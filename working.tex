%%%%%%%%%%%%%%%%%%%%%%%%%%%%%%%%%%%%%%%%%%%%%%%%%%%%%%%%%%%%%%%%%%%%%%%%%%%%%%%%%%%%%%%%%%%%%%%%%%%%%%%%%%%%%%%%%%%%%%%%

\documentclass{sig-alternate}

\usepackage[T1]{fontenc}

\usepackage[lighttt]{lmodern}

\usepackage[british]{babel}%
\usepackage{cleveref}%
\usepackage{csquotes}%
\usepackage[shortcuts]{extdash}%
\usepackage[numbers]{natbib}

\usepackage{listings}
\usepackage{subcaption}

\usepackage{pgf}

\lstset{language=python}
\lstset{basicstyle=\ttfamily\small}
\lstset{keywordstyle=\ttfamily\bfseries}
\lstset{frame=single}

\newenvironment{FunctionList}{%
\lstset{basicstyle=\ttfamily\bfseries\small}
\begin{list}{}{\leftmargin=5pt}
}{\end{list}\lstset{basicstyle=\ttfamily\small}}

\newcommand{\picalc}{\(\pi\)-calculus }

\hyphenation{poli-cies}
%\hyphenation{tech-ni-cal}

%%%%%%%%%%%%%%%%%%%%%%%%%%%%%%%%%%%%%%%%%%%%%%%%%%%%%%%%%%%%%%%%%%%%%%%%%%%%%%%%%%%%%%%%%%%%%%%%%%%%%%%%%%%%%%%%%%%%%%%%

\title{Simulating Variance in Socio-Technical Behaviours using Executable Workflow Fuzzing}

%\author{Tom Wallis \and Tim Storer}

% \institute{Tom Wallis
%   \at School of Computing Science, University of Glasgow, Glasgow \\
%   \email{twallisgm@gmail.com}
%   \and
%   Tim Storer
%   \at School of Computing Science, University of Glasgow, Glasgow \\
%   \email{timothy.storer@glasgow.ac.uk}
%   }  
  
% \date{Received: XXX / Accepted: XXX}

% \date{}    Do we want a date on the paper? Don't know how these things are done. 

%%%%%%%%%%%%%%%%%%%%%%%%%%%%%%%%%%%%%%%%%%%%%%%%%%%%%%%%%%%%%%%%%%%%%%%%%%%%%%%%%%%%%%%%%%%%%%%%%%%%%%%%%%%%%%%%%%%%%%%%

\begin{document}

%%%%%%%%%%%%%%%%%%%%%%%%%%%%%%%%%%%%%%%%%%%%%%%%%%%%%%%%%%%%%%%%%%%%%%%%%%%%%%%%%%%%%%%%%%%%%%%%%%%%%%%%%%%%%%%%%%%%%%%%

\maketitle

%%%%%%%%%%%%%%%%%%%%%%%%%%%%%%%%%%%%%%%%%%%%%%%%%%%%%%%%%%%%%%%%%%%%%%%%%%%%%%%%%%%%%%%%%%%%%%%%%%%%%%%%%%%%%%%%%%%%%%%%

\begin{abstract}
  %% TODO SHORTEN!


  The engineering of large scale, complex software based socio-technical systems is still very much a craft, dependent
  on methods based on trial, error and subsequent revision.  A particular difficulty is the lack of modelling tools that
  support the analysis and prediction of actor behaviours in and around socio-technical systems.  Existing notations
  such as activity diagrams, business process modelling languages or Petri Nets assume that behaviour can be described
  as idealised workflows.  Unfortunately, the behaviour in a socio-technical system is highly contingent and subject to
  considerable variability as actors react to changing conditions and identify potential optimisations to their
  practices.  As a consequence, existing techniques result in models that either lack sufficient detail to capture the
  effect of subtle contingencies; are too narrow to make useful assessments about the larger system; are unable to
  capture evolution in behaviours; or are so complex that analysis and interpretation becomes intractable.

  This paper presents Fuzzi Moss, a novel method for simulating the effect of contingent behaviour in socio-technical
  systems using software code fuzzing.  A socio-technical system is represented as an object-oriented domain model
  compromising one or more classes.  Behaviours that operate on the system model state are described separately as a set
  of idealised executable workflows.  These workflows are then annotated with code fuzzers, declarations of how the
  workflow could be dynamically altered (removing steps, duplicating steps, introducing new steps for example) when the
  workflow is executed.  A simulation is then configured by executing the fuzzed workflow on an instance of the domain
  model.  Results collected from the domain model state inform predictions as to how an idealised workflow will perform
  in the presence of contingent behaviour in the real world.
 
  We have implemented a proof of concept tool to evaluate our method. Fuzzi Moss is as a small Python library,
  comprising a decorator for annotating executable workflows with desired fuzzers, a mechanism for dynamically
  inspecting the structure of a workflow and applying fuzzers during execution and a library of core fuzzers.  We
  evaluate our approach to simulating socio-technical behaviours by applying Fuzzi Moss to a socio-technical system case
  study.  The results of the case study demonstrate the feasibility of the fuzzing method.  Finally, We explore the
  wider applications of fuzzing in socio-technical systems modelling and discuss the next steps for the research.

\end{abstract}

%%%%%%%%%%%%%%%%%%%%%%%%%%%%%%%%%%%%%%%%%%%%%%%%%%%%%%%%%%%%%%%%%%%%%%%%%%%%%%%%%%%%%%%%%%%%%%%%%%%%%%%%%%%%%%%%%%%%%%%%

\section{Introduction}
\label{sec:introduction}

%%%%%%%%%%%%%%%%%%%%%%%%%%%%%%%%%%%%%%%%%%%%%%%%%%%%%%%%%%%%%%%%%%%%%%%%%%%%%%%%%%%%%%%%%%%%%%%%%%%%%%%%%%%%%%%%%%%%%%%%

\emph{Socio-technical systems} are large scale, complex models, representing the interactions between a diverse set of
actors including individual technical artifacts, human operators, organisational structures
\citep{Susman1976,elhassan07role,sommerville08socio}.  The behaviour of such systems are influenced by a complex
interplay of factors, including formally defined business processses, legal or regulatory standards, technological
evolution, organisational culture or norms and interpersonal relationships and responsibilities
\citep{bade07structures,pentland05organisational}. Examples of such systems with complex workflows involving multiple
actors include emergency vehicle dispatch \citep{robinson96limited}, electronic voting systems
\citep{bryans04towards,lock07observations}, patient care in a neo-natal unit \citep{baxter07evaluating} and electronic
stock exchange infrastructures \citep{cftc-sec10findings}.  Systems of this form confound traditional approaches to
modelling, simulating and predicting behaviour for several reasons:

\begin{itemize}

\item Socio-technical systems are simultaneously \emph{very large and heterogeneous}, comprising a a mix of autonomous
  actors, each with their own behaviours \cite{crabtree00ethnomethodologically}.  Systems engineering has traditionally
  approached the problem of scale through the development of models that abstract complex behaviours and model them as
  emergent system properties \cite{vespignani11model}.  However, these stochastic treatments do not capture the complex
  interactions that occur between heterogeneous actors, with interactions occurring across different scales of activity.
  For example, \citet{lock07observations} observed the disruptions caused to a national election in Scotland caused by a
  variety of small scale technical system defects.

\item The behaviour is contingent on unpredictable circumstances, including both factors in the environment and
  concerning the system actors.  For example, the time and manner in which a task, such as developing a new feature for
  a software system, is completed may vary considerably between actors with different training and experiences.
  Similarly, the decision to work on a task at all may depend on unpredictable and uncontrollable external circumstances
  (such as a power outage).  In these circumstances, actors may also take it upon themselves to complete tasks outwith
  expected workflows in order to discharge their responsibilities, by working from a nearby cafe for example, even if
  this violates organisational security policies.  As \citet{besnard03human} note, such adaptations often make the human
  actors the dependable parts of a socio-technical system.

\item Behaviour is continually evolving, as the autonomous actors in a system adapt to new circumstances, discover
  optimisations to their workflows, adapt the workflow to suit local organisational priorities or take shortcuts
  \citep{bonen79evolutionary,Lyytinen2008,anderson04heterogeneous}.  As a consequence, the \emph{de facto} behaviour
  exhibited within a system may differ from that envisaged by system architects in idealised workflows.  For example, a
  ward manager in a hospital may delay releasing beds for re-allocation by wider hospital management in the anticipation
  that these will be required by incoming patients later in the day \citep{dewsbury07responsibility}.  This evolution of
  practice may quickly invalidate expected models of behaviour.

\end{itemize}

We contend that due to these challenges, modelling socio-technical system behaviours using conventional systems
engineering methods will typically either result in a model that is tractable, but lacks the necessary detail of the
underlying system to provide informative results; so narrow in scope as to be uninformative about the behaviour of the
wider system of interest; or so large and complex as to be intractable for analysis, whether manual or automated.
Consequently, the design and construction of systems at this scale is still very much a craft, lacking the methods and
tools to support modelling and predictive simulation available in other engineering disciplines.

The research contribution of this paper is to present and evaluate a novel environment, Fuzzi Moss, for simulating
complex and contingent behaviour in socio-technical systems which addresses this challenge.  In our approach, we provide
for a separation of concerns between the model of a problem domain, models of idealised socio-technical actor behaviour
and the influence of contingent factors that complicate the actual execution of idealised workflows in practice.  The
separation of concerns is achieved by modelling:

\begin{itemize}

\item The problem domain as collection of classes implemented in the Python programming language.

\item Idealised workflows descriptions as executable Python classes in the agent oriented modelling framework,
  Theatre\_Ag \citep{theatreag}.

\item Contingent behaviour as \emph{dynamic fuzzing aspects} that can alter the flow of execution in workflow
  descriptions during the execution of a simulation, using the PyDySoFu library \citep{wallis2017pydysofu}.

\end{itemize}

Both the Theatre\_Ag framework and PyDySoFu libraries were implemented specifically for this work.

Critical to the approach is our hypothesis that:

\begin{quotation}
  Hypothesis: Dynamic fuzzing of workflow descriptions can represent the effect of complex and contingent behaviour by
  actors in socio-technical systems, when following idealised workflows.
\end{quotation}

To test this hypothesis, an example socio-technical case study of team based software development was developed.  The
case study compares the performance of different software development processes when a software development team follows
idealised workflows that have been subject to contingent behaviour.  Development processes are compared based on their
effect on the emergent properties of the simulated system under development, specifically features implemented and mean
time to failure.

The rest of this paper is structured as follows.  Section \ref{sec:related} discusses related work, covering existing
techniques for modelling socio-technical workflows and other applications of code fuzzing in software engineering.
Section \ref{sec:fuzzi-moss} presents the method for constructing models of socio-technical systems, associated
workflows and denoting desired fuzzings.  Where relevant, this section also discusses details of the implementation
details for Fuzzi Moss.  Section \ref{sec:evaluation} presents the case study evaluation of the method and Section
\ref{sec:conclusions} discusses conclusions and future work, as well as noting the potential for applying fuzzing to
other forms of socio-technical models.

%%%%%%%%%%%%%%%%%%%%%%%%%%%%%%%%%%%%%%%%%%%%%%%%%%%%%%%%%%%%%%%%%%%%%%%%%%%%%%%%%%%%%%%%%%%%%%%%%%%%%%%%%%%%%%%%%%%%%%%%

\section{Related Work}

\label{sec:related}

%%%%%%%%%%%%%%%%%%%%%%%%%%%%%%%%%%%%%%%%%%%%%%%%%%%%%%%%%%%%%%%%%%%%%%%%%%%%%%%%%%%%%%%%%%%%%%%%%%%%%%%%%%%%%%%%%%%%%%%%

This section presents a literature review of the development of models and of behaviours in socio-technical systems.
The difficulties of developing modelling techniques that accommodate the inherent scale, complexity, contingency and
dynamism of socio-technical systems are highlighted.  In addition, existing applications of software fuzzing are
reviewed with respect to their relevance to the present work.

Graphical notations have received considerable attention, perhaps due to their perceived efficacy in communicating
requirements between users, customers and system architects.  These modelling languages include workflow based
approaches such as UML activity diagrams \citep{omg2010omguml}, BPMN \citep{omg2011omgbpmn}, YAWL
\citep{hofstede2010yawl} and OBASHI \cite{obashimethodology}; and goal based approaches such as KaOS
\citep{Werneck2009}, \emph{i*} \citep{yu1995} and responsibility modelling \citep{sommerville09responsibility}.

Activity diagrams are perhaps the most commonly known workflow language, due to incorporation in the UML standard
\citep{omg2010omguml}.  The notation supports the modelling of the flow of control across a directed graph of
activities, with arcs representing transitions in control.  Additional nodes are provided for denoting entry and exit
points, as well as decision branches.  The notation is based on the Petri Net formalism and includes support for
concurrent flows through the chart, as well as workflow forking and merging. The semi-formal nature of the UML standard
enable the automatic parsing of graphical models, using CASE tools such as the Eclipse Modelling Framework
\citep{EMFManual}.  An advantage of this approach is that models can be used for negotiation between project
stakeholders, whilst also being used for simulations to predict system behaviour.

The Business Process Model and Notation (BPMN) is an alternative OMG standard for modelling workflows, with similar core
notation and semantics for modelling workflows \citep{omg2011omgbpmn}.  Unlike activity diagrams, however, BPMN provides
a richer notation for expressing more complex aspects of activities, such as differentiating between tasks, activities
and transactions; triggering and orchestrating concurrent activities using messages; the identification of information
resources need to realise an activity; and the orchestration of activities across organisational boundaries
\citep{White2004}.  The notation is intended to support the generation of executable business processes expressed as web
services, however, it can also be employed in other workflow contexts.

Yet Another Workflow Language (YAWL) provides similar capabilities to activity diagrams for modelling workflows, as well
as being supported by CASE tools for graphical modelling \citep{hofstede2010yawl}.  However, unlike activity diagrams,
YAWL is based on the \picalc\citep{Aalst2004}.  The notation also provides for a richer range of workflow requirements
than activity diagrams, including sophisticated forking and merging rules, separation between workflow specifications
and executions and resourcing and data requirements.

The OBASHI (Ownership, Business, Application, System, Hardware, Infrastructure) methodology and notation
\citep{obashimethodology} is designed for modelling business processes across enterprise infrastructures.  The notation
is intended for capturing the movement of data through a business process and revealing the associated dependencies on
underlying infrastructure such as software systems, servers and network communications.  The language also provides a
means for mapping these flows to higher level concerns, such as business rationale and ultimate organisational owner. In
contrast to other workflow notations, flows are based on the movement of data rather than control.

%%%%

Describing socio-technical behaviour using workflow notations can be difficult, because of the basic assumption that all
contingencies in a workflow can be completely described at a given level of granularity, and that more complex details
can be encapsulated within coarser grained activities with well defined interfaces.  As argued in Section
\ref{sec:introduction}, socio-technical behaviours are inherently complex, contingent and evolutionary, making such
refinement based techniques difficult to apply.  As \citet{israilidis13ignorance} have argued, the unknowns in a
socio-technical system may be far more significant than the knowns. Several authors have therefore discussed alternative
techniques for modelling socio-technical systems with support for contingent behaviour
\citep{yu1995,dardenne93goal,Herrmann1999,sommerville09deriving}.

Both \emph{i*}\citet{yu1995} and KaOS \citet{dardenne93goal} are goal oriented notations for modelling socio-technical
systems \citep{Werneck2009}.  In contrast to workflows, goal oriented approaches primarily capture the intents of actors
(what they are seeking to achieve).  Goals can be de-composed into a sub-goal hierarchy using logical operators to
express the form of decomposition. Goals can also be annotated with strategies and/or resource requirements to support
automated analysis.  \citeauthor{yu1995} argued that socio-technical systems should be viewed as collections of
collaborating actors, each with their own (potentially conflicting) objectives.  Eliciting and analysing the actors
intents allows the inter-dependencies between actors and the overall behaviour of the system to be understood, without
the need for explicit models of individual workflows.

Other authors have extended goal oriented approaches to provide greater flexibility.  \citet{sommerville09deriving}
argued that stakeholders often struggle to express their behaviour within a socio-technical system in terms of goals.
Instead, \citeauthor{sommerville09deriving} argue that the concept of \emph{responsibilities}, the duties held by an
actor in a system, are a more intuitive means of describing system behaviours that also capture a variety of contingent
behaviours.  A notation for expressing the relationships between responsibilities and resources in order to identify
dependencies within a system is provided.  Earlier work on responsibility modelling also provided mechanisms with
annotating responsibilities with indicative workflows, expressing the means by which responsibilities \emph{could} be
executed \citep{dewsbury07responsibility}.

Despite providing for contingency, a limitation of the goal and responsibility approaches is the need for complete model
descriptions. \citet{Herrmann1999} introduced techniques for annotating goal oriented system models in the SeeMe
notation with vagueness.  The notation enables a modeller to denote where vagueness may be present in a model due to
abstraction (i.e. consistent vagueness) and due to omission (inconsistent vagueness).  In addition,
\citeauthor{Herrmann1999} provide notation for indicating that a model is thought to be complete, containing all
pertinent details.  However, the annotations are not accompanied by a formal semantics, or other means of supporting
automated analysis.

We are not aware of other applications of fuzzing techniques to modelling contingent behaviours in socio-technical
systems.  However, software code fuzzing (or mutation) is employed in software quality assurance in order to
automatically generate program variants.  \emph{Mutation operators} in such applications may alter the value of
literals, swap arithmetic or other operators, or change the ordering of arguments to a function call, for example.
Applying different combinations of mutation operators creates a population of mutants of the target program.  One
application of this technique is mutation testing, in which the generation of program variants is used to simulate the
introduction of defects and evaluate the effectiveness of an application's test suite in detecting regressions
\citep{demillo78hints}.  A test is considered to have detected a mutant if the application of the test to the mutant
fails.  A test suite that detects a higher proportion of mutants is considered to have good coverage of the target
program.

The effectiveness of mutant generation is significantly influenced by choice of mutant operators to apply, since the
search space of potential mutants to be tested is very large and many mutants will reveal the same test suite
deficiencies \citep{takanen08fuzzing}.  Generation of mutants based on an understanding of a system's specification
allows mutant generation to be focused on a system's intended behaviour.  It can therefore be expected that applying
code fuzzing to simulating socio-technical behaviours requires an understanding of the likely variants to behaviour that
may occur in a workflow in order to generate realistic simulations.

There are a variety of existing tools that incorporate fuzzing functionality for mutation testing, including PiTest
\citep{coles14pitest} for Java and MutPy \citep{mutpy26} and PyMuTester \citep{pymuttester} for Python.
\citet{storer15ringneck-repos} has also developed a tool for mutation testing Maven component assembly
specifications. All these tools work by constructing and then manipulating abstract syntax trees of target programs.
The result is a population of statically generated mutant programs that can be evaluated using the target program's own
test suite.  A disadvantage of this mechanism (for the purposes of modelling socio-technical systems) is that the
mutants are generated statically, prior to program execution.  Our own implementation of code fuzzing is motivated by
the desire to simulate dynamic contingent behaviour, that can vary from the idealised model each time a fuzzed step in a
workflow is executed.

%%%%%%%%%%%%%%%%%%%%%%%%%%%%%%%%%%%%%%%%%%%%%%%%%%%%%%%%%%%%%%%%%%%%%%%%%%%%%%%%%%%%%%%%%%%%%%%%%%%%%%%%%%%%%%%%%%%%%%%%

\section{Problem Domain Model}

%%%%%%%%%%%%%%%%%%%%%%%%%%%%%%%%%%%%%%%%%%%%%%%%%%%%%%%%%%%%%%%%%%%%%%%%%%%%%%%%%%%%%%%%%%%%%%%%%%%%%%%%%%%%%%%%%%%%%%%%

In this section we introduce our approach to modelling a problem domain and associated idealised workflows in
socio-technical systems.  We have chosen to present the approach through an example case study of team based software
development, in which we will explore the efficacy of two development workflows: waterfall and test driven development.
We have adopted an object-oriented approach to modelling the elements of the problem domain \citep{bennett06object}, so
that the elements of the domain are described as a collection of Python classes.  Figure \ref{fig:feature-class-diagram}
shows the class diagram for the case study.  The diagram shows classes for:

\begin{figure}
  
  \centering
  \includegraphics{floats/class-diagram-1}
  
  \caption{Class diagram of the software development problem domain case study, using the UML notation.}
  \label{fig:feature-class-diagram}
  
\end{figure}


\begin{itemize}
\item Features, representing user-facing specifications of the system's functionality.
\item Code chunks, representing the implementation details of the features, which may have dependencies on other chunks
  in the system.
\item Bugs introduced into chunks during the completion of features.  
\item Software systems which aggregate all the source artefacts of a software project, including features, chunks, bugs
  and tests.
\item Version control servers and clients for coordinating distributed development of a project.

\end{itemize}

Methods are implemented for these classes that provide them with behaviours in the problem domain. Many of these
behaviours have side effects which are modelled stochastically. For example:

\begin{itemize}

\item Features can be extended through the addition of code chunks.  Each time a new chunk is added to a feature other
  chunks may also need to be modified, potentially creating further dependencies between chunks or introducing bugs.

\item Tests can be exercised resulting in the detection of bugs.  The more tests created for a feature, the greater the
  probability of detecting a given bug.

\item Features can be debugged (resulting in the removal of bugs) or refactored (resulting in the reduction in the
  number of dependencies).

\item Software systems can be operated, which may cause bugs in the system to manifest themselves, causing a system
  halt.

\end{itemize}

Very few restrictions are placed on the implementation of the problem domain classes.  Operations can accept a variety
of arguments, modify object state, invoke operations on other problem domain classes and return values as desired.
However The operations on the problem domain should be modelled as atomic and independent, allowing them to be safely
invoked in any combination.  The \lstinline!@Property! decorator can be used to improve the readability of code, as
normal.

Idealised socio-technical workflows are collections of tasks that operate on a common state.  Tasks are implemented as
Python methods, with all the tasks associated with the same workflow and operating on the same state collected together
in a single Python class.  For the purposes of the software development case study, workflows were created for
interacting with a version control server in an update-merge-commit cycle; specification of new features in a system;
implementation of features; development of tests to exercise features; and debugging tests that revealed bugs and
refactoring of features.  Workflows can also be organised hierarchically, so for example, the workflows for modifying
the system artefacts depend on the change management workflow in order to coordinate changes within a team.  Further
workflows were implemented for coordinating the overall team activities by following the waterfall and test driven
development methodologies.  Figure \ref{fig:debugging} shows the Python code for the debuggin workflow, while
\ref{fig:tdd} shows the Python code for the Test Driven Development workflow.

\begin{figure*}

  \begin{subfigure}{\linewidth}
\begin{lstlisting}
class Debugging(object):

    is_workflow = True

    @default_cost(1)
    def debug(self, feature, bug, random):
        feature.debug(random, bug)

    @default_cost()
    def debug_test(self, test, random):
        while True:
            try:
                test.exercise()
                break
            except BugEncounteredException as e:
                self.debug(test.feature, e.bug, random)
                self.change_management.commit_changes(random)

    @default_cost()
    def debug_feature(self, logical_name, random):

        self.change_management.checkout()

        feature = self.change_management.centralised_vcs_client.working_copy.get_feature(logical_name)

        for test in feature.tests:
            self.debug_test(test, random)

    @default_cost()
    def debug_system(self, random):
        self.change_management.checkout()
        for test in self.change_management.centralised_vcs_client.working_copy.tests:
            self.debug_test(test, random)

\end{lstlisting}

    \caption{Debugging workflow}
    \label{fig:debugging}
  \end{subfigure}


  \vspace{10pt}

  \begin{subfigure}{\linewidth}
\begin{lstlisting}
class TestDrivenDevelopment(object):

    is_workflow = True

    @default_cost()
    def implement_feature_tdd(self, user_story, random):
        self.specification.add_feature(user_story.logical_name, user_story.size, random)
        self.testing.test_per_chunk_ratio(user_story.logical_name, random)
        self.implementation.implement_feature(user_story.logical_name, random)
        self.debugging.debug_feature(user_story.logical_name, random)
        self.refactoring.refactor_feature(user_story.logical_name, random)

    @default_cost()
    def work_from_backlog(self, product_backlog, random):
        while True:
            try:
                user_story = product_backlog.get(block=False)
                self.implement_feature_tdd(user_story, random)
            except Empty:
                break
\end{lstlisting}
    \caption{Test Driven Development workflow}
    \label{fig:tdd}
  \end{subfigure}

  \vspace{10pt}

  \caption{Examples of workflow classes implemented in Theatre\_Ag.  Constructors are omitted for brevity.}
\end{figure*}

Theatre\_Ag represents simulations as \emph{episodes}, which provide a cast of actors with an initial set of directions
called an \emph{improvisation} to operate on a problem domain model.  Two improvisations were implemented for the
software development case study in order to initiate simulations for the waterfall and test driven development
workflows.


%%%%%%%%%%%%%%%%%%%%%%%%%%%%%%%%%%%%%%%%%%%%%%%%%%%%%%%%%%%%%%%%%%%%%%%%%%%%%%%%%%%%%%%%%%%%%%%%%%%%%%%%%%%%%%%%%%%%%%%%

\section{Workflow Fuzzing}

%%%%%%%%%%%%%%%%%%%%%%%%%%%%%%%%%%%%%%%%%%%%%%%%%%%%%%%%%%%%%%%%%%%%%%%%%%%%%%%%%%%%%%%%%%%%%%%%%%%%%%%%%%%%%%%%%%%%%%%%

Specification of socio-technical workflow fuzzers using aspects implemented using the pydysofu
fuzzing framework 


%%%%%%%%%%%%%%%%%%%%%%%%%%%%%%%%%%%%%%%%%%%%%%%%%%%%%%%%%%%%%%%%%%%%%%%%%%%%%%%%%%%%%%%%%%%%%%%%%%%%%%%%%%%%%%%%%%%%%%%%

\section{Method and Implementation}
\label{sec:fuzzi-moss}

%%%%%%%%%%%%%%%%%%%%%%%%%%%%%%%%%%%%%%%%%%%%%%%%%%%%%%%%%%%%%%%%%%%%%%%%%%%%%%%%%%%%%%%%%%%%%%%%%%%%%%%%%%%%%%%%%%%%%%%%

Implementation details of the the proof of concept library, Fuzzi Moss, are also presented.  Source code for the Fuzzi
Moss library is available from the project's GitHub repository \citep{wallis2016fuzzi-moss-scm}.  The library was
implemented in the Python programming language.  The Python syntax is designed to be readable, meaning that domain
models and workflows could be expressed in structured natural language.  Further, Python is an interpreted
object-oriented language, that treats function definitions as first class constructs.  This made the development of a
proof of concept fuzzing library convenient.  The core implementation is less than 100 lines of program code, including
blank lines and source code documentation, whilst the library of utility fuzzers is implemented in less than 300
lines. Source code for the case study is also available in a separate GitHub repository
\citep{storer2016softdev-workflow-scm} and comprises approximately 1000 lines of program code.  Some of the example code
fragments shown below have been modified to ease explanation (removing self arguments from function definitions, for
example).


%%%%%%%%%%%%%%%%%%%%%%%%%%%%%%%%%%%%%%%%%%%%%%%%%%%%%%%%%%%%%%%%%%%%%%%%%%%%%%%%%%%%%%%%%%%%%%%%%%%%%%%%%%%%%%%%%%%%%%%%

\subsection{Workflow Modelling}

%%%%%%%%%%%%%%%%%%%%%%%%%%%%%%%%%%%%%%%%%%%%%%%%%%%%%%%%%%%%%%%%%%%%%%%%%%%%%%%%%%%%%%%%%%%%%%%%%%%%%%%%%%%%%%%%%%%%%%%%

The next stage of the method concerns the development of descriptions of socio-technical behaviours of actors in the
problem domain, using workflows.  Workflows direct the sequence of actions by an actor on the artefacts in the problem
domain.  Behaviours are modelled as \emph{idealised} workflows, i.e. the behaviour desired or expected of an actor in a
socio-technical system without concern for contingencies.  Workflows are modelled as activity diagrams, which enables
the construction of a hierarchy of workflow descriptions, supporting modularity and reuse.

Two related workflows are illustrated in Figure \ref{fig:workflow-partial}.  Figure \ref{fig:workflow-tdd} illustrates a
Test Driven Development workflow, in which a software developer specifies an implementation, creates a test case,
implements and debugs the feature and finally refactors the functional implementation.  Figure
\ref{fig:workflow-refactoring} illustrates the more fine grained workflow for refactoring.  A developer, continues to
refactor the feature until the measured coupling rate is less than a desired maximum.  Both workflows are executed by
invoking operations on a \lstinline!Developer!  instance that tracks the cost of performing actions on the development
project.

\begin{figure*}
  \centering

  \begin{subfigure}[b]{.45\linewidth}
    \centering
    \includegraphics{floats/tdd-workflow-1}

    \caption{Test driven development}
    \label{fig:workflow-tdd}
  \end{subfigure}
  \begin{subfigure}[b]{.45\linewidth}
    \centering
    \includegraphics{floats/refactoring-workflow-1}
    
    \caption{Refactoring}
    \label{fig:workflow-refactoring}
  \end{subfigure}
  
  \

  \caption{Partial socio-technical workflows for software development expressed as UML activity diagrams.}

  \label{fig:workflow-partial}
\end{figure*}

We currently limit the use of activity diagram notation to begin, end, activity and conditional branches, with
transitions indicating the flow of control between steps.  Fuzzi Moss does not currently support concurrent behaviours
that would be represented by fork and join nodes, although this feature is discussed in Section \ref{sec:conclusions}.
The workflow models are implemented as collections of Python functions.  As for the domain actions, implementation of
the workflow functions is flexible. Workflow functions can be parameterised and may also return values as desired.
Workflow functions can be encapsulated into classes to support model maintenance, although this is not necessary for the
method itself.  One restriction that is made is that workflow functions should not contain nested function definitions,
as these may be inadvertently fuzzed by the Fuzzi Moss mechanism.

%%%%%%%%%%%%%%%%%%%%%%%%%%%%%%%%%%%%%%%%%%%%%%%%%%%%%%%%%%%%%%%%%%%%%%%%%%%%%%%%%%%%%%%%%%%%%%%%%%%%%%%%%%%%%%%%%%%%%%%%

\subsection{Specifying the Fuzzers for a Workflow}

%%%%%%%%%%%%%%%%%%%%%%%%%%%%%%%%%%%%%%%%%%%%%%%%%%%%%%%%%%%%%%%%%%%%%%%%%%%%%%%%%%%%%%%%%%%%%%%%%%%%%%%%%%%%%%%%%%%%%%%%

The workflow functions implemented in Python are denoted as eligible for fuzzing by applying the \lstinline!@fuzz!
decorator provided in the Fuzzi Moss package. The decorator's constructor accepts one argument, a pointer to a
\emph{fuzzer} function, that will be used in the fuzzing mechanism described below.  An example of the application of a
fuzzer (identity) is shown in Figure \ref{fig:fuzz}. Python's decorator mechanism intercepts invocations of the
decorated function and allows these to be substituted with an alternative function to be invoked.  The Fuzzi Moss
decorator exploits this mechanism by defining and returning a \lstinline!wrap()!  function that is able to dynamically
fuzz the decorated function each time it is invoked.  The wrap function accepts the same set of arguments as the
decorated function and should be expected to return the same values. The decorator's \lstinline!__call__()!  method,
which intercepts the invocation of the decorated function, defines the wrap function as follows.

\begin{enumerate}

\item The abstract syntax tree (AST) for the decorated function is constructed, using the Python \lstinline!inspect!
  package to recover the function's source code and the \lstinline!ast!  package to build the AST from the source.  This
  reference AST is then cached for future reference.

\item A copy of the AST is made.  An AST visitor (called a transformer) is then constructed and given the decorator's
  fuzzer attribute as an argument.  The copied syntax tree is then passed to the visitor.

\item The visitor identifies the function definition node in the AST and applies the fuzzer to the function definition's
  body.  The body is represented as a list of Python statements in the AST.  The function body is replaced by the value
  returned by the fuzzer.

\item Finally, the now fuzzed AST is compiled to Python byte code.  The byte code of the fuzzed function is then
  substituted for the decorated function's byte code and the function is invoked with its original parameters.  Control
  is then passed back to the Python interpreter, allowing any returned values to be passed back as normal.

\end{enumerate}


 \begin{figure}
   \centering
  \begin{lstlisting}
@fuzz(identity) def _refactor_feature(developer, feature): while len(feature.dependencies) > \
self.target_dependencies_per_feature: developer.refactor(feature)
\end{lstlisting}
  
   \caption{Example fuzz decorator applied to a software development workflow.}
   \label{fig:fuzz}
 \end{figure}

 The generic definition of the fuzzer function (accept and return a list of Python statement ASTs) allows a user to
 define a wide range of fuzzing mechanisms to suit their problem domain.  To demonstrate this flexibility, a suite of
 fuzzers is provided in Fuzzi Moss that allow the modular construction of more complex fuzzer behaviours.  The behaviour
 of each of the fuzzers available in Fuzzi Moss are described in the following sub sections.

%%%%%%%%%%%%%%%%%%%%%%%%%%%%%%%%%%%%%%%%%%%%%%%%%%%%%%%%%%%%%%%%%%%%%%%%%%%%%%%%%%%%%%%%%%%%%%%%%%%%%%%%%%%%%%%%%%%%%%%%

 \subsection{Simple Fuzzers}

%%%%%%%%%%%%%%%%%%%%%%%%%%%%%%%%%%%%%%%%%%%%%%%%%%%%%%%%%%%%%%%%%%%%%%%%%%%%%%%%%%%%%%%%%%%%%%%%%%%%%%%%%%%%%%%%%%%%%%%%

 A simple fuzzer accepts and returns a list of \lstinline!ast.Statement! objects and can be declared for use directly
 within a fuzz decorator.  The following simple fuzzers are defined in Fuzzi Moss.

 \begin{FunctionList}

 \item\lstinline!identity! returns the input list of statements.  The identity fuzzer is used as a default for the fuzz
   decorator, but is also useful when building composite fuzzers.

 \item\lstinline!replace_steps_with_passes! returns a list of \lstinline!pass!  statements of the same length as the
   input.  Replacing statements with pass is safer than removing the statement, since a Python function must be defined
   with at least one statement.

 \item\lstinline!duplicate_steps! returns a list containing the input sequence repeated twice.

 \item\lstinline!shuffle_steps! returns a randomly shuffled list of the input.  A Python Random object
   \lstinline!fuzzi_moss_random! is used as a random source.

 \item \lstinline!swap_if_blocks! switches the body and orelse blocks of all if statements in the input.

 \end{FunctionList}

 The application of simple fuzzers was demonstrated in Figure \ref{fig:fuzz}.  Note that the fuzzer is supplied to the
 fuzz decorator as a function pointer, rather than as an evaluated function call.

%%%%%%%%%%%%%%%%%%%%%%%%%%%%%%%%%%%%%%%%%%%%%%%%%%%%%%%%%%%%%%%%%%%%%%%%%%%%%%%%%%%%%%%%%%%%%%%%%%%%%%%%%%%%%%%%%%%%%%%%

 \subsection{Fuzzing Filters}

%%%%%%%%%%%%%%%%%%%%%%%%%%%%%%%%%%%%%%%%%%%%%%%%%%%%%%%%%%%%%%%%%%%%%%%%%%%%%%%%%%%%%%%%%%%%%%%%%%%%%%%%%%%%%%%%%%%%%%%%

 Sometimes it is desirable to restrict the application of simple fuzzers to particular portions of the body of a
 workflow function.  In this situation the \lstinline!filter_steps()! fuzzer can be used.  Filter steps takes two
 arguments: a filter function pointer and a fuzzer function pointer.  These are used to define a nested
 \lstinline!_filter_step! fuzzer that behaves like a simple fuzzer as described above.

 A filter function accepts a list of statements and returns a list of tuples.  Each tuple denotes the start and end
 index for a block of statements in the input.  Each block is then fuzzed using the supplied fuzzer.  Blocks not
 specified by the filter are not affected by the fuzzer, such that the returned sequence of statements may contain a mix
 of fuzzed and un-fuzzed statements.

 The following filters are available for use in conjunction with a fuzzer.

 \begin{FunctionList}

 \item \lstinline!choose_last_step! returns the start and end index of the last statement in the input,
   i.e. \lstinline!input[-2:-1]!

 \item \lstinline!choose_random_steps(n)!  returns $n$ length 1 sub-blocks randomly selected from the input steps.

 \item \lstinline!exclude_control_structures(target)! permits the exclusion of control structure statements by type as
   specified by the target input argument. Supported structures are \lstinline!for!, \lstinline!while!, \lstinline!if!,
   \lstinline!try!-\lstinline!except! and \lstinline!return!.

 \end{FunctionList}

 An invert filter is also provided which inverts the selection provided by its argument. Several pre-defined filtering
 fuzzers are implemented in Fuzzi Moss combining some of the predefined filters and simple fuzzers.  These provide
 commonly required fuzzers, including removing or duplicating the last step or a random step. A
 \lstinline!choose_identity! filter is also included for completeness.  Figure \ref{fig:filter} illustrates the
 application of a filter steps fuzzer that randomly shuffles all but the last step.

\begin{figure}
  \centering

\begin{lstlisting}
@fuzz( filter_steps( invert(choose_last_step), shuffle_steps ) )
def work(self, system, developer, schedule):

self.complete_specification(schedule, system) self.implement_features(developer, system)
self.implement_test_suite(developer, system) self.debug_system(developer, system) self.refactor_system(developer,
system)

\end{lstlisting}

  \caption{Using filters to control the application of step replacement to all but the last step in a Waterfall software
    development workflow.}
  \label{fig:filter}
\end{figure}

%%%%%%%%%%%%%%%%%%%%%%%%%%%%%%%%%%%%%%%%%%%%%%%%%%%%%%%%%%%%%%%%%%%%%%%%%%%%%%%%%%%%%%%%%%%%%%%%%%%%%%%%%%%%%%%%%%%%%%%%

\subsection{Composite Fuzzers}

%%%%%%%%%%%%%%%%%%%%%%%%%%%%%%%%%%%%%%%%%%%%%%%%%%%%%%%%%%%%%%%%%%%%%%%%%%%%%%%%%%%%%%%%%%%%%%%%%%%%%%%%%%%%%%%%%%%%%%%%

More complex fuzzers can be assembled using composite fuzzing functions that are implemented in a similar manner to
filtering.


\begin{FunctionList}
\item \lstinline!in_sequence(fuzzers)! applies each fuzzer found in the input list of fuzzers in sequence to a function
  body.

\item \lstinline!choose_from(distribution)! selects a fuzzer to apply at random from the supplied probability
  distribution.  The distribution is defined as a list of weight, fuzzer tuples.

\item \lstinline!on_condition_that(condition, fuzzer)! %
  applies the specified fuzzer if the specified condition holds.  The condition may be a literal Boolean value, a
  function pointer or a Python expression.  Lambda expressions are not supported.

\item \lstinline!recurse_into_nested_steps(target)! identifies control structure statements and applies the supplied
  fuzzer to their body blocks.  The recursion can be limited to particular control structures using the option
  \lstinline!target! argument.
\end{FunctionList}

An example of applying a composite fuzzer to a workflow is shown in Figure \ref{fig:composite}.  The figure shows a
workflow for enhancing the quality of a software system.  A developer adds a test to a feature in order to detect the
presence of bugs, or prevent the introduction of bugs during refactoring.  The developer then performs some debugging
work, followed by some refactoring.  The fuzzer applied to the workflow is the \lstinline!choose_from!.  The supplied
distribution will fuzz the workflow 5\% of the time on average, removing one step at random when it does so.  The figure
therefore shows a convenient way of modelling the occasional random omission of steps in a workflow.

\begin{figure}
  \centering

\begin{lstlisting}
@fuzz( choose_from([ (0.95, identity), (0.05, remove_random_step) ]) ) def _enhance_system_quality( self, feature,
developer):

developer.add_test(feature) self._debug_feature(developer, feature) self._refactor_feature(developer, feature)
\end{lstlisting}
  
  \caption{Application of a composite fuzzer to a workflow.}
  \label{fig:composite}
\end{figure}

%%%%%%%%%%%%%%%%%%%%%%%%%%%%%%%%%%%%%%%%%%%%%%%%%%%%%%%%%%%%%%%%%%%%%%%%%%%%%%%%%%%%%%%%%%%%%%%%%%%%%%%%%%%%%%%%%%%%%%%%

\subsection{Control Structure Fuzzers}

%%%%%%%%%%%%%%%%%%%%%%%%%%%%%%%%%%%%%%%%%%%%%%%%%%%%%%%%%%%%%%%%%%%%%%%%%%%%%%%%%%%%%%%%%%%%%%%%%%%%%%%%%%%%%%%%%%%%%%%%

The final set of fuzzers provide for manipulation of control structure conditions and iterators.  Both of these fuzzers
inspect the input list of statements for compatible statement types to fuzz and apply the fuzzing to every one found.

\begin{FunctionList}

\item \lstinline!replace_condition_with(condition)! replaces conditions discovered in control structure statements with
  the specified condition.  The condition may be a literal Boolean value, a function pointer or a Python expression
  defined in a string.  Lambda expressions are not supported.

\item \lstinline!replace_for_iterator_with(iterator)! replaces iterators discovered in for loops with the specified
  iterator.  The replacement iterator must be specified as a list containing numerical or string literal types only.

\end{FunctionList}

Control structure fuzzers are useful for introducing variability in decision making into a workflow.  For example, a
developer may decide to stop working on refactoring a feature before the number of dependencies on other parts of a
system are reduced below the specified level, even if this is required by the idealised workflow.  Similarly, a
developer may skip the application of a task to a feature through an error of omission.

%%%%%%%%%%%%%%%%%%%%%%%%%%%%%%%%%%%%%%%%%%%%%%%%%%%%%%%%%%%%%%%%%%%%%%%%%%%%%%%%%%%%%%%%%%%%%%%%%%%%%%%%%%%%%%%%%%%%%%%%

\subsection{Defining and Evaluating Scenarios}

%%%%%%%%%%%%%%%%%%%%%%%%%%%%%%%%%%%%%%%%%%%%%%%%%%%%%%%%%%%%%%%%%%%%%%%%%%%%%%%%%%%%%%%%%%%%%%%%%%%%%%%%%%%%%%%%%%%%%%%%

Experiments in Fuzzi Moss are created by defining scenarios with initial conditions (available resources, participants
and so on), as well as a schedule of work items.  The schedule of work items is specified as part of the problem domain,
so may be altered during the execution of a workflow, or as a result of external events during a simulation.  For
example, the specified set of features for a software system may be altered as development proceeds and requirements
become better understood.

By convention, workflows are organised as Python classes, and contain a top-level \lstinline!work()! method to initiate
activity.  Invoking this method causes the simulation to be executed.  Additional behaviour may also be simulated in
order to evaluate the behaviour of the workflow.  For example, a software system may be operated within a simulation
after a development workflow has been executed in order to estimate quality assurance characteristics such as mean
operations to failure.

The Fuzzi Moss package exposes variables that can be used to configure the global behaviour of the fuzz decorator and
support the management of experiments. Specifically:

\begin{itemize}
\item A Python random object, \lstinline!fuzzi_moss_random!.  This object is used as a source of randomness where
  required by the Fuzzi Moss fuzzers.  The object can be seeded or even completely replaced as desired by the modeller.

\item A Boolean variable, \lstinline!enable_fuzzings! is provided, which can be used to control whether fuzzings are
  applied.  This feature is useful for executing the fuzzed workflows in their idealised state, since this provides for
  comparison with fuzzed workflows.

\end{itemize}

The dynamic application of fuzzings to socio-technical behaviours in Fuzzi Moss, mean that different sets of fuzzings
may be applied to different runs of the same scenario.  Many problem domains will also incorporate some form of
probabilistic functionality in order to represent stochastic elements of a problem domain.  Therefore, it is anticipated
that the characteristics of a workflow should be measured across multiple runs of the same scenario.

%%%%%%%%%%%%%%%%%%%%%%%%%%%%%%%%%%%%%%%%%%%%%%%%%%%%%%%%%%%%%%%%%%%%%%%%%%%%%%%%%%%%%%%%%%%%%%%%%%%%%%%%%%%%%%%%%%%%%%%%

\section{Evaluation}
\label{sec:evaluation}

\begin{figure*}
  \centering
    %% Creator: Matplotlib, PGF backend
%%
%% To include the figure in your LaTeX document, write
%%   \input{<filename>.pgf}
%%
%% Make sure the required packages are loaded in your preamble
%%   \usepackage{pgf}
%%
%% Figures using additional raster images can only be included by \input if
%% they are in the same directory as the main LaTeX file. For loading figures
%% from other directories you can use the `import` package
%%   \usepackage{import}
%% and then include the figures with
%%   \import{<path to file>}{<filename>.pgf}
%%
%% Matplotlib used the following preamble
%%   \usepackage{fontspec}
%%   \setmainfont{Times New Roman}
%%   \setsansfont{Verdana}
%%   \setmonofont{Courier New}
%%
\begingroup%
\makeatletter%
\begin{pgfpicture}%
\pgfpathrectangle{\pgfpointorigin}{\pgfqpoint{6.880358in}{2.989457in}}%
\pgfusepath{use as bounding box, clip}%
\begin{pgfscope}%
\pgfsetbuttcap%
\pgfsetmiterjoin%
\definecolor{currentfill}{rgb}{1.000000,1.000000,1.000000}%
\pgfsetfillcolor{currentfill}%
\pgfsetlinewidth{0.000000pt}%
\definecolor{currentstroke}{rgb}{1.000000,1.000000,1.000000}%
\pgfsetstrokecolor{currentstroke}%
\pgfsetdash{}{0pt}%
\pgfpathmoveto{\pgfqpoint{0.000000in}{0.000000in}}%
\pgfpathlineto{\pgfqpoint{6.880358in}{0.000000in}}%
\pgfpathlineto{\pgfqpoint{6.880358in}{2.989457in}}%
\pgfpathlineto{\pgfqpoint{0.000000in}{2.989457in}}%
\pgfpathclose%
\pgfusepath{fill}%
\end{pgfscope}%
\begin{pgfscope}%
\pgfsetbuttcap%
\pgfsetmiterjoin%
\definecolor{currentfill}{rgb}{1.000000,1.000000,1.000000}%
\pgfsetfillcolor{currentfill}%
\pgfsetlinewidth{0.000000pt}%
\definecolor{currentstroke}{rgb}{0.000000,0.000000,0.000000}%
\pgfsetstrokecolor{currentstroke}%
\pgfsetstrokeopacity{0.000000}%
\pgfsetdash{}{0pt}%
\pgfpathmoveto{\pgfqpoint{0.457963in}{0.528059in}}%
\pgfpathlineto{\pgfqpoint{6.657963in}{0.528059in}}%
\pgfpathlineto{\pgfqpoint{6.657963in}{2.813774in}}%
\pgfpathlineto{\pgfqpoint{0.457963in}{2.813774in}}%
\pgfpathclose%
\pgfusepath{fill}%
\end{pgfscope}%
\begin{pgfscope}%
\pgfpathrectangle{\pgfqpoint{0.457963in}{0.528059in}}{\pgfqpoint{6.200000in}{2.285714in}} %
\pgfusepath{clip}%
\pgfsetbuttcap%
\pgfsetroundjoin%
\definecolor{currentfill}{rgb}{0.833333,0.833333,1.000000}%
\pgfsetfillcolor{currentfill}%
\pgfsetlinewidth{1.003750pt}%
\definecolor{currentstroke}{rgb}{0.833333,0.833333,1.000000}%
\pgfsetstrokecolor{currentstroke}%
\pgfsetdash{}{0pt}%
\pgfpathmoveto{\pgfqpoint{0.457963in}{0.823534in}}%
\pgfpathcurveto{\pgfqpoint{0.466200in}{0.823534in}}{\pgfqpoint{0.474100in}{0.826806in}}{\pgfqpoint{0.479924in}{0.832630in}}%
\pgfpathcurveto{\pgfqpoint{0.485748in}{0.838454in}}{\pgfqpoint{0.489020in}{0.846354in}}{\pgfqpoint{0.489020in}{0.854590in}}%
\pgfpathcurveto{\pgfqpoint{0.489020in}{0.862826in}}{\pgfqpoint{0.485748in}{0.870726in}}{\pgfqpoint{0.479924in}{0.876550in}}%
\pgfpathcurveto{\pgfqpoint{0.474100in}{0.882374in}}{\pgfqpoint{0.466200in}{0.885647in}}{\pgfqpoint{0.457963in}{0.885647in}}%
\pgfpathcurveto{\pgfqpoint{0.449727in}{0.885647in}}{\pgfqpoint{0.441827in}{0.882374in}}{\pgfqpoint{0.436003in}{0.876550in}}%
\pgfpathcurveto{\pgfqpoint{0.430179in}{0.870726in}}{\pgfqpoint{0.426907in}{0.862826in}}{\pgfqpoint{0.426907in}{0.854590in}}%
\pgfpathcurveto{\pgfqpoint{0.426907in}{0.846354in}}{\pgfqpoint{0.430179in}{0.838454in}}{\pgfqpoint{0.436003in}{0.832630in}}%
\pgfpathcurveto{\pgfqpoint{0.441827in}{0.826806in}}{\pgfqpoint{0.449727in}{0.823534in}}{\pgfqpoint{0.457963in}{0.823534in}}%
\pgfpathclose%
\pgfusepath{stroke,fill}%
\end{pgfscope}%
\begin{pgfscope}%
\pgfpathrectangle{\pgfqpoint{0.457963in}{0.528059in}}{\pgfqpoint{6.200000in}{2.285714in}} %
\pgfusepath{clip}%
\pgfsetbuttcap%
\pgfsetroundjoin%
\definecolor{currentfill}{rgb}{0.833333,0.833333,1.000000}%
\pgfsetfillcolor{currentfill}%
\pgfsetlinewidth{1.003750pt}%
\definecolor{currentstroke}{rgb}{0.833333,0.833333,1.000000}%
\pgfsetstrokecolor{currentstroke}%
\pgfsetdash{}{0pt}%
\pgfpathmoveto{\pgfqpoint{0.457963in}{0.823534in}}%
\pgfpathcurveto{\pgfqpoint{0.466200in}{0.823534in}}{\pgfqpoint{0.474100in}{0.826806in}}{\pgfqpoint{0.479924in}{0.832630in}}%
\pgfpathcurveto{\pgfqpoint{0.485748in}{0.838454in}}{\pgfqpoint{0.489020in}{0.846354in}}{\pgfqpoint{0.489020in}{0.854590in}}%
\pgfpathcurveto{\pgfqpoint{0.489020in}{0.862826in}}{\pgfqpoint{0.485748in}{0.870726in}}{\pgfqpoint{0.479924in}{0.876550in}}%
\pgfpathcurveto{\pgfqpoint{0.474100in}{0.882374in}}{\pgfqpoint{0.466200in}{0.885647in}}{\pgfqpoint{0.457963in}{0.885647in}}%
\pgfpathcurveto{\pgfqpoint{0.449727in}{0.885647in}}{\pgfqpoint{0.441827in}{0.882374in}}{\pgfqpoint{0.436003in}{0.876550in}}%
\pgfpathcurveto{\pgfqpoint{0.430179in}{0.870726in}}{\pgfqpoint{0.426907in}{0.862826in}}{\pgfqpoint{0.426907in}{0.854590in}}%
\pgfpathcurveto{\pgfqpoint{0.426907in}{0.846354in}}{\pgfqpoint{0.430179in}{0.838454in}}{\pgfqpoint{0.436003in}{0.832630in}}%
\pgfpathcurveto{\pgfqpoint{0.441827in}{0.826806in}}{\pgfqpoint{0.449727in}{0.823534in}}{\pgfqpoint{0.457963in}{0.823534in}}%
\pgfpathclose%
\pgfusepath{stroke,fill}%
\end{pgfscope}%
\begin{pgfscope}%
\pgfpathrectangle{\pgfqpoint{0.457963in}{0.528059in}}{\pgfqpoint{6.200000in}{2.285714in}} %
\pgfusepath{clip}%
\pgfsetbuttcap%
\pgfsetroundjoin%
\definecolor{currentfill}{rgb}{0.833333,0.833333,1.000000}%
\pgfsetfillcolor{currentfill}%
\pgfsetlinewidth{1.003750pt}%
\definecolor{currentstroke}{rgb}{0.833333,0.833333,1.000000}%
\pgfsetstrokecolor{currentstroke}%
\pgfsetdash{}{0pt}%
\pgfpathmoveto{\pgfqpoint{0.457963in}{0.823534in}}%
\pgfpathcurveto{\pgfqpoint{0.466200in}{0.823534in}}{\pgfqpoint{0.474100in}{0.826806in}}{\pgfqpoint{0.479924in}{0.832630in}}%
\pgfpathcurveto{\pgfqpoint{0.485748in}{0.838454in}}{\pgfqpoint{0.489020in}{0.846354in}}{\pgfqpoint{0.489020in}{0.854590in}}%
\pgfpathcurveto{\pgfqpoint{0.489020in}{0.862826in}}{\pgfqpoint{0.485748in}{0.870726in}}{\pgfqpoint{0.479924in}{0.876550in}}%
\pgfpathcurveto{\pgfqpoint{0.474100in}{0.882374in}}{\pgfqpoint{0.466200in}{0.885647in}}{\pgfqpoint{0.457963in}{0.885647in}}%
\pgfpathcurveto{\pgfqpoint{0.449727in}{0.885647in}}{\pgfqpoint{0.441827in}{0.882374in}}{\pgfqpoint{0.436003in}{0.876550in}}%
\pgfpathcurveto{\pgfqpoint{0.430179in}{0.870726in}}{\pgfqpoint{0.426907in}{0.862826in}}{\pgfqpoint{0.426907in}{0.854590in}}%
\pgfpathcurveto{\pgfqpoint{0.426907in}{0.846354in}}{\pgfqpoint{0.430179in}{0.838454in}}{\pgfqpoint{0.436003in}{0.832630in}}%
\pgfpathcurveto{\pgfqpoint{0.441827in}{0.826806in}}{\pgfqpoint{0.449727in}{0.823534in}}{\pgfqpoint{0.457963in}{0.823534in}}%
\pgfpathclose%
\pgfusepath{stroke,fill}%
\end{pgfscope}%
\begin{pgfscope}%
\pgfpathrectangle{\pgfqpoint{0.457963in}{0.528059in}}{\pgfqpoint{6.200000in}{2.285714in}} %
\pgfusepath{clip}%
\pgfsetbuttcap%
\pgfsetroundjoin%
\definecolor{currentfill}{rgb}{0.833333,0.833333,1.000000}%
\pgfsetfillcolor{currentfill}%
\pgfsetlinewidth{1.003750pt}%
\definecolor{currentstroke}{rgb}{0.833333,0.833333,1.000000}%
\pgfsetstrokecolor{currentstroke}%
\pgfsetdash{}{0pt}%
\pgfpathmoveto{\pgfqpoint{0.457963in}{0.823534in}}%
\pgfpathcurveto{\pgfqpoint{0.466200in}{0.823534in}}{\pgfqpoint{0.474100in}{0.826806in}}{\pgfqpoint{0.479924in}{0.832630in}}%
\pgfpathcurveto{\pgfqpoint{0.485748in}{0.838454in}}{\pgfqpoint{0.489020in}{0.846354in}}{\pgfqpoint{0.489020in}{0.854590in}}%
\pgfpathcurveto{\pgfqpoint{0.489020in}{0.862826in}}{\pgfqpoint{0.485748in}{0.870726in}}{\pgfqpoint{0.479924in}{0.876550in}}%
\pgfpathcurveto{\pgfqpoint{0.474100in}{0.882374in}}{\pgfqpoint{0.466200in}{0.885647in}}{\pgfqpoint{0.457963in}{0.885647in}}%
\pgfpathcurveto{\pgfqpoint{0.449727in}{0.885647in}}{\pgfqpoint{0.441827in}{0.882374in}}{\pgfqpoint{0.436003in}{0.876550in}}%
\pgfpathcurveto{\pgfqpoint{0.430179in}{0.870726in}}{\pgfqpoint{0.426907in}{0.862826in}}{\pgfqpoint{0.426907in}{0.854590in}}%
\pgfpathcurveto{\pgfqpoint{0.426907in}{0.846354in}}{\pgfqpoint{0.430179in}{0.838454in}}{\pgfqpoint{0.436003in}{0.832630in}}%
\pgfpathcurveto{\pgfqpoint{0.441827in}{0.826806in}}{\pgfqpoint{0.449727in}{0.823534in}}{\pgfqpoint{0.457963in}{0.823534in}}%
\pgfpathclose%
\pgfusepath{stroke,fill}%
\end{pgfscope}%
\begin{pgfscope}%
\pgfpathrectangle{\pgfqpoint{0.457963in}{0.528059in}}{\pgfqpoint{6.200000in}{2.285714in}} %
\pgfusepath{clip}%
\pgfsetbuttcap%
\pgfsetroundjoin%
\definecolor{currentfill}{rgb}{0.833333,0.833333,1.000000}%
\pgfsetfillcolor{currentfill}%
\pgfsetlinewidth{1.003750pt}%
\definecolor{currentstroke}{rgb}{0.833333,0.833333,1.000000}%
\pgfsetstrokecolor{currentstroke}%
\pgfsetdash{}{0pt}%
\pgfpathmoveto{\pgfqpoint{0.457963in}{0.823534in}}%
\pgfpathcurveto{\pgfqpoint{0.466200in}{0.823534in}}{\pgfqpoint{0.474100in}{0.826806in}}{\pgfqpoint{0.479924in}{0.832630in}}%
\pgfpathcurveto{\pgfqpoint{0.485748in}{0.838454in}}{\pgfqpoint{0.489020in}{0.846354in}}{\pgfqpoint{0.489020in}{0.854590in}}%
\pgfpathcurveto{\pgfqpoint{0.489020in}{0.862826in}}{\pgfqpoint{0.485748in}{0.870726in}}{\pgfqpoint{0.479924in}{0.876550in}}%
\pgfpathcurveto{\pgfqpoint{0.474100in}{0.882374in}}{\pgfqpoint{0.466200in}{0.885647in}}{\pgfqpoint{0.457963in}{0.885647in}}%
\pgfpathcurveto{\pgfqpoint{0.449727in}{0.885647in}}{\pgfqpoint{0.441827in}{0.882374in}}{\pgfqpoint{0.436003in}{0.876550in}}%
\pgfpathcurveto{\pgfqpoint{0.430179in}{0.870726in}}{\pgfqpoint{0.426907in}{0.862826in}}{\pgfqpoint{0.426907in}{0.854590in}}%
\pgfpathcurveto{\pgfqpoint{0.426907in}{0.846354in}}{\pgfqpoint{0.430179in}{0.838454in}}{\pgfqpoint{0.436003in}{0.832630in}}%
\pgfpathcurveto{\pgfqpoint{0.441827in}{0.826806in}}{\pgfqpoint{0.449727in}{0.823534in}}{\pgfqpoint{0.457963in}{0.823534in}}%
\pgfpathclose%
\pgfusepath{stroke,fill}%
\end{pgfscope}%
\begin{pgfscope}%
\pgfpathrectangle{\pgfqpoint{0.457963in}{0.528059in}}{\pgfqpoint{6.200000in}{2.285714in}} %
\pgfusepath{clip}%
\pgfsetbuttcap%
\pgfsetroundjoin%
\definecolor{currentfill}{rgb}{0.833333,0.833333,1.000000}%
\pgfsetfillcolor{currentfill}%
\pgfsetlinewidth{1.003750pt}%
\definecolor{currentstroke}{rgb}{0.833333,0.833333,1.000000}%
\pgfsetstrokecolor{currentstroke}%
\pgfsetdash{}{0pt}%
\pgfpathmoveto{\pgfqpoint{0.457963in}{0.823534in}}%
\pgfpathcurveto{\pgfqpoint{0.466200in}{0.823534in}}{\pgfqpoint{0.474100in}{0.826806in}}{\pgfqpoint{0.479924in}{0.832630in}}%
\pgfpathcurveto{\pgfqpoint{0.485748in}{0.838454in}}{\pgfqpoint{0.489020in}{0.846354in}}{\pgfqpoint{0.489020in}{0.854590in}}%
\pgfpathcurveto{\pgfqpoint{0.489020in}{0.862826in}}{\pgfqpoint{0.485748in}{0.870726in}}{\pgfqpoint{0.479924in}{0.876550in}}%
\pgfpathcurveto{\pgfqpoint{0.474100in}{0.882374in}}{\pgfqpoint{0.466200in}{0.885647in}}{\pgfqpoint{0.457963in}{0.885647in}}%
\pgfpathcurveto{\pgfqpoint{0.449727in}{0.885647in}}{\pgfqpoint{0.441827in}{0.882374in}}{\pgfqpoint{0.436003in}{0.876550in}}%
\pgfpathcurveto{\pgfqpoint{0.430179in}{0.870726in}}{\pgfqpoint{0.426907in}{0.862826in}}{\pgfqpoint{0.426907in}{0.854590in}}%
\pgfpathcurveto{\pgfqpoint{0.426907in}{0.846354in}}{\pgfqpoint{0.430179in}{0.838454in}}{\pgfqpoint{0.436003in}{0.832630in}}%
\pgfpathcurveto{\pgfqpoint{0.441827in}{0.826806in}}{\pgfqpoint{0.449727in}{0.823534in}}{\pgfqpoint{0.457963in}{0.823534in}}%
\pgfpathclose%
\pgfusepath{stroke,fill}%
\end{pgfscope}%
\begin{pgfscope}%
\pgfpathrectangle{\pgfqpoint{0.457963in}{0.528059in}}{\pgfqpoint{6.200000in}{2.285714in}} %
\pgfusepath{clip}%
\pgfsetbuttcap%
\pgfsetroundjoin%
\definecolor{currentfill}{rgb}{0.833333,0.833333,1.000000}%
\pgfsetfillcolor{currentfill}%
\pgfsetlinewidth{1.003750pt}%
\definecolor{currentstroke}{rgb}{0.833333,0.833333,1.000000}%
\pgfsetstrokecolor{currentstroke}%
\pgfsetdash{}{0pt}%
\pgfpathmoveto{\pgfqpoint{0.468297in}{0.823534in}}%
\pgfpathcurveto{\pgfqpoint{0.476533in}{0.823534in}}{\pgfqpoint{0.484433in}{0.826806in}}{\pgfqpoint{0.490257in}{0.832630in}}%
\pgfpathcurveto{\pgfqpoint{0.496081in}{0.838454in}}{\pgfqpoint{0.499353in}{0.846354in}}{\pgfqpoint{0.499353in}{0.854590in}}%
\pgfpathcurveto{\pgfqpoint{0.499353in}{0.862826in}}{\pgfqpoint{0.496081in}{0.870726in}}{\pgfqpoint{0.490257in}{0.876550in}}%
\pgfpathcurveto{\pgfqpoint{0.484433in}{0.882374in}}{\pgfqpoint{0.476533in}{0.885647in}}{\pgfqpoint{0.468297in}{0.885647in}}%
\pgfpathcurveto{\pgfqpoint{0.460060in}{0.885647in}}{\pgfqpoint{0.452160in}{0.882374in}}{\pgfqpoint{0.446336in}{0.876550in}}%
\pgfpathcurveto{\pgfqpoint{0.440512in}{0.870726in}}{\pgfqpoint{0.437240in}{0.862826in}}{\pgfqpoint{0.437240in}{0.854590in}}%
\pgfpathcurveto{\pgfqpoint{0.437240in}{0.846354in}}{\pgfqpoint{0.440512in}{0.838454in}}{\pgfqpoint{0.446336in}{0.832630in}}%
\pgfpathcurveto{\pgfqpoint{0.452160in}{0.826806in}}{\pgfqpoint{0.460060in}{0.823534in}}{\pgfqpoint{0.468297in}{0.823534in}}%
\pgfpathclose%
\pgfusepath{stroke,fill}%
\end{pgfscope}%
\begin{pgfscope}%
\pgfpathrectangle{\pgfqpoint{0.457963in}{0.528059in}}{\pgfqpoint{6.200000in}{2.285714in}} %
\pgfusepath{clip}%
\pgfsetbuttcap%
\pgfsetroundjoin%
\definecolor{currentfill}{rgb}{0.833333,0.833333,1.000000}%
\pgfsetfillcolor{currentfill}%
\pgfsetlinewidth{1.003750pt}%
\definecolor{currentstroke}{rgb}{0.833333,0.833333,1.000000}%
\pgfsetstrokecolor{currentstroke}%
\pgfsetdash{}{0pt}%
\pgfpathmoveto{\pgfqpoint{0.468297in}{0.823534in}}%
\pgfpathcurveto{\pgfqpoint{0.476533in}{0.823534in}}{\pgfqpoint{0.484433in}{0.826806in}}{\pgfqpoint{0.490257in}{0.832630in}}%
\pgfpathcurveto{\pgfqpoint{0.496081in}{0.838454in}}{\pgfqpoint{0.499353in}{0.846354in}}{\pgfqpoint{0.499353in}{0.854590in}}%
\pgfpathcurveto{\pgfqpoint{0.499353in}{0.862826in}}{\pgfqpoint{0.496081in}{0.870726in}}{\pgfqpoint{0.490257in}{0.876550in}}%
\pgfpathcurveto{\pgfqpoint{0.484433in}{0.882374in}}{\pgfqpoint{0.476533in}{0.885647in}}{\pgfqpoint{0.468297in}{0.885647in}}%
\pgfpathcurveto{\pgfqpoint{0.460060in}{0.885647in}}{\pgfqpoint{0.452160in}{0.882374in}}{\pgfqpoint{0.446336in}{0.876550in}}%
\pgfpathcurveto{\pgfqpoint{0.440512in}{0.870726in}}{\pgfqpoint{0.437240in}{0.862826in}}{\pgfqpoint{0.437240in}{0.854590in}}%
\pgfpathcurveto{\pgfqpoint{0.437240in}{0.846354in}}{\pgfqpoint{0.440512in}{0.838454in}}{\pgfqpoint{0.446336in}{0.832630in}}%
\pgfpathcurveto{\pgfqpoint{0.452160in}{0.826806in}}{\pgfqpoint{0.460060in}{0.823534in}}{\pgfqpoint{0.468297in}{0.823534in}}%
\pgfpathclose%
\pgfusepath{stroke,fill}%
\end{pgfscope}%
\begin{pgfscope}%
\pgfpathrectangle{\pgfqpoint{0.457963in}{0.528059in}}{\pgfqpoint{6.200000in}{2.285714in}} %
\pgfusepath{clip}%
\pgfsetbuttcap%
\pgfsetroundjoin%
\definecolor{currentfill}{rgb}{0.833333,0.833333,1.000000}%
\pgfsetfillcolor{currentfill}%
\pgfsetlinewidth{1.003750pt}%
\definecolor{currentstroke}{rgb}{0.833333,0.833333,1.000000}%
\pgfsetstrokecolor{currentstroke}%
\pgfsetdash{}{0pt}%
\pgfpathmoveto{\pgfqpoint{0.519963in}{0.823534in}}%
\pgfpathcurveto{\pgfqpoint{0.528200in}{0.823534in}}{\pgfqpoint{0.536100in}{0.826806in}}{\pgfqpoint{0.541924in}{0.832630in}}%
\pgfpathcurveto{\pgfqpoint{0.547748in}{0.838454in}}{\pgfqpoint{0.551020in}{0.846354in}}{\pgfqpoint{0.551020in}{0.854590in}}%
\pgfpathcurveto{\pgfqpoint{0.551020in}{0.862826in}}{\pgfqpoint{0.547748in}{0.870726in}}{\pgfqpoint{0.541924in}{0.876550in}}%
\pgfpathcurveto{\pgfqpoint{0.536100in}{0.882374in}}{\pgfqpoint{0.528200in}{0.885647in}}{\pgfqpoint{0.519963in}{0.885647in}}%
\pgfpathcurveto{\pgfqpoint{0.511727in}{0.885647in}}{\pgfqpoint{0.503827in}{0.882374in}}{\pgfqpoint{0.498003in}{0.876550in}}%
\pgfpathcurveto{\pgfqpoint{0.492179in}{0.870726in}}{\pgfqpoint{0.488907in}{0.862826in}}{\pgfqpoint{0.488907in}{0.854590in}}%
\pgfpathcurveto{\pgfqpoint{0.488907in}{0.846354in}}{\pgfqpoint{0.492179in}{0.838454in}}{\pgfqpoint{0.498003in}{0.832630in}}%
\pgfpathcurveto{\pgfqpoint{0.503827in}{0.826806in}}{\pgfqpoint{0.511727in}{0.823534in}}{\pgfqpoint{0.519963in}{0.823534in}}%
\pgfpathclose%
\pgfusepath{stroke,fill}%
\end{pgfscope}%
\begin{pgfscope}%
\pgfpathrectangle{\pgfqpoint{0.457963in}{0.528059in}}{\pgfqpoint{6.200000in}{2.285714in}} %
\pgfusepath{clip}%
\pgfsetbuttcap%
\pgfsetroundjoin%
\definecolor{currentfill}{rgb}{0.833333,0.833333,1.000000}%
\pgfsetfillcolor{currentfill}%
\pgfsetlinewidth{1.003750pt}%
\definecolor{currentstroke}{rgb}{0.833333,0.833333,1.000000}%
\pgfsetstrokecolor{currentstroke}%
\pgfsetdash{}{0pt}%
\pgfpathmoveto{\pgfqpoint{0.561297in}{0.823534in}}%
\pgfpathcurveto{\pgfqpoint{0.569533in}{0.823534in}}{\pgfqpoint{0.577433in}{0.826806in}}{\pgfqpoint{0.583257in}{0.832630in}}%
\pgfpathcurveto{\pgfqpoint{0.589081in}{0.838454in}}{\pgfqpoint{0.592353in}{0.846354in}}{\pgfqpoint{0.592353in}{0.854590in}}%
\pgfpathcurveto{\pgfqpoint{0.592353in}{0.862826in}}{\pgfqpoint{0.589081in}{0.870726in}}{\pgfqpoint{0.583257in}{0.876550in}}%
\pgfpathcurveto{\pgfqpoint{0.577433in}{0.882374in}}{\pgfqpoint{0.569533in}{0.885647in}}{\pgfqpoint{0.561297in}{0.885647in}}%
\pgfpathcurveto{\pgfqpoint{0.553060in}{0.885647in}}{\pgfqpoint{0.545160in}{0.882374in}}{\pgfqpoint{0.539336in}{0.876550in}}%
\pgfpathcurveto{\pgfqpoint{0.533512in}{0.870726in}}{\pgfqpoint{0.530240in}{0.862826in}}{\pgfqpoint{0.530240in}{0.854590in}}%
\pgfpathcurveto{\pgfqpoint{0.530240in}{0.846354in}}{\pgfqpoint{0.533512in}{0.838454in}}{\pgfqpoint{0.539336in}{0.832630in}}%
\pgfpathcurveto{\pgfqpoint{0.545160in}{0.826806in}}{\pgfqpoint{0.553060in}{0.823534in}}{\pgfqpoint{0.561297in}{0.823534in}}%
\pgfpathclose%
\pgfusepath{stroke,fill}%
\end{pgfscope}%
\begin{pgfscope}%
\pgfpathrectangle{\pgfqpoint{0.457963in}{0.528059in}}{\pgfqpoint{6.200000in}{2.285714in}} %
\pgfusepath{clip}%
\pgfsetbuttcap%
\pgfsetroundjoin%
\definecolor{currentfill}{rgb}{0.833333,0.833333,1.000000}%
\pgfsetfillcolor{currentfill}%
\pgfsetlinewidth{1.003750pt}%
\definecolor{currentstroke}{rgb}{0.833333,0.833333,1.000000}%
\pgfsetstrokecolor{currentstroke}%
\pgfsetdash{}{0pt}%
\pgfpathmoveto{\pgfqpoint{0.602630in}{0.810472in}}%
\pgfpathcurveto{\pgfqpoint{0.610866in}{0.810472in}}{\pgfqpoint{0.618766in}{0.813745in}}{\pgfqpoint{0.624590in}{0.819569in}}%
\pgfpathcurveto{\pgfqpoint{0.630414in}{0.825393in}}{\pgfqpoint{0.633686in}{0.833293in}}{\pgfqpoint{0.633686in}{0.841529in}}%
\pgfpathcurveto{\pgfqpoint{0.633686in}{0.849765in}}{\pgfqpoint{0.630414in}{0.857665in}}{\pgfqpoint{0.624590in}{0.863489in}}%
\pgfpathcurveto{\pgfqpoint{0.618766in}{0.869313in}}{\pgfqpoint{0.610866in}{0.872585in}}{\pgfqpoint{0.602630in}{0.872585in}}%
\pgfpathcurveto{\pgfqpoint{0.594394in}{0.872585in}}{\pgfqpoint{0.586494in}{0.869313in}}{\pgfqpoint{0.580670in}{0.863489in}}%
\pgfpathcurveto{\pgfqpoint{0.574846in}{0.857665in}}{\pgfqpoint{0.571574in}{0.849765in}}{\pgfqpoint{0.571574in}{0.841529in}}%
\pgfpathcurveto{\pgfqpoint{0.571574in}{0.833293in}}{\pgfqpoint{0.574846in}{0.825393in}}{\pgfqpoint{0.580670in}{0.819569in}}%
\pgfpathcurveto{\pgfqpoint{0.586494in}{0.813745in}}{\pgfqpoint{0.594394in}{0.810472in}}{\pgfqpoint{0.602630in}{0.810472in}}%
\pgfpathclose%
\pgfusepath{stroke,fill}%
\end{pgfscope}%
\begin{pgfscope}%
\pgfpathrectangle{\pgfqpoint{0.457963in}{0.528059in}}{\pgfqpoint{6.200000in}{2.285714in}} %
\pgfusepath{clip}%
\pgfsetbuttcap%
\pgfsetroundjoin%
\definecolor{currentfill}{rgb}{0.833333,0.833333,1.000000}%
\pgfsetfillcolor{currentfill}%
\pgfsetlinewidth{1.003750pt}%
\definecolor{currentstroke}{rgb}{0.833333,0.833333,1.000000}%
\pgfsetstrokecolor{currentstroke}%
\pgfsetdash{}{0pt}%
\pgfpathmoveto{\pgfqpoint{0.612963in}{0.810472in}}%
\pgfpathcurveto{\pgfqpoint{0.621200in}{0.810472in}}{\pgfqpoint{0.629100in}{0.813745in}}{\pgfqpoint{0.634924in}{0.819569in}}%
\pgfpathcurveto{\pgfqpoint{0.640748in}{0.825393in}}{\pgfqpoint{0.644020in}{0.833293in}}{\pgfqpoint{0.644020in}{0.841529in}}%
\pgfpathcurveto{\pgfqpoint{0.644020in}{0.849765in}}{\pgfqpoint{0.640748in}{0.857665in}}{\pgfqpoint{0.634924in}{0.863489in}}%
\pgfpathcurveto{\pgfqpoint{0.629100in}{0.869313in}}{\pgfqpoint{0.621200in}{0.872585in}}{\pgfqpoint{0.612963in}{0.872585in}}%
\pgfpathcurveto{\pgfqpoint{0.604727in}{0.872585in}}{\pgfqpoint{0.596827in}{0.869313in}}{\pgfqpoint{0.591003in}{0.863489in}}%
\pgfpathcurveto{\pgfqpoint{0.585179in}{0.857665in}}{\pgfqpoint{0.581907in}{0.849765in}}{\pgfqpoint{0.581907in}{0.841529in}}%
\pgfpathcurveto{\pgfqpoint{0.581907in}{0.833293in}}{\pgfqpoint{0.585179in}{0.825393in}}{\pgfqpoint{0.591003in}{0.819569in}}%
\pgfpathcurveto{\pgfqpoint{0.596827in}{0.813745in}}{\pgfqpoint{0.604727in}{0.810472in}}{\pgfqpoint{0.612963in}{0.810472in}}%
\pgfpathclose%
\pgfusepath{stroke,fill}%
\end{pgfscope}%
\begin{pgfscope}%
\pgfpathrectangle{\pgfqpoint{0.457963in}{0.528059in}}{\pgfqpoint{6.200000in}{2.285714in}} %
\pgfusepath{clip}%
\pgfsetbuttcap%
\pgfsetroundjoin%
\definecolor{currentfill}{rgb}{0.833333,0.833333,1.000000}%
\pgfsetfillcolor{currentfill}%
\pgfsetlinewidth{1.003750pt}%
\definecolor{currentstroke}{rgb}{0.833333,0.833333,1.000000}%
\pgfsetstrokecolor{currentstroke}%
\pgfsetdash{}{0pt}%
\pgfpathmoveto{\pgfqpoint{0.612963in}{0.823534in}}%
\pgfpathcurveto{\pgfqpoint{0.621200in}{0.823534in}}{\pgfqpoint{0.629100in}{0.826806in}}{\pgfqpoint{0.634924in}{0.832630in}}%
\pgfpathcurveto{\pgfqpoint{0.640748in}{0.838454in}}{\pgfqpoint{0.644020in}{0.846354in}}{\pgfqpoint{0.644020in}{0.854590in}}%
\pgfpathcurveto{\pgfqpoint{0.644020in}{0.862826in}}{\pgfqpoint{0.640748in}{0.870726in}}{\pgfqpoint{0.634924in}{0.876550in}}%
\pgfpathcurveto{\pgfqpoint{0.629100in}{0.882374in}}{\pgfqpoint{0.621200in}{0.885647in}}{\pgfqpoint{0.612963in}{0.885647in}}%
\pgfpathcurveto{\pgfqpoint{0.604727in}{0.885647in}}{\pgfqpoint{0.596827in}{0.882374in}}{\pgfqpoint{0.591003in}{0.876550in}}%
\pgfpathcurveto{\pgfqpoint{0.585179in}{0.870726in}}{\pgfqpoint{0.581907in}{0.862826in}}{\pgfqpoint{0.581907in}{0.854590in}}%
\pgfpathcurveto{\pgfqpoint{0.581907in}{0.846354in}}{\pgfqpoint{0.585179in}{0.838454in}}{\pgfqpoint{0.591003in}{0.832630in}}%
\pgfpathcurveto{\pgfqpoint{0.596827in}{0.826806in}}{\pgfqpoint{0.604727in}{0.823534in}}{\pgfqpoint{0.612963in}{0.823534in}}%
\pgfpathclose%
\pgfusepath{stroke,fill}%
\end{pgfscope}%
\begin{pgfscope}%
\pgfpathrectangle{\pgfqpoint{0.457963in}{0.528059in}}{\pgfqpoint{6.200000in}{2.285714in}} %
\pgfusepath{clip}%
\pgfsetbuttcap%
\pgfsetroundjoin%
\definecolor{currentfill}{rgb}{0.833333,0.833333,1.000000}%
\pgfsetfillcolor{currentfill}%
\pgfsetlinewidth{1.003750pt}%
\definecolor{currentstroke}{rgb}{0.833333,0.833333,1.000000}%
\pgfsetstrokecolor{currentstroke}%
\pgfsetdash{}{0pt}%
\pgfpathmoveto{\pgfqpoint{0.850630in}{0.771289in}}%
\pgfpathcurveto{\pgfqpoint{0.858866in}{0.771289in}}{\pgfqpoint{0.866766in}{0.774561in}}{\pgfqpoint{0.872590in}{0.780385in}}%
\pgfpathcurveto{\pgfqpoint{0.878414in}{0.786209in}}{\pgfqpoint{0.881686in}{0.794109in}}{\pgfqpoint{0.881686in}{0.802345in}}%
\pgfpathcurveto{\pgfqpoint{0.881686in}{0.810581in}}{\pgfqpoint{0.878414in}{0.818481in}}{\pgfqpoint{0.872590in}{0.824305in}}%
\pgfpathcurveto{\pgfqpoint{0.866766in}{0.830129in}}{\pgfqpoint{0.858866in}{0.833402in}}{\pgfqpoint{0.850630in}{0.833402in}}%
\pgfpathcurveto{\pgfqpoint{0.842394in}{0.833402in}}{\pgfqpoint{0.834494in}{0.830129in}}{\pgfqpoint{0.828670in}{0.824305in}}%
\pgfpathcurveto{\pgfqpoint{0.822846in}{0.818481in}}{\pgfqpoint{0.819574in}{0.810581in}}{\pgfqpoint{0.819574in}{0.802345in}}%
\pgfpathcurveto{\pgfqpoint{0.819574in}{0.794109in}}{\pgfqpoint{0.822846in}{0.786209in}}{\pgfqpoint{0.828670in}{0.780385in}}%
\pgfpathcurveto{\pgfqpoint{0.834494in}{0.774561in}}{\pgfqpoint{0.842394in}{0.771289in}}{\pgfqpoint{0.850630in}{0.771289in}}%
\pgfpathclose%
\pgfusepath{stroke,fill}%
\end{pgfscope}%
\begin{pgfscope}%
\pgfpathrectangle{\pgfqpoint{0.457963in}{0.528059in}}{\pgfqpoint{6.200000in}{2.285714in}} %
\pgfusepath{clip}%
\pgfsetbuttcap%
\pgfsetroundjoin%
\definecolor{currentfill}{rgb}{0.833333,0.833333,1.000000}%
\pgfsetfillcolor{currentfill}%
\pgfsetlinewidth{1.003750pt}%
\definecolor{currentstroke}{rgb}{0.833333,0.833333,1.000000}%
\pgfsetstrokecolor{currentstroke}%
\pgfsetdash{}{0pt}%
\pgfpathmoveto{\pgfqpoint{0.922963in}{0.758227in}}%
\pgfpathcurveto{\pgfqpoint{0.931200in}{0.758227in}}{\pgfqpoint{0.939100in}{0.761500in}}{\pgfqpoint{0.944924in}{0.767324in}}%
\pgfpathcurveto{\pgfqpoint{0.950748in}{0.773148in}}{\pgfqpoint{0.954020in}{0.781048in}}{\pgfqpoint{0.954020in}{0.789284in}}%
\pgfpathcurveto{\pgfqpoint{0.954020in}{0.797520in}}{\pgfqpoint{0.950748in}{0.805420in}}{\pgfqpoint{0.944924in}{0.811244in}}%
\pgfpathcurveto{\pgfqpoint{0.939100in}{0.817068in}}{\pgfqpoint{0.931200in}{0.820340in}}{\pgfqpoint{0.922963in}{0.820340in}}%
\pgfpathcurveto{\pgfqpoint{0.914727in}{0.820340in}}{\pgfqpoint{0.906827in}{0.817068in}}{\pgfqpoint{0.901003in}{0.811244in}}%
\pgfpathcurveto{\pgfqpoint{0.895179in}{0.805420in}}{\pgfqpoint{0.891907in}{0.797520in}}{\pgfqpoint{0.891907in}{0.789284in}}%
\pgfpathcurveto{\pgfqpoint{0.891907in}{0.781048in}}{\pgfqpoint{0.895179in}{0.773148in}}{\pgfqpoint{0.901003in}{0.767324in}}%
\pgfpathcurveto{\pgfqpoint{0.906827in}{0.761500in}}{\pgfqpoint{0.914727in}{0.758227in}}{\pgfqpoint{0.922963in}{0.758227in}}%
\pgfpathclose%
\pgfusepath{stroke,fill}%
\end{pgfscope}%
\begin{pgfscope}%
\pgfpathrectangle{\pgfqpoint{0.457963in}{0.528059in}}{\pgfqpoint{6.200000in}{2.285714in}} %
\pgfusepath{clip}%
\pgfsetbuttcap%
\pgfsetroundjoin%
\definecolor{currentfill}{rgb}{0.833333,0.833333,1.000000}%
\pgfsetfillcolor{currentfill}%
\pgfsetlinewidth{1.003750pt}%
\definecolor{currentstroke}{rgb}{0.833333,0.833333,1.000000}%
\pgfsetstrokecolor{currentstroke}%
\pgfsetdash{}{0pt}%
\pgfpathmoveto{\pgfqpoint{1.139963in}{0.810472in}}%
\pgfpathcurveto{\pgfqpoint{1.148200in}{0.810472in}}{\pgfqpoint{1.156100in}{0.813745in}}{\pgfqpoint{1.161924in}{0.819569in}}%
\pgfpathcurveto{\pgfqpoint{1.167748in}{0.825393in}}{\pgfqpoint{1.171020in}{0.833293in}}{\pgfqpoint{1.171020in}{0.841529in}}%
\pgfpathcurveto{\pgfqpoint{1.171020in}{0.849765in}}{\pgfqpoint{1.167748in}{0.857665in}}{\pgfqpoint{1.161924in}{0.863489in}}%
\pgfpathcurveto{\pgfqpoint{1.156100in}{0.869313in}}{\pgfqpoint{1.148200in}{0.872585in}}{\pgfqpoint{1.139963in}{0.872585in}}%
\pgfpathcurveto{\pgfqpoint{1.131727in}{0.872585in}}{\pgfqpoint{1.123827in}{0.869313in}}{\pgfqpoint{1.118003in}{0.863489in}}%
\pgfpathcurveto{\pgfqpoint{1.112179in}{0.857665in}}{\pgfqpoint{1.108907in}{0.849765in}}{\pgfqpoint{1.108907in}{0.841529in}}%
\pgfpathcurveto{\pgfqpoint{1.108907in}{0.833293in}}{\pgfqpoint{1.112179in}{0.825393in}}{\pgfqpoint{1.118003in}{0.819569in}}%
\pgfpathcurveto{\pgfqpoint{1.123827in}{0.813745in}}{\pgfqpoint{1.131727in}{0.810472in}}{\pgfqpoint{1.139963in}{0.810472in}}%
\pgfpathclose%
\pgfusepath{stroke,fill}%
\end{pgfscope}%
\begin{pgfscope}%
\pgfpathrectangle{\pgfqpoint{0.457963in}{0.528059in}}{\pgfqpoint{6.200000in}{2.285714in}} %
\pgfusepath{clip}%
\pgfsetbuttcap%
\pgfsetroundjoin%
\definecolor{currentfill}{rgb}{0.833333,0.833333,1.000000}%
\pgfsetfillcolor{currentfill}%
\pgfsetlinewidth{1.003750pt}%
\definecolor{currentstroke}{rgb}{0.833333,0.833333,1.000000}%
\pgfsetstrokecolor{currentstroke}%
\pgfsetdash{}{0pt}%
\pgfpathmoveto{\pgfqpoint{1.170963in}{0.823534in}}%
\pgfpathcurveto{\pgfqpoint{1.179200in}{0.823534in}}{\pgfqpoint{1.187100in}{0.826806in}}{\pgfqpoint{1.192924in}{0.832630in}}%
\pgfpathcurveto{\pgfqpoint{1.198748in}{0.838454in}}{\pgfqpoint{1.202020in}{0.846354in}}{\pgfqpoint{1.202020in}{0.854590in}}%
\pgfpathcurveto{\pgfqpoint{1.202020in}{0.862826in}}{\pgfqpoint{1.198748in}{0.870726in}}{\pgfqpoint{1.192924in}{0.876550in}}%
\pgfpathcurveto{\pgfqpoint{1.187100in}{0.882374in}}{\pgfqpoint{1.179200in}{0.885647in}}{\pgfqpoint{1.170963in}{0.885647in}}%
\pgfpathcurveto{\pgfqpoint{1.162727in}{0.885647in}}{\pgfqpoint{1.154827in}{0.882374in}}{\pgfqpoint{1.149003in}{0.876550in}}%
\pgfpathcurveto{\pgfqpoint{1.143179in}{0.870726in}}{\pgfqpoint{1.139907in}{0.862826in}}{\pgfqpoint{1.139907in}{0.854590in}}%
\pgfpathcurveto{\pgfqpoint{1.139907in}{0.846354in}}{\pgfqpoint{1.143179in}{0.838454in}}{\pgfqpoint{1.149003in}{0.832630in}}%
\pgfpathcurveto{\pgfqpoint{1.154827in}{0.826806in}}{\pgfqpoint{1.162727in}{0.823534in}}{\pgfqpoint{1.170963in}{0.823534in}}%
\pgfpathclose%
\pgfusepath{stroke,fill}%
\end{pgfscope}%
\begin{pgfscope}%
\pgfpathrectangle{\pgfqpoint{0.457963in}{0.528059in}}{\pgfqpoint{6.200000in}{2.285714in}} %
\pgfusepath{clip}%
\pgfsetbuttcap%
\pgfsetroundjoin%
\definecolor{currentfill}{rgb}{0.833333,0.833333,1.000000}%
\pgfsetfillcolor{currentfill}%
\pgfsetlinewidth{1.003750pt}%
\definecolor{currentstroke}{rgb}{0.833333,0.833333,1.000000}%
\pgfsetstrokecolor{currentstroke}%
\pgfsetdash{}{0pt}%
\pgfpathmoveto{\pgfqpoint{1.470630in}{0.758227in}}%
\pgfpathcurveto{\pgfqpoint{1.478866in}{0.758227in}}{\pgfqpoint{1.486766in}{0.761500in}}{\pgfqpoint{1.492590in}{0.767324in}}%
\pgfpathcurveto{\pgfqpoint{1.498414in}{0.773148in}}{\pgfqpoint{1.501686in}{0.781048in}}{\pgfqpoint{1.501686in}{0.789284in}}%
\pgfpathcurveto{\pgfqpoint{1.501686in}{0.797520in}}{\pgfqpoint{1.498414in}{0.805420in}}{\pgfqpoint{1.492590in}{0.811244in}}%
\pgfpathcurveto{\pgfqpoint{1.486766in}{0.817068in}}{\pgfqpoint{1.478866in}{0.820340in}}{\pgfqpoint{1.470630in}{0.820340in}}%
\pgfpathcurveto{\pgfqpoint{1.462394in}{0.820340in}}{\pgfqpoint{1.454494in}{0.817068in}}{\pgfqpoint{1.448670in}{0.811244in}}%
\pgfpathcurveto{\pgfqpoint{1.442846in}{0.805420in}}{\pgfqpoint{1.439574in}{0.797520in}}{\pgfqpoint{1.439574in}{0.789284in}}%
\pgfpathcurveto{\pgfqpoint{1.439574in}{0.781048in}}{\pgfqpoint{1.442846in}{0.773148in}}{\pgfqpoint{1.448670in}{0.767324in}}%
\pgfpathcurveto{\pgfqpoint{1.454494in}{0.761500in}}{\pgfqpoint{1.462394in}{0.758227in}}{\pgfqpoint{1.470630in}{0.758227in}}%
\pgfpathclose%
\pgfusepath{stroke,fill}%
\end{pgfscope}%
\begin{pgfscope}%
\pgfpathrectangle{\pgfqpoint{0.457963in}{0.528059in}}{\pgfqpoint{6.200000in}{2.285714in}} %
\pgfusepath{clip}%
\pgfsetbuttcap%
\pgfsetroundjoin%
\definecolor{currentfill}{rgb}{0.833333,0.833333,1.000000}%
\pgfsetfillcolor{currentfill}%
\pgfsetlinewidth{1.003750pt}%
\definecolor{currentstroke}{rgb}{0.833333,0.833333,1.000000}%
\pgfsetstrokecolor{currentstroke}%
\pgfsetdash{}{0pt}%
\pgfpathmoveto{\pgfqpoint{1.821963in}{0.810472in}}%
\pgfpathcurveto{\pgfqpoint{1.830200in}{0.810472in}}{\pgfqpoint{1.838100in}{0.813745in}}{\pgfqpoint{1.843924in}{0.819569in}}%
\pgfpathcurveto{\pgfqpoint{1.849748in}{0.825393in}}{\pgfqpoint{1.853020in}{0.833293in}}{\pgfqpoint{1.853020in}{0.841529in}}%
\pgfpathcurveto{\pgfqpoint{1.853020in}{0.849765in}}{\pgfqpoint{1.849748in}{0.857665in}}{\pgfqpoint{1.843924in}{0.863489in}}%
\pgfpathcurveto{\pgfqpoint{1.838100in}{0.869313in}}{\pgfqpoint{1.830200in}{0.872585in}}{\pgfqpoint{1.821963in}{0.872585in}}%
\pgfpathcurveto{\pgfqpoint{1.813727in}{0.872585in}}{\pgfqpoint{1.805827in}{0.869313in}}{\pgfqpoint{1.800003in}{0.863489in}}%
\pgfpathcurveto{\pgfqpoint{1.794179in}{0.857665in}}{\pgfqpoint{1.790907in}{0.849765in}}{\pgfqpoint{1.790907in}{0.841529in}}%
\pgfpathcurveto{\pgfqpoint{1.790907in}{0.833293in}}{\pgfqpoint{1.794179in}{0.825393in}}{\pgfqpoint{1.800003in}{0.819569in}}%
\pgfpathcurveto{\pgfqpoint{1.805827in}{0.813745in}}{\pgfqpoint{1.813727in}{0.810472in}}{\pgfqpoint{1.821963in}{0.810472in}}%
\pgfpathclose%
\pgfusepath{stroke,fill}%
\end{pgfscope}%
\begin{pgfscope}%
\pgfpathrectangle{\pgfqpoint{0.457963in}{0.528059in}}{\pgfqpoint{6.200000in}{2.285714in}} %
\pgfusepath{clip}%
\pgfsetbuttcap%
\pgfsetroundjoin%
\definecolor{currentfill}{rgb}{0.833333,0.833333,1.000000}%
\pgfsetfillcolor{currentfill}%
\pgfsetlinewidth{1.003750pt}%
\definecolor{currentstroke}{rgb}{0.833333,0.833333,1.000000}%
\pgfsetstrokecolor{currentstroke}%
\pgfsetdash{}{0pt}%
\pgfpathmoveto{\pgfqpoint{2.193963in}{0.627615in}}%
\pgfpathcurveto{\pgfqpoint{2.202200in}{0.627615in}}{\pgfqpoint{2.210100in}{0.630887in}}{\pgfqpoint{2.215924in}{0.636711in}}%
\pgfpathcurveto{\pgfqpoint{2.221748in}{0.642535in}}{\pgfqpoint{2.225020in}{0.650435in}}{\pgfqpoint{2.225020in}{0.658672in}}%
\pgfpathcurveto{\pgfqpoint{2.225020in}{0.666908in}}{\pgfqpoint{2.221748in}{0.674808in}}{\pgfqpoint{2.215924in}{0.680632in}}%
\pgfpathcurveto{\pgfqpoint{2.210100in}{0.686456in}}{\pgfqpoint{2.202200in}{0.689728in}}{\pgfqpoint{2.193963in}{0.689728in}}%
\pgfpathcurveto{\pgfqpoint{2.185727in}{0.689728in}}{\pgfqpoint{2.177827in}{0.686456in}}{\pgfqpoint{2.172003in}{0.680632in}}%
\pgfpathcurveto{\pgfqpoint{2.166179in}{0.674808in}}{\pgfqpoint{2.162907in}{0.666908in}}{\pgfqpoint{2.162907in}{0.658672in}}%
\pgfpathcurveto{\pgfqpoint{2.162907in}{0.650435in}}{\pgfqpoint{2.166179in}{0.642535in}}{\pgfqpoint{2.172003in}{0.636711in}}%
\pgfpathcurveto{\pgfqpoint{2.177827in}{0.630887in}}{\pgfqpoint{2.185727in}{0.627615in}}{\pgfqpoint{2.193963in}{0.627615in}}%
\pgfpathclose%
\pgfusepath{stroke,fill}%
\end{pgfscope}%
\begin{pgfscope}%
\pgfpathrectangle{\pgfqpoint{0.457963in}{0.528059in}}{\pgfqpoint{6.200000in}{2.285714in}} %
\pgfusepath{clip}%
\pgfsetbuttcap%
\pgfsetroundjoin%
\definecolor{currentfill}{rgb}{0.666667,0.666667,1.000000}%
\pgfsetfillcolor{currentfill}%
\pgfsetlinewidth{1.003750pt}%
\definecolor{currentstroke}{rgb}{0.666667,0.666667,1.000000}%
\pgfsetstrokecolor{currentstroke}%
\pgfsetdash{}{0pt}%
\pgfpathmoveto{\pgfqpoint{0.457963in}{1.150064in}}%
\pgfpathcurveto{\pgfqpoint{0.466200in}{1.150064in}}{\pgfqpoint{0.474100in}{1.153336in}}{\pgfqpoint{0.479924in}{1.159160in}}%
\pgfpathcurveto{\pgfqpoint{0.485748in}{1.164984in}}{\pgfqpoint{0.489020in}{1.172884in}}{\pgfqpoint{0.489020in}{1.181121in}}%
\pgfpathcurveto{\pgfqpoint{0.489020in}{1.189357in}}{\pgfqpoint{0.485748in}{1.197257in}}{\pgfqpoint{0.479924in}{1.203081in}}%
\pgfpathcurveto{\pgfqpoint{0.474100in}{1.208905in}}{\pgfqpoint{0.466200in}{1.212177in}}{\pgfqpoint{0.457963in}{1.212177in}}%
\pgfpathcurveto{\pgfqpoint{0.449727in}{1.212177in}}{\pgfqpoint{0.441827in}{1.208905in}}{\pgfqpoint{0.436003in}{1.203081in}}%
\pgfpathcurveto{\pgfqpoint{0.430179in}{1.197257in}}{\pgfqpoint{0.426907in}{1.189357in}}{\pgfqpoint{0.426907in}{1.181121in}}%
\pgfpathcurveto{\pgfqpoint{0.426907in}{1.172884in}}{\pgfqpoint{0.430179in}{1.164984in}}{\pgfqpoint{0.436003in}{1.159160in}}%
\pgfpathcurveto{\pgfqpoint{0.441827in}{1.153336in}}{\pgfqpoint{0.449727in}{1.150064in}}{\pgfqpoint{0.457963in}{1.150064in}}%
\pgfpathclose%
\pgfusepath{stroke,fill}%
\end{pgfscope}%
\begin{pgfscope}%
\pgfpathrectangle{\pgfqpoint{0.457963in}{0.528059in}}{\pgfqpoint{6.200000in}{2.285714in}} %
\pgfusepath{clip}%
\pgfsetbuttcap%
\pgfsetroundjoin%
\definecolor{currentfill}{rgb}{0.666667,0.666667,1.000000}%
\pgfsetfillcolor{currentfill}%
\pgfsetlinewidth{1.003750pt}%
\definecolor{currentstroke}{rgb}{0.666667,0.666667,1.000000}%
\pgfsetstrokecolor{currentstroke}%
\pgfsetdash{}{0pt}%
\pgfpathmoveto{\pgfqpoint{0.457963in}{1.150064in}}%
\pgfpathcurveto{\pgfqpoint{0.466200in}{1.150064in}}{\pgfqpoint{0.474100in}{1.153336in}}{\pgfqpoint{0.479924in}{1.159160in}}%
\pgfpathcurveto{\pgfqpoint{0.485748in}{1.164984in}}{\pgfqpoint{0.489020in}{1.172884in}}{\pgfqpoint{0.489020in}{1.181121in}}%
\pgfpathcurveto{\pgfqpoint{0.489020in}{1.189357in}}{\pgfqpoint{0.485748in}{1.197257in}}{\pgfqpoint{0.479924in}{1.203081in}}%
\pgfpathcurveto{\pgfqpoint{0.474100in}{1.208905in}}{\pgfqpoint{0.466200in}{1.212177in}}{\pgfqpoint{0.457963in}{1.212177in}}%
\pgfpathcurveto{\pgfqpoint{0.449727in}{1.212177in}}{\pgfqpoint{0.441827in}{1.208905in}}{\pgfqpoint{0.436003in}{1.203081in}}%
\pgfpathcurveto{\pgfqpoint{0.430179in}{1.197257in}}{\pgfqpoint{0.426907in}{1.189357in}}{\pgfqpoint{0.426907in}{1.181121in}}%
\pgfpathcurveto{\pgfqpoint{0.426907in}{1.172884in}}{\pgfqpoint{0.430179in}{1.164984in}}{\pgfqpoint{0.436003in}{1.159160in}}%
\pgfpathcurveto{\pgfqpoint{0.441827in}{1.153336in}}{\pgfqpoint{0.449727in}{1.150064in}}{\pgfqpoint{0.457963in}{1.150064in}}%
\pgfpathclose%
\pgfusepath{stroke,fill}%
\end{pgfscope}%
\begin{pgfscope}%
\pgfpathrectangle{\pgfqpoint{0.457963in}{0.528059in}}{\pgfqpoint{6.200000in}{2.285714in}} %
\pgfusepath{clip}%
\pgfsetbuttcap%
\pgfsetroundjoin%
\definecolor{currentfill}{rgb}{0.666667,0.666667,1.000000}%
\pgfsetfillcolor{currentfill}%
\pgfsetlinewidth{1.003750pt}%
\definecolor{currentstroke}{rgb}{0.666667,0.666667,1.000000}%
\pgfsetstrokecolor{currentstroke}%
\pgfsetdash{}{0pt}%
\pgfpathmoveto{\pgfqpoint{0.457963in}{1.150064in}}%
\pgfpathcurveto{\pgfqpoint{0.466200in}{1.150064in}}{\pgfqpoint{0.474100in}{1.153336in}}{\pgfqpoint{0.479924in}{1.159160in}}%
\pgfpathcurveto{\pgfqpoint{0.485748in}{1.164984in}}{\pgfqpoint{0.489020in}{1.172884in}}{\pgfqpoint{0.489020in}{1.181121in}}%
\pgfpathcurveto{\pgfqpoint{0.489020in}{1.189357in}}{\pgfqpoint{0.485748in}{1.197257in}}{\pgfqpoint{0.479924in}{1.203081in}}%
\pgfpathcurveto{\pgfqpoint{0.474100in}{1.208905in}}{\pgfqpoint{0.466200in}{1.212177in}}{\pgfqpoint{0.457963in}{1.212177in}}%
\pgfpathcurveto{\pgfqpoint{0.449727in}{1.212177in}}{\pgfqpoint{0.441827in}{1.208905in}}{\pgfqpoint{0.436003in}{1.203081in}}%
\pgfpathcurveto{\pgfqpoint{0.430179in}{1.197257in}}{\pgfqpoint{0.426907in}{1.189357in}}{\pgfqpoint{0.426907in}{1.181121in}}%
\pgfpathcurveto{\pgfqpoint{0.426907in}{1.172884in}}{\pgfqpoint{0.430179in}{1.164984in}}{\pgfqpoint{0.436003in}{1.159160in}}%
\pgfpathcurveto{\pgfqpoint{0.441827in}{1.153336in}}{\pgfqpoint{0.449727in}{1.150064in}}{\pgfqpoint{0.457963in}{1.150064in}}%
\pgfpathclose%
\pgfusepath{stroke,fill}%
\end{pgfscope}%
\begin{pgfscope}%
\pgfpathrectangle{\pgfqpoint{0.457963in}{0.528059in}}{\pgfqpoint{6.200000in}{2.285714in}} %
\pgfusepath{clip}%
\pgfsetbuttcap%
\pgfsetroundjoin%
\definecolor{currentfill}{rgb}{0.666667,0.666667,1.000000}%
\pgfsetfillcolor{currentfill}%
\pgfsetlinewidth{1.003750pt}%
\definecolor{currentstroke}{rgb}{0.666667,0.666667,1.000000}%
\pgfsetstrokecolor{currentstroke}%
\pgfsetdash{}{0pt}%
\pgfpathmoveto{\pgfqpoint{0.457963in}{1.150064in}}%
\pgfpathcurveto{\pgfqpoint{0.466200in}{1.150064in}}{\pgfqpoint{0.474100in}{1.153336in}}{\pgfqpoint{0.479924in}{1.159160in}}%
\pgfpathcurveto{\pgfqpoint{0.485748in}{1.164984in}}{\pgfqpoint{0.489020in}{1.172884in}}{\pgfqpoint{0.489020in}{1.181121in}}%
\pgfpathcurveto{\pgfqpoint{0.489020in}{1.189357in}}{\pgfqpoint{0.485748in}{1.197257in}}{\pgfqpoint{0.479924in}{1.203081in}}%
\pgfpathcurveto{\pgfqpoint{0.474100in}{1.208905in}}{\pgfqpoint{0.466200in}{1.212177in}}{\pgfqpoint{0.457963in}{1.212177in}}%
\pgfpathcurveto{\pgfqpoint{0.449727in}{1.212177in}}{\pgfqpoint{0.441827in}{1.208905in}}{\pgfqpoint{0.436003in}{1.203081in}}%
\pgfpathcurveto{\pgfqpoint{0.430179in}{1.197257in}}{\pgfqpoint{0.426907in}{1.189357in}}{\pgfqpoint{0.426907in}{1.181121in}}%
\pgfpathcurveto{\pgfqpoint{0.426907in}{1.172884in}}{\pgfqpoint{0.430179in}{1.164984in}}{\pgfqpoint{0.436003in}{1.159160in}}%
\pgfpathcurveto{\pgfqpoint{0.441827in}{1.153336in}}{\pgfqpoint{0.449727in}{1.150064in}}{\pgfqpoint{0.457963in}{1.150064in}}%
\pgfpathclose%
\pgfusepath{stroke,fill}%
\end{pgfscope}%
\begin{pgfscope}%
\pgfpathrectangle{\pgfqpoint{0.457963in}{0.528059in}}{\pgfqpoint{6.200000in}{2.285714in}} %
\pgfusepath{clip}%
\pgfsetbuttcap%
\pgfsetroundjoin%
\definecolor{currentfill}{rgb}{0.666667,0.666667,1.000000}%
\pgfsetfillcolor{currentfill}%
\pgfsetlinewidth{1.003750pt}%
\definecolor{currentstroke}{rgb}{0.666667,0.666667,1.000000}%
\pgfsetstrokecolor{currentstroke}%
\pgfsetdash{}{0pt}%
\pgfpathmoveto{\pgfqpoint{0.457963in}{1.150064in}}%
\pgfpathcurveto{\pgfqpoint{0.466200in}{1.150064in}}{\pgfqpoint{0.474100in}{1.153336in}}{\pgfqpoint{0.479924in}{1.159160in}}%
\pgfpathcurveto{\pgfqpoint{0.485748in}{1.164984in}}{\pgfqpoint{0.489020in}{1.172884in}}{\pgfqpoint{0.489020in}{1.181121in}}%
\pgfpathcurveto{\pgfqpoint{0.489020in}{1.189357in}}{\pgfqpoint{0.485748in}{1.197257in}}{\pgfqpoint{0.479924in}{1.203081in}}%
\pgfpathcurveto{\pgfqpoint{0.474100in}{1.208905in}}{\pgfqpoint{0.466200in}{1.212177in}}{\pgfqpoint{0.457963in}{1.212177in}}%
\pgfpathcurveto{\pgfqpoint{0.449727in}{1.212177in}}{\pgfqpoint{0.441827in}{1.208905in}}{\pgfqpoint{0.436003in}{1.203081in}}%
\pgfpathcurveto{\pgfqpoint{0.430179in}{1.197257in}}{\pgfqpoint{0.426907in}{1.189357in}}{\pgfqpoint{0.426907in}{1.181121in}}%
\pgfpathcurveto{\pgfqpoint{0.426907in}{1.172884in}}{\pgfqpoint{0.430179in}{1.164984in}}{\pgfqpoint{0.436003in}{1.159160in}}%
\pgfpathcurveto{\pgfqpoint{0.441827in}{1.153336in}}{\pgfqpoint{0.449727in}{1.150064in}}{\pgfqpoint{0.457963in}{1.150064in}}%
\pgfpathclose%
\pgfusepath{stroke,fill}%
\end{pgfscope}%
\begin{pgfscope}%
\pgfpathrectangle{\pgfqpoint{0.457963in}{0.528059in}}{\pgfqpoint{6.200000in}{2.285714in}} %
\pgfusepath{clip}%
\pgfsetbuttcap%
\pgfsetroundjoin%
\definecolor{currentfill}{rgb}{0.666667,0.666667,1.000000}%
\pgfsetfillcolor{currentfill}%
\pgfsetlinewidth{1.003750pt}%
\definecolor{currentstroke}{rgb}{0.666667,0.666667,1.000000}%
\pgfsetstrokecolor{currentstroke}%
\pgfsetdash{}{0pt}%
\pgfpathmoveto{\pgfqpoint{0.468297in}{1.150064in}}%
\pgfpathcurveto{\pgfqpoint{0.476533in}{1.150064in}}{\pgfqpoint{0.484433in}{1.153336in}}{\pgfqpoint{0.490257in}{1.159160in}}%
\pgfpathcurveto{\pgfqpoint{0.496081in}{1.164984in}}{\pgfqpoint{0.499353in}{1.172884in}}{\pgfqpoint{0.499353in}{1.181121in}}%
\pgfpathcurveto{\pgfqpoint{0.499353in}{1.189357in}}{\pgfqpoint{0.496081in}{1.197257in}}{\pgfqpoint{0.490257in}{1.203081in}}%
\pgfpathcurveto{\pgfqpoint{0.484433in}{1.208905in}}{\pgfqpoint{0.476533in}{1.212177in}}{\pgfqpoint{0.468297in}{1.212177in}}%
\pgfpathcurveto{\pgfqpoint{0.460060in}{1.212177in}}{\pgfqpoint{0.452160in}{1.208905in}}{\pgfqpoint{0.446336in}{1.203081in}}%
\pgfpathcurveto{\pgfqpoint{0.440512in}{1.197257in}}{\pgfqpoint{0.437240in}{1.189357in}}{\pgfqpoint{0.437240in}{1.181121in}}%
\pgfpathcurveto{\pgfqpoint{0.437240in}{1.172884in}}{\pgfqpoint{0.440512in}{1.164984in}}{\pgfqpoint{0.446336in}{1.159160in}}%
\pgfpathcurveto{\pgfqpoint{0.452160in}{1.153336in}}{\pgfqpoint{0.460060in}{1.150064in}}{\pgfqpoint{0.468297in}{1.150064in}}%
\pgfpathclose%
\pgfusepath{stroke,fill}%
\end{pgfscope}%
\begin{pgfscope}%
\pgfpathrectangle{\pgfqpoint{0.457963in}{0.528059in}}{\pgfqpoint{6.200000in}{2.285714in}} %
\pgfusepath{clip}%
\pgfsetbuttcap%
\pgfsetroundjoin%
\definecolor{currentfill}{rgb}{0.666667,0.666667,1.000000}%
\pgfsetfillcolor{currentfill}%
\pgfsetlinewidth{1.003750pt}%
\definecolor{currentstroke}{rgb}{0.666667,0.666667,1.000000}%
\pgfsetstrokecolor{currentstroke}%
\pgfsetdash{}{0pt}%
\pgfpathmoveto{\pgfqpoint{0.530297in}{1.137003in}}%
\pgfpathcurveto{\pgfqpoint{0.538533in}{1.137003in}}{\pgfqpoint{0.546433in}{1.140275in}}{\pgfqpoint{0.552257in}{1.146099in}}%
\pgfpathcurveto{\pgfqpoint{0.558081in}{1.151923in}}{\pgfqpoint{0.561353in}{1.159823in}}{\pgfqpoint{0.561353in}{1.168059in}}%
\pgfpathcurveto{\pgfqpoint{0.561353in}{1.176296in}}{\pgfqpoint{0.558081in}{1.184196in}}{\pgfqpoint{0.552257in}{1.190020in}}%
\pgfpathcurveto{\pgfqpoint{0.546433in}{1.195844in}}{\pgfqpoint{0.538533in}{1.199116in}}{\pgfqpoint{0.530297in}{1.199116in}}%
\pgfpathcurveto{\pgfqpoint{0.522060in}{1.199116in}}{\pgfqpoint{0.514160in}{1.195844in}}{\pgfqpoint{0.508336in}{1.190020in}}%
\pgfpathcurveto{\pgfqpoint{0.502512in}{1.184196in}}{\pgfqpoint{0.499240in}{1.176296in}}{\pgfqpoint{0.499240in}{1.168059in}}%
\pgfpathcurveto{\pgfqpoint{0.499240in}{1.159823in}}{\pgfqpoint{0.502512in}{1.151923in}}{\pgfqpoint{0.508336in}{1.146099in}}%
\pgfpathcurveto{\pgfqpoint{0.514160in}{1.140275in}}{\pgfqpoint{0.522060in}{1.137003in}}{\pgfqpoint{0.530297in}{1.137003in}}%
\pgfpathclose%
\pgfusepath{stroke,fill}%
\end{pgfscope}%
\begin{pgfscope}%
\pgfpathrectangle{\pgfqpoint{0.457963in}{0.528059in}}{\pgfqpoint{6.200000in}{2.285714in}} %
\pgfusepath{clip}%
\pgfsetbuttcap%
\pgfsetroundjoin%
\definecolor{currentfill}{rgb}{0.666667,0.666667,1.000000}%
\pgfsetfillcolor{currentfill}%
\pgfsetlinewidth{1.003750pt}%
\definecolor{currentstroke}{rgb}{0.666667,0.666667,1.000000}%
\pgfsetstrokecolor{currentstroke}%
\pgfsetdash{}{0pt}%
\pgfpathmoveto{\pgfqpoint{0.530297in}{1.137003in}}%
\pgfpathcurveto{\pgfqpoint{0.538533in}{1.137003in}}{\pgfqpoint{0.546433in}{1.140275in}}{\pgfqpoint{0.552257in}{1.146099in}}%
\pgfpathcurveto{\pgfqpoint{0.558081in}{1.151923in}}{\pgfqpoint{0.561353in}{1.159823in}}{\pgfqpoint{0.561353in}{1.168059in}}%
\pgfpathcurveto{\pgfqpoint{0.561353in}{1.176296in}}{\pgfqpoint{0.558081in}{1.184196in}}{\pgfqpoint{0.552257in}{1.190020in}}%
\pgfpathcurveto{\pgfqpoint{0.546433in}{1.195844in}}{\pgfqpoint{0.538533in}{1.199116in}}{\pgfqpoint{0.530297in}{1.199116in}}%
\pgfpathcurveto{\pgfqpoint{0.522060in}{1.199116in}}{\pgfqpoint{0.514160in}{1.195844in}}{\pgfqpoint{0.508336in}{1.190020in}}%
\pgfpathcurveto{\pgfqpoint{0.502512in}{1.184196in}}{\pgfqpoint{0.499240in}{1.176296in}}{\pgfqpoint{0.499240in}{1.168059in}}%
\pgfpathcurveto{\pgfqpoint{0.499240in}{1.159823in}}{\pgfqpoint{0.502512in}{1.151923in}}{\pgfqpoint{0.508336in}{1.146099in}}%
\pgfpathcurveto{\pgfqpoint{0.514160in}{1.140275in}}{\pgfqpoint{0.522060in}{1.137003in}}{\pgfqpoint{0.530297in}{1.137003in}}%
\pgfpathclose%
\pgfusepath{stroke,fill}%
\end{pgfscope}%
\begin{pgfscope}%
\pgfpathrectangle{\pgfqpoint{0.457963in}{0.528059in}}{\pgfqpoint{6.200000in}{2.285714in}} %
\pgfusepath{clip}%
\pgfsetbuttcap%
\pgfsetroundjoin%
\definecolor{currentfill}{rgb}{0.666667,0.666667,1.000000}%
\pgfsetfillcolor{currentfill}%
\pgfsetlinewidth{1.003750pt}%
\definecolor{currentstroke}{rgb}{0.666667,0.666667,1.000000}%
\pgfsetstrokecolor{currentstroke}%
\pgfsetdash{}{0pt}%
\pgfpathmoveto{\pgfqpoint{0.540630in}{1.137003in}}%
\pgfpathcurveto{\pgfqpoint{0.548866in}{1.137003in}}{\pgfqpoint{0.556766in}{1.140275in}}{\pgfqpoint{0.562590in}{1.146099in}}%
\pgfpathcurveto{\pgfqpoint{0.568414in}{1.151923in}}{\pgfqpoint{0.571686in}{1.159823in}}{\pgfqpoint{0.571686in}{1.168059in}}%
\pgfpathcurveto{\pgfqpoint{0.571686in}{1.176296in}}{\pgfqpoint{0.568414in}{1.184196in}}{\pgfqpoint{0.562590in}{1.190020in}}%
\pgfpathcurveto{\pgfqpoint{0.556766in}{1.195844in}}{\pgfqpoint{0.548866in}{1.199116in}}{\pgfqpoint{0.540630in}{1.199116in}}%
\pgfpathcurveto{\pgfqpoint{0.532394in}{1.199116in}}{\pgfqpoint{0.524494in}{1.195844in}}{\pgfqpoint{0.518670in}{1.190020in}}%
\pgfpathcurveto{\pgfqpoint{0.512846in}{1.184196in}}{\pgfqpoint{0.509574in}{1.176296in}}{\pgfqpoint{0.509574in}{1.168059in}}%
\pgfpathcurveto{\pgfqpoint{0.509574in}{1.159823in}}{\pgfqpoint{0.512846in}{1.151923in}}{\pgfqpoint{0.518670in}{1.146099in}}%
\pgfpathcurveto{\pgfqpoint{0.524494in}{1.140275in}}{\pgfqpoint{0.532394in}{1.137003in}}{\pgfqpoint{0.540630in}{1.137003in}}%
\pgfpathclose%
\pgfusepath{stroke,fill}%
\end{pgfscope}%
\begin{pgfscope}%
\pgfpathrectangle{\pgfqpoint{0.457963in}{0.528059in}}{\pgfqpoint{6.200000in}{2.285714in}} %
\pgfusepath{clip}%
\pgfsetbuttcap%
\pgfsetroundjoin%
\definecolor{currentfill}{rgb}{0.666667,0.666667,1.000000}%
\pgfsetfillcolor{currentfill}%
\pgfsetlinewidth{1.003750pt}%
\definecolor{currentstroke}{rgb}{0.666667,0.666667,1.000000}%
\pgfsetstrokecolor{currentstroke}%
\pgfsetdash{}{0pt}%
\pgfpathmoveto{\pgfqpoint{0.643963in}{1.123942in}}%
\pgfpathcurveto{\pgfqpoint{0.652200in}{1.123942in}}{\pgfqpoint{0.660100in}{1.127214in}}{\pgfqpoint{0.665924in}{1.133038in}}%
\pgfpathcurveto{\pgfqpoint{0.671748in}{1.138862in}}{\pgfqpoint{0.675020in}{1.146762in}}{\pgfqpoint{0.675020in}{1.154998in}}%
\pgfpathcurveto{\pgfqpoint{0.675020in}{1.163234in}}{\pgfqpoint{0.671748in}{1.171135in}}{\pgfqpoint{0.665924in}{1.176958in}}%
\pgfpathcurveto{\pgfqpoint{0.660100in}{1.182782in}}{\pgfqpoint{0.652200in}{1.186055in}}{\pgfqpoint{0.643963in}{1.186055in}}%
\pgfpathcurveto{\pgfqpoint{0.635727in}{1.186055in}}{\pgfqpoint{0.627827in}{1.182782in}}{\pgfqpoint{0.622003in}{1.176958in}}%
\pgfpathcurveto{\pgfqpoint{0.616179in}{1.171135in}}{\pgfqpoint{0.612907in}{1.163234in}}{\pgfqpoint{0.612907in}{1.154998in}}%
\pgfpathcurveto{\pgfqpoint{0.612907in}{1.146762in}}{\pgfqpoint{0.616179in}{1.138862in}}{\pgfqpoint{0.622003in}{1.133038in}}%
\pgfpathcurveto{\pgfqpoint{0.627827in}{1.127214in}}{\pgfqpoint{0.635727in}{1.123942in}}{\pgfqpoint{0.643963in}{1.123942in}}%
\pgfpathclose%
\pgfusepath{stroke,fill}%
\end{pgfscope}%
\begin{pgfscope}%
\pgfpathrectangle{\pgfqpoint{0.457963in}{0.528059in}}{\pgfqpoint{6.200000in}{2.285714in}} %
\pgfusepath{clip}%
\pgfsetbuttcap%
\pgfsetroundjoin%
\definecolor{currentfill}{rgb}{0.666667,0.666667,1.000000}%
\pgfsetfillcolor{currentfill}%
\pgfsetlinewidth{1.003750pt}%
\definecolor{currentstroke}{rgb}{0.666667,0.666667,1.000000}%
\pgfsetstrokecolor{currentstroke}%
\pgfsetdash{}{0pt}%
\pgfpathmoveto{\pgfqpoint{0.695630in}{1.150064in}}%
\pgfpathcurveto{\pgfqpoint{0.703866in}{1.150064in}}{\pgfqpoint{0.711766in}{1.153336in}}{\pgfqpoint{0.717590in}{1.159160in}}%
\pgfpathcurveto{\pgfqpoint{0.723414in}{1.164984in}}{\pgfqpoint{0.726686in}{1.172884in}}{\pgfqpoint{0.726686in}{1.181121in}}%
\pgfpathcurveto{\pgfqpoint{0.726686in}{1.189357in}}{\pgfqpoint{0.723414in}{1.197257in}}{\pgfqpoint{0.717590in}{1.203081in}}%
\pgfpathcurveto{\pgfqpoint{0.711766in}{1.208905in}}{\pgfqpoint{0.703866in}{1.212177in}}{\pgfqpoint{0.695630in}{1.212177in}}%
\pgfpathcurveto{\pgfqpoint{0.687394in}{1.212177in}}{\pgfqpoint{0.679494in}{1.208905in}}{\pgfqpoint{0.673670in}{1.203081in}}%
\pgfpathcurveto{\pgfqpoint{0.667846in}{1.197257in}}{\pgfqpoint{0.664574in}{1.189357in}}{\pgfqpoint{0.664574in}{1.181121in}}%
\pgfpathcurveto{\pgfqpoint{0.664574in}{1.172884in}}{\pgfqpoint{0.667846in}{1.164984in}}{\pgfqpoint{0.673670in}{1.159160in}}%
\pgfpathcurveto{\pgfqpoint{0.679494in}{1.153336in}}{\pgfqpoint{0.687394in}{1.150064in}}{\pgfqpoint{0.695630in}{1.150064in}}%
\pgfpathclose%
\pgfusepath{stroke,fill}%
\end{pgfscope}%
\begin{pgfscope}%
\pgfpathrectangle{\pgfqpoint{0.457963in}{0.528059in}}{\pgfqpoint{6.200000in}{2.285714in}} %
\pgfusepath{clip}%
\pgfsetbuttcap%
\pgfsetroundjoin%
\definecolor{currentfill}{rgb}{0.666667,0.666667,1.000000}%
\pgfsetfillcolor{currentfill}%
\pgfsetlinewidth{1.003750pt}%
\definecolor{currentstroke}{rgb}{0.666667,0.666667,1.000000}%
\pgfsetstrokecolor{currentstroke}%
\pgfsetdash{}{0pt}%
\pgfpathmoveto{\pgfqpoint{0.757630in}{1.150064in}}%
\pgfpathcurveto{\pgfqpoint{0.765866in}{1.150064in}}{\pgfqpoint{0.773766in}{1.153336in}}{\pgfqpoint{0.779590in}{1.159160in}}%
\pgfpathcurveto{\pgfqpoint{0.785414in}{1.164984in}}{\pgfqpoint{0.788686in}{1.172884in}}{\pgfqpoint{0.788686in}{1.181121in}}%
\pgfpathcurveto{\pgfqpoint{0.788686in}{1.189357in}}{\pgfqpoint{0.785414in}{1.197257in}}{\pgfqpoint{0.779590in}{1.203081in}}%
\pgfpathcurveto{\pgfqpoint{0.773766in}{1.208905in}}{\pgfqpoint{0.765866in}{1.212177in}}{\pgfqpoint{0.757630in}{1.212177in}}%
\pgfpathcurveto{\pgfqpoint{0.749394in}{1.212177in}}{\pgfqpoint{0.741494in}{1.208905in}}{\pgfqpoint{0.735670in}{1.203081in}}%
\pgfpathcurveto{\pgfqpoint{0.729846in}{1.197257in}}{\pgfqpoint{0.726574in}{1.189357in}}{\pgfqpoint{0.726574in}{1.181121in}}%
\pgfpathcurveto{\pgfqpoint{0.726574in}{1.172884in}}{\pgfqpoint{0.729846in}{1.164984in}}{\pgfqpoint{0.735670in}{1.159160in}}%
\pgfpathcurveto{\pgfqpoint{0.741494in}{1.153336in}}{\pgfqpoint{0.749394in}{1.150064in}}{\pgfqpoint{0.757630in}{1.150064in}}%
\pgfpathclose%
\pgfusepath{stroke,fill}%
\end{pgfscope}%
\begin{pgfscope}%
\pgfpathrectangle{\pgfqpoint{0.457963in}{0.528059in}}{\pgfqpoint{6.200000in}{2.285714in}} %
\pgfusepath{clip}%
\pgfsetbuttcap%
\pgfsetroundjoin%
\definecolor{currentfill}{rgb}{0.666667,0.666667,1.000000}%
\pgfsetfillcolor{currentfill}%
\pgfsetlinewidth{1.003750pt}%
\definecolor{currentstroke}{rgb}{0.666667,0.666667,1.000000}%
\pgfsetstrokecolor{currentstroke}%
\pgfsetdash{}{0pt}%
\pgfpathmoveto{\pgfqpoint{0.933297in}{1.071697in}}%
\pgfpathcurveto{\pgfqpoint{0.941533in}{1.071697in}}{\pgfqpoint{0.949433in}{1.074969in}}{\pgfqpoint{0.955257in}{1.080793in}}%
\pgfpathcurveto{\pgfqpoint{0.961081in}{1.086617in}}{\pgfqpoint{0.964353in}{1.094517in}}{\pgfqpoint{0.964353in}{1.102753in}}%
\pgfpathcurveto{\pgfqpoint{0.964353in}{1.110990in}}{\pgfqpoint{0.961081in}{1.118890in}}{\pgfqpoint{0.955257in}{1.124714in}}%
\pgfpathcurveto{\pgfqpoint{0.949433in}{1.130538in}}{\pgfqpoint{0.941533in}{1.133810in}}{\pgfqpoint{0.933297in}{1.133810in}}%
\pgfpathcurveto{\pgfqpoint{0.925060in}{1.133810in}}{\pgfqpoint{0.917160in}{1.130538in}}{\pgfqpoint{0.911336in}{1.124714in}}%
\pgfpathcurveto{\pgfqpoint{0.905512in}{1.118890in}}{\pgfqpoint{0.902240in}{1.110990in}}{\pgfqpoint{0.902240in}{1.102753in}}%
\pgfpathcurveto{\pgfqpoint{0.902240in}{1.094517in}}{\pgfqpoint{0.905512in}{1.086617in}}{\pgfqpoint{0.911336in}{1.080793in}}%
\pgfpathcurveto{\pgfqpoint{0.917160in}{1.074969in}}{\pgfqpoint{0.925060in}{1.071697in}}{\pgfqpoint{0.933297in}{1.071697in}}%
\pgfpathclose%
\pgfusepath{stroke,fill}%
\end{pgfscope}%
\begin{pgfscope}%
\pgfpathrectangle{\pgfqpoint{0.457963in}{0.528059in}}{\pgfqpoint{6.200000in}{2.285714in}} %
\pgfusepath{clip}%
\pgfsetbuttcap%
\pgfsetroundjoin%
\definecolor{currentfill}{rgb}{0.666667,0.666667,1.000000}%
\pgfsetfillcolor{currentfill}%
\pgfsetlinewidth{1.003750pt}%
\definecolor{currentstroke}{rgb}{0.666667,0.666667,1.000000}%
\pgfsetstrokecolor{currentstroke}%
\pgfsetdash{}{0pt}%
\pgfpathmoveto{\pgfqpoint{0.995297in}{1.084758in}}%
\pgfpathcurveto{\pgfqpoint{1.003533in}{1.084758in}}{\pgfqpoint{1.011433in}{1.088030in}}{\pgfqpoint{1.017257in}{1.093854in}}%
\pgfpathcurveto{\pgfqpoint{1.023081in}{1.099678in}}{\pgfqpoint{1.026353in}{1.107578in}}{\pgfqpoint{1.026353in}{1.115815in}}%
\pgfpathcurveto{\pgfqpoint{1.026353in}{1.124051in}}{\pgfqpoint{1.023081in}{1.131951in}}{\pgfqpoint{1.017257in}{1.137775in}}%
\pgfpathcurveto{\pgfqpoint{1.011433in}{1.143599in}}{\pgfqpoint{1.003533in}{1.146871in}}{\pgfqpoint{0.995297in}{1.146871in}}%
\pgfpathcurveto{\pgfqpoint{0.987060in}{1.146871in}}{\pgfqpoint{0.979160in}{1.143599in}}{\pgfqpoint{0.973336in}{1.137775in}}%
\pgfpathcurveto{\pgfqpoint{0.967512in}{1.131951in}}{\pgfqpoint{0.964240in}{1.124051in}}{\pgfqpoint{0.964240in}{1.115815in}}%
\pgfpathcurveto{\pgfqpoint{0.964240in}{1.107578in}}{\pgfqpoint{0.967512in}{1.099678in}}{\pgfqpoint{0.973336in}{1.093854in}}%
\pgfpathcurveto{\pgfqpoint{0.979160in}{1.088030in}}{\pgfqpoint{0.987060in}{1.084758in}}{\pgfqpoint{0.995297in}{1.084758in}}%
\pgfpathclose%
\pgfusepath{stroke,fill}%
\end{pgfscope}%
\begin{pgfscope}%
\pgfpathrectangle{\pgfqpoint{0.457963in}{0.528059in}}{\pgfqpoint{6.200000in}{2.285714in}} %
\pgfusepath{clip}%
\pgfsetbuttcap%
\pgfsetroundjoin%
\definecolor{currentfill}{rgb}{0.666667,0.666667,1.000000}%
\pgfsetfillcolor{currentfill}%
\pgfsetlinewidth{1.003750pt}%
\definecolor{currentstroke}{rgb}{0.666667,0.666667,1.000000}%
\pgfsetstrokecolor{currentstroke}%
\pgfsetdash{}{0pt}%
\pgfpathmoveto{\pgfqpoint{1.232963in}{1.137003in}}%
\pgfpathcurveto{\pgfqpoint{1.241200in}{1.137003in}}{\pgfqpoint{1.249100in}{1.140275in}}{\pgfqpoint{1.254924in}{1.146099in}}%
\pgfpathcurveto{\pgfqpoint{1.260748in}{1.151923in}}{\pgfqpoint{1.264020in}{1.159823in}}{\pgfqpoint{1.264020in}{1.168059in}}%
\pgfpathcurveto{\pgfqpoint{1.264020in}{1.176296in}}{\pgfqpoint{1.260748in}{1.184196in}}{\pgfqpoint{1.254924in}{1.190020in}}%
\pgfpathcurveto{\pgfqpoint{1.249100in}{1.195844in}}{\pgfqpoint{1.241200in}{1.199116in}}{\pgfqpoint{1.232963in}{1.199116in}}%
\pgfpathcurveto{\pgfqpoint{1.224727in}{1.199116in}}{\pgfqpoint{1.216827in}{1.195844in}}{\pgfqpoint{1.211003in}{1.190020in}}%
\pgfpathcurveto{\pgfqpoint{1.205179in}{1.184196in}}{\pgfqpoint{1.201907in}{1.176296in}}{\pgfqpoint{1.201907in}{1.168059in}}%
\pgfpathcurveto{\pgfqpoint{1.201907in}{1.159823in}}{\pgfqpoint{1.205179in}{1.151923in}}{\pgfqpoint{1.211003in}{1.146099in}}%
\pgfpathcurveto{\pgfqpoint{1.216827in}{1.140275in}}{\pgfqpoint{1.224727in}{1.137003in}}{\pgfqpoint{1.232963in}{1.137003in}}%
\pgfpathclose%
\pgfusepath{stroke,fill}%
\end{pgfscope}%
\begin{pgfscope}%
\pgfpathrectangle{\pgfqpoint{0.457963in}{0.528059in}}{\pgfqpoint{6.200000in}{2.285714in}} %
\pgfusepath{clip}%
\pgfsetbuttcap%
\pgfsetroundjoin%
\definecolor{currentfill}{rgb}{0.666667,0.666667,1.000000}%
\pgfsetfillcolor{currentfill}%
\pgfsetlinewidth{1.003750pt}%
\definecolor{currentstroke}{rgb}{0.666667,0.666667,1.000000}%
\pgfsetstrokecolor{currentstroke}%
\pgfsetdash{}{0pt}%
\pgfpathmoveto{\pgfqpoint{1.336297in}{1.123942in}}%
\pgfpathcurveto{\pgfqpoint{1.344533in}{1.123942in}}{\pgfqpoint{1.352433in}{1.127214in}}{\pgfqpoint{1.358257in}{1.133038in}}%
\pgfpathcurveto{\pgfqpoint{1.364081in}{1.138862in}}{\pgfqpoint{1.367353in}{1.146762in}}{\pgfqpoint{1.367353in}{1.154998in}}%
\pgfpathcurveto{\pgfqpoint{1.367353in}{1.163234in}}{\pgfqpoint{1.364081in}{1.171135in}}{\pgfqpoint{1.358257in}{1.176958in}}%
\pgfpathcurveto{\pgfqpoint{1.352433in}{1.182782in}}{\pgfqpoint{1.344533in}{1.186055in}}{\pgfqpoint{1.336297in}{1.186055in}}%
\pgfpathcurveto{\pgfqpoint{1.328060in}{1.186055in}}{\pgfqpoint{1.320160in}{1.182782in}}{\pgfqpoint{1.314336in}{1.176958in}}%
\pgfpathcurveto{\pgfqpoint{1.308512in}{1.171135in}}{\pgfqpoint{1.305240in}{1.163234in}}{\pgfqpoint{1.305240in}{1.154998in}}%
\pgfpathcurveto{\pgfqpoint{1.305240in}{1.146762in}}{\pgfqpoint{1.308512in}{1.138862in}}{\pgfqpoint{1.314336in}{1.133038in}}%
\pgfpathcurveto{\pgfqpoint{1.320160in}{1.127214in}}{\pgfqpoint{1.328060in}{1.123942in}}{\pgfqpoint{1.336297in}{1.123942in}}%
\pgfpathclose%
\pgfusepath{stroke,fill}%
\end{pgfscope}%
\begin{pgfscope}%
\pgfpathrectangle{\pgfqpoint{0.457963in}{0.528059in}}{\pgfqpoint{6.200000in}{2.285714in}} %
\pgfusepath{clip}%
\pgfsetbuttcap%
\pgfsetroundjoin%
\definecolor{currentfill}{rgb}{0.666667,0.666667,1.000000}%
\pgfsetfillcolor{currentfill}%
\pgfsetlinewidth{1.003750pt}%
\definecolor{currentstroke}{rgb}{0.666667,0.666667,1.000000}%
\pgfsetstrokecolor{currentstroke}%
\pgfsetdash{}{0pt}%
\pgfpathmoveto{\pgfqpoint{1.728963in}{0.954146in}}%
\pgfpathcurveto{\pgfqpoint{1.737200in}{0.954146in}}{\pgfqpoint{1.745100in}{0.957418in}}{\pgfqpoint{1.750924in}{0.963242in}}%
\pgfpathcurveto{\pgfqpoint{1.756748in}{0.969066in}}{\pgfqpoint{1.760020in}{0.976966in}}{\pgfqpoint{1.760020in}{0.985202in}}%
\pgfpathcurveto{\pgfqpoint{1.760020in}{0.993439in}}{\pgfqpoint{1.756748in}{1.001339in}}{\pgfqpoint{1.750924in}{1.007163in}}%
\pgfpathcurveto{\pgfqpoint{1.745100in}{1.012986in}}{\pgfqpoint{1.737200in}{1.016259in}}{\pgfqpoint{1.728963in}{1.016259in}}%
\pgfpathcurveto{\pgfqpoint{1.720727in}{1.016259in}}{\pgfqpoint{1.712827in}{1.012986in}}{\pgfqpoint{1.707003in}{1.007163in}}%
\pgfpathcurveto{\pgfqpoint{1.701179in}{1.001339in}}{\pgfqpoint{1.697907in}{0.993439in}}{\pgfqpoint{1.697907in}{0.985202in}}%
\pgfpathcurveto{\pgfqpoint{1.697907in}{0.976966in}}{\pgfqpoint{1.701179in}{0.969066in}}{\pgfqpoint{1.707003in}{0.963242in}}%
\pgfpathcurveto{\pgfqpoint{1.712827in}{0.957418in}}{\pgfqpoint{1.720727in}{0.954146in}}{\pgfqpoint{1.728963in}{0.954146in}}%
\pgfpathclose%
\pgfusepath{stroke,fill}%
\end{pgfscope}%
\begin{pgfscope}%
\pgfpathrectangle{\pgfqpoint{0.457963in}{0.528059in}}{\pgfqpoint{6.200000in}{2.285714in}} %
\pgfusepath{clip}%
\pgfsetbuttcap%
\pgfsetroundjoin%
\definecolor{currentfill}{rgb}{0.666667,0.666667,1.000000}%
\pgfsetfillcolor{currentfill}%
\pgfsetlinewidth{1.003750pt}%
\definecolor{currentstroke}{rgb}{0.666667,0.666667,1.000000}%
\pgfsetstrokecolor{currentstroke}%
\pgfsetdash{}{0pt}%
\pgfpathmoveto{\pgfqpoint{2.297297in}{1.071697in}}%
\pgfpathcurveto{\pgfqpoint{2.305533in}{1.071697in}}{\pgfqpoint{2.313433in}{1.074969in}}{\pgfqpoint{2.319257in}{1.080793in}}%
\pgfpathcurveto{\pgfqpoint{2.325081in}{1.086617in}}{\pgfqpoint{2.328353in}{1.094517in}}{\pgfqpoint{2.328353in}{1.102753in}}%
\pgfpathcurveto{\pgfqpoint{2.328353in}{1.110990in}}{\pgfqpoint{2.325081in}{1.118890in}}{\pgfqpoint{2.319257in}{1.124714in}}%
\pgfpathcurveto{\pgfqpoint{2.313433in}{1.130538in}}{\pgfqpoint{2.305533in}{1.133810in}}{\pgfqpoint{2.297297in}{1.133810in}}%
\pgfpathcurveto{\pgfqpoint{2.289060in}{1.133810in}}{\pgfqpoint{2.281160in}{1.130538in}}{\pgfqpoint{2.275336in}{1.124714in}}%
\pgfpathcurveto{\pgfqpoint{2.269512in}{1.118890in}}{\pgfqpoint{2.266240in}{1.110990in}}{\pgfqpoint{2.266240in}{1.102753in}}%
\pgfpathcurveto{\pgfqpoint{2.266240in}{1.094517in}}{\pgfqpoint{2.269512in}{1.086617in}}{\pgfqpoint{2.275336in}{1.080793in}}%
\pgfpathcurveto{\pgfqpoint{2.281160in}{1.074969in}}{\pgfqpoint{2.289060in}{1.071697in}}{\pgfqpoint{2.297297in}{1.071697in}}%
\pgfpathclose%
\pgfusepath{stroke,fill}%
\end{pgfscope}%
\begin{pgfscope}%
\pgfpathrectangle{\pgfqpoint{0.457963in}{0.528059in}}{\pgfqpoint{6.200000in}{2.285714in}} %
\pgfusepath{clip}%
\pgfsetbuttcap%
\pgfsetroundjoin%
\definecolor{currentfill}{rgb}{0.666667,0.666667,1.000000}%
\pgfsetfillcolor{currentfill}%
\pgfsetlinewidth{1.003750pt}%
\definecolor{currentstroke}{rgb}{0.666667,0.666667,1.000000}%
\pgfsetstrokecolor{currentstroke}%
\pgfsetdash{}{0pt}%
\pgfpathmoveto{\pgfqpoint{3.516630in}{0.941085in}}%
\pgfpathcurveto{\pgfqpoint{3.524866in}{0.941085in}}{\pgfqpoint{3.532766in}{0.944357in}}{\pgfqpoint{3.538590in}{0.950181in}}%
\pgfpathcurveto{\pgfqpoint{3.544414in}{0.956005in}}{\pgfqpoint{3.547686in}{0.963905in}}{\pgfqpoint{3.547686in}{0.972141in}}%
\pgfpathcurveto{\pgfqpoint{3.547686in}{0.980377in}}{\pgfqpoint{3.544414in}{0.988277in}}{\pgfqpoint{3.538590in}{0.994101in}}%
\pgfpathcurveto{\pgfqpoint{3.532766in}{0.999925in}}{\pgfqpoint{3.524866in}{1.003198in}}{\pgfqpoint{3.516630in}{1.003198in}}%
\pgfpathcurveto{\pgfqpoint{3.508394in}{1.003198in}}{\pgfqpoint{3.500494in}{0.999925in}}{\pgfqpoint{3.494670in}{0.994101in}}%
\pgfpathcurveto{\pgfqpoint{3.488846in}{0.988277in}}{\pgfqpoint{3.485574in}{0.980377in}}{\pgfqpoint{3.485574in}{0.972141in}}%
\pgfpathcurveto{\pgfqpoint{3.485574in}{0.963905in}}{\pgfqpoint{3.488846in}{0.956005in}}{\pgfqpoint{3.494670in}{0.950181in}}%
\pgfpathcurveto{\pgfqpoint{3.500494in}{0.944357in}}{\pgfqpoint{3.508394in}{0.941085in}}{\pgfqpoint{3.516630in}{0.941085in}}%
\pgfpathclose%
\pgfusepath{stroke,fill}%
\end{pgfscope}%
\begin{pgfscope}%
\pgfpathrectangle{\pgfqpoint{0.457963in}{0.528059in}}{\pgfqpoint{6.200000in}{2.285714in}} %
\pgfusepath{clip}%
\pgfsetbuttcap%
\pgfsetroundjoin%
\definecolor{currentfill}{rgb}{0.666667,0.666667,1.000000}%
\pgfsetfillcolor{currentfill}%
\pgfsetlinewidth{1.003750pt}%
\definecolor{currentstroke}{rgb}{0.666667,0.666667,1.000000}%
\pgfsetstrokecolor{currentstroke}%
\pgfsetdash{}{0pt}%
\pgfpathmoveto{\pgfqpoint{4.301963in}{0.797411in}}%
\pgfpathcurveto{\pgfqpoint{4.310200in}{0.797411in}}{\pgfqpoint{4.318100in}{0.800683in}}{\pgfqpoint{4.323924in}{0.806507in}}%
\pgfpathcurveto{\pgfqpoint{4.329748in}{0.812331in}}{\pgfqpoint{4.333020in}{0.820231in}}{\pgfqpoint{4.333020in}{0.828468in}}%
\pgfpathcurveto{\pgfqpoint{4.333020in}{0.836704in}}{\pgfqpoint{4.329748in}{0.844604in}}{\pgfqpoint{4.323924in}{0.850428in}}%
\pgfpathcurveto{\pgfqpoint{4.318100in}{0.856252in}}{\pgfqpoint{4.310200in}{0.859524in}}{\pgfqpoint{4.301963in}{0.859524in}}%
\pgfpathcurveto{\pgfqpoint{4.293727in}{0.859524in}}{\pgfqpoint{4.285827in}{0.856252in}}{\pgfqpoint{4.280003in}{0.850428in}}%
\pgfpathcurveto{\pgfqpoint{4.274179in}{0.844604in}}{\pgfqpoint{4.270907in}{0.836704in}}{\pgfqpoint{4.270907in}{0.828468in}}%
\pgfpathcurveto{\pgfqpoint{4.270907in}{0.820231in}}{\pgfqpoint{4.274179in}{0.812331in}}{\pgfqpoint{4.280003in}{0.806507in}}%
\pgfpathcurveto{\pgfqpoint{4.285827in}{0.800683in}}{\pgfqpoint{4.293727in}{0.797411in}}{\pgfqpoint{4.301963in}{0.797411in}}%
\pgfpathclose%
\pgfusepath{stroke,fill}%
\end{pgfscope}%
\begin{pgfscope}%
\pgfpathrectangle{\pgfqpoint{0.457963in}{0.528059in}}{\pgfqpoint{6.200000in}{2.285714in}} %
\pgfusepath{clip}%
\pgfsetbuttcap%
\pgfsetroundjoin%
\definecolor{currentfill}{rgb}{0.500000,0.500000,1.000000}%
\pgfsetfillcolor{currentfill}%
\pgfsetlinewidth{1.003750pt}%
\definecolor{currentstroke}{rgb}{0.500000,0.500000,1.000000}%
\pgfsetstrokecolor{currentstroke}%
\pgfsetdash{}{0pt}%
\pgfpathmoveto{\pgfqpoint{0.457963in}{1.476595in}}%
\pgfpathcurveto{\pgfqpoint{0.466200in}{1.476595in}}{\pgfqpoint{0.474100in}{1.479867in}}{\pgfqpoint{0.479924in}{1.485691in}}%
\pgfpathcurveto{\pgfqpoint{0.485748in}{1.491515in}}{\pgfqpoint{0.489020in}{1.499415in}}{\pgfqpoint{0.489020in}{1.507651in}}%
\pgfpathcurveto{\pgfqpoint{0.489020in}{1.515888in}}{\pgfqpoint{0.485748in}{1.523788in}}{\pgfqpoint{0.479924in}{1.529612in}}%
\pgfpathcurveto{\pgfqpoint{0.474100in}{1.535435in}}{\pgfqpoint{0.466200in}{1.538708in}}{\pgfqpoint{0.457963in}{1.538708in}}%
\pgfpathcurveto{\pgfqpoint{0.449727in}{1.538708in}}{\pgfqpoint{0.441827in}{1.535435in}}{\pgfqpoint{0.436003in}{1.529612in}}%
\pgfpathcurveto{\pgfqpoint{0.430179in}{1.523788in}}{\pgfqpoint{0.426907in}{1.515888in}}{\pgfqpoint{0.426907in}{1.507651in}}%
\pgfpathcurveto{\pgfqpoint{0.426907in}{1.499415in}}{\pgfqpoint{0.430179in}{1.491515in}}{\pgfqpoint{0.436003in}{1.485691in}}%
\pgfpathcurveto{\pgfqpoint{0.441827in}{1.479867in}}{\pgfqpoint{0.449727in}{1.476595in}}{\pgfqpoint{0.457963in}{1.476595in}}%
\pgfpathclose%
\pgfusepath{stroke,fill}%
\end{pgfscope}%
\begin{pgfscope}%
\pgfpathrectangle{\pgfqpoint{0.457963in}{0.528059in}}{\pgfqpoint{6.200000in}{2.285714in}} %
\pgfusepath{clip}%
\pgfsetbuttcap%
\pgfsetroundjoin%
\definecolor{currentfill}{rgb}{0.500000,0.500000,1.000000}%
\pgfsetfillcolor{currentfill}%
\pgfsetlinewidth{1.003750pt}%
\definecolor{currentstroke}{rgb}{0.500000,0.500000,1.000000}%
\pgfsetstrokecolor{currentstroke}%
\pgfsetdash{}{0pt}%
\pgfpathmoveto{\pgfqpoint{0.457963in}{1.476595in}}%
\pgfpathcurveto{\pgfqpoint{0.466200in}{1.476595in}}{\pgfqpoint{0.474100in}{1.479867in}}{\pgfqpoint{0.479924in}{1.485691in}}%
\pgfpathcurveto{\pgfqpoint{0.485748in}{1.491515in}}{\pgfqpoint{0.489020in}{1.499415in}}{\pgfqpoint{0.489020in}{1.507651in}}%
\pgfpathcurveto{\pgfqpoint{0.489020in}{1.515888in}}{\pgfqpoint{0.485748in}{1.523788in}}{\pgfqpoint{0.479924in}{1.529612in}}%
\pgfpathcurveto{\pgfqpoint{0.474100in}{1.535435in}}{\pgfqpoint{0.466200in}{1.538708in}}{\pgfqpoint{0.457963in}{1.538708in}}%
\pgfpathcurveto{\pgfqpoint{0.449727in}{1.538708in}}{\pgfqpoint{0.441827in}{1.535435in}}{\pgfqpoint{0.436003in}{1.529612in}}%
\pgfpathcurveto{\pgfqpoint{0.430179in}{1.523788in}}{\pgfqpoint{0.426907in}{1.515888in}}{\pgfqpoint{0.426907in}{1.507651in}}%
\pgfpathcurveto{\pgfqpoint{0.426907in}{1.499415in}}{\pgfqpoint{0.430179in}{1.491515in}}{\pgfqpoint{0.436003in}{1.485691in}}%
\pgfpathcurveto{\pgfqpoint{0.441827in}{1.479867in}}{\pgfqpoint{0.449727in}{1.476595in}}{\pgfqpoint{0.457963in}{1.476595in}}%
\pgfpathclose%
\pgfusepath{stroke,fill}%
\end{pgfscope}%
\begin{pgfscope}%
\pgfpathrectangle{\pgfqpoint{0.457963in}{0.528059in}}{\pgfqpoint{6.200000in}{2.285714in}} %
\pgfusepath{clip}%
\pgfsetbuttcap%
\pgfsetroundjoin%
\definecolor{currentfill}{rgb}{0.500000,0.500000,1.000000}%
\pgfsetfillcolor{currentfill}%
\pgfsetlinewidth{1.003750pt}%
\definecolor{currentstroke}{rgb}{0.500000,0.500000,1.000000}%
\pgfsetstrokecolor{currentstroke}%
\pgfsetdash{}{0pt}%
\pgfpathmoveto{\pgfqpoint{0.468297in}{1.476595in}}%
\pgfpathcurveto{\pgfqpoint{0.476533in}{1.476595in}}{\pgfqpoint{0.484433in}{1.479867in}}{\pgfqpoint{0.490257in}{1.485691in}}%
\pgfpathcurveto{\pgfqpoint{0.496081in}{1.491515in}}{\pgfqpoint{0.499353in}{1.499415in}}{\pgfqpoint{0.499353in}{1.507651in}}%
\pgfpathcurveto{\pgfqpoint{0.499353in}{1.515888in}}{\pgfqpoint{0.496081in}{1.523788in}}{\pgfqpoint{0.490257in}{1.529612in}}%
\pgfpathcurveto{\pgfqpoint{0.484433in}{1.535435in}}{\pgfqpoint{0.476533in}{1.538708in}}{\pgfqpoint{0.468297in}{1.538708in}}%
\pgfpathcurveto{\pgfqpoint{0.460060in}{1.538708in}}{\pgfqpoint{0.452160in}{1.535435in}}{\pgfqpoint{0.446336in}{1.529612in}}%
\pgfpathcurveto{\pgfqpoint{0.440512in}{1.523788in}}{\pgfqpoint{0.437240in}{1.515888in}}{\pgfqpoint{0.437240in}{1.507651in}}%
\pgfpathcurveto{\pgfqpoint{0.437240in}{1.499415in}}{\pgfqpoint{0.440512in}{1.491515in}}{\pgfqpoint{0.446336in}{1.485691in}}%
\pgfpathcurveto{\pgfqpoint{0.452160in}{1.479867in}}{\pgfqpoint{0.460060in}{1.476595in}}{\pgfqpoint{0.468297in}{1.476595in}}%
\pgfpathclose%
\pgfusepath{stroke,fill}%
\end{pgfscope}%
\begin{pgfscope}%
\pgfpathrectangle{\pgfqpoint{0.457963in}{0.528059in}}{\pgfqpoint{6.200000in}{2.285714in}} %
\pgfusepath{clip}%
\pgfsetbuttcap%
\pgfsetroundjoin%
\definecolor{currentfill}{rgb}{0.500000,0.500000,1.000000}%
\pgfsetfillcolor{currentfill}%
\pgfsetlinewidth{1.003750pt}%
\definecolor{currentstroke}{rgb}{0.500000,0.500000,1.000000}%
\pgfsetstrokecolor{currentstroke}%
\pgfsetdash{}{0pt}%
\pgfpathmoveto{\pgfqpoint{0.468297in}{1.476595in}}%
\pgfpathcurveto{\pgfqpoint{0.476533in}{1.476595in}}{\pgfqpoint{0.484433in}{1.479867in}}{\pgfqpoint{0.490257in}{1.485691in}}%
\pgfpathcurveto{\pgfqpoint{0.496081in}{1.491515in}}{\pgfqpoint{0.499353in}{1.499415in}}{\pgfqpoint{0.499353in}{1.507651in}}%
\pgfpathcurveto{\pgfqpoint{0.499353in}{1.515888in}}{\pgfqpoint{0.496081in}{1.523788in}}{\pgfqpoint{0.490257in}{1.529612in}}%
\pgfpathcurveto{\pgfqpoint{0.484433in}{1.535435in}}{\pgfqpoint{0.476533in}{1.538708in}}{\pgfqpoint{0.468297in}{1.538708in}}%
\pgfpathcurveto{\pgfqpoint{0.460060in}{1.538708in}}{\pgfqpoint{0.452160in}{1.535435in}}{\pgfqpoint{0.446336in}{1.529612in}}%
\pgfpathcurveto{\pgfqpoint{0.440512in}{1.523788in}}{\pgfqpoint{0.437240in}{1.515888in}}{\pgfqpoint{0.437240in}{1.507651in}}%
\pgfpathcurveto{\pgfqpoint{0.437240in}{1.499415in}}{\pgfqpoint{0.440512in}{1.491515in}}{\pgfqpoint{0.446336in}{1.485691in}}%
\pgfpathcurveto{\pgfqpoint{0.452160in}{1.479867in}}{\pgfqpoint{0.460060in}{1.476595in}}{\pgfqpoint{0.468297in}{1.476595in}}%
\pgfpathclose%
\pgfusepath{stroke,fill}%
\end{pgfscope}%
\begin{pgfscope}%
\pgfpathrectangle{\pgfqpoint{0.457963in}{0.528059in}}{\pgfqpoint{6.200000in}{2.285714in}} %
\pgfusepath{clip}%
\pgfsetbuttcap%
\pgfsetroundjoin%
\definecolor{currentfill}{rgb}{0.500000,0.500000,1.000000}%
\pgfsetfillcolor{currentfill}%
\pgfsetlinewidth{1.003750pt}%
\definecolor{currentstroke}{rgb}{0.500000,0.500000,1.000000}%
\pgfsetstrokecolor{currentstroke}%
\pgfsetdash{}{0pt}%
\pgfpathmoveto{\pgfqpoint{0.478630in}{1.476595in}}%
\pgfpathcurveto{\pgfqpoint{0.486866in}{1.476595in}}{\pgfqpoint{0.494766in}{1.479867in}}{\pgfqpoint{0.500590in}{1.485691in}}%
\pgfpathcurveto{\pgfqpoint{0.506414in}{1.491515in}}{\pgfqpoint{0.509686in}{1.499415in}}{\pgfqpoint{0.509686in}{1.507651in}}%
\pgfpathcurveto{\pgfqpoint{0.509686in}{1.515888in}}{\pgfqpoint{0.506414in}{1.523788in}}{\pgfqpoint{0.500590in}{1.529612in}}%
\pgfpathcurveto{\pgfqpoint{0.494766in}{1.535435in}}{\pgfqpoint{0.486866in}{1.538708in}}{\pgfqpoint{0.478630in}{1.538708in}}%
\pgfpathcurveto{\pgfqpoint{0.470394in}{1.538708in}}{\pgfqpoint{0.462494in}{1.535435in}}{\pgfqpoint{0.456670in}{1.529612in}}%
\pgfpathcurveto{\pgfqpoint{0.450846in}{1.523788in}}{\pgfqpoint{0.447574in}{1.515888in}}{\pgfqpoint{0.447574in}{1.507651in}}%
\pgfpathcurveto{\pgfqpoint{0.447574in}{1.499415in}}{\pgfqpoint{0.450846in}{1.491515in}}{\pgfqpoint{0.456670in}{1.485691in}}%
\pgfpathcurveto{\pgfqpoint{0.462494in}{1.479867in}}{\pgfqpoint{0.470394in}{1.476595in}}{\pgfqpoint{0.478630in}{1.476595in}}%
\pgfpathclose%
\pgfusepath{stroke,fill}%
\end{pgfscope}%
\begin{pgfscope}%
\pgfpathrectangle{\pgfqpoint{0.457963in}{0.528059in}}{\pgfqpoint{6.200000in}{2.285714in}} %
\pgfusepath{clip}%
\pgfsetbuttcap%
\pgfsetroundjoin%
\definecolor{currentfill}{rgb}{0.500000,0.500000,1.000000}%
\pgfsetfillcolor{currentfill}%
\pgfsetlinewidth{1.003750pt}%
\definecolor{currentstroke}{rgb}{0.500000,0.500000,1.000000}%
\pgfsetstrokecolor{currentstroke}%
\pgfsetdash{}{0pt}%
\pgfpathmoveto{\pgfqpoint{0.488963in}{1.476595in}}%
\pgfpathcurveto{\pgfqpoint{0.497200in}{1.476595in}}{\pgfqpoint{0.505100in}{1.479867in}}{\pgfqpoint{0.510924in}{1.485691in}}%
\pgfpathcurveto{\pgfqpoint{0.516748in}{1.491515in}}{\pgfqpoint{0.520020in}{1.499415in}}{\pgfqpoint{0.520020in}{1.507651in}}%
\pgfpathcurveto{\pgfqpoint{0.520020in}{1.515888in}}{\pgfqpoint{0.516748in}{1.523788in}}{\pgfqpoint{0.510924in}{1.529612in}}%
\pgfpathcurveto{\pgfqpoint{0.505100in}{1.535435in}}{\pgfqpoint{0.497200in}{1.538708in}}{\pgfqpoint{0.488963in}{1.538708in}}%
\pgfpathcurveto{\pgfqpoint{0.480727in}{1.538708in}}{\pgfqpoint{0.472827in}{1.535435in}}{\pgfqpoint{0.467003in}{1.529612in}}%
\pgfpathcurveto{\pgfqpoint{0.461179in}{1.523788in}}{\pgfqpoint{0.457907in}{1.515888in}}{\pgfqpoint{0.457907in}{1.507651in}}%
\pgfpathcurveto{\pgfqpoint{0.457907in}{1.499415in}}{\pgfqpoint{0.461179in}{1.491515in}}{\pgfqpoint{0.467003in}{1.485691in}}%
\pgfpathcurveto{\pgfqpoint{0.472827in}{1.479867in}}{\pgfqpoint{0.480727in}{1.476595in}}{\pgfqpoint{0.488963in}{1.476595in}}%
\pgfpathclose%
\pgfusepath{stroke,fill}%
\end{pgfscope}%
\begin{pgfscope}%
\pgfpathrectangle{\pgfqpoint{0.457963in}{0.528059in}}{\pgfqpoint{6.200000in}{2.285714in}} %
\pgfusepath{clip}%
\pgfsetbuttcap%
\pgfsetroundjoin%
\definecolor{currentfill}{rgb}{0.500000,0.500000,1.000000}%
\pgfsetfillcolor{currentfill}%
\pgfsetlinewidth{1.003750pt}%
\definecolor{currentstroke}{rgb}{0.500000,0.500000,1.000000}%
\pgfsetstrokecolor{currentstroke}%
\pgfsetdash{}{0pt}%
\pgfpathmoveto{\pgfqpoint{0.509630in}{1.463534in}}%
\pgfpathcurveto{\pgfqpoint{0.517866in}{1.463534in}}{\pgfqpoint{0.525766in}{1.466806in}}{\pgfqpoint{0.531590in}{1.472630in}}%
\pgfpathcurveto{\pgfqpoint{0.537414in}{1.478454in}}{\pgfqpoint{0.540686in}{1.486354in}}{\pgfqpoint{0.540686in}{1.494590in}}%
\pgfpathcurveto{\pgfqpoint{0.540686in}{1.502826in}}{\pgfqpoint{0.537414in}{1.510726in}}{\pgfqpoint{0.531590in}{1.516550in}}%
\pgfpathcurveto{\pgfqpoint{0.525766in}{1.522374in}}{\pgfqpoint{0.517866in}{1.525647in}}{\pgfqpoint{0.509630in}{1.525647in}}%
\pgfpathcurveto{\pgfqpoint{0.501394in}{1.525647in}}{\pgfqpoint{0.493494in}{1.522374in}}{\pgfqpoint{0.487670in}{1.516550in}}%
\pgfpathcurveto{\pgfqpoint{0.481846in}{1.510726in}}{\pgfqpoint{0.478574in}{1.502826in}}{\pgfqpoint{0.478574in}{1.494590in}}%
\pgfpathcurveto{\pgfqpoint{0.478574in}{1.486354in}}{\pgfqpoint{0.481846in}{1.478454in}}{\pgfqpoint{0.487670in}{1.472630in}}%
\pgfpathcurveto{\pgfqpoint{0.493494in}{1.466806in}}{\pgfqpoint{0.501394in}{1.463534in}}{\pgfqpoint{0.509630in}{1.463534in}}%
\pgfpathclose%
\pgfusepath{stroke,fill}%
\end{pgfscope}%
\begin{pgfscope}%
\pgfpathrectangle{\pgfqpoint{0.457963in}{0.528059in}}{\pgfqpoint{6.200000in}{2.285714in}} %
\pgfusepath{clip}%
\pgfsetbuttcap%
\pgfsetroundjoin%
\definecolor{currentfill}{rgb}{0.500000,0.500000,1.000000}%
\pgfsetfillcolor{currentfill}%
\pgfsetlinewidth{1.003750pt}%
\definecolor{currentstroke}{rgb}{0.500000,0.500000,1.000000}%
\pgfsetstrokecolor{currentstroke}%
\pgfsetdash{}{0pt}%
\pgfpathmoveto{\pgfqpoint{0.540630in}{1.476595in}}%
\pgfpathcurveto{\pgfqpoint{0.548866in}{1.476595in}}{\pgfqpoint{0.556766in}{1.479867in}}{\pgfqpoint{0.562590in}{1.485691in}}%
\pgfpathcurveto{\pgfqpoint{0.568414in}{1.491515in}}{\pgfqpoint{0.571686in}{1.499415in}}{\pgfqpoint{0.571686in}{1.507651in}}%
\pgfpathcurveto{\pgfqpoint{0.571686in}{1.515888in}}{\pgfqpoint{0.568414in}{1.523788in}}{\pgfqpoint{0.562590in}{1.529612in}}%
\pgfpathcurveto{\pgfqpoint{0.556766in}{1.535435in}}{\pgfqpoint{0.548866in}{1.538708in}}{\pgfqpoint{0.540630in}{1.538708in}}%
\pgfpathcurveto{\pgfqpoint{0.532394in}{1.538708in}}{\pgfqpoint{0.524494in}{1.535435in}}{\pgfqpoint{0.518670in}{1.529612in}}%
\pgfpathcurveto{\pgfqpoint{0.512846in}{1.523788in}}{\pgfqpoint{0.509574in}{1.515888in}}{\pgfqpoint{0.509574in}{1.507651in}}%
\pgfpathcurveto{\pgfqpoint{0.509574in}{1.499415in}}{\pgfqpoint{0.512846in}{1.491515in}}{\pgfqpoint{0.518670in}{1.485691in}}%
\pgfpathcurveto{\pgfqpoint{0.524494in}{1.479867in}}{\pgfqpoint{0.532394in}{1.476595in}}{\pgfqpoint{0.540630in}{1.476595in}}%
\pgfpathclose%
\pgfusepath{stroke,fill}%
\end{pgfscope}%
\begin{pgfscope}%
\pgfpathrectangle{\pgfqpoint{0.457963in}{0.528059in}}{\pgfqpoint{6.200000in}{2.285714in}} %
\pgfusepath{clip}%
\pgfsetbuttcap%
\pgfsetroundjoin%
\definecolor{currentfill}{rgb}{0.500000,0.500000,1.000000}%
\pgfsetfillcolor{currentfill}%
\pgfsetlinewidth{1.003750pt}%
\definecolor{currentstroke}{rgb}{0.500000,0.500000,1.000000}%
\pgfsetstrokecolor{currentstroke}%
\pgfsetdash{}{0pt}%
\pgfpathmoveto{\pgfqpoint{0.623297in}{1.463534in}}%
\pgfpathcurveto{\pgfqpoint{0.631533in}{1.463534in}}{\pgfqpoint{0.639433in}{1.466806in}}{\pgfqpoint{0.645257in}{1.472630in}}%
\pgfpathcurveto{\pgfqpoint{0.651081in}{1.478454in}}{\pgfqpoint{0.654353in}{1.486354in}}{\pgfqpoint{0.654353in}{1.494590in}}%
\pgfpathcurveto{\pgfqpoint{0.654353in}{1.502826in}}{\pgfqpoint{0.651081in}{1.510726in}}{\pgfqpoint{0.645257in}{1.516550in}}%
\pgfpathcurveto{\pgfqpoint{0.639433in}{1.522374in}}{\pgfqpoint{0.631533in}{1.525647in}}{\pgfqpoint{0.623297in}{1.525647in}}%
\pgfpathcurveto{\pgfqpoint{0.615060in}{1.525647in}}{\pgfqpoint{0.607160in}{1.522374in}}{\pgfqpoint{0.601336in}{1.516550in}}%
\pgfpathcurveto{\pgfqpoint{0.595512in}{1.510726in}}{\pgfqpoint{0.592240in}{1.502826in}}{\pgfqpoint{0.592240in}{1.494590in}}%
\pgfpathcurveto{\pgfqpoint{0.592240in}{1.486354in}}{\pgfqpoint{0.595512in}{1.478454in}}{\pgfqpoint{0.601336in}{1.472630in}}%
\pgfpathcurveto{\pgfqpoint{0.607160in}{1.466806in}}{\pgfqpoint{0.615060in}{1.463534in}}{\pgfqpoint{0.623297in}{1.463534in}}%
\pgfpathclose%
\pgfusepath{stroke,fill}%
\end{pgfscope}%
\begin{pgfscope}%
\pgfpathrectangle{\pgfqpoint{0.457963in}{0.528059in}}{\pgfqpoint{6.200000in}{2.285714in}} %
\pgfusepath{clip}%
\pgfsetbuttcap%
\pgfsetroundjoin%
\definecolor{currentfill}{rgb}{0.500000,0.500000,1.000000}%
\pgfsetfillcolor{currentfill}%
\pgfsetlinewidth{1.003750pt}%
\definecolor{currentstroke}{rgb}{0.500000,0.500000,1.000000}%
\pgfsetstrokecolor{currentstroke}%
\pgfsetdash{}{0pt}%
\pgfpathmoveto{\pgfqpoint{0.633630in}{1.450472in}}%
\pgfpathcurveto{\pgfqpoint{0.641866in}{1.450472in}}{\pgfqpoint{0.649766in}{1.453745in}}{\pgfqpoint{0.655590in}{1.459569in}}%
\pgfpathcurveto{\pgfqpoint{0.661414in}{1.465393in}}{\pgfqpoint{0.664686in}{1.473293in}}{\pgfqpoint{0.664686in}{1.481529in}}%
\pgfpathcurveto{\pgfqpoint{0.664686in}{1.489765in}}{\pgfqpoint{0.661414in}{1.497665in}}{\pgfqpoint{0.655590in}{1.503489in}}%
\pgfpathcurveto{\pgfqpoint{0.649766in}{1.509313in}}{\pgfqpoint{0.641866in}{1.512585in}}{\pgfqpoint{0.633630in}{1.512585in}}%
\pgfpathcurveto{\pgfqpoint{0.625394in}{1.512585in}}{\pgfqpoint{0.617494in}{1.509313in}}{\pgfqpoint{0.611670in}{1.503489in}}%
\pgfpathcurveto{\pgfqpoint{0.605846in}{1.497665in}}{\pgfqpoint{0.602574in}{1.489765in}}{\pgfqpoint{0.602574in}{1.481529in}}%
\pgfpathcurveto{\pgfqpoint{0.602574in}{1.473293in}}{\pgfqpoint{0.605846in}{1.465393in}}{\pgfqpoint{0.611670in}{1.459569in}}%
\pgfpathcurveto{\pgfqpoint{0.617494in}{1.453745in}}{\pgfqpoint{0.625394in}{1.450472in}}{\pgfqpoint{0.633630in}{1.450472in}}%
\pgfpathclose%
\pgfusepath{stroke,fill}%
\end{pgfscope}%
\begin{pgfscope}%
\pgfpathrectangle{\pgfqpoint{0.457963in}{0.528059in}}{\pgfqpoint{6.200000in}{2.285714in}} %
\pgfusepath{clip}%
\pgfsetbuttcap%
\pgfsetroundjoin%
\definecolor{currentfill}{rgb}{0.500000,0.500000,1.000000}%
\pgfsetfillcolor{currentfill}%
\pgfsetlinewidth{1.003750pt}%
\definecolor{currentstroke}{rgb}{0.500000,0.500000,1.000000}%
\pgfsetstrokecolor{currentstroke}%
\pgfsetdash{}{0pt}%
\pgfpathmoveto{\pgfqpoint{0.664630in}{1.437411in}}%
\pgfpathcurveto{\pgfqpoint{0.672866in}{1.437411in}}{\pgfqpoint{0.680766in}{1.440683in}}{\pgfqpoint{0.686590in}{1.446507in}}%
\pgfpathcurveto{\pgfqpoint{0.692414in}{1.452331in}}{\pgfqpoint{0.695686in}{1.460231in}}{\pgfqpoint{0.695686in}{1.468468in}}%
\pgfpathcurveto{\pgfqpoint{0.695686in}{1.476704in}}{\pgfqpoint{0.692414in}{1.484604in}}{\pgfqpoint{0.686590in}{1.490428in}}%
\pgfpathcurveto{\pgfqpoint{0.680766in}{1.496252in}}{\pgfqpoint{0.672866in}{1.499524in}}{\pgfqpoint{0.664630in}{1.499524in}}%
\pgfpathcurveto{\pgfqpoint{0.656394in}{1.499524in}}{\pgfqpoint{0.648494in}{1.496252in}}{\pgfqpoint{0.642670in}{1.490428in}}%
\pgfpathcurveto{\pgfqpoint{0.636846in}{1.484604in}}{\pgfqpoint{0.633574in}{1.476704in}}{\pgfqpoint{0.633574in}{1.468468in}}%
\pgfpathcurveto{\pgfqpoint{0.633574in}{1.460231in}}{\pgfqpoint{0.636846in}{1.452331in}}{\pgfqpoint{0.642670in}{1.446507in}}%
\pgfpathcurveto{\pgfqpoint{0.648494in}{1.440683in}}{\pgfqpoint{0.656394in}{1.437411in}}{\pgfqpoint{0.664630in}{1.437411in}}%
\pgfpathclose%
\pgfusepath{stroke,fill}%
\end{pgfscope}%
\begin{pgfscope}%
\pgfpathrectangle{\pgfqpoint{0.457963in}{0.528059in}}{\pgfqpoint{6.200000in}{2.285714in}} %
\pgfusepath{clip}%
\pgfsetbuttcap%
\pgfsetroundjoin%
\definecolor{currentfill}{rgb}{0.500000,0.500000,1.000000}%
\pgfsetfillcolor{currentfill}%
\pgfsetlinewidth{1.003750pt}%
\definecolor{currentstroke}{rgb}{0.500000,0.500000,1.000000}%
\pgfsetstrokecolor{currentstroke}%
\pgfsetdash{}{0pt}%
\pgfpathmoveto{\pgfqpoint{0.943630in}{1.463534in}}%
\pgfpathcurveto{\pgfqpoint{0.951866in}{1.463534in}}{\pgfqpoint{0.959766in}{1.466806in}}{\pgfqpoint{0.965590in}{1.472630in}}%
\pgfpathcurveto{\pgfqpoint{0.971414in}{1.478454in}}{\pgfqpoint{0.974686in}{1.486354in}}{\pgfqpoint{0.974686in}{1.494590in}}%
\pgfpathcurveto{\pgfqpoint{0.974686in}{1.502826in}}{\pgfqpoint{0.971414in}{1.510726in}}{\pgfqpoint{0.965590in}{1.516550in}}%
\pgfpathcurveto{\pgfqpoint{0.959766in}{1.522374in}}{\pgfqpoint{0.951866in}{1.525647in}}{\pgfqpoint{0.943630in}{1.525647in}}%
\pgfpathcurveto{\pgfqpoint{0.935394in}{1.525647in}}{\pgfqpoint{0.927494in}{1.522374in}}{\pgfqpoint{0.921670in}{1.516550in}}%
\pgfpathcurveto{\pgfqpoint{0.915846in}{1.510726in}}{\pgfqpoint{0.912574in}{1.502826in}}{\pgfqpoint{0.912574in}{1.494590in}}%
\pgfpathcurveto{\pgfqpoint{0.912574in}{1.486354in}}{\pgfqpoint{0.915846in}{1.478454in}}{\pgfqpoint{0.921670in}{1.472630in}}%
\pgfpathcurveto{\pgfqpoint{0.927494in}{1.466806in}}{\pgfqpoint{0.935394in}{1.463534in}}{\pgfqpoint{0.943630in}{1.463534in}}%
\pgfpathclose%
\pgfusepath{stroke,fill}%
\end{pgfscope}%
\begin{pgfscope}%
\pgfpathrectangle{\pgfqpoint{0.457963in}{0.528059in}}{\pgfqpoint{6.200000in}{2.285714in}} %
\pgfusepath{clip}%
\pgfsetbuttcap%
\pgfsetroundjoin%
\definecolor{currentfill}{rgb}{0.500000,0.500000,1.000000}%
\pgfsetfillcolor{currentfill}%
\pgfsetlinewidth{1.003750pt}%
\definecolor{currentstroke}{rgb}{0.500000,0.500000,1.000000}%
\pgfsetstrokecolor{currentstroke}%
\pgfsetdash{}{0pt}%
\pgfpathmoveto{\pgfqpoint{0.953963in}{1.437411in}}%
\pgfpathcurveto{\pgfqpoint{0.962200in}{1.437411in}}{\pgfqpoint{0.970100in}{1.440683in}}{\pgfqpoint{0.975924in}{1.446507in}}%
\pgfpathcurveto{\pgfqpoint{0.981748in}{1.452331in}}{\pgfqpoint{0.985020in}{1.460231in}}{\pgfqpoint{0.985020in}{1.468468in}}%
\pgfpathcurveto{\pgfqpoint{0.985020in}{1.476704in}}{\pgfqpoint{0.981748in}{1.484604in}}{\pgfqpoint{0.975924in}{1.490428in}}%
\pgfpathcurveto{\pgfqpoint{0.970100in}{1.496252in}}{\pgfqpoint{0.962200in}{1.499524in}}{\pgfqpoint{0.953963in}{1.499524in}}%
\pgfpathcurveto{\pgfqpoint{0.945727in}{1.499524in}}{\pgfqpoint{0.937827in}{1.496252in}}{\pgfqpoint{0.932003in}{1.490428in}}%
\pgfpathcurveto{\pgfqpoint{0.926179in}{1.484604in}}{\pgfqpoint{0.922907in}{1.476704in}}{\pgfqpoint{0.922907in}{1.468468in}}%
\pgfpathcurveto{\pgfqpoint{0.922907in}{1.460231in}}{\pgfqpoint{0.926179in}{1.452331in}}{\pgfqpoint{0.932003in}{1.446507in}}%
\pgfpathcurveto{\pgfqpoint{0.937827in}{1.440683in}}{\pgfqpoint{0.945727in}{1.437411in}}{\pgfqpoint{0.953963in}{1.437411in}}%
\pgfpathclose%
\pgfusepath{stroke,fill}%
\end{pgfscope}%
\begin{pgfscope}%
\pgfpathrectangle{\pgfqpoint{0.457963in}{0.528059in}}{\pgfqpoint{6.200000in}{2.285714in}} %
\pgfusepath{clip}%
\pgfsetbuttcap%
\pgfsetroundjoin%
\definecolor{currentfill}{rgb}{0.500000,0.500000,1.000000}%
\pgfsetfillcolor{currentfill}%
\pgfsetlinewidth{1.003750pt}%
\definecolor{currentstroke}{rgb}{0.500000,0.500000,1.000000}%
\pgfsetstrokecolor{currentstroke}%
\pgfsetdash{}{0pt}%
\pgfpathmoveto{\pgfqpoint{1.584297in}{1.345983in}}%
\pgfpathcurveto{\pgfqpoint{1.592533in}{1.345983in}}{\pgfqpoint{1.600433in}{1.349255in}}{\pgfqpoint{1.606257in}{1.355079in}}%
\pgfpathcurveto{\pgfqpoint{1.612081in}{1.360903in}}{\pgfqpoint{1.615353in}{1.368803in}}{\pgfqpoint{1.615353in}{1.377039in}}%
\pgfpathcurveto{\pgfqpoint{1.615353in}{1.385275in}}{\pgfqpoint{1.612081in}{1.393175in}}{\pgfqpoint{1.606257in}{1.398999in}}%
\pgfpathcurveto{\pgfqpoint{1.600433in}{1.404823in}}{\pgfqpoint{1.592533in}{1.408096in}}{\pgfqpoint{1.584297in}{1.408096in}}%
\pgfpathcurveto{\pgfqpoint{1.576060in}{1.408096in}}{\pgfqpoint{1.568160in}{1.404823in}}{\pgfqpoint{1.562336in}{1.398999in}}%
\pgfpathcurveto{\pgfqpoint{1.556512in}{1.393175in}}{\pgfqpoint{1.553240in}{1.385275in}}{\pgfqpoint{1.553240in}{1.377039in}}%
\pgfpathcurveto{\pgfqpoint{1.553240in}{1.368803in}}{\pgfqpoint{1.556512in}{1.360903in}}{\pgfqpoint{1.562336in}{1.355079in}}%
\pgfpathcurveto{\pgfqpoint{1.568160in}{1.349255in}}{\pgfqpoint{1.576060in}{1.345983in}}{\pgfqpoint{1.584297in}{1.345983in}}%
\pgfpathclose%
\pgfusepath{stroke,fill}%
\end{pgfscope}%
\begin{pgfscope}%
\pgfpathrectangle{\pgfqpoint{0.457963in}{0.528059in}}{\pgfqpoint{6.200000in}{2.285714in}} %
\pgfusepath{clip}%
\pgfsetbuttcap%
\pgfsetroundjoin%
\definecolor{currentfill}{rgb}{0.500000,0.500000,1.000000}%
\pgfsetfillcolor{currentfill}%
\pgfsetlinewidth{1.003750pt}%
\definecolor{currentstroke}{rgb}{0.500000,0.500000,1.000000}%
\pgfsetstrokecolor{currentstroke}%
\pgfsetdash{}{0pt}%
\pgfpathmoveto{\pgfqpoint{1.759963in}{1.463534in}}%
\pgfpathcurveto{\pgfqpoint{1.768200in}{1.463534in}}{\pgfqpoint{1.776100in}{1.466806in}}{\pgfqpoint{1.781924in}{1.472630in}}%
\pgfpathcurveto{\pgfqpoint{1.787748in}{1.478454in}}{\pgfqpoint{1.791020in}{1.486354in}}{\pgfqpoint{1.791020in}{1.494590in}}%
\pgfpathcurveto{\pgfqpoint{1.791020in}{1.502826in}}{\pgfqpoint{1.787748in}{1.510726in}}{\pgfqpoint{1.781924in}{1.516550in}}%
\pgfpathcurveto{\pgfqpoint{1.776100in}{1.522374in}}{\pgfqpoint{1.768200in}{1.525647in}}{\pgfqpoint{1.759963in}{1.525647in}}%
\pgfpathcurveto{\pgfqpoint{1.751727in}{1.525647in}}{\pgfqpoint{1.743827in}{1.522374in}}{\pgfqpoint{1.738003in}{1.516550in}}%
\pgfpathcurveto{\pgfqpoint{1.732179in}{1.510726in}}{\pgfqpoint{1.728907in}{1.502826in}}{\pgfqpoint{1.728907in}{1.494590in}}%
\pgfpathcurveto{\pgfqpoint{1.728907in}{1.486354in}}{\pgfqpoint{1.732179in}{1.478454in}}{\pgfqpoint{1.738003in}{1.472630in}}%
\pgfpathcurveto{\pgfqpoint{1.743827in}{1.466806in}}{\pgfqpoint{1.751727in}{1.463534in}}{\pgfqpoint{1.759963in}{1.463534in}}%
\pgfpathclose%
\pgfusepath{stroke,fill}%
\end{pgfscope}%
\begin{pgfscope}%
\pgfpathrectangle{\pgfqpoint{0.457963in}{0.528059in}}{\pgfqpoint{6.200000in}{2.285714in}} %
\pgfusepath{clip}%
\pgfsetbuttcap%
\pgfsetroundjoin%
\definecolor{currentfill}{rgb}{0.500000,0.500000,1.000000}%
\pgfsetfillcolor{currentfill}%
\pgfsetlinewidth{1.003750pt}%
\definecolor{currentstroke}{rgb}{0.500000,0.500000,1.000000}%
\pgfsetstrokecolor{currentstroke}%
\pgfsetdash{}{0pt}%
\pgfpathmoveto{\pgfqpoint{2.710630in}{1.450472in}}%
\pgfpathcurveto{\pgfqpoint{2.718866in}{1.450472in}}{\pgfqpoint{2.726766in}{1.453745in}}{\pgfqpoint{2.732590in}{1.459569in}}%
\pgfpathcurveto{\pgfqpoint{2.738414in}{1.465393in}}{\pgfqpoint{2.741686in}{1.473293in}}{\pgfqpoint{2.741686in}{1.481529in}}%
\pgfpathcurveto{\pgfqpoint{2.741686in}{1.489765in}}{\pgfqpoint{2.738414in}{1.497665in}}{\pgfqpoint{2.732590in}{1.503489in}}%
\pgfpathcurveto{\pgfqpoint{2.726766in}{1.509313in}}{\pgfqpoint{2.718866in}{1.512585in}}{\pgfqpoint{2.710630in}{1.512585in}}%
\pgfpathcurveto{\pgfqpoint{2.702394in}{1.512585in}}{\pgfqpoint{2.694494in}{1.509313in}}{\pgfqpoint{2.688670in}{1.503489in}}%
\pgfpathcurveto{\pgfqpoint{2.682846in}{1.497665in}}{\pgfqpoint{2.679574in}{1.489765in}}{\pgfqpoint{2.679574in}{1.481529in}}%
\pgfpathcurveto{\pgfqpoint{2.679574in}{1.473293in}}{\pgfqpoint{2.682846in}{1.465393in}}{\pgfqpoint{2.688670in}{1.459569in}}%
\pgfpathcurveto{\pgfqpoint{2.694494in}{1.453745in}}{\pgfqpoint{2.702394in}{1.450472in}}{\pgfqpoint{2.710630in}{1.450472in}}%
\pgfpathclose%
\pgfusepath{stroke,fill}%
\end{pgfscope}%
\begin{pgfscope}%
\pgfpathrectangle{\pgfqpoint{0.457963in}{0.528059in}}{\pgfqpoint{6.200000in}{2.285714in}} %
\pgfusepath{clip}%
\pgfsetbuttcap%
\pgfsetroundjoin%
\definecolor{currentfill}{rgb}{0.500000,0.500000,1.000000}%
\pgfsetfillcolor{currentfill}%
\pgfsetlinewidth{1.003750pt}%
\definecolor{currentstroke}{rgb}{0.500000,0.500000,1.000000}%
\pgfsetstrokecolor{currentstroke}%
\pgfsetdash{}{0pt}%
\pgfpathmoveto{\pgfqpoint{2.751963in}{1.150064in}}%
\pgfpathcurveto{\pgfqpoint{2.760200in}{1.150064in}}{\pgfqpoint{2.768100in}{1.153336in}}{\pgfqpoint{2.773924in}{1.159160in}}%
\pgfpathcurveto{\pgfqpoint{2.779748in}{1.164984in}}{\pgfqpoint{2.783020in}{1.172884in}}{\pgfqpoint{2.783020in}{1.181121in}}%
\pgfpathcurveto{\pgfqpoint{2.783020in}{1.189357in}}{\pgfqpoint{2.779748in}{1.197257in}}{\pgfqpoint{2.773924in}{1.203081in}}%
\pgfpathcurveto{\pgfqpoint{2.768100in}{1.208905in}}{\pgfqpoint{2.760200in}{1.212177in}}{\pgfqpoint{2.751963in}{1.212177in}}%
\pgfpathcurveto{\pgfqpoint{2.743727in}{1.212177in}}{\pgfqpoint{2.735827in}{1.208905in}}{\pgfqpoint{2.730003in}{1.203081in}}%
\pgfpathcurveto{\pgfqpoint{2.724179in}{1.197257in}}{\pgfqpoint{2.720907in}{1.189357in}}{\pgfqpoint{2.720907in}{1.181121in}}%
\pgfpathcurveto{\pgfqpoint{2.720907in}{1.172884in}}{\pgfqpoint{2.724179in}{1.164984in}}{\pgfqpoint{2.730003in}{1.159160in}}%
\pgfpathcurveto{\pgfqpoint{2.735827in}{1.153336in}}{\pgfqpoint{2.743727in}{1.150064in}}{\pgfqpoint{2.751963in}{1.150064in}}%
\pgfpathclose%
\pgfusepath{stroke,fill}%
\end{pgfscope}%
\begin{pgfscope}%
\pgfpathrectangle{\pgfqpoint{0.457963in}{0.528059in}}{\pgfqpoint{6.200000in}{2.285714in}} %
\pgfusepath{clip}%
\pgfsetbuttcap%
\pgfsetroundjoin%
\definecolor{currentfill}{rgb}{0.500000,0.500000,1.000000}%
\pgfsetfillcolor{currentfill}%
\pgfsetlinewidth{1.003750pt}%
\definecolor{currentstroke}{rgb}{0.500000,0.500000,1.000000}%
\pgfsetstrokecolor{currentstroke}%
\pgfsetdash{}{0pt}%
\pgfpathmoveto{\pgfqpoint{3.041297in}{1.359044in}}%
\pgfpathcurveto{\pgfqpoint{3.049533in}{1.359044in}}{\pgfqpoint{3.057433in}{1.362316in}}{\pgfqpoint{3.063257in}{1.368140in}}%
\pgfpathcurveto{\pgfqpoint{3.069081in}{1.373964in}}{\pgfqpoint{3.072353in}{1.381864in}}{\pgfqpoint{3.072353in}{1.390100in}}%
\pgfpathcurveto{\pgfqpoint{3.072353in}{1.398337in}}{\pgfqpoint{3.069081in}{1.406237in}}{\pgfqpoint{3.063257in}{1.412061in}}%
\pgfpathcurveto{\pgfqpoint{3.057433in}{1.417884in}}{\pgfqpoint{3.049533in}{1.421157in}}{\pgfqpoint{3.041297in}{1.421157in}}%
\pgfpathcurveto{\pgfqpoint{3.033060in}{1.421157in}}{\pgfqpoint{3.025160in}{1.417884in}}{\pgfqpoint{3.019336in}{1.412061in}}%
\pgfpathcurveto{\pgfqpoint{3.013512in}{1.406237in}}{\pgfqpoint{3.010240in}{1.398337in}}{\pgfqpoint{3.010240in}{1.390100in}}%
\pgfpathcurveto{\pgfqpoint{3.010240in}{1.381864in}}{\pgfqpoint{3.013512in}{1.373964in}}{\pgfqpoint{3.019336in}{1.368140in}}%
\pgfpathcurveto{\pgfqpoint{3.025160in}{1.362316in}}{\pgfqpoint{3.033060in}{1.359044in}}{\pgfqpoint{3.041297in}{1.359044in}}%
\pgfpathclose%
\pgfusepath{stroke,fill}%
\end{pgfscope}%
\begin{pgfscope}%
\pgfpathrectangle{\pgfqpoint{0.457963in}{0.528059in}}{\pgfqpoint{6.200000in}{2.285714in}} %
\pgfusepath{clip}%
\pgfsetbuttcap%
\pgfsetroundjoin%
\definecolor{currentfill}{rgb}{0.500000,0.500000,1.000000}%
\pgfsetfillcolor{currentfill}%
\pgfsetlinewidth{1.003750pt}%
\definecolor{currentstroke}{rgb}{0.500000,0.500000,1.000000}%
\pgfsetstrokecolor{currentstroke}%
\pgfsetdash{}{0pt}%
\pgfpathmoveto{\pgfqpoint{4.384630in}{1.163125in}}%
\pgfpathcurveto{\pgfqpoint{4.392866in}{1.163125in}}{\pgfqpoint{4.400766in}{1.166398in}}{\pgfqpoint{4.406590in}{1.172222in}}%
\pgfpathcurveto{\pgfqpoint{4.412414in}{1.178046in}}{\pgfqpoint{4.415686in}{1.185946in}}{\pgfqpoint{4.415686in}{1.194182in}}%
\pgfpathcurveto{\pgfqpoint{4.415686in}{1.202418in}}{\pgfqpoint{4.412414in}{1.210318in}}{\pgfqpoint{4.406590in}{1.216142in}}%
\pgfpathcurveto{\pgfqpoint{4.400766in}{1.221966in}}{\pgfqpoint{4.392866in}{1.225238in}}{\pgfqpoint{4.384630in}{1.225238in}}%
\pgfpathcurveto{\pgfqpoint{4.376394in}{1.225238in}}{\pgfqpoint{4.368494in}{1.221966in}}{\pgfqpoint{4.362670in}{1.216142in}}%
\pgfpathcurveto{\pgfqpoint{4.356846in}{1.210318in}}{\pgfqpoint{4.353574in}{1.202418in}}{\pgfqpoint{4.353574in}{1.194182in}}%
\pgfpathcurveto{\pgfqpoint{4.353574in}{1.185946in}}{\pgfqpoint{4.356846in}{1.178046in}}{\pgfqpoint{4.362670in}{1.172222in}}%
\pgfpathcurveto{\pgfqpoint{4.368494in}{1.166398in}}{\pgfqpoint{4.376394in}{1.163125in}}{\pgfqpoint{4.384630in}{1.163125in}}%
\pgfpathclose%
\pgfusepath{stroke,fill}%
\end{pgfscope}%
\begin{pgfscope}%
\pgfpathrectangle{\pgfqpoint{0.457963in}{0.528059in}}{\pgfqpoint{6.200000in}{2.285714in}} %
\pgfusepath{clip}%
\pgfsetbuttcap%
\pgfsetroundjoin%
\definecolor{currentfill}{rgb}{0.500000,0.500000,1.000000}%
\pgfsetfillcolor{currentfill}%
\pgfsetlinewidth{1.003750pt}%
\definecolor{currentstroke}{rgb}{0.500000,0.500000,1.000000}%
\pgfsetstrokecolor{currentstroke}%
\pgfsetdash{}{0pt}%
\pgfpathmoveto{\pgfqpoint{5.934630in}{1.071697in}}%
\pgfpathcurveto{\pgfqpoint{5.942866in}{1.071697in}}{\pgfqpoint{5.950766in}{1.074969in}}{\pgfqpoint{5.956590in}{1.080793in}}%
\pgfpathcurveto{\pgfqpoint{5.962414in}{1.086617in}}{\pgfqpoint{5.965686in}{1.094517in}}{\pgfqpoint{5.965686in}{1.102753in}}%
\pgfpathcurveto{\pgfqpoint{5.965686in}{1.110990in}}{\pgfqpoint{5.962414in}{1.118890in}}{\pgfqpoint{5.956590in}{1.124714in}}%
\pgfpathcurveto{\pgfqpoint{5.950766in}{1.130538in}}{\pgfqpoint{5.942866in}{1.133810in}}{\pgfqpoint{5.934630in}{1.133810in}}%
\pgfpathcurveto{\pgfqpoint{5.926394in}{1.133810in}}{\pgfqpoint{5.918494in}{1.130538in}}{\pgfqpoint{5.912670in}{1.124714in}}%
\pgfpathcurveto{\pgfqpoint{5.906846in}{1.118890in}}{\pgfqpoint{5.903574in}{1.110990in}}{\pgfqpoint{5.903574in}{1.102753in}}%
\pgfpathcurveto{\pgfqpoint{5.903574in}{1.094517in}}{\pgfqpoint{5.906846in}{1.086617in}}{\pgfqpoint{5.912670in}{1.080793in}}%
\pgfpathcurveto{\pgfqpoint{5.918494in}{1.074969in}}{\pgfqpoint{5.926394in}{1.071697in}}{\pgfqpoint{5.934630in}{1.071697in}}%
\pgfpathclose%
\pgfusepath{stroke,fill}%
\end{pgfscope}%
\begin{pgfscope}%
\pgfpathrectangle{\pgfqpoint{0.457963in}{0.528059in}}{\pgfqpoint{6.200000in}{2.285714in}} %
\pgfusepath{clip}%
\pgfsetbuttcap%
\pgfsetroundjoin%
\definecolor{currentfill}{rgb}{0.333333,0.333333,1.000000}%
\pgfsetfillcolor{currentfill}%
\pgfsetlinewidth{1.003750pt}%
\definecolor{currentstroke}{rgb}{0.333333,0.333333,1.000000}%
\pgfsetstrokecolor{currentstroke}%
\pgfsetdash{}{0pt}%
\pgfpathmoveto{\pgfqpoint{0.457963in}{1.803125in}}%
\pgfpathcurveto{\pgfqpoint{0.466200in}{1.803125in}}{\pgfqpoint{0.474100in}{1.806398in}}{\pgfqpoint{0.479924in}{1.812222in}}%
\pgfpathcurveto{\pgfqpoint{0.485748in}{1.818046in}}{\pgfqpoint{0.489020in}{1.825946in}}{\pgfqpoint{0.489020in}{1.834182in}}%
\pgfpathcurveto{\pgfqpoint{0.489020in}{1.842418in}}{\pgfqpoint{0.485748in}{1.850318in}}{\pgfqpoint{0.479924in}{1.856142in}}%
\pgfpathcurveto{\pgfqpoint{0.474100in}{1.861966in}}{\pgfqpoint{0.466200in}{1.865238in}}{\pgfqpoint{0.457963in}{1.865238in}}%
\pgfpathcurveto{\pgfqpoint{0.449727in}{1.865238in}}{\pgfqpoint{0.441827in}{1.861966in}}{\pgfqpoint{0.436003in}{1.856142in}}%
\pgfpathcurveto{\pgfqpoint{0.430179in}{1.850318in}}{\pgfqpoint{0.426907in}{1.842418in}}{\pgfqpoint{0.426907in}{1.834182in}}%
\pgfpathcurveto{\pgfqpoint{0.426907in}{1.825946in}}{\pgfqpoint{0.430179in}{1.818046in}}{\pgfqpoint{0.436003in}{1.812222in}}%
\pgfpathcurveto{\pgfqpoint{0.441827in}{1.806398in}}{\pgfqpoint{0.449727in}{1.803125in}}{\pgfqpoint{0.457963in}{1.803125in}}%
\pgfpathclose%
\pgfusepath{stroke,fill}%
\end{pgfscope}%
\begin{pgfscope}%
\pgfpathrectangle{\pgfqpoint{0.457963in}{0.528059in}}{\pgfqpoint{6.200000in}{2.285714in}} %
\pgfusepath{clip}%
\pgfsetbuttcap%
\pgfsetroundjoin%
\definecolor{currentfill}{rgb}{0.333333,0.333333,1.000000}%
\pgfsetfillcolor{currentfill}%
\pgfsetlinewidth{1.003750pt}%
\definecolor{currentstroke}{rgb}{0.333333,0.333333,1.000000}%
\pgfsetstrokecolor{currentstroke}%
\pgfsetdash{}{0pt}%
\pgfpathmoveto{\pgfqpoint{0.457963in}{1.803125in}}%
\pgfpathcurveto{\pgfqpoint{0.466200in}{1.803125in}}{\pgfqpoint{0.474100in}{1.806398in}}{\pgfqpoint{0.479924in}{1.812222in}}%
\pgfpathcurveto{\pgfqpoint{0.485748in}{1.818046in}}{\pgfqpoint{0.489020in}{1.825946in}}{\pgfqpoint{0.489020in}{1.834182in}}%
\pgfpathcurveto{\pgfqpoint{0.489020in}{1.842418in}}{\pgfqpoint{0.485748in}{1.850318in}}{\pgfqpoint{0.479924in}{1.856142in}}%
\pgfpathcurveto{\pgfqpoint{0.474100in}{1.861966in}}{\pgfqpoint{0.466200in}{1.865238in}}{\pgfqpoint{0.457963in}{1.865238in}}%
\pgfpathcurveto{\pgfqpoint{0.449727in}{1.865238in}}{\pgfqpoint{0.441827in}{1.861966in}}{\pgfqpoint{0.436003in}{1.856142in}}%
\pgfpathcurveto{\pgfqpoint{0.430179in}{1.850318in}}{\pgfqpoint{0.426907in}{1.842418in}}{\pgfqpoint{0.426907in}{1.834182in}}%
\pgfpathcurveto{\pgfqpoint{0.426907in}{1.825946in}}{\pgfqpoint{0.430179in}{1.818046in}}{\pgfqpoint{0.436003in}{1.812222in}}%
\pgfpathcurveto{\pgfqpoint{0.441827in}{1.806398in}}{\pgfqpoint{0.449727in}{1.803125in}}{\pgfqpoint{0.457963in}{1.803125in}}%
\pgfpathclose%
\pgfusepath{stroke,fill}%
\end{pgfscope}%
\begin{pgfscope}%
\pgfpathrectangle{\pgfqpoint{0.457963in}{0.528059in}}{\pgfqpoint{6.200000in}{2.285714in}} %
\pgfusepath{clip}%
\pgfsetbuttcap%
\pgfsetroundjoin%
\definecolor{currentfill}{rgb}{0.333333,0.333333,1.000000}%
\pgfsetfillcolor{currentfill}%
\pgfsetlinewidth{1.003750pt}%
\definecolor{currentstroke}{rgb}{0.333333,0.333333,1.000000}%
\pgfsetstrokecolor{currentstroke}%
\pgfsetdash{}{0pt}%
\pgfpathmoveto{\pgfqpoint{0.457963in}{1.803125in}}%
\pgfpathcurveto{\pgfqpoint{0.466200in}{1.803125in}}{\pgfqpoint{0.474100in}{1.806398in}}{\pgfqpoint{0.479924in}{1.812222in}}%
\pgfpathcurveto{\pgfqpoint{0.485748in}{1.818046in}}{\pgfqpoint{0.489020in}{1.825946in}}{\pgfqpoint{0.489020in}{1.834182in}}%
\pgfpathcurveto{\pgfqpoint{0.489020in}{1.842418in}}{\pgfqpoint{0.485748in}{1.850318in}}{\pgfqpoint{0.479924in}{1.856142in}}%
\pgfpathcurveto{\pgfqpoint{0.474100in}{1.861966in}}{\pgfqpoint{0.466200in}{1.865238in}}{\pgfqpoint{0.457963in}{1.865238in}}%
\pgfpathcurveto{\pgfqpoint{0.449727in}{1.865238in}}{\pgfqpoint{0.441827in}{1.861966in}}{\pgfqpoint{0.436003in}{1.856142in}}%
\pgfpathcurveto{\pgfqpoint{0.430179in}{1.850318in}}{\pgfqpoint{0.426907in}{1.842418in}}{\pgfqpoint{0.426907in}{1.834182in}}%
\pgfpathcurveto{\pgfqpoint{0.426907in}{1.825946in}}{\pgfqpoint{0.430179in}{1.818046in}}{\pgfqpoint{0.436003in}{1.812222in}}%
\pgfpathcurveto{\pgfqpoint{0.441827in}{1.806398in}}{\pgfqpoint{0.449727in}{1.803125in}}{\pgfqpoint{0.457963in}{1.803125in}}%
\pgfpathclose%
\pgfusepath{stroke,fill}%
\end{pgfscope}%
\begin{pgfscope}%
\pgfpathrectangle{\pgfqpoint{0.457963in}{0.528059in}}{\pgfqpoint{6.200000in}{2.285714in}} %
\pgfusepath{clip}%
\pgfsetbuttcap%
\pgfsetroundjoin%
\definecolor{currentfill}{rgb}{0.333333,0.333333,1.000000}%
\pgfsetfillcolor{currentfill}%
\pgfsetlinewidth{1.003750pt}%
\definecolor{currentstroke}{rgb}{0.333333,0.333333,1.000000}%
\pgfsetstrokecolor{currentstroke}%
\pgfsetdash{}{0pt}%
\pgfpathmoveto{\pgfqpoint{0.457963in}{1.803125in}}%
\pgfpathcurveto{\pgfqpoint{0.466200in}{1.803125in}}{\pgfqpoint{0.474100in}{1.806398in}}{\pgfqpoint{0.479924in}{1.812222in}}%
\pgfpathcurveto{\pgfqpoint{0.485748in}{1.818046in}}{\pgfqpoint{0.489020in}{1.825946in}}{\pgfqpoint{0.489020in}{1.834182in}}%
\pgfpathcurveto{\pgfqpoint{0.489020in}{1.842418in}}{\pgfqpoint{0.485748in}{1.850318in}}{\pgfqpoint{0.479924in}{1.856142in}}%
\pgfpathcurveto{\pgfqpoint{0.474100in}{1.861966in}}{\pgfqpoint{0.466200in}{1.865238in}}{\pgfqpoint{0.457963in}{1.865238in}}%
\pgfpathcurveto{\pgfqpoint{0.449727in}{1.865238in}}{\pgfqpoint{0.441827in}{1.861966in}}{\pgfqpoint{0.436003in}{1.856142in}}%
\pgfpathcurveto{\pgfqpoint{0.430179in}{1.850318in}}{\pgfqpoint{0.426907in}{1.842418in}}{\pgfqpoint{0.426907in}{1.834182in}}%
\pgfpathcurveto{\pgfqpoint{0.426907in}{1.825946in}}{\pgfqpoint{0.430179in}{1.818046in}}{\pgfqpoint{0.436003in}{1.812222in}}%
\pgfpathcurveto{\pgfqpoint{0.441827in}{1.806398in}}{\pgfqpoint{0.449727in}{1.803125in}}{\pgfqpoint{0.457963in}{1.803125in}}%
\pgfpathclose%
\pgfusepath{stroke,fill}%
\end{pgfscope}%
\begin{pgfscope}%
\pgfpathrectangle{\pgfqpoint{0.457963in}{0.528059in}}{\pgfqpoint{6.200000in}{2.285714in}} %
\pgfusepath{clip}%
\pgfsetbuttcap%
\pgfsetroundjoin%
\definecolor{currentfill}{rgb}{0.333333,0.333333,1.000000}%
\pgfsetfillcolor{currentfill}%
\pgfsetlinewidth{1.003750pt}%
\definecolor{currentstroke}{rgb}{0.333333,0.333333,1.000000}%
\pgfsetstrokecolor{currentstroke}%
\pgfsetdash{}{0pt}%
\pgfpathmoveto{\pgfqpoint{0.468297in}{1.803125in}}%
\pgfpathcurveto{\pgfqpoint{0.476533in}{1.803125in}}{\pgfqpoint{0.484433in}{1.806398in}}{\pgfqpoint{0.490257in}{1.812222in}}%
\pgfpathcurveto{\pgfqpoint{0.496081in}{1.818046in}}{\pgfqpoint{0.499353in}{1.825946in}}{\pgfqpoint{0.499353in}{1.834182in}}%
\pgfpathcurveto{\pgfqpoint{0.499353in}{1.842418in}}{\pgfqpoint{0.496081in}{1.850318in}}{\pgfqpoint{0.490257in}{1.856142in}}%
\pgfpathcurveto{\pgfqpoint{0.484433in}{1.861966in}}{\pgfqpoint{0.476533in}{1.865238in}}{\pgfqpoint{0.468297in}{1.865238in}}%
\pgfpathcurveto{\pgfqpoint{0.460060in}{1.865238in}}{\pgfqpoint{0.452160in}{1.861966in}}{\pgfqpoint{0.446336in}{1.856142in}}%
\pgfpathcurveto{\pgfqpoint{0.440512in}{1.850318in}}{\pgfqpoint{0.437240in}{1.842418in}}{\pgfqpoint{0.437240in}{1.834182in}}%
\pgfpathcurveto{\pgfqpoint{0.437240in}{1.825946in}}{\pgfqpoint{0.440512in}{1.818046in}}{\pgfqpoint{0.446336in}{1.812222in}}%
\pgfpathcurveto{\pgfqpoint{0.452160in}{1.806398in}}{\pgfqpoint{0.460060in}{1.803125in}}{\pgfqpoint{0.468297in}{1.803125in}}%
\pgfpathclose%
\pgfusepath{stroke,fill}%
\end{pgfscope}%
\begin{pgfscope}%
\pgfpathrectangle{\pgfqpoint{0.457963in}{0.528059in}}{\pgfqpoint{6.200000in}{2.285714in}} %
\pgfusepath{clip}%
\pgfsetbuttcap%
\pgfsetroundjoin%
\definecolor{currentfill}{rgb}{0.333333,0.333333,1.000000}%
\pgfsetfillcolor{currentfill}%
\pgfsetlinewidth{1.003750pt}%
\definecolor{currentstroke}{rgb}{0.333333,0.333333,1.000000}%
\pgfsetstrokecolor{currentstroke}%
\pgfsetdash{}{0pt}%
\pgfpathmoveto{\pgfqpoint{0.488963in}{1.803125in}}%
\pgfpathcurveto{\pgfqpoint{0.497200in}{1.803125in}}{\pgfqpoint{0.505100in}{1.806398in}}{\pgfqpoint{0.510924in}{1.812222in}}%
\pgfpathcurveto{\pgfqpoint{0.516748in}{1.818046in}}{\pgfqpoint{0.520020in}{1.825946in}}{\pgfqpoint{0.520020in}{1.834182in}}%
\pgfpathcurveto{\pgfqpoint{0.520020in}{1.842418in}}{\pgfqpoint{0.516748in}{1.850318in}}{\pgfqpoint{0.510924in}{1.856142in}}%
\pgfpathcurveto{\pgfqpoint{0.505100in}{1.861966in}}{\pgfqpoint{0.497200in}{1.865238in}}{\pgfqpoint{0.488963in}{1.865238in}}%
\pgfpathcurveto{\pgfqpoint{0.480727in}{1.865238in}}{\pgfqpoint{0.472827in}{1.861966in}}{\pgfqpoint{0.467003in}{1.856142in}}%
\pgfpathcurveto{\pgfqpoint{0.461179in}{1.850318in}}{\pgfqpoint{0.457907in}{1.842418in}}{\pgfqpoint{0.457907in}{1.834182in}}%
\pgfpathcurveto{\pgfqpoint{0.457907in}{1.825946in}}{\pgfqpoint{0.461179in}{1.818046in}}{\pgfqpoint{0.467003in}{1.812222in}}%
\pgfpathcurveto{\pgfqpoint{0.472827in}{1.806398in}}{\pgfqpoint{0.480727in}{1.803125in}}{\pgfqpoint{0.488963in}{1.803125in}}%
\pgfpathclose%
\pgfusepath{stroke,fill}%
\end{pgfscope}%
\begin{pgfscope}%
\pgfpathrectangle{\pgfqpoint{0.457963in}{0.528059in}}{\pgfqpoint{6.200000in}{2.285714in}} %
\pgfusepath{clip}%
\pgfsetbuttcap%
\pgfsetroundjoin%
\definecolor{currentfill}{rgb}{0.333333,0.333333,1.000000}%
\pgfsetfillcolor{currentfill}%
\pgfsetlinewidth{1.003750pt}%
\definecolor{currentstroke}{rgb}{0.333333,0.333333,1.000000}%
\pgfsetstrokecolor{currentstroke}%
\pgfsetdash{}{0pt}%
\pgfpathmoveto{\pgfqpoint{0.550963in}{1.790064in}}%
\pgfpathcurveto{\pgfqpoint{0.559200in}{1.790064in}}{\pgfqpoint{0.567100in}{1.793336in}}{\pgfqpoint{0.572924in}{1.799160in}}%
\pgfpathcurveto{\pgfqpoint{0.578748in}{1.804984in}}{\pgfqpoint{0.582020in}{1.812884in}}{\pgfqpoint{0.582020in}{1.821121in}}%
\pgfpathcurveto{\pgfqpoint{0.582020in}{1.829357in}}{\pgfqpoint{0.578748in}{1.837257in}}{\pgfqpoint{0.572924in}{1.843081in}}%
\pgfpathcurveto{\pgfqpoint{0.567100in}{1.848905in}}{\pgfqpoint{0.559200in}{1.852177in}}{\pgfqpoint{0.550963in}{1.852177in}}%
\pgfpathcurveto{\pgfqpoint{0.542727in}{1.852177in}}{\pgfqpoint{0.534827in}{1.848905in}}{\pgfqpoint{0.529003in}{1.843081in}}%
\pgfpathcurveto{\pgfqpoint{0.523179in}{1.837257in}}{\pgfqpoint{0.519907in}{1.829357in}}{\pgfqpoint{0.519907in}{1.821121in}}%
\pgfpathcurveto{\pgfqpoint{0.519907in}{1.812884in}}{\pgfqpoint{0.523179in}{1.804984in}}{\pgfqpoint{0.529003in}{1.799160in}}%
\pgfpathcurveto{\pgfqpoint{0.534827in}{1.793336in}}{\pgfqpoint{0.542727in}{1.790064in}}{\pgfqpoint{0.550963in}{1.790064in}}%
\pgfpathclose%
\pgfusepath{stroke,fill}%
\end{pgfscope}%
\begin{pgfscope}%
\pgfpathrectangle{\pgfqpoint{0.457963in}{0.528059in}}{\pgfqpoint{6.200000in}{2.285714in}} %
\pgfusepath{clip}%
\pgfsetbuttcap%
\pgfsetroundjoin%
\definecolor{currentfill}{rgb}{0.333333,0.333333,1.000000}%
\pgfsetfillcolor{currentfill}%
\pgfsetlinewidth{1.003750pt}%
\definecolor{currentstroke}{rgb}{0.333333,0.333333,1.000000}%
\pgfsetstrokecolor{currentstroke}%
\pgfsetdash{}{0pt}%
\pgfpathmoveto{\pgfqpoint{0.581963in}{1.803125in}}%
\pgfpathcurveto{\pgfqpoint{0.590200in}{1.803125in}}{\pgfqpoint{0.598100in}{1.806398in}}{\pgfqpoint{0.603924in}{1.812222in}}%
\pgfpathcurveto{\pgfqpoint{0.609748in}{1.818046in}}{\pgfqpoint{0.613020in}{1.825946in}}{\pgfqpoint{0.613020in}{1.834182in}}%
\pgfpathcurveto{\pgfqpoint{0.613020in}{1.842418in}}{\pgfqpoint{0.609748in}{1.850318in}}{\pgfqpoint{0.603924in}{1.856142in}}%
\pgfpathcurveto{\pgfqpoint{0.598100in}{1.861966in}}{\pgfqpoint{0.590200in}{1.865238in}}{\pgfqpoint{0.581963in}{1.865238in}}%
\pgfpathcurveto{\pgfqpoint{0.573727in}{1.865238in}}{\pgfqpoint{0.565827in}{1.861966in}}{\pgfqpoint{0.560003in}{1.856142in}}%
\pgfpathcurveto{\pgfqpoint{0.554179in}{1.850318in}}{\pgfqpoint{0.550907in}{1.842418in}}{\pgfqpoint{0.550907in}{1.834182in}}%
\pgfpathcurveto{\pgfqpoint{0.550907in}{1.825946in}}{\pgfqpoint{0.554179in}{1.818046in}}{\pgfqpoint{0.560003in}{1.812222in}}%
\pgfpathcurveto{\pgfqpoint{0.565827in}{1.806398in}}{\pgfqpoint{0.573727in}{1.803125in}}{\pgfqpoint{0.581963in}{1.803125in}}%
\pgfpathclose%
\pgfusepath{stroke,fill}%
\end{pgfscope}%
\begin{pgfscope}%
\pgfpathrectangle{\pgfqpoint{0.457963in}{0.528059in}}{\pgfqpoint{6.200000in}{2.285714in}} %
\pgfusepath{clip}%
\pgfsetbuttcap%
\pgfsetroundjoin%
\definecolor{currentfill}{rgb}{0.333333,0.333333,1.000000}%
\pgfsetfillcolor{currentfill}%
\pgfsetlinewidth{1.003750pt}%
\definecolor{currentstroke}{rgb}{0.333333,0.333333,1.000000}%
\pgfsetstrokecolor{currentstroke}%
\pgfsetdash{}{0pt}%
\pgfpathmoveto{\pgfqpoint{0.685297in}{1.803125in}}%
\pgfpathcurveto{\pgfqpoint{0.693533in}{1.803125in}}{\pgfqpoint{0.701433in}{1.806398in}}{\pgfqpoint{0.707257in}{1.812222in}}%
\pgfpathcurveto{\pgfqpoint{0.713081in}{1.818046in}}{\pgfqpoint{0.716353in}{1.825946in}}{\pgfqpoint{0.716353in}{1.834182in}}%
\pgfpathcurveto{\pgfqpoint{0.716353in}{1.842418in}}{\pgfqpoint{0.713081in}{1.850318in}}{\pgfqpoint{0.707257in}{1.856142in}}%
\pgfpathcurveto{\pgfqpoint{0.701433in}{1.861966in}}{\pgfqpoint{0.693533in}{1.865238in}}{\pgfqpoint{0.685297in}{1.865238in}}%
\pgfpathcurveto{\pgfqpoint{0.677060in}{1.865238in}}{\pgfqpoint{0.669160in}{1.861966in}}{\pgfqpoint{0.663336in}{1.856142in}}%
\pgfpathcurveto{\pgfqpoint{0.657512in}{1.850318in}}{\pgfqpoint{0.654240in}{1.842418in}}{\pgfqpoint{0.654240in}{1.834182in}}%
\pgfpathcurveto{\pgfqpoint{0.654240in}{1.825946in}}{\pgfqpoint{0.657512in}{1.818046in}}{\pgfqpoint{0.663336in}{1.812222in}}%
\pgfpathcurveto{\pgfqpoint{0.669160in}{1.806398in}}{\pgfqpoint{0.677060in}{1.803125in}}{\pgfqpoint{0.685297in}{1.803125in}}%
\pgfpathclose%
\pgfusepath{stroke,fill}%
\end{pgfscope}%
\begin{pgfscope}%
\pgfpathrectangle{\pgfqpoint{0.457963in}{0.528059in}}{\pgfqpoint{6.200000in}{2.285714in}} %
\pgfusepath{clip}%
\pgfsetbuttcap%
\pgfsetroundjoin%
\definecolor{currentfill}{rgb}{0.333333,0.333333,1.000000}%
\pgfsetfillcolor{currentfill}%
\pgfsetlinewidth{1.003750pt}%
\definecolor{currentstroke}{rgb}{0.333333,0.333333,1.000000}%
\pgfsetstrokecolor{currentstroke}%
\pgfsetdash{}{0pt}%
\pgfpathmoveto{\pgfqpoint{0.747297in}{1.777003in}}%
\pgfpathcurveto{\pgfqpoint{0.755533in}{1.777003in}}{\pgfqpoint{0.763433in}{1.780275in}}{\pgfqpoint{0.769257in}{1.786099in}}%
\pgfpathcurveto{\pgfqpoint{0.775081in}{1.791923in}}{\pgfqpoint{0.778353in}{1.799823in}}{\pgfqpoint{0.778353in}{1.808059in}}%
\pgfpathcurveto{\pgfqpoint{0.778353in}{1.816296in}}{\pgfqpoint{0.775081in}{1.824196in}}{\pgfqpoint{0.769257in}{1.830020in}}%
\pgfpathcurveto{\pgfqpoint{0.763433in}{1.835844in}}{\pgfqpoint{0.755533in}{1.839116in}}{\pgfqpoint{0.747297in}{1.839116in}}%
\pgfpathcurveto{\pgfqpoint{0.739060in}{1.839116in}}{\pgfqpoint{0.731160in}{1.835844in}}{\pgfqpoint{0.725336in}{1.830020in}}%
\pgfpathcurveto{\pgfqpoint{0.719512in}{1.824196in}}{\pgfqpoint{0.716240in}{1.816296in}}{\pgfqpoint{0.716240in}{1.808059in}}%
\pgfpathcurveto{\pgfqpoint{0.716240in}{1.799823in}}{\pgfqpoint{0.719512in}{1.791923in}}{\pgfqpoint{0.725336in}{1.786099in}}%
\pgfpathcurveto{\pgfqpoint{0.731160in}{1.780275in}}{\pgfqpoint{0.739060in}{1.777003in}}{\pgfqpoint{0.747297in}{1.777003in}}%
\pgfpathclose%
\pgfusepath{stroke,fill}%
\end{pgfscope}%
\begin{pgfscope}%
\pgfpathrectangle{\pgfqpoint{0.457963in}{0.528059in}}{\pgfqpoint{6.200000in}{2.285714in}} %
\pgfusepath{clip}%
\pgfsetbuttcap%
\pgfsetroundjoin%
\definecolor{currentfill}{rgb}{0.333333,0.333333,1.000000}%
\pgfsetfillcolor{currentfill}%
\pgfsetlinewidth{1.003750pt}%
\definecolor{currentstroke}{rgb}{0.333333,0.333333,1.000000}%
\pgfsetstrokecolor{currentstroke}%
\pgfsetdash{}{0pt}%
\pgfpathmoveto{\pgfqpoint{1.077963in}{1.803125in}}%
\pgfpathcurveto{\pgfqpoint{1.086200in}{1.803125in}}{\pgfqpoint{1.094100in}{1.806398in}}{\pgfqpoint{1.099924in}{1.812222in}}%
\pgfpathcurveto{\pgfqpoint{1.105748in}{1.818046in}}{\pgfqpoint{1.109020in}{1.825946in}}{\pgfqpoint{1.109020in}{1.834182in}}%
\pgfpathcurveto{\pgfqpoint{1.109020in}{1.842418in}}{\pgfqpoint{1.105748in}{1.850318in}}{\pgfqpoint{1.099924in}{1.856142in}}%
\pgfpathcurveto{\pgfqpoint{1.094100in}{1.861966in}}{\pgfqpoint{1.086200in}{1.865238in}}{\pgfqpoint{1.077963in}{1.865238in}}%
\pgfpathcurveto{\pgfqpoint{1.069727in}{1.865238in}}{\pgfqpoint{1.061827in}{1.861966in}}{\pgfqpoint{1.056003in}{1.856142in}}%
\pgfpathcurveto{\pgfqpoint{1.050179in}{1.850318in}}{\pgfqpoint{1.046907in}{1.842418in}}{\pgfqpoint{1.046907in}{1.834182in}}%
\pgfpathcurveto{\pgfqpoint{1.046907in}{1.825946in}}{\pgfqpoint{1.050179in}{1.818046in}}{\pgfqpoint{1.056003in}{1.812222in}}%
\pgfpathcurveto{\pgfqpoint{1.061827in}{1.806398in}}{\pgfqpoint{1.069727in}{1.803125in}}{\pgfqpoint{1.077963in}{1.803125in}}%
\pgfpathclose%
\pgfusepath{stroke,fill}%
\end{pgfscope}%
\begin{pgfscope}%
\pgfpathrectangle{\pgfqpoint{0.457963in}{0.528059in}}{\pgfqpoint{6.200000in}{2.285714in}} %
\pgfusepath{clip}%
\pgfsetbuttcap%
\pgfsetroundjoin%
\definecolor{currentfill}{rgb}{0.333333,0.333333,1.000000}%
\pgfsetfillcolor{currentfill}%
\pgfsetlinewidth{1.003750pt}%
\definecolor{currentstroke}{rgb}{0.333333,0.333333,1.000000}%
\pgfsetstrokecolor{currentstroke}%
\pgfsetdash{}{0pt}%
\pgfpathmoveto{\pgfqpoint{1.315630in}{1.698636in}}%
\pgfpathcurveto{\pgfqpoint{1.323866in}{1.698636in}}{\pgfqpoint{1.331766in}{1.701908in}}{\pgfqpoint{1.337590in}{1.707732in}}%
\pgfpathcurveto{\pgfqpoint{1.343414in}{1.713556in}}{\pgfqpoint{1.346686in}{1.721456in}}{\pgfqpoint{1.346686in}{1.729692in}}%
\pgfpathcurveto{\pgfqpoint{1.346686in}{1.737928in}}{\pgfqpoint{1.343414in}{1.745828in}}{\pgfqpoint{1.337590in}{1.751652in}}%
\pgfpathcurveto{\pgfqpoint{1.331766in}{1.757476in}}{\pgfqpoint{1.323866in}{1.760749in}}{\pgfqpoint{1.315630in}{1.760749in}}%
\pgfpathcurveto{\pgfqpoint{1.307394in}{1.760749in}}{\pgfqpoint{1.299494in}{1.757476in}}{\pgfqpoint{1.293670in}{1.751652in}}%
\pgfpathcurveto{\pgfqpoint{1.287846in}{1.745828in}}{\pgfqpoint{1.284574in}{1.737928in}}{\pgfqpoint{1.284574in}{1.729692in}}%
\pgfpathcurveto{\pgfqpoint{1.284574in}{1.721456in}}{\pgfqpoint{1.287846in}{1.713556in}}{\pgfqpoint{1.293670in}{1.707732in}}%
\pgfpathcurveto{\pgfqpoint{1.299494in}{1.701908in}}{\pgfqpoint{1.307394in}{1.698636in}}{\pgfqpoint{1.315630in}{1.698636in}}%
\pgfpathclose%
\pgfusepath{stroke,fill}%
\end{pgfscope}%
\begin{pgfscope}%
\pgfpathrectangle{\pgfqpoint{0.457963in}{0.528059in}}{\pgfqpoint{6.200000in}{2.285714in}} %
\pgfusepath{clip}%
\pgfsetbuttcap%
\pgfsetroundjoin%
\definecolor{currentfill}{rgb}{0.333333,0.333333,1.000000}%
\pgfsetfillcolor{currentfill}%
\pgfsetlinewidth{1.003750pt}%
\definecolor{currentstroke}{rgb}{0.333333,0.333333,1.000000}%
\pgfsetstrokecolor{currentstroke}%
\pgfsetdash{}{0pt}%
\pgfpathmoveto{\pgfqpoint{1.532630in}{1.803125in}}%
\pgfpathcurveto{\pgfqpoint{1.540866in}{1.803125in}}{\pgfqpoint{1.548766in}{1.806398in}}{\pgfqpoint{1.554590in}{1.812222in}}%
\pgfpathcurveto{\pgfqpoint{1.560414in}{1.818046in}}{\pgfqpoint{1.563686in}{1.825946in}}{\pgfqpoint{1.563686in}{1.834182in}}%
\pgfpathcurveto{\pgfqpoint{1.563686in}{1.842418in}}{\pgfqpoint{1.560414in}{1.850318in}}{\pgfqpoint{1.554590in}{1.856142in}}%
\pgfpathcurveto{\pgfqpoint{1.548766in}{1.861966in}}{\pgfqpoint{1.540866in}{1.865238in}}{\pgfqpoint{1.532630in}{1.865238in}}%
\pgfpathcurveto{\pgfqpoint{1.524394in}{1.865238in}}{\pgfqpoint{1.516494in}{1.861966in}}{\pgfqpoint{1.510670in}{1.856142in}}%
\pgfpathcurveto{\pgfqpoint{1.504846in}{1.850318in}}{\pgfqpoint{1.501574in}{1.842418in}}{\pgfqpoint{1.501574in}{1.834182in}}%
\pgfpathcurveto{\pgfqpoint{1.501574in}{1.825946in}}{\pgfqpoint{1.504846in}{1.818046in}}{\pgfqpoint{1.510670in}{1.812222in}}%
\pgfpathcurveto{\pgfqpoint{1.516494in}{1.806398in}}{\pgfqpoint{1.524394in}{1.803125in}}{\pgfqpoint{1.532630in}{1.803125in}}%
\pgfpathclose%
\pgfusepath{stroke,fill}%
\end{pgfscope}%
\begin{pgfscope}%
\pgfpathrectangle{\pgfqpoint{0.457963in}{0.528059in}}{\pgfqpoint{6.200000in}{2.285714in}} %
\pgfusepath{clip}%
\pgfsetbuttcap%
\pgfsetroundjoin%
\definecolor{currentfill}{rgb}{0.333333,0.333333,1.000000}%
\pgfsetfillcolor{currentfill}%
\pgfsetlinewidth{1.003750pt}%
\definecolor{currentstroke}{rgb}{0.333333,0.333333,1.000000}%
\pgfsetstrokecolor{currentstroke}%
\pgfsetdash{}{0pt}%
\pgfpathmoveto{\pgfqpoint{1.832297in}{1.633329in}}%
\pgfpathcurveto{\pgfqpoint{1.840533in}{1.633329in}}{\pgfqpoint{1.848433in}{1.636602in}}{\pgfqpoint{1.854257in}{1.642426in}}%
\pgfpathcurveto{\pgfqpoint{1.860081in}{1.648250in}}{\pgfqpoint{1.863353in}{1.656150in}}{\pgfqpoint{1.863353in}{1.664386in}}%
\pgfpathcurveto{\pgfqpoint{1.863353in}{1.672622in}}{\pgfqpoint{1.860081in}{1.680522in}}{\pgfqpoint{1.854257in}{1.686346in}}%
\pgfpathcurveto{\pgfqpoint{1.848433in}{1.692170in}}{\pgfqpoint{1.840533in}{1.695442in}}{\pgfqpoint{1.832297in}{1.695442in}}%
\pgfpathcurveto{\pgfqpoint{1.824060in}{1.695442in}}{\pgfqpoint{1.816160in}{1.692170in}}{\pgfqpoint{1.810336in}{1.686346in}}%
\pgfpathcurveto{\pgfqpoint{1.804512in}{1.680522in}}{\pgfqpoint{1.801240in}{1.672622in}}{\pgfqpoint{1.801240in}{1.664386in}}%
\pgfpathcurveto{\pgfqpoint{1.801240in}{1.656150in}}{\pgfqpoint{1.804512in}{1.648250in}}{\pgfqpoint{1.810336in}{1.642426in}}%
\pgfpathcurveto{\pgfqpoint{1.816160in}{1.636602in}}{\pgfqpoint{1.824060in}{1.633329in}}{\pgfqpoint{1.832297in}{1.633329in}}%
\pgfpathclose%
\pgfusepath{stroke,fill}%
\end{pgfscope}%
\begin{pgfscope}%
\pgfpathrectangle{\pgfqpoint{0.457963in}{0.528059in}}{\pgfqpoint{6.200000in}{2.285714in}} %
\pgfusepath{clip}%
\pgfsetbuttcap%
\pgfsetroundjoin%
\definecolor{currentfill}{rgb}{0.333333,0.333333,1.000000}%
\pgfsetfillcolor{currentfill}%
\pgfsetlinewidth{1.003750pt}%
\definecolor{currentstroke}{rgb}{0.333333,0.333333,1.000000}%
\pgfsetstrokecolor{currentstroke}%
\pgfsetdash{}{0pt}%
\pgfpathmoveto{\pgfqpoint{2.565963in}{1.777003in}}%
\pgfpathcurveto{\pgfqpoint{2.574200in}{1.777003in}}{\pgfqpoint{2.582100in}{1.780275in}}{\pgfqpoint{2.587924in}{1.786099in}}%
\pgfpathcurveto{\pgfqpoint{2.593748in}{1.791923in}}{\pgfqpoint{2.597020in}{1.799823in}}{\pgfqpoint{2.597020in}{1.808059in}}%
\pgfpathcurveto{\pgfqpoint{2.597020in}{1.816296in}}{\pgfqpoint{2.593748in}{1.824196in}}{\pgfqpoint{2.587924in}{1.830020in}}%
\pgfpathcurveto{\pgfqpoint{2.582100in}{1.835844in}}{\pgfqpoint{2.574200in}{1.839116in}}{\pgfqpoint{2.565963in}{1.839116in}}%
\pgfpathcurveto{\pgfqpoint{2.557727in}{1.839116in}}{\pgfqpoint{2.549827in}{1.835844in}}{\pgfqpoint{2.544003in}{1.830020in}}%
\pgfpathcurveto{\pgfqpoint{2.538179in}{1.824196in}}{\pgfqpoint{2.534907in}{1.816296in}}{\pgfqpoint{2.534907in}{1.808059in}}%
\pgfpathcurveto{\pgfqpoint{2.534907in}{1.799823in}}{\pgfqpoint{2.538179in}{1.791923in}}{\pgfqpoint{2.544003in}{1.786099in}}%
\pgfpathcurveto{\pgfqpoint{2.549827in}{1.780275in}}{\pgfqpoint{2.557727in}{1.777003in}}{\pgfqpoint{2.565963in}{1.777003in}}%
\pgfpathclose%
\pgfusepath{stroke,fill}%
\end{pgfscope}%
\begin{pgfscope}%
\pgfpathrectangle{\pgfqpoint{0.457963in}{0.528059in}}{\pgfqpoint{6.200000in}{2.285714in}} %
\pgfusepath{clip}%
\pgfsetbuttcap%
\pgfsetroundjoin%
\definecolor{currentfill}{rgb}{0.333333,0.333333,1.000000}%
\pgfsetfillcolor{currentfill}%
\pgfsetlinewidth{1.003750pt}%
\definecolor{currentstroke}{rgb}{0.333333,0.333333,1.000000}%
\pgfsetstrokecolor{currentstroke}%
\pgfsetdash{}{0pt}%
\pgfpathmoveto{\pgfqpoint{3.320297in}{1.372105in}}%
\pgfpathcurveto{\pgfqpoint{3.328533in}{1.372105in}}{\pgfqpoint{3.336433in}{1.375377in}}{\pgfqpoint{3.342257in}{1.381201in}}%
\pgfpathcurveto{\pgfqpoint{3.348081in}{1.387025in}}{\pgfqpoint{3.351353in}{1.394925in}}{\pgfqpoint{3.351353in}{1.403161in}}%
\pgfpathcurveto{\pgfqpoint{3.351353in}{1.411398in}}{\pgfqpoint{3.348081in}{1.419298in}}{\pgfqpoint{3.342257in}{1.425122in}}%
\pgfpathcurveto{\pgfqpoint{3.336433in}{1.430946in}}{\pgfqpoint{3.328533in}{1.434218in}}{\pgfqpoint{3.320297in}{1.434218in}}%
\pgfpathcurveto{\pgfqpoint{3.312060in}{1.434218in}}{\pgfqpoint{3.304160in}{1.430946in}}{\pgfqpoint{3.298336in}{1.425122in}}%
\pgfpathcurveto{\pgfqpoint{3.292512in}{1.419298in}}{\pgfqpoint{3.289240in}{1.411398in}}{\pgfqpoint{3.289240in}{1.403161in}}%
\pgfpathcurveto{\pgfqpoint{3.289240in}{1.394925in}}{\pgfqpoint{3.292512in}{1.387025in}}{\pgfqpoint{3.298336in}{1.381201in}}%
\pgfpathcurveto{\pgfqpoint{3.304160in}{1.375377in}}{\pgfqpoint{3.312060in}{1.372105in}}{\pgfqpoint{3.320297in}{1.372105in}}%
\pgfpathclose%
\pgfusepath{stroke,fill}%
\end{pgfscope}%
\begin{pgfscope}%
\pgfpathrectangle{\pgfqpoint{0.457963in}{0.528059in}}{\pgfqpoint{6.200000in}{2.285714in}} %
\pgfusepath{clip}%
\pgfsetbuttcap%
\pgfsetroundjoin%
\definecolor{currentfill}{rgb}{0.333333,0.333333,1.000000}%
\pgfsetfillcolor{currentfill}%
\pgfsetlinewidth{1.003750pt}%
\definecolor{currentstroke}{rgb}{0.333333,0.333333,1.000000}%
\pgfsetstrokecolor{currentstroke}%
\pgfsetdash{}{0pt}%
\pgfpathmoveto{\pgfqpoint{4.281297in}{1.568023in}}%
\pgfpathcurveto{\pgfqpoint{4.289533in}{1.568023in}}{\pgfqpoint{4.297433in}{1.571296in}}{\pgfqpoint{4.303257in}{1.577120in}}%
\pgfpathcurveto{\pgfqpoint{4.309081in}{1.582944in}}{\pgfqpoint{4.312353in}{1.590844in}}{\pgfqpoint{4.312353in}{1.599080in}}%
\pgfpathcurveto{\pgfqpoint{4.312353in}{1.607316in}}{\pgfqpoint{4.309081in}{1.615216in}}{\pgfqpoint{4.303257in}{1.621040in}}%
\pgfpathcurveto{\pgfqpoint{4.297433in}{1.626864in}}{\pgfqpoint{4.289533in}{1.630136in}}{\pgfqpoint{4.281297in}{1.630136in}}%
\pgfpathcurveto{\pgfqpoint{4.273060in}{1.630136in}}{\pgfqpoint{4.265160in}{1.626864in}}{\pgfqpoint{4.259336in}{1.621040in}}%
\pgfpathcurveto{\pgfqpoint{4.253512in}{1.615216in}}{\pgfqpoint{4.250240in}{1.607316in}}{\pgfqpoint{4.250240in}{1.599080in}}%
\pgfpathcurveto{\pgfqpoint{4.250240in}{1.590844in}}{\pgfqpoint{4.253512in}{1.582944in}}{\pgfqpoint{4.259336in}{1.577120in}}%
\pgfpathcurveto{\pgfqpoint{4.265160in}{1.571296in}}{\pgfqpoint{4.273060in}{1.568023in}}{\pgfqpoint{4.281297in}{1.568023in}}%
\pgfpathclose%
\pgfusepath{stroke,fill}%
\end{pgfscope}%
\begin{pgfscope}%
\pgfpathrectangle{\pgfqpoint{0.457963in}{0.528059in}}{\pgfqpoint{6.200000in}{2.285714in}} %
\pgfusepath{clip}%
\pgfsetbuttcap%
\pgfsetroundjoin%
\definecolor{currentfill}{rgb}{0.333333,0.333333,1.000000}%
\pgfsetfillcolor{currentfill}%
\pgfsetlinewidth{1.003750pt}%
\definecolor{currentstroke}{rgb}{0.333333,0.333333,1.000000}%
\pgfsetstrokecolor{currentstroke}%
\pgfsetdash{}{0pt}%
\pgfpathmoveto{\pgfqpoint{4.343297in}{1.737819in}}%
\pgfpathcurveto{\pgfqpoint{4.351533in}{1.737819in}}{\pgfqpoint{4.359433in}{1.741092in}}{\pgfqpoint{4.365257in}{1.746916in}}%
\pgfpathcurveto{\pgfqpoint{4.371081in}{1.752739in}}{\pgfqpoint{4.374353in}{1.760639in}}{\pgfqpoint{4.374353in}{1.768876in}}%
\pgfpathcurveto{\pgfqpoint{4.374353in}{1.777112in}}{\pgfqpoint{4.371081in}{1.785012in}}{\pgfqpoint{4.365257in}{1.790836in}}%
\pgfpathcurveto{\pgfqpoint{4.359433in}{1.796660in}}{\pgfqpoint{4.351533in}{1.799932in}}{\pgfqpoint{4.343297in}{1.799932in}}%
\pgfpathcurveto{\pgfqpoint{4.335060in}{1.799932in}}{\pgfqpoint{4.327160in}{1.796660in}}{\pgfqpoint{4.321336in}{1.790836in}}%
\pgfpathcurveto{\pgfqpoint{4.315512in}{1.785012in}}{\pgfqpoint{4.312240in}{1.777112in}}{\pgfqpoint{4.312240in}{1.768876in}}%
\pgfpathcurveto{\pgfqpoint{4.312240in}{1.760639in}}{\pgfqpoint{4.315512in}{1.752739in}}{\pgfqpoint{4.321336in}{1.746916in}}%
\pgfpathcurveto{\pgfqpoint{4.327160in}{1.741092in}}{\pgfqpoint{4.335060in}{1.737819in}}{\pgfqpoint{4.343297in}{1.737819in}}%
\pgfpathclose%
\pgfusepath{stroke,fill}%
\end{pgfscope}%
\begin{pgfscope}%
\pgfpathrectangle{\pgfqpoint{0.457963in}{0.528059in}}{\pgfqpoint{6.200000in}{2.285714in}} %
\pgfusepath{clip}%
\pgfsetbuttcap%
\pgfsetroundjoin%
\definecolor{currentfill}{rgb}{0.333333,0.333333,1.000000}%
\pgfsetfillcolor{currentfill}%
\pgfsetlinewidth{1.003750pt}%
\definecolor{currentstroke}{rgb}{0.333333,0.333333,1.000000}%
\pgfsetstrokecolor{currentstroke}%
\pgfsetdash{}{0pt}%
\pgfpathmoveto{\pgfqpoint{4.890963in}{1.372105in}}%
\pgfpathcurveto{\pgfqpoint{4.899200in}{1.372105in}}{\pgfqpoint{4.907100in}{1.375377in}}{\pgfqpoint{4.912924in}{1.381201in}}%
\pgfpathcurveto{\pgfqpoint{4.918748in}{1.387025in}}{\pgfqpoint{4.922020in}{1.394925in}}{\pgfqpoint{4.922020in}{1.403161in}}%
\pgfpathcurveto{\pgfqpoint{4.922020in}{1.411398in}}{\pgfqpoint{4.918748in}{1.419298in}}{\pgfqpoint{4.912924in}{1.425122in}}%
\pgfpathcurveto{\pgfqpoint{4.907100in}{1.430946in}}{\pgfqpoint{4.899200in}{1.434218in}}{\pgfqpoint{4.890963in}{1.434218in}}%
\pgfpathcurveto{\pgfqpoint{4.882727in}{1.434218in}}{\pgfqpoint{4.874827in}{1.430946in}}{\pgfqpoint{4.869003in}{1.425122in}}%
\pgfpathcurveto{\pgfqpoint{4.863179in}{1.419298in}}{\pgfqpoint{4.859907in}{1.411398in}}{\pgfqpoint{4.859907in}{1.403161in}}%
\pgfpathcurveto{\pgfqpoint{4.859907in}{1.394925in}}{\pgfqpoint{4.863179in}{1.387025in}}{\pgfqpoint{4.869003in}{1.381201in}}%
\pgfpathcurveto{\pgfqpoint{4.874827in}{1.375377in}}{\pgfqpoint{4.882727in}{1.372105in}}{\pgfqpoint{4.890963in}{1.372105in}}%
\pgfpathclose%
\pgfusepath{stroke,fill}%
\end{pgfscope}%
\begin{pgfscope}%
\pgfpathrectangle{\pgfqpoint{0.457963in}{0.528059in}}{\pgfqpoint{6.200000in}{2.285714in}} %
\pgfusepath{clip}%
\pgfsetbuttcap%
\pgfsetroundjoin%
\definecolor{currentfill}{rgb}{0.333333,0.333333,1.000000}%
\pgfsetfillcolor{currentfill}%
\pgfsetlinewidth{1.003750pt}%
\definecolor{currentstroke}{rgb}{0.333333,0.333333,1.000000}%
\pgfsetstrokecolor{currentstroke}%
\pgfsetdash{}{0pt}%
\pgfpathmoveto{\pgfqpoint{6.234297in}{1.110880in}}%
\pgfpathcurveto{\pgfqpoint{6.242533in}{1.110880in}}{\pgfqpoint{6.250433in}{1.114153in}}{\pgfqpoint{6.256257in}{1.119977in}}%
\pgfpathcurveto{\pgfqpoint{6.262081in}{1.125801in}}{\pgfqpoint{6.265353in}{1.133701in}}{\pgfqpoint{6.265353in}{1.141937in}}%
\pgfpathcurveto{\pgfqpoint{6.265353in}{1.150173in}}{\pgfqpoint{6.262081in}{1.158073in}}{\pgfqpoint{6.256257in}{1.163897in}}%
\pgfpathcurveto{\pgfqpoint{6.250433in}{1.169721in}}{\pgfqpoint{6.242533in}{1.172993in}}{\pgfqpoint{6.234297in}{1.172993in}}%
\pgfpathcurveto{\pgfqpoint{6.226060in}{1.172993in}}{\pgfqpoint{6.218160in}{1.169721in}}{\pgfqpoint{6.212336in}{1.163897in}}%
\pgfpathcurveto{\pgfqpoint{6.206512in}{1.158073in}}{\pgfqpoint{6.203240in}{1.150173in}}{\pgfqpoint{6.203240in}{1.141937in}}%
\pgfpathcurveto{\pgfqpoint{6.203240in}{1.133701in}}{\pgfqpoint{6.206512in}{1.125801in}}{\pgfqpoint{6.212336in}{1.119977in}}%
\pgfpathcurveto{\pgfqpoint{6.218160in}{1.114153in}}{\pgfqpoint{6.226060in}{1.110880in}}{\pgfqpoint{6.234297in}{1.110880in}}%
\pgfpathclose%
\pgfusepath{stroke,fill}%
\end{pgfscope}%
\begin{pgfscope}%
\pgfpathrectangle{\pgfqpoint{0.457963in}{0.528059in}}{\pgfqpoint{6.200000in}{2.285714in}} %
\pgfusepath{clip}%
\pgfsetbuttcap%
\pgfsetroundjoin%
\definecolor{currentfill}{rgb}{0.166667,0.166667,1.000000}%
\pgfsetfillcolor{currentfill}%
\pgfsetlinewidth{1.003750pt}%
\definecolor{currentstroke}{rgb}{0.166667,0.166667,1.000000}%
\pgfsetstrokecolor{currentstroke}%
\pgfsetdash{}{0pt}%
\pgfpathmoveto{\pgfqpoint{0.457963in}{2.129656in}}%
\pgfpathcurveto{\pgfqpoint{0.466200in}{2.129656in}}{\pgfqpoint{0.474100in}{2.132928in}}{\pgfqpoint{0.479924in}{2.138752in}}%
\pgfpathcurveto{\pgfqpoint{0.485748in}{2.144576in}}{\pgfqpoint{0.489020in}{2.152476in}}{\pgfqpoint{0.489020in}{2.160713in}}%
\pgfpathcurveto{\pgfqpoint{0.489020in}{2.168949in}}{\pgfqpoint{0.485748in}{2.176849in}}{\pgfqpoint{0.479924in}{2.182673in}}%
\pgfpathcurveto{\pgfqpoint{0.474100in}{2.188497in}}{\pgfqpoint{0.466200in}{2.191769in}}{\pgfqpoint{0.457963in}{2.191769in}}%
\pgfpathcurveto{\pgfqpoint{0.449727in}{2.191769in}}{\pgfqpoint{0.441827in}{2.188497in}}{\pgfqpoint{0.436003in}{2.182673in}}%
\pgfpathcurveto{\pgfqpoint{0.430179in}{2.176849in}}{\pgfqpoint{0.426907in}{2.168949in}}{\pgfqpoint{0.426907in}{2.160713in}}%
\pgfpathcurveto{\pgfqpoint{0.426907in}{2.152476in}}{\pgfqpoint{0.430179in}{2.144576in}}{\pgfqpoint{0.436003in}{2.138752in}}%
\pgfpathcurveto{\pgfqpoint{0.441827in}{2.132928in}}{\pgfqpoint{0.449727in}{2.129656in}}{\pgfqpoint{0.457963in}{2.129656in}}%
\pgfpathclose%
\pgfusepath{stroke,fill}%
\end{pgfscope}%
\begin{pgfscope}%
\pgfpathrectangle{\pgfqpoint{0.457963in}{0.528059in}}{\pgfqpoint{6.200000in}{2.285714in}} %
\pgfusepath{clip}%
\pgfsetbuttcap%
\pgfsetroundjoin%
\definecolor{currentfill}{rgb}{0.166667,0.166667,1.000000}%
\pgfsetfillcolor{currentfill}%
\pgfsetlinewidth{1.003750pt}%
\definecolor{currentstroke}{rgb}{0.166667,0.166667,1.000000}%
\pgfsetstrokecolor{currentstroke}%
\pgfsetdash{}{0pt}%
\pgfpathmoveto{\pgfqpoint{0.457963in}{2.129656in}}%
\pgfpathcurveto{\pgfqpoint{0.466200in}{2.129656in}}{\pgfqpoint{0.474100in}{2.132928in}}{\pgfqpoint{0.479924in}{2.138752in}}%
\pgfpathcurveto{\pgfqpoint{0.485748in}{2.144576in}}{\pgfqpoint{0.489020in}{2.152476in}}{\pgfqpoint{0.489020in}{2.160713in}}%
\pgfpathcurveto{\pgfqpoint{0.489020in}{2.168949in}}{\pgfqpoint{0.485748in}{2.176849in}}{\pgfqpoint{0.479924in}{2.182673in}}%
\pgfpathcurveto{\pgfqpoint{0.474100in}{2.188497in}}{\pgfqpoint{0.466200in}{2.191769in}}{\pgfqpoint{0.457963in}{2.191769in}}%
\pgfpathcurveto{\pgfqpoint{0.449727in}{2.191769in}}{\pgfqpoint{0.441827in}{2.188497in}}{\pgfqpoint{0.436003in}{2.182673in}}%
\pgfpathcurveto{\pgfqpoint{0.430179in}{2.176849in}}{\pgfqpoint{0.426907in}{2.168949in}}{\pgfqpoint{0.426907in}{2.160713in}}%
\pgfpathcurveto{\pgfqpoint{0.426907in}{2.152476in}}{\pgfqpoint{0.430179in}{2.144576in}}{\pgfqpoint{0.436003in}{2.138752in}}%
\pgfpathcurveto{\pgfqpoint{0.441827in}{2.132928in}}{\pgfqpoint{0.449727in}{2.129656in}}{\pgfqpoint{0.457963in}{2.129656in}}%
\pgfpathclose%
\pgfusepath{stroke,fill}%
\end{pgfscope}%
\begin{pgfscope}%
\pgfpathrectangle{\pgfqpoint{0.457963in}{0.528059in}}{\pgfqpoint{6.200000in}{2.285714in}} %
\pgfusepath{clip}%
\pgfsetbuttcap%
\pgfsetroundjoin%
\definecolor{currentfill}{rgb}{0.166667,0.166667,1.000000}%
\pgfsetfillcolor{currentfill}%
\pgfsetlinewidth{1.003750pt}%
\definecolor{currentstroke}{rgb}{0.166667,0.166667,1.000000}%
\pgfsetstrokecolor{currentstroke}%
\pgfsetdash{}{0pt}%
\pgfpathmoveto{\pgfqpoint{0.457963in}{2.129656in}}%
\pgfpathcurveto{\pgfqpoint{0.466200in}{2.129656in}}{\pgfqpoint{0.474100in}{2.132928in}}{\pgfqpoint{0.479924in}{2.138752in}}%
\pgfpathcurveto{\pgfqpoint{0.485748in}{2.144576in}}{\pgfqpoint{0.489020in}{2.152476in}}{\pgfqpoint{0.489020in}{2.160713in}}%
\pgfpathcurveto{\pgfqpoint{0.489020in}{2.168949in}}{\pgfqpoint{0.485748in}{2.176849in}}{\pgfqpoint{0.479924in}{2.182673in}}%
\pgfpathcurveto{\pgfqpoint{0.474100in}{2.188497in}}{\pgfqpoint{0.466200in}{2.191769in}}{\pgfqpoint{0.457963in}{2.191769in}}%
\pgfpathcurveto{\pgfqpoint{0.449727in}{2.191769in}}{\pgfqpoint{0.441827in}{2.188497in}}{\pgfqpoint{0.436003in}{2.182673in}}%
\pgfpathcurveto{\pgfqpoint{0.430179in}{2.176849in}}{\pgfqpoint{0.426907in}{2.168949in}}{\pgfqpoint{0.426907in}{2.160713in}}%
\pgfpathcurveto{\pgfqpoint{0.426907in}{2.152476in}}{\pgfqpoint{0.430179in}{2.144576in}}{\pgfqpoint{0.436003in}{2.138752in}}%
\pgfpathcurveto{\pgfqpoint{0.441827in}{2.132928in}}{\pgfqpoint{0.449727in}{2.129656in}}{\pgfqpoint{0.457963in}{2.129656in}}%
\pgfpathclose%
\pgfusepath{stroke,fill}%
\end{pgfscope}%
\begin{pgfscope}%
\pgfpathrectangle{\pgfqpoint{0.457963in}{0.528059in}}{\pgfqpoint{6.200000in}{2.285714in}} %
\pgfusepath{clip}%
\pgfsetbuttcap%
\pgfsetroundjoin%
\definecolor{currentfill}{rgb}{0.166667,0.166667,1.000000}%
\pgfsetfillcolor{currentfill}%
\pgfsetlinewidth{1.003750pt}%
\definecolor{currentstroke}{rgb}{0.166667,0.166667,1.000000}%
\pgfsetstrokecolor{currentstroke}%
\pgfsetdash{}{0pt}%
\pgfpathmoveto{\pgfqpoint{0.457963in}{2.129656in}}%
\pgfpathcurveto{\pgfqpoint{0.466200in}{2.129656in}}{\pgfqpoint{0.474100in}{2.132928in}}{\pgfqpoint{0.479924in}{2.138752in}}%
\pgfpathcurveto{\pgfqpoint{0.485748in}{2.144576in}}{\pgfqpoint{0.489020in}{2.152476in}}{\pgfqpoint{0.489020in}{2.160713in}}%
\pgfpathcurveto{\pgfqpoint{0.489020in}{2.168949in}}{\pgfqpoint{0.485748in}{2.176849in}}{\pgfqpoint{0.479924in}{2.182673in}}%
\pgfpathcurveto{\pgfqpoint{0.474100in}{2.188497in}}{\pgfqpoint{0.466200in}{2.191769in}}{\pgfqpoint{0.457963in}{2.191769in}}%
\pgfpathcurveto{\pgfqpoint{0.449727in}{2.191769in}}{\pgfqpoint{0.441827in}{2.188497in}}{\pgfqpoint{0.436003in}{2.182673in}}%
\pgfpathcurveto{\pgfqpoint{0.430179in}{2.176849in}}{\pgfqpoint{0.426907in}{2.168949in}}{\pgfqpoint{0.426907in}{2.160713in}}%
\pgfpathcurveto{\pgfqpoint{0.426907in}{2.152476in}}{\pgfqpoint{0.430179in}{2.144576in}}{\pgfqpoint{0.436003in}{2.138752in}}%
\pgfpathcurveto{\pgfqpoint{0.441827in}{2.132928in}}{\pgfqpoint{0.449727in}{2.129656in}}{\pgfqpoint{0.457963in}{2.129656in}}%
\pgfpathclose%
\pgfusepath{stroke,fill}%
\end{pgfscope}%
\begin{pgfscope}%
\pgfpathrectangle{\pgfqpoint{0.457963in}{0.528059in}}{\pgfqpoint{6.200000in}{2.285714in}} %
\pgfusepath{clip}%
\pgfsetbuttcap%
\pgfsetroundjoin%
\definecolor{currentfill}{rgb}{0.166667,0.166667,1.000000}%
\pgfsetfillcolor{currentfill}%
\pgfsetlinewidth{1.003750pt}%
\definecolor{currentstroke}{rgb}{0.166667,0.166667,1.000000}%
\pgfsetstrokecolor{currentstroke}%
\pgfsetdash{}{0pt}%
\pgfpathmoveto{\pgfqpoint{0.478630in}{2.129656in}}%
\pgfpathcurveto{\pgfqpoint{0.486866in}{2.129656in}}{\pgfqpoint{0.494766in}{2.132928in}}{\pgfqpoint{0.500590in}{2.138752in}}%
\pgfpathcurveto{\pgfqpoint{0.506414in}{2.144576in}}{\pgfqpoint{0.509686in}{2.152476in}}{\pgfqpoint{0.509686in}{2.160713in}}%
\pgfpathcurveto{\pgfqpoint{0.509686in}{2.168949in}}{\pgfqpoint{0.506414in}{2.176849in}}{\pgfqpoint{0.500590in}{2.182673in}}%
\pgfpathcurveto{\pgfqpoint{0.494766in}{2.188497in}}{\pgfqpoint{0.486866in}{2.191769in}}{\pgfqpoint{0.478630in}{2.191769in}}%
\pgfpathcurveto{\pgfqpoint{0.470394in}{2.191769in}}{\pgfqpoint{0.462494in}{2.188497in}}{\pgfqpoint{0.456670in}{2.182673in}}%
\pgfpathcurveto{\pgfqpoint{0.450846in}{2.176849in}}{\pgfqpoint{0.447574in}{2.168949in}}{\pgfqpoint{0.447574in}{2.160713in}}%
\pgfpathcurveto{\pgfqpoint{0.447574in}{2.152476in}}{\pgfqpoint{0.450846in}{2.144576in}}{\pgfqpoint{0.456670in}{2.138752in}}%
\pgfpathcurveto{\pgfqpoint{0.462494in}{2.132928in}}{\pgfqpoint{0.470394in}{2.129656in}}{\pgfqpoint{0.478630in}{2.129656in}}%
\pgfpathclose%
\pgfusepath{stroke,fill}%
\end{pgfscope}%
\begin{pgfscope}%
\pgfpathrectangle{\pgfqpoint{0.457963in}{0.528059in}}{\pgfqpoint{6.200000in}{2.285714in}} %
\pgfusepath{clip}%
\pgfsetbuttcap%
\pgfsetroundjoin%
\definecolor{currentfill}{rgb}{0.166667,0.166667,1.000000}%
\pgfsetfillcolor{currentfill}%
\pgfsetlinewidth{1.003750pt}%
\definecolor{currentstroke}{rgb}{0.166667,0.166667,1.000000}%
\pgfsetstrokecolor{currentstroke}%
\pgfsetdash{}{0pt}%
\pgfpathmoveto{\pgfqpoint{0.488963in}{2.129656in}}%
\pgfpathcurveto{\pgfqpoint{0.497200in}{2.129656in}}{\pgfqpoint{0.505100in}{2.132928in}}{\pgfqpoint{0.510924in}{2.138752in}}%
\pgfpathcurveto{\pgfqpoint{0.516748in}{2.144576in}}{\pgfqpoint{0.520020in}{2.152476in}}{\pgfqpoint{0.520020in}{2.160713in}}%
\pgfpathcurveto{\pgfqpoint{0.520020in}{2.168949in}}{\pgfqpoint{0.516748in}{2.176849in}}{\pgfqpoint{0.510924in}{2.182673in}}%
\pgfpathcurveto{\pgfqpoint{0.505100in}{2.188497in}}{\pgfqpoint{0.497200in}{2.191769in}}{\pgfqpoint{0.488963in}{2.191769in}}%
\pgfpathcurveto{\pgfqpoint{0.480727in}{2.191769in}}{\pgfqpoint{0.472827in}{2.188497in}}{\pgfqpoint{0.467003in}{2.182673in}}%
\pgfpathcurveto{\pgfqpoint{0.461179in}{2.176849in}}{\pgfqpoint{0.457907in}{2.168949in}}{\pgfqpoint{0.457907in}{2.160713in}}%
\pgfpathcurveto{\pgfqpoint{0.457907in}{2.152476in}}{\pgfqpoint{0.461179in}{2.144576in}}{\pgfqpoint{0.467003in}{2.138752in}}%
\pgfpathcurveto{\pgfqpoint{0.472827in}{2.132928in}}{\pgfqpoint{0.480727in}{2.129656in}}{\pgfqpoint{0.488963in}{2.129656in}}%
\pgfpathclose%
\pgfusepath{stroke,fill}%
\end{pgfscope}%
\begin{pgfscope}%
\pgfpathrectangle{\pgfqpoint{0.457963in}{0.528059in}}{\pgfqpoint{6.200000in}{2.285714in}} %
\pgfusepath{clip}%
\pgfsetbuttcap%
\pgfsetroundjoin%
\definecolor{currentfill}{rgb}{0.166667,0.166667,1.000000}%
\pgfsetfillcolor{currentfill}%
\pgfsetlinewidth{1.003750pt}%
\definecolor{currentstroke}{rgb}{0.166667,0.166667,1.000000}%
\pgfsetstrokecolor{currentstroke}%
\pgfsetdash{}{0pt}%
\pgfpathmoveto{\pgfqpoint{0.581963in}{2.103534in}}%
\pgfpathcurveto{\pgfqpoint{0.590200in}{2.103534in}}{\pgfqpoint{0.598100in}{2.106806in}}{\pgfqpoint{0.603924in}{2.112630in}}%
\pgfpathcurveto{\pgfqpoint{0.609748in}{2.118454in}}{\pgfqpoint{0.613020in}{2.126354in}}{\pgfqpoint{0.613020in}{2.134590in}}%
\pgfpathcurveto{\pgfqpoint{0.613020in}{2.142826in}}{\pgfqpoint{0.609748in}{2.150726in}}{\pgfqpoint{0.603924in}{2.156550in}}%
\pgfpathcurveto{\pgfqpoint{0.598100in}{2.162374in}}{\pgfqpoint{0.590200in}{2.165647in}}{\pgfqpoint{0.581963in}{2.165647in}}%
\pgfpathcurveto{\pgfqpoint{0.573727in}{2.165647in}}{\pgfqpoint{0.565827in}{2.162374in}}{\pgfqpoint{0.560003in}{2.156550in}}%
\pgfpathcurveto{\pgfqpoint{0.554179in}{2.150726in}}{\pgfqpoint{0.550907in}{2.142826in}}{\pgfqpoint{0.550907in}{2.134590in}}%
\pgfpathcurveto{\pgfqpoint{0.550907in}{2.126354in}}{\pgfqpoint{0.554179in}{2.118454in}}{\pgfqpoint{0.560003in}{2.112630in}}%
\pgfpathcurveto{\pgfqpoint{0.565827in}{2.106806in}}{\pgfqpoint{0.573727in}{2.103534in}}{\pgfqpoint{0.581963in}{2.103534in}}%
\pgfpathclose%
\pgfusepath{stroke,fill}%
\end{pgfscope}%
\begin{pgfscope}%
\pgfpathrectangle{\pgfqpoint{0.457963in}{0.528059in}}{\pgfqpoint{6.200000in}{2.285714in}} %
\pgfusepath{clip}%
\pgfsetbuttcap%
\pgfsetroundjoin%
\definecolor{currentfill}{rgb}{0.166667,0.166667,1.000000}%
\pgfsetfillcolor{currentfill}%
\pgfsetlinewidth{1.003750pt}%
\definecolor{currentstroke}{rgb}{0.166667,0.166667,1.000000}%
\pgfsetstrokecolor{currentstroke}%
\pgfsetdash{}{0pt}%
\pgfpathmoveto{\pgfqpoint{0.581963in}{2.129656in}}%
\pgfpathcurveto{\pgfqpoint{0.590200in}{2.129656in}}{\pgfqpoint{0.598100in}{2.132928in}}{\pgfqpoint{0.603924in}{2.138752in}}%
\pgfpathcurveto{\pgfqpoint{0.609748in}{2.144576in}}{\pgfqpoint{0.613020in}{2.152476in}}{\pgfqpoint{0.613020in}{2.160713in}}%
\pgfpathcurveto{\pgfqpoint{0.613020in}{2.168949in}}{\pgfqpoint{0.609748in}{2.176849in}}{\pgfqpoint{0.603924in}{2.182673in}}%
\pgfpathcurveto{\pgfqpoint{0.598100in}{2.188497in}}{\pgfqpoint{0.590200in}{2.191769in}}{\pgfqpoint{0.581963in}{2.191769in}}%
\pgfpathcurveto{\pgfqpoint{0.573727in}{2.191769in}}{\pgfqpoint{0.565827in}{2.188497in}}{\pgfqpoint{0.560003in}{2.182673in}}%
\pgfpathcurveto{\pgfqpoint{0.554179in}{2.176849in}}{\pgfqpoint{0.550907in}{2.168949in}}{\pgfqpoint{0.550907in}{2.160713in}}%
\pgfpathcurveto{\pgfqpoint{0.550907in}{2.152476in}}{\pgfqpoint{0.554179in}{2.144576in}}{\pgfqpoint{0.560003in}{2.138752in}}%
\pgfpathcurveto{\pgfqpoint{0.565827in}{2.132928in}}{\pgfqpoint{0.573727in}{2.129656in}}{\pgfqpoint{0.581963in}{2.129656in}}%
\pgfpathclose%
\pgfusepath{stroke,fill}%
\end{pgfscope}%
\begin{pgfscope}%
\pgfpathrectangle{\pgfqpoint{0.457963in}{0.528059in}}{\pgfqpoint{6.200000in}{2.285714in}} %
\pgfusepath{clip}%
\pgfsetbuttcap%
\pgfsetroundjoin%
\definecolor{currentfill}{rgb}{0.166667,0.166667,1.000000}%
\pgfsetfillcolor{currentfill}%
\pgfsetlinewidth{1.003750pt}%
\definecolor{currentstroke}{rgb}{0.166667,0.166667,1.000000}%
\pgfsetstrokecolor{currentstroke}%
\pgfsetdash{}{0pt}%
\pgfpathmoveto{\pgfqpoint{0.685297in}{2.103534in}}%
\pgfpathcurveto{\pgfqpoint{0.693533in}{2.103534in}}{\pgfqpoint{0.701433in}{2.106806in}}{\pgfqpoint{0.707257in}{2.112630in}}%
\pgfpathcurveto{\pgfqpoint{0.713081in}{2.118454in}}{\pgfqpoint{0.716353in}{2.126354in}}{\pgfqpoint{0.716353in}{2.134590in}}%
\pgfpathcurveto{\pgfqpoint{0.716353in}{2.142826in}}{\pgfqpoint{0.713081in}{2.150726in}}{\pgfqpoint{0.707257in}{2.156550in}}%
\pgfpathcurveto{\pgfqpoint{0.701433in}{2.162374in}}{\pgfqpoint{0.693533in}{2.165647in}}{\pgfqpoint{0.685297in}{2.165647in}}%
\pgfpathcurveto{\pgfqpoint{0.677060in}{2.165647in}}{\pgfqpoint{0.669160in}{2.162374in}}{\pgfqpoint{0.663336in}{2.156550in}}%
\pgfpathcurveto{\pgfqpoint{0.657512in}{2.150726in}}{\pgfqpoint{0.654240in}{2.142826in}}{\pgfqpoint{0.654240in}{2.134590in}}%
\pgfpathcurveto{\pgfqpoint{0.654240in}{2.126354in}}{\pgfqpoint{0.657512in}{2.118454in}}{\pgfqpoint{0.663336in}{2.112630in}}%
\pgfpathcurveto{\pgfqpoint{0.669160in}{2.106806in}}{\pgfqpoint{0.677060in}{2.103534in}}{\pgfqpoint{0.685297in}{2.103534in}}%
\pgfpathclose%
\pgfusepath{stroke,fill}%
\end{pgfscope}%
\begin{pgfscope}%
\pgfpathrectangle{\pgfqpoint{0.457963in}{0.528059in}}{\pgfqpoint{6.200000in}{2.285714in}} %
\pgfusepath{clip}%
\pgfsetbuttcap%
\pgfsetroundjoin%
\definecolor{currentfill}{rgb}{0.166667,0.166667,1.000000}%
\pgfsetfillcolor{currentfill}%
\pgfsetlinewidth{1.003750pt}%
\definecolor{currentstroke}{rgb}{0.166667,0.166667,1.000000}%
\pgfsetstrokecolor{currentstroke}%
\pgfsetdash{}{0pt}%
\pgfpathmoveto{\pgfqpoint{0.891963in}{2.129656in}}%
\pgfpathcurveto{\pgfqpoint{0.900200in}{2.129656in}}{\pgfqpoint{0.908100in}{2.132928in}}{\pgfqpoint{0.913924in}{2.138752in}}%
\pgfpathcurveto{\pgfqpoint{0.919748in}{2.144576in}}{\pgfqpoint{0.923020in}{2.152476in}}{\pgfqpoint{0.923020in}{2.160713in}}%
\pgfpathcurveto{\pgfqpoint{0.923020in}{2.168949in}}{\pgfqpoint{0.919748in}{2.176849in}}{\pgfqpoint{0.913924in}{2.182673in}}%
\pgfpathcurveto{\pgfqpoint{0.908100in}{2.188497in}}{\pgfqpoint{0.900200in}{2.191769in}}{\pgfqpoint{0.891963in}{2.191769in}}%
\pgfpathcurveto{\pgfqpoint{0.883727in}{2.191769in}}{\pgfqpoint{0.875827in}{2.188497in}}{\pgfqpoint{0.870003in}{2.182673in}}%
\pgfpathcurveto{\pgfqpoint{0.864179in}{2.176849in}}{\pgfqpoint{0.860907in}{2.168949in}}{\pgfqpoint{0.860907in}{2.160713in}}%
\pgfpathcurveto{\pgfqpoint{0.860907in}{2.152476in}}{\pgfqpoint{0.864179in}{2.144576in}}{\pgfqpoint{0.870003in}{2.138752in}}%
\pgfpathcurveto{\pgfqpoint{0.875827in}{2.132928in}}{\pgfqpoint{0.883727in}{2.129656in}}{\pgfqpoint{0.891963in}{2.129656in}}%
\pgfpathclose%
\pgfusepath{stroke,fill}%
\end{pgfscope}%
\begin{pgfscope}%
\pgfpathrectangle{\pgfqpoint{0.457963in}{0.528059in}}{\pgfqpoint{6.200000in}{2.285714in}} %
\pgfusepath{clip}%
\pgfsetbuttcap%
\pgfsetroundjoin%
\definecolor{currentfill}{rgb}{0.166667,0.166667,1.000000}%
\pgfsetfillcolor{currentfill}%
\pgfsetlinewidth{1.003750pt}%
\definecolor{currentstroke}{rgb}{0.166667,0.166667,1.000000}%
\pgfsetstrokecolor{currentstroke}%
\pgfsetdash{}{0pt}%
\pgfpathmoveto{\pgfqpoint{1.212297in}{2.103534in}}%
\pgfpathcurveto{\pgfqpoint{1.220533in}{2.103534in}}{\pgfqpoint{1.228433in}{2.106806in}}{\pgfqpoint{1.234257in}{2.112630in}}%
\pgfpathcurveto{\pgfqpoint{1.240081in}{2.118454in}}{\pgfqpoint{1.243353in}{2.126354in}}{\pgfqpoint{1.243353in}{2.134590in}}%
\pgfpathcurveto{\pgfqpoint{1.243353in}{2.142826in}}{\pgfqpoint{1.240081in}{2.150726in}}{\pgfqpoint{1.234257in}{2.156550in}}%
\pgfpathcurveto{\pgfqpoint{1.228433in}{2.162374in}}{\pgfqpoint{1.220533in}{2.165647in}}{\pgfqpoint{1.212297in}{2.165647in}}%
\pgfpathcurveto{\pgfqpoint{1.204060in}{2.165647in}}{\pgfqpoint{1.196160in}{2.162374in}}{\pgfqpoint{1.190336in}{2.156550in}}%
\pgfpathcurveto{\pgfqpoint{1.184512in}{2.150726in}}{\pgfqpoint{1.181240in}{2.142826in}}{\pgfqpoint{1.181240in}{2.134590in}}%
\pgfpathcurveto{\pgfqpoint{1.181240in}{2.126354in}}{\pgfqpoint{1.184512in}{2.118454in}}{\pgfqpoint{1.190336in}{2.112630in}}%
\pgfpathcurveto{\pgfqpoint{1.196160in}{2.106806in}}{\pgfqpoint{1.204060in}{2.103534in}}{\pgfqpoint{1.212297in}{2.103534in}}%
\pgfpathclose%
\pgfusepath{stroke,fill}%
\end{pgfscope}%
\begin{pgfscope}%
\pgfpathrectangle{\pgfqpoint{0.457963in}{0.528059in}}{\pgfqpoint{6.200000in}{2.285714in}} %
\pgfusepath{clip}%
\pgfsetbuttcap%
\pgfsetroundjoin%
\definecolor{currentfill}{rgb}{0.166667,0.166667,1.000000}%
\pgfsetfillcolor{currentfill}%
\pgfsetlinewidth{1.003750pt}%
\definecolor{currentstroke}{rgb}{0.166667,0.166667,1.000000}%
\pgfsetstrokecolor{currentstroke}%
\pgfsetdash{}{0pt}%
\pgfpathmoveto{\pgfqpoint{1.666963in}{2.129656in}}%
\pgfpathcurveto{\pgfqpoint{1.675200in}{2.129656in}}{\pgfqpoint{1.683100in}{2.132928in}}{\pgfqpoint{1.688924in}{2.138752in}}%
\pgfpathcurveto{\pgfqpoint{1.694748in}{2.144576in}}{\pgfqpoint{1.698020in}{2.152476in}}{\pgfqpoint{1.698020in}{2.160713in}}%
\pgfpathcurveto{\pgfqpoint{1.698020in}{2.168949in}}{\pgfqpoint{1.694748in}{2.176849in}}{\pgfqpoint{1.688924in}{2.182673in}}%
\pgfpathcurveto{\pgfqpoint{1.683100in}{2.188497in}}{\pgfqpoint{1.675200in}{2.191769in}}{\pgfqpoint{1.666963in}{2.191769in}}%
\pgfpathcurveto{\pgfqpoint{1.658727in}{2.191769in}}{\pgfqpoint{1.650827in}{2.188497in}}{\pgfqpoint{1.645003in}{2.182673in}}%
\pgfpathcurveto{\pgfqpoint{1.639179in}{2.176849in}}{\pgfqpoint{1.635907in}{2.168949in}}{\pgfqpoint{1.635907in}{2.160713in}}%
\pgfpathcurveto{\pgfqpoint{1.635907in}{2.152476in}}{\pgfqpoint{1.639179in}{2.144576in}}{\pgfqpoint{1.645003in}{2.138752in}}%
\pgfpathcurveto{\pgfqpoint{1.650827in}{2.132928in}}{\pgfqpoint{1.658727in}{2.129656in}}{\pgfqpoint{1.666963in}{2.129656in}}%
\pgfpathclose%
\pgfusepath{stroke,fill}%
\end{pgfscope}%
\begin{pgfscope}%
\pgfpathrectangle{\pgfqpoint{0.457963in}{0.528059in}}{\pgfqpoint{6.200000in}{2.285714in}} %
\pgfusepath{clip}%
\pgfsetbuttcap%
\pgfsetroundjoin%
\definecolor{currentfill}{rgb}{0.166667,0.166667,1.000000}%
\pgfsetfillcolor{currentfill}%
\pgfsetlinewidth{1.003750pt}%
\definecolor{currentstroke}{rgb}{0.166667,0.166667,1.000000}%
\pgfsetstrokecolor{currentstroke}%
\pgfsetdash{}{0pt}%
\pgfpathmoveto{\pgfqpoint{1.677297in}{1.972921in}}%
\pgfpathcurveto{\pgfqpoint{1.685533in}{1.972921in}}{\pgfqpoint{1.693433in}{1.976194in}}{\pgfqpoint{1.699257in}{1.982018in}}%
\pgfpathcurveto{\pgfqpoint{1.705081in}{1.987841in}}{\pgfqpoint{1.708353in}{1.995742in}}{\pgfqpoint{1.708353in}{2.003978in}}%
\pgfpathcurveto{\pgfqpoint{1.708353in}{2.012214in}}{\pgfqpoint{1.705081in}{2.020114in}}{\pgfqpoint{1.699257in}{2.025938in}}%
\pgfpathcurveto{\pgfqpoint{1.693433in}{2.031762in}}{\pgfqpoint{1.685533in}{2.035034in}}{\pgfqpoint{1.677297in}{2.035034in}}%
\pgfpathcurveto{\pgfqpoint{1.669060in}{2.035034in}}{\pgfqpoint{1.661160in}{2.031762in}}{\pgfqpoint{1.655336in}{2.025938in}}%
\pgfpathcurveto{\pgfqpoint{1.649512in}{2.020114in}}{\pgfqpoint{1.646240in}{2.012214in}}{\pgfqpoint{1.646240in}{2.003978in}}%
\pgfpathcurveto{\pgfqpoint{1.646240in}{1.995742in}}{\pgfqpoint{1.649512in}{1.987841in}}{\pgfqpoint{1.655336in}{1.982018in}}%
\pgfpathcurveto{\pgfqpoint{1.661160in}{1.976194in}}{\pgfqpoint{1.669060in}{1.972921in}}{\pgfqpoint{1.677297in}{1.972921in}}%
\pgfpathclose%
\pgfusepath{stroke,fill}%
\end{pgfscope}%
\begin{pgfscope}%
\pgfpathrectangle{\pgfqpoint{0.457963in}{0.528059in}}{\pgfqpoint{6.200000in}{2.285714in}} %
\pgfusepath{clip}%
\pgfsetbuttcap%
\pgfsetroundjoin%
\definecolor{currentfill}{rgb}{0.166667,0.166667,1.000000}%
\pgfsetfillcolor{currentfill}%
\pgfsetlinewidth{1.003750pt}%
\definecolor{currentstroke}{rgb}{0.166667,0.166667,1.000000}%
\pgfsetstrokecolor{currentstroke}%
\pgfsetdash{}{0pt}%
\pgfpathmoveto{\pgfqpoint{2.782963in}{1.803125in}}%
\pgfpathcurveto{\pgfqpoint{2.791200in}{1.803125in}}{\pgfqpoint{2.799100in}{1.806398in}}{\pgfqpoint{2.804924in}{1.812222in}}%
\pgfpathcurveto{\pgfqpoint{2.810748in}{1.818046in}}{\pgfqpoint{2.814020in}{1.825946in}}{\pgfqpoint{2.814020in}{1.834182in}}%
\pgfpathcurveto{\pgfqpoint{2.814020in}{1.842418in}}{\pgfqpoint{2.810748in}{1.850318in}}{\pgfqpoint{2.804924in}{1.856142in}}%
\pgfpathcurveto{\pgfqpoint{2.799100in}{1.861966in}}{\pgfqpoint{2.791200in}{1.865238in}}{\pgfqpoint{2.782963in}{1.865238in}}%
\pgfpathcurveto{\pgfqpoint{2.774727in}{1.865238in}}{\pgfqpoint{2.766827in}{1.861966in}}{\pgfqpoint{2.761003in}{1.856142in}}%
\pgfpathcurveto{\pgfqpoint{2.755179in}{1.850318in}}{\pgfqpoint{2.751907in}{1.842418in}}{\pgfqpoint{2.751907in}{1.834182in}}%
\pgfpathcurveto{\pgfqpoint{2.751907in}{1.825946in}}{\pgfqpoint{2.755179in}{1.818046in}}{\pgfqpoint{2.761003in}{1.812222in}}%
\pgfpathcurveto{\pgfqpoint{2.766827in}{1.806398in}}{\pgfqpoint{2.774727in}{1.803125in}}{\pgfqpoint{2.782963in}{1.803125in}}%
\pgfpathclose%
\pgfusepath{stroke,fill}%
\end{pgfscope}%
\begin{pgfscope}%
\pgfpathrectangle{\pgfqpoint{0.457963in}{0.528059in}}{\pgfqpoint{6.200000in}{2.285714in}} %
\pgfusepath{clip}%
\pgfsetbuttcap%
\pgfsetroundjoin%
\definecolor{currentfill}{rgb}{0.166667,0.166667,1.000000}%
\pgfsetfillcolor{currentfill}%
\pgfsetlinewidth{1.003750pt}%
\definecolor{currentstroke}{rgb}{0.166667,0.166667,1.000000}%
\pgfsetstrokecolor{currentstroke}%
\pgfsetdash{}{0pt}%
\pgfpathmoveto{\pgfqpoint{3.030963in}{2.077411in}}%
\pgfpathcurveto{\pgfqpoint{3.039200in}{2.077411in}}{\pgfqpoint{3.047100in}{2.080683in}}{\pgfqpoint{3.052924in}{2.086507in}}%
\pgfpathcurveto{\pgfqpoint{3.058748in}{2.092331in}}{\pgfqpoint{3.062020in}{2.100231in}}{\pgfqpoint{3.062020in}{2.108468in}}%
\pgfpathcurveto{\pgfqpoint{3.062020in}{2.116704in}}{\pgfqpoint{3.058748in}{2.124604in}}{\pgfqpoint{3.052924in}{2.130428in}}%
\pgfpathcurveto{\pgfqpoint{3.047100in}{2.136252in}}{\pgfqpoint{3.039200in}{2.139524in}}{\pgfqpoint{3.030963in}{2.139524in}}%
\pgfpathcurveto{\pgfqpoint{3.022727in}{2.139524in}}{\pgfqpoint{3.014827in}{2.136252in}}{\pgfqpoint{3.009003in}{2.130428in}}%
\pgfpathcurveto{\pgfqpoint{3.003179in}{2.124604in}}{\pgfqpoint{2.999907in}{2.116704in}}{\pgfqpoint{2.999907in}{2.108468in}}%
\pgfpathcurveto{\pgfqpoint{2.999907in}{2.100231in}}{\pgfqpoint{3.003179in}{2.092331in}}{\pgfqpoint{3.009003in}{2.086507in}}%
\pgfpathcurveto{\pgfqpoint{3.014827in}{2.080683in}}{\pgfqpoint{3.022727in}{2.077411in}}{\pgfqpoint{3.030963in}{2.077411in}}%
\pgfpathclose%
\pgfusepath{stroke,fill}%
\end{pgfscope}%
\begin{pgfscope}%
\pgfpathrectangle{\pgfqpoint{0.457963in}{0.528059in}}{\pgfqpoint{6.200000in}{2.285714in}} %
\pgfusepath{clip}%
\pgfsetbuttcap%
\pgfsetroundjoin%
\definecolor{currentfill}{rgb}{0.166667,0.166667,1.000000}%
\pgfsetfillcolor{currentfill}%
\pgfsetlinewidth{1.003750pt}%
\definecolor{currentstroke}{rgb}{0.166667,0.166667,1.000000}%
\pgfsetstrokecolor{currentstroke}%
\pgfsetdash{}{0pt}%
\pgfpathmoveto{\pgfqpoint{3.919630in}{1.607207in}}%
\pgfpathcurveto{\pgfqpoint{3.927866in}{1.607207in}}{\pgfqpoint{3.935766in}{1.610479in}}{\pgfqpoint{3.941590in}{1.616303in}}%
\pgfpathcurveto{\pgfqpoint{3.947414in}{1.622127in}}{\pgfqpoint{3.950686in}{1.630027in}}{\pgfqpoint{3.950686in}{1.638264in}}%
\pgfpathcurveto{\pgfqpoint{3.950686in}{1.646500in}}{\pgfqpoint{3.947414in}{1.654400in}}{\pgfqpoint{3.941590in}{1.660224in}}%
\pgfpathcurveto{\pgfqpoint{3.935766in}{1.666048in}}{\pgfqpoint{3.927866in}{1.669320in}}{\pgfqpoint{3.919630in}{1.669320in}}%
\pgfpathcurveto{\pgfqpoint{3.911394in}{1.669320in}}{\pgfqpoint{3.903494in}{1.666048in}}{\pgfqpoint{3.897670in}{1.660224in}}%
\pgfpathcurveto{\pgfqpoint{3.891846in}{1.654400in}}{\pgfqpoint{3.888574in}{1.646500in}}{\pgfqpoint{3.888574in}{1.638264in}}%
\pgfpathcurveto{\pgfqpoint{3.888574in}{1.630027in}}{\pgfqpoint{3.891846in}{1.622127in}}{\pgfqpoint{3.897670in}{1.616303in}}%
\pgfpathcurveto{\pgfqpoint{3.903494in}{1.610479in}}{\pgfqpoint{3.911394in}{1.607207in}}{\pgfqpoint{3.919630in}{1.607207in}}%
\pgfpathclose%
\pgfusepath{stroke,fill}%
\end{pgfscope}%
\begin{pgfscope}%
\pgfpathrectangle{\pgfqpoint{0.457963in}{0.528059in}}{\pgfqpoint{6.200000in}{2.285714in}} %
\pgfusepath{clip}%
\pgfsetbuttcap%
\pgfsetroundjoin%
\definecolor{currentfill}{rgb}{0.166667,0.166667,1.000000}%
\pgfsetfillcolor{currentfill}%
\pgfsetlinewidth{1.003750pt}%
\definecolor{currentstroke}{rgb}{0.166667,0.166667,1.000000}%
\pgfsetstrokecolor{currentstroke}%
\pgfsetdash{}{0pt}%
\pgfpathmoveto{\pgfqpoint{4.064297in}{1.946799in}}%
\pgfpathcurveto{\pgfqpoint{4.072533in}{1.946799in}}{\pgfqpoint{4.080433in}{1.950071in}}{\pgfqpoint{4.086257in}{1.955895in}}%
\pgfpathcurveto{\pgfqpoint{4.092081in}{1.961719in}}{\pgfqpoint{4.095353in}{1.969619in}}{\pgfqpoint{4.095353in}{1.977855in}}%
\pgfpathcurveto{\pgfqpoint{4.095353in}{1.986092in}}{\pgfqpoint{4.092081in}{1.993992in}}{\pgfqpoint{4.086257in}{1.999816in}}%
\pgfpathcurveto{\pgfqpoint{4.080433in}{2.005640in}}{\pgfqpoint{4.072533in}{2.008912in}}{\pgfqpoint{4.064297in}{2.008912in}}%
\pgfpathcurveto{\pgfqpoint{4.056060in}{2.008912in}}{\pgfqpoint{4.048160in}{2.005640in}}{\pgfqpoint{4.042336in}{1.999816in}}%
\pgfpathcurveto{\pgfqpoint{4.036512in}{1.993992in}}{\pgfqpoint{4.033240in}{1.986092in}}{\pgfqpoint{4.033240in}{1.977855in}}%
\pgfpathcurveto{\pgfqpoint{4.033240in}{1.969619in}}{\pgfqpoint{4.036512in}{1.961719in}}{\pgfqpoint{4.042336in}{1.955895in}}%
\pgfpathcurveto{\pgfqpoint{4.048160in}{1.950071in}}{\pgfqpoint{4.056060in}{1.946799in}}{\pgfqpoint{4.064297in}{1.946799in}}%
\pgfpathclose%
\pgfusepath{stroke,fill}%
\end{pgfscope}%
\begin{pgfscope}%
\pgfpathrectangle{\pgfqpoint{0.457963in}{0.528059in}}{\pgfqpoint{6.200000in}{2.285714in}} %
\pgfusepath{clip}%
\pgfsetbuttcap%
\pgfsetroundjoin%
\definecolor{currentfill}{rgb}{0.166667,0.166667,1.000000}%
\pgfsetfillcolor{currentfill}%
\pgfsetlinewidth{1.003750pt}%
\definecolor{currentstroke}{rgb}{0.166667,0.166667,1.000000}%
\pgfsetstrokecolor{currentstroke}%
\pgfsetdash{}{0pt}%
\pgfpathmoveto{\pgfqpoint{4.673963in}{1.711697in}}%
\pgfpathcurveto{\pgfqpoint{4.682200in}{1.711697in}}{\pgfqpoint{4.690100in}{1.714969in}}{\pgfqpoint{4.695924in}{1.720793in}}%
\pgfpathcurveto{\pgfqpoint{4.701748in}{1.726617in}}{\pgfqpoint{4.705020in}{1.734517in}}{\pgfqpoint{4.705020in}{1.742753in}}%
\pgfpathcurveto{\pgfqpoint{4.705020in}{1.750990in}}{\pgfqpoint{4.701748in}{1.758890in}}{\pgfqpoint{4.695924in}{1.764714in}}%
\pgfpathcurveto{\pgfqpoint{4.690100in}{1.770538in}}{\pgfqpoint{4.682200in}{1.773810in}}{\pgfqpoint{4.673963in}{1.773810in}}%
\pgfpathcurveto{\pgfqpoint{4.665727in}{1.773810in}}{\pgfqpoint{4.657827in}{1.770538in}}{\pgfqpoint{4.652003in}{1.764714in}}%
\pgfpathcurveto{\pgfqpoint{4.646179in}{1.758890in}}{\pgfqpoint{4.642907in}{1.750990in}}{\pgfqpoint{4.642907in}{1.742753in}}%
\pgfpathcurveto{\pgfqpoint{4.642907in}{1.734517in}}{\pgfqpoint{4.646179in}{1.726617in}}{\pgfqpoint{4.652003in}{1.720793in}}%
\pgfpathcurveto{\pgfqpoint{4.657827in}{1.714969in}}{\pgfqpoint{4.665727in}{1.711697in}}{\pgfqpoint{4.673963in}{1.711697in}}%
\pgfpathclose%
\pgfusepath{stroke,fill}%
\end{pgfscope}%
\begin{pgfscope}%
\pgfpathrectangle{\pgfqpoint{0.457963in}{0.528059in}}{\pgfqpoint{6.200000in}{2.285714in}} %
\pgfusepath{clip}%
\pgfsetbuttcap%
\pgfsetroundjoin%
\definecolor{currentfill}{rgb}{0.166667,0.166667,1.000000}%
\pgfsetfillcolor{currentfill}%
\pgfsetlinewidth{1.003750pt}%
\definecolor{currentstroke}{rgb}{0.166667,0.166667,1.000000}%
\pgfsetstrokecolor{currentstroke}%
\pgfsetdash{}{0pt}%
\pgfpathmoveto{\pgfqpoint{4.963297in}{1.306799in}}%
\pgfpathcurveto{\pgfqpoint{4.971533in}{1.306799in}}{\pgfqpoint{4.979433in}{1.310071in}}{\pgfqpoint{4.985257in}{1.315895in}}%
\pgfpathcurveto{\pgfqpoint{4.991081in}{1.321719in}}{\pgfqpoint{4.994353in}{1.329619in}}{\pgfqpoint{4.994353in}{1.337855in}}%
\pgfpathcurveto{\pgfqpoint{4.994353in}{1.346092in}}{\pgfqpoint{4.991081in}{1.353992in}}{\pgfqpoint{4.985257in}{1.359816in}}%
\pgfpathcurveto{\pgfqpoint{4.979433in}{1.365640in}}{\pgfqpoint{4.971533in}{1.368912in}}{\pgfqpoint{4.963297in}{1.368912in}}%
\pgfpathcurveto{\pgfqpoint{4.955060in}{1.368912in}}{\pgfqpoint{4.947160in}{1.365640in}}{\pgfqpoint{4.941336in}{1.359816in}}%
\pgfpathcurveto{\pgfqpoint{4.935512in}{1.353992in}}{\pgfqpoint{4.932240in}{1.346092in}}{\pgfqpoint{4.932240in}{1.337855in}}%
\pgfpathcurveto{\pgfqpoint{4.932240in}{1.329619in}}{\pgfqpoint{4.935512in}{1.321719in}}{\pgfqpoint{4.941336in}{1.315895in}}%
\pgfpathcurveto{\pgfqpoint{4.947160in}{1.310071in}}{\pgfqpoint{4.955060in}{1.306799in}}{\pgfqpoint{4.963297in}{1.306799in}}%
\pgfpathclose%
\pgfusepath{stroke,fill}%
\end{pgfscope}%
\begin{pgfscope}%
\pgfpathrectangle{\pgfqpoint{0.457963in}{0.528059in}}{\pgfqpoint{6.200000in}{2.285714in}} %
\pgfusepath{clip}%
\pgfsetbuttcap%
\pgfsetroundjoin%
\definecolor{currentfill}{rgb}{0.166667,0.166667,1.000000}%
\pgfsetfillcolor{currentfill}%
\pgfsetlinewidth{1.003750pt}%
\definecolor{currentstroke}{rgb}{0.166667,0.166667,1.000000}%
\pgfsetstrokecolor{currentstroke}%
\pgfsetdash{}{0pt}%
\pgfpathmoveto{\pgfqpoint{5.789963in}{1.123942in}}%
\pgfpathcurveto{\pgfqpoint{5.798200in}{1.123942in}}{\pgfqpoint{5.806100in}{1.127214in}}{\pgfqpoint{5.811924in}{1.133038in}}%
\pgfpathcurveto{\pgfqpoint{5.817748in}{1.138862in}}{\pgfqpoint{5.821020in}{1.146762in}}{\pgfqpoint{5.821020in}{1.154998in}}%
\pgfpathcurveto{\pgfqpoint{5.821020in}{1.163234in}}{\pgfqpoint{5.817748in}{1.171135in}}{\pgfqpoint{5.811924in}{1.176958in}}%
\pgfpathcurveto{\pgfqpoint{5.806100in}{1.182782in}}{\pgfqpoint{5.798200in}{1.186055in}}{\pgfqpoint{5.789963in}{1.186055in}}%
\pgfpathcurveto{\pgfqpoint{5.781727in}{1.186055in}}{\pgfqpoint{5.773827in}{1.182782in}}{\pgfqpoint{5.768003in}{1.176958in}}%
\pgfpathcurveto{\pgfqpoint{5.762179in}{1.171135in}}{\pgfqpoint{5.758907in}{1.163234in}}{\pgfqpoint{5.758907in}{1.154998in}}%
\pgfpathcurveto{\pgfqpoint{5.758907in}{1.146762in}}{\pgfqpoint{5.762179in}{1.138862in}}{\pgfqpoint{5.768003in}{1.133038in}}%
\pgfpathcurveto{\pgfqpoint{5.773827in}{1.127214in}}{\pgfqpoint{5.781727in}{1.123942in}}{\pgfqpoint{5.789963in}{1.123942in}}%
\pgfpathclose%
\pgfusepath{stroke,fill}%
\end{pgfscope}%
\begin{pgfscope}%
\pgfpathrectangle{\pgfqpoint{0.457963in}{0.528059in}}{\pgfqpoint{6.200000in}{2.285714in}} %
\pgfusepath{clip}%
\pgfsetbuttcap%
\pgfsetroundjoin%
\definecolor{currentfill}{rgb}{0.000000,0.000000,1.000000}%
\pgfsetfillcolor{currentfill}%
\pgfsetlinewidth{1.003750pt}%
\definecolor{currentstroke}{rgb}{0.000000,0.000000,1.000000}%
\pgfsetstrokecolor{currentstroke}%
\pgfsetdash{}{0pt}%
\pgfpathmoveto{\pgfqpoint{0.457963in}{2.456187in}}%
\pgfpathcurveto{\pgfqpoint{0.466200in}{2.456187in}}{\pgfqpoint{0.474100in}{2.459459in}}{\pgfqpoint{0.479924in}{2.465283in}}%
\pgfpathcurveto{\pgfqpoint{0.485748in}{2.471107in}}{\pgfqpoint{0.489020in}{2.479007in}}{\pgfqpoint{0.489020in}{2.487243in}}%
\pgfpathcurveto{\pgfqpoint{0.489020in}{2.495479in}}{\pgfqpoint{0.485748in}{2.503379in}}{\pgfqpoint{0.479924in}{2.509203in}}%
\pgfpathcurveto{\pgfqpoint{0.474100in}{2.515027in}}{\pgfqpoint{0.466200in}{2.518300in}}{\pgfqpoint{0.457963in}{2.518300in}}%
\pgfpathcurveto{\pgfqpoint{0.449727in}{2.518300in}}{\pgfqpoint{0.441827in}{2.515027in}}{\pgfqpoint{0.436003in}{2.509203in}}%
\pgfpathcurveto{\pgfqpoint{0.430179in}{2.503379in}}{\pgfqpoint{0.426907in}{2.495479in}}{\pgfqpoint{0.426907in}{2.487243in}}%
\pgfpathcurveto{\pgfqpoint{0.426907in}{2.479007in}}{\pgfqpoint{0.430179in}{2.471107in}}{\pgfqpoint{0.436003in}{2.465283in}}%
\pgfpathcurveto{\pgfqpoint{0.441827in}{2.459459in}}{\pgfqpoint{0.449727in}{2.456187in}}{\pgfqpoint{0.457963in}{2.456187in}}%
\pgfpathclose%
\pgfusepath{stroke,fill}%
\end{pgfscope}%
\begin{pgfscope}%
\pgfpathrectangle{\pgfqpoint{0.457963in}{0.528059in}}{\pgfqpoint{6.200000in}{2.285714in}} %
\pgfusepath{clip}%
\pgfsetbuttcap%
\pgfsetroundjoin%
\definecolor{currentfill}{rgb}{0.000000,0.000000,1.000000}%
\pgfsetfillcolor{currentfill}%
\pgfsetlinewidth{1.003750pt}%
\definecolor{currentstroke}{rgb}{0.000000,0.000000,1.000000}%
\pgfsetstrokecolor{currentstroke}%
\pgfsetdash{}{0pt}%
\pgfpathmoveto{\pgfqpoint{0.457963in}{2.456187in}}%
\pgfpathcurveto{\pgfqpoint{0.466200in}{2.456187in}}{\pgfqpoint{0.474100in}{2.459459in}}{\pgfqpoint{0.479924in}{2.465283in}}%
\pgfpathcurveto{\pgfqpoint{0.485748in}{2.471107in}}{\pgfqpoint{0.489020in}{2.479007in}}{\pgfqpoint{0.489020in}{2.487243in}}%
\pgfpathcurveto{\pgfqpoint{0.489020in}{2.495479in}}{\pgfqpoint{0.485748in}{2.503379in}}{\pgfqpoint{0.479924in}{2.509203in}}%
\pgfpathcurveto{\pgfqpoint{0.474100in}{2.515027in}}{\pgfqpoint{0.466200in}{2.518300in}}{\pgfqpoint{0.457963in}{2.518300in}}%
\pgfpathcurveto{\pgfqpoint{0.449727in}{2.518300in}}{\pgfqpoint{0.441827in}{2.515027in}}{\pgfqpoint{0.436003in}{2.509203in}}%
\pgfpathcurveto{\pgfqpoint{0.430179in}{2.503379in}}{\pgfqpoint{0.426907in}{2.495479in}}{\pgfqpoint{0.426907in}{2.487243in}}%
\pgfpathcurveto{\pgfqpoint{0.426907in}{2.479007in}}{\pgfqpoint{0.430179in}{2.471107in}}{\pgfqpoint{0.436003in}{2.465283in}}%
\pgfpathcurveto{\pgfqpoint{0.441827in}{2.459459in}}{\pgfqpoint{0.449727in}{2.456187in}}{\pgfqpoint{0.457963in}{2.456187in}}%
\pgfpathclose%
\pgfusepath{stroke,fill}%
\end{pgfscope}%
\begin{pgfscope}%
\pgfpathrectangle{\pgfqpoint{0.457963in}{0.528059in}}{\pgfqpoint{6.200000in}{2.285714in}} %
\pgfusepath{clip}%
\pgfsetbuttcap%
\pgfsetroundjoin%
\definecolor{currentfill}{rgb}{0.000000,0.000000,1.000000}%
\pgfsetfillcolor{currentfill}%
\pgfsetlinewidth{1.003750pt}%
\definecolor{currentstroke}{rgb}{0.000000,0.000000,1.000000}%
\pgfsetstrokecolor{currentstroke}%
\pgfsetdash{}{0pt}%
\pgfpathmoveto{\pgfqpoint{0.457963in}{2.456187in}}%
\pgfpathcurveto{\pgfqpoint{0.466200in}{2.456187in}}{\pgfqpoint{0.474100in}{2.459459in}}{\pgfqpoint{0.479924in}{2.465283in}}%
\pgfpathcurveto{\pgfqpoint{0.485748in}{2.471107in}}{\pgfqpoint{0.489020in}{2.479007in}}{\pgfqpoint{0.489020in}{2.487243in}}%
\pgfpathcurveto{\pgfqpoint{0.489020in}{2.495479in}}{\pgfqpoint{0.485748in}{2.503379in}}{\pgfqpoint{0.479924in}{2.509203in}}%
\pgfpathcurveto{\pgfqpoint{0.474100in}{2.515027in}}{\pgfqpoint{0.466200in}{2.518300in}}{\pgfqpoint{0.457963in}{2.518300in}}%
\pgfpathcurveto{\pgfqpoint{0.449727in}{2.518300in}}{\pgfqpoint{0.441827in}{2.515027in}}{\pgfqpoint{0.436003in}{2.509203in}}%
\pgfpathcurveto{\pgfqpoint{0.430179in}{2.503379in}}{\pgfqpoint{0.426907in}{2.495479in}}{\pgfqpoint{0.426907in}{2.487243in}}%
\pgfpathcurveto{\pgfqpoint{0.426907in}{2.479007in}}{\pgfqpoint{0.430179in}{2.471107in}}{\pgfqpoint{0.436003in}{2.465283in}}%
\pgfpathcurveto{\pgfqpoint{0.441827in}{2.459459in}}{\pgfqpoint{0.449727in}{2.456187in}}{\pgfqpoint{0.457963in}{2.456187in}}%
\pgfpathclose%
\pgfusepath{stroke,fill}%
\end{pgfscope}%
\begin{pgfscope}%
\pgfpathrectangle{\pgfqpoint{0.457963in}{0.528059in}}{\pgfqpoint{6.200000in}{2.285714in}} %
\pgfusepath{clip}%
\pgfsetbuttcap%
\pgfsetroundjoin%
\definecolor{currentfill}{rgb}{0.000000,0.000000,1.000000}%
\pgfsetfillcolor{currentfill}%
\pgfsetlinewidth{1.003750pt}%
\definecolor{currentstroke}{rgb}{0.000000,0.000000,1.000000}%
\pgfsetstrokecolor{currentstroke}%
\pgfsetdash{}{0pt}%
\pgfpathmoveto{\pgfqpoint{0.478630in}{2.456187in}}%
\pgfpathcurveto{\pgfqpoint{0.486866in}{2.456187in}}{\pgfqpoint{0.494766in}{2.459459in}}{\pgfqpoint{0.500590in}{2.465283in}}%
\pgfpathcurveto{\pgfqpoint{0.506414in}{2.471107in}}{\pgfqpoint{0.509686in}{2.479007in}}{\pgfqpoint{0.509686in}{2.487243in}}%
\pgfpathcurveto{\pgfqpoint{0.509686in}{2.495479in}}{\pgfqpoint{0.506414in}{2.503379in}}{\pgfqpoint{0.500590in}{2.509203in}}%
\pgfpathcurveto{\pgfqpoint{0.494766in}{2.515027in}}{\pgfqpoint{0.486866in}{2.518300in}}{\pgfqpoint{0.478630in}{2.518300in}}%
\pgfpathcurveto{\pgfqpoint{0.470394in}{2.518300in}}{\pgfqpoint{0.462494in}{2.515027in}}{\pgfqpoint{0.456670in}{2.509203in}}%
\pgfpathcurveto{\pgfqpoint{0.450846in}{2.503379in}}{\pgfqpoint{0.447574in}{2.495479in}}{\pgfqpoint{0.447574in}{2.487243in}}%
\pgfpathcurveto{\pgfqpoint{0.447574in}{2.479007in}}{\pgfqpoint{0.450846in}{2.471107in}}{\pgfqpoint{0.456670in}{2.465283in}}%
\pgfpathcurveto{\pgfqpoint{0.462494in}{2.459459in}}{\pgfqpoint{0.470394in}{2.456187in}}{\pgfqpoint{0.478630in}{2.456187in}}%
\pgfpathclose%
\pgfusepath{stroke,fill}%
\end{pgfscope}%
\begin{pgfscope}%
\pgfpathrectangle{\pgfqpoint{0.457963in}{0.528059in}}{\pgfqpoint{6.200000in}{2.285714in}} %
\pgfusepath{clip}%
\pgfsetbuttcap%
\pgfsetroundjoin%
\definecolor{currentfill}{rgb}{0.000000,0.000000,1.000000}%
\pgfsetfillcolor{currentfill}%
\pgfsetlinewidth{1.003750pt}%
\definecolor{currentstroke}{rgb}{0.000000,0.000000,1.000000}%
\pgfsetstrokecolor{currentstroke}%
\pgfsetdash{}{0pt}%
\pgfpathmoveto{\pgfqpoint{0.519963in}{2.456187in}}%
\pgfpathcurveto{\pgfqpoint{0.528200in}{2.456187in}}{\pgfqpoint{0.536100in}{2.459459in}}{\pgfqpoint{0.541924in}{2.465283in}}%
\pgfpathcurveto{\pgfqpoint{0.547748in}{2.471107in}}{\pgfqpoint{0.551020in}{2.479007in}}{\pgfqpoint{0.551020in}{2.487243in}}%
\pgfpathcurveto{\pgfqpoint{0.551020in}{2.495479in}}{\pgfqpoint{0.547748in}{2.503379in}}{\pgfqpoint{0.541924in}{2.509203in}}%
\pgfpathcurveto{\pgfqpoint{0.536100in}{2.515027in}}{\pgfqpoint{0.528200in}{2.518300in}}{\pgfqpoint{0.519963in}{2.518300in}}%
\pgfpathcurveto{\pgfqpoint{0.511727in}{2.518300in}}{\pgfqpoint{0.503827in}{2.515027in}}{\pgfqpoint{0.498003in}{2.509203in}}%
\pgfpathcurveto{\pgfqpoint{0.492179in}{2.503379in}}{\pgfqpoint{0.488907in}{2.495479in}}{\pgfqpoint{0.488907in}{2.487243in}}%
\pgfpathcurveto{\pgfqpoint{0.488907in}{2.479007in}}{\pgfqpoint{0.492179in}{2.471107in}}{\pgfqpoint{0.498003in}{2.465283in}}%
\pgfpathcurveto{\pgfqpoint{0.503827in}{2.459459in}}{\pgfqpoint{0.511727in}{2.456187in}}{\pgfqpoint{0.519963in}{2.456187in}}%
\pgfpathclose%
\pgfusepath{stroke,fill}%
\end{pgfscope}%
\begin{pgfscope}%
\pgfpathrectangle{\pgfqpoint{0.457963in}{0.528059in}}{\pgfqpoint{6.200000in}{2.285714in}} %
\pgfusepath{clip}%
\pgfsetbuttcap%
\pgfsetroundjoin%
\definecolor{currentfill}{rgb}{0.000000,0.000000,1.000000}%
\pgfsetfillcolor{currentfill}%
\pgfsetlinewidth{1.003750pt}%
\definecolor{currentstroke}{rgb}{0.000000,0.000000,1.000000}%
\pgfsetstrokecolor{currentstroke}%
\pgfsetdash{}{0pt}%
\pgfpathmoveto{\pgfqpoint{0.623297in}{2.443125in}}%
\pgfpathcurveto{\pgfqpoint{0.631533in}{2.443125in}}{\pgfqpoint{0.639433in}{2.446398in}}{\pgfqpoint{0.645257in}{2.452222in}}%
\pgfpathcurveto{\pgfqpoint{0.651081in}{2.458046in}}{\pgfqpoint{0.654353in}{2.465946in}}{\pgfqpoint{0.654353in}{2.474182in}}%
\pgfpathcurveto{\pgfqpoint{0.654353in}{2.482418in}}{\pgfqpoint{0.651081in}{2.490318in}}{\pgfqpoint{0.645257in}{2.496142in}}%
\pgfpathcurveto{\pgfqpoint{0.639433in}{2.501966in}}{\pgfqpoint{0.631533in}{2.505238in}}{\pgfqpoint{0.623297in}{2.505238in}}%
\pgfpathcurveto{\pgfqpoint{0.615060in}{2.505238in}}{\pgfqpoint{0.607160in}{2.501966in}}{\pgfqpoint{0.601336in}{2.496142in}}%
\pgfpathcurveto{\pgfqpoint{0.595512in}{2.490318in}}{\pgfqpoint{0.592240in}{2.482418in}}{\pgfqpoint{0.592240in}{2.474182in}}%
\pgfpathcurveto{\pgfqpoint{0.592240in}{2.465946in}}{\pgfqpoint{0.595512in}{2.458046in}}{\pgfqpoint{0.601336in}{2.452222in}}%
\pgfpathcurveto{\pgfqpoint{0.607160in}{2.446398in}}{\pgfqpoint{0.615060in}{2.443125in}}{\pgfqpoint{0.623297in}{2.443125in}}%
\pgfpathclose%
\pgfusepath{stroke,fill}%
\end{pgfscope}%
\begin{pgfscope}%
\pgfpathrectangle{\pgfqpoint{0.457963in}{0.528059in}}{\pgfqpoint{6.200000in}{2.285714in}} %
\pgfusepath{clip}%
\pgfsetbuttcap%
\pgfsetroundjoin%
\definecolor{currentfill}{rgb}{0.000000,0.000000,1.000000}%
\pgfsetfillcolor{currentfill}%
\pgfsetlinewidth{1.003750pt}%
\definecolor{currentstroke}{rgb}{0.000000,0.000000,1.000000}%
\pgfsetstrokecolor{currentstroke}%
\pgfsetdash{}{0pt}%
\pgfpathmoveto{\pgfqpoint{0.716297in}{2.430064in}}%
\pgfpathcurveto{\pgfqpoint{0.724533in}{2.430064in}}{\pgfqpoint{0.732433in}{2.433336in}}{\pgfqpoint{0.738257in}{2.439160in}}%
\pgfpathcurveto{\pgfqpoint{0.744081in}{2.444984in}}{\pgfqpoint{0.747353in}{2.452884in}}{\pgfqpoint{0.747353in}{2.461121in}}%
\pgfpathcurveto{\pgfqpoint{0.747353in}{2.469357in}}{\pgfqpoint{0.744081in}{2.477257in}}{\pgfqpoint{0.738257in}{2.483081in}}%
\pgfpathcurveto{\pgfqpoint{0.732433in}{2.488905in}}{\pgfqpoint{0.724533in}{2.492177in}}{\pgfqpoint{0.716297in}{2.492177in}}%
\pgfpathcurveto{\pgfqpoint{0.708060in}{2.492177in}}{\pgfqpoint{0.700160in}{2.488905in}}{\pgfqpoint{0.694336in}{2.483081in}}%
\pgfpathcurveto{\pgfqpoint{0.688512in}{2.477257in}}{\pgfqpoint{0.685240in}{2.469357in}}{\pgfqpoint{0.685240in}{2.461121in}}%
\pgfpathcurveto{\pgfqpoint{0.685240in}{2.452884in}}{\pgfqpoint{0.688512in}{2.444984in}}{\pgfqpoint{0.694336in}{2.439160in}}%
\pgfpathcurveto{\pgfqpoint{0.700160in}{2.433336in}}{\pgfqpoint{0.708060in}{2.430064in}}{\pgfqpoint{0.716297in}{2.430064in}}%
\pgfpathclose%
\pgfusepath{stroke,fill}%
\end{pgfscope}%
\begin{pgfscope}%
\pgfpathrectangle{\pgfqpoint{0.457963in}{0.528059in}}{\pgfqpoint{6.200000in}{2.285714in}} %
\pgfusepath{clip}%
\pgfsetbuttcap%
\pgfsetroundjoin%
\definecolor{currentfill}{rgb}{0.000000,0.000000,1.000000}%
\pgfsetfillcolor{currentfill}%
\pgfsetlinewidth{1.003750pt}%
\definecolor{currentstroke}{rgb}{0.000000,0.000000,1.000000}%
\pgfsetstrokecolor{currentstroke}%
\pgfsetdash{}{0pt}%
\pgfpathmoveto{\pgfqpoint{0.747297in}{2.456187in}}%
\pgfpathcurveto{\pgfqpoint{0.755533in}{2.456187in}}{\pgfqpoint{0.763433in}{2.459459in}}{\pgfqpoint{0.769257in}{2.465283in}}%
\pgfpathcurveto{\pgfqpoint{0.775081in}{2.471107in}}{\pgfqpoint{0.778353in}{2.479007in}}{\pgfqpoint{0.778353in}{2.487243in}}%
\pgfpathcurveto{\pgfqpoint{0.778353in}{2.495479in}}{\pgfqpoint{0.775081in}{2.503379in}}{\pgfqpoint{0.769257in}{2.509203in}}%
\pgfpathcurveto{\pgfqpoint{0.763433in}{2.515027in}}{\pgfqpoint{0.755533in}{2.518300in}}{\pgfqpoint{0.747297in}{2.518300in}}%
\pgfpathcurveto{\pgfqpoint{0.739060in}{2.518300in}}{\pgfqpoint{0.731160in}{2.515027in}}{\pgfqpoint{0.725336in}{2.509203in}}%
\pgfpathcurveto{\pgfqpoint{0.719512in}{2.503379in}}{\pgfqpoint{0.716240in}{2.495479in}}{\pgfqpoint{0.716240in}{2.487243in}}%
\pgfpathcurveto{\pgfqpoint{0.716240in}{2.479007in}}{\pgfqpoint{0.719512in}{2.471107in}}{\pgfqpoint{0.725336in}{2.465283in}}%
\pgfpathcurveto{\pgfqpoint{0.731160in}{2.459459in}}{\pgfqpoint{0.739060in}{2.456187in}}{\pgfqpoint{0.747297in}{2.456187in}}%
\pgfpathclose%
\pgfusepath{stroke,fill}%
\end{pgfscope}%
\begin{pgfscope}%
\pgfpathrectangle{\pgfqpoint{0.457963in}{0.528059in}}{\pgfqpoint{6.200000in}{2.285714in}} %
\pgfusepath{clip}%
\pgfsetbuttcap%
\pgfsetroundjoin%
\definecolor{currentfill}{rgb}{0.000000,0.000000,1.000000}%
\pgfsetfillcolor{currentfill}%
\pgfsetlinewidth{1.003750pt}%
\definecolor{currentstroke}{rgb}{0.000000,0.000000,1.000000}%
\pgfsetstrokecolor{currentstroke}%
\pgfsetdash{}{0pt}%
\pgfpathmoveto{\pgfqpoint{0.902297in}{2.377819in}}%
\pgfpathcurveto{\pgfqpoint{0.910533in}{2.377819in}}{\pgfqpoint{0.918433in}{2.381092in}}{\pgfqpoint{0.924257in}{2.386916in}}%
\pgfpathcurveto{\pgfqpoint{0.930081in}{2.392739in}}{\pgfqpoint{0.933353in}{2.400639in}}{\pgfqpoint{0.933353in}{2.408876in}}%
\pgfpathcurveto{\pgfqpoint{0.933353in}{2.417112in}}{\pgfqpoint{0.930081in}{2.425012in}}{\pgfqpoint{0.924257in}{2.430836in}}%
\pgfpathcurveto{\pgfqpoint{0.918433in}{2.436660in}}{\pgfqpoint{0.910533in}{2.439932in}}{\pgfqpoint{0.902297in}{2.439932in}}%
\pgfpathcurveto{\pgfqpoint{0.894060in}{2.439932in}}{\pgfqpoint{0.886160in}{2.436660in}}{\pgfqpoint{0.880336in}{2.430836in}}%
\pgfpathcurveto{\pgfqpoint{0.874512in}{2.425012in}}{\pgfqpoint{0.871240in}{2.417112in}}{\pgfqpoint{0.871240in}{2.408876in}}%
\pgfpathcurveto{\pgfqpoint{0.871240in}{2.400639in}}{\pgfqpoint{0.874512in}{2.392739in}}{\pgfqpoint{0.880336in}{2.386916in}}%
\pgfpathcurveto{\pgfqpoint{0.886160in}{2.381092in}}{\pgfqpoint{0.894060in}{2.377819in}}{\pgfqpoint{0.902297in}{2.377819in}}%
\pgfpathclose%
\pgfusepath{stroke,fill}%
\end{pgfscope}%
\begin{pgfscope}%
\pgfpathrectangle{\pgfqpoint{0.457963in}{0.528059in}}{\pgfqpoint{6.200000in}{2.285714in}} %
\pgfusepath{clip}%
\pgfsetbuttcap%
\pgfsetroundjoin%
\definecolor{currentfill}{rgb}{0.000000,0.000000,1.000000}%
\pgfsetfillcolor{currentfill}%
\pgfsetlinewidth{1.003750pt}%
\definecolor{currentstroke}{rgb}{0.000000,0.000000,1.000000}%
\pgfsetstrokecolor{currentstroke}%
\pgfsetdash{}{0pt}%
\pgfpathmoveto{\pgfqpoint{1.181297in}{2.443125in}}%
\pgfpathcurveto{\pgfqpoint{1.189533in}{2.443125in}}{\pgfqpoint{1.197433in}{2.446398in}}{\pgfqpoint{1.203257in}{2.452222in}}%
\pgfpathcurveto{\pgfqpoint{1.209081in}{2.458046in}}{\pgfqpoint{1.212353in}{2.465946in}}{\pgfqpoint{1.212353in}{2.474182in}}%
\pgfpathcurveto{\pgfqpoint{1.212353in}{2.482418in}}{\pgfqpoint{1.209081in}{2.490318in}}{\pgfqpoint{1.203257in}{2.496142in}}%
\pgfpathcurveto{\pgfqpoint{1.197433in}{2.501966in}}{\pgfqpoint{1.189533in}{2.505238in}}{\pgfqpoint{1.181297in}{2.505238in}}%
\pgfpathcurveto{\pgfqpoint{1.173060in}{2.505238in}}{\pgfqpoint{1.165160in}{2.501966in}}{\pgfqpoint{1.159336in}{2.496142in}}%
\pgfpathcurveto{\pgfqpoint{1.153512in}{2.490318in}}{\pgfqpoint{1.150240in}{2.482418in}}{\pgfqpoint{1.150240in}{2.474182in}}%
\pgfpathcurveto{\pgfqpoint{1.150240in}{2.465946in}}{\pgfqpoint{1.153512in}{2.458046in}}{\pgfqpoint{1.159336in}{2.452222in}}%
\pgfpathcurveto{\pgfqpoint{1.165160in}{2.446398in}}{\pgfqpoint{1.173060in}{2.443125in}}{\pgfqpoint{1.181297in}{2.443125in}}%
\pgfpathclose%
\pgfusepath{stroke,fill}%
\end{pgfscope}%
\begin{pgfscope}%
\pgfpathrectangle{\pgfqpoint{0.457963in}{0.528059in}}{\pgfqpoint{6.200000in}{2.285714in}} %
\pgfusepath{clip}%
\pgfsetbuttcap%
\pgfsetroundjoin%
\definecolor{currentfill}{rgb}{0.000000,0.000000,1.000000}%
\pgfsetfillcolor{currentfill}%
\pgfsetlinewidth{1.003750pt}%
\definecolor{currentstroke}{rgb}{0.000000,0.000000,1.000000}%
\pgfsetstrokecolor{currentstroke}%
\pgfsetdash{}{0pt}%
\pgfpathmoveto{\pgfqpoint{1.429297in}{2.364758in}}%
\pgfpathcurveto{\pgfqpoint{1.437533in}{2.364758in}}{\pgfqpoint{1.445433in}{2.368030in}}{\pgfqpoint{1.451257in}{2.373854in}}%
\pgfpathcurveto{\pgfqpoint{1.457081in}{2.379678in}}{\pgfqpoint{1.460353in}{2.387578in}}{\pgfqpoint{1.460353in}{2.395815in}}%
\pgfpathcurveto{\pgfqpoint{1.460353in}{2.404051in}}{\pgfqpoint{1.457081in}{2.411951in}}{\pgfqpoint{1.451257in}{2.417775in}}%
\pgfpathcurveto{\pgfqpoint{1.445433in}{2.423599in}}{\pgfqpoint{1.437533in}{2.426871in}}{\pgfqpoint{1.429297in}{2.426871in}}%
\pgfpathcurveto{\pgfqpoint{1.421060in}{2.426871in}}{\pgfqpoint{1.413160in}{2.423599in}}{\pgfqpoint{1.407336in}{2.417775in}}%
\pgfpathcurveto{\pgfqpoint{1.401512in}{2.411951in}}{\pgfqpoint{1.398240in}{2.404051in}}{\pgfqpoint{1.398240in}{2.395815in}}%
\pgfpathcurveto{\pgfqpoint{1.398240in}{2.387578in}}{\pgfqpoint{1.401512in}{2.379678in}}{\pgfqpoint{1.407336in}{2.373854in}}%
\pgfpathcurveto{\pgfqpoint{1.413160in}{2.368030in}}{\pgfqpoint{1.421060in}{2.364758in}}{\pgfqpoint{1.429297in}{2.364758in}}%
\pgfpathclose%
\pgfusepath{stroke,fill}%
\end{pgfscope}%
\begin{pgfscope}%
\pgfpathrectangle{\pgfqpoint{0.457963in}{0.528059in}}{\pgfqpoint{6.200000in}{2.285714in}} %
\pgfusepath{clip}%
\pgfsetbuttcap%
\pgfsetroundjoin%
\definecolor{currentfill}{rgb}{0.000000,0.000000,1.000000}%
\pgfsetfillcolor{currentfill}%
\pgfsetlinewidth{1.003750pt}%
\definecolor{currentstroke}{rgb}{0.000000,0.000000,1.000000}%
\pgfsetstrokecolor{currentstroke}%
\pgfsetdash{}{0pt}%
\pgfpathmoveto{\pgfqpoint{1.429297in}{2.456187in}}%
\pgfpathcurveto{\pgfqpoint{1.437533in}{2.456187in}}{\pgfqpoint{1.445433in}{2.459459in}}{\pgfqpoint{1.451257in}{2.465283in}}%
\pgfpathcurveto{\pgfqpoint{1.457081in}{2.471107in}}{\pgfqpoint{1.460353in}{2.479007in}}{\pgfqpoint{1.460353in}{2.487243in}}%
\pgfpathcurveto{\pgfqpoint{1.460353in}{2.495479in}}{\pgfqpoint{1.457081in}{2.503379in}}{\pgfqpoint{1.451257in}{2.509203in}}%
\pgfpathcurveto{\pgfqpoint{1.445433in}{2.515027in}}{\pgfqpoint{1.437533in}{2.518300in}}{\pgfqpoint{1.429297in}{2.518300in}}%
\pgfpathcurveto{\pgfqpoint{1.421060in}{2.518300in}}{\pgfqpoint{1.413160in}{2.515027in}}{\pgfqpoint{1.407336in}{2.509203in}}%
\pgfpathcurveto{\pgfqpoint{1.401512in}{2.503379in}}{\pgfqpoint{1.398240in}{2.495479in}}{\pgfqpoint{1.398240in}{2.487243in}}%
\pgfpathcurveto{\pgfqpoint{1.398240in}{2.479007in}}{\pgfqpoint{1.401512in}{2.471107in}}{\pgfqpoint{1.407336in}{2.465283in}}%
\pgfpathcurveto{\pgfqpoint{1.413160in}{2.459459in}}{\pgfqpoint{1.421060in}{2.456187in}}{\pgfqpoint{1.429297in}{2.456187in}}%
\pgfpathclose%
\pgfusepath{stroke,fill}%
\end{pgfscope}%
\begin{pgfscope}%
\pgfpathrectangle{\pgfqpoint{0.457963in}{0.528059in}}{\pgfqpoint{6.200000in}{2.285714in}} %
\pgfusepath{clip}%
\pgfsetbuttcap%
\pgfsetroundjoin%
\definecolor{currentfill}{rgb}{0.000000,0.000000,1.000000}%
\pgfsetfillcolor{currentfill}%
\pgfsetlinewidth{1.003750pt}%
\definecolor{currentstroke}{rgb}{0.000000,0.000000,1.000000}%
\pgfsetstrokecolor{currentstroke}%
\pgfsetdash{}{0pt}%
\pgfpathmoveto{\pgfqpoint{1.687630in}{2.443125in}}%
\pgfpathcurveto{\pgfqpoint{1.695866in}{2.443125in}}{\pgfqpoint{1.703766in}{2.446398in}}{\pgfqpoint{1.709590in}{2.452222in}}%
\pgfpathcurveto{\pgfqpoint{1.715414in}{2.458046in}}{\pgfqpoint{1.718686in}{2.465946in}}{\pgfqpoint{1.718686in}{2.474182in}}%
\pgfpathcurveto{\pgfqpoint{1.718686in}{2.482418in}}{\pgfqpoint{1.715414in}{2.490318in}}{\pgfqpoint{1.709590in}{2.496142in}}%
\pgfpathcurveto{\pgfqpoint{1.703766in}{2.501966in}}{\pgfqpoint{1.695866in}{2.505238in}}{\pgfqpoint{1.687630in}{2.505238in}}%
\pgfpathcurveto{\pgfqpoint{1.679394in}{2.505238in}}{\pgfqpoint{1.671494in}{2.501966in}}{\pgfqpoint{1.665670in}{2.496142in}}%
\pgfpathcurveto{\pgfqpoint{1.659846in}{2.490318in}}{\pgfqpoint{1.656574in}{2.482418in}}{\pgfqpoint{1.656574in}{2.474182in}}%
\pgfpathcurveto{\pgfqpoint{1.656574in}{2.465946in}}{\pgfqpoint{1.659846in}{2.458046in}}{\pgfqpoint{1.665670in}{2.452222in}}%
\pgfpathcurveto{\pgfqpoint{1.671494in}{2.446398in}}{\pgfqpoint{1.679394in}{2.443125in}}{\pgfqpoint{1.687630in}{2.443125in}}%
\pgfpathclose%
\pgfusepath{stroke,fill}%
\end{pgfscope}%
\begin{pgfscope}%
\pgfpathrectangle{\pgfqpoint{0.457963in}{0.528059in}}{\pgfqpoint{6.200000in}{2.285714in}} %
\pgfusepath{clip}%
\pgfsetbuttcap%
\pgfsetroundjoin%
\definecolor{currentfill}{rgb}{0.000000,0.000000,1.000000}%
\pgfsetfillcolor{currentfill}%
\pgfsetlinewidth{1.003750pt}%
\definecolor{currentstroke}{rgb}{0.000000,0.000000,1.000000}%
\pgfsetstrokecolor{currentstroke}%
\pgfsetdash{}{0pt}%
\pgfpathmoveto{\pgfqpoint{3.113630in}{2.116595in}}%
\pgfpathcurveto{\pgfqpoint{3.121866in}{2.116595in}}{\pgfqpoint{3.129766in}{2.119867in}}{\pgfqpoint{3.135590in}{2.125691in}}%
\pgfpathcurveto{\pgfqpoint{3.141414in}{2.131515in}}{\pgfqpoint{3.144686in}{2.139415in}}{\pgfqpoint{3.144686in}{2.147651in}}%
\pgfpathcurveto{\pgfqpoint{3.144686in}{2.155888in}}{\pgfqpoint{3.141414in}{2.163788in}}{\pgfqpoint{3.135590in}{2.169612in}}%
\pgfpathcurveto{\pgfqpoint{3.129766in}{2.175435in}}{\pgfqpoint{3.121866in}{2.178708in}}{\pgfqpoint{3.113630in}{2.178708in}}%
\pgfpathcurveto{\pgfqpoint{3.105394in}{2.178708in}}{\pgfqpoint{3.097494in}{2.175435in}}{\pgfqpoint{3.091670in}{2.169612in}}%
\pgfpathcurveto{\pgfqpoint{3.085846in}{2.163788in}}{\pgfqpoint{3.082574in}{2.155888in}}{\pgfqpoint{3.082574in}{2.147651in}}%
\pgfpathcurveto{\pgfqpoint{3.082574in}{2.139415in}}{\pgfqpoint{3.085846in}{2.131515in}}{\pgfqpoint{3.091670in}{2.125691in}}%
\pgfpathcurveto{\pgfqpoint{3.097494in}{2.119867in}}{\pgfqpoint{3.105394in}{2.116595in}}{\pgfqpoint{3.113630in}{2.116595in}}%
\pgfpathclose%
\pgfusepath{stroke,fill}%
\end{pgfscope}%
\begin{pgfscope}%
\pgfpathrectangle{\pgfqpoint{0.457963in}{0.528059in}}{\pgfqpoint{6.200000in}{2.285714in}} %
\pgfusepath{clip}%
\pgfsetbuttcap%
\pgfsetroundjoin%
\definecolor{currentfill}{rgb}{0.000000,0.000000,1.000000}%
\pgfsetfillcolor{currentfill}%
\pgfsetlinewidth{1.003750pt}%
\definecolor{currentstroke}{rgb}{0.000000,0.000000,1.000000}%
\pgfsetstrokecolor{currentstroke}%
\pgfsetdash{}{0pt}%
\pgfpathmoveto{\pgfqpoint{3.526963in}{2.247207in}}%
\pgfpathcurveto{\pgfqpoint{3.535200in}{2.247207in}}{\pgfqpoint{3.543100in}{2.250479in}}{\pgfqpoint{3.548924in}{2.256303in}}%
\pgfpathcurveto{\pgfqpoint{3.554748in}{2.262127in}}{\pgfqpoint{3.558020in}{2.270027in}}{\pgfqpoint{3.558020in}{2.278264in}}%
\pgfpathcurveto{\pgfqpoint{3.558020in}{2.286500in}}{\pgfqpoint{3.554748in}{2.294400in}}{\pgfqpoint{3.548924in}{2.300224in}}%
\pgfpathcurveto{\pgfqpoint{3.543100in}{2.306048in}}{\pgfqpoint{3.535200in}{2.309320in}}{\pgfqpoint{3.526963in}{2.309320in}}%
\pgfpathcurveto{\pgfqpoint{3.518727in}{2.309320in}}{\pgfqpoint{3.510827in}{2.306048in}}{\pgfqpoint{3.505003in}{2.300224in}}%
\pgfpathcurveto{\pgfqpoint{3.499179in}{2.294400in}}{\pgfqpoint{3.495907in}{2.286500in}}{\pgfqpoint{3.495907in}{2.278264in}}%
\pgfpathcurveto{\pgfqpoint{3.495907in}{2.270027in}}{\pgfqpoint{3.499179in}{2.262127in}}{\pgfqpoint{3.505003in}{2.256303in}}%
\pgfpathcurveto{\pgfqpoint{3.510827in}{2.250479in}}{\pgfqpoint{3.518727in}{2.247207in}}{\pgfqpoint{3.526963in}{2.247207in}}%
\pgfpathclose%
\pgfusepath{stroke,fill}%
\end{pgfscope}%
\begin{pgfscope}%
\pgfpathrectangle{\pgfqpoint{0.457963in}{0.528059in}}{\pgfqpoint{6.200000in}{2.285714in}} %
\pgfusepath{clip}%
\pgfsetbuttcap%
\pgfsetroundjoin%
\definecolor{currentfill}{rgb}{0.000000,0.000000,1.000000}%
\pgfsetfillcolor{currentfill}%
\pgfsetlinewidth{1.003750pt}%
\definecolor{currentstroke}{rgb}{0.000000,0.000000,1.000000}%
\pgfsetstrokecolor{currentstroke}%
\pgfsetdash{}{0pt}%
\pgfpathmoveto{\pgfqpoint{3.712963in}{1.933738in}}%
\pgfpathcurveto{\pgfqpoint{3.721200in}{1.933738in}}{\pgfqpoint{3.729100in}{1.937010in}}{\pgfqpoint{3.734924in}{1.942834in}}%
\pgfpathcurveto{\pgfqpoint{3.740748in}{1.948658in}}{\pgfqpoint{3.744020in}{1.956558in}}{\pgfqpoint{3.744020in}{1.964794in}}%
\pgfpathcurveto{\pgfqpoint{3.744020in}{1.973030in}}{\pgfqpoint{3.740748in}{1.980930in}}{\pgfqpoint{3.734924in}{1.986754in}}%
\pgfpathcurveto{\pgfqpoint{3.729100in}{1.992578in}}{\pgfqpoint{3.721200in}{1.995851in}}{\pgfqpoint{3.712963in}{1.995851in}}%
\pgfpathcurveto{\pgfqpoint{3.704727in}{1.995851in}}{\pgfqpoint{3.696827in}{1.992578in}}{\pgfqpoint{3.691003in}{1.986754in}}%
\pgfpathcurveto{\pgfqpoint{3.685179in}{1.980930in}}{\pgfqpoint{3.681907in}{1.973030in}}{\pgfqpoint{3.681907in}{1.964794in}}%
\pgfpathcurveto{\pgfqpoint{3.681907in}{1.956558in}}{\pgfqpoint{3.685179in}{1.948658in}}{\pgfqpoint{3.691003in}{1.942834in}}%
\pgfpathcurveto{\pgfqpoint{3.696827in}{1.937010in}}{\pgfqpoint{3.704727in}{1.933738in}}{\pgfqpoint{3.712963in}{1.933738in}}%
\pgfpathclose%
\pgfusepath{stroke,fill}%
\end{pgfscope}%
\begin{pgfscope}%
\pgfpathrectangle{\pgfqpoint{0.457963in}{0.528059in}}{\pgfqpoint{6.200000in}{2.285714in}} %
\pgfusepath{clip}%
\pgfsetbuttcap%
\pgfsetroundjoin%
\definecolor{currentfill}{rgb}{0.000000,0.000000,1.000000}%
\pgfsetfillcolor{currentfill}%
\pgfsetlinewidth{1.003750pt}%
\definecolor{currentstroke}{rgb}{0.000000,0.000000,1.000000}%
\pgfsetstrokecolor{currentstroke}%
\pgfsetdash{}{0pt}%
\pgfpathmoveto{\pgfqpoint{4.818630in}{1.763942in}}%
\pgfpathcurveto{\pgfqpoint{4.826866in}{1.763942in}}{\pgfqpoint{4.834766in}{1.767214in}}{\pgfqpoint{4.840590in}{1.773038in}}%
\pgfpathcurveto{\pgfqpoint{4.846414in}{1.778862in}}{\pgfqpoint{4.849686in}{1.786762in}}{\pgfqpoint{4.849686in}{1.794998in}}%
\pgfpathcurveto{\pgfqpoint{4.849686in}{1.803234in}}{\pgfqpoint{4.846414in}{1.811135in}}{\pgfqpoint{4.840590in}{1.816958in}}%
\pgfpathcurveto{\pgfqpoint{4.834766in}{1.822782in}}{\pgfqpoint{4.826866in}{1.826055in}}{\pgfqpoint{4.818630in}{1.826055in}}%
\pgfpathcurveto{\pgfqpoint{4.810394in}{1.826055in}}{\pgfqpoint{4.802494in}{1.822782in}}{\pgfqpoint{4.796670in}{1.816958in}}%
\pgfpathcurveto{\pgfqpoint{4.790846in}{1.811135in}}{\pgfqpoint{4.787574in}{1.803234in}}{\pgfqpoint{4.787574in}{1.794998in}}%
\pgfpathcurveto{\pgfqpoint{4.787574in}{1.786762in}}{\pgfqpoint{4.790846in}{1.778862in}}{\pgfqpoint{4.796670in}{1.773038in}}%
\pgfpathcurveto{\pgfqpoint{4.802494in}{1.767214in}}{\pgfqpoint{4.810394in}{1.763942in}}{\pgfqpoint{4.818630in}{1.763942in}}%
\pgfpathclose%
\pgfusepath{stroke,fill}%
\end{pgfscope}%
\begin{pgfscope}%
\pgfpathrectangle{\pgfqpoint{0.457963in}{0.528059in}}{\pgfqpoint{6.200000in}{2.285714in}} %
\pgfusepath{clip}%
\pgfsetbuttcap%
\pgfsetroundjoin%
\definecolor{currentfill}{rgb}{0.000000,0.000000,1.000000}%
\pgfsetfillcolor{currentfill}%
\pgfsetlinewidth{1.003750pt}%
\definecolor{currentstroke}{rgb}{0.000000,0.000000,1.000000}%
\pgfsetstrokecolor{currentstroke}%
\pgfsetdash{}{0pt}%
\pgfpathmoveto{\pgfqpoint{4.921963in}{1.306799in}}%
\pgfpathcurveto{\pgfqpoint{4.930200in}{1.306799in}}{\pgfqpoint{4.938100in}{1.310071in}}{\pgfqpoint{4.943924in}{1.315895in}}%
\pgfpathcurveto{\pgfqpoint{4.949748in}{1.321719in}}{\pgfqpoint{4.953020in}{1.329619in}}{\pgfqpoint{4.953020in}{1.337855in}}%
\pgfpathcurveto{\pgfqpoint{4.953020in}{1.346092in}}{\pgfqpoint{4.949748in}{1.353992in}}{\pgfqpoint{4.943924in}{1.359816in}}%
\pgfpathcurveto{\pgfqpoint{4.938100in}{1.365640in}}{\pgfqpoint{4.930200in}{1.368912in}}{\pgfqpoint{4.921963in}{1.368912in}}%
\pgfpathcurveto{\pgfqpoint{4.913727in}{1.368912in}}{\pgfqpoint{4.905827in}{1.365640in}}{\pgfqpoint{4.900003in}{1.359816in}}%
\pgfpathcurveto{\pgfqpoint{4.894179in}{1.353992in}}{\pgfqpoint{4.890907in}{1.346092in}}{\pgfqpoint{4.890907in}{1.337855in}}%
\pgfpathcurveto{\pgfqpoint{4.890907in}{1.329619in}}{\pgfqpoint{4.894179in}{1.321719in}}{\pgfqpoint{4.900003in}{1.315895in}}%
\pgfpathcurveto{\pgfqpoint{4.905827in}{1.310071in}}{\pgfqpoint{4.913727in}{1.306799in}}{\pgfqpoint{4.921963in}{1.306799in}}%
\pgfpathclose%
\pgfusepath{stroke,fill}%
\end{pgfscope}%
\begin{pgfscope}%
\pgfpathrectangle{\pgfqpoint{0.457963in}{0.528059in}}{\pgfqpoint{6.200000in}{2.285714in}} %
\pgfusepath{clip}%
\pgfsetbuttcap%
\pgfsetroundjoin%
\definecolor{currentfill}{rgb}{0.000000,0.000000,1.000000}%
\pgfsetfillcolor{currentfill}%
\pgfsetlinewidth{1.003750pt}%
\definecolor{currentstroke}{rgb}{0.000000,0.000000,1.000000}%
\pgfsetstrokecolor{currentstroke}%
\pgfsetdash{}{0pt}%
\pgfpathmoveto{\pgfqpoint{5.014963in}{1.750880in}}%
\pgfpathcurveto{\pgfqpoint{5.023200in}{1.750880in}}{\pgfqpoint{5.031100in}{1.754153in}}{\pgfqpoint{5.036924in}{1.759977in}}%
\pgfpathcurveto{\pgfqpoint{5.042748in}{1.765801in}}{\pgfqpoint{5.046020in}{1.773701in}}{\pgfqpoint{5.046020in}{1.781937in}}%
\pgfpathcurveto{\pgfqpoint{5.046020in}{1.790173in}}{\pgfqpoint{5.042748in}{1.798073in}}{\pgfqpoint{5.036924in}{1.803897in}}%
\pgfpathcurveto{\pgfqpoint{5.031100in}{1.809721in}}{\pgfqpoint{5.023200in}{1.812993in}}{\pgfqpoint{5.014963in}{1.812993in}}%
\pgfpathcurveto{\pgfqpoint{5.006727in}{1.812993in}}{\pgfqpoint{4.998827in}{1.809721in}}{\pgfqpoint{4.993003in}{1.803897in}}%
\pgfpathcurveto{\pgfqpoint{4.987179in}{1.798073in}}{\pgfqpoint{4.983907in}{1.790173in}}{\pgfqpoint{4.983907in}{1.781937in}}%
\pgfpathcurveto{\pgfqpoint{4.983907in}{1.773701in}}{\pgfqpoint{4.987179in}{1.765801in}}{\pgfqpoint{4.993003in}{1.759977in}}%
\pgfpathcurveto{\pgfqpoint{4.998827in}{1.754153in}}{\pgfqpoint{5.006727in}{1.750880in}}{\pgfqpoint{5.014963in}{1.750880in}}%
\pgfpathclose%
\pgfusepath{stroke,fill}%
\end{pgfscope}%
\begin{pgfscope}%
\pgfpathrectangle{\pgfqpoint{0.457963in}{0.528059in}}{\pgfqpoint{6.200000in}{2.285714in}} %
\pgfusepath{clip}%
\pgfsetbuttcap%
\pgfsetroundjoin%
\definecolor{currentfill}{rgb}{0.000000,0.000000,1.000000}%
\pgfsetfillcolor{currentfill}%
\pgfsetlinewidth{1.003750pt}%
\definecolor{currentstroke}{rgb}{0.000000,0.000000,1.000000}%
\pgfsetstrokecolor{currentstroke}%
\pgfsetdash{}{0pt}%
\pgfpathmoveto{\pgfqpoint{5.469630in}{1.123942in}}%
\pgfpathcurveto{\pgfqpoint{5.477866in}{1.123942in}}{\pgfqpoint{5.485766in}{1.127214in}}{\pgfqpoint{5.491590in}{1.133038in}}%
\pgfpathcurveto{\pgfqpoint{5.497414in}{1.138862in}}{\pgfqpoint{5.500686in}{1.146762in}}{\pgfqpoint{5.500686in}{1.154998in}}%
\pgfpathcurveto{\pgfqpoint{5.500686in}{1.163234in}}{\pgfqpoint{5.497414in}{1.171135in}}{\pgfqpoint{5.491590in}{1.176958in}}%
\pgfpathcurveto{\pgfqpoint{5.485766in}{1.182782in}}{\pgfqpoint{5.477866in}{1.186055in}}{\pgfqpoint{5.469630in}{1.186055in}}%
\pgfpathcurveto{\pgfqpoint{5.461394in}{1.186055in}}{\pgfqpoint{5.453494in}{1.182782in}}{\pgfqpoint{5.447670in}{1.176958in}}%
\pgfpathcurveto{\pgfqpoint{5.441846in}{1.171135in}}{\pgfqpoint{5.438574in}{1.163234in}}{\pgfqpoint{5.438574in}{1.154998in}}%
\pgfpathcurveto{\pgfqpoint{5.438574in}{1.146762in}}{\pgfqpoint{5.441846in}{1.138862in}}{\pgfqpoint{5.447670in}{1.133038in}}%
\pgfpathcurveto{\pgfqpoint{5.453494in}{1.127214in}}{\pgfqpoint{5.461394in}{1.123942in}}{\pgfqpoint{5.469630in}{1.123942in}}%
\pgfpathclose%
\pgfusepath{stroke,fill}%
\end{pgfscope}%
\begin{pgfscope}%
\pgfpathrectangle{\pgfqpoint{0.457963in}{0.528059in}}{\pgfqpoint{6.200000in}{2.285714in}} %
\pgfusepath{clip}%
\pgfsetbuttcap%
\pgfsetroundjoin%
\definecolor{currentfill}{rgb}{1.000000,0.833333,0.833333}%
\pgfsetfillcolor{currentfill}%
\pgfsetlinewidth{1.003750pt}%
\definecolor{currentstroke}{rgb}{1.000000,0.833333,0.833333}%
\pgfsetstrokecolor{currentstroke}%
\pgfsetdash{}{0pt}%
\pgfpathmoveto{\pgfqpoint{0.457963in}{0.823534in}}%
\pgfpathcurveto{\pgfqpoint{0.466200in}{0.823534in}}{\pgfqpoint{0.474100in}{0.826806in}}{\pgfqpoint{0.479924in}{0.832630in}}%
\pgfpathcurveto{\pgfqpoint{0.485748in}{0.838454in}}{\pgfqpoint{0.489020in}{0.846354in}}{\pgfqpoint{0.489020in}{0.854590in}}%
\pgfpathcurveto{\pgfqpoint{0.489020in}{0.862826in}}{\pgfqpoint{0.485748in}{0.870726in}}{\pgfqpoint{0.479924in}{0.876550in}}%
\pgfpathcurveto{\pgfqpoint{0.474100in}{0.882374in}}{\pgfqpoint{0.466200in}{0.885647in}}{\pgfqpoint{0.457963in}{0.885647in}}%
\pgfpathcurveto{\pgfqpoint{0.449727in}{0.885647in}}{\pgfqpoint{0.441827in}{0.882374in}}{\pgfqpoint{0.436003in}{0.876550in}}%
\pgfpathcurveto{\pgfqpoint{0.430179in}{0.870726in}}{\pgfqpoint{0.426907in}{0.862826in}}{\pgfqpoint{0.426907in}{0.854590in}}%
\pgfpathcurveto{\pgfqpoint{0.426907in}{0.846354in}}{\pgfqpoint{0.430179in}{0.838454in}}{\pgfqpoint{0.436003in}{0.832630in}}%
\pgfpathcurveto{\pgfqpoint{0.441827in}{0.826806in}}{\pgfqpoint{0.449727in}{0.823534in}}{\pgfqpoint{0.457963in}{0.823534in}}%
\pgfpathclose%
\pgfusepath{stroke,fill}%
\end{pgfscope}%
\begin{pgfscope}%
\pgfpathrectangle{\pgfqpoint{0.457963in}{0.528059in}}{\pgfqpoint{6.200000in}{2.285714in}} %
\pgfusepath{clip}%
\pgfsetbuttcap%
\pgfsetroundjoin%
\definecolor{currentfill}{rgb}{1.000000,0.833333,0.833333}%
\pgfsetfillcolor{currentfill}%
\pgfsetlinewidth{1.003750pt}%
\definecolor{currentstroke}{rgb}{1.000000,0.833333,0.833333}%
\pgfsetstrokecolor{currentstroke}%
\pgfsetdash{}{0pt}%
\pgfpathmoveto{\pgfqpoint{0.457963in}{0.823534in}}%
\pgfpathcurveto{\pgfqpoint{0.466200in}{0.823534in}}{\pgfqpoint{0.474100in}{0.826806in}}{\pgfqpoint{0.479924in}{0.832630in}}%
\pgfpathcurveto{\pgfqpoint{0.485748in}{0.838454in}}{\pgfqpoint{0.489020in}{0.846354in}}{\pgfqpoint{0.489020in}{0.854590in}}%
\pgfpathcurveto{\pgfqpoint{0.489020in}{0.862826in}}{\pgfqpoint{0.485748in}{0.870726in}}{\pgfqpoint{0.479924in}{0.876550in}}%
\pgfpathcurveto{\pgfqpoint{0.474100in}{0.882374in}}{\pgfqpoint{0.466200in}{0.885647in}}{\pgfqpoint{0.457963in}{0.885647in}}%
\pgfpathcurveto{\pgfqpoint{0.449727in}{0.885647in}}{\pgfqpoint{0.441827in}{0.882374in}}{\pgfqpoint{0.436003in}{0.876550in}}%
\pgfpathcurveto{\pgfqpoint{0.430179in}{0.870726in}}{\pgfqpoint{0.426907in}{0.862826in}}{\pgfqpoint{0.426907in}{0.854590in}}%
\pgfpathcurveto{\pgfqpoint{0.426907in}{0.846354in}}{\pgfqpoint{0.430179in}{0.838454in}}{\pgfqpoint{0.436003in}{0.832630in}}%
\pgfpathcurveto{\pgfqpoint{0.441827in}{0.826806in}}{\pgfqpoint{0.449727in}{0.823534in}}{\pgfqpoint{0.457963in}{0.823534in}}%
\pgfpathclose%
\pgfusepath{stroke,fill}%
\end{pgfscope}%
\begin{pgfscope}%
\pgfpathrectangle{\pgfqpoint{0.457963in}{0.528059in}}{\pgfqpoint{6.200000in}{2.285714in}} %
\pgfusepath{clip}%
\pgfsetbuttcap%
\pgfsetroundjoin%
\definecolor{currentfill}{rgb}{1.000000,0.833333,0.833333}%
\pgfsetfillcolor{currentfill}%
\pgfsetlinewidth{1.003750pt}%
\definecolor{currentstroke}{rgb}{1.000000,0.833333,0.833333}%
\pgfsetstrokecolor{currentstroke}%
\pgfsetdash{}{0pt}%
\pgfpathmoveto{\pgfqpoint{0.457963in}{0.823534in}}%
\pgfpathcurveto{\pgfqpoint{0.466200in}{0.823534in}}{\pgfqpoint{0.474100in}{0.826806in}}{\pgfqpoint{0.479924in}{0.832630in}}%
\pgfpathcurveto{\pgfqpoint{0.485748in}{0.838454in}}{\pgfqpoint{0.489020in}{0.846354in}}{\pgfqpoint{0.489020in}{0.854590in}}%
\pgfpathcurveto{\pgfqpoint{0.489020in}{0.862826in}}{\pgfqpoint{0.485748in}{0.870726in}}{\pgfqpoint{0.479924in}{0.876550in}}%
\pgfpathcurveto{\pgfqpoint{0.474100in}{0.882374in}}{\pgfqpoint{0.466200in}{0.885647in}}{\pgfqpoint{0.457963in}{0.885647in}}%
\pgfpathcurveto{\pgfqpoint{0.449727in}{0.885647in}}{\pgfqpoint{0.441827in}{0.882374in}}{\pgfqpoint{0.436003in}{0.876550in}}%
\pgfpathcurveto{\pgfqpoint{0.430179in}{0.870726in}}{\pgfqpoint{0.426907in}{0.862826in}}{\pgfqpoint{0.426907in}{0.854590in}}%
\pgfpathcurveto{\pgfqpoint{0.426907in}{0.846354in}}{\pgfqpoint{0.430179in}{0.838454in}}{\pgfqpoint{0.436003in}{0.832630in}}%
\pgfpathcurveto{\pgfqpoint{0.441827in}{0.826806in}}{\pgfqpoint{0.449727in}{0.823534in}}{\pgfqpoint{0.457963in}{0.823534in}}%
\pgfpathclose%
\pgfusepath{stroke,fill}%
\end{pgfscope}%
\begin{pgfscope}%
\pgfpathrectangle{\pgfqpoint{0.457963in}{0.528059in}}{\pgfqpoint{6.200000in}{2.285714in}} %
\pgfusepath{clip}%
\pgfsetbuttcap%
\pgfsetroundjoin%
\definecolor{currentfill}{rgb}{1.000000,0.833333,0.833333}%
\pgfsetfillcolor{currentfill}%
\pgfsetlinewidth{1.003750pt}%
\definecolor{currentstroke}{rgb}{1.000000,0.833333,0.833333}%
\pgfsetstrokecolor{currentstroke}%
\pgfsetdash{}{0pt}%
\pgfpathmoveto{\pgfqpoint{0.457963in}{0.823534in}}%
\pgfpathcurveto{\pgfqpoint{0.466200in}{0.823534in}}{\pgfqpoint{0.474100in}{0.826806in}}{\pgfqpoint{0.479924in}{0.832630in}}%
\pgfpathcurveto{\pgfqpoint{0.485748in}{0.838454in}}{\pgfqpoint{0.489020in}{0.846354in}}{\pgfqpoint{0.489020in}{0.854590in}}%
\pgfpathcurveto{\pgfqpoint{0.489020in}{0.862826in}}{\pgfqpoint{0.485748in}{0.870726in}}{\pgfqpoint{0.479924in}{0.876550in}}%
\pgfpathcurveto{\pgfqpoint{0.474100in}{0.882374in}}{\pgfqpoint{0.466200in}{0.885647in}}{\pgfqpoint{0.457963in}{0.885647in}}%
\pgfpathcurveto{\pgfqpoint{0.449727in}{0.885647in}}{\pgfqpoint{0.441827in}{0.882374in}}{\pgfqpoint{0.436003in}{0.876550in}}%
\pgfpathcurveto{\pgfqpoint{0.430179in}{0.870726in}}{\pgfqpoint{0.426907in}{0.862826in}}{\pgfqpoint{0.426907in}{0.854590in}}%
\pgfpathcurveto{\pgfqpoint{0.426907in}{0.846354in}}{\pgfqpoint{0.430179in}{0.838454in}}{\pgfqpoint{0.436003in}{0.832630in}}%
\pgfpathcurveto{\pgfqpoint{0.441827in}{0.826806in}}{\pgfqpoint{0.449727in}{0.823534in}}{\pgfqpoint{0.457963in}{0.823534in}}%
\pgfpathclose%
\pgfusepath{stroke,fill}%
\end{pgfscope}%
\begin{pgfscope}%
\pgfpathrectangle{\pgfqpoint{0.457963in}{0.528059in}}{\pgfqpoint{6.200000in}{2.285714in}} %
\pgfusepath{clip}%
\pgfsetbuttcap%
\pgfsetroundjoin%
\definecolor{currentfill}{rgb}{1.000000,0.833333,0.833333}%
\pgfsetfillcolor{currentfill}%
\pgfsetlinewidth{1.003750pt}%
\definecolor{currentstroke}{rgb}{1.000000,0.833333,0.833333}%
\pgfsetstrokecolor{currentstroke}%
\pgfsetdash{}{0pt}%
\pgfpathmoveto{\pgfqpoint{0.457963in}{0.823534in}}%
\pgfpathcurveto{\pgfqpoint{0.466200in}{0.823534in}}{\pgfqpoint{0.474100in}{0.826806in}}{\pgfqpoint{0.479924in}{0.832630in}}%
\pgfpathcurveto{\pgfqpoint{0.485748in}{0.838454in}}{\pgfqpoint{0.489020in}{0.846354in}}{\pgfqpoint{0.489020in}{0.854590in}}%
\pgfpathcurveto{\pgfqpoint{0.489020in}{0.862826in}}{\pgfqpoint{0.485748in}{0.870726in}}{\pgfqpoint{0.479924in}{0.876550in}}%
\pgfpathcurveto{\pgfqpoint{0.474100in}{0.882374in}}{\pgfqpoint{0.466200in}{0.885647in}}{\pgfqpoint{0.457963in}{0.885647in}}%
\pgfpathcurveto{\pgfqpoint{0.449727in}{0.885647in}}{\pgfqpoint{0.441827in}{0.882374in}}{\pgfqpoint{0.436003in}{0.876550in}}%
\pgfpathcurveto{\pgfqpoint{0.430179in}{0.870726in}}{\pgfqpoint{0.426907in}{0.862826in}}{\pgfqpoint{0.426907in}{0.854590in}}%
\pgfpathcurveto{\pgfqpoint{0.426907in}{0.846354in}}{\pgfqpoint{0.430179in}{0.838454in}}{\pgfqpoint{0.436003in}{0.832630in}}%
\pgfpathcurveto{\pgfqpoint{0.441827in}{0.826806in}}{\pgfqpoint{0.449727in}{0.823534in}}{\pgfqpoint{0.457963in}{0.823534in}}%
\pgfpathclose%
\pgfusepath{stroke,fill}%
\end{pgfscope}%
\begin{pgfscope}%
\pgfpathrectangle{\pgfqpoint{0.457963in}{0.528059in}}{\pgfqpoint{6.200000in}{2.285714in}} %
\pgfusepath{clip}%
\pgfsetbuttcap%
\pgfsetroundjoin%
\definecolor{currentfill}{rgb}{1.000000,0.833333,0.833333}%
\pgfsetfillcolor{currentfill}%
\pgfsetlinewidth{1.003750pt}%
\definecolor{currentstroke}{rgb}{1.000000,0.833333,0.833333}%
\pgfsetstrokecolor{currentstroke}%
\pgfsetdash{}{0pt}%
\pgfpathmoveto{\pgfqpoint{0.457963in}{0.823534in}}%
\pgfpathcurveto{\pgfqpoint{0.466200in}{0.823534in}}{\pgfqpoint{0.474100in}{0.826806in}}{\pgfqpoint{0.479924in}{0.832630in}}%
\pgfpathcurveto{\pgfqpoint{0.485748in}{0.838454in}}{\pgfqpoint{0.489020in}{0.846354in}}{\pgfqpoint{0.489020in}{0.854590in}}%
\pgfpathcurveto{\pgfqpoint{0.489020in}{0.862826in}}{\pgfqpoint{0.485748in}{0.870726in}}{\pgfqpoint{0.479924in}{0.876550in}}%
\pgfpathcurveto{\pgfqpoint{0.474100in}{0.882374in}}{\pgfqpoint{0.466200in}{0.885647in}}{\pgfqpoint{0.457963in}{0.885647in}}%
\pgfpathcurveto{\pgfqpoint{0.449727in}{0.885647in}}{\pgfqpoint{0.441827in}{0.882374in}}{\pgfqpoint{0.436003in}{0.876550in}}%
\pgfpathcurveto{\pgfqpoint{0.430179in}{0.870726in}}{\pgfqpoint{0.426907in}{0.862826in}}{\pgfqpoint{0.426907in}{0.854590in}}%
\pgfpathcurveto{\pgfqpoint{0.426907in}{0.846354in}}{\pgfqpoint{0.430179in}{0.838454in}}{\pgfqpoint{0.436003in}{0.832630in}}%
\pgfpathcurveto{\pgfqpoint{0.441827in}{0.826806in}}{\pgfqpoint{0.449727in}{0.823534in}}{\pgfqpoint{0.457963in}{0.823534in}}%
\pgfpathclose%
\pgfusepath{stroke,fill}%
\end{pgfscope}%
\begin{pgfscope}%
\pgfpathrectangle{\pgfqpoint{0.457963in}{0.528059in}}{\pgfqpoint{6.200000in}{2.285714in}} %
\pgfusepath{clip}%
\pgfsetbuttcap%
\pgfsetroundjoin%
\definecolor{currentfill}{rgb}{1.000000,0.833333,0.833333}%
\pgfsetfillcolor{currentfill}%
\pgfsetlinewidth{1.003750pt}%
\definecolor{currentstroke}{rgb}{1.000000,0.833333,0.833333}%
\pgfsetstrokecolor{currentstroke}%
\pgfsetdash{}{0pt}%
\pgfpathmoveto{\pgfqpoint{0.468297in}{0.810472in}}%
\pgfpathcurveto{\pgfqpoint{0.476533in}{0.810472in}}{\pgfqpoint{0.484433in}{0.813745in}}{\pgfqpoint{0.490257in}{0.819569in}}%
\pgfpathcurveto{\pgfqpoint{0.496081in}{0.825393in}}{\pgfqpoint{0.499353in}{0.833293in}}{\pgfqpoint{0.499353in}{0.841529in}}%
\pgfpathcurveto{\pgfqpoint{0.499353in}{0.849765in}}{\pgfqpoint{0.496081in}{0.857665in}}{\pgfqpoint{0.490257in}{0.863489in}}%
\pgfpathcurveto{\pgfqpoint{0.484433in}{0.869313in}}{\pgfqpoint{0.476533in}{0.872585in}}{\pgfqpoint{0.468297in}{0.872585in}}%
\pgfpathcurveto{\pgfqpoint{0.460060in}{0.872585in}}{\pgfqpoint{0.452160in}{0.869313in}}{\pgfqpoint{0.446336in}{0.863489in}}%
\pgfpathcurveto{\pgfqpoint{0.440512in}{0.857665in}}{\pgfqpoint{0.437240in}{0.849765in}}{\pgfqpoint{0.437240in}{0.841529in}}%
\pgfpathcurveto{\pgfqpoint{0.437240in}{0.833293in}}{\pgfqpoint{0.440512in}{0.825393in}}{\pgfqpoint{0.446336in}{0.819569in}}%
\pgfpathcurveto{\pgfqpoint{0.452160in}{0.813745in}}{\pgfqpoint{0.460060in}{0.810472in}}{\pgfqpoint{0.468297in}{0.810472in}}%
\pgfpathclose%
\pgfusepath{stroke,fill}%
\end{pgfscope}%
\begin{pgfscope}%
\pgfpathrectangle{\pgfqpoint{0.457963in}{0.528059in}}{\pgfqpoint{6.200000in}{2.285714in}} %
\pgfusepath{clip}%
\pgfsetbuttcap%
\pgfsetroundjoin%
\definecolor{currentfill}{rgb}{1.000000,0.833333,0.833333}%
\pgfsetfillcolor{currentfill}%
\pgfsetlinewidth{1.003750pt}%
\definecolor{currentstroke}{rgb}{1.000000,0.833333,0.833333}%
\pgfsetstrokecolor{currentstroke}%
\pgfsetdash{}{0pt}%
\pgfpathmoveto{\pgfqpoint{0.478630in}{0.823534in}}%
\pgfpathcurveto{\pgfqpoint{0.486866in}{0.823534in}}{\pgfqpoint{0.494766in}{0.826806in}}{\pgfqpoint{0.500590in}{0.832630in}}%
\pgfpathcurveto{\pgfqpoint{0.506414in}{0.838454in}}{\pgfqpoint{0.509686in}{0.846354in}}{\pgfqpoint{0.509686in}{0.854590in}}%
\pgfpathcurveto{\pgfqpoint{0.509686in}{0.862826in}}{\pgfqpoint{0.506414in}{0.870726in}}{\pgfqpoint{0.500590in}{0.876550in}}%
\pgfpathcurveto{\pgfqpoint{0.494766in}{0.882374in}}{\pgfqpoint{0.486866in}{0.885647in}}{\pgfqpoint{0.478630in}{0.885647in}}%
\pgfpathcurveto{\pgfqpoint{0.470394in}{0.885647in}}{\pgfqpoint{0.462494in}{0.882374in}}{\pgfqpoint{0.456670in}{0.876550in}}%
\pgfpathcurveto{\pgfqpoint{0.450846in}{0.870726in}}{\pgfqpoint{0.447574in}{0.862826in}}{\pgfqpoint{0.447574in}{0.854590in}}%
\pgfpathcurveto{\pgfqpoint{0.447574in}{0.846354in}}{\pgfqpoint{0.450846in}{0.838454in}}{\pgfqpoint{0.456670in}{0.832630in}}%
\pgfpathcurveto{\pgfqpoint{0.462494in}{0.826806in}}{\pgfqpoint{0.470394in}{0.823534in}}{\pgfqpoint{0.478630in}{0.823534in}}%
\pgfpathclose%
\pgfusepath{stroke,fill}%
\end{pgfscope}%
\begin{pgfscope}%
\pgfpathrectangle{\pgfqpoint{0.457963in}{0.528059in}}{\pgfqpoint{6.200000in}{2.285714in}} %
\pgfusepath{clip}%
\pgfsetbuttcap%
\pgfsetroundjoin%
\definecolor{currentfill}{rgb}{1.000000,0.833333,0.833333}%
\pgfsetfillcolor{currentfill}%
\pgfsetlinewidth{1.003750pt}%
\definecolor{currentstroke}{rgb}{1.000000,0.833333,0.833333}%
\pgfsetstrokecolor{currentstroke}%
\pgfsetdash{}{0pt}%
\pgfpathmoveto{\pgfqpoint{0.499297in}{0.784350in}}%
\pgfpathcurveto{\pgfqpoint{0.507533in}{0.784350in}}{\pgfqpoint{0.515433in}{0.787622in}}{\pgfqpoint{0.521257in}{0.793446in}}%
\pgfpathcurveto{\pgfqpoint{0.527081in}{0.799270in}}{\pgfqpoint{0.530353in}{0.807170in}}{\pgfqpoint{0.530353in}{0.815406in}}%
\pgfpathcurveto{\pgfqpoint{0.530353in}{0.823643in}}{\pgfqpoint{0.527081in}{0.831543in}}{\pgfqpoint{0.521257in}{0.837367in}}%
\pgfpathcurveto{\pgfqpoint{0.515433in}{0.843191in}}{\pgfqpoint{0.507533in}{0.846463in}}{\pgfqpoint{0.499297in}{0.846463in}}%
\pgfpathcurveto{\pgfqpoint{0.491060in}{0.846463in}}{\pgfqpoint{0.483160in}{0.843191in}}{\pgfqpoint{0.477336in}{0.837367in}}%
\pgfpathcurveto{\pgfqpoint{0.471512in}{0.831543in}}{\pgfqpoint{0.468240in}{0.823643in}}{\pgfqpoint{0.468240in}{0.815406in}}%
\pgfpathcurveto{\pgfqpoint{0.468240in}{0.807170in}}{\pgfqpoint{0.471512in}{0.799270in}}{\pgfqpoint{0.477336in}{0.793446in}}%
\pgfpathcurveto{\pgfqpoint{0.483160in}{0.787622in}}{\pgfqpoint{0.491060in}{0.784350in}}{\pgfqpoint{0.499297in}{0.784350in}}%
\pgfpathclose%
\pgfusepath{stroke,fill}%
\end{pgfscope}%
\begin{pgfscope}%
\pgfpathrectangle{\pgfqpoint{0.457963in}{0.528059in}}{\pgfqpoint{6.200000in}{2.285714in}} %
\pgfusepath{clip}%
\pgfsetbuttcap%
\pgfsetroundjoin%
\definecolor{currentfill}{rgb}{1.000000,0.833333,0.833333}%
\pgfsetfillcolor{currentfill}%
\pgfsetlinewidth{1.003750pt}%
\definecolor{currentstroke}{rgb}{1.000000,0.833333,0.833333}%
\pgfsetstrokecolor{currentstroke}%
\pgfsetdash{}{0pt}%
\pgfpathmoveto{\pgfqpoint{0.519963in}{0.784350in}}%
\pgfpathcurveto{\pgfqpoint{0.528200in}{0.784350in}}{\pgfqpoint{0.536100in}{0.787622in}}{\pgfqpoint{0.541924in}{0.793446in}}%
\pgfpathcurveto{\pgfqpoint{0.547748in}{0.799270in}}{\pgfqpoint{0.551020in}{0.807170in}}{\pgfqpoint{0.551020in}{0.815406in}}%
\pgfpathcurveto{\pgfqpoint{0.551020in}{0.823643in}}{\pgfqpoint{0.547748in}{0.831543in}}{\pgfqpoint{0.541924in}{0.837367in}}%
\pgfpathcurveto{\pgfqpoint{0.536100in}{0.843191in}}{\pgfqpoint{0.528200in}{0.846463in}}{\pgfqpoint{0.519963in}{0.846463in}}%
\pgfpathcurveto{\pgfqpoint{0.511727in}{0.846463in}}{\pgfqpoint{0.503827in}{0.843191in}}{\pgfqpoint{0.498003in}{0.837367in}}%
\pgfpathcurveto{\pgfqpoint{0.492179in}{0.831543in}}{\pgfqpoint{0.488907in}{0.823643in}}{\pgfqpoint{0.488907in}{0.815406in}}%
\pgfpathcurveto{\pgfqpoint{0.488907in}{0.807170in}}{\pgfqpoint{0.492179in}{0.799270in}}{\pgfqpoint{0.498003in}{0.793446in}}%
\pgfpathcurveto{\pgfqpoint{0.503827in}{0.787622in}}{\pgfqpoint{0.511727in}{0.784350in}}{\pgfqpoint{0.519963in}{0.784350in}}%
\pgfpathclose%
\pgfusepath{stroke,fill}%
\end{pgfscope}%
\begin{pgfscope}%
\pgfpathrectangle{\pgfqpoint{0.457963in}{0.528059in}}{\pgfqpoint{6.200000in}{2.285714in}} %
\pgfusepath{clip}%
\pgfsetbuttcap%
\pgfsetroundjoin%
\definecolor{currentfill}{rgb}{1.000000,0.833333,0.833333}%
\pgfsetfillcolor{currentfill}%
\pgfsetlinewidth{1.003750pt}%
\definecolor{currentstroke}{rgb}{1.000000,0.833333,0.833333}%
\pgfsetstrokecolor{currentstroke}%
\pgfsetdash{}{0pt}%
\pgfpathmoveto{\pgfqpoint{0.519963in}{0.823534in}}%
\pgfpathcurveto{\pgfqpoint{0.528200in}{0.823534in}}{\pgfqpoint{0.536100in}{0.826806in}}{\pgfqpoint{0.541924in}{0.832630in}}%
\pgfpathcurveto{\pgfqpoint{0.547748in}{0.838454in}}{\pgfqpoint{0.551020in}{0.846354in}}{\pgfqpoint{0.551020in}{0.854590in}}%
\pgfpathcurveto{\pgfqpoint{0.551020in}{0.862826in}}{\pgfqpoint{0.547748in}{0.870726in}}{\pgfqpoint{0.541924in}{0.876550in}}%
\pgfpathcurveto{\pgfqpoint{0.536100in}{0.882374in}}{\pgfqpoint{0.528200in}{0.885647in}}{\pgfqpoint{0.519963in}{0.885647in}}%
\pgfpathcurveto{\pgfqpoint{0.511727in}{0.885647in}}{\pgfqpoint{0.503827in}{0.882374in}}{\pgfqpoint{0.498003in}{0.876550in}}%
\pgfpathcurveto{\pgfqpoint{0.492179in}{0.870726in}}{\pgfqpoint{0.488907in}{0.862826in}}{\pgfqpoint{0.488907in}{0.854590in}}%
\pgfpathcurveto{\pgfqpoint{0.488907in}{0.846354in}}{\pgfqpoint{0.492179in}{0.838454in}}{\pgfqpoint{0.498003in}{0.832630in}}%
\pgfpathcurveto{\pgfqpoint{0.503827in}{0.826806in}}{\pgfqpoint{0.511727in}{0.823534in}}{\pgfqpoint{0.519963in}{0.823534in}}%
\pgfpathclose%
\pgfusepath{stroke,fill}%
\end{pgfscope}%
\begin{pgfscope}%
\pgfpathrectangle{\pgfqpoint{0.457963in}{0.528059in}}{\pgfqpoint{6.200000in}{2.285714in}} %
\pgfusepath{clip}%
\pgfsetbuttcap%
\pgfsetroundjoin%
\definecolor{currentfill}{rgb}{1.000000,0.833333,0.833333}%
\pgfsetfillcolor{currentfill}%
\pgfsetlinewidth{1.003750pt}%
\definecolor{currentstroke}{rgb}{1.000000,0.833333,0.833333}%
\pgfsetstrokecolor{currentstroke}%
\pgfsetdash{}{0pt}%
\pgfpathmoveto{\pgfqpoint{0.550963in}{0.784350in}}%
\pgfpathcurveto{\pgfqpoint{0.559200in}{0.784350in}}{\pgfqpoint{0.567100in}{0.787622in}}{\pgfqpoint{0.572924in}{0.793446in}}%
\pgfpathcurveto{\pgfqpoint{0.578748in}{0.799270in}}{\pgfqpoint{0.582020in}{0.807170in}}{\pgfqpoint{0.582020in}{0.815406in}}%
\pgfpathcurveto{\pgfqpoint{0.582020in}{0.823643in}}{\pgfqpoint{0.578748in}{0.831543in}}{\pgfqpoint{0.572924in}{0.837367in}}%
\pgfpathcurveto{\pgfqpoint{0.567100in}{0.843191in}}{\pgfqpoint{0.559200in}{0.846463in}}{\pgfqpoint{0.550963in}{0.846463in}}%
\pgfpathcurveto{\pgfqpoint{0.542727in}{0.846463in}}{\pgfqpoint{0.534827in}{0.843191in}}{\pgfqpoint{0.529003in}{0.837367in}}%
\pgfpathcurveto{\pgfqpoint{0.523179in}{0.831543in}}{\pgfqpoint{0.519907in}{0.823643in}}{\pgfqpoint{0.519907in}{0.815406in}}%
\pgfpathcurveto{\pgfqpoint{0.519907in}{0.807170in}}{\pgfqpoint{0.523179in}{0.799270in}}{\pgfqpoint{0.529003in}{0.793446in}}%
\pgfpathcurveto{\pgfqpoint{0.534827in}{0.787622in}}{\pgfqpoint{0.542727in}{0.784350in}}{\pgfqpoint{0.550963in}{0.784350in}}%
\pgfpathclose%
\pgfusepath{stroke,fill}%
\end{pgfscope}%
\begin{pgfscope}%
\pgfpathrectangle{\pgfqpoint{0.457963in}{0.528059in}}{\pgfqpoint{6.200000in}{2.285714in}} %
\pgfusepath{clip}%
\pgfsetbuttcap%
\pgfsetroundjoin%
\definecolor{currentfill}{rgb}{1.000000,0.833333,0.833333}%
\pgfsetfillcolor{currentfill}%
\pgfsetlinewidth{1.003750pt}%
\definecolor{currentstroke}{rgb}{1.000000,0.833333,0.833333}%
\pgfsetstrokecolor{currentstroke}%
\pgfsetdash{}{0pt}%
\pgfpathmoveto{\pgfqpoint{0.612963in}{0.823534in}}%
\pgfpathcurveto{\pgfqpoint{0.621200in}{0.823534in}}{\pgfqpoint{0.629100in}{0.826806in}}{\pgfqpoint{0.634924in}{0.832630in}}%
\pgfpathcurveto{\pgfqpoint{0.640748in}{0.838454in}}{\pgfqpoint{0.644020in}{0.846354in}}{\pgfqpoint{0.644020in}{0.854590in}}%
\pgfpathcurveto{\pgfqpoint{0.644020in}{0.862826in}}{\pgfqpoint{0.640748in}{0.870726in}}{\pgfqpoint{0.634924in}{0.876550in}}%
\pgfpathcurveto{\pgfqpoint{0.629100in}{0.882374in}}{\pgfqpoint{0.621200in}{0.885647in}}{\pgfqpoint{0.612963in}{0.885647in}}%
\pgfpathcurveto{\pgfqpoint{0.604727in}{0.885647in}}{\pgfqpoint{0.596827in}{0.882374in}}{\pgfqpoint{0.591003in}{0.876550in}}%
\pgfpathcurveto{\pgfqpoint{0.585179in}{0.870726in}}{\pgfqpoint{0.581907in}{0.862826in}}{\pgfqpoint{0.581907in}{0.854590in}}%
\pgfpathcurveto{\pgfqpoint{0.581907in}{0.846354in}}{\pgfqpoint{0.585179in}{0.838454in}}{\pgfqpoint{0.591003in}{0.832630in}}%
\pgfpathcurveto{\pgfqpoint{0.596827in}{0.826806in}}{\pgfqpoint{0.604727in}{0.823534in}}{\pgfqpoint{0.612963in}{0.823534in}}%
\pgfpathclose%
\pgfusepath{stroke,fill}%
\end{pgfscope}%
\begin{pgfscope}%
\pgfpathrectangle{\pgfqpoint{0.457963in}{0.528059in}}{\pgfqpoint{6.200000in}{2.285714in}} %
\pgfusepath{clip}%
\pgfsetbuttcap%
\pgfsetroundjoin%
\definecolor{currentfill}{rgb}{1.000000,0.833333,0.833333}%
\pgfsetfillcolor{currentfill}%
\pgfsetlinewidth{1.003750pt}%
\definecolor{currentstroke}{rgb}{1.000000,0.833333,0.833333}%
\pgfsetstrokecolor{currentstroke}%
\pgfsetdash{}{0pt}%
\pgfpathmoveto{\pgfqpoint{0.984963in}{0.692921in}}%
\pgfpathcurveto{\pgfqpoint{0.993200in}{0.692921in}}{\pgfqpoint{1.001100in}{0.696194in}}{\pgfqpoint{1.006924in}{0.702018in}}%
\pgfpathcurveto{\pgfqpoint{1.012748in}{0.707841in}}{\pgfqpoint{1.016020in}{0.715742in}}{\pgfqpoint{1.016020in}{0.723978in}}%
\pgfpathcurveto{\pgfqpoint{1.016020in}{0.732214in}}{\pgfqpoint{1.012748in}{0.740114in}}{\pgfqpoint{1.006924in}{0.745938in}}%
\pgfpathcurveto{\pgfqpoint{1.001100in}{0.751762in}}{\pgfqpoint{0.993200in}{0.755034in}}{\pgfqpoint{0.984963in}{0.755034in}}%
\pgfpathcurveto{\pgfqpoint{0.976727in}{0.755034in}}{\pgfqpoint{0.968827in}{0.751762in}}{\pgfqpoint{0.963003in}{0.745938in}}%
\pgfpathcurveto{\pgfqpoint{0.957179in}{0.740114in}}{\pgfqpoint{0.953907in}{0.732214in}}{\pgfqpoint{0.953907in}{0.723978in}}%
\pgfpathcurveto{\pgfqpoint{0.953907in}{0.715742in}}{\pgfqpoint{0.957179in}{0.707841in}}{\pgfqpoint{0.963003in}{0.702018in}}%
\pgfpathcurveto{\pgfqpoint{0.968827in}{0.696194in}}{\pgfqpoint{0.976727in}{0.692921in}}{\pgfqpoint{0.984963in}{0.692921in}}%
\pgfpathclose%
\pgfusepath{stroke,fill}%
\end{pgfscope}%
\begin{pgfscope}%
\pgfpathrectangle{\pgfqpoint{0.457963in}{0.528059in}}{\pgfqpoint{6.200000in}{2.285714in}} %
\pgfusepath{clip}%
\pgfsetbuttcap%
\pgfsetroundjoin%
\definecolor{currentfill}{rgb}{1.000000,0.833333,0.833333}%
\pgfsetfillcolor{currentfill}%
\pgfsetlinewidth{1.003750pt}%
\definecolor{currentstroke}{rgb}{1.000000,0.833333,0.833333}%
\pgfsetstrokecolor{currentstroke}%
\pgfsetdash{}{0pt}%
\pgfpathmoveto{\pgfqpoint{1.098630in}{0.823534in}}%
\pgfpathcurveto{\pgfqpoint{1.106866in}{0.823534in}}{\pgfqpoint{1.114766in}{0.826806in}}{\pgfqpoint{1.120590in}{0.832630in}}%
\pgfpathcurveto{\pgfqpoint{1.126414in}{0.838454in}}{\pgfqpoint{1.129686in}{0.846354in}}{\pgfqpoint{1.129686in}{0.854590in}}%
\pgfpathcurveto{\pgfqpoint{1.129686in}{0.862826in}}{\pgfqpoint{1.126414in}{0.870726in}}{\pgfqpoint{1.120590in}{0.876550in}}%
\pgfpathcurveto{\pgfqpoint{1.114766in}{0.882374in}}{\pgfqpoint{1.106866in}{0.885647in}}{\pgfqpoint{1.098630in}{0.885647in}}%
\pgfpathcurveto{\pgfqpoint{1.090394in}{0.885647in}}{\pgfqpoint{1.082494in}{0.882374in}}{\pgfqpoint{1.076670in}{0.876550in}}%
\pgfpathcurveto{\pgfqpoint{1.070846in}{0.870726in}}{\pgfqpoint{1.067574in}{0.862826in}}{\pgfqpoint{1.067574in}{0.854590in}}%
\pgfpathcurveto{\pgfqpoint{1.067574in}{0.846354in}}{\pgfqpoint{1.070846in}{0.838454in}}{\pgfqpoint{1.076670in}{0.832630in}}%
\pgfpathcurveto{\pgfqpoint{1.082494in}{0.826806in}}{\pgfqpoint{1.090394in}{0.823534in}}{\pgfqpoint{1.098630in}{0.823534in}}%
\pgfpathclose%
\pgfusepath{stroke,fill}%
\end{pgfscope}%
\begin{pgfscope}%
\pgfpathrectangle{\pgfqpoint{0.457963in}{0.528059in}}{\pgfqpoint{6.200000in}{2.285714in}} %
\pgfusepath{clip}%
\pgfsetbuttcap%
\pgfsetroundjoin%
\definecolor{currentfill}{rgb}{1.000000,0.833333,0.833333}%
\pgfsetfillcolor{currentfill}%
\pgfsetlinewidth{1.003750pt}%
\definecolor{currentstroke}{rgb}{1.000000,0.833333,0.833333}%
\pgfsetstrokecolor{currentstroke}%
\pgfsetdash{}{0pt}%
\pgfpathmoveto{\pgfqpoint{1.305297in}{0.810472in}}%
\pgfpathcurveto{\pgfqpoint{1.313533in}{0.810472in}}{\pgfqpoint{1.321433in}{0.813745in}}{\pgfqpoint{1.327257in}{0.819569in}}%
\pgfpathcurveto{\pgfqpoint{1.333081in}{0.825393in}}{\pgfqpoint{1.336353in}{0.833293in}}{\pgfqpoint{1.336353in}{0.841529in}}%
\pgfpathcurveto{\pgfqpoint{1.336353in}{0.849765in}}{\pgfqpoint{1.333081in}{0.857665in}}{\pgfqpoint{1.327257in}{0.863489in}}%
\pgfpathcurveto{\pgfqpoint{1.321433in}{0.869313in}}{\pgfqpoint{1.313533in}{0.872585in}}{\pgfqpoint{1.305297in}{0.872585in}}%
\pgfpathcurveto{\pgfqpoint{1.297060in}{0.872585in}}{\pgfqpoint{1.289160in}{0.869313in}}{\pgfqpoint{1.283336in}{0.863489in}}%
\pgfpathcurveto{\pgfqpoint{1.277512in}{0.857665in}}{\pgfqpoint{1.274240in}{0.849765in}}{\pgfqpoint{1.274240in}{0.841529in}}%
\pgfpathcurveto{\pgfqpoint{1.274240in}{0.833293in}}{\pgfqpoint{1.277512in}{0.825393in}}{\pgfqpoint{1.283336in}{0.819569in}}%
\pgfpathcurveto{\pgfqpoint{1.289160in}{0.813745in}}{\pgfqpoint{1.297060in}{0.810472in}}{\pgfqpoint{1.305297in}{0.810472in}}%
\pgfpathclose%
\pgfusepath{stroke,fill}%
\end{pgfscope}%
\begin{pgfscope}%
\pgfpathrectangle{\pgfqpoint{0.457963in}{0.528059in}}{\pgfqpoint{6.200000in}{2.285714in}} %
\pgfusepath{clip}%
\pgfsetbuttcap%
\pgfsetroundjoin%
\definecolor{currentfill}{rgb}{1.000000,0.833333,0.833333}%
\pgfsetfillcolor{currentfill}%
\pgfsetlinewidth{1.003750pt}%
\definecolor{currentstroke}{rgb}{1.000000,0.833333,0.833333}%
\pgfsetstrokecolor{currentstroke}%
\pgfsetdash{}{0pt}%
\pgfpathmoveto{\pgfqpoint{1.584297in}{0.588432in}}%
\pgfpathcurveto{\pgfqpoint{1.592533in}{0.588432in}}{\pgfqpoint{1.600433in}{0.591704in}}{\pgfqpoint{1.606257in}{0.597528in}}%
\pgfpathcurveto{\pgfqpoint{1.612081in}{0.603352in}}{\pgfqpoint{1.615353in}{0.611252in}}{\pgfqpoint{1.615353in}{0.619488in}}%
\pgfpathcurveto{\pgfqpoint{1.615353in}{0.627724in}}{\pgfqpoint{1.612081in}{0.635624in}}{\pgfqpoint{1.606257in}{0.641448in}}%
\pgfpathcurveto{\pgfqpoint{1.600433in}{0.647272in}}{\pgfqpoint{1.592533in}{0.650545in}}{\pgfqpoint{1.584297in}{0.650545in}}%
\pgfpathcurveto{\pgfqpoint{1.576060in}{0.650545in}}{\pgfqpoint{1.568160in}{0.647272in}}{\pgfqpoint{1.562336in}{0.641448in}}%
\pgfpathcurveto{\pgfqpoint{1.556512in}{0.635624in}}{\pgfqpoint{1.553240in}{0.627724in}}{\pgfqpoint{1.553240in}{0.619488in}}%
\pgfpathcurveto{\pgfqpoint{1.553240in}{0.611252in}}{\pgfqpoint{1.556512in}{0.603352in}}{\pgfqpoint{1.562336in}{0.597528in}}%
\pgfpathcurveto{\pgfqpoint{1.568160in}{0.591704in}}{\pgfqpoint{1.576060in}{0.588432in}}{\pgfqpoint{1.584297in}{0.588432in}}%
\pgfpathclose%
\pgfusepath{stroke,fill}%
\end{pgfscope}%
\begin{pgfscope}%
\pgfpathrectangle{\pgfqpoint{0.457963in}{0.528059in}}{\pgfqpoint{6.200000in}{2.285714in}} %
\pgfusepath{clip}%
\pgfsetbuttcap%
\pgfsetroundjoin%
\definecolor{currentfill}{rgb}{1.000000,0.833333,0.833333}%
\pgfsetfillcolor{currentfill}%
\pgfsetlinewidth{1.003750pt}%
\definecolor{currentstroke}{rgb}{1.000000,0.833333,0.833333}%
\pgfsetstrokecolor{currentstroke}%
\pgfsetdash{}{0pt}%
\pgfpathmoveto{\pgfqpoint{1.656630in}{0.758227in}}%
\pgfpathcurveto{\pgfqpoint{1.664866in}{0.758227in}}{\pgfqpoint{1.672766in}{0.761500in}}{\pgfqpoint{1.678590in}{0.767324in}}%
\pgfpathcurveto{\pgfqpoint{1.684414in}{0.773148in}}{\pgfqpoint{1.687686in}{0.781048in}}{\pgfqpoint{1.687686in}{0.789284in}}%
\pgfpathcurveto{\pgfqpoint{1.687686in}{0.797520in}}{\pgfqpoint{1.684414in}{0.805420in}}{\pgfqpoint{1.678590in}{0.811244in}}%
\pgfpathcurveto{\pgfqpoint{1.672766in}{0.817068in}}{\pgfqpoint{1.664866in}{0.820340in}}{\pgfqpoint{1.656630in}{0.820340in}}%
\pgfpathcurveto{\pgfqpoint{1.648394in}{0.820340in}}{\pgfqpoint{1.640494in}{0.817068in}}{\pgfqpoint{1.634670in}{0.811244in}}%
\pgfpathcurveto{\pgfqpoint{1.628846in}{0.805420in}}{\pgfqpoint{1.625574in}{0.797520in}}{\pgfqpoint{1.625574in}{0.789284in}}%
\pgfpathcurveto{\pgfqpoint{1.625574in}{0.781048in}}{\pgfqpoint{1.628846in}{0.773148in}}{\pgfqpoint{1.634670in}{0.767324in}}%
\pgfpathcurveto{\pgfqpoint{1.640494in}{0.761500in}}{\pgfqpoint{1.648394in}{0.758227in}}{\pgfqpoint{1.656630in}{0.758227in}}%
\pgfpathclose%
\pgfusepath{stroke,fill}%
\end{pgfscope}%
\begin{pgfscope}%
\pgfpathrectangle{\pgfqpoint{0.457963in}{0.528059in}}{\pgfqpoint{6.200000in}{2.285714in}} %
\pgfusepath{clip}%
\pgfsetbuttcap%
\pgfsetroundjoin%
\definecolor{currentfill}{rgb}{1.000000,0.833333,0.833333}%
\pgfsetfillcolor{currentfill}%
\pgfsetlinewidth{1.003750pt}%
\definecolor{currentstroke}{rgb}{1.000000,0.833333,0.833333}%
\pgfsetstrokecolor{currentstroke}%
\pgfsetdash{}{0pt}%
\pgfpathmoveto{\pgfqpoint{1.677297in}{0.705983in}}%
\pgfpathcurveto{\pgfqpoint{1.685533in}{0.705983in}}{\pgfqpoint{1.693433in}{0.709255in}}{\pgfqpoint{1.699257in}{0.715079in}}%
\pgfpathcurveto{\pgfqpoint{1.705081in}{0.720903in}}{\pgfqpoint{1.708353in}{0.728803in}}{\pgfqpoint{1.708353in}{0.737039in}}%
\pgfpathcurveto{\pgfqpoint{1.708353in}{0.745275in}}{\pgfqpoint{1.705081in}{0.753175in}}{\pgfqpoint{1.699257in}{0.758999in}}%
\pgfpathcurveto{\pgfqpoint{1.693433in}{0.764823in}}{\pgfqpoint{1.685533in}{0.768096in}}{\pgfqpoint{1.677297in}{0.768096in}}%
\pgfpathcurveto{\pgfqpoint{1.669060in}{0.768096in}}{\pgfqpoint{1.661160in}{0.764823in}}{\pgfqpoint{1.655336in}{0.758999in}}%
\pgfpathcurveto{\pgfqpoint{1.649512in}{0.753175in}}{\pgfqpoint{1.646240in}{0.745275in}}{\pgfqpoint{1.646240in}{0.737039in}}%
\pgfpathcurveto{\pgfqpoint{1.646240in}{0.728803in}}{\pgfqpoint{1.649512in}{0.720903in}}{\pgfqpoint{1.655336in}{0.715079in}}%
\pgfpathcurveto{\pgfqpoint{1.661160in}{0.709255in}}{\pgfqpoint{1.669060in}{0.705983in}}{\pgfqpoint{1.677297in}{0.705983in}}%
\pgfpathclose%
\pgfusepath{stroke,fill}%
\end{pgfscope}%
\begin{pgfscope}%
\pgfpathrectangle{\pgfqpoint{0.457963in}{0.528059in}}{\pgfqpoint{6.200000in}{2.285714in}} %
\pgfusepath{clip}%
\pgfsetbuttcap%
\pgfsetroundjoin%
\definecolor{currentfill}{rgb}{1.000000,0.833333,0.833333}%
\pgfsetfillcolor{currentfill}%
\pgfsetlinewidth{1.003750pt}%
\definecolor{currentstroke}{rgb}{1.000000,0.833333,0.833333}%
\pgfsetstrokecolor{currentstroke}%
\pgfsetdash{}{0pt}%
\pgfpathmoveto{\pgfqpoint{1.759963in}{0.784350in}}%
\pgfpathcurveto{\pgfqpoint{1.768200in}{0.784350in}}{\pgfqpoint{1.776100in}{0.787622in}}{\pgfqpoint{1.781924in}{0.793446in}}%
\pgfpathcurveto{\pgfqpoint{1.787748in}{0.799270in}}{\pgfqpoint{1.791020in}{0.807170in}}{\pgfqpoint{1.791020in}{0.815406in}}%
\pgfpathcurveto{\pgfqpoint{1.791020in}{0.823643in}}{\pgfqpoint{1.787748in}{0.831543in}}{\pgfqpoint{1.781924in}{0.837367in}}%
\pgfpathcurveto{\pgfqpoint{1.776100in}{0.843191in}}{\pgfqpoint{1.768200in}{0.846463in}}{\pgfqpoint{1.759963in}{0.846463in}}%
\pgfpathcurveto{\pgfqpoint{1.751727in}{0.846463in}}{\pgfqpoint{1.743827in}{0.843191in}}{\pgfqpoint{1.738003in}{0.837367in}}%
\pgfpathcurveto{\pgfqpoint{1.732179in}{0.831543in}}{\pgfqpoint{1.728907in}{0.823643in}}{\pgfqpoint{1.728907in}{0.815406in}}%
\pgfpathcurveto{\pgfqpoint{1.728907in}{0.807170in}}{\pgfqpoint{1.732179in}{0.799270in}}{\pgfqpoint{1.738003in}{0.793446in}}%
\pgfpathcurveto{\pgfqpoint{1.743827in}{0.787622in}}{\pgfqpoint{1.751727in}{0.784350in}}{\pgfqpoint{1.759963in}{0.784350in}}%
\pgfpathclose%
\pgfusepath{stroke,fill}%
\end{pgfscope}%
\begin{pgfscope}%
\pgfpathrectangle{\pgfqpoint{0.457963in}{0.528059in}}{\pgfqpoint{6.200000in}{2.285714in}} %
\pgfusepath{clip}%
\pgfsetbuttcap%
\pgfsetroundjoin%
\definecolor{currentfill}{rgb}{1.000000,0.666667,0.666667}%
\pgfsetfillcolor{currentfill}%
\pgfsetlinewidth{1.003750pt}%
\definecolor{currentstroke}{rgb}{1.000000,0.666667,0.666667}%
\pgfsetstrokecolor{currentstroke}%
\pgfsetdash{}{0pt}%
\pgfpathmoveto{\pgfqpoint{0.457963in}{1.150064in}}%
\pgfpathcurveto{\pgfqpoint{0.466200in}{1.150064in}}{\pgfqpoint{0.474100in}{1.153336in}}{\pgfqpoint{0.479924in}{1.159160in}}%
\pgfpathcurveto{\pgfqpoint{0.485748in}{1.164984in}}{\pgfqpoint{0.489020in}{1.172884in}}{\pgfqpoint{0.489020in}{1.181121in}}%
\pgfpathcurveto{\pgfqpoint{0.489020in}{1.189357in}}{\pgfqpoint{0.485748in}{1.197257in}}{\pgfqpoint{0.479924in}{1.203081in}}%
\pgfpathcurveto{\pgfqpoint{0.474100in}{1.208905in}}{\pgfqpoint{0.466200in}{1.212177in}}{\pgfqpoint{0.457963in}{1.212177in}}%
\pgfpathcurveto{\pgfqpoint{0.449727in}{1.212177in}}{\pgfqpoint{0.441827in}{1.208905in}}{\pgfqpoint{0.436003in}{1.203081in}}%
\pgfpathcurveto{\pgfqpoint{0.430179in}{1.197257in}}{\pgfqpoint{0.426907in}{1.189357in}}{\pgfqpoint{0.426907in}{1.181121in}}%
\pgfpathcurveto{\pgfqpoint{0.426907in}{1.172884in}}{\pgfqpoint{0.430179in}{1.164984in}}{\pgfqpoint{0.436003in}{1.159160in}}%
\pgfpathcurveto{\pgfqpoint{0.441827in}{1.153336in}}{\pgfqpoint{0.449727in}{1.150064in}}{\pgfqpoint{0.457963in}{1.150064in}}%
\pgfpathclose%
\pgfusepath{stroke,fill}%
\end{pgfscope}%
\begin{pgfscope}%
\pgfpathrectangle{\pgfqpoint{0.457963in}{0.528059in}}{\pgfqpoint{6.200000in}{2.285714in}} %
\pgfusepath{clip}%
\pgfsetbuttcap%
\pgfsetroundjoin%
\definecolor{currentfill}{rgb}{1.000000,0.666667,0.666667}%
\pgfsetfillcolor{currentfill}%
\pgfsetlinewidth{1.003750pt}%
\definecolor{currentstroke}{rgb}{1.000000,0.666667,0.666667}%
\pgfsetstrokecolor{currentstroke}%
\pgfsetdash{}{0pt}%
\pgfpathmoveto{\pgfqpoint{0.457963in}{1.150064in}}%
\pgfpathcurveto{\pgfqpoint{0.466200in}{1.150064in}}{\pgfqpoint{0.474100in}{1.153336in}}{\pgfqpoint{0.479924in}{1.159160in}}%
\pgfpathcurveto{\pgfqpoint{0.485748in}{1.164984in}}{\pgfqpoint{0.489020in}{1.172884in}}{\pgfqpoint{0.489020in}{1.181121in}}%
\pgfpathcurveto{\pgfqpoint{0.489020in}{1.189357in}}{\pgfqpoint{0.485748in}{1.197257in}}{\pgfqpoint{0.479924in}{1.203081in}}%
\pgfpathcurveto{\pgfqpoint{0.474100in}{1.208905in}}{\pgfqpoint{0.466200in}{1.212177in}}{\pgfqpoint{0.457963in}{1.212177in}}%
\pgfpathcurveto{\pgfqpoint{0.449727in}{1.212177in}}{\pgfqpoint{0.441827in}{1.208905in}}{\pgfqpoint{0.436003in}{1.203081in}}%
\pgfpathcurveto{\pgfqpoint{0.430179in}{1.197257in}}{\pgfqpoint{0.426907in}{1.189357in}}{\pgfqpoint{0.426907in}{1.181121in}}%
\pgfpathcurveto{\pgfqpoint{0.426907in}{1.172884in}}{\pgfqpoint{0.430179in}{1.164984in}}{\pgfqpoint{0.436003in}{1.159160in}}%
\pgfpathcurveto{\pgfqpoint{0.441827in}{1.153336in}}{\pgfqpoint{0.449727in}{1.150064in}}{\pgfqpoint{0.457963in}{1.150064in}}%
\pgfpathclose%
\pgfusepath{stroke,fill}%
\end{pgfscope}%
\begin{pgfscope}%
\pgfpathrectangle{\pgfqpoint{0.457963in}{0.528059in}}{\pgfqpoint{6.200000in}{2.285714in}} %
\pgfusepath{clip}%
\pgfsetbuttcap%
\pgfsetroundjoin%
\definecolor{currentfill}{rgb}{1.000000,0.666667,0.666667}%
\pgfsetfillcolor{currentfill}%
\pgfsetlinewidth{1.003750pt}%
\definecolor{currentstroke}{rgb}{1.000000,0.666667,0.666667}%
\pgfsetstrokecolor{currentstroke}%
\pgfsetdash{}{0pt}%
\pgfpathmoveto{\pgfqpoint{0.457963in}{1.150064in}}%
\pgfpathcurveto{\pgfqpoint{0.466200in}{1.150064in}}{\pgfqpoint{0.474100in}{1.153336in}}{\pgfqpoint{0.479924in}{1.159160in}}%
\pgfpathcurveto{\pgfqpoint{0.485748in}{1.164984in}}{\pgfqpoint{0.489020in}{1.172884in}}{\pgfqpoint{0.489020in}{1.181121in}}%
\pgfpathcurveto{\pgfqpoint{0.489020in}{1.189357in}}{\pgfqpoint{0.485748in}{1.197257in}}{\pgfqpoint{0.479924in}{1.203081in}}%
\pgfpathcurveto{\pgfqpoint{0.474100in}{1.208905in}}{\pgfqpoint{0.466200in}{1.212177in}}{\pgfqpoint{0.457963in}{1.212177in}}%
\pgfpathcurveto{\pgfqpoint{0.449727in}{1.212177in}}{\pgfqpoint{0.441827in}{1.208905in}}{\pgfqpoint{0.436003in}{1.203081in}}%
\pgfpathcurveto{\pgfqpoint{0.430179in}{1.197257in}}{\pgfqpoint{0.426907in}{1.189357in}}{\pgfqpoint{0.426907in}{1.181121in}}%
\pgfpathcurveto{\pgfqpoint{0.426907in}{1.172884in}}{\pgfqpoint{0.430179in}{1.164984in}}{\pgfqpoint{0.436003in}{1.159160in}}%
\pgfpathcurveto{\pgfqpoint{0.441827in}{1.153336in}}{\pgfqpoint{0.449727in}{1.150064in}}{\pgfqpoint{0.457963in}{1.150064in}}%
\pgfpathclose%
\pgfusepath{stroke,fill}%
\end{pgfscope}%
\begin{pgfscope}%
\pgfpathrectangle{\pgfqpoint{0.457963in}{0.528059in}}{\pgfqpoint{6.200000in}{2.285714in}} %
\pgfusepath{clip}%
\pgfsetbuttcap%
\pgfsetroundjoin%
\definecolor{currentfill}{rgb}{1.000000,0.666667,0.666667}%
\pgfsetfillcolor{currentfill}%
\pgfsetlinewidth{1.003750pt}%
\definecolor{currentstroke}{rgb}{1.000000,0.666667,0.666667}%
\pgfsetstrokecolor{currentstroke}%
\pgfsetdash{}{0pt}%
\pgfpathmoveto{\pgfqpoint{0.457963in}{1.150064in}}%
\pgfpathcurveto{\pgfqpoint{0.466200in}{1.150064in}}{\pgfqpoint{0.474100in}{1.153336in}}{\pgfqpoint{0.479924in}{1.159160in}}%
\pgfpathcurveto{\pgfqpoint{0.485748in}{1.164984in}}{\pgfqpoint{0.489020in}{1.172884in}}{\pgfqpoint{0.489020in}{1.181121in}}%
\pgfpathcurveto{\pgfqpoint{0.489020in}{1.189357in}}{\pgfqpoint{0.485748in}{1.197257in}}{\pgfqpoint{0.479924in}{1.203081in}}%
\pgfpathcurveto{\pgfqpoint{0.474100in}{1.208905in}}{\pgfqpoint{0.466200in}{1.212177in}}{\pgfqpoint{0.457963in}{1.212177in}}%
\pgfpathcurveto{\pgfqpoint{0.449727in}{1.212177in}}{\pgfqpoint{0.441827in}{1.208905in}}{\pgfqpoint{0.436003in}{1.203081in}}%
\pgfpathcurveto{\pgfqpoint{0.430179in}{1.197257in}}{\pgfqpoint{0.426907in}{1.189357in}}{\pgfqpoint{0.426907in}{1.181121in}}%
\pgfpathcurveto{\pgfqpoint{0.426907in}{1.172884in}}{\pgfqpoint{0.430179in}{1.164984in}}{\pgfqpoint{0.436003in}{1.159160in}}%
\pgfpathcurveto{\pgfqpoint{0.441827in}{1.153336in}}{\pgfqpoint{0.449727in}{1.150064in}}{\pgfqpoint{0.457963in}{1.150064in}}%
\pgfpathclose%
\pgfusepath{stroke,fill}%
\end{pgfscope}%
\begin{pgfscope}%
\pgfpathrectangle{\pgfqpoint{0.457963in}{0.528059in}}{\pgfqpoint{6.200000in}{2.285714in}} %
\pgfusepath{clip}%
\pgfsetbuttcap%
\pgfsetroundjoin%
\definecolor{currentfill}{rgb}{1.000000,0.666667,0.666667}%
\pgfsetfillcolor{currentfill}%
\pgfsetlinewidth{1.003750pt}%
\definecolor{currentstroke}{rgb}{1.000000,0.666667,0.666667}%
\pgfsetstrokecolor{currentstroke}%
\pgfsetdash{}{0pt}%
\pgfpathmoveto{\pgfqpoint{0.457963in}{1.150064in}}%
\pgfpathcurveto{\pgfqpoint{0.466200in}{1.150064in}}{\pgfqpoint{0.474100in}{1.153336in}}{\pgfqpoint{0.479924in}{1.159160in}}%
\pgfpathcurveto{\pgfqpoint{0.485748in}{1.164984in}}{\pgfqpoint{0.489020in}{1.172884in}}{\pgfqpoint{0.489020in}{1.181121in}}%
\pgfpathcurveto{\pgfqpoint{0.489020in}{1.189357in}}{\pgfqpoint{0.485748in}{1.197257in}}{\pgfqpoint{0.479924in}{1.203081in}}%
\pgfpathcurveto{\pgfqpoint{0.474100in}{1.208905in}}{\pgfqpoint{0.466200in}{1.212177in}}{\pgfqpoint{0.457963in}{1.212177in}}%
\pgfpathcurveto{\pgfqpoint{0.449727in}{1.212177in}}{\pgfqpoint{0.441827in}{1.208905in}}{\pgfqpoint{0.436003in}{1.203081in}}%
\pgfpathcurveto{\pgfqpoint{0.430179in}{1.197257in}}{\pgfqpoint{0.426907in}{1.189357in}}{\pgfqpoint{0.426907in}{1.181121in}}%
\pgfpathcurveto{\pgfqpoint{0.426907in}{1.172884in}}{\pgfqpoint{0.430179in}{1.164984in}}{\pgfqpoint{0.436003in}{1.159160in}}%
\pgfpathcurveto{\pgfqpoint{0.441827in}{1.153336in}}{\pgfqpoint{0.449727in}{1.150064in}}{\pgfqpoint{0.457963in}{1.150064in}}%
\pgfpathclose%
\pgfusepath{stroke,fill}%
\end{pgfscope}%
\begin{pgfscope}%
\pgfpathrectangle{\pgfqpoint{0.457963in}{0.528059in}}{\pgfqpoint{6.200000in}{2.285714in}} %
\pgfusepath{clip}%
\pgfsetbuttcap%
\pgfsetroundjoin%
\definecolor{currentfill}{rgb}{1.000000,0.666667,0.666667}%
\pgfsetfillcolor{currentfill}%
\pgfsetlinewidth{1.003750pt}%
\definecolor{currentstroke}{rgb}{1.000000,0.666667,0.666667}%
\pgfsetstrokecolor{currentstroke}%
\pgfsetdash{}{0pt}%
\pgfpathmoveto{\pgfqpoint{0.468297in}{1.123942in}}%
\pgfpathcurveto{\pgfqpoint{0.476533in}{1.123942in}}{\pgfqpoint{0.484433in}{1.127214in}}{\pgfqpoint{0.490257in}{1.133038in}}%
\pgfpathcurveto{\pgfqpoint{0.496081in}{1.138862in}}{\pgfqpoint{0.499353in}{1.146762in}}{\pgfqpoint{0.499353in}{1.154998in}}%
\pgfpathcurveto{\pgfqpoint{0.499353in}{1.163234in}}{\pgfqpoint{0.496081in}{1.171135in}}{\pgfqpoint{0.490257in}{1.176958in}}%
\pgfpathcurveto{\pgfqpoint{0.484433in}{1.182782in}}{\pgfqpoint{0.476533in}{1.186055in}}{\pgfqpoint{0.468297in}{1.186055in}}%
\pgfpathcurveto{\pgfqpoint{0.460060in}{1.186055in}}{\pgfqpoint{0.452160in}{1.182782in}}{\pgfqpoint{0.446336in}{1.176958in}}%
\pgfpathcurveto{\pgfqpoint{0.440512in}{1.171135in}}{\pgfqpoint{0.437240in}{1.163234in}}{\pgfqpoint{0.437240in}{1.154998in}}%
\pgfpathcurveto{\pgfqpoint{0.437240in}{1.146762in}}{\pgfqpoint{0.440512in}{1.138862in}}{\pgfqpoint{0.446336in}{1.133038in}}%
\pgfpathcurveto{\pgfqpoint{0.452160in}{1.127214in}}{\pgfqpoint{0.460060in}{1.123942in}}{\pgfqpoint{0.468297in}{1.123942in}}%
\pgfpathclose%
\pgfusepath{stroke,fill}%
\end{pgfscope}%
\begin{pgfscope}%
\pgfpathrectangle{\pgfqpoint{0.457963in}{0.528059in}}{\pgfqpoint{6.200000in}{2.285714in}} %
\pgfusepath{clip}%
\pgfsetbuttcap%
\pgfsetroundjoin%
\definecolor{currentfill}{rgb}{1.000000,0.666667,0.666667}%
\pgfsetfillcolor{currentfill}%
\pgfsetlinewidth{1.003750pt}%
\definecolor{currentstroke}{rgb}{1.000000,0.666667,0.666667}%
\pgfsetstrokecolor{currentstroke}%
\pgfsetdash{}{0pt}%
\pgfpathmoveto{\pgfqpoint{0.561297in}{1.150064in}}%
\pgfpathcurveto{\pgfqpoint{0.569533in}{1.150064in}}{\pgfqpoint{0.577433in}{1.153336in}}{\pgfqpoint{0.583257in}{1.159160in}}%
\pgfpathcurveto{\pgfqpoint{0.589081in}{1.164984in}}{\pgfqpoint{0.592353in}{1.172884in}}{\pgfqpoint{0.592353in}{1.181121in}}%
\pgfpathcurveto{\pgfqpoint{0.592353in}{1.189357in}}{\pgfqpoint{0.589081in}{1.197257in}}{\pgfqpoint{0.583257in}{1.203081in}}%
\pgfpathcurveto{\pgfqpoint{0.577433in}{1.208905in}}{\pgfqpoint{0.569533in}{1.212177in}}{\pgfqpoint{0.561297in}{1.212177in}}%
\pgfpathcurveto{\pgfqpoint{0.553060in}{1.212177in}}{\pgfqpoint{0.545160in}{1.208905in}}{\pgfqpoint{0.539336in}{1.203081in}}%
\pgfpathcurveto{\pgfqpoint{0.533512in}{1.197257in}}{\pgfqpoint{0.530240in}{1.189357in}}{\pgfqpoint{0.530240in}{1.181121in}}%
\pgfpathcurveto{\pgfqpoint{0.530240in}{1.172884in}}{\pgfqpoint{0.533512in}{1.164984in}}{\pgfqpoint{0.539336in}{1.159160in}}%
\pgfpathcurveto{\pgfqpoint{0.545160in}{1.153336in}}{\pgfqpoint{0.553060in}{1.150064in}}{\pgfqpoint{0.561297in}{1.150064in}}%
\pgfpathclose%
\pgfusepath{stroke,fill}%
\end{pgfscope}%
\begin{pgfscope}%
\pgfpathrectangle{\pgfqpoint{0.457963in}{0.528059in}}{\pgfqpoint{6.200000in}{2.285714in}} %
\pgfusepath{clip}%
\pgfsetbuttcap%
\pgfsetroundjoin%
\definecolor{currentfill}{rgb}{1.000000,0.666667,0.666667}%
\pgfsetfillcolor{currentfill}%
\pgfsetlinewidth{1.003750pt}%
\definecolor{currentstroke}{rgb}{1.000000,0.666667,0.666667}%
\pgfsetstrokecolor{currentstroke}%
\pgfsetdash{}{0pt}%
\pgfpathmoveto{\pgfqpoint{0.664630in}{0.914962in}}%
\pgfpathcurveto{\pgfqpoint{0.672866in}{0.914962in}}{\pgfqpoint{0.680766in}{0.918234in}}{\pgfqpoint{0.686590in}{0.924058in}}%
\pgfpathcurveto{\pgfqpoint{0.692414in}{0.929882in}}{\pgfqpoint{0.695686in}{0.937782in}}{\pgfqpoint{0.695686in}{0.946019in}}%
\pgfpathcurveto{\pgfqpoint{0.695686in}{0.954255in}}{\pgfqpoint{0.692414in}{0.962155in}}{\pgfqpoint{0.686590in}{0.967979in}}%
\pgfpathcurveto{\pgfqpoint{0.680766in}{0.973803in}}{\pgfqpoint{0.672866in}{0.977075in}}{\pgfqpoint{0.664630in}{0.977075in}}%
\pgfpathcurveto{\pgfqpoint{0.656394in}{0.977075in}}{\pgfqpoint{0.648494in}{0.973803in}}{\pgfqpoint{0.642670in}{0.967979in}}%
\pgfpathcurveto{\pgfqpoint{0.636846in}{0.962155in}}{\pgfqpoint{0.633574in}{0.954255in}}{\pgfqpoint{0.633574in}{0.946019in}}%
\pgfpathcurveto{\pgfqpoint{0.633574in}{0.937782in}}{\pgfqpoint{0.636846in}{0.929882in}}{\pgfqpoint{0.642670in}{0.924058in}}%
\pgfpathcurveto{\pgfqpoint{0.648494in}{0.918234in}}{\pgfqpoint{0.656394in}{0.914962in}}{\pgfqpoint{0.664630in}{0.914962in}}%
\pgfpathclose%
\pgfusepath{stroke,fill}%
\end{pgfscope}%
\begin{pgfscope}%
\pgfpathrectangle{\pgfqpoint{0.457963in}{0.528059in}}{\pgfqpoint{6.200000in}{2.285714in}} %
\pgfusepath{clip}%
\pgfsetbuttcap%
\pgfsetroundjoin%
\definecolor{currentfill}{rgb}{1.000000,0.666667,0.666667}%
\pgfsetfillcolor{currentfill}%
\pgfsetlinewidth{1.003750pt}%
\definecolor{currentstroke}{rgb}{1.000000,0.666667,0.666667}%
\pgfsetstrokecolor{currentstroke}%
\pgfsetdash{}{0pt}%
\pgfpathmoveto{\pgfqpoint{0.664630in}{1.071697in}}%
\pgfpathcurveto{\pgfqpoint{0.672866in}{1.071697in}}{\pgfqpoint{0.680766in}{1.074969in}}{\pgfqpoint{0.686590in}{1.080793in}}%
\pgfpathcurveto{\pgfqpoint{0.692414in}{1.086617in}}{\pgfqpoint{0.695686in}{1.094517in}}{\pgfqpoint{0.695686in}{1.102753in}}%
\pgfpathcurveto{\pgfqpoint{0.695686in}{1.110990in}}{\pgfqpoint{0.692414in}{1.118890in}}{\pgfqpoint{0.686590in}{1.124714in}}%
\pgfpathcurveto{\pgfqpoint{0.680766in}{1.130538in}}{\pgfqpoint{0.672866in}{1.133810in}}{\pgfqpoint{0.664630in}{1.133810in}}%
\pgfpathcurveto{\pgfqpoint{0.656394in}{1.133810in}}{\pgfqpoint{0.648494in}{1.130538in}}{\pgfqpoint{0.642670in}{1.124714in}}%
\pgfpathcurveto{\pgfqpoint{0.636846in}{1.118890in}}{\pgfqpoint{0.633574in}{1.110990in}}{\pgfqpoint{0.633574in}{1.102753in}}%
\pgfpathcurveto{\pgfqpoint{0.633574in}{1.094517in}}{\pgfqpoint{0.636846in}{1.086617in}}{\pgfqpoint{0.642670in}{1.080793in}}%
\pgfpathcurveto{\pgfqpoint{0.648494in}{1.074969in}}{\pgfqpoint{0.656394in}{1.071697in}}{\pgfqpoint{0.664630in}{1.071697in}}%
\pgfpathclose%
\pgfusepath{stroke,fill}%
\end{pgfscope}%
\begin{pgfscope}%
\pgfpathrectangle{\pgfqpoint{0.457963in}{0.528059in}}{\pgfqpoint{6.200000in}{2.285714in}} %
\pgfusepath{clip}%
\pgfsetbuttcap%
\pgfsetroundjoin%
\definecolor{currentfill}{rgb}{1.000000,0.666667,0.666667}%
\pgfsetfillcolor{currentfill}%
\pgfsetlinewidth{1.003750pt}%
\definecolor{currentstroke}{rgb}{1.000000,0.666667,0.666667}%
\pgfsetstrokecolor{currentstroke}%
\pgfsetdash{}{0pt}%
\pgfpathmoveto{\pgfqpoint{0.695630in}{0.993329in}}%
\pgfpathcurveto{\pgfqpoint{0.703866in}{0.993329in}}{\pgfqpoint{0.711766in}{0.996602in}}{\pgfqpoint{0.717590in}{1.002426in}}%
\pgfpathcurveto{\pgfqpoint{0.723414in}{1.008250in}}{\pgfqpoint{0.726686in}{1.016150in}}{\pgfqpoint{0.726686in}{1.024386in}}%
\pgfpathcurveto{\pgfqpoint{0.726686in}{1.032622in}}{\pgfqpoint{0.723414in}{1.040522in}}{\pgfqpoint{0.717590in}{1.046346in}}%
\pgfpathcurveto{\pgfqpoint{0.711766in}{1.052170in}}{\pgfqpoint{0.703866in}{1.055442in}}{\pgfqpoint{0.695630in}{1.055442in}}%
\pgfpathcurveto{\pgfqpoint{0.687394in}{1.055442in}}{\pgfqpoint{0.679494in}{1.052170in}}{\pgfqpoint{0.673670in}{1.046346in}}%
\pgfpathcurveto{\pgfqpoint{0.667846in}{1.040522in}}{\pgfqpoint{0.664574in}{1.032622in}}{\pgfqpoint{0.664574in}{1.024386in}}%
\pgfpathcurveto{\pgfqpoint{0.664574in}{1.016150in}}{\pgfqpoint{0.667846in}{1.008250in}}{\pgfqpoint{0.673670in}{1.002426in}}%
\pgfpathcurveto{\pgfqpoint{0.679494in}{0.996602in}}{\pgfqpoint{0.687394in}{0.993329in}}{\pgfqpoint{0.695630in}{0.993329in}}%
\pgfpathclose%
\pgfusepath{stroke,fill}%
\end{pgfscope}%
\begin{pgfscope}%
\pgfpathrectangle{\pgfqpoint{0.457963in}{0.528059in}}{\pgfqpoint{6.200000in}{2.285714in}} %
\pgfusepath{clip}%
\pgfsetbuttcap%
\pgfsetroundjoin%
\definecolor{currentfill}{rgb}{1.000000,0.666667,0.666667}%
\pgfsetfillcolor{currentfill}%
\pgfsetlinewidth{1.003750pt}%
\definecolor{currentstroke}{rgb}{1.000000,0.666667,0.666667}%
\pgfsetstrokecolor{currentstroke}%
\pgfsetdash{}{0pt}%
\pgfpathmoveto{\pgfqpoint{0.695630in}{1.137003in}}%
\pgfpathcurveto{\pgfqpoint{0.703866in}{1.137003in}}{\pgfqpoint{0.711766in}{1.140275in}}{\pgfqpoint{0.717590in}{1.146099in}}%
\pgfpathcurveto{\pgfqpoint{0.723414in}{1.151923in}}{\pgfqpoint{0.726686in}{1.159823in}}{\pgfqpoint{0.726686in}{1.168059in}}%
\pgfpathcurveto{\pgfqpoint{0.726686in}{1.176296in}}{\pgfqpoint{0.723414in}{1.184196in}}{\pgfqpoint{0.717590in}{1.190020in}}%
\pgfpathcurveto{\pgfqpoint{0.711766in}{1.195844in}}{\pgfqpoint{0.703866in}{1.199116in}}{\pgfqpoint{0.695630in}{1.199116in}}%
\pgfpathcurveto{\pgfqpoint{0.687394in}{1.199116in}}{\pgfqpoint{0.679494in}{1.195844in}}{\pgfqpoint{0.673670in}{1.190020in}}%
\pgfpathcurveto{\pgfqpoint{0.667846in}{1.184196in}}{\pgfqpoint{0.664574in}{1.176296in}}{\pgfqpoint{0.664574in}{1.168059in}}%
\pgfpathcurveto{\pgfqpoint{0.664574in}{1.159823in}}{\pgfqpoint{0.667846in}{1.151923in}}{\pgfqpoint{0.673670in}{1.146099in}}%
\pgfpathcurveto{\pgfqpoint{0.679494in}{1.140275in}}{\pgfqpoint{0.687394in}{1.137003in}}{\pgfqpoint{0.695630in}{1.137003in}}%
\pgfpathclose%
\pgfusepath{stroke,fill}%
\end{pgfscope}%
\begin{pgfscope}%
\pgfpathrectangle{\pgfqpoint{0.457963in}{0.528059in}}{\pgfqpoint{6.200000in}{2.285714in}} %
\pgfusepath{clip}%
\pgfsetbuttcap%
\pgfsetroundjoin%
\definecolor{currentfill}{rgb}{1.000000,0.666667,0.666667}%
\pgfsetfillcolor{currentfill}%
\pgfsetlinewidth{1.003750pt}%
\definecolor{currentstroke}{rgb}{1.000000,0.666667,0.666667}%
\pgfsetstrokecolor{currentstroke}%
\pgfsetdash{}{0pt}%
\pgfpathmoveto{\pgfqpoint{0.695630in}{1.150064in}}%
\pgfpathcurveto{\pgfqpoint{0.703866in}{1.150064in}}{\pgfqpoint{0.711766in}{1.153336in}}{\pgfqpoint{0.717590in}{1.159160in}}%
\pgfpathcurveto{\pgfqpoint{0.723414in}{1.164984in}}{\pgfqpoint{0.726686in}{1.172884in}}{\pgfqpoint{0.726686in}{1.181121in}}%
\pgfpathcurveto{\pgfqpoint{0.726686in}{1.189357in}}{\pgfqpoint{0.723414in}{1.197257in}}{\pgfqpoint{0.717590in}{1.203081in}}%
\pgfpathcurveto{\pgfqpoint{0.711766in}{1.208905in}}{\pgfqpoint{0.703866in}{1.212177in}}{\pgfqpoint{0.695630in}{1.212177in}}%
\pgfpathcurveto{\pgfqpoint{0.687394in}{1.212177in}}{\pgfqpoint{0.679494in}{1.208905in}}{\pgfqpoint{0.673670in}{1.203081in}}%
\pgfpathcurveto{\pgfqpoint{0.667846in}{1.197257in}}{\pgfqpoint{0.664574in}{1.189357in}}{\pgfqpoint{0.664574in}{1.181121in}}%
\pgfpathcurveto{\pgfqpoint{0.664574in}{1.172884in}}{\pgfqpoint{0.667846in}{1.164984in}}{\pgfqpoint{0.673670in}{1.159160in}}%
\pgfpathcurveto{\pgfqpoint{0.679494in}{1.153336in}}{\pgfqpoint{0.687394in}{1.150064in}}{\pgfqpoint{0.695630in}{1.150064in}}%
\pgfpathclose%
\pgfusepath{stroke,fill}%
\end{pgfscope}%
\begin{pgfscope}%
\pgfpathrectangle{\pgfqpoint{0.457963in}{0.528059in}}{\pgfqpoint{6.200000in}{2.285714in}} %
\pgfusepath{clip}%
\pgfsetbuttcap%
\pgfsetroundjoin%
\definecolor{currentfill}{rgb}{1.000000,0.666667,0.666667}%
\pgfsetfillcolor{currentfill}%
\pgfsetlinewidth{1.003750pt}%
\definecolor{currentstroke}{rgb}{1.000000,0.666667,0.666667}%
\pgfsetstrokecolor{currentstroke}%
\pgfsetdash{}{0pt}%
\pgfpathmoveto{\pgfqpoint{0.736963in}{1.019452in}}%
\pgfpathcurveto{\pgfqpoint{0.745200in}{1.019452in}}{\pgfqpoint{0.753100in}{1.022724in}}{\pgfqpoint{0.758924in}{1.028548in}}%
\pgfpathcurveto{\pgfqpoint{0.764748in}{1.034372in}}{\pgfqpoint{0.768020in}{1.042272in}}{\pgfqpoint{0.768020in}{1.050508in}}%
\pgfpathcurveto{\pgfqpoint{0.768020in}{1.058745in}}{\pgfqpoint{0.764748in}{1.066645in}}{\pgfqpoint{0.758924in}{1.072469in}}%
\pgfpathcurveto{\pgfqpoint{0.753100in}{1.078293in}}{\pgfqpoint{0.745200in}{1.081565in}}{\pgfqpoint{0.736963in}{1.081565in}}%
\pgfpathcurveto{\pgfqpoint{0.728727in}{1.081565in}}{\pgfqpoint{0.720827in}{1.078293in}}{\pgfqpoint{0.715003in}{1.072469in}}%
\pgfpathcurveto{\pgfqpoint{0.709179in}{1.066645in}}{\pgfqpoint{0.705907in}{1.058745in}}{\pgfqpoint{0.705907in}{1.050508in}}%
\pgfpathcurveto{\pgfqpoint{0.705907in}{1.042272in}}{\pgfqpoint{0.709179in}{1.034372in}}{\pgfqpoint{0.715003in}{1.028548in}}%
\pgfpathcurveto{\pgfqpoint{0.720827in}{1.022724in}}{\pgfqpoint{0.728727in}{1.019452in}}{\pgfqpoint{0.736963in}{1.019452in}}%
\pgfpathclose%
\pgfusepath{stroke,fill}%
\end{pgfscope}%
\begin{pgfscope}%
\pgfpathrectangle{\pgfqpoint{0.457963in}{0.528059in}}{\pgfqpoint{6.200000in}{2.285714in}} %
\pgfusepath{clip}%
\pgfsetbuttcap%
\pgfsetroundjoin%
\definecolor{currentfill}{rgb}{1.000000,0.666667,0.666667}%
\pgfsetfillcolor{currentfill}%
\pgfsetlinewidth{1.003750pt}%
\definecolor{currentstroke}{rgb}{1.000000,0.666667,0.666667}%
\pgfsetstrokecolor{currentstroke}%
\pgfsetdash{}{0pt}%
\pgfpathmoveto{\pgfqpoint{0.953963in}{1.137003in}}%
\pgfpathcurveto{\pgfqpoint{0.962200in}{1.137003in}}{\pgfqpoint{0.970100in}{1.140275in}}{\pgfqpoint{0.975924in}{1.146099in}}%
\pgfpathcurveto{\pgfqpoint{0.981748in}{1.151923in}}{\pgfqpoint{0.985020in}{1.159823in}}{\pgfqpoint{0.985020in}{1.168059in}}%
\pgfpathcurveto{\pgfqpoint{0.985020in}{1.176296in}}{\pgfqpoint{0.981748in}{1.184196in}}{\pgfqpoint{0.975924in}{1.190020in}}%
\pgfpathcurveto{\pgfqpoint{0.970100in}{1.195844in}}{\pgfqpoint{0.962200in}{1.199116in}}{\pgfqpoint{0.953963in}{1.199116in}}%
\pgfpathcurveto{\pgfqpoint{0.945727in}{1.199116in}}{\pgfqpoint{0.937827in}{1.195844in}}{\pgfqpoint{0.932003in}{1.190020in}}%
\pgfpathcurveto{\pgfqpoint{0.926179in}{1.184196in}}{\pgfqpoint{0.922907in}{1.176296in}}{\pgfqpoint{0.922907in}{1.168059in}}%
\pgfpathcurveto{\pgfqpoint{0.922907in}{1.159823in}}{\pgfqpoint{0.926179in}{1.151923in}}{\pgfqpoint{0.932003in}{1.146099in}}%
\pgfpathcurveto{\pgfqpoint{0.937827in}{1.140275in}}{\pgfqpoint{0.945727in}{1.137003in}}{\pgfqpoint{0.953963in}{1.137003in}}%
\pgfpathclose%
\pgfusepath{stroke,fill}%
\end{pgfscope}%
\begin{pgfscope}%
\pgfpathrectangle{\pgfqpoint{0.457963in}{0.528059in}}{\pgfqpoint{6.200000in}{2.285714in}} %
\pgfusepath{clip}%
\pgfsetbuttcap%
\pgfsetroundjoin%
\definecolor{currentfill}{rgb}{1.000000,0.666667,0.666667}%
\pgfsetfillcolor{currentfill}%
\pgfsetlinewidth{1.003750pt}%
\definecolor{currentstroke}{rgb}{1.000000,0.666667,0.666667}%
\pgfsetstrokecolor{currentstroke}%
\pgfsetdash{}{0pt}%
\pgfpathmoveto{\pgfqpoint{1.222630in}{1.137003in}}%
\pgfpathcurveto{\pgfqpoint{1.230866in}{1.137003in}}{\pgfqpoint{1.238766in}{1.140275in}}{\pgfqpoint{1.244590in}{1.146099in}}%
\pgfpathcurveto{\pgfqpoint{1.250414in}{1.151923in}}{\pgfqpoint{1.253686in}{1.159823in}}{\pgfqpoint{1.253686in}{1.168059in}}%
\pgfpathcurveto{\pgfqpoint{1.253686in}{1.176296in}}{\pgfqpoint{1.250414in}{1.184196in}}{\pgfqpoint{1.244590in}{1.190020in}}%
\pgfpathcurveto{\pgfqpoint{1.238766in}{1.195844in}}{\pgfqpoint{1.230866in}{1.199116in}}{\pgfqpoint{1.222630in}{1.199116in}}%
\pgfpathcurveto{\pgfqpoint{1.214394in}{1.199116in}}{\pgfqpoint{1.206494in}{1.195844in}}{\pgfqpoint{1.200670in}{1.190020in}}%
\pgfpathcurveto{\pgfqpoint{1.194846in}{1.184196in}}{\pgfqpoint{1.191574in}{1.176296in}}{\pgfqpoint{1.191574in}{1.168059in}}%
\pgfpathcurveto{\pgfqpoint{1.191574in}{1.159823in}}{\pgfqpoint{1.194846in}{1.151923in}}{\pgfqpoint{1.200670in}{1.146099in}}%
\pgfpathcurveto{\pgfqpoint{1.206494in}{1.140275in}}{\pgfqpoint{1.214394in}{1.137003in}}{\pgfqpoint{1.222630in}{1.137003in}}%
\pgfpathclose%
\pgfusepath{stroke,fill}%
\end{pgfscope}%
\begin{pgfscope}%
\pgfpathrectangle{\pgfqpoint{0.457963in}{0.528059in}}{\pgfqpoint{6.200000in}{2.285714in}} %
\pgfusepath{clip}%
\pgfsetbuttcap%
\pgfsetroundjoin%
\definecolor{currentfill}{rgb}{1.000000,0.666667,0.666667}%
\pgfsetfillcolor{currentfill}%
\pgfsetlinewidth{1.003750pt}%
\definecolor{currentstroke}{rgb}{1.000000,0.666667,0.666667}%
\pgfsetstrokecolor{currentstroke}%
\pgfsetdash{}{0pt}%
\pgfpathmoveto{\pgfqpoint{1.615297in}{1.137003in}}%
\pgfpathcurveto{\pgfqpoint{1.623533in}{1.137003in}}{\pgfqpoint{1.631433in}{1.140275in}}{\pgfqpoint{1.637257in}{1.146099in}}%
\pgfpathcurveto{\pgfqpoint{1.643081in}{1.151923in}}{\pgfqpoint{1.646353in}{1.159823in}}{\pgfqpoint{1.646353in}{1.168059in}}%
\pgfpathcurveto{\pgfqpoint{1.646353in}{1.176296in}}{\pgfqpoint{1.643081in}{1.184196in}}{\pgfqpoint{1.637257in}{1.190020in}}%
\pgfpathcurveto{\pgfqpoint{1.631433in}{1.195844in}}{\pgfqpoint{1.623533in}{1.199116in}}{\pgfqpoint{1.615297in}{1.199116in}}%
\pgfpathcurveto{\pgfqpoint{1.607060in}{1.199116in}}{\pgfqpoint{1.599160in}{1.195844in}}{\pgfqpoint{1.593336in}{1.190020in}}%
\pgfpathcurveto{\pgfqpoint{1.587512in}{1.184196in}}{\pgfqpoint{1.584240in}{1.176296in}}{\pgfqpoint{1.584240in}{1.168059in}}%
\pgfpathcurveto{\pgfqpoint{1.584240in}{1.159823in}}{\pgfqpoint{1.587512in}{1.151923in}}{\pgfqpoint{1.593336in}{1.146099in}}%
\pgfpathcurveto{\pgfqpoint{1.599160in}{1.140275in}}{\pgfqpoint{1.607060in}{1.137003in}}{\pgfqpoint{1.615297in}{1.137003in}}%
\pgfpathclose%
\pgfusepath{stroke,fill}%
\end{pgfscope}%
\begin{pgfscope}%
\pgfpathrectangle{\pgfqpoint{0.457963in}{0.528059in}}{\pgfqpoint{6.200000in}{2.285714in}} %
\pgfusepath{clip}%
\pgfsetbuttcap%
\pgfsetroundjoin%
\definecolor{currentfill}{rgb}{1.000000,0.666667,0.666667}%
\pgfsetfillcolor{currentfill}%
\pgfsetlinewidth{1.003750pt}%
\definecolor{currentstroke}{rgb}{1.000000,0.666667,0.666667}%
\pgfsetstrokecolor{currentstroke}%
\pgfsetdash{}{0pt}%
\pgfpathmoveto{\pgfqpoint{1.708297in}{1.071697in}}%
\pgfpathcurveto{\pgfqpoint{1.716533in}{1.071697in}}{\pgfqpoint{1.724433in}{1.074969in}}{\pgfqpoint{1.730257in}{1.080793in}}%
\pgfpathcurveto{\pgfqpoint{1.736081in}{1.086617in}}{\pgfqpoint{1.739353in}{1.094517in}}{\pgfqpoint{1.739353in}{1.102753in}}%
\pgfpathcurveto{\pgfqpoint{1.739353in}{1.110990in}}{\pgfqpoint{1.736081in}{1.118890in}}{\pgfqpoint{1.730257in}{1.124714in}}%
\pgfpathcurveto{\pgfqpoint{1.724433in}{1.130538in}}{\pgfqpoint{1.716533in}{1.133810in}}{\pgfqpoint{1.708297in}{1.133810in}}%
\pgfpathcurveto{\pgfqpoint{1.700060in}{1.133810in}}{\pgfqpoint{1.692160in}{1.130538in}}{\pgfqpoint{1.686336in}{1.124714in}}%
\pgfpathcurveto{\pgfqpoint{1.680512in}{1.118890in}}{\pgfqpoint{1.677240in}{1.110990in}}{\pgfqpoint{1.677240in}{1.102753in}}%
\pgfpathcurveto{\pgfqpoint{1.677240in}{1.094517in}}{\pgfqpoint{1.680512in}{1.086617in}}{\pgfqpoint{1.686336in}{1.080793in}}%
\pgfpathcurveto{\pgfqpoint{1.692160in}{1.074969in}}{\pgfqpoint{1.700060in}{1.071697in}}{\pgfqpoint{1.708297in}{1.071697in}}%
\pgfpathclose%
\pgfusepath{stroke,fill}%
\end{pgfscope}%
\begin{pgfscope}%
\pgfpathrectangle{\pgfqpoint{0.457963in}{0.528059in}}{\pgfqpoint{6.200000in}{2.285714in}} %
\pgfusepath{clip}%
\pgfsetbuttcap%
\pgfsetroundjoin%
\definecolor{currentfill}{rgb}{1.000000,0.666667,0.666667}%
\pgfsetfillcolor{currentfill}%
\pgfsetlinewidth{1.003750pt}%
\definecolor{currentstroke}{rgb}{1.000000,0.666667,0.666667}%
\pgfsetstrokecolor{currentstroke}%
\pgfsetdash{}{0pt}%
\pgfpathmoveto{\pgfqpoint{2.100963in}{0.993329in}}%
\pgfpathcurveto{\pgfqpoint{2.109200in}{0.993329in}}{\pgfqpoint{2.117100in}{0.996602in}}{\pgfqpoint{2.122924in}{1.002426in}}%
\pgfpathcurveto{\pgfqpoint{2.128748in}{1.008250in}}{\pgfqpoint{2.132020in}{1.016150in}}{\pgfqpoint{2.132020in}{1.024386in}}%
\pgfpathcurveto{\pgfqpoint{2.132020in}{1.032622in}}{\pgfqpoint{2.128748in}{1.040522in}}{\pgfqpoint{2.122924in}{1.046346in}}%
\pgfpathcurveto{\pgfqpoint{2.117100in}{1.052170in}}{\pgfqpoint{2.109200in}{1.055442in}}{\pgfqpoint{2.100963in}{1.055442in}}%
\pgfpathcurveto{\pgfqpoint{2.092727in}{1.055442in}}{\pgfqpoint{2.084827in}{1.052170in}}{\pgfqpoint{2.079003in}{1.046346in}}%
\pgfpathcurveto{\pgfqpoint{2.073179in}{1.040522in}}{\pgfqpoint{2.069907in}{1.032622in}}{\pgfqpoint{2.069907in}{1.024386in}}%
\pgfpathcurveto{\pgfqpoint{2.069907in}{1.016150in}}{\pgfqpoint{2.073179in}{1.008250in}}{\pgfqpoint{2.079003in}{1.002426in}}%
\pgfpathcurveto{\pgfqpoint{2.084827in}{0.996602in}}{\pgfqpoint{2.092727in}{0.993329in}}{\pgfqpoint{2.100963in}{0.993329in}}%
\pgfpathclose%
\pgfusepath{stroke,fill}%
\end{pgfscope}%
\begin{pgfscope}%
\pgfpathrectangle{\pgfqpoint{0.457963in}{0.528059in}}{\pgfqpoint{6.200000in}{2.285714in}} %
\pgfusepath{clip}%
\pgfsetbuttcap%
\pgfsetroundjoin%
\definecolor{currentfill}{rgb}{1.000000,0.666667,0.666667}%
\pgfsetfillcolor{currentfill}%
\pgfsetlinewidth{1.003750pt}%
\definecolor{currentstroke}{rgb}{1.000000,0.666667,0.666667}%
\pgfsetstrokecolor{currentstroke}%
\pgfsetdash{}{0pt}%
\pgfpathmoveto{\pgfqpoint{2.142297in}{0.627615in}}%
\pgfpathcurveto{\pgfqpoint{2.150533in}{0.627615in}}{\pgfqpoint{2.158433in}{0.630887in}}{\pgfqpoint{2.164257in}{0.636711in}}%
\pgfpathcurveto{\pgfqpoint{2.170081in}{0.642535in}}{\pgfqpoint{2.173353in}{0.650435in}}{\pgfqpoint{2.173353in}{0.658672in}}%
\pgfpathcurveto{\pgfqpoint{2.173353in}{0.666908in}}{\pgfqpoint{2.170081in}{0.674808in}}{\pgfqpoint{2.164257in}{0.680632in}}%
\pgfpathcurveto{\pgfqpoint{2.158433in}{0.686456in}}{\pgfqpoint{2.150533in}{0.689728in}}{\pgfqpoint{2.142297in}{0.689728in}}%
\pgfpathcurveto{\pgfqpoint{2.134060in}{0.689728in}}{\pgfqpoint{2.126160in}{0.686456in}}{\pgfqpoint{2.120336in}{0.680632in}}%
\pgfpathcurveto{\pgfqpoint{2.114512in}{0.674808in}}{\pgfqpoint{2.111240in}{0.666908in}}{\pgfqpoint{2.111240in}{0.658672in}}%
\pgfpathcurveto{\pgfqpoint{2.111240in}{0.650435in}}{\pgfqpoint{2.114512in}{0.642535in}}{\pgfqpoint{2.120336in}{0.636711in}}%
\pgfpathcurveto{\pgfqpoint{2.126160in}{0.630887in}}{\pgfqpoint{2.134060in}{0.627615in}}{\pgfqpoint{2.142297in}{0.627615in}}%
\pgfpathclose%
\pgfusepath{stroke,fill}%
\end{pgfscope}%
\begin{pgfscope}%
\pgfpathrectangle{\pgfqpoint{0.457963in}{0.528059in}}{\pgfqpoint{6.200000in}{2.285714in}} %
\pgfusepath{clip}%
\pgfsetbuttcap%
\pgfsetroundjoin%
\definecolor{currentfill}{rgb}{1.000000,0.666667,0.666667}%
\pgfsetfillcolor{currentfill}%
\pgfsetlinewidth{1.003750pt}%
\definecolor{currentstroke}{rgb}{1.000000,0.666667,0.666667}%
\pgfsetstrokecolor{currentstroke}%
\pgfsetdash{}{0pt}%
\pgfpathmoveto{\pgfqpoint{2.576297in}{0.875778in}}%
\pgfpathcurveto{\pgfqpoint{2.584533in}{0.875778in}}{\pgfqpoint{2.592433in}{0.879051in}}{\pgfqpoint{2.598257in}{0.884875in}}%
\pgfpathcurveto{\pgfqpoint{2.604081in}{0.890699in}}{\pgfqpoint{2.607353in}{0.898599in}}{\pgfqpoint{2.607353in}{0.906835in}}%
\pgfpathcurveto{\pgfqpoint{2.607353in}{0.915071in}}{\pgfqpoint{2.604081in}{0.922971in}}{\pgfqpoint{2.598257in}{0.928795in}}%
\pgfpathcurveto{\pgfqpoint{2.592433in}{0.934619in}}{\pgfqpoint{2.584533in}{0.937891in}}{\pgfqpoint{2.576297in}{0.937891in}}%
\pgfpathcurveto{\pgfqpoint{2.568060in}{0.937891in}}{\pgfqpoint{2.560160in}{0.934619in}}{\pgfqpoint{2.554336in}{0.928795in}}%
\pgfpathcurveto{\pgfqpoint{2.548512in}{0.922971in}}{\pgfqpoint{2.545240in}{0.915071in}}{\pgfqpoint{2.545240in}{0.906835in}}%
\pgfpathcurveto{\pgfqpoint{2.545240in}{0.898599in}}{\pgfqpoint{2.548512in}{0.890699in}}{\pgfqpoint{2.554336in}{0.884875in}}%
\pgfpathcurveto{\pgfqpoint{2.560160in}{0.879051in}}{\pgfqpoint{2.568060in}{0.875778in}}{\pgfqpoint{2.576297in}{0.875778in}}%
\pgfpathclose%
\pgfusepath{stroke,fill}%
\end{pgfscope}%
\begin{pgfscope}%
\pgfpathrectangle{\pgfqpoint{0.457963in}{0.528059in}}{\pgfqpoint{6.200000in}{2.285714in}} %
\pgfusepath{clip}%
\pgfsetbuttcap%
\pgfsetroundjoin%
\definecolor{currentfill}{rgb}{1.000000,0.500000,0.500000}%
\pgfsetfillcolor{currentfill}%
\pgfsetlinewidth{1.003750pt}%
\definecolor{currentstroke}{rgb}{1.000000,0.500000,0.500000}%
\pgfsetstrokecolor{currentstroke}%
\pgfsetdash{}{0pt}%
\pgfpathmoveto{\pgfqpoint{0.457963in}{1.476595in}}%
\pgfpathcurveto{\pgfqpoint{0.466200in}{1.476595in}}{\pgfqpoint{0.474100in}{1.479867in}}{\pgfqpoint{0.479924in}{1.485691in}}%
\pgfpathcurveto{\pgfqpoint{0.485748in}{1.491515in}}{\pgfqpoint{0.489020in}{1.499415in}}{\pgfqpoint{0.489020in}{1.507651in}}%
\pgfpathcurveto{\pgfqpoint{0.489020in}{1.515888in}}{\pgfqpoint{0.485748in}{1.523788in}}{\pgfqpoint{0.479924in}{1.529612in}}%
\pgfpathcurveto{\pgfqpoint{0.474100in}{1.535435in}}{\pgfqpoint{0.466200in}{1.538708in}}{\pgfqpoint{0.457963in}{1.538708in}}%
\pgfpathcurveto{\pgfqpoint{0.449727in}{1.538708in}}{\pgfqpoint{0.441827in}{1.535435in}}{\pgfqpoint{0.436003in}{1.529612in}}%
\pgfpathcurveto{\pgfqpoint{0.430179in}{1.523788in}}{\pgfqpoint{0.426907in}{1.515888in}}{\pgfqpoint{0.426907in}{1.507651in}}%
\pgfpathcurveto{\pgfqpoint{0.426907in}{1.499415in}}{\pgfqpoint{0.430179in}{1.491515in}}{\pgfqpoint{0.436003in}{1.485691in}}%
\pgfpathcurveto{\pgfqpoint{0.441827in}{1.479867in}}{\pgfqpoint{0.449727in}{1.476595in}}{\pgfqpoint{0.457963in}{1.476595in}}%
\pgfpathclose%
\pgfusepath{stroke,fill}%
\end{pgfscope}%
\begin{pgfscope}%
\pgfpathrectangle{\pgfqpoint{0.457963in}{0.528059in}}{\pgfqpoint{6.200000in}{2.285714in}} %
\pgfusepath{clip}%
\pgfsetbuttcap%
\pgfsetroundjoin%
\definecolor{currentfill}{rgb}{1.000000,0.500000,0.500000}%
\pgfsetfillcolor{currentfill}%
\pgfsetlinewidth{1.003750pt}%
\definecolor{currentstroke}{rgb}{1.000000,0.500000,0.500000}%
\pgfsetstrokecolor{currentstroke}%
\pgfsetdash{}{0pt}%
\pgfpathmoveto{\pgfqpoint{0.457963in}{1.476595in}}%
\pgfpathcurveto{\pgfqpoint{0.466200in}{1.476595in}}{\pgfqpoint{0.474100in}{1.479867in}}{\pgfqpoint{0.479924in}{1.485691in}}%
\pgfpathcurveto{\pgfqpoint{0.485748in}{1.491515in}}{\pgfqpoint{0.489020in}{1.499415in}}{\pgfqpoint{0.489020in}{1.507651in}}%
\pgfpathcurveto{\pgfqpoint{0.489020in}{1.515888in}}{\pgfqpoint{0.485748in}{1.523788in}}{\pgfqpoint{0.479924in}{1.529612in}}%
\pgfpathcurveto{\pgfqpoint{0.474100in}{1.535435in}}{\pgfqpoint{0.466200in}{1.538708in}}{\pgfqpoint{0.457963in}{1.538708in}}%
\pgfpathcurveto{\pgfqpoint{0.449727in}{1.538708in}}{\pgfqpoint{0.441827in}{1.535435in}}{\pgfqpoint{0.436003in}{1.529612in}}%
\pgfpathcurveto{\pgfqpoint{0.430179in}{1.523788in}}{\pgfqpoint{0.426907in}{1.515888in}}{\pgfqpoint{0.426907in}{1.507651in}}%
\pgfpathcurveto{\pgfqpoint{0.426907in}{1.499415in}}{\pgfqpoint{0.430179in}{1.491515in}}{\pgfqpoint{0.436003in}{1.485691in}}%
\pgfpathcurveto{\pgfqpoint{0.441827in}{1.479867in}}{\pgfqpoint{0.449727in}{1.476595in}}{\pgfqpoint{0.457963in}{1.476595in}}%
\pgfpathclose%
\pgfusepath{stroke,fill}%
\end{pgfscope}%
\begin{pgfscope}%
\pgfpathrectangle{\pgfqpoint{0.457963in}{0.528059in}}{\pgfqpoint{6.200000in}{2.285714in}} %
\pgfusepath{clip}%
\pgfsetbuttcap%
\pgfsetroundjoin%
\definecolor{currentfill}{rgb}{1.000000,0.500000,0.500000}%
\pgfsetfillcolor{currentfill}%
\pgfsetlinewidth{1.003750pt}%
\definecolor{currentstroke}{rgb}{1.000000,0.500000,0.500000}%
\pgfsetstrokecolor{currentstroke}%
\pgfsetdash{}{0pt}%
\pgfpathmoveto{\pgfqpoint{0.457963in}{1.476595in}}%
\pgfpathcurveto{\pgfqpoint{0.466200in}{1.476595in}}{\pgfqpoint{0.474100in}{1.479867in}}{\pgfqpoint{0.479924in}{1.485691in}}%
\pgfpathcurveto{\pgfqpoint{0.485748in}{1.491515in}}{\pgfqpoint{0.489020in}{1.499415in}}{\pgfqpoint{0.489020in}{1.507651in}}%
\pgfpathcurveto{\pgfqpoint{0.489020in}{1.515888in}}{\pgfqpoint{0.485748in}{1.523788in}}{\pgfqpoint{0.479924in}{1.529612in}}%
\pgfpathcurveto{\pgfqpoint{0.474100in}{1.535435in}}{\pgfqpoint{0.466200in}{1.538708in}}{\pgfqpoint{0.457963in}{1.538708in}}%
\pgfpathcurveto{\pgfqpoint{0.449727in}{1.538708in}}{\pgfqpoint{0.441827in}{1.535435in}}{\pgfqpoint{0.436003in}{1.529612in}}%
\pgfpathcurveto{\pgfqpoint{0.430179in}{1.523788in}}{\pgfqpoint{0.426907in}{1.515888in}}{\pgfqpoint{0.426907in}{1.507651in}}%
\pgfpathcurveto{\pgfqpoint{0.426907in}{1.499415in}}{\pgfqpoint{0.430179in}{1.491515in}}{\pgfqpoint{0.436003in}{1.485691in}}%
\pgfpathcurveto{\pgfqpoint{0.441827in}{1.479867in}}{\pgfqpoint{0.449727in}{1.476595in}}{\pgfqpoint{0.457963in}{1.476595in}}%
\pgfpathclose%
\pgfusepath{stroke,fill}%
\end{pgfscope}%
\begin{pgfscope}%
\pgfpathrectangle{\pgfqpoint{0.457963in}{0.528059in}}{\pgfqpoint{6.200000in}{2.285714in}} %
\pgfusepath{clip}%
\pgfsetbuttcap%
\pgfsetroundjoin%
\definecolor{currentfill}{rgb}{1.000000,0.500000,0.500000}%
\pgfsetfillcolor{currentfill}%
\pgfsetlinewidth{1.003750pt}%
\definecolor{currentstroke}{rgb}{1.000000,0.500000,0.500000}%
\pgfsetstrokecolor{currentstroke}%
\pgfsetdash{}{0pt}%
\pgfpathmoveto{\pgfqpoint{0.468297in}{1.437411in}}%
\pgfpathcurveto{\pgfqpoint{0.476533in}{1.437411in}}{\pgfqpoint{0.484433in}{1.440683in}}{\pgfqpoint{0.490257in}{1.446507in}}%
\pgfpathcurveto{\pgfqpoint{0.496081in}{1.452331in}}{\pgfqpoint{0.499353in}{1.460231in}}{\pgfqpoint{0.499353in}{1.468468in}}%
\pgfpathcurveto{\pgfqpoint{0.499353in}{1.476704in}}{\pgfqpoint{0.496081in}{1.484604in}}{\pgfqpoint{0.490257in}{1.490428in}}%
\pgfpathcurveto{\pgfqpoint{0.484433in}{1.496252in}}{\pgfqpoint{0.476533in}{1.499524in}}{\pgfqpoint{0.468297in}{1.499524in}}%
\pgfpathcurveto{\pgfqpoint{0.460060in}{1.499524in}}{\pgfqpoint{0.452160in}{1.496252in}}{\pgfqpoint{0.446336in}{1.490428in}}%
\pgfpathcurveto{\pgfqpoint{0.440512in}{1.484604in}}{\pgfqpoint{0.437240in}{1.476704in}}{\pgfqpoint{0.437240in}{1.468468in}}%
\pgfpathcurveto{\pgfqpoint{0.437240in}{1.460231in}}{\pgfqpoint{0.440512in}{1.452331in}}{\pgfqpoint{0.446336in}{1.446507in}}%
\pgfpathcurveto{\pgfqpoint{0.452160in}{1.440683in}}{\pgfqpoint{0.460060in}{1.437411in}}{\pgfqpoint{0.468297in}{1.437411in}}%
\pgfpathclose%
\pgfusepath{stroke,fill}%
\end{pgfscope}%
\begin{pgfscope}%
\pgfpathrectangle{\pgfqpoint{0.457963in}{0.528059in}}{\pgfqpoint{6.200000in}{2.285714in}} %
\pgfusepath{clip}%
\pgfsetbuttcap%
\pgfsetroundjoin%
\definecolor{currentfill}{rgb}{1.000000,0.500000,0.500000}%
\pgfsetfillcolor{currentfill}%
\pgfsetlinewidth{1.003750pt}%
\definecolor{currentstroke}{rgb}{1.000000,0.500000,0.500000}%
\pgfsetstrokecolor{currentstroke}%
\pgfsetdash{}{0pt}%
\pgfpathmoveto{\pgfqpoint{0.468297in}{1.437411in}}%
\pgfpathcurveto{\pgfqpoint{0.476533in}{1.437411in}}{\pgfqpoint{0.484433in}{1.440683in}}{\pgfqpoint{0.490257in}{1.446507in}}%
\pgfpathcurveto{\pgfqpoint{0.496081in}{1.452331in}}{\pgfqpoint{0.499353in}{1.460231in}}{\pgfqpoint{0.499353in}{1.468468in}}%
\pgfpathcurveto{\pgfqpoint{0.499353in}{1.476704in}}{\pgfqpoint{0.496081in}{1.484604in}}{\pgfqpoint{0.490257in}{1.490428in}}%
\pgfpathcurveto{\pgfqpoint{0.484433in}{1.496252in}}{\pgfqpoint{0.476533in}{1.499524in}}{\pgfqpoint{0.468297in}{1.499524in}}%
\pgfpathcurveto{\pgfqpoint{0.460060in}{1.499524in}}{\pgfqpoint{0.452160in}{1.496252in}}{\pgfqpoint{0.446336in}{1.490428in}}%
\pgfpathcurveto{\pgfqpoint{0.440512in}{1.484604in}}{\pgfqpoint{0.437240in}{1.476704in}}{\pgfqpoint{0.437240in}{1.468468in}}%
\pgfpathcurveto{\pgfqpoint{0.437240in}{1.460231in}}{\pgfqpoint{0.440512in}{1.452331in}}{\pgfqpoint{0.446336in}{1.446507in}}%
\pgfpathcurveto{\pgfqpoint{0.452160in}{1.440683in}}{\pgfqpoint{0.460060in}{1.437411in}}{\pgfqpoint{0.468297in}{1.437411in}}%
\pgfpathclose%
\pgfusepath{stroke,fill}%
\end{pgfscope}%
\begin{pgfscope}%
\pgfpathrectangle{\pgfqpoint{0.457963in}{0.528059in}}{\pgfqpoint{6.200000in}{2.285714in}} %
\pgfusepath{clip}%
\pgfsetbuttcap%
\pgfsetroundjoin%
\definecolor{currentfill}{rgb}{1.000000,0.500000,0.500000}%
\pgfsetfillcolor{currentfill}%
\pgfsetlinewidth{1.003750pt}%
\definecolor{currentstroke}{rgb}{1.000000,0.500000,0.500000}%
\pgfsetstrokecolor{currentstroke}%
\pgfsetdash{}{0pt}%
\pgfpathmoveto{\pgfqpoint{0.478630in}{1.476595in}}%
\pgfpathcurveto{\pgfqpoint{0.486866in}{1.476595in}}{\pgfqpoint{0.494766in}{1.479867in}}{\pgfqpoint{0.500590in}{1.485691in}}%
\pgfpathcurveto{\pgfqpoint{0.506414in}{1.491515in}}{\pgfqpoint{0.509686in}{1.499415in}}{\pgfqpoint{0.509686in}{1.507651in}}%
\pgfpathcurveto{\pgfqpoint{0.509686in}{1.515888in}}{\pgfqpoint{0.506414in}{1.523788in}}{\pgfqpoint{0.500590in}{1.529612in}}%
\pgfpathcurveto{\pgfqpoint{0.494766in}{1.535435in}}{\pgfqpoint{0.486866in}{1.538708in}}{\pgfqpoint{0.478630in}{1.538708in}}%
\pgfpathcurveto{\pgfqpoint{0.470394in}{1.538708in}}{\pgfqpoint{0.462494in}{1.535435in}}{\pgfqpoint{0.456670in}{1.529612in}}%
\pgfpathcurveto{\pgfqpoint{0.450846in}{1.523788in}}{\pgfqpoint{0.447574in}{1.515888in}}{\pgfqpoint{0.447574in}{1.507651in}}%
\pgfpathcurveto{\pgfqpoint{0.447574in}{1.499415in}}{\pgfqpoint{0.450846in}{1.491515in}}{\pgfqpoint{0.456670in}{1.485691in}}%
\pgfpathcurveto{\pgfqpoint{0.462494in}{1.479867in}}{\pgfqpoint{0.470394in}{1.476595in}}{\pgfqpoint{0.478630in}{1.476595in}}%
\pgfpathclose%
\pgfusepath{stroke,fill}%
\end{pgfscope}%
\begin{pgfscope}%
\pgfpathrectangle{\pgfqpoint{0.457963in}{0.528059in}}{\pgfqpoint{6.200000in}{2.285714in}} %
\pgfusepath{clip}%
\pgfsetbuttcap%
\pgfsetroundjoin%
\definecolor{currentfill}{rgb}{1.000000,0.500000,0.500000}%
\pgfsetfillcolor{currentfill}%
\pgfsetlinewidth{1.003750pt}%
\definecolor{currentstroke}{rgb}{1.000000,0.500000,0.500000}%
\pgfsetstrokecolor{currentstroke}%
\pgfsetdash{}{0pt}%
\pgfpathmoveto{\pgfqpoint{0.488963in}{1.437411in}}%
\pgfpathcurveto{\pgfqpoint{0.497200in}{1.437411in}}{\pgfqpoint{0.505100in}{1.440683in}}{\pgfqpoint{0.510924in}{1.446507in}}%
\pgfpathcurveto{\pgfqpoint{0.516748in}{1.452331in}}{\pgfqpoint{0.520020in}{1.460231in}}{\pgfqpoint{0.520020in}{1.468468in}}%
\pgfpathcurveto{\pgfqpoint{0.520020in}{1.476704in}}{\pgfqpoint{0.516748in}{1.484604in}}{\pgfqpoint{0.510924in}{1.490428in}}%
\pgfpathcurveto{\pgfqpoint{0.505100in}{1.496252in}}{\pgfqpoint{0.497200in}{1.499524in}}{\pgfqpoint{0.488963in}{1.499524in}}%
\pgfpathcurveto{\pgfqpoint{0.480727in}{1.499524in}}{\pgfqpoint{0.472827in}{1.496252in}}{\pgfqpoint{0.467003in}{1.490428in}}%
\pgfpathcurveto{\pgfqpoint{0.461179in}{1.484604in}}{\pgfqpoint{0.457907in}{1.476704in}}{\pgfqpoint{0.457907in}{1.468468in}}%
\pgfpathcurveto{\pgfqpoint{0.457907in}{1.460231in}}{\pgfqpoint{0.461179in}{1.452331in}}{\pgfqpoint{0.467003in}{1.446507in}}%
\pgfpathcurveto{\pgfqpoint{0.472827in}{1.440683in}}{\pgfqpoint{0.480727in}{1.437411in}}{\pgfqpoint{0.488963in}{1.437411in}}%
\pgfpathclose%
\pgfusepath{stroke,fill}%
\end{pgfscope}%
\begin{pgfscope}%
\pgfpathrectangle{\pgfqpoint{0.457963in}{0.528059in}}{\pgfqpoint{6.200000in}{2.285714in}} %
\pgfusepath{clip}%
\pgfsetbuttcap%
\pgfsetroundjoin%
\definecolor{currentfill}{rgb}{1.000000,0.500000,0.500000}%
\pgfsetfillcolor{currentfill}%
\pgfsetlinewidth{1.003750pt}%
\definecolor{currentstroke}{rgb}{1.000000,0.500000,0.500000}%
\pgfsetstrokecolor{currentstroke}%
\pgfsetdash{}{0pt}%
\pgfpathmoveto{\pgfqpoint{0.499297in}{1.476595in}}%
\pgfpathcurveto{\pgfqpoint{0.507533in}{1.476595in}}{\pgfqpoint{0.515433in}{1.479867in}}{\pgfqpoint{0.521257in}{1.485691in}}%
\pgfpathcurveto{\pgfqpoint{0.527081in}{1.491515in}}{\pgfqpoint{0.530353in}{1.499415in}}{\pgfqpoint{0.530353in}{1.507651in}}%
\pgfpathcurveto{\pgfqpoint{0.530353in}{1.515888in}}{\pgfqpoint{0.527081in}{1.523788in}}{\pgfqpoint{0.521257in}{1.529612in}}%
\pgfpathcurveto{\pgfqpoint{0.515433in}{1.535435in}}{\pgfqpoint{0.507533in}{1.538708in}}{\pgfqpoint{0.499297in}{1.538708in}}%
\pgfpathcurveto{\pgfqpoint{0.491060in}{1.538708in}}{\pgfqpoint{0.483160in}{1.535435in}}{\pgfqpoint{0.477336in}{1.529612in}}%
\pgfpathcurveto{\pgfqpoint{0.471512in}{1.523788in}}{\pgfqpoint{0.468240in}{1.515888in}}{\pgfqpoint{0.468240in}{1.507651in}}%
\pgfpathcurveto{\pgfqpoint{0.468240in}{1.499415in}}{\pgfqpoint{0.471512in}{1.491515in}}{\pgfqpoint{0.477336in}{1.485691in}}%
\pgfpathcurveto{\pgfqpoint{0.483160in}{1.479867in}}{\pgfqpoint{0.491060in}{1.476595in}}{\pgfqpoint{0.499297in}{1.476595in}}%
\pgfpathclose%
\pgfusepath{stroke,fill}%
\end{pgfscope}%
\begin{pgfscope}%
\pgfpathrectangle{\pgfqpoint{0.457963in}{0.528059in}}{\pgfqpoint{6.200000in}{2.285714in}} %
\pgfusepath{clip}%
\pgfsetbuttcap%
\pgfsetroundjoin%
\definecolor{currentfill}{rgb}{1.000000,0.500000,0.500000}%
\pgfsetfillcolor{currentfill}%
\pgfsetlinewidth{1.003750pt}%
\definecolor{currentstroke}{rgb}{1.000000,0.500000,0.500000}%
\pgfsetstrokecolor{currentstroke}%
\pgfsetdash{}{0pt}%
\pgfpathmoveto{\pgfqpoint{0.550963in}{1.476595in}}%
\pgfpathcurveto{\pgfqpoint{0.559200in}{1.476595in}}{\pgfqpoint{0.567100in}{1.479867in}}{\pgfqpoint{0.572924in}{1.485691in}}%
\pgfpathcurveto{\pgfqpoint{0.578748in}{1.491515in}}{\pgfqpoint{0.582020in}{1.499415in}}{\pgfqpoint{0.582020in}{1.507651in}}%
\pgfpathcurveto{\pgfqpoint{0.582020in}{1.515888in}}{\pgfqpoint{0.578748in}{1.523788in}}{\pgfqpoint{0.572924in}{1.529612in}}%
\pgfpathcurveto{\pgfqpoint{0.567100in}{1.535435in}}{\pgfqpoint{0.559200in}{1.538708in}}{\pgfqpoint{0.550963in}{1.538708in}}%
\pgfpathcurveto{\pgfqpoint{0.542727in}{1.538708in}}{\pgfqpoint{0.534827in}{1.535435in}}{\pgfqpoint{0.529003in}{1.529612in}}%
\pgfpathcurveto{\pgfqpoint{0.523179in}{1.523788in}}{\pgfqpoint{0.519907in}{1.515888in}}{\pgfqpoint{0.519907in}{1.507651in}}%
\pgfpathcurveto{\pgfqpoint{0.519907in}{1.499415in}}{\pgfqpoint{0.523179in}{1.491515in}}{\pgfqpoint{0.529003in}{1.485691in}}%
\pgfpathcurveto{\pgfqpoint{0.534827in}{1.479867in}}{\pgfqpoint{0.542727in}{1.476595in}}{\pgfqpoint{0.550963in}{1.476595in}}%
\pgfpathclose%
\pgfusepath{stroke,fill}%
\end{pgfscope}%
\begin{pgfscope}%
\pgfpathrectangle{\pgfqpoint{0.457963in}{0.528059in}}{\pgfqpoint{6.200000in}{2.285714in}} %
\pgfusepath{clip}%
\pgfsetbuttcap%
\pgfsetroundjoin%
\definecolor{currentfill}{rgb}{1.000000,0.500000,0.500000}%
\pgfsetfillcolor{currentfill}%
\pgfsetlinewidth{1.003750pt}%
\definecolor{currentstroke}{rgb}{1.000000,0.500000,0.500000}%
\pgfsetstrokecolor{currentstroke}%
\pgfsetdash{}{0pt}%
\pgfpathmoveto{\pgfqpoint{0.705963in}{1.241493in}}%
\pgfpathcurveto{\pgfqpoint{0.714200in}{1.241493in}}{\pgfqpoint{0.722100in}{1.244765in}}{\pgfqpoint{0.727924in}{1.250589in}}%
\pgfpathcurveto{\pgfqpoint{0.733748in}{1.256413in}}{\pgfqpoint{0.737020in}{1.264313in}}{\pgfqpoint{0.737020in}{1.272549in}}%
\pgfpathcurveto{\pgfqpoint{0.737020in}{1.280786in}}{\pgfqpoint{0.733748in}{1.288686in}}{\pgfqpoint{0.727924in}{1.294510in}}%
\pgfpathcurveto{\pgfqpoint{0.722100in}{1.300333in}}{\pgfqpoint{0.714200in}{1.303606in}}{\pgfqpoint{0.705963in}{1.303606in}}%
\pgfpathcurveto{\pgfqpoint{0.697727in}{1.303606in}}{\pgfqpoint{0.689827in}{1.300333in}}{\pgfqpoint{0.684003in}{1.294510in}}%
\pgfpathcurveto{\pgfqpoint{0.678179in}{1.288686in}}{\pgfqpoint{0.674907in}{1.280786in}}{\pgfqpoint{0.674907in}{1.272549in}}%
\pgfpathcurveto{\pgfqpoint{0.674907in}{1.264313in}}{\pgfqpoint{0.678179in}{1.256413in}}{\pgfqpoint{0.684003in}{1.250589in}}%
\pgfpathcurveto{\pgfqpoint{0.689827in}{1.244765in}}{\pgfqpoint{0.697727in}{1.241493in}}{\pgfqpoint{0.705963in}{1.241493in}}%
\pgfpathclose%
\pgfusepath{stroke,fill}%
\end{pgfscope}%
\begin{pgfscope}%
\pgfpathrectangle{\pgfqpoint{0.457963in}{0.528059in}}{\pgfqpoint{6.200000in}{2.285714in}} %
\pgfusepath{clip}%
\pgfsetbuttcap%
\pgfsetroundjoin%
\definecolor{currentfill}{rgb}{1.000000,0.500000,0.500000}%
\pgfsetfillcolor{currentfill}%
\pgfsetlinewidth{1.003750pt}%
\definecolor{currentstroke}{rgb}{1.000000,0.500000,0.500000}%
\pgfsetstrokecolor{currentstroke}%
\pgfsetdash{}{0pt}%
\pgfpathmoveto{\pgfqpoint{0.788630in}{1.437411in}}%
\pgfpathcurveto{\pgfqpoint{0.796866in}{1.437411in}}{\pgfqpoint{0.804766in}{1.440683in}}{\pgfqpoint{0.810590in}{1.446507in}}%
\pgfpathcurveto{\pgfqpoint{0.816414in}{1.452331in}}{\pgfqpoint{0.819686in}{1.460231in}}{\pgfqpoint{0.819686in}{1.468468in}}%
\pgfpathcurveto{\pgfqpoint{0.819686in}{1.476704in}}{\pgfqpoint{0.816414in}{1.484604in}}{\pgfqpoint{0.810590in}{1.490428in}}%
\pgfpathcurveto{\pgfqpoint{0.804766in}{1.496252in}}{\pgfqpoint{0.796866in}{1.499524in}}{\pgfqpoint{0.788630in}{1.499524in}}%
\pgfpathcurveto{\pgfqpoint{0.780394in}{1.499524in}}{\pgfqpoint{0.772494in}{1.496252in}}{\pgfqpoint{0.766670in}{1.490428in}}%
\pgfpathcurveto{\pgfqpoint{0.760846in}{1.484604in}}{\pgfqpoint{0.757574in}{1.476704in}}{\pgfqpoint{0.757574in}{1.468468in}}%
\pgfpathcurveto{\pgfqpoint{0.757574in}{1.460231in}}{\pgfqpoint{0.760846in}{1.452331in}}{\pgfqpoint{0.766670in}{1.446507in}}%
\pgfpathcurveto{\pgfqpoint{0.772494in}{1.440683in}}{\pgfqpoint{0.780394in}{1.437411in}}{\pgfqpoint{0.788630in}{1.437411in}}%
\pgfpathclose%
\pgfusepath{stroke,fill}%
\end{pgfscope}%
\begin{pgfscope}%
\pgfpathrectangle{\pgfqpoint{0.457963in}{0.528059in}}{\pgfqpoint{6.200000in}{2.285714in}} %
\pgfusepath{clip}%
\pgfsetbuttcap%
\pgfsetroundjoin%
\definecolor{currentfill}{rgb}{1.000000,0.500000,0.500000}%
\pgfsetfillcolor{currentfill}%
\pgfsetlinewidth{1.003750pt}%
\definecolor{currentstroke}{rgb}{1.000000,0.500000,0.500000}%
\pgfsetstrokecolor{currentstroke}%
\pgfsetdash{}{0pt}%
\pgfpathmoveto{\pgfqpoint{0.829963in}{1.241493in}}%
\pgfpathcurveto{\pgfqpoint{0.838200in}{1.241493in}}{\pgfqpoint{0.846100in}{1.244765in}}{\pgfqpoint{0.851924in}{1.250589in}}%
\pgfpathcurveto{\pgfqpoint{0.857748in}{1.256413in}}{\pgfqpoint{0.861020in}{1.264313in}}{\pgfqpoint{0.861020in}{1.272549in}}%
\pgfpathcurveto{\pgfqpoint{0.861020in}{1.280786in}}{\pgfqpoint{0.857748in}{1.288686in}}{\pgfqpoint{0.851924in}{1.294510in}}%
\pgfpathcurveto{\pgfqpoint{0.846100in}{1.300333in}}{\pgfqpoint{0.838200in}{1.303606in}}{\pgfqpoint{0.829963in}{1.303606in}}%
\pgfpathcurveto{\pgfqpoint{0.821727in}{1.303606in}}{\pgfqpoint{0.813827in}{1.300333in}}{\pgfqpoint{0.808003in}{1.294510in}}%
\pgfpathcurveto{\pgfqpoint{0.802179in}{1.288686in}}{\pgfqpoint{0.798907in}{1.280786in}}{\pgfqpoint{0.798907in}{1.272549in}}%
\pgfpathcurveto{\pgfqpoint{0.798907in}{1.264313in}}{\pgfqpoint{0.802179in}{1.256413in}}{\pgfqpoint{0.808003in}{1.250589in}}%
\pgfpathcurveto{\pgfqpoint{0.813827in}{1.244765in}}{\pgfqpoint{0.821727in}{1.241493in}}{\pgfqpoint{0.829963in}{1.241493in}}%
\pgfpathclose%
\pgfusepath{stroke,fill}%
\end{pgfscope}%
\begin{pgfscope}%
\pgfpathrectangle{\pgfqpoint{0.457963in}{0.528059in}}{\pgfqpoint{6.200000in}{2.285714in}} %
\pgfusepath{clip}%
\pgfsetbuttcap%
\pgfsetroundjoin%
\definecolor{currentfill}{rgb}{1.000000,0.500000,0.500000}%
\pgfsetfillcolor{currentfill}%
\pgfsetlinewidth{1.003750pt}%
\definecolor{currentstroke}{rgb}{1.000000,0.500000,0.500000}%
\pgfsetstrokecolor{currentstroke}%
\pgfsetdash{}{0pt}%
\pgfpathmoveto{\pgfqpoint{1.036630in}{1.476595in}}%
\pgfpathcurveto{\pgfqpoint{1.044866in}{1.476595in}}{\pgfqpoint{1.052766in}{1.479867in}}{\pgfqpoint{1.058590in}{1.485691in}}%
\pgfpathcurveto{\pgfqpoint{1.064414in}{1.491515in}}{\pgfqpoint{1.067686in}{1.499415in}}{\pgfqpoint{1.067686in}{1.507651in}}%
\pgfpathcurveto{\pgfqpoint{1.067686in}{1.515888in}}{\pgfqpoint{1.064414in}{1.523788in}}{\pgfqpoint{1.058590in}{1.529612in}}%
\pgfpathcurveto{\pgfqpoint{1.052766in}{1.535435in}}{\pgfqpoint{1.044866in}{1.538708in}}{\pgfqpoint{1.036630in}{1.538708in}}%
\pgfpathcurveto{\pgfqpoint{1.028394in}{1.538708in}}{\pgfqpoint{1.020494in}{1.535435in}}{\pgfqpoint{1.014670in}{1.529612in}}%
\pgfpathcurveto{\pgfqpoint{1.008846in}{1.523788in}}{\pgfqpoint{1.005574in}{1.515888in}}{\pgfqpoint{1.005574in}{1.507651in}}%
\pgfpathcurveto{\pgfqpoint{1.005574in}{1.499415in}}{\pgfqpoint{1.008846in}{1.491515in}}{\pgfqpoint{1.014670in}{1.485691in}}%
\pgfpathcurveto{\pgfqpoint{1.020494in}{1.479867in}}{\pgfqpoint{1.028394in}{1.476595in}}{\pgfqpoint{1.036630in}{1.476595in}}%
\pgfpathclose%
\pgfusepath{stroke,fill}%
\end{pgfscope}%
\begin{pgfscope}%
\pgfpathrectangle{\pgfqpoint{0.457963in}{0.528059in}}{\pgfqpoint{6.200000in}{2.285714in}} %
\pgfusepath{clip}%
\pgfsetbuttcap%
\pgfsetroundjoin%
\definecolor{currentfill}{rgb}{1.000000,0.500000,0.500000}%
\pgfsetfillcolor{currentfill}%
\pgfsetlinewidth{1.003750pt}%
\definecolor{currentstroke}{rgb}{1.000000,0.500000,0.500000}%
\pgfsetstrokecolor{currentstroke}%
\pgfsetdash{}{0pt}%
\pgfpathmoveto{\pgfqpoint{1.160630in}{1.084758in}}%
\pgfpathcurveto{\pgfqpoint{1.168866in}{1.084758in}}{\pgfqpoint{1.176766in}{1.088030in}}{\pgfqpoint{1.182590in}{1.093854in}}%
\pgfpathcurveto{\pgfqpoint{1.188414in}{1.099678in}}{\pgfqpoint{1.191686in}{1.107578in}}{\pgfqpoint{1.191686in}{1.115815in}}%
\pgfpathcurveto{\pgfqpoint{1.191686in}{1.124051in}}{\pgfqpoint{1.188414in}{1.131951in}}{\pgfqpoint{1.182590in}{1.137775in}}%
\pgfpathcurveto{\pgfqpoint{1.176766in}{1.143599in}}{\pgfqpoint{1.168866in}{1.146871in}}{\pgfqpoint{1.160630in}{1.146871in}}%
\pgfpathcurveto{\pgfqpoint{1.152394in}{1.146871in}}{\pgfqpoint{1.144494in}{1.143599in}}{\pgfqpoint{1.138670in}{1.137775in}}%
\pgfpathcurveto{\pgfqpoint{1.132846in}{1.131951in}}{\pgfqpoint{1.129574in}{1.124051in}}{\pgfqpoint{1.129574in}{1.115815in}}%
\pgfpathcurveto{\pgfqpoint{1.129574in}{1.107578in}}{\pgfqpoint{1.132846in}{1.099678in}}{\pgfqpoint{1.138670in}{1.093854in}}%
\pgfpathcurveto{\pgfqpoint{1.144494in}{1.088030in}}{\pgfqpoint{1.152394in}{1.084758in}}{\pgfqpoint{1.160630in}{1.084758in}}%
\pgfpathclose%
\pgfusepath{stroke,fill}%
\end{pgfscope}%
\begin{pgfscope}%
\pgfpathrectangle{\pgfqpoint{0.457963in}{0.528059in}}{\pgfqpoint{6.200000in}{2.285714in}} %
\pgfusepath{clip}%
\pgfsetbuttcap%
\pgfsetroundjoin%
\definecolor{currentfill}{rgb}{1.000000,0.500000,0.500000}%
\pgfsetfillcolor{currentfill}%
\pgfsetlinewidth{1.003750pt}%
\definecolor{currentstroke}{rgb}{1.000000,0.500000,0.500000}%
\pgfsetstrokecolor{currentstroke}%
\pgfsetdash{}{0pt}%
\pgfpathmoveto{\pgfqpoint{1.294963in}{0.771289in}}%
\pgfpathcurveto{\pgfqpoint{1.303200in}{0.771289in}}{\pgfqpoint{1.311100in}{0.774561in}}{\pgfqpoint{1.316924in}{0.780385in}}%
\pgfpathcurveto{\pgfqpoint{1.322748in}{0.786209in}}{\pgfqpoint{1.326020in}{0.794109in}}{\pgfqpoint{1.326020in}{0.802345in}}%
\pgfpathcurveto{\pgfqpoint{1.326020in}{0.810581in}}{\pgfqpoint{1.322748in}{0.818481in}}{\pgfqpoint{1.316924in}{0.824305in}}%
\pgfpathcurveto{\pgfqpoint{1.311100in}{0.830129in}}{\pgfqpoint{1.303200in}{0.833402in}}{\pgfqpoint{1.294963in}{0.833402in}}%
\pgfpathcurveto{\pgfqpoint{1.286727in}{0.833402in}}{\pgfqpoint{1.278827in}{0.830129in}}{\pgfqpoint{1.273003in}{0.824305in}}%
\pgfpathcurveto{\pgfqpoint{1.267179in}{0.818481in}}{\pgfqpoint{1.263907in}{0.810581in}}{\pgfqpoint{1.263907in}{0.802345in}}%
\pgfpathcurveto{\pgfqpoint{1.263907in}{0.794109in}}{\pgfqpoint{1.267179in}{0.786209in}}{\pgfqpoint{1.273003in}{0.780385in}}%
\pgfpathcurveto{\pgfqpoint{1.278827in}{0.774561in}}{\pgfqpoint{1.286727in}{0.771289in}}{\pgfqpoint{1.294963in}{0.771289in}}%
\pgfpathclose%
\pgfusepath{stroke,fill}%
\end{pgfscope}%
\begin{pgfscope}%
\pgfpathrectangle{\pgfqpoint{0.457963in}{0.528059in}}{\pgfqpoint{6.200000in}{2.285714in}} %
\pgfusepath{clip}%
\pgfsetbuttcap%
\pgfsetroundjoin%
\definecolor{currentfill}{rgb}{1.000000,0.500000,0.500000}%
\pgfsetfillcolor{currentfill}%
\pgfsetlinewidth{1.003750pt}%
\definecolor{currentstroke}{rgb}{1.000000,0.500000,0.500000}%
\pgfsetstrokecolor{currentstroke}%
\pgfsetdash{}{0pt}%
\pgfpathmoveto{\pgfqpoint{1.532630in}{1.437411in}}%
\pgfpathcurveto{\pgfqpoint{1.540866in}{1.437411in}}{\pgfqpoint{1.548766in}{1.440683in}}{\pgfqpoint{1.554590in}{1.446507in}}%
\pgfpathcurveto{\pgfqpoint{1.560414in}{1.452331in}}{\pgfqpoint{1.563686in}{1.460231in}}{\pgfqpoint{1.563686in}{1.468468in}}%
\pgfpathcurveto{\pgfqpoint{1.563686in}{1.476704in}}{\pgfqpoint{1.560414in}{1.484604in}}{\pgfqpoint{1.554590in}{1.490428in}}%
\pgfpathcurveto{\pgfqpoint{1.548766in}{1.496252in}}{\pgfqpoint{1.540866in}{1.499524in}}{\pgfqpoint{1.532630in}{1.499524in}}%
\pgfpathcurveto{\pgfqpoint{1.524394in}{1.499524in}}{\pgfqpoint{1.516494in}{1.496252in}}{\pgfqpoint{1.510670in}{1.490428in}}%
\pgfpathcurveto{\pgfqpoint{1.504846in}{1.484604in}}{\pgfqpoint{1.501574in}{1.476704in}}{\pgfqpoint{1.501574in}{1.468468in}}%
\pgfpathcurveto{\pgfqpoint{1.501574in}{1.460231in}}{\pgfqpoint{1.504846in}{1.452331in}}{\pgfqpoint{1.510670in}{1.446507in}}%
\pgfpathcurveto{\pgfqpoint{1.516494in}{1.440683in}}{\pgfqpoint{1.524394in}{1.437411in}}{\pgfqpoint{1.532630in}{1.437411in}}%
\pgfpathclose%
\pgfusepath{stroke,fill}%
\end{pgfscope}%
\begin{pgfscope}%
\pgfpathrectangle{\pgfqpoint{0.457963in}{0.528059in}}{\pgfqpoint{6.200000in}{2.285714in}} %
\pgfusepath{clip}%
\pgfsetbuttcap%
\pgfsetroundjoin%
\definecolor{currentfill}{rgb}{1.000000,0.500000,0.500000}%
\pgfsetfillcolor{currentfill}%
\pgfsetlinewidth{1.003750pt}%
\definecolor{currentstroke}{rgb}{1.000000,0.500000,0.500000}%
\pgfsetstrokecolor{currentstroke}%
\pgfsetdash{}{0pt}%
\pgfpathmoveto{\pgfqpoint{1.821963in}{1.385166in}}%
\pgfpathcurveto{\pgfqpoint{1.830200in}{1.385166in}}{\pgfqpoint{1.838100in}{1.388439in}}{\pgfqpoint{1.843924in}{1.394262in}}%
\pgfpathcurveto{\pgfqpoint{1.849748in}{1.400086in}}{\pgfqpoint{1.853020in}{1.407986in}}{\pgfqpoint{1.853020in}{1.416223in}}%
\pgfpathcurveto{\pgfqpoint{1.853020in}{1.424459in}}{\pgfqpoint{1.849748in}{1.432359in}}{\pgfqpoint{1.843924in}{1.438183in}}%
\pgfpathcurveto{\pgfqpoint{1.838100in}{1.444007in}}{\pgfqpoint{1.830200in}{1.447279in}}{\pgfqpoint{1.821963in}{1.447279in}}%
\pgfpathcurveto{\pgfqpoint{1.813727in}{1.447279in}}{\pgfqpoint{1.805827in}{1.444007in}}{\pgfqpoint{1.800003in}{1.438183in}}%
\pgfpathcurveto{\pgfqpoint{1.794179in}{1.432359in}}{\pgfqpoint{1.790907in}{1.424459in}}{\pgfqpoint{1.790907in}{1.416223in}}%
\pgfpathcurveto{\pgfqpoint{1.790907in}{1.407986in}}{\pgfqpoint{1.794179in}{1.400086in}}{\pgfqpoint{1.800003in}{1.394262in}}%
\pgfpathcurveto{\pgfqpoint{1.805827in}{1.388439in}}{\pgfqpoint{1.813727in}{1.385166in}}{\pgfqpoint{1.821963in}{1.385166in}}%
\pgfpathclose%
\pgfusepath{stroke,fill}%
\end{pgfscope}%
\begin{pgfscope}%
\pgfpathrectangle{\pgfqpoint{0.457963in}{0.528059in}}{\pgfqpoint{6.200000in}{2.285714in}} %
\pgfusepath{clip}%
\pgfsetbuttcap%
\pgfsetroundjoin%
\definecolor{currentfill}{rgb}{1.000000,0.500000,0.500000}%
\pgfsetfillcolor{currentfill}%
\pgfsetlinewidth{1.003750pt}%
\definecolor{currentstroke}{rgb}{1.000000,0.500000,0.500000}%
\pgfsetstrokecolor{currentstroke}%
\pgfsetdash{}{0pt}%
\pgfpathmoveto{\pgfqpoint{1.925297in}{1.398227in}}%
\pgfpathcurveto{\pgfqpoint{1.933533in}{1.398227in}}{\pgfqpoint{1.941433in}{1.401500in}}{\pgfqpoint{1.947257in}{1.407324in}}%
\pgfpathcurveto{\pgfqpoint{1.953081in}{1.413148in}}{\pgfqpoint{1.956353in}{1.421048in}}{\pgfqpoint{1.956353in}{1.429284in}}%
\pgfpathcurveto{\pgfqpoint{1.956353in}{1.437520in}}{\pgfqpoint{1.953081in}{1.445420in}}{\pgfqpoint{1.947257in}{1.451244in}}%
\pgfpathcurveto{\pgfqpoint{1.941433in}{1.457068in}}{\pgfqpoint{1.933533in}{1.460340in}}{\pgfqpoint{1.925297in}{1.460340in}}%
\pgfpathcurveto{\pgfqpoint{1.917060in}{1.460340in}}{\pgfqpoint{1.909160in}{1.457068in}}{\pgfqpoint{1.903336in}{1.451244in}}%
\pgfpathcurveto{\pgfqpoint{1.897512in}{1.445420in}}{\pgfqpoint{1.894240in}{1.437520in}}{\pgfqpoint{1.894240in}{1.429284in}}%
\pgfpathcurveto{\pgfqpoint{1.894240in}{1.421048in}}{\pgfqpoint{1.897512in}{1.413148in}}{\pgfqpoint{1.903336in}{1.407324in}}%
\pgfpathcurveto{\pgfqpoint{1.909160in}{1.401500in}}{\pgfqpoint{1.917060in}{1.398227in}}{\pgfqpoint{1.925297in}{1.398227in}}%
\pgfpathclose%
\pgfusepath{stroke,fill}%
\end{pgfscope}%
\begin{pgfscope}%
\pgfpathrectangle{\pgfqpoint{0.457963in}{0.528059in}}{\pgfqpoint{6.200000in}{2.285714in}} %
\pgfusepath{clip}%
\pgfsetbuttcap%
\pgfsetroundjoin%
\definecolor{currentfill}{rgb}{1.000000,0.500000,0.500000}%
\pgfsetfillcolor{currentfill}%
\pgfsetlinewidth{1.003750pt}%
\definecolor{currentstroke}{rgb}{1.000000,0.500000,0.500000}%
\pgfsetstrokecolor{currentstroke}%
\pgfsetdash{}{0pt}%
\pgfpathmoveto{\pgfqpoint{2.565963in}{1.110880in}}%
\pgfpathcurveto{\pgfqpoint{2.574200in}{1.110880in}}{\pgfqpoint{2.582100in}{1.114153in}}{\pgfqpoint{2.587924in}{1.119977in}}%
\pgfpathcurveto{\pgfqpoint{2.593748in}{1.125801in}}{\pgfqpoint{2.597020in}{1.133701in}}{\pgfqpoint{2.597020in}{1.141937in}}%
\pgfpathcurveto{\pgfqpoint{2.597020in}{1.150173in}}{\pgfqpoint{2.593748in}{1.158073in}}{\pgfqpoint{2.587924in}{1.163897in}}%
\pgfpathcurveto{\pgfqpoint{2.582100in}{1.169721in}}{\pgfqpoint{2.574200in}{1.172993in}}{\pgfqpoint{2.565963in}{1.172993in}}%
\pgfpathcurveto{\pgfqpoint{2.557727in}{1.172993in}}{\pgfqpoint{2.549827in}{1.169721in}}{\pgfqpoint{2.544003in}{1.163897in}}%
\pgfpathcurveto{\pgfqpoint{2.538179in}{1.158073in}}{\pgfqpoint{2.534907in}{1.150173in}}{\pgfqpoint{2.534907in}{1.141937in}}%
\pgfpathcurveto{\pgfqpoint{2.534907in}{1.133701in}}{\pgfqpoint{2.538179in}{1.125801in}}{\pgfqpoint{2.544003in}{1.119977in}}%
\pgfpathcurveto{\pgfqpoint{2.549827in}{1.114153in}}{\pgfqpoint{2.557727in}{1.110880in}}{\pgfqpoint{2.565963in}{1.110880in}}%
\pgfpathclose%
\pgfusepath{stroke,fill}%
\end{pgfscope}%
\begin{pgfscope}%
\pgfpathrectangle{\pgfqpoint{0.457963in}{0.528059in}}{\pgfqpoint{6.200000in}{2.285714in}} %
\pgfusepath{clip}%
\pgfsetbuttcap%
\pgfsetroundjoin%
\definecolor{currentfill}{rgb}{1.000000,0.500000,0.500000}%
\pgfsetfillcolor{currentfill}%
\pgfsetlinewidth{1.003750pt}%
\definecolor{currentstroke}{rgb}{1.000000,0.500000,0.500000}%
\pgfsetstrokecolor{currentstroke}%
\pgfsetdash{}{0pt}%
\pgfpathmoveto{\pgfqpoint{2.968963in}{1.019452in}}%
\pgfpathcurveto{\pgfqpoint{2.977200in}{1.019452in}}{\pgfqpoint{2.985100in}{1.022724in}}{\pgfqpoint{2.990924in}{1.028548in}}%
\pgfpathcurveto{\pgfqpoint{2.996748in}{1.034372in}}{\pgfqpoint{3.000020in}{1.042272in}}{\pgfqpoint{3.000020in}{1.050508in}}%
\pgfpathcurveto{\pgfqpoint{3.000020in}{1.058745in}}{\pgfqpoint{2.996748in}{1.066645in}}{\pgfqpoint{2.990924in}{1.072469in}}%
\pgfpathcurveto{\pgfqpoint{2.985100in}{1.078293in}}{\pgfqpoint{2.977200in}{1.081565in}}{\pgfqpoint{2.968963in}{1.081565in}}%
\pgfpathcurveto{\pgfqpoint{2.960727in}{1.081565in}}{\pgfqpoint{2.952827in}{1.078293in}}{\pgfqpoint{2.947003in}{1.072469in}}%
\pgfpathcurveto{\pgfqpoint{2.941179in}{1.066645in}}{\pgfqpoint{2.937907in}{1.058745in}}{\pgfqpoint{2.937907in}{1.050508in}}%
\pgfpathcurveto{\pgfqpoint{2.937907in}{1.042272in}}{\pgfqpoint{2.941179in}{1.034372in}}{\pgfqpoint{2.947003in}{1.028548in}}%
\pgfpathcurveto{\pgfqpoint{2.952827in}{1.022724in}}{\pgfqpoint{2.960727in}{1.019452in}}{\pgfqpoint{2.968963in}{1.019452in}}%
\pgfpathclose%
\pgfusepath{stroke,fill}%
\end{pgfscope}%
\begin{pgfscope}%
\pgfpathrectangle{\pgfqpoint{0.457963in}{0.528059in}}{\pgfqpoint{6.200000in}{2.285714in}} %
\pgfusepath{clip}%
\pgfsetbuttcap%
\pgfsetroundjoin%
\definecolor{currentfill}{rgb}{1.000000,0.333333,0.333333}%
\pgfsetfillcolor{currentfill}%
\pgfsetlinewidth{1.003750pt}%
\definecolor{currentstroke}{rgb}{1.000000,0.333333,0.333333}%
\pgfsetstrokecolor{currentstroke}%
\pgfsetdash{}{0pt}%
\pgfpathmoveto{\pgfqpoint{0.457963in}{1.803125in}}%
\pgfpathcurveto{\pgfqpoint{0.466200in}{1.803125in}}{\pgfqpoint{0.474100in}{1.806398in}}{\pgfqpoint{0.479924in}{1.812222in}}%
\pgfpathcurveto{\pgfqpoint{0.485748in}{1.818046in}}{\pgfqpoint{0.489020in}{1.825946in}}{\pgfqpoint{0.489020in}{1.834182in}}%
\pgfpathcurveto{\pgfqpoint{0.489020in}{1.842418in}}{\pgfqpoint{0.485748in}{1.850318in}}{\pgfqpoint{0.479924in}{1.856142in}}%
\pgfpathcurveto{\pgfqpoint{0.474100in}{1.861966in}}{\pgfqpoint{0.466200in}{1.865238in}}{\pgfqpoint{0.457963in}{1.865238in}}%
\pgfpathcurveto{\pgfqpoint{0.449727in}{1.865238in}}{\pgfqpoint{0.441827in}{1.861966in}}{\pgfqpoint{0.436003in}{1.856142in}}%
\pgfpathcurveto{\pgfqpoint{0.430179in}{1.850318in}}{\pgfqpoint{0.426907in}{1.842418in}}{\pgfqpoint{0.426907in}{1.834182in}}%
\pgfpathcurveto{\pgfqpoint{0.426907in}{1.825946in}}{\pgfqpoint{0.430179in}{1.818046in}}{\pgfqpoint{0.436003in}{1.812222in}}%
\pgfpathcurveto{\pgfqpoint{0.441827in}{1.806398in}}{\pgfqpoint{0.449727in}{1.803125in}}{\pgfqpoint{0.457963in}{1.803125in}}%
\pgfpathclose%
\pgfusepath{stroke,fill}%
\end{pgfscope}%
\begin{pgfscope}%
\pgfpathrectangle{\pgfqpoint{0.457963in}{0.528059in}}{\pgfqpoint{6.200000in}{2.285714in}} %
\pgfusepath{clip}%
\pgfsetbuttcap%
\pgfsetroundjoin%
\definecolor{currentfill}{rgb}{1.000000,0.333333,0.333333}%
\pgfsetfillcolor{currentfill}%
\pgfsetlinewidth{1.003750pt}%
\definecolor{currentstroke}{rgb}{1.000000,0.333333,0.333333}%
\pgfsetstrokecolor{currentstroke}%
\pgfsetdash{}{0pt}%
\pgfpathmoveto{\pgfqpoint{0.457963in}{1.803125in}}%
\pgfpathcurveto{\pgfqpoint{0.466200in}{1.803125in}}{\pgfqpoint{0.474100in}{1.806398in}}{\pgfqpoint{0.479924in}{1.812222in}}%
\pgfpathcurveto{\pgfqpoint{0.485748in}{1.818046in}}{\pgfqpoint{0.489020in}{1.825946in}}{\pgfqpoint{0.489020in}{1.834182in}}%
\pgfpathcurveto{\pgfqpoint{0.489020in}{1.842418in}}{\pgfqpoint{0.485748in}{1.850318in}}{\pgfqpoint{0.479924in}{1.856142in}}%
\pgfpathcurveto{\pgfqpoint{0.474100in}{1.861966in}}{\pgfqpoint{0.466200in}{1.865238in}}{\pgfqpoint{0.457963in}{1.865238in}}%
\pgfpathcurveto{\pgfqpoint{0.449727in}{1.865238in}}{\pgfqpoint{0.441827in}{1.861966in}}{\pgfqpoint{0.436003in}{1.856142in}}%
\pgfpathcurveto{\pgfqpoint{0.430179in}{1.850318in}}{\pgfqpoint{0.426907in}{1.842418in}}{\pgfqpoint{0.426907in}{1.834182in}}%
\pgfpathcurveto{\pgfqpoint{0.426907in}{1.825946in}}{\pgfqpoint{0.430179in}{1.818046in}}{\pgfqpoint{0.436003in}{1.812222in}}%
\pgfpathcurveto{\pgfqpoint{0.441827in}{1.806398in}}{\pgfqpoint{0.449727in}{1.803125in}}{\pgfqpoint{0.457963in}{1.803125in}}%
\pgfpathclose%
\pgfusepath{stroke,fill}%
\end{pgfscope}%
\begin{pgfscope}%
\pgfpathrectangle{\pgfqpoint{0.457963in}{0.528059in}}{\pgfqpoint{6.200000in}{2.285714in}} %
\pgfusepath{clip}%
\pgfsetbuttcap%
\pgfsetroundjoin%
\definecolor{currentfill}{rgb}{1.000000,0.333333,0.333333}%
\pgfsetfillcolor{currentfill}%
\pgfsetlinewidth{1.003750pt}%
\definecolor{currentstroke}{rgb}{1.000000,0.333333,0.333333}%
\pgfsetstrokecolor{currentstroke}%
\pgfsetdash{}{0pt}%
\pgfpathmoveto{\pgfqpoint{0.457963in}{1.803125in}}%
\pgfpathcurveto{\pgfqpoint{0.466200in}{1.803125in}}{\pgfqpoint{0.474100in}{1.806398in}}{\pgfqpoint{0.479924in}{1.812222in}}%
\pgfpathcurveto{\pgfqpoint{0.485748in}{1.818046in}}{\pgfqpoint{0.489020in}{1.825946in}}{\pgfqpoint{0.489020in}{1.834182in}}%
\pgfpathcurveto{\pgfqpoint{0.489020in}{1.842418in}}{\pgfqpoint{0.485748in}{1.850318in}}{\pgfqpoint{0.479924in}{1.856142in}}%
\pgfpathcurveto{\pgfqpoint{0.474100in}{1.861966in}}{\pgfqpoint{0.466200in}{1.865238in}}{\pgfqpoint{0.457963in}{1.865238in}}%
\pgfpathcurveto{\pgfqpoint{0.449727in}{1.865238in}}{\pgfqpoint{0.441827in}{1.861966in}}{\pgfqpoint{0.436003in}{1.856142in}}%
\pgfpathcurveto{\pgfqpoint{0.430179in}{1.850318in}}{\pgfqpoint{0.426907in}{1.842418in}}{\pgfqpoint{0.426907in}{1.834182in}}%
\pgfpathcurveto{\pgfqpoint{0.426907in}{1.825946in}}{\pgfqpoint{0.430179in}{1.818046in}}{\pgfqpoint{0.436003in}{1.812222in}}%
\pgfpathcurveto{\pgfqpoint{0.441827in}{1.806398in}}{\pgfqpoint{0.449727in}{1.803125in}}{\pgfqpoint{0.457963in}{1.803125in}}%
\pgfpathclose%
\pgfusepath{stroke,fill}%
\end{pgfscope}%
\begin{pgfscope}%
\pgfpathrectangle{\pgfqpoint{0.457963in}{0.528059in}}{\pgfqpoint{6.200000in}{2.285714in}} %
\pgfusepath{clip}%
\pgfsetbuttcap%
\pgfsetroundjoin%
\definecolor{currentfill}{rgb}{1.000000,0.333333,0.333333}%
\pgfsetfillcolor{currentfill}%
\pgfsetlinewidth{1.003750pt}%
\definecolor{currentstroke}{rgb}{1.000000,0.333333,0.333333}%
\pgfsetstrokecolor{currentstroke}%
\pgfsetdash{}{0pt}%
\pgfpathmoveto{\pgfqpoint{0.457963in}{1.803125in}}%
\pgfpathcurveto{\pgfqpoint{0.466200in}{1.803125in}}{\pgfqpoint{0.474100in}{1.806398in}}{\pgfqpoint{0.479924in}{1.812222in}}%
\pgfpathcurveto{\pgfqpoint{0.485748in}{1.818046in}}{\pgfqpoint{0.489020in}{1.825946in}}{\pgfqpoint{0.489020in}{1.834182in}}%
\pgfpathcurveto{\pgfqpoint{0.489020in}{1.842418in}}{\pgfqpoint{0.485748in}{1.850318in}}{\pgfqpoint{0.479924in}{1.856142in}}%
\pgfpathcurveto{\pgfqpoint{0.474100in}{1.861966in}}{\pgfqpoint{0.466200in}{1.865238in}}{\pgfqpoint{0.457963in}{1.865238in}}%
\pgfpathcurveto{\pgfqpoint{0.449727in}{1.865238in}}{\pgfqpoint{0.441827in}{1.861966in}}{\pgfqpoint{0.436003in}{1.856142in}}%
\pgfpathcurveto{\pgfqpoint{0.430179in}{1.850318in}}{\pgfqpoint{0.426907in}{1.842418in}}{\pgfqpoint{0.426907in}{1.834182in}}%
\pgfpathcurveto{\pgfqpoint{0.426907in}{1.825946in}}{\pgfqpoint{0.430179in}{1.818046in}}{\pgfqpoint{0.436003in}{1.812222in}}%
\pgfpathcurveto{\pgfqpoint{0.441827in}{1.806398in}}{\pgfqpoint{0.449727in}{1.803125in}}{\pgfqpoint{0.457963in}{1.803125in}}%
\pgfpathclose%
\pgfusepath{stroke,fill}%
\end{pgfscope}%
\begin{pgfscope}%
\pgfpathrectangle{\pgfqpoint{0.457963in}{0.528059in}}{\pgfqpoint{6.200000in}{2.285714in}} %
\pgfusepath{clip}%
\pgfsetbuttcap%
\pgfsetroundjoin%
\definecolor{currentfill}{rgb}{1.000000,0.333333,0.333333}%
\pgfsetfillcolor{currentfill}%
\pgfsetlinewidth{1.003750pt}%
\definecolor{currentstroke}{rgb}{1.000000,0.333333,0.333333}%
\pgfsetstrokecolor{currentstroke}%
\pgfsetdash{}{0pt}%
\pgfpathmoveto{\pgfqpoint{0.457963in}{1.803125in}}%
\pgfpathcurveto{\pgfqpoint{0.466200in}{1.803125in}}{\pgfqpoint{0.474100in}{1.806398in}}{\pgfqpoint{0.479924in}{1.812222in}}%
\pgfpathcurveto{\pgfqpoint{0.485748in}{1.818046in}}{\pgfqpoint{0.489020in}{1.825946in}}{\pgfqpoint{0.489020in}{1.834182in}}%
\pgfpathcurveto{\pgfqpoint{0.489020in}{1.842418in}}{\pgfqpoint{0.485748in}{1.850318in}}{\pgfqpoint{0.479924in}{1.856142in}}%
\pgfpathcurveto{\pgfqpoint{0.474100in}{1.861966in}}{\pgfqpoint{0.466200in}{1.865238in}}{\pgfqpoint{0.457963in}{1.865238in}}%
\pgfpathcurveto{\pgfqpoint{0.449727in}{1.865238in}}{\pgfqpoint{0.441827in}{1.861966in}}{\pgfqpoint{0.436003in}{1.856142in}}%
\pgfpathcurveto{\pgfqpoint{0.430179in}{1.850318in}}{\pgfqpoint{0.426907in}{1.842418in}}{\pgfqpoint{0.426907in}{1.834182in}}%
\pgfpathcurveto{\pgfqpoint{0.426907in}{1.825946in}}{\pgfqpoint{0.430179in}{1.818046in}}{\pgfqpoint{0.436003in}{1.812222in}}%
\pgfpathcurveto{\pgfqpoint{0.441827in}{1.806398in}}{\pgfqpoint{0.449727in}{1.803125in}}{\pgfqpoint{0.457963in}{1.803125in}}%
\pgfpathclose%
\pgfusepath{stroke,fill}%
\end{pgfscope}%
\begin{pgfscope}%
\pgfpathrectangle{\pgfqpoint{0.457963in}{0.528059in}}{\pgfqpoint{6.200000in}{2.285714in}} %
\pgfusepath{clip}%
\pgfsetbuttcap%
\pgfsetroundjoin%
\definecolor{currentfill}{rgb}{1.000000,0.333333,0.333333}%
\pgfsetfillcolor{currentfill}%
\pgfsetlinewidth{1.003750pt}%
\definecolor{currentstroke}{rgb}{1.000000,0.333333,0.333333}%
\pgfsetstrokecolor{currentstroke}%
\pgfsetdash{}{0pt}%
\pgfpathmoveto{\pgfqpoint{0.488963in}{1.803125in}}%
\pgfpathcurveto{\pgfqpoint{0.497200in}{1.803125in}}{\pgfqpoint{0.505100in}{1.806398in}}{\pgfqpoint{0.510924in}{1.812222in}}%
\pgfpathcurveto{\pgfqpoint{0.516748in}{1.818046in}}{\pgfqpoint{0.520020in}{1.825946in}}{\pgfqpoint{0.520020in}{1.834182in}}%
\pgfpathcurveto{\pgfqpoint{0.520020in}{1.842418in}}{\pgfqpoint{0.516748in}{1.850318in}}{\pgfqpoint{0.510924in}{1.856142in}}%
\pgfpathcurveto{\pgfqpoint{0.505100in}{1.861966in}}{\pgfqpoint{0.497200in}{1.865238in}}{\pgfqpoint{0.488963in}{1.865238in}}%
\pgfpathcurveto{\pgfqpoint{0.480727in}{1.865238in}}{\pgfqpoint{0.472827in}{1.861966in}}{\pgfqpoint{0.467003in}{1.856142in}}%
\pgfpathcurveto{\pgfqpoint{0.461179in}{1.850318in}}{\pgfqpoint{0.457907in}{1.842418in}}{\pgfqpoint{0.457907in}{1.834182in}}%
\pgfpathcurveto{\pgfqpoint{0.457907in}{1.825946in}}{\pgfqpoint{0.461179in}{1.818046in}}{\pgfqpoint{0.467003in}{1.812222in}}%
\pgfpathcurveto{\pgfqpoint{0.472827in}{1.806398in}}{\pgfqpoint{0.480727in}{1.803125in}}{\pgfqpoint{0.488963in}{1.803125in}}%
\pgfpathclose%
\pgfusepath{stroke,fill}%
\end{pgfscope}%
\begin{pgfscope}%
\pgfpathrectangle{\pgfqpoint{0.457963in}{0.528059in}}{\pgfqpoint{6.200000in}{2.285714in}} %
\pgfusepath{clip}%
\pgfsetbuttcap%
\pgfsetroundjoin%
\definecolor{currentfill}{rgb}{1.000000,0.333333,0.333333}%
\pgfsetfillcolor{currentfill}%
\pgfsetlinewidth{1.003750pt}%
\definecolor{currentstroke}{rgb}{1.000000,0.333333,0.333333}%
\pgfsetstrokecolor{currentstroke}%
\pgfsetdash{}{0pt}%
\pgfpathmoveto{\pgfqpoint{0.530297in}{1.750880in}}%
\pgfpathcurveto{\pgfqpoint{0.538533in}{1.750880in}}{\pgfqpoint{0.546433in}{1.754153in}}{\pgfqpoint{0.552257in}{1.759977in}}%
\pgfpathcurveto{\pgfqpoint{0.558081in}{1.765801in}}{\pgfqpoint{0.561353in}{1.773701in}}{\pgfqpoint{0.561353in}{1.781937in}}%
\pgfpathcurveto{\pgfqpoint{0.561353in}{1.790173in}}{\pgfqpoint{0.558081in}{1.798073in}}{\pgfqpoint{0.552257in}{1.803897in}}%
\pgfpathcurveto{\pgfqpoint{0.546433in}{1.809721in}}{\pgfqpoint{0.538533in}{1.812993in}}{\pgfqpoint{0.530297in}{1.812993in}}%
\pgfpathcurveto{\pgfqpoint{0.522060in}{1.812993in}}{\pgfqpoint{0.514160in}{1.809721in}}{\pgfqpoint{0.508336in}{1.803897in}}%
\pgfpathcurveto{\pgfqpoint{0.502512in}{1.798073in}}{\pgfqpoint{0.499240in}{1.790173in}}{\pgfqpoint{0.499240in}{1.781937in}}%
\pgfpathcurveto{\pgfqpoint{0.499240in}{1.773701in}}{\pgfqpoint{0.502512in}{1.765801in}}{\pgfqpoint{0.508336in}{1.759977in}}%
\pgfpathcurveto{\pgfqpoint{0.514160in}{1.754153in}}{\pgfqpoint{0.522060in}{1.750880in}}{\pgfqpoint{0.530297in}{1.750880in}}%
\pgfpathclose%
\pgfusepath{stroke,fill}%
\end{pgfscope}%
\begin{pgfscope}%
\pgfpathrectangle{\pgfqpoint{0.457963in}{0.528059in}}{\pgfqpoint{6.200000in}{2.285714in}} %
\pgfusepath{clip}%
\pgfsetbuttcap%
\pgfsetroundjoin%
\definecolor{currentfill}{rgb}{1.000000,0.333333,0.333333}%
\pgfsetfillcolor{currentfill}%
\pgfsetlinewidth{1.003750pt}%
\definecolor{currentstroke}{rgb}{1.000000,0.333333,0.333333}%
\pgfsetstrokecolor{currentstroke}%
\pgfsetdash{}{0pt}%
\pgfpathmoveto{\pgfqpoint{0.540630in}{1.594146in}}%
\pgfpathcurveto{\pgfqpoint{0.548866in}{1.594146in}}{\pgfqpoint{0.556766in}{1.597418in}}{\pgfqpoint{0.562590in}{1.603242in}}%
\pgfpathcurveto{\pgfqpoint{0.568414in}{1.609066in}}{\pgfqpoint{0.571686in}{1.616966in}}{\pgfqpoint{0.571686in}{1.625202in}}%
\pgfpathcurveto{\pgfqpoint{0.571686in}{1.633439in}}{\pgfqpoint{0.568414in}{1.641339in}}{\pgfqpoint{0.562590in}{1.647163in}}%
\pgfpathcurveto{\pgfqpoint{0.556766in}{1.652986in}}{\pgfqpoint{0.548866in}{1.656259in}}{\pgfqpoint{0.540630in}{1.656259in}}%
\pgfpathcurveto{\pgfqpoint{0.532394in}{1.656259in}}{\pgfqpoint{0.524494in}{1.652986in}}{\pgfqpoint{0.518670in}{1.647163in}}%
\pgfpathcurveto{\pgfqpoint{0.512846in}{1.641339in}}{\pgfqpoint{0.509574in}{1.633439in}}{\pgfqpoint{0.509574in}{1.625202in}}%
\pgfpathcurveto{\pgfqpoint{0.509574in}{1.616966in}}{\pgfqpoint{0.512846in}{1.609066in}}{\pgfqpoint{0.518670in}{1.603242in}}%
\pgfpathcurveto{\pgfqpoint{0.524494in}{1.597418in}}{\pgfqpoint{0.532394in}{1.594146in}}{\pgfqpoint{0.540630in}{1.594146in}}%
\pgfpathclose%
\pgfusepath{stroke,fill}%
\end{pgfscope}%
\begin{pgfscope}%
\pgfpathrectangle{\pgfqpoint{0.457963in}{0.528059in}}{\pgfqpoint{6.200000in}{2.285714in}} %
\pgfusepath{clip}%
\pgfsetbuttcap%
\pgfsetroundjoin%
\definecolor{currentfill}{rgb}{1.000000,0.333333,0.333333}%
\pgfsetfillcolor{currentfill}%
\pgfsetlinewidth{1.003750pt}%
\definecolor{currentstroke}{rgb}{1.000000,0.333333,0.333333}%
\pgfsetstrokecolor{currentstroke}%
\pgfsetdash{}{0pt}%
\pgfpathmoveto{\pgfqpoint{0.581963in}{1.750880in}}%
\pgfpathcurveto{\pgfqpoint{0.590200in}{1.750880in}}{\pgfqpoint{0.598100in}{1.754153in}}{\pgfqpoint{0.603924in}{1.759977in}}%
\pgfpathcurveto{\pgfqpoint{0.609748in}{1.765801in}}{\pgfqpoint{0.613020in}{1.773701in}}{\pgfqpoint{0.613020in}{1.781937in}}%
\pgfpathcurveto{\pgfqpoint{0.613020in}{1.790173in}}{\pgfqpoint{0.609748in}{1.798073in}}{\pgfqpoint{0.603924in}{1.803897in}}%
\pgfpathcurveto{\pgfqpoint{0.598100in}{1.809721in}}{\pgfqpoint{0.590200in}{1.812993in}}{\pgfqpoint{0.581963in}{1.812993in}}%
\pgfpathcurveto{\pgfqpoint{0.573727in}{1.812993in}}{\pgfqpoint{0.565827in}{1.809721in}}{\pgfqpoint{0.560003in}{1.803897in}}%
\pgfpathcurveto{\pgfqpoint{0.554179in}{1.798073in}}{\pgfqpoint{0.550907in}{1.790173in}}{\pgfqpoint{0.550907in}{1.781937in}}%
\pgfpathcurveto{\pgfqpoint{0.550907in}{1.773701in}}{\pgfqpoint{0.554179in}{1.765801in}}{\pgfqpoint{0.560003in}{1.759977in}}%
\pgfpathcurveto{\pgfqpoint{0.565827in}{1.754153in}}{\pgfqpoint{0.573727in}{1.750880in}}{\pgfqpoint{0.581963in}{1.750880in}}%
\pgfpathclose%
\pgfusepath{stroke,fill}%
\end{pgfscope}%
\begin{pgfscope}%
\pgfpathrectangle{\pgfqpoint{0.457963in}{0.528059in}}{\pgfqpoint{6.200000in}{2.285714in}} %
\pgfusepath{clip}%
\pgfsetbuttcap%
\pgfsetroundjoin%
\definecolor{currentfill}{rgb}{1.000000,0.333333,0.333333}%
\pgfsetfillcolor{currentfill}%
\pgfsetlinewidth{1.003750pt}%
\definecolor{currentstroke}{rgb}{1.000000,0.333333,0.333333}%
\pgfsetstrokecolor{currentstroke}%
\pgfsetdash{}{0pt}%
\pgfpathmoveto{\pgfqpoint{0.592297in}{1.790064in}}%
\pgfpathcurveto{\pgfqpoint{0.600533in}{1.790064in}}{\pgfqpoint{0.608433in}{1.793336in}}{\pgfqpoint{0.614257in}{1.799160in}}%
\pgfpathcurveto{\pgfqpoint{0.620081in}{1.804984in}}{\pgfqpoint{0.623353in}{1.812884in}}{\pgfqpoint{0.623353in}{1.821121in}}%
\pgfpathcurveto{\pgfqpoint{0.623353in}{1.829357in}}{\pgfqpoint{0.620081in}{1.837257in}}{\pgfqpoint{0.614257in}{1.843081in}}%
\pgfpathcurveto{\pgfqpoint{0.608433in}{1.848905in}}{\pgfqpoint{0.600533in}{1.852177in}}{\pgfqpoint{0.592297in}{1.852177in}}%
\pgfpathcurveto{\pgfqpoint{0.584060in}{1.852177in}}{\pgfqpoint{0.576160in}{1.848905in}}{\pgfqpoint{0.570336in}{1.843081in}}%
\pgfpathcurveto{\pgfqpoint{0.564512in}{1.837257in}}{\pgfqpoint{0.561240in}{1.829357in}}{\pgfqpoint{0.561240in}{1.821121in}}%
\pgfpathcurveto{\pgfqpoint{0.561240in}{1.812884in}}{\pgfqpoint{0.564512in}{1.804984in}}{\pgfqpoint{0.570336in}{1.799160in}}%
\pgfpathcurveto{\pgfqpoint{0.576160in}{1.793336in}}{\pgfqpoint{0.584060in}{1.790064in}}{\pgfqpoint{0.592297in}{1.790064in}}%
\pgfpathclose%
\pgfusepath{stroke,fill}%
\end{pgfscope}%
\begin{pgfscope}%
\pgfpathrectangle{\pgfqpoint{0.457963in}{0.528059in}}{\pgfqpoint{6.200000in}{2.285714in}} %
\pgfusepath{clip}%
\pgfsetbuttcap%
\pgfsetroundjoin%
\definecolor{currentfill}{rgb}{1.000000,0.333333,0.333333}%
\pgfsetfillcolor{currentfill}%
\pgfsetlinewidth{1.003750pt}%
\definecolor{currentstroke}{rgb}{1.000000,0.333333,0.333333}%
\pgfsetstrokecolor{currentstroke}%
\pgfsetdash{}{0pt}%
\pgfpathmoveto{\pgfqpoint{0.643963in}{1.777003in}}%
\pgfpathcurveto{\pgfqpoint{0.652200in}{1.777003in}}{\pgfqpoint{0.660100in}{1.780275in}}{\pgfqpoint{0.665924in}{1.786099in}}%
\pgfpathcurveto{\pgfqpoint{0.671748in}{1.791923in}}{\pgfqpoint{0.675020in}{1.799823in}}{\pgfqpoint{0.675020in}{1.808059in}}%
\pgfpathcurveto{\pgfqpoint{0.675020in}{1.816296in}}{\pgfqpoint{0.671748in}{1.824196in}}{\pgfqpoint{0.665924in}{1.830020in}}%
\pgfpathcurveto{\pgfqpoint{0.660100in}{1.835844in}}{\pgfqpoint{0.652200in}{1.839116in}}{\pgfqpoint{0.643963in}{1.839116in}}%
\pgfpathcurveto{\pgfqpoint{0.635727in}{1.839116in}}{\pgfqpoint{0.627827in}{1.835844in}}{\pgfqpoint{0.622003in}{1.830020in}}%
\pgfpathcurveto{\pgfqpoint{0.616179in}{1.824196in}}{\pgfqpoint{0.612907in}{1.816296in}}{\pgfqpoint{0.612907in}{1.808059in}}%
\pgfpathcurveto{\pgfqpoint{0.612907in}{1.799823in}}{\pgfqpoint{0.616179in}{1.791923in}}{\pgfqpoint{0.622003in}{1.786099in}}%
\pgfpathcurveto{\pgfqpoint{0.627827in}{1.780275in}}{\pgfqpoint{0.635727in}{1.777003in}}{\pgfqpoint{0.643963in}{1.777003in}}%
\pgfpathclose%
\pgfusepath{stroke,fill}%
\end{pgfscope}%
\begin{pgfscope}%
\pgfpathrectangle{\pgfqpoint{0.457963in}{0.528059in}}{\pgfqpoint{6.200000in}{2.285714in}} %
\pgfusepath{clip}%
\pgfsetbuttcap%
\pgfsetroundjoin%
\definecolor{currentfill}{rgb}{1.000000,0.333333,0.333333}%
\pgfsetfillcolor{currentfill}%
\pgfsetlinewidth{1.003750pt}%
\definecolor{currentstroke}{rgb}{1.000000,0.333333,0.333333}%
\pgfsetstrokecolor{currentstroke}%
\pgfsetdash{}{0pt}%
\pgfpathmoveto{\pgfqpoint{0.881630in}{1.803125in}}%
\pgfpathcurveto{\pgfqpoint{0.889866in}{1.803125in}}{\pgfqpoint{0.897766in}{1.806398in}}{\pgfqpoint{0.903590in}{1.812222in}}%
\pgfpathcurveto{\pgfqpoint{0.909414in}{1.818046in}}{\pgfqpoint{0.912686in}{1.825946in}}{\pgfqpoint{0.912686in}{1.834182in}}%
\pgfpathcurveto{\pgfqpoint{0.912686in}{1.842418in}}{\pgfqpoint{0.909414in}{1.850318in}}{\pgfqpoint{0.903590in}{1.856142in}}%
\pgfpathcurveto{\pgfqpoint{0.897766in}{1.861966in}}{\pgfqpoint{0.889866in}{1.865238in}}{\pgfqpoint{0.881630in}{1.865238in}}%
\pgfpathcurveto{\pgfqpoint{0.873394in}{1.865238in}}{\pgfqpoint{0.865494in}{1.861966in}}{\pgfqpoint{0.859670in}{1.856142in}}%
\pgfpathcurveto{\pgfqpoint{0.853846in}{1.850318in}}{\pgfqpoint{0.850574in}{1.842418in}}{\pgfqpoint{0.850574in}{1.834182in}}%
\pgfpathcurveto{\pgfqpoint{0.850574in}{1.825946in}}{\pgfqpoint{0.853846in}{1.818046in}}{\pgfqpoint{0.859670in}{1.812222in}}%
\pgfpathcurveto{\pgfqpoint{0.865494in}{1.806398in}}{\pgfqpoint{0.873394in}{1.803125in}}{\pgfqpoint{0.881630in}{1.803125in}}%
\pgfpathclose%
\pgfusepath{stroke,fill}%
\end{pgfscope}%
\begin{pgfscope}%
\pgfpathrectangle{\pgfqpoint{0.457963in}{0.528059in}}{\pgfqpoint{6.200000in}{2.285714in}} %
\pgfusepath{clip}%
\pgfsetbuttcap%
\pgfsetroundjoin%
\definecolor{currentfill}{rgb}{1.000000,0.333333,0.333333}%
\pgfsetfillcolor{currentfill}%
\pgfsetlinewidth{1.003750pt}%
\definecolor{currentstroke}{rgb}{1.000000,0.333333,0.333333}%
\pgfsetstrokecolor{currentstroke}%
\pgfsetdash{}{0pt}%
\pgfpathmoveto{\pgfqpoint{1.139963in}{1.019452in}}%
\pgfpathcurveto{\pgfqpoint{1.148200in}{1.019452in}}{\pgfqpoint{1.156100in}{1.022724in}}{\pgfqpoint{1.161924in}{1.028548in}}%
\pgfpathcurveto{\pgfqpoint{1.167748in}{1.034372in}}{\pgfqpoint{1.171020in}{1.042272in}}{\pgfqpoint{1.171020in}{1.050508in}}%
\pgfpathcurveto{\pgfqpoint{1.171020in}{1.058745in}}{\pgfqpoint{1.167748in}{1.066645in}}{\pgfqpoint{1.161924in}{1.072469in}}%
\pgfpathcurveto{\pgfqpoint{1.156100in}{1.078293in}}{\pgfqpoint{1.148200in}{1.081565in}}{\pgfqpoint{1.139963in}{1.081565in}}%
\pgfpathcurveto{\pgfqpoint{1.131727in}{1.081565in}}{\pgfqpoint{1.123827in}{1.078293in}}{\pgfqpoint{1.118003in}{1.072469in}}%
\pgfpathcurveto{\pgfqpoint{1.112179in}{1.066645in}}{\pgfqpoint{1.108907in}{1.058745in}}{\pgfqpoint{1.108907in}{1.050508in}}%
\pgfpathcurveto{\pgfqpoint{1.108907in}{1.042272in}}{\pgfqpoint{1.112179in}{1.034372in}}{\pgfqpoint{1.118003in}{1.028548in}}%
\pgfpathcurveto{\pgfqpoint{1.123827in}{1.022724in}}{\pgfqpoint{1.131727in}{1.019452in}}{\pgfqpoint{1.139963in}{1.019452in}}%
\pgfpathclose%
\pgfusepath{stroke,fill}%
\end{pgfscope}%
\begin{pgfscope}%
\pgfpathrectangle{\pgfqpoint{0.457963in}{0.528059in}}{\pgfqpoint{6.200000in}{2.285714in}} %
\pgfusepath{clip}%
\pgfsetbuttcap%
\pgfsetroundjoin%
\definecolor{currentfill}{rgb}{1.000000,0.333333,0.333333}%
\pgfsetfillcolor{currentfill}%
\pgfsetlinewidth{1.003750pt}%
\definecolor{currentstroke}{rgb}{1.000000,0.333333,0.333333}%
\pgfsetstrokecolor{currentstroke}%
\pgfsetdash{}{0pt}%
\pgfpathmoveto{\pgfqpoint{1.305297in}{1.659452in}}%
\pgfpathcurveto{\pgfqpoint{1.313533in}{1.659452in}}{\pgfqpoint{1.321433in}{1.662724in}}{\pgfqpoint{1.327257in}{1.668548in}}%
\pgfpathcurveto{\pgfqpoint{1.333081in}{1.674372in}}{\pgfqpoint{1.336353in}{1.682272in}}{\pgfqpoint{1.336353in}{1.690508in}}%
\pgfpathcurveto{\pgfqpoint{1.336353in}{1.698745in}}{\pgfqpoint{1.333081in}{1.706645in}}{\pgfqpoint{1.327257in}{1.712469in}}%
\pgfpathcurveto{\pgfqpoint{1.321433in}{1.718293in}}{\pgfqpoint{1.313533in}{1.721565in}}{\pgfqpoint{1.305297in}{1.721565in}}%
\pgfpathcurveto{\pgfqpoint{1.297060in}{1.721565in}}{\pgfqpoint{1.289160in}{1.718293in}}{\pgfqpoint{1.283336in}{1.712469in}}%
\pgfpathcurveto{\pgfqpoint{1.277512in}{1.706645in}}{\pgfqpoint{1.274240in}{1.698745in}}{\pgfqpoint{1.274240in}{1.690508in}}%
\pgfpathcurveto{\pgfqpoint{1.274240in}{1.682272in}}{\pgfqpoint{1.277512in}{1.674372in}}{\pgfqpoint{1.283336in}{1.668548in}}%
\pgfpathcurveto{\pgfqpoint{1.289160in}{1.662724in}}{\pgfqpoint{1.297060in}{1.659452in}}{\pgfqpoint{1.305297in}{1.659452in}}%
\pgfpathclose%
\pgfusepath{stroke,fill}%
\end{pgfscope}%
\begin{pgfscope}%
\pgfpathrectangle{\pgfqpoint{0.457963in}{0.528059in}}{\pgfqpoint{6.200000in}{2.285714in}} %
\pgfusepath{clip}%
\pgfsetbuttcap%
\pgfsetroundjoin%
\definecolor{currentfill}{rgb}{1.000000,0.333333,0.333333}%
\pgfsetfillcolor{currentfill}%
\pgfsetlinewidth{1.003750pt}%
\definecolor{currentstroke}{rgb}{1.000000,0.333333,0.333333}%
\pgfsetstrokecolor{currentstroke}%
\pgfsetdash{}{0pt}%
\pgfpathmoveto{\pgfqpoint{1.677297in}{1.763942in}}%
\pgfpathcurveto{\pgfqpoint{1.685533in}{1.763942in}}{\pgfqpoint{1.693433in}{1.767214in}}{\pgfqpoint{1.699257in}{1.773038in}}%
\pgfpathcurveto{\pgfqpoint{1.705081in}{1.778862in}}{\pgfqpoint{1.708353in}{1.786762in}}{\pgfqpoint{1.708353in}{1.794998in}}%
\pgfpathcurveto{\pgfqpoint{1.708353in}{1.803234in}}{\pgfqpoint{1.705081in}{1.811135in}}{\pgfqpoint{1.699257in}{1.816958in}}%
\pgfpathcurveto{\pgfqpoint{1.693433in}{1.822782in}}{\pgfqpoint{1.685533in}{1.826055in}}{\pgfqpoint{1.677297in}{1.826055in}}%
\pgfpathcurveto{\pgfqpoint{1.669060in}{1.826055in}}{\pgfqpoint{1.661160in}{1.822782in}}{\pgfqpoint{1.655336in}{1.816958in}}%
\pgfpathcurveto{\pgfqpoint{1.649512in}{1.811135in}}{\pgfqpoint{1.646240in}{1.803234in}}{\pgfqpoint{1.646240in}{1.794998in}}%
\pgfpathcurveto{\pgfqpoint{1.646240in}{1.786762in}}{\pgfqpoint{1.649512in}{1.778862in}}{\pgfqpoint{1.655336in}{1.773038in}}%
\pgfpathcurveto{\pgfqpoint{1.661160in}{1.767214in}}{\pgfqpoint{1.669060in}{1.763942in}}{\pgfqpoint{1.677297in}{1.763942in}}%
\pgfpathclose%
\pgfusepath{stroke,fill}%
\end{pgfscope}%
\begin{pgfscope}%
\pgfpathrectangle{\pgfqpoint{0.457963in}{0.528059in}}{\pgfqpoint{6.200000in}{2.285714in}} %
\pgfusepath{clip}%
\pgfsetbuttcap%
\pgfsetroundjoin%
\definecolor{currentfill}{rgb}{1.000000,0.333333,0.333333}%
\pgfsetfillcolor{currentfill}%
\pgfsetlinewidth{1.003750pt}%
\definecolor{currentstroke}{rgb}{1.000000,0.333333,0.333333}%
\pgfsetstrokecolor{currentstroke}%
\pgfsetdash{}{0pt}%
\pgfpathmoveto{\pgfqpoint{1.945963in}{1.724758in}}%
\pgfpathcurveto{\pgfqpoint{1.954200in}{1.724758in}}{\pgfqpoint{1.962100in}{1.728030in}}{\pgfqpoint{1.967924in}{1.733854in}}%
\pgfpathcurveto{\pgfqpoint{1.973748in}{1.739678in}}{\pgfqpoint{1.977020in}{1.747578in}}{\pgfqpoint{1.977020in}{1.755815in}}%
\pgfpathcurveto{\pgfqpoint{1.977020in}{1.764051in}}{\pgfqpoint{1.973748in}{1.771951in}}{\pgfqpoint{1.967924in}{1.777775in}}%
\pgfpathcurveto{\pgfqpoint{1.962100in}{1.783599in}}{\pgfqpoint{1.954200in}{1.786871in}}{\pgfqpoint{1.945963in}{1.786871in}}%
\pgfpathcurveto{\pgfqpoint{1.937727in}{1.786871in}}{\pgfqpoint{1.929827in}{1.783599in}}{\pgfqpoint{1.924003in}{1.777775in}}%
\pgfpathcurveto{\pgfqpoint{1.918179in}{1.771951in}}{\pgfqpoint{1.914907in}{1.764051in}}{\pgfqpoint{1.914907in}{1.755815in}}%
\pgfpathcurveto{\pgfqpoint{1.914907in}{1.747578in}}{\pgfqpoint{1.918179in}{1.739678in}}{\pgfqpoint{1.924003in}{1.733854in}}%
\pgfpathcurveto{\pgfqpoint{1.929827in}{1.728030in}}{\pgfqpoint{1.937727in}{1.724758in}}{\pgfqpoint{1.945963in}{1.724758in}}%
\pgfpathclose%
\pgfusepath{stroke,fill}%
\end{pgfscope}%
\begin{pgfscope}%
\pgfpathrectangle{\pgfqpoint{0.457963in}{0.528059in}}{\pgfqpoint{6.200000in}{2.285714in}} %
\pgfusepath{clip}%
\pgfsetbuttcap%
\pgfsetroundjoin%
\definecolor{currentfill}{rgb}{1.000000,0.333333,0.333333}%
\pgfsetfillcolor{currentfill}%
\pgfsetlinewidth{1.003750pt}%
\definecolor{currentstroke}{rgb}{1.000000,0.333333,0.333333}%
\pgfsetstrokecolor{currentstroke}%
\pgfsetdash{}{0pt}%
\pgfpathmoveto{\pgfqpoint{2.038963in}{0.705983in}}%
\pgfpathcurveto{\pgfqpoint{2.047200in}{0.705983in}}{\pgfqpoint{2.055100in}{0.709255in}}{\pgfqpoint{2.060924in}{0.715079in}}%
\pgfpathcurveto{\pgfqpoint{2.066748in}{0.720903in}}{\pgfqpoint{2.070020in}{0.728803in}}{\pgfqpoint{2.070020in}{0.737039in}}%
\pgfpathcurveto{\pgfqpoint{2.070020in}{0.745275in}}{\pgfqpoint{2.066748in}{0.753175in}}{\pgfqpoint{2.060924in}{0.758999in}}%
\pgfpathcurveto{\pgfqpoint{2.055100in}{0.764823in}}{\pgfqpoint{2.047200in}{0.768096in}}{\pgfqpoint{2.038963in}{0.768096in}}%
\pgfpathcurveto{\pgfqpoint{2.030727in}{0.768096in}}{\pgfqpoint{2.022827in}{0.764823in}}{\pgfqpoint{2.017003in}{0.758999in}}%
\pgfpathcurveto{\pgfqpoint{2.011179in}{0.753175in}}{\pgfqpoint{2.007907in}{0.745275in}}{\pgfqpoint{2.007907in}{0.737039in}}%
\pgfpathcurveto{\pgfqpoint{2.007907in}{0.728803in}}{\pgfqpoint{2.011179in}{0.720903in}}{\pgfqpoint{2.017003in}{0.715079in}}%
\pgfpathcurveto{\pgfqpoint{2.022827in}{0.709255in}}{\pgfqpoint{2.030727in}{0.705983in}}{\pgfqpoint{2.038963in}{0.705983in}}%
\pgfpathclose%
\pgfusepath{stroke,fill}%
\end{pgfscope}%
\begin{pgfscope}%
\pgfpathrectangle{\pgfqpoint{0.457963in}{0.528059in}}{\pgfqpoint{6.200000in}{2.285714in}} %
\pgfusepath{clip}%
\pgfsetbuttcap%
\pgfsetroundjoin%
\definecolor{currentfill}{rgb}{1.000000,0.333333,0.333333}%
\pgfsetfillcolor{currentfill}%
\pgfsetlinewidth{1.003750pt}%
\definecolor{currentstroke}{rgb}{1.000000,0.333333,0.333333}%
\pgfsetstrokecolor{currentstroke}%
\pgfsetdash{}{0pt}%
\pgfpathmoveto{\pgfqpoint{2.152630in}{1.633329in}}%
\pgfpathcurveto{\pgfqpoint{2.160866in}{1.633329in}}{\pgfqpoint{2.168766in}{1.636602in}}{\pgfqpoint{2.174590in}{1.642426in}}%
\pgfpathcurveto{\pgfqpoint{2.180414in}{1.648250in}}{\pgfqpoint{2.183686in}{1.656150in}}{\pgfqpoint{2.183686in}{1.664386in}}%
\pgfpathcurveto{\pgfqpoint{2.183686in}{1.672622in}}{\pgfqpoint{2.180414in}{1.680522in}}{\pgfqpoint{2.174590in}{1.686346in}}%
\pgfpathcurveto{\pgfqpoint{2.168766in}{1.692170in}}{\pgfqpoint{2.160866in}{1.695442in}}{\pgfqpoint{2.152630in}{1.695442in}}%
\pgfpathcurveto{\pgfqpoint{2.144394in}{1.695442in}}{\pgfqpoint{2.136494in}{1.692170in}}{\pgfqpoint{2.130670in}{1.686346in}}%
\pgfpathcurveto{\pgfqpoint{2.124846in}{1.680522in}}{\pgfqpoint{2.121574in}{1.672622in}}{\pgfqpoint{2.121574in}{1.664386in}}%
\pgfpathcurveto{\pgfqpoint{2.121574in}{1.656150in}}{\pgfqpoint{2.124846in}{1.648250in}}{\pgfqpoint{2.130670in}{1.642426in}}%
\pgfpathcurveto{\pgfqpoint{2.136494in}{1.636602in}}{\pgfqpoint{2.144394in}{1.633329in}}{\pgfqpoint{2.152630in}{1.633329in}}%
\pgfpathclose%
\pgfusepath{stroke,fill}%
\end{pgfscope}%
\begin{pgfscope}%
\pgfpathrectangle{\pgfqpoint{0.457963in}{0.528059in}}{\pgfqpoint{6.200000in}{2.285714in}} %
\pgfusepath{clip}%
\pgfsetbuttcap%
\pgfsetroundjoin%
\definecolor{currentfill}{rgb}{1.000000,0.333333,0.333333}%
\pgfsetfillcolor{currentfill}%
\pgfsetlinewidth{1.003750pt}%
\definecolor{currentstroke}{rgb}{1.000000,0.333333,0.333333}%
\pgfsetstrokecolor{currentstroke}%
\pgfsetdash{}{0pt}%
\pgfpathmoveto{\pgfqpoint{2.204297in}{1.163125in}}%
\pgfpathcurveto{\pgfqpoint{2.212533in}{1.163125in}}{\pgfqpoint{2.220433in}{1.166398in}}{\pgfqpoint{2.226257in}{1.172222in}}%
\pgfpathcurveto{\pgfqpoint{2.232081in}{1.178046in}}{\pgfqpoint{2.235353in}{1.185946in}}{\pgfqpoint{2.235353in}{1.194182in}}%
\pgfpathcurveto{\pgfqpoint{2.235353in}{1.202418in}}{\pgfqpoint{2.232081in}{1.210318in}}{\pgfqpoint{2.226257in}{1.216142in}}%
\pgfpathcurveto{\pgfqpoint{2.220433in}{1.221966in}}{\pgfqpoint{2.212533in}{1.225238in}}{\pgfqpoint{2.204297in}{1.225238in}}%
\pgfpathcurveto{\pgfqpoint{2.196060in}{1.225238in}}{\pgfqpoint{2.188160in}{1.221966in}}{\pgfqpoint{2.182336in}{1.216142in}}%
\pgfpathcurveto{\pgfqpoint{2.176512in}{1.210318in}}{\pgfqpoint{2.173240in}{1.202418in}}{\pgfqpoint{2.173240in}{1.194182in}}%
\pgfpathcurveto{\pgfqpoint{2.173240in}{1.185946in}}{\pgfqpoint{2.176512in}{1.178046in}}{\pgfqpoint{2.182336in}{1.172222in}}%
\pgfpathcurveto{\pgfqpoint{2.188160in}{1.166398in}}{\pgfqpoint{2.196060in}{1.163125in}}{\pgfqpoint{2.204297in}{1.163125in}}%
\pgfpathclose%
\pgfusepath{stroke,fill}%
\end{pgfscope}%
\begin{pgfscope}%
\pgfpathrectangle{\pgfqpoint{0.457963in}{0.528059in}}{\pgfqpoint{6.200000in}{2.285714in}} %
\pgfusepath{clip}%
\pgfsetbuttcap%
\pgfsetroundjoin%
\definecolor{currentfill}{rgb}{1.000000,0.333333,0.333333}%
\pgfsetfillcolor{currentfill}%
\pgfsetlinewidth{1.003750pt}%
\definecolor{currentstroke}{rgb}{1.000000,0.333333,0.333333}%
\pgfsetstrokecolor{currentstroke}%
\pgfsetdash{}{0pt}%
\pgfpathmoveto{\pgfqpoint{3.630297in}{0.993329in}}%
\pgfpathcurveto{\pgfqpoint{3.638533in}{0.993329in}}{\pgfqpoint{3.646433in}{0.996602in}}{\pgfqpoint{3.652257in}{1.002426in}}%
\pgfpathcurveto{\pgfqpoint{3.658081in}{1.008250in}}{\pgfqpoint{3.661353in}{1.016150in}}{\pgfqpoint{3.661353in}{1.024386in}}%
\pgfpathcurveto{\pgfqpoint{3.661353in}{1.032622in}}{\pgfqpoint{3.658081in}{1.040522in}}{\pgfqpoint{3.652257in}{1.046346in}}%
\pgfpathcurveto{\pgfqpoint{3.646433in}{1.052170in}}{\pgfqpoint{3.638533in}{1.055442in}}{\pgfqpoint{3.630297in}{1.055442in}}%
\pgfpathcurveto{\pgfqpoint{3.622060in}{1.055442in}}{\pgfqpoint{3.614160in}{1.052170in}}{\pgfqpoint{3.608336in}{1.046346in}}%
\pgfpathcurveto{\pgfqpoint{3.602512in}{1.040522in}}{\pgfqpoint{3.599240in}{1.032622in}}{\pgfqpoint{3.599240in}{1.024386in}}%
\pgfpathcurveto{\pgfqpoint{3.599240in}{1.016150in}}{\pgfqpoint{3.602512in}{1.008250in}}{\pgfqpoint{3.608336in}{1.002426in}}%
\pgfpathcurveto{\pgfqpoint{3.614160in}{0.996602in}}{\pgfqpoint{3.622060in}{0.993329in}}{\pgfqpoint{3.630297in}{0.993329in}}%
\pgfpathclose%
\pgfusepath{stroke,fill}%
\end{pgfscope}%
\begin{pgfscope}%
\pgfpathrectangle{\pgfqpoint{0.457963in}{0.528059in}}{\pgfqpoint{6.200000in}{2.285714in}} %
\pgfusepath{clip}%
\pgfsetbuttcap%
\pgfsetroundjoin%
\definecolor{currentfill}{rgb}{1.000000,0.166667,0.166667}%
\pgfsetfillcolor{currentfill}%
\pgfsetlinewidth{1.003750pt}%
\definecolor{currentstroke}{rgb}{1.000000,0.166667,0.166667}%
\pgfsetstrokecolor{currentstroke}%
\pgfsetdash{}{0pt}%
\pgfpathmoveto{\pgfqpoint{0.457963in}{2.129656in}}%
\pgfpathcurveto{\pgfqpoint{0.466200in}{2.129656in}}{\pgfqpoint{0.474100in}{2.132928in}}{\pgfqpoint{0.479924in}{2.138752in}}%
\pgfpathcurveto{\pgfqpoint{0.485748in}{2.144576in}}{\pgfqpoint{0.489020in}{2.152476in}}{\pgfqpoint{0.489020in}{2.160713in}}%
\pgfpathcurveto{\pgfqpoint{0.489020in}{2.168949in}}{\pgfqpoint{0.485748in}{2.176849in}}{\pgfqpoint{0.479924in}{2.182673in}}%
\pgfpathcurveto{\pgfqpoint{0.474100in}{2.188497in}}{\pgfqpoint{0.466200in}{2.191769in}}{\pgfqpoint{0.457963in}{2.191769in}}%
\pgfpathcurveto{\pgfqpoint{0.449727in}{2.191769in}}{\pgfqpoint{0.441827in}{2.188497in}}{\pgfqpoint{0.436003in}{2.182673in}}%
\pgfpathcurveto{\pgfqpoint{0.430179in}{2.176849in}}{\pgfqpoint{0.426907in}{2.168949in}}{\pgfqpoint{0.426907in}{2.160713in}}%
\pgfpathcurveto{\pgfqpoint{0.426907in}{2.152476in}}{\pgfqpoint{0.430179in}{2.144576in}}{\pgfqpoint{0.436003in}{2.138752in}}%
\pgfpathcurveto{\pgfqpoint{0.441827in}{2.132928in}}{\pgfqpoint{0.449727in}{2.129656in}}{\pgfqpoint{0.457963in}{2.129656in}}%
\pgfpathclose%
\pgfusepath{stroke,fill}%
\end{pgfscope}%
\begin{pgfscope}%
\pgfpathrectangle{\pgfqpoint{0.457963in}{0.528059in}}{\pgfqpoint{6.200000in}{2.285714in}} %
\pgfusepath{clip}%
\pgfsetbuttcap%
\pgfsetroundjoin%
\definecolor{currentfill}{rgb}{1.000000,0.166667,0.166667}%
\pgfsetfillcolor{currentfill}%
\pgfsetlinewidth{1.003750pt}%
\definecolor{currentstroke}{rgb}{1.000000,0.166667,0.166667}%
\pgfsetstrokecolor{currentstroke}%
\pgfsetdash{}{0pt}%
\pgfpathmoveto{\pgfqpoint{0.457963in}{2.129656in}}%
\pgfpathcurveto{\pgfqpoint{0.466200in}{2.129656in}}{\pgfqpoint{0.474100in}{2.132928in}}{\pgfqpoint{0.479924in}{2.138752in}}%
\pgfpathcurveto{\pgfqpoint{0.485748in}{2.144576in}}{\pgfqpoint{0.489020in}{2.152476in}}{\pgfqpoint{0.489020in}{2.160713in}}%
\pgfpathcurveto{\pgfqpoint{0.489020in}{2.168949in}}{\pgfqpoint{0.485748in}{2.176849in}}{\pgfqpoint{0.479924in}{2.182673in}}%
\pgfpathcurveto{\pgfqpoint{0.474100in}{2.188497in}}{\pgfqpoint{0.466200in}{2.191769in}}{\pgfqpoint{0.457963in}{2.191769in}}%
\pgfpathcurveto{\pgfqpoint{0.449727in}{2.191769in}}{\pgfqpoint{0.441827in}{2.188497in}}{\pgfqpoint{0.436003in}{2.182673in}}%
\pgfpathcurveto{\pgfqpoint{0.430179in}{2.176849in}}{\pgfqpoint{0.426907in}{2.168949in}}{\pgfqpoint{0.426907in}{2.160713in}}%
\pgfpathcurveto{\pgfqpoint{0.426907in}{2.152476in}}{\pgfqpoint{0.430179in}{2.144576in}}{\pgfqpoint{0.436003in}{2.138752in}}%
\pgfpathcurveto{\pgfqpoint{0.441827in}{2.132928in}}{\pgfqpoint{0.449727in}{2.129656in}}{\pgfqpoint{0.457963in}{2.129656in}}%
\pgfpathclose%
\pgfusepath{stroke,fill}%
\end{pgfscope}%
\begin{pgfscope}%
\pgfpathrectangle{\pgfqpoint{0.457963in}{0.528059in}}{\pgfqpoint{6.200000in}{2.285714in}} %
\pgfusepath{clip}%
\pgfsetbuttcap%
\pgfsetroundjoin%
\definecolor{currentfill}{rgb}{1.000000,0.166667,0.166667}%
\pgfsetfillcolor{currentfill}%
\pgfsetlinewidth{1.003750pt}%
\definecolor{currentstroke}{rgb}{1.000000,0.166667,0.166667}%
\pgfsetstrokecolor{currentstroke}%
\pgfsetdash{}{0pt}%
\pgfpathmoveto{\pgfqpoint{0.457963in}{2.129656in}}%
\pgfpathcurveto{\pgfqpoint{0.466200in}{2.129656in}}{\pgfqpoint{0.474100in}{2.132928in}}{\pgfqpoint{0.479924in}{2.138752in}}%
\pgfpathcurveto{\pgfqpoint{0.485748in}{2.144576in}}{\pgfqpoint{0.489020in}{2.152476in}}{\pgfqpoint{0.489020in}{2.160713in}}%
\pgfpathcurveto{\pgfqpoint{0.489020in}{2.168949in}}{\pgfqpoint{0.485748in}{2.176849in}}{\pgfqpoint{0.479924in}{2.182673in}}%
\pgfpathcurveto{\pgfqpoint{0.474100in}{2.188497in}}{\pgfqpoint{0.466200in}{2.191769in}}{\pgfqpoint{0.457963in}{2.191769in}}%
\pgfpathcurveto{\pgfqpoint{0.449727in}{2.191769in}}{\pgfqpoint{0.441827in}{2.188497in}}{\pgfqpoint{0.436003in}{2.182673in}}%
\pgfpathcurveto{\pgfqpoint{0.430179in}{2.176849in}}{\pgfqpoint{0.426907in}{2.168949in}}{\pgfqpoint{0.426907in}{2.160713in}}%
\pgfpathcurveto{\pgfqpoint{0.426907in}{2.152476in}}{\pgfqpoint{0.430179in}{2.144576in}}{\pgfqpoint{0.436003in}{2.138752in}}%
\pgfpathcurveto{\pgfqpoint{0.441827in}{2.132928in}}{\pgfqpoint{0.449727in}{2.129656in}}{\pgfqpoint{0.457963in}{2.129656in}}%
\pgfpathclose%
\pgfusepath{stroke,fill}%
\end{pgfscope}%
\begin{pgfscope}%
\pgfpathrectangle{\pgfqpoint{0.457963in}{0.528059in}}{\pgfqpoint{6.200000in}{2.285714in}} %
\pgfusepath{clip}%
\pgfsetbuttcap%
\pgfsetroundjoin%
\definecolor{currentfill}{rgb}{1.000000,0.166667,0.166667}%
\pgfsetfillcolor{currentfill}%
\pgfsetlinewidth{1.003750pt}%
\definecolor{currentstroke}{rgb}{1.000000,0.166667,0.166667}%
\pgfsetstrokecolor{currentstroke}%
\pgfsetdash{}{0pt}%
\pgfpathmoveto{\pgfqpoint{0.457963in}{2.129656in}}%
\pgfpathcurveto{\pgfqpoint{0.466200in}{2.129656in}}{\pgfqpoint{0.474100in}{2.132928in}}{\pgfqpoint{0.479924in}{2.138752in}}%
\pgfpathcurveto{\pgfqpoint{0.485748in}{2.144576in}}{\pgfqpoint{0.489020in}{2.152476in}}{\pgfqpoint{0.489020in}{2.160713in}}%
\pgfpathcurveto{\pgfqpoint{0.489020in}{2.168949in}}{\pgfqpoint{0.485748in}{2.176849in}}{\pgfqpoint{0.479924in}{2.182673in}}%
\pgfpathcurveto{\pgfqpoint{0.474100in}{2.188497in}}{\pgfqpoint{0.466200in}{2.191769in}}{\pgfqpoint{0.457963in}{2.191769in}}%
\pgfpathcurveto{\pgfqpoint{0.449727in}{2.191769in}}{\pgfqpoint{0.441827in}{2.188497in}}{\pgfqpoint{0.436003in}{2.182673in}}%
\pgfpathcurveto{\pgfqpoint{0.430179in}{2.176849in}}{\pgfqpoint{0.426907in}{2.168949in}}{\pgfqpoint{0.426907in}{2.160713in}}%
\pgfpathcurveto{\pgfqpoint{0.426907in}{2.152476in}}{\pgfqpoint{0.430179in}{2.144576in}}{\pgfqpoint{0.436003in}{2.138752in}}%
\pgfpathcurveto{\pgfqpoint{0.441827in}{2.132928in}}{\pgfqpoint{0.449727in}{2.129656in}}{\pgfqpoint{0.457963in}{2.129656in}}%
\pgfpathclose%
\pgfusepath{stroke,fill}%
\end{pgfscope}%
\begin{pgfscope}%
\pgfpathrectangle{\pgfqpoint{0.457963in}{0.528059in}}{\pgfqpoint{6.200000in}{2.285714in}} %
\pgfusepath{clip}%
\pgfsetbuttcap%
\pgfsetroundjoin%
\definecolor{currentfill}{rgb}{1.000000,0.166667,0.166667}%
\pgfsetfillcolor{currentfill}%
\pgfsetlinewidth{1.003750pt}%
\definecolor{currentstroke}{rgb}{1.000000,0.166667,0.166667}%
\pgfsetstrokecolor{currentstroke}%
\pgfsetdash{}{0pt}%
\pgfpathmoveto{\pgfqpoint{0.457963in}{2.129656in}}%
\pgfpathcurveto{\pgfqpoint{0.466200in}{2.129656in}}{\pgfqpoint{0.474100in}{2.132928in}}{\pgfqpoint{0.479924in}{2.138752in}}%
\pgfpathcurveto{\pgfqpoint{0.485748in}{2.144576in}}{\pgfqpoint{0.489020in}{2.152476in}}{\pgfqpoint{0.489020in}{2.160713in}}%
\pgfpathcurveto{\pgfqpoint{0.489020in}{2.168949in}}{\pgfqpoint{0.485748in}{2.176849in}}{\pgfqpoint{0.479924in}{2.182673in}}%
\pgfpathcurveto{\pgfqpoint{0.474100in}{2.188497in}}{\pgfqpoint{0.466200in}{2.191769in}}{\pgfqpoint{0.457963in}{2.191769in}}%
\pgfpathcurveto{\pgfqpoint{0.449727in}{2.191769in}}{\pgfqpoint{0.441827in}{2.188497in}}{\pgfqpoint{0.436003in}{2.182673in}}%
\pgfpathcurveto{\pgfqpoint{0.430179in}{2.176849in}}{\pgfqpoint{0.426907in}{2.168949in}}{\pgfqpoint{0.426907in}{2.160713in}}%
\pgfpathcurveto{\pgfqpoint{0.426907in}{2.152476in}}{\pgfqpoint{0.430179in}{2.144576in}}{\pgfqpoint{0.436003in}{2.138752in}}%
\pgfpathcurveto{\pgfqpoint{0.441827in}{2.132928in}}{\pgfqpoint{0.449727in}{2.129656in}}{\pgfqpoint{0.457963in}{2.129656in}}%
\pgfpathclose%
\pgfusepath{stroke,fill}%
\end{pgfscope}%
\begin{pgfscope}%
\pgfpathrectangle{\pgfqpoint{0.457963in}{0.528059in}}{\pgfqpoint{6.200000in}{2.285714in}} %
\pgfusepath{clip}%
\pgfsetbuttcap%
\pgfsetroundjoin%
\definecolor{currentfill}{rgb}{1.000000,0.166667,0.166667}%
\pgfsetfillcolor{currentfill}%
\pgfsetlinewidth{1.003750pt}%
\definecolor{currentstroke}{rgb}{1.000000,0.166667,0.166667}%
\pgfsetstrokecolor{currentstroke}%
\pgfsetdash{}{0pt}%
\pgfpathmoveto{\pgfqpoint{0.468297in}{2.064350in}}%
\pgfpathcurveto{\pgfqpoint{0.476533in}{2.064350in}}{\pgfqpoint{0.484433in}{2.067622in}}{\pgfqpoint{0.490257in}{2.073446in}}%
\pgfpathcurveto{\pgfqpoint{0.496081in}{2.079270in}}{\pgfqpoint{0.499353in}{2.087170in}}{\pgfqpoint{0.499353in}{2.095406in}}%
\pgfpathcurveto{\pgfqpoint{0.499353in}{2.103643in}}{\pgfqpoint{0.496081in}{2.111543in}}{\pgfqpoint{0.490257in}{2.117367in}}%
\pgfpathcurveto{\pgfqpoint{0.484433in}{2.123191in}}{\pgfqpoint{0.476533in}{2.126463in}}{\pgfqpoint{0.468297in}{2.126463in}}%
\pgfpathcurveto{\pgfqpoint{0.460060in}{2.126463in}}{\pgfqpoint{0.452160in}{2.123191in}}{\pgfqpoint{0.446336in}{2.117367in}}%
\pgfpathcurveto{\pgfqpoint{0.440512in}{2.111543in}}{\pgfqpoint{0.437240in}{2.103643in}}{\pgfqpoint{0.437240in}{2.095406in}}%
\pgfpathcurveto{\pgfqpoint{0.437240in}{2.087170in}}{\pgfqpoint{0.440512in}{2.079270in}}{\pgfqpoint{0.446336in}{2.073446in}}%
\pgfpathcurveto{\pgfqpoint{0.452160in}{2.067622in}}{\pgfqpoint{0.460060in}{2.064350in}}{\pgfqpoint{0.468297in}{2.064350in}}%
\pgfpathclose%
\pgfusepath{stroke,fill}%
\end{pgfscope}%
\begin{pgfscope}%
\pgfpathrectangle{\pgfqpoint{0.457963in}{0.528059in}}{\pgfqpoint{6.200000in}{2.285714in}} %
\pgfusepath{clip}%
\pgfsetbuttcap%
\pgfsetroundjoin%
\definecolor{currentfill}{rgb}{1.000000,0.166667,0.166667}%
\pgfsetfillcolor{currentfill}%
\pgfsetlinewidth{1.003750pt}%
\definecolor{currentstroke}{rgb}{1.000000,0.166667,0.166667}%
\pgfsetstrokecolor{currentstroke}%
\pgfsetdash{}{0pt}%
\pgfpathmoveto{\pgfqpoint{0.519963in}{2.064350in}}%
\pgfpathcurveto{\pgfqpoint{0.528200in}{2.064350in}}{\pgfqpoint{0.536100in}{2.067622in}}{\pgfqpoint{0.541924in}{2.073446in}}%
\pgfpathcurveto{\pgfqpoint{0.547748in}{2.079270in}}{\pgfqpoint{0.551020in}{2.087170in}}{\pgfqpoint{0.551020in}{2.095406in}}%
\pgfpathcurveto{\pgfqpoint{0.551020in}{2.103643in}}{\pgfqpoint{0.547748in}{2.111543in}}{\pgfqpoint{0.541924in}{2.117367in}}%
\pgfpathcurveto{\pgfqpoint{0.536100in}{2.123191in}}{\pgfqpoint{0.528200in}{2.126463in}}{\pgfqpoint{0.519963in}{2.126463in}}%
\pgfpathcurveto{\pgfqpoint{0.511727in}{2.126463in}}{\pgfqpoint{0.503827in}{2.123191in}}{\pgfqpoint{0.498003in}{2.117367in}}%
\pgfpathcurveto{\pgfqpoint{0.492179in}{2.111543in}}{\pgfqpoint{0.488907in}{2.103643in}}{\pgfqpoint{0.488907in}{2.095406in}}%
\pgfpathcurveto{\pgfqpoint{0.488907in}{2.087170in}}{\pgfqpoint{0.492179in}{2.079270in}}{\pgfqpoint{0.498003in}{2.073446in}}%
\pgfpathcurveto{\pgfqpoint{0.503827in}{2.067622in}}{\pgfqpoint{0.511727in}{2.064350in}}{\pgfqpoint{0.519963in}{2.064350in}}%
\pgfpathclose%
\pgfusepath{stroke,fill}%
\end{pgfscope}%
\begin{pgfscope}%
\pgfpathrectangle{\pgfqpoint{0.457963in}{0.528059in}}{\pgfqpoint{6.200000in}{2.285714in}} %
\pgfusepath{clip}%
\pgfsetbuttcap%
\pgfsetroundjoin%
\definecolor{currentfill}{rgb}{1.000000,0.166667,0.166667}%
\pgfsetfillcolor{currentfill}%
\pgfsetlinewidth{1.003750pt}%
\definecolor{currentstroke}{rgb}{1.000000,0.166667,0.166667}%
\pgfsetstrokecolor{currentstroke}%
\pgfsetdash{}{0pt}%
\pgfpathmoveto{\pgfqpoint{0.530297in}{2.129656in}}%
\pgfpathcurveto{\pgfqpoint{0.538533in}{2.129656in}}{\pgfqpoint{0.546433in}{2.132928in}}{\pgfqpoint{0.552257in}{2.138752in}}%
\pgfpathcurveto{\pgfqpoint{0.558081in}{2.144576in}}{\pgfqpoint{0.561353in}{2.152476in}}{\pgfqpoint{0.561353in}{2.160713in}}%
\pgfpathcurveto{\pgfqpoint{0.561353in}{2.168949in}}{\pgfqpoint{0.558081in}{2.176849in}}{\pgfqpoint{0.552257in}{2.182673in}}%
\pgfpathcurveto{\pgfqpoint{0.546433in}{2.188497in}}{\pgfqpoint{0.538533in}{2.191769in}}{\pgfqpoint{0.530297in}{2.191769in}}%
\pgfpathcurveto{\pgfqpoint{0.522060in}{2.191769in}}{\pgfqpoint{0.514160in}{2.188497in}}{\pgfqpoint{0.508336in}{2.182673in}}%
\pgfpathcurveto{\pgfqpoint{0.502512in}{2.176849in}}{\pgfqpoint{0.499240in}{2.168949in}}{\pgfqpoint{0.499240in}{2.160713in}}%
\pgfpathcurveto{\pgfqpoint{0.499240in}{2.152476in}}{\pgfqpoint{0.502512in}{2.144576in}}{\pgfqpoint{0.508336in}{2.138752in}}%
\pgfpathcurveto{\pgfqpoint{0.514160in}{2.132928in}}{\pgfqpoint{0.522060in}{2.129656in}}{\pgfqpoint{0.530297in}{2.129656in}}%
\pgfpathclose%
\pgfusepath{stroke,fill}%
\end{pgfscope}%
\begin{pgfscope}%
\pgfpathrectangle{\pgfqpoint{0.457963in}{0.528059in}}{\pgfqpoint{6.200000in}{2.285714in}} %
\pgfusepath{clip}%
\pgfsetbuttcap%
\pgfsetroundjoin%
\definecolor{currentfill}{rgb}{1.000000,0.166667,0.166667}%
\pgfsetfillcolor{currentfill}%
\pgfsetlinewidth{1.003750pt}%
\definecolor{currentstroke}{rgb}{1.000000,0.166667,0.166667}%
\pgfsetstrokecolor{currentstroke}%
\pgfsetdash{}{0pt}%
\pgfpathmoveto{\pgfqpoint{0.571630in}{1.672513in}}%
\pgfpathcurveto{\pgfqpoint{0.579866in}{1.672513in}}{\pgfqpoint{0.587766in}{1.675785in}}{\pgfqpoint{0.593590in}{1.681609in}}%
\pgfpathcurveto{\pgfqpoint{0.599414in}{1.687433in}}{\pgfqpoint{0.602686in}{1.695333in}}{\pgfqpoint{0.602686in}{1.703570in}}%
\pgfpathcurveto{\pgfqpoint{0.602686in}{1.711806in}}{\pgfqpoint{0.599414in}{1.719706in}}{\pgfqpoint{0.593590in}{1.725530in}}%
\pgfpathcurveto{\pgfqpoint{0.587766in}{1.731354in}}{\pgfqpoint{0.579866in}{1.734626in}}{\pgfqpoint{0.571630in}{1.734626in}}%
\pgfpathcurveto{\pgfqpoint{0.563394in}{1.734626in}}{\pgfqpoint{0.555494in}{1.731354in}}{\pgfqpoint{0.549670in}{1.725530in}}%
\pgfpathcurveto{\pgfqpoint{0.543846in}{1.719706in}}{\pgfqpoint{0.540574in}{1.711806in}}{\pgfqpoint{0.540574in}{1.703570in}}%
\pgfpathcurveto{\pgfqpoint{0.540574in}{1.695333in}}{\pgfqpoint{0.543846in}{1.687433in}}{\pgfqpoint{0.549670in}{1.681609in}}%
\pgfpathcurveto{\pgfqpoint{0.555494in}{1.675785in}}{\pgfqpoint{0.563394in}{1.672513in}}{\pgfqpoint{0.571630in}{1.672513in}}%
\pgfpathclose%
\pgfusepath{stroke,fill}%
\end{pgfscope}%
\begin{pgfscope}%
\pgfpathrectangle{\pgfqpoint{0.457963in}{0.528059in}}{\pgfqpoint{6.200000in}{2.285714in}} %
\pgfusepath{clip}%
\pgfsetbuttcap%
\pgfsetroundjoin%
\definecolor{currentfill}{rgb}{1.000000,0.166667,0.166667}%
\pgfsetfillcolor{currentfill}%
\pgfsetlinewidth{1.003750pt}%
\definecolor{currentstroke}{rgb}{1.000000,0.166667,0.166667}%
\pgfsetstrokecolor{currentstroke}%
\pgfsetdash{}{0pt}%
\pgfpathmoveto{\pgfqpoint{0.571630in}{2.116595in}}%
\pgfpathcurveto{\pgfqpoint{0.579866in}{2.116595in}}{\pgfqpoint{0.587766in}{2.119867in}}{\pgfqpoint{0.593590in}{2.125691in}}%
\pgfpathcurveto{\pgfqpoint{0.599414in}{2.131515in}}{\pgfqpoint{0.602686in}{2.139415in}}{\pgfqpoint{0.602686in}{2.147651in}}%
\pgfpathcurveto{\pgfqpoint{0.602686in}{2.155888in}}{\pgfqpoint{0.599414in}{2.163788in}}{\pgfqpoint{0.593590in}{2.169612in}}%
\pgfpathcurveto{\pgfqpoint{0.587766in}{2.175435in}}{\pgfqpoint{0.579866in}{2.178708in}}{\pgfqpoint{0.571630in}{2.178708in}}%
\pgfpathcurveto{\pgfqpoint{0.563394in}{2.178708in}}{\pgfqpoint{0.555494in}{2.175435in}}{\pgfqpoint{0.549670in}{2.169612in}}%
\pgfpathcurveto{\pgfqpoint{0.543846in}{2.163788in}}{\pgfqpoint{0.540574in}{2.155888in}}{\pgfqpoint{0.540574in}{2.147651in}}%
\pgfpathcurveto{\pgfqpoint{0.540574in}{2.139415in}}{\pgfqpoint{0.543846in}{2.131515in}}{\pgfqpoint{0.549670in}{2.125691in}}%
\pgfpathcurveto{\pgfqpoint{0.555494in}{2.119867in}}{\pgfqpoint{0.563394in}{2.116595in}}{\pgfqpoint{0.571630in}{2.116595in}}%
\pgfpathclose%
\pgfusepath{stroke,fill}%
\end{pgfscope}%
\begin{pgfscope}%
\pgfpathrectangle{\pgfqpoint{0.457963in}{0.528059in}}{\pgfqpoint{6.200000in}{2.285714in}} %
\pgfusepath{clip}%
\pgfsetbuttcap%
\pgfsetroundjoin%
\definecolor{currentfill}{rgb}{1.000000,0.166667,0.166667}%
\pgfsetfillcolor{currentfill}%
\pgfsetlinewidth{1.003750pt}%
\definecolor{currentstroke}{rgb}{1.000000,0.166667,0.166667}%
\pgfsetstrokecolor{currentstroke}%
\pgfsetdash{}{0pt}%
\pgfpathmoveto{\pgfqpoint{0.674963in}{2.129656in}}%
\pgfpathcurveto{\pgfqpoint{0.683200in}{2.129656in}}{\pgfqpoint{0.691100in}{2.132928in}}{\pgfqpoint{0.696924in}{2.138752in}}%
\pgfpathcurveto{\pgfqpoint{0.702748in}{2.144576in}}{\pgfqpoint{0.706020in}{2.152476in}}{\pgfqpoint{0.706020in}{2.160713in}}%
\pgfpathcurveto{\pgfqpoint{0.706020in}{2.168949in}}{\pgfqpoint{0.702748in}{2.176849in}}{\pgfqpoint{0.696924in}{2.182673in}}%
\pgfpathcurveto{\pgfqpoint{0.691100in}{2.188497in}}{\pgfqpoint{0.683200in}{2.191769in}}{\pgfqpoint{0.674963in}{2.191769in}}%
\pgfpathcurveto{\pgfqpoint{0.666727in}{2.191769in}}{\pgfqpoint{0.658827in}{2.188497in}}{\pgfqpoint{0.653003in}{2.182673in}}%
\pgfpathcurveto{\pgfqpoint{0.647179in}{2.176849in}}{\pgfqpoint{0.643907in}{2.168949in}}{\pgfqpoint{0.643907in}{2.160713in}}%
\pgfpathcurveto{\pgfqpoint{0.643907in}{2.152476in}}{\pgfqpoint{0.647179in}{2.144576in}}{\pgfqpoint{0.653003in}{2.138752in}}%
\pgfpathcurveto{\pgfqpoint{0.658827in}{2.132928in}}{\pgfqpoint{0.666727in}{2.129656in}}{\pgfqpoint{0.674963in}{2.129656in}}%
\pgfpathclose%
\pgfusepath{stroke,fill}%
\end{pgfscope}%
\begin{pgfscope}%
\pgfpathrectangle{\pgfqpoint{0.457963in}{0.528059in}}{\pgfqpoint{6.200000in}{2.285714in}} %
\pgfusepath{clip}%
\pgfsetbuttcap%
\pgfsetroundjoin%
\definecolor{currentfill}{rgb}{1.000000,0.166667,0.166667}%
\pgfsetfillcolor{currentfill}%
\pgfsetlinewidth{1.003750pt}%
\definecolor{currentstroke}{rgb}{1.000000,0.166667,0.166667}%
\pgfsetstrokecolor{currentstroke}%
\pgfsetdash{}{0pt}%
\pgfpathmoveto{\pgfqpoint{0.860963in}{1.607207in}}%
\pgfpathcurveto{\pgfqpoint{0.869200in}{1.607207in}}{\pgfqpoint{0.877100in}{1.610479in}}{\pgfqpoint{0.882924in}{1.616303in}}%
\pgfpathcurveto{\pgfqpoint{0.888748in}{1.622127in}}{\pgfqpoint{0.892020in}{1.630027in}}{\pgfqpoint{0.892020in}{1.638264in}}%
\pgfpathcurveto{\pgfqpoint{0.892020in}{1.646500in}}{\pgfqpoint{0.888748in}{1.654400in}}{\pgfqpoint{0.882924in}{1.660224in}}%
\pgfpathcurveto{\pgfqpoint{0.877100in}{1.666048in}}{\pgfqpoint{0.869200in}{1.669320in}}{\pgfqpoint{0.860963in}{1.669320in}}%
\pgfpathcurveto{\pgfqpoint{0.852727in}{1.669320in}}{\pgfqpoint{0.844827in}{1.666048in}}{\pgfqpoint{0.839003in}{1.660224in}}%
\pgfpathcurveto{\pgfqpoint{0.833179in}{1.654400in}}{\pgfqpoint{0.829907in}{1.646500in}}{\pgfqpoint{0.829907in}{1.638264in}}%
\pgfpathcurveto{\pgfqpoint{0.829907in}{1.630027in}}{\pgfqpoint{0.833179in}{1.622127in}}{\pgfqpoint{0.839003in}{1.616303in}}%
\pgfpathcurveto{\pgfqpoint{0.844827in}{1.610479in}}{\pgfqpoint{0.852727in}{1.607207in}}{\pgfqpoint{0.860963in}{1.607207in}}%
\pgfpathclose%
\pgfusepath{stroke,fill}%
\end{pgfscope}%
\begin{pgfscope}%
\pgfpathrectangle{\pgfqpoint{0.457963in}{0.528059in}}{\pgfqpoint{6.200000in}{2.285714in}} %
\pgfusepath{clip}%
\pgfsetbuttcap%
\pgfsetroundjoin%
\definecolor{currentfill}{rgb}{1.000000,0.166667,0.166667}%
\pgfsetfillcolor{currentfill}%
\pgfsetlinewidth{1.003750pt}%
\definecolor{currentstroke}{rgb}{1.000000,0.166667,0.166667}%
\pgfsetstrokecolor{currentstroke}%
\pgfsetdash{}{0pt}%
\pgfpathmoveto{\pgfqpoint{1.222630in}{2.090472in}}%
\pgfpathcurveto{\pgfqpoint{1.230866in}{2.090472in}}{\pgfqpoint{1.238766in}{2.093745in}}{\pgfqpoint{1.244590in}{2.099569in}}%
\pgfpathcurveto{\pgfqpoint{1.250414in}{2.105393in}}{\pgfqpoint{1.253686in}{2.113293in}}{\pgfqpoint{1.253686in}{2.121529in}}%
\pgfpathcurveto{\pgfqpoint{1.253686in}{2.129765in}}{\pgfqpoint{1.250414in}{2.137665in}}{\pgfqpoint{1.244590in}{2.143489in}}%
\pgfpathcurveto{\pgfqpoint{1.238766in}{2.149313in}}{\pgfqpoint{1.230866in}{2.152585in}}{\pgfqpoint{1.222630in}{2.152585in}}%
\pgfpathcurveto{\pgfqpoint{1.214394in}{2.152585in}}{\pgfqpoint{1.206494in}{2.149313in}}{\pgfqpoint{1.200670in}{2.143489in}}%
\pgfpathcurveto{\pgfqpoint{1.194846in}{2.137665in}}{\pgfqpoint{1.191574in}{2.129765in}}{\pgfqpoint{1.191574in}{2.121529in}}%
\pgfpathcurveto{\pgfqpoint{1.191574in}{2.113293in}}{\pgfqpoint{1.194846in}{2.105393in}}{\pgfqpoint{1.200670in}{2.099569in}}%
\pgfpathcurveto{\pgfqpoint{1.206494in}{2.093745in}}{\pgfqpoint{1.214394in}{2.090472in}}{\pgfqpoint{1.222630in}{2.090472in}}%
\pgfpathclose%
\pgfusepath{stroke,fill}%
\end{pgfscope}%
\begin{pgfscope}%
\pgfpathrectangle{\pgfqpoint{0.457963in}{0.528059in}}{\pgfqpoint{6.200000in}{2.285714in}} %
\pgfusepath{clip}%
\pgfsetbuttcap%
\pgfsetroundjoin%
\definecolor{currentfill}{rgb}{1.000000,0.166667,0.166667}%
\pgfsetfillcolor{currentfill}%
\pgfsetlinewidth{1.003750pt}%
\definecolor{currentstroke}{rgb}{1.000000,0.166667,0.166667}%
\pgfsetstrokecolor{currentstroke}%
\pgfsetdash{}{0pt}%
\pgfpathmoveto{\pgfqpoint{1.387963in}{2.116595in}}%
\pgfpathcurveto{\pgfqpoint{1.396200in}{2.116595in}}{\pgfqpoint{1.404100in}{2.119867in}}{\pgfqpoint{1.409924in}{2.125691in}}%
\pgfpathcurveto{\pgfqpoint{1.415748in}{2.131515in}}{\pgfqpoint{1.419020in}{2.139415in}}{\pgfqpoint{1.419020in}{2.147651in}}%
\pgfpathcurveto{\pgfqpoint{1.419020in}{2.155888in}}{\pgfqpoint{1.415748in}{2.163788in}}{\pgfqpoint{1.409924in}{2.169612in}}%
\pgfpathcurveto{\pgfqpoint{1.404100in}{2.175435in}}{\pgfqpoint{1.396200in}{2.178708in}}{\pgfqpoint{1.387963in}{2.178708in}}%
\pgfpathcurveto{\pgfqpoint{1.379727in}{2.178708in}}{\pgfqpoint{1.371827in}{2.175435in}}{\pgfqpoint{1.366003in}{2.169612in}}%
\pgfpathcurveto{\pgfqpoint{1.360179in}{2.163788in}}{\pgfqpoint{1.356907in}{2.155888in}}{\pgfqpoint{1.356907in}{2.147651in}}%
\pgfpathcurveto{\pgfqpoint{1.356907in}{2.139415in}}{\pgfqpoint{1.360179in}{2.131515in}}{\pgfqpoint{1.366003in}{2.125691in}}%
\pgfpathcurveto{\pgfqpoint{1.371827in}{2.119867in}}{\pgfqpoint{1.379727in}{2.116595in}}{\pgfqpoint{1.387963in}{2.116595in}}%
\pgfpathclose%
\pgfusepath{stroke,fill}%
\end{pgfscope}%
\begin{pgfscope}%
\pgfpathrectangle{\pgfqpoint{0.457963in}{0.528059in}}{\pgfqpoint{6.200000in}{2.285714in}} %
\pgfusepath{clip}%
\pgfsetbuttcap%
\pgfsetroundjoin%
\definecolor{currentfill}{rgb}{1.000000,0.166667,0.166667}%
\pgfsetfillcolor{currentfill}%
\pgfsetlinewidth{1.003750pt}%
\definecolor{currentstroke}{rgb}{1.000000,0.166667,0.166667}%
\pgfsetstrokecolor{currentstroke}%
\pgfsetdash{}{0pt}%
\pgfpathmoveto{\pgfqpoint{1.460297in}{1.084758in}}%
\pgfpathcurveto{\pgfqpoint{1.468533in}{1.084758in}}{\pgfqpoint{1.476433in}{1.088030in}}{\pgfqpoint{1.482257in}{1.093854in}}%
\pgfpathcurveto{\pgfqpoint{1.488081in}{1.099678in}}{\pgfqpoint{1.491353in}{1.107578in}}{\pgfqpoint{1.491353in}{1.115815in}}%
\pgfpathcurveto{\pgfqpoint{1.491353in}{1.124051in}}{\pgfqpoint{1.488081in}{1.131951in}}{\pgfqpoint{1.482257in}{1.137775in}}%
\pgfpathcurveto{\pgfqpoint{1.476433in}{1.143599in}}{\pgfqpoint{1.468533in}{1.146871in}}{\pgfqpoint{1.460297in}{1.146871in}}%
\pgfpathcurveto{\pgfqpoint{1.452060in}{1.146871in}}{\pgfqpoint{1.444160in}{1.143599in}}{\pgfqpoint{1.438336in}{1.137775in}}%
\pgfpathcurveto{\pgfqpoint{1.432512in}{1.131951in}}{\pgfqpoint{1.429240in}{1.124051in}}{\pgfqpoint{1.429240in}{1.115815in}}%
\pgfpathcurveto{\pgfqpoint{1.429240in}{1.107578in}}{\pgfqpoint{1.432512in}{1.099678in}}{\pgfqpoint{1.438336in}{1.093854in}}%
\pgfpathcurveto{\pgfqpoint{1.444160in}{1.088030in}}{\pgfqpoint{1.452060in}{1.084758in}}{\pgfqpoint{1.460297in}{1.084758in}}%
\pgfpathclose%
\pgfusepath{stroke,fill}%
\end{pgfscope}%
\begin{pgfscope}%
\pgfpathrectangle{\pgfqpoint{0.457963in}{0.528059in}}{\pgfqpoint{6.200000in}{2.285714in}} %
\pgfusepath{clip}%
\pgfsetbuttcap%
\pgfsetroundjoin%
\definecolor{currentfill}{rgb}{1.000000,0.166667,0.166667}%
\pgfsetfillcolor{currentfill}%
\pgfsetlinewidth{1.003750pt}%
\definecolor{currentstroke}{rgb}{1.000000,0.166667,0.166667}%
\pgfsetstrokecolor{currentstroke}%
\pgfsetdash{}{0pt}%
\pgfpathmoveto{\pgfqpoint{1.790963in}{1.737819in}}%
\pgfpathcurveto{\pgfqpoint{1.799200in}{1.737819in}}{\pgfqpoint{1.807100in}{1.741092in}}{\pgfqpoint{1.812924in}{1.746916in}}%
\pgfpathcurveto{\pgfqpoint{1.818748in}{1.752739in}}{\pgfqpoint{1.822020in}{1.760639in}}{\pgfqpoint{1.822020in}{1.768876in}}%
\pgfpathcurveto{\pgfqpoint{1.822020in}{1.777112in}}{\pgfqpoint{1.818748in}{1.785012in}}{\pgfqpoint{1.812924in}{1.790836in}}%
\pgfpathcurveto{\pgfqpoint{1.807100in}{1.796660in}}{\pgfqpoint{1.799200in}{1.799932in}}{\pgfqpoint{1.790963in}{1.799932in}}%
\pgfpathcurveto{\pgfqpoint{1.782727in}{1.799932in}}{\pgfqpoint{1.774827in}{1.796660in}}{\pgfqpoint{1.769003in}{1.790836in}}%
\pgfpathcurveto{\pgfqpoint{1.763179in}{1.785012in}}{\pgfqpoint{1.759907in}{1.777112in}}{\pgfqpoint{1.759907in}{1.768876in}}%
\pgfpathcurveto{\pgfqpoint{1.759907in}{1.760639in}}{\pgfqpoint{1.763179in}{1.752739in}}{\pgfqpoint{1.769003in}{1.746916in}}%
\pgfpathcurveto{\pgfqpoint{1.774827in}{1.741092in}}{\pgfqpoint{1.782727in}{1.737819in}}{\pgfqpoint{1.790963in}{1.737819in}}%
\pgfpathclose%
\pgfusepath{stroke,fill}%
\end{pgfscope}%
\begin{pgfscope}%
\pgfpathrectangle{\pgfqpoint{0.457963in}{0.528059in}}{\pgfqpoint{6.200000in}{2.285714in}} %
\pgfusepath{clip}%
\pgfsetbuttcap%
\pgfsetroundjoin%
\definecolor{currentfill}{rgb}{1.000000,0.166667,0.166667}%
\pgfsetfillcolor{currentfill}%
\pgfsetlinewidth{1.003750pt}%
\definecolor{currentstroke}{rgb}{1.000000,0.166667,0.166667}%
\pgfsetstrokecolor{currentstroke}%
\pgfsetdash{}{0pt}%
\pgfpathmoveto{\pgfqpoint{2.028630in}{1.959860in}}%
\pgfpathcurveto{\pgfqpoint{2.036866in}{1.959860in}}{\pgfqpoint{2.044766in}{1.963132in}}{\pgfqpoint{2.050590in}{1.968956in}}%
\pgfpathcurveto{\pgfqpoint{2.056414in}{1.974780in}}{\pgfqpoint{2.059686in}{1.982680in}}{\pgfqpoint{2.059686in}{1.990917in}}%
\pgfpathcurveto{\pgfqpoint{2.059686in}{1.999153in}}{\pgfqpoint{2.056414in}{2.007053in}}{\pgfqpoint{2.050590in}{2.012877in}}%
\pgfpathcurveto{\pgfqpoint{2.044766in}{2.018701in}}{\pgfqpoint{2.036866in}{2.021973in}}{\pgfqpoint{2.028630in}{2.021973in}}%
\pgfpathcurveto{\pgfqpoint{2.020394in}{2.021973in}}{\pgfqpoint{2.012494in}{2.018701in}}{\pgfqpoint{2.006670in}{2.012877in}}%
\pgfpathcurveto{\pgfqpoint{2.000846in}{2.007053in}}{\pgfqpoint{1.997574in}{1.999153in}}{\pgfqpoint{1.997574in}{1.990917in}}%
\pgfpathcurveto{\pgfqpoint{1.997574in}{1.982680in}}{\pgfqpoint{2.000846in}{1.974780in}}{\pgfqpoint{2.006670in}{1.968956in}}%
\pgfpathcurveto{\pgfqpoint{2.012494in}{1.963132in}}{\pgfqpoint{2.020394in}{1.959860in}}{\pgfqpoint{2.028630in}{1.959860in}}%
\pgfpathclose%
\pgfusepath{stroke,fill}%
\end{pgfscope}%
\begin{pgfscope}%
\pgfpathrectangle{\pgfqpoint{0.457963in}{0.528059in}}{\pgfqpoint{6.200000in}{2.285714in}} %
\pgfusepath{clip}%
\pgfsetbuttcap%
\pgfsetroundjoin%
\definecolor{currentfill}{rgb}{1.000000,0.166667,0.166667}%
\pgfsetfillcolor{currentfill}%
\pgfsetlinewidth{1.003750pt}%
\definecolor{currentstroke}{rgb}{1.000000,0.166667,0.166667}%
\pgfsetstrokecolor{currentstroke}%
\pgfsetdash{}{0pt}%
\pgfpathmoveto{\pgfqpoint{2.090630in}{2.116595in}}%
\pgfpathcurveto{\pgfqpoint{2.098866in}{2.116595in}}{\pgfqpoint{2.106766in}{2.119867in}}{\pgfqpoint{2.112590in}{2.125691in}}%
\pgfpathcurveto{\pgfqpoint{2.118414in}{2.131515in}}{\pgfqpoint{2.121686in}{2.139415in}}{\pgfqpoint{2.121686in}{2.147651in}}%
\pgfpathcurveto{\pgfqpoint{2.121686in}{2.155888in}}{\pgfqpoint{2.118414in}{2.163788in}}{\pgfqpoint{2.112590in}{2.169612in}}%
\pgfpathcurveto{\pgfqpoint{2.106766in}{2.175435in}}{\pgfqpoint{2.098866in}{2.178708in}}{\pgfqpoint{2.090630in}{2.178708in}}%
\pgfpathcurveto{\pgfqpoint{2.082394in}{2.178708in}}{\pgfqpoint{2.074494in}{2.175435in}}{\pgfqpoint{2.068670in}{2.169612in}}%
\pgfpathcurveto{\pgfqpoint{2.062846in}{2.163788in}}{\pgfqpoint{2.059574in}{2.155888in}}{\pgfqpoint{2.059574in}{2.147651in}}%
\pgfpathcurveto{\pgfqpoint{2.059574in}{2.139415in}}{\pgfqpoint{2.062846in}{2.131515in}}{\pgfqpoint{2.068670in}{2.125691in}}%
\pgfpathcurveto{\pgfqpoint{2.074494in}{2.119867in}}{\pgfqpoint{2.082394in}{2.116595in}}{\pgfqpoint{2.090630in}{2.116595in}}%
\pgfpathclose%
\pgfusepath{stroke,fill}%
\end{pgfscope}%
\begin{pgfscope}%
\pgfpathrectangle{\pgfqpoint{0.457963in}{0.528059in}}{\pgfqpoint{6.200000in}{2.285714in}} %
\pgfusepath{clip}%
\pgfsetbuttcap%
\pgfsetroundjoin%
\definecolor{currentfill}{rgb}{1.000000,0.166667,0.166667}%
\pgfsetfillcolor{currentfill}%
\pgfsetlinewidth{1.003750pt}%
\definecolor{currentstroke}{rgb}{1.000000,0.166667,0.166667}%
\pgfsetstrokecolor{currentstroke}%
\pgfsetdash{}{0pt}%
\pgfpathmoveto{\pgfqpoint{2.400630in}{1.345983in}}%
\pgfpathcurveto{\pgfqpoint{2.408866in}{1.345983in}}{\pgfqpoint{2.416766in}{1.349255in}}{\pgfqpoint{2.422590in}{1.355079in}}%
\pgfpathcurveto{\pgfqpoint{2.428414in}{1.360903in}}{\pgfqpoint{2.431686in}{1.368803in}}{\pgfqpoint{2.431686in}{1.377039in}}%
\pgfpathcurveto{\pgfqpoint{2.431686in}{1.385275in}}{\pgfqpoint{2.428414in}{1.393175in}}{\pgfqpoint{2.422590in}{1.398999in}}%
\pgfpathcurveto{\pgfqpoint{2.416766in}{1.404823in}}{\pgfqpoint{2.408866in}{1.408096in}}{\pgfqpoint{2.400630in}{1.408096in}}%
\pgfpathcurveto{\pgfqpoint{2.392394in}{1.408096in}}{\pgfqpoint{2.384494in}{1.404823in}}{\pgfqpoint{2.378670in}{1.398999in}}%
\pgfpathcurveto{\pgfqpoint{2.372846in}{1.393175in}}{\pgfqpoint{2.369574in}{1.385275in}}{\pgfqpoint{2.369574in}{1.377039in}}%
\pgfpathcurveto{\pgfqpoint{2.369574in}{1.368803in}}{\pgfqpoint{2.372846in}{1.360903in}}{\pgfqpoint{2.378670in}{1.355079in}}%
\pgfpathcurveto{\pgfqpoint{2.384494in}{1.349255in}}{\pgfqpoint{2.392394in}{1.345983in}}{\pgfqpoint{2.400630in}{1.345983in}}%
\pgfpathclose%
\pgfusepath{stroke,fill}%
\end{pgfscope}%
\begin{pgfscope}%
\pgfpathrectangle{\pgfqpoint{0.457963in}{0.528059in}}{\pgfqpoint{6.200000in}{2.285714in}} %
\pgfusepath{clip}%
\pgfsetbuttcap%
\pgfsetroundjoin%
\definecolor{currentfill}{rgb}{1.000000,0.166667,0.166667}%
\pgfsetfillcolor{currentfill}%
\pgfsetlinewidth{1.003750pt}%
\definecolor{currentstroke}{rgb}{1.000000,0.166667,0.166667}%
\pgfsetstrokecolor{currentstroke}%
\pgfsetdash{}{0pt}%
\pgfpathmoveto{\pgfqpoint{3.361630in}{1.071697in}}%
\pgfpathcurveto{\pgfqpoint{3.369866in}{1.071697in}}{\pgfqpoint{3.377766in}{1.074969in}}{\pgfqpoint{3.383590in}{1.080793in}}%
\pgfpathcurveto{\pgfqpoint{3.389414in}{1.086617in}}{\pgfqpoint{3.392686in}{1.094517in}}{\pgfqpoint{3.392686in}{1.102753in}}%
\pgfpathcurveto{\pgfqpoint{3.392686in}{1.110990in}}{\pgfqpoint{3.389414in}{1.118890in}}{\pgfqpoint{3.383590in}{1.124714in}}%
\pgfpathcurveto{\pgfqpoint{3.377766in}{1.130538in}}{\pgfqpoint{3.369866in}{1.133810in}}{\pgfqpoint{3.361630in}{1.133810in}}%
\pgfpathcurveto{\pgfqpoint{3.353394in}{1.133810in}}{\pgfqpoint{3.345494in}{1.130538in}}{\pgfqpoint{3.339670in}{1.124714in}}%
\pgfpathcurveto{\pgfqpoint{3.333846in}{1.118890in}}{\pgfqpoint{3.330574in}{1.110990in}}{\pgfqpoint{3.330574in}{1.102753in}}%
\pgfpathcurveto{\pgfqpoint{3.330574in}{1.094517in}}{\pgfqpoint{3.333846in}{1.086617in}}{\pgfqpoint{3.339670in}{1.080793in}}%
\pgfpathcurveto{\pgfqpoint{3.345494in}{1.074969in}}{\pgfqpoint{3.353394in}{1.071697in}}{\pgfqpoint{3.361630in}{1.071697in}}%
\pgfpathclose%
\pgfusepath{stroke,fill}%
\end{pgfscope}%
\begin{pgfscope}%
\pgfpathrectangle{\pgfqpoint{0.457963in}{0.528059in}}{\pgfqpoint{6.200000in}{2.285714in}} %
\pgfusepath{clip}%
\pgfsetbuttcap%
\pgfsetroundjoin%
\definecolor{currentfill}{rgb}{1.000000,0.000000,0.000000}%
\pgfsetfillcolor{currentfill}%
\pgfsetlinewidth{1.003750pt}%
\definecolor{currentstroke}{rgb}{1.000000,0.000000,0.000000}%
\pgfsetstrokecolor{currentstroke}%
\pgfsetdash{}{0pt}%
\pgfpathmoveto{\pgfqpoint{0.457963in}{2.456187in}}%
\pgfpathcurveto{\pgfqpoint{0.466200in}{2.456187in}}{\pgfqpoint{0.474100in}{2.459459in}}{\pgfqpoint{0.479924in}{2.465283in}}%
\pgfpathcurveto{\pgfqpoint{0.485748in}{2.471107in}}{\pgfqpoint{0.489020in}{2.479007in}}{\pgfqpoint{0.489020in}{2.487243in}}%
\pgfpathcurveto{\pgfqpoint{0.489020in}{2.495479in}}{\pgfqpoint{0.485748in}{2.503379in}}{\pgfqpoint{0.479924in}{2.509203in}}%
\pgfpathcurveto{\pgfqpoint{0.474100in}{2.515027in}}{\pgfqpoint{0.466200in}{2.518300in}}{\pgfqpoint{0.457963in}{2.518300in}}%
\pgfpathcurveto{\pgfqpoint{0.449727in}{2.518300in}}{\pgfqpoint{0.441827in}{2.515027in}}{\pgfqpoint{0.436003in}{2.509203in}}%
\pgfpathcurveto{\pgfqpoint{0.430179in}{2.503379in}}{\pgfqpoint{0.426907in}{2.495479in}}{\pgfqpoint{0.426907in}{2.487243in}}%
\pgfpathcurveto{\pgfqpoint{0.426907in}{2.479007in}}{\pgfqpoint{0.430179in}{2.471107in}}{\pgfqpoint{0.436003in}{2.465283in}}%
\pgfpathcurveto{\pgfqpoint{0.441827in}{2.459459in}}{\pgfqpoint{0.449727in}{2.456187in}}{\pgfqpoint{0.457963in}{2.456187in}}%
\pgfpathclose%
\pgfusepath{stroke,fill}%
\end{pgfscope}%
\begin{pgfscope}%
\pgfpathrectangle{\pgfqpoint{0.457963in}{0.528059in}}{\pgfqpoint{6.200000in}{2.285714in}} %
\pgfusepath{clip}%
\pgfsetbuttcap%
\pgfsetroundjoin%
\definecolor{currentfill}{rgb}{1.000000,0.000000,0.000000}%
\pgfsetfillcolor{currentfill}%
\pgfsetlinewidth{1.003750pt}%
\definecolor{currentstroke}{rgb}{1.000000,0.000000,0.000000}%
\pgfsetstrokecolor{currentstroke}%
\pgfsetdash{}{0pt}%
\pgfpathmoveto{\pgfqpoint{0.457963in}{2.456187in}}%
\pgfpathcurveto{\pgfqpoint{0.466200in}{2.456187in}}{\pgfqpoint{0.474100in}{2.459459in}}{\pgfqpoint{0.479924in}{2.465283in}}%
\pgfpathcurveto{\pgfqpoint{0.485748in}{2.471107in}}{\pgfqpoint{0.489020in}{2.479007in}}{\pgfqpoint{0.489020in}{2.487243in}}%
\pgfpathcurveto{\pgfqpoint{0.489020in}{2.495479in}}{\pgfqpoint{0.485748in}{2.503379in}}{\pgfqpoint{0.479924in}{2.509203in}}%
\pgfpathcurveto{\pgfqpoint{0.474100in}{2.515027in}}{\pgfqpoint{0.466200in}{2.518300in}}{\pgfqpoint{0.457963in}{2.518300in}}%
\pgfpathcurveto{\pgfqpoint{0.449727in}{2.518300in}}{\pgfqpoint{0.441827in}{2.515027in}}{\pgfqpoint{0.436003in}{2.509203in}}%
\pgfpathcurveto{\pgfqpoint{0.430179in}{2.503379in}}{\pgfqpoint{0.426907in}{2.495479in}}{\pgfqpoint{0.426907in}{2.487243in}}%
\pgfpathcurveto{\pgfqpoint{0.426907in}{2.479007in}}{\pgfqpoint{0.430179in}{2.471107in}}{\pgfqpoint{0.436003in}{2.465283in}}%
\pgfpathcurveto{\pgfqpoint{0.441827in}{2.459459in}}{\pgfqpoint{0.449727in}{2.456187in}}{\pgfqpoint{0.457963in}{2.456187in}}%
\pgfpathclose%
\pgfusepath{stroke,fill}%
\end{pgfscope}%
\begin{pgfscope}%
\pgfpathrectangle{\pgfqpoint{0.457963in}{0.528059in}}{\pgfqpoint{6.200000in}{2.285714in}} %
\pgfusepath{clip}%
\pgfsetbuttcap%
\pgfsetroundjoin%
\definecolor{currentfill}{rgb}{1.000000,0.000000,0.000000}%
\pgfsetfillcolor{currentfill}%
\pgfsetlinewidth{1.003750pt}%
\definecolor{currentstroke}{rgb}{1.000000,0.000000,0.000000}%
\pgfsetstrokecolor{currentstroke}%
\pgfsetdash{}{0pt}%
\pgfpathmoveto{\pgfqpoint{0.457963in}{2.456187in}}%
\pgfpathcurveto{\pgfqpoint{0.466200in}{2.456187in}}{\pgfqpoint{0.474100in}{2.459459in}}{\pgfqpoint{0.479924in}{2.465283in}}%
\pgfpathcurveto{\pgfqpoint{0.485748in}{2.471107in}}{\pgfqpoint{0.489020in}{2.479007in}}{\pgfqpoint{0.489020in}{2.487243in}}%
\pgfpathcurveto{\pgfqpoint{0.489020in}{2.495479in}}{\pgfqpoint{0.485748in}{2.503379in}}{\pgfqpoint{0.479924in}{2.509203in}}%
\pgfpathcurveto{\pgfqpoint{0.474100in}{2.515027in}}{\pgfqpoint{0.466200in}{2.518300in}}{\pgfqpoint{0.457963in}{2.518300in}}%
\pgfpathcurveto{\pgfqpoint{0.449727in}{2.518300in}}{\pgfqpoint{0.441827in}{2.515027in}}{\pgfqpoint{0.436003in}{2.509203in}}%
\pgfpathcurveto{\pgfqpoint{0.430179in}{2.503379in}}{\pgfqpoint{0.426907in}{2.495479in}}{\pgfqpoint{0.426907in}{2.487243in}}%
\pgfpathcurveto{\pgfqpoint{0.426907in}{2.479007in}}{\pgfqpoint{0.430179in}{2.471107in}}{\pgfqpoint{0.436003in}{2.465283in}}%
\pgfpathcurveto{\pgfqpoint{0.441827in}{2.459459in}}{\pgfqpoint{0.449727in}{2.456187in}}{\pgfqpoint{0.457963in}{2.456187in}}%
\pgfpathclose%
\pgfusepath{stroke,fill}%
\end{pgfscope}%
\begin{pgfscope}%
\pgfpathrectangle{\pgfqpoint{0.457963in}{0.528059in}}{\pgfqpoint{6.200000in}{2.285714in}} %
\pgfusepath{clip}%
\pgfsetbuttcap%
\pgfsetroundjoin%
\definecolor{currentfill}{rgb}{1.000000,0.000000,0.000000}%
\pgfsetfillcolor{currentfill}%
\pgfsetlinewidth{1.003750pt}%
\definecolor{currentstroke}{rgb}{1.000000,0.000000,0.000000}%
\pgfsetstrokecolor{currentstroke}%
\pgfsetdash{}{0pt}%
\pgfpathmoveto{\pgfqpoint{0.457963in}{2.456187in}}%
\pgfpathcurveto{\pgfqpoint{0.466200in}{2.456187in}}{\pgfqpoint{0.474100in}{2.459459in}}{\pgfqpoint{0.479924in}{2.465283in}}%
\pgfpathcurveto{\pgfqpoint{0.485748in}{2.471107in}}{\pgfqpoint{0.489020in}{2.479007in}}{\pgfqpoint{0.489020in}{2.487243in}}%
\pgfpathcurveto{\pgfqpoint{0.489020in}{2.495479in}}{\pgfqpoint{0.485748in}{2.503379in}}{\pgfqpoint{0.479924in}{2.509203in}}%
\pgfpathcurveto{\pgfqpoint{0.474100in}{2.515027in}}{\pgfqpoint{0.466200in}{2.518300in}}{\pgfqpoint{0.457963in}{2.518300in}}%
\pgfpathcurveto{\pgfqpoint{0.449727in}{2.518300in}}{\pgfqpoint{0.441827in}{2.515027in}}{\pgfqpoint{0.436003in}{2.509203in}}%
\pgfpathcurveto{\pgfqpoint{0.430179in}{2.503379in}}{\pgfqpoint{0.426907in}{2.495479in}}{\pgfqpoint{0.426907in}{2.487243in}}%
\pgfpathcurveto{\pgfqpoint{0.426907in}{2.479007in}}{\pgfqpoint{0.430179in}{2.471107in}}{\pgfqpoint{0.436003in}{2.465283in}}%
\pgfpathcurveto{\pgfqpoint{0.441827in}{2.459459in}}{\pgfqpoint{0.449727in}{2.456187in}}{\pgfqpoint{0.457963in}{2.456187in}}%
\pgfpathclose%
\pgfusepath{stroke,fill}%
\end{pgfscope}%
\begin{pgfscope}%
\pgfpathrectangle{\pgfqpoint{0.457963in}{0.528059in}}{\pgfqpoint{6.200000in}{2.285714in}} %
\pgfusepath{clip}%
\pgfsetbuttcap%
\pgfsetroundjoin%
\definecolor{currentfill}{rgb}{1.000000,0.000000,0.000000}%
\pgfsetfillcolor{currentfill}%
\pgfsetlinewidth{1.003750pt}%
\definecolor{currentstroke}{rgb}{1.000000,0.000000,0.000000}%
\pgfsetstrokecolor{currentstroke}%
\pgfsetdash{}{0pt}%
\pgfpathmoveto{\pgfqpoint{0.468297in}{2.377819in}}%
\pgfpathcurveto{\pgfqpoint{0.476533in}{2.377819in}}{\pgfqpoint{0.484433in}{2.381092in}}{\pgfqpoint{0.490257in}{2.386916in}}%
\pgfpathcurveto{\pgfqpoint{0.496081in}{2.392739in}}{\pgfqpoint{0.499353in}{2.400639in}}{\pgfqpoint{0.499353in}{2.408876in}}%
\pgfpathcurveto{\pgfqpoint{0.499353in}{2.417112in}}{\pgfqpoint{0.496081in}{2.425012in}}{\pgfqpoint{0.490257in}{2.430836in}}%
\pgfpathcurveto{\pgfqpoint{0.484433in}{2.436660in}}{\pgfqpoint{0.476533in}{2.439932in}}{\pgfqpoint{0.468297in}{2.439932in}}%
\pgfpathcurveto{\pgfqpoint{0.460060in}{2.439932in}}{\pgfqpoint{0.452160in}{2.436660in}}{\pgfqpoint{0.446336in}{2.430836in}}%
\pgfpathcurveto{\pgfqpoint{0.440512in}{2.425012in}}{\pgfqpoint{0.437240in}{2.417112in}}{\pgfqpoint{0.437240in}{2.408876in}}%
\pgfpathcurveto{\pgfqpoint{0.437240in}{2.400639in}}{\pgfqpoint{0.440512in}{2.392739in}}{\pgfqpoint{0.446336in}{2.386916in}}%
\pgfpathcurveto{\pgfqpoint{0.452160in}{2.381092in}}{\pgfqpoint{0.460060in}{2.377819in}}{\pgfqpoint{0.468297in}{2.377819in}}%
\pgfpathclose%
\pgfusepath{stroke,fill}%
\end{pgfscope}%
\begin{pgfscope}%
\pgfpathrectangle{\pgfqpoint{0.457963in}{0.528059in}}{\pgfqpoint{6.200000in}{2.285714in}} %
\pgfusepath{clip}%
\pgfsetbuttcap%
\pgfsetroundjoin%
\definecolor{currentfill}{rgb}{1.000000,0.000000,0.000000}%
\pgfsetfillcolor{currentfill}%
\pgfsetlinewidth{1.003750pt}%
\definecolor{currentstroke}{rgb}{1.000000,0.000000,0.000000}%
\pgfsetstrokecolor{currentstroke}%
\pgfsetdash{}{0pt}%
\pgfpathmoveto{\pgfqpoint{0.509630in}{2.221085in}}%
\pgfpathcurveto{\pgfqpoint{0.517866in}{2.221085in}}{\pgfqpoint{0.525766in}{2.224357in}}{\pgfqpoint{0.531590in}{2.230181in}}%
\pgfpathcurveto{\pgfqpoint{0.537414in}{2.236005in}}{\pgfqpoint{0.540686in}{2.243905in}}{\pgfqpoint{0.540686in}{2.252141in}}%
\pgfpathcurveto{\pgfqpoint{0.540686in}{2.260377in}}{\pgfqpoint{0.537414in}{2.268277in}}{\pgfqpoint{0.531590in}{2.274101in}}%
\pgfpathcurveto{\pgfqpoint{0.525766in}{2.279925in}}{\pgfqpoint{0.517866in}{2.283198in}}{\pgfqpoint{0.509630in}{2.283198in}}%
\pgfpathcurveto{\pgfqpoint{0.501394in}{2.283198in}}{\pgfqpoint{0.493494in}{2.279925in}}{\pgfqpoint{0.487670in}{2.274101in}}%
\pgfpathcurveto{\pgfqpoint{0.481846in}{2.268277in}}{\pgfqpoint{0.478574in}{2.260377in}}{\pgfqpoint{0.478574in}{2.252141in}}%
\pgfpathcurveto{\pgfqpoint{0.478574in}{2.243905in}}{\pgfqpoint{0.481846in}{2.236005in}}{\pgfqpoint{0.487670in}{2.230181in}}%
\pgfpathcurveto{\pgfqpoint{0.493494in}{2.224357in}}{\pgfqpoint{0.501394in}{2.221085in}}{\pgfqpoint{0.509630in}{2.221085in}}%
\pgfpathclose%
\pgfusepath{stroke,fill}%
\end{pgfscope}%
\begin{pgfscope}%
\pgfpathrectangle{\pgfqpoint{0.457963in}{0.528059in}}{\pgfqpoint{6.200000in}{2.285714in}} %
\pgfusepath{clip}%
\pgfsetbuttcap%
\pgfsetroundjoin%
\definecolor{currentfill}{rgb}{1.000000,0.000000,0.000000}%
\pgfsetfillcolor{currentfill}%
\pgfsetlinewidth{1.003750pt}%
\definecolor{currentstroke}{rgb}{1.000000,0.000000,0.000000}%
\pgfsetstrokecolor{currentstroke}%
\pgfsetdash{}{0pt}%
\pgfpathmoveto{\pgfqpoint{0.540630in}{2.456187in}}%
\pgfpathcurveto{\pgfqpoint{0.548866in}{2.456187in}}{\pgfqpoint{0.556766in}{2.459459in}}{\pgfqpoint{0.562590in}{2.465283in}}%
\pgfpathcurveto{\pgfqpoint{0.568414in}{2.471107in}}{\pgfqpoint{0.571686in}{2.479007in}}{\pgfqpoint{0.571686in}{2.487243in}}%
\pgfpathcurveto{\pgfqpoint{0.571686in}{2.495479in}}{\pgfqpoint{0.568414in}{2.503379in}}{\pgfqpoint{0.562590in}{2.509203in}}%
\pgfpathcurveto{\pgfqpoint{0.556766in}{2.515027in}}{\pgfqpoint{0.548866in}{2.518300in}}{\pgfqpoint{0.540630in}{2.518300in}}%
\pgfpathcurveto{\pgfqpoint{0.532394in}{2.518300in}}{\pgfqpoint{0.524494in}{2.515027in}}{\pgfqpoint{0.518670in}{2.509203in}}%
\pgfpathcurveto{\pgfqpoint{0.512846in}{2.503379in}}{\pgfqpoint{0.509574in}{2.495479in}}{\pgfqpoint{0.509574in}{2.487243in}}%
\pgfpathcurveto{\pgfqpoint{0.509574in}{2.479007in}}{\pgfqpoint{0.512846in}{2.471107in}}{\pgfqpoint{0.518670in}{2.465283in}}%
\pgfpathcurveto{\pgfqpoint{0.524494in}{2.459459in}}{\pgfqpoint{0.532394in}{2.456187in}}{\pgfqpoint{0.540630in}{2.456187in}}%
\pgfpathclose%
\pgfusepath{stroke,fill}%
\end{pgfscope}%
\begin{pgfscope}%
\pgfpathrectangle{\pgfqpoint{0.457963in}{0.528059in}}{\pgfqpoint{6.200000in}{2.285714in}} %
\pgfusepath{clip}%
\pgfsetbuttcap%
\pgfsetroundjoin%
\definecolor{currentfill}{rgb}{1.000000,0.000000,0.000000}%
\pgfsetfillcolor{currentfill}%
\pgfsetlinewidth{1.003750pt}%
\definecolor{currentstroke}{rgb}{1.000000,0.000000,0.000000}%
\pgfsetstrokecolor{currentstroke}%
\pgfsetdash{}{0pt}%
\pgfpathmoveto{\pgfqpoint{0.747297in}{2.221085in}}%
\pgfpathcurveto{\pgfqpoint{0.755533in}{2.221085in}}{\pgfqpoint{0.763433in}{2.224357in}}{\pgfqpoint{0.769257in}{2.230181in}}%
\pgfpathcurveto{\pgfqpoint{0.775081in}{2.236005in}}{\pgfqpoint{0.778353in}{2.243905in}}{\pgfqpoint{0.778353in}{2.252141in}}%
\pgfpathcurveto{\pgfqpoint{0.778353in}{2.260377in}}{\pgfqpoint{0.775081in}{2.268277in}}{\pgfqpoint{0.769257in}{2.274101in}}%
\pgfpathcurveto{\pgfqpoint{0.763433in}{2.279925in}}{\pgfqpoint{0.755533in}{2.283198in}}{\pgfqpoint{0.747297in}{2.283198in}}%
\pgfpathcurveto{\pgfqpoint{0.739060in}{2.283198in}}{\pgfqpoint{0.731160in}{2.279925in}}{\pgfqpoint{0.725336in}{2.274101in}}%
\pgfpathcurveto{\pgfqpoint{0.719512in}{2.268277in}}{\pgfqpoint{0.716240in}{2.260377in}}{\pgfqpoint{0.716240in}{2.252141in}}%
\pgfpathcurveto{\pgfqpoint{0.716240in}{2.243905in}}{\pgfqpoint{0.719512in}{2.236005in}}{\pgfqpoint{0.725336in}{2.230181in}}%
\pgfpathcurveto{\pgfqpoint{0.731160in}{2.224357in}}{\pgfqpoint{0.739060in}{2.221085in}}{\pgfqpoint{0.747297in}{2.221085in}}%
\pgfpathclose%
\pgfusepath{stroke,fill}%
\end{pgfscope}%
\begin{pgfscope}%
\pgfpathrectangle{\pgfqpoint{0.457963in}{0.528059in}}{\pgfqpoint{6.200000in}{2.285714in}} %
\pgfusepath{clip}%
\pgfsetbuttcap%
\pgfsetroundjoin%
\definecolor{currentfill}{rgb}{1.000000,0.000000,0.000000}%
\pgfsetfillcolor{currentfill}%
\pgfsetlinewidth{1.003750pt}%
\definecolor{currentstroke}{rgb}{1.000000,0.000000,0.000000}%
\pgfsetstrokecolor{currentstroke}%
\pgfsetdash{}{0pt}%
\pgfpathmoveto{\pgfqpoint{0.809297in}{2.456187in}}%
\pgfpathcurveto{\pgfqpoint{0.817533in}{2.456187in}}{\pgfqpoint{0.825433in}{2.459459in}}{\pgfqpoint{0.831257in}{2.465283in}}%
\pgfpathcurveto{\pgfqpoint{0.837081in}{2.471107in}}{\pgfqpoint{0.840353in}{2.479007in}}{\pgfqpoint{0.840353in}{2.487243in}}%
\pgfpathcurveto{\pgfqpoint{0.840353in}{2.495479in}}{\pgfqpoint{0.837081in}{2.503379in}}{\pgfqpoint{0.831257in}{2.509203in}}%
\pgfpathcurveto{\pgfqpoint{0.825433in}{2.515027in}}{\pgfqpoint{0.817533in}{2.518300in}}{\pgfqpoint{0.809297in}{2.518300in}}%
\pgfpathcurveto{\pgfqpoint{0.801060in}{2.518300in}}{\pgfqpoint{0.793160in}{2.515027in}}{\pgfqpoint{0.787336in}{2.509203in}}%
\pgfpathcurveto{\pgfqpoint{0.781512in}{2.503379in}}{\pgfqpoint{0.778240in}{2.495479in}}{\pgfqpoint{0.778240in}{2.487243in}}%
\pgfpathcurveto{\pgfqpoint{0.778240in}{2.479007in}}{\pgfqpoint{0.781512in}{2.471107in}}{\pgfqpoint{0.787336in}{2.465283in}}%
\pgfpathcurveto{\pgfqpoint{0.793160in}{2.459459in}}{\pgfqpoint{0.801060in}{2.456187in}}{\pgfqpoint{0.809297in}{2.456187in}}%
\pgfpathclose%
\pgfusepath{stroke,fill}%
\end{pgfscope}%
\begin{pgfscope}%
\pgfpathrectangle{\pgfqpoint{0.457963in}{0.528059in}}{\pgfqpoint{6.200000in}{2.285714in}} %
\pgfusepath{clip}%
\pgfsetbuttcap%
\pgfsetroundjoin%
\definecolor{currentfill}{rgb}{1.000000,0.000000,0.000000}%
\pgfsetfillcolor{currentfill}%
\pgfsetlinewidth{1.003750pt}%
\definecolor{currentstroke}{rgb}{1.000000,0.000000,0.000000}%
\pgfsetstrokecolor{currentstroke}%
\pgfsetdash{}{0pt}%
\pgfpathmoveto{\pgfqpoint{0.891963in}{1.829248in}}%
\pgfpathcurveto{\pgfqpoint{0.900200in}{1.829248in}}{\pgfqpoint{0.908100in}{1.832520in}}{\pgfqpoint{0.913924in}{1.838344in}}%
\pgfpathcurveto{\pgfqpoint{0.919748in}{1.844168in}}{\pgfqpoint{0.923020in}{1.852068in}}{\pgfqpoint{0.923020in}{1.860304in}}%
\pgfpathcurveto{\pgfqpoint{0.923020in}{1.868541in}}{\pgfqpoint{0.919748in}{1.876441in}}{\pgfqpoint{0.913924in}{1.882265in}}%
\pgfpathcurveto{\pgfqpoint{0.908100in}{1.888089in}}{\pgfqpoint{0.900200in}{1.891361in}}{\pgfqpoint{0.891963in}{1.891361in}}%
\pgfpathcurveto{\pgfqpoint{0.883727in}{1.891361in}}{\pgfqpoint{0.875827in}{1.888089in}}{\pgfqpoint{0.870003in}{1.882265in}}%
\pgfpathcurveto{\pgfqpoint{0.864179in}{1.876441in}}{\pgfqpoint{0.860907in}{1.868541in}}{\pgfqpoint{0.860907in}{1.860304in}}%
\pgfpathcurveto{\pgfqpoint{0.860907in}{1.852068in}}{\pgfqpoint{0.864179in}{1.844168in}}{\pgfqpoint{0.870003in}{1.838344in}}%
\pgfpathcurveto{\pgfqpoint{0.875827in}{1.832520in}}{\pgfqpoint{0.883727in}{1.829248in}}{\pgfqpoint{0.891963in}{1.829248in}}%
\pgfpathclose%
\pgfusepath{stroke,fill}%
\end{pgfscope}%
\begin{pgfscope}%
\pgfpathrectangle{\pgfqpoint{0.457963in}{0.528059in}}{\pgfqpoint{6.200000in}{2.285714in}} %
\pgfusepath{clip}%
\pgfsetbuttcap%
\pgfsetroundjoin%
\definecolor{currentfill}{rgb}{1.000000,0.000000,0.000000}%
\pgfsetfillcolor{currentfill}%
\pgfsetlinewidth{1.003750pt}%
\definecolor{currentstroke}{rgb}{1.000000,0.000000,0.000000}%
\pgfsetstrokecolor{currentstroke}%
\pgfsetdash{}{0pt}%
\pgfpathmoveto{\pgfqpoint{0.984963in}{2.456187in}}%
\pgfpathcurveto{\pgfqpoint{0.993200in}{2.456187in}}{\pgfqpoint{1.001100in}{2.459459in}}{\pgfqpoint{1.006924in}{2.465283in}}%
\pgfpathcurveto{\pgfqpoint{1.012748in}{2.471107in}}{\pgfqpoint{1.016020in}{2.479007in}}{\pgfqpoint{1.016020in}{2.487243in}}%
\pgfpathcurveto{\pgfqpoint{1.016020in}{2.495479in}}{\pgfqpoint{1.012748in}{2.503379in}}{\pgfqpoint{1.006924in}{2.509203in}}%
\pgfpathcurveto{\pgfqpoint{1.001100in}{2.515027in}}{\pgfqpoint{0.993200in}{2.518300in}}{\pgfqpoint{0.984963in}{2.518300in}}%
\pgfpathcurveto{\pgfqpoint{0.976727in}{2.518300in}}{\pgfqpoint{0.968827in}{2.515027in}}{\pgfqpoint{0.963003in}{2.509203in}}%
\pgfpathcurveto{\pgfqpoint{0.957179in}{2.503379in}}{\pgfqpoint{0.953907in}{2.495479in}}{\pgfqpoint{0.953907in}{2.487243in}}%
\pgfpathcurveto{\pgfqpoint{0.953907in}{2.479007in}}{\pgfqpoint{0.957179in}{2.471107in}}{\pgfqpoint{0.963003in}{2.465283in}}%
\pgfpathcurveto{\pgfqpoint{0.968827in}{2.459459in}}{\pgfqpoint{0.976727in}{2.456187in}}{\pgfqpoint{0.984963in}{2.456187in}}%
\pgfpathclose%
\pgfusepath{stroke,fill}%
\end{pgfscope}%
\begin{pgfscope}%
\pgfpathrectangle{\pgfqpoint{0.457963in}{0.528059in}}{\pgfqpoint{6.200000in}{2.285714in}} %
\pgfusepath{clip}%
\pgfsetbuttcap%
\pgfsetroundjoin%
\definecolor{currentfill}{rgb}{1.000000,0.000000,0.000000}%
\pgfsetfillcolor{currentfill}%
\pgfsetlinewidth{1.003750pt}%
\definecolor{currentstroke}{rgb}{1.000000,0.000000,0.000000}%
\pgfsetstrokecolor{currentstroke}%
\pgfsetdash{}{0pt}%
\pgfpathmoveto{\pgfqpoint{1.057297in}{1.280676in}}%
\pgfpathcurveto{\pgfqpoint{1.065533in}{1.280676in}}{\pgfqpoint{1.073433in}{1.283949in}}{\pgfqpoint{1.079257in}{1.289773in}}%
\pgfpathcurveto{\pgfqpoint{1.085081in}{1.295597in}}{\pgfqpoint{1.088353in}{1.303497in}}{\pgfqpoint{1.088353in}{1.311733in}}%
\pgfpathcurveto{\pgfqpoint{1.088353in}{1.319969in}}{\pgfqpoint{1.085081in}{1.327869in}}{\pgfqpoint{1.079257in}{1.333693in}}%
\pgfpathcurveto{\pgfqpoint{1.073433in}{1.339517in}}{\pgfqpoint{1.065533in}{1.342789in}}{\pgfqpoint{1.057297in}{1.342789in}}%
\pgfpathcurveto{\pgfqpoint{1.049060in}{1.342789in}}{\pgfqpoint{1.041160in}{1.339517in}}{\pgfqpoint{1.035336in}{1.333693in}}%
\pgfpathcurveto{\pgfqpoint{1.029512in}{1.327869in}}{\pgfqpoint{1.026240in}{1.319969in}}{\pgfqpoint{1.026240in}{1.311733in}}%
\pgfpathcurveto{\pgfqpoint{1.026240in}{1.303497in}}{\pgfqpoint{1.029512in}{1.295597in}}{\pgfqpoint{1.035336in}{1.289773in}}%
\pgfpathcurveto{\pgfqpoint{1.041160in}{1.283949in}}{\pgfqpoint{1.049060in}{1.280676in}}{\pgfqpoint{1.057297in}{1.280676in}}%
\pgfpathclose%
\pgfusepath{stroke,fill}%
\end{pgfscope}%
\begin{pgfscope}%
\pgfpathrectangle{\pgfqpoint{0.457963in}{0.528059in}}{\pgfqpoint{6.200000in}{2.285714in}} %
\pgfusepath{clip}%
\pgfsetbuttcap%
\pgfsetroundjoin%
\definecolor{currentfill}{rgb}{1.000000,0.000000,0.000000}%
\pgfsetfillcolor{currentfill}%
\pgfsetlinewidth{1.003750pt}%
\definecolor{currentstroke}{rgb}{1.000000,0.000000,0.000000}%
\pgfsetstrokecolor{currentstroke}%
\pgfsetdash{}{0pt}%
\pgfpathmoveto{\pgfqpoint{1.315630in}{2.377819in}}%
\pgfpathcurveto{\pgfqpoint{1.323866in}{2.377819in}}{\pgfqpoint{1.331766in}{2.381092in}}{\pgfqpoint{1.337590in}{2.386916in}}%
\pgfpathcurveto{\pgfqpoint{1.343414in}{2.392739in}}{\pgfqpoint{1.346686in}{2.400639in}}{\pgfqpoint{1.346686in}{2.408876in}}%
\pgfpathcurveto{\pgfqpoint{1.346686in}{2.417112in}}{\pgfqpoint{1.343414in}{2.425012in}}{\pgfqpoint{1.337590in}{2.430836in}}%
\pgfpathcurveto{\pgfqpoint{1.331766in}{2.436660in}}{\pgfqpoint{1.323866in}{2.439932in}}{\pgfqpoint{1.315630in}{2.439932in}}%
\pgfpathcurveto{\pgfqpoint{1.307394in}{2.439932in}}{\pgfqpoint{1.299494in}{2.436660in}}{\pgfqpoint{1.293670in}{2.430836in}}%
\pgfpathcurveto{\pgfqpoint{1.287846in}{2.425012in}}{\pgfqpoint{1.284574in}{2.417112in}}{\pgfqpoint{1.284574in}{2.408876in}}%
\pgfpathcurveto{\pgfqpoint{1.284574in}{2.400639in}}{\pgfqpoint{1.287846in}{2.392739in}}{\pgfqpoint{1.293670in}{2.386916in}}%
\pgfpathcurveto{\pgfqpoint{1.299494in}{2.381092in}}{\pgfqpoint{1.307394in}{2.377819in}}{\pgfqpoint{1.315630in}{2.377819in}}%
\pgfpathclose%
\pgfusepath{stroke,fill}%
\end{pgfscope}%
\begin{pgfscope}%
\pgfpathrectangle{\pgfqpoint{0.457963in}{0.528059in}}{\pgfqpoint{6.200000in}{2.285714in}} %
\pgfusepath{clip}%
\pgfsetbuttcap%
\pgfsetroundjoin%
\definecolor{currentfill}{rgb}{1.000000,0.000000,0.000000}%
\pgfsetfillcolor{currentfill}%
\pgfsetlinewidth{1.003750pt}%
\definecolor{currentstroke}{rgb}{1.000000,0.000000,0.000000}%
\pgfsetstrokecolor{currentstroke}%
\pgfsetdash{}{0pt}%
\pgfpathmoveto{\pgfqpoint{1.449963in}{0.967207in}}%
\pgfpathcurveto{\pgfqpoint{1.458200in}{0.967207in}}{\pgfqpoint{1.466100in}{0.970479in}}{\pgfqpoint{1.471924in}{0.976303in}}%
\pgfpathcurveto{\pgfqpoint{1.477748in}{0.982127in}}{\pgfqpoint{1.481020in}{0.990027in}}{\pgfqpoint{1.481020in}{0.998264in}}%
\pgfpathcurveto{\pgfqpoint{1.481020in}{1.006500in}}{\pgfqpoint{1.477748in}{1.014400in}}{\pgfqpoint{1.471924in}{1.020224in}}%
\pgfpathcurveto{\pgfqpoint{1.466100in}{1.026048in}}{\pgfqpoint{1.458200in}{1.029320in}}{\pgfqpoint{1.449963in}{1.029320in}}%
\pgfpathcurveto{\pgfqpoint{1.441727in}{1.029320in}}{\pgfqpoint{1.433827in}{1.026048in}}{\pgfqpoint{1.428003in}{1.020224in}}%
\pgfpathcurveto{\pgfqpoint{1.422179in}{1.014400in}}{\pgfqpoint{1.418907in}{1.006500in}}{\pgfqpoint{1.418907in}{0.998264in}}%
\pgfpathcurveto{\pgfqpoint{1.418907in}{0.990027in}}{\pgfqpoint{1.422179in}{0.982127in}}{\pgfqpoint{1.428003in}{0.976303in}}%
\pgfpathcurveto{\pgfqpoint{1.433827in}{0.970479in}}{\pgfqpoint{1.441727in}{0.967207in}}{\pgfqpoint{1.449963in}{0.967207in}}%
\pgfpathclose%
\pgfusepath{stroke,fill}%
\end{pgfscope}%
\begin{pgfscope}%
\pgfpathrectangle{\pgfqpoint{0.457963in}{0.528059in}}{\pgfqpoint{6.200000in}{2.285714in}} %
\pgfusepath{clip}%
\pgfsetbuttcap%
\pgfsetroundjoin%
\definecolor{currentfill}{rgb}{1.000000,0.000000,0.000000}%
\pgfsetfillcolor{currentfill}%
\pgfsetlinewidth{1.003750pt}%
\definecolor{currentstroke}{rgb}{1.000000,0.000000,0.000000}%
\pgfsetstrokecolor{currentstroke}%
\pgfsetdash{}{0pt}%
\pgfpathmoveto{\pgfqpoint{1.460297in}{2.443125in}}%
\pgfpathcurveto{\pgfqpoint{1.468533in}{2.443125in}}{\pgfqpoint{1.476433in}{2.446398in}}{\pgfqpoint{1.482257in}{2.452222in}}%
\pgfpathcurveto{\pgfqpoint{1.488081in}{2.458046in}}{\pgfqpoint{1.491353in}{2.465946in}}{\pgfqpoint{1.491353in}{2.474182in}}%
\pgfpathcurveto{\pgfqpoint{1.491353in}{2.482418in}}{\pgfqpoint{1.488081in}{2.490318in}}{\pgfqpoint{1.482257in}{2.496142in}}%
\pgfpathcurveto{\pgfqpoint{1.476433in}{2.501966in}}{\pgfqpoint{1.468533in}{2.505238in}}{\pgfqpoint{1.460297in}{2.505238in}}%
\pgfpathcurveto{\pgfqpoint{1.452060in}{2.505238in}}{\pgfqpoint{1.444160in}{2.501966in}}{\pgfqpoint{1.438336in}{2.496142in}}%
\pgfpathcurveto{\pgfqpoint{1.432512in}{2.490318in}}{\pgfqpoint{1.429240in}{2.482418in}}{\pgfqpoint{1.429240in}{2.474182in}}%
\pgfpathcurveto{\pgfqpoint{1.429240in}{2.465946in}}{\pgfqpoint{1.432512in}{2.458046in}}{\pgfqpoint{1.438336in}{2.452222in}}%
\pgfpathcurveto{\pgfqpoint{1.444160in}{2.446398in}}{\pgfqpoint{1.452060in}{2.443125in}}{\pgfqpoint{1.460297in}{2.443125in}}%
\pgfpathclose%
\pgfusepath{stroke,fill}%
\end{pgfscope}%
\begin{pgfscope}%
\pgfpathrectangle{\pgfqpoint{0.457963in}{0.528059in}}{\pgfqpoint{6.200000in}{2.285714in}} %
\pgfusepath{clip}%
\pgfsetbuttcap%
\pgfsetroundjoin%
\definecolor{currentfill}{rgb}{1.000000,0.000000,0.000000}%
\pgfsetfillcolor{currentfill}%
\pgfsetlinewidth{1.003750pt}%
\definecolor{currentstroke}{rgb}{1.000000,0.000000,0.000000}%
\pgfsetstrokecolor{currentstroke}%
\pgfsetdash{}{0pt}%
\pgfpathmoveto{\pgfqpoint{1.821963in}{2.273329in}}%
\pgfpathcurveto{\pgfqpoint{1.830200in}{2.273329in}}{\pgfqpoint{1.838100in}{2.276602in}}{\pgfqpoint{1.843924in}{2.282426in}}%
\pgfpathcurveto{\pgfqpoint{1.849748in}{2.288250in}}{\pgfqpoint{1.853020in}{2.296150in}}{\pgfqpoint{1.853020in}{2.304386in}}%
\pgfpathcurveto{\pgfqpoint{1.853020in}{2.312622in}}{\pgfqpoint{1.849748in}{2.320522in}}{\pgfqpoint{1.843924in}{2.326346in}}%
\pgfpathcurveto{\pgfqpoint{1.838100in}{2.332170in}}{\pgfqpoint{1.830200in}{2.335442in}}{\pgfqpoint{1.821963in}{2.335442in}}%
\pgfpathcurveto{\pgfqpoint{1.813727in}{2.335442in}}{\pgfqpoint{1.805827in}{2.332170in}}{\pgfqpoint{1.800003in}{2.326346in}}%
\pgfpathcurveto{\pgfqpoint{1.794179in}{2.320522in}}{\pgfqpoint{1.790907in}{2.312622in}}{\pgfqpoint{1.790907in}{2.304386in}}%
\pgfpathcurveto{\pgfqpoint{1.790907in}{2.296150in}}{\pgfqpoint{1.794179in}{2.288250in}}{\pgfqpoint{1.800003in}{2.282426in}}%
\pgfpathcurveto{\pgfqpoint{1.805827in}{2.276602in}}{\pgfqpoint{1.813727in}{2.273329in}}{\pgfqpoint{1.821963in}{2.273329in}}%
\pgfpathclose%
\pgfusepath{stroke,fill}%
\end{pgfscope}%
\begin{pgfscope}%
\pgfpathrectangle{\pgfqpoint{0.457963in}{0.528059in}}{\pgfqpoint{6.200000in}{2.285714in}} %
\pgfusepath{clip}%
\pgfsetbuttcap%
\pgfsetroundjoin%
\definecolor{currentfill}{rgb}{1.000000,0.000000,0.000000}%
\pgfsetfillcolor{currentfill}%
\pgfsetlinewidth{1.003750pt}%
\definecolor{currentstroke}{rgb}{1.000000,0.000000,0.000000}%
\pgfsetstrokecolor{currentstroke}%
\pgfsetdash{}{0pt}%
\pgfpathmoveto{\pgfqpoint{2.204297in}{2.116595in}}%
\pgfpathcurveto{\pgfqpoint{2.212533in}{2.116595in}}{\pgfqpoint{2.220433in}{2.119867in}}{\pgfqpoint{2.226257in}{2.125691in}}%
\pgfpathcurveto{\pgfqpoint{2.232081in}{2.131515in}}{\pgfqpoint{2.235353in}{2.139415in}}{\pgfqpoint{2.235353in}{2.147651in}}%
\pgfpathcurveto{\pgfqpoint{2.235353in}{2.155888in}}{\pgfqpoint{2.232081in}{2.163788in}}{\pgfqpoint{2.226257in}{2.169612in}}%
\pgfpathcurveto{\pgfqpoint{2.220433in}{2.175435in}}{\pgfqpoint{2.212533in}{2.178708in}}{\pgfqpoint{2.204297in}{2.178708in}}%
\pgfpathcurveto{\pgfqpoint{2.196060in}{2.178708in}}{\pgfqpoint{2.188160in}{2.175435in}}{\pgfqpoint{2.182336in}{2.169612in}}%
\pgfpathcurveto{\pgfqpoint{2.176512in}{2.163788in}}{\pgfqpoint{2.173240in}{2.155888in}}{\pgfqpoint{2.173240in}{2.147651in}}%
\pgfpathcurveto{\pgfqpoint{2.173240in}{2.139415in}}{\pgfqpoint{2.176512in}{2.131515in}}{\pgfqpoint{2.182336in}{2.125691in}}%
\pgfpathcurveto{\pgfqpoint{2.188160in}{2.119867in}}{\pgfqpoint{2.196060in}{2.116595in}}{\pgfqpoint{2.204297in}{2.116595in}}%
\pgfpathclose%
\pgfusepath{stroke,fill}%
\end{pgfscope}%
\begin{pgfscope}%
\pgfpathrectangle{\pgfqpoint{0.457963in}{0.528059in}}{\pgfqpoint{6.200000in}{2.285714in}} %
\pgfusepath{clip}%
\pgfsetbuttcap%
\pgfsetroundjoin%
\definecolor{currentfill}{rgb}{1.000000,0.000000,0.000000}%
\pgfsetfillcolor{currentfill}%
\pgfsetlinewidth{1.003750pt}%
\definecolor{currentstroke}{rgb}{1.000000,0.000000,0.000000}%
\pgfsetstrokecolor{currentstroke}%
\pgfsetdash{}{0pt}%
\pgfpathmoveto{\pgfqpoint{2.214630in}{1.959860in}}%
\pgfpathcurveto{\pgfqpoint{2.222866in}{1.959860in}}{\pgfqpoint{2.230766in}{1.963132in}}{\pgfqpoint{2.236590in}{1.968956in}}%
\pgfpathcurveto{\pgfqpoint{2.242414in}{1.974780in}}{\pgfqpoint{2.245686in}{1.982680in}}{\pgfqpoint{2.245686in}{1.990917in}}%
\pgfpathcurveto{\pgfqpoint{2.245686in}{1.999153in}}{\pgfqpoint{2.242414in}{2.007053in}}{\pgfqpoint{2.236590in}{2.012877in}}%
\pgfpathcurveto{\pgfqpoint{2.230766in}{2.018701in}}{\pgfqpoint{2.222866in}{2.021973in}}{\pgfqpoint{2.214630in}{2.021973in}}%
\pgfpathcurveto{\pgfqpoint{2.206394in}{2.021973in}}{\pgfqpoint{2.198494in}{2.018701in}}{\pgfqpoint{2.192670in}{2.012877in}}%
\pgfpathcurveto{\pgfqpoint{2.186846in}{2.007053in}}{\pgfqpoint{2.183574in}{1.999153in}}{\pgfqpoint{2.183574in}{1.990917in}}%
\pgfpathcurveto{\pgfqpoint{2.183574in}{1.982680in}}{\pgfqpoint{2.186846in}{1.974780in}}{\pgfqpoint{2.192670in}{1.968956in}}%
\pgfpathcurveto{\pgfqpoint{2.198494in}{1.963132in}}{\pgfqpoint{2.206394in}{1.959860in}}{\pgfqpoint{2.214630in}{1.959860in}}%
\pgfpathclose%
\pgfusepath{stroke,fill}%
\end{pgfscope}%
\begin{pgfscope}%
\pgfpathrectangle{\pgfqpoint{0.457963in}{0.528059in}}{\pgfqpoint{6.200000in}{2.285714in}} %
\pgfusepath{clip}%
\pgfsetbuttcap%
\pgfsetroundjoin%
\definecolor{currentfill}{rgb}{1.000000,0.000000,0.000000}%
\pgfsetfillcolor{currentfill}%
\pgfsetlinewidth{1.003750pt}%
\definecolor{currentstroke}{rgb}{1.000000,0.000000,0.000000}%
\pgfsetstrokecolor{currentstroke}%
\pgfsetdash{}{0pt}%
\pgfpathmoveto{\pgfqpoint{2.865630in}{1.280676in}}%
\pgfpathcurveto{\pgfqpoint{2.873866in}{1.280676in}}{\pgfqpoint{2.881766in}{1.283949in}}{\pgfqpoint{2.887590in}{1.289773in}}%
\pgfpathcurveto{\pgfqpoint{2.893414in}{1.295597in}}{\pgfqpoint{2.896686in}{1.303497in}}{\pgfqpoint{2.896686in}{1.311733in}}%
\pgfpathcurveto{\pgfqpoint{2.896686in}{1.319969in}}{\pgfqpoint{2.893414in}{1.327869in}}{\pgfqpoint{2.887590in}{1.333693in}}%
\pgfpathcurveto{\pgfqpoint{2.881766in}{1.339517in}}{\pgfqpoint{2.873866in}{1.342789in}}{\pgfqpoint{2.865630in}{1.342789in}}%
\pgfpathcurveto{\pgfqpoint{2.857394in}{1.342789in}}{\pgfqpoint{2.849494in}{1.339517in}}{\pgfqpoint{2.843670in}{1.333693in}}%
\pgfpathcurveto{\pgfqpoint{2.837846in}{1.327869in}}{\pgfqpoint{2.834574in}{1.319969in}}{\pgfqpoint{2.834574in}{1.311733in}}%
\pgfpathcurveto{\pgfqpoint{2.834574in}{1.303497in}}{\pgfqpoint{2.837846in}{1.295597in}}{\pgfqpoint{2.843670in}{1.289773in}}%
\pgfpathcurveto{\pgfqpoint{2.849494in}{1.283949in}}{\pgfqpoint{2.857394in}{1.280676in}}{\pgfqpoint{2.865630in}{1.280676in}}%
\pgfpathclose%
\pgfusepath{stroke,fill}%
\end{pgfscope}%
\begin{pgfscope}%
\pgfpathrectangle{\pgfqpoint{0.457963in}{0.528059in}}{\pgfqpoint{6.200000in}{2.285714in}} %
\pgfusepath{clip}%
\pgfsetbuttcap%
\pgfsetroundjoin%
\definecolor{currentfill}{rgb}{1.000000,0.000000,0.000000}%
\pgfsetfillcolor{currentfill}%
\pgfsetlinewidth{1.003750pt}%
\definecolor{currentstroke}{rgb}{1.000000,0.000000,0.000000}%
\pgfsetstrokecolor{currentstroke}%
\pgfsetdash{}{0pt}%
\pgfpathmoveto{\pgfqpoint{3.258297in}{1.084758in}}%
\pgfpathcurveto{\pgfqpoint{3.266533in}{1.084758in}}{\pgfqpoint{3.274433in}{1.088030in}}{\pgfqpoint{3.280257in}{1.093854in}}%
\pgfpathcurveto{\pgfqpoint{3.286081in}{1.099678in}}{\pgfqpoint{3.289353in}{1.107578in}}{\pgfqpoint{3.289353in}{1.115815in}}%
\pgfpathcurveto{\pgfqpoint{3.289353in}{1.124051in}}{\pgfqpoint{3.286081in}{1.131951in}}{\pgfqpoint{3.280257in}{1.137775in}}%
\pgfpathcurveto{\pgfqpoint{3.274433in}{1.143599in}}{\pgfqpoint{3.266533in}{1.146871in}}{\pgfqpoint{3.258297in}{1.146871in}}%
\pgfpathcurveto{\pgfqpoint{3.250060in}{1.146871in}}{\pgfqpoint{3.242160in}{1.143599in}}{\pgfqpoint{3.236336in}{1.137775in}}%
\pgfpathcurveto{\pgfqpoint{3.230512in}{1.131951in}}{\pgfqpoint{3.227240in}{1.124051in}}{\pgfqpoint{3.227240in}{1.115815in}}%
\pgfpathcurveto{\pgfqpoint{3.227240in}{1.107578in}}{\pgfqpoint{3.230512in}{1.099678in}}{\pgfqpoint{3.236336in}{1.093854in}}%
\pgfpathcurveto{\pgfqpoint{3.242160in}{1.088030in}}{\pgfqpoint{3.250060in}{1.084758in}}{\pgfqpoint{3.258297in}{1.084758in}}%
\pgfpathclose%
\pgfusepath{stroke,fill}%
\end{pgfscope}%
\begin{pgfscope}%
\pgfpathrectangle{\pgfqpoint{0.457963in}{0.528059in}}{\pgfqpoint{6.200000in}{2.285714in}} %
\pgfusepath{clip}%
\pgfsetbuttcap%
\pgfsetroundjoin%
\definecolor{currentfill}{rgb}{0.833333,0.833333,1.000000}%
\pgfsetfillcolor{currentfill}%
\pgfsetlinewidth{1.003750pt}%
\definecolor{currentstroke}{rgb}{0.833333,0.833333,1.000000}%
\pgfsetstrokecolor{currentstroke}%
\pgfsetdash{}{0pt}%
\pgfpathmoveto{\pgfqpoint{0.457963in}{0.823534in}}%
\pgfpathcurveto{\pgfqpoint{0.466200in}{0.823534in}}{\pgfqpoint{0.474100in}{0.826806in}}{\pgfqpoint{0.479924in}{0.832630in}}%
\pgfpathcurveto{\pgfqpoint{0.485748in}{0.838454in}}{\pgfqpoint{0.489020in}{0.846354in}}{\pgfqpoint{0.489020in}{0.854590in}}%
\pgfpathcurveto{\pgfqpoint{0.489020in}{0.862826in}}{\pgfqpoint{0.485748in}{0.870726in}}{\pgfqpoint{0.479924in}{0.876550in}}%
\pgfpathcurveto{\pgfqpoint{0.474100in}{0.882374in}}{\pgfqpoint{0.466200in}{0.885647in}}{\pgfqpoint{0.457963in}{0.885647in}}%
\pgfpathcurveto{\pgfqpoint{0.449727in}{0.885647in}}{\pgfqpoint{0.441827in}{0.882374in}}{\pgfqpoint{0.436003in}{0.876550in}}%
\pgfpathcurveto{\pgfqpoint{0.430179in}{0.870726in}}{\pgfqpoint{0.426907in}{0.862826in}}{\pgfqpoint{0.426907in}{0.854590in}}%
\pgfpathcurveto{\pgfqpoint{0.426907in}{0.846354in}}{\pgfqpoint{0.430179in}{0.838454in}}{\pgfqpoint{0.436003in}{0.832630in}}%
\pgfpathcurveto{\pgfqpoint{0.441827in}{0.826806in}}{\pgfqpoint{0.449727in}{0.823534in}}{\pgfqpoint{0.457963in}{0.823534in}}%
\pgfpathclose%
\pgfusepath{stroke,fill}%
\end{pgfscope}%
\begin{pgfscope}%
\pgfpathrectangle{\pgfqpoint{0.457963in}{0.528059in}}{\pgfqpoint{6.200000in}{2.285714in}} %
\pgfusepath{clip}%
\pgfsetbuttcap%
\pgfsetroundjoin%
\definecolor{currentfill}{rgb}{0.833333,0.833333,1.000000}%
\pgfsetfillcolor{currentfill}%
\pgfsetlinewidth{1.003750pt}%
\definecolor{currentstroke}{rgb}{0.833333,0.833333,1.000000}%
\pgfsetstrokecolor{currentstroke}%
\pgfsetdash{}{0pt}%
\pgfpathmoveto{\pgfqpoint{0.457963in}{0.823534in}}%
\pgfpathcurveto{\pgfqpoint{0.466200in}{0.823534in}}{\pgfqpoint{0.474100in}{0.826806in}}{\pgfqpoint{0.479924in}{0.832630in}}%
\pgfpathcurveto{\pgfqpoint{0.485748in}{0.838454in}}{\pgfqpoint{0.489020in}{0.846354in}}{\pgfqpoint{0.489020in}{0.854590in}}%
\pgfpathcurveto{\pgfqpoint{0.489020in}{0.862826in}}{\pgfqpoint{0.485748in}{0.870726in}}{\pgfqpoint{0.479924in}{0.876550in}}%
\pgfpathcurveto{\pgfqpoint{0.474100in}{0.882374in}}{\pgfqpoint{0.466200in}{0.885647in}}{\pgfqpoint{0.457963in}{0.885647in}}%
\pgfpathcurveto{\pgfqpoint{0.449727in}{0.885647in}}{\pgfqpoint{0.441827in}{0.882374in}}{\pgfqpoint{0.436003in}{0.876550in}}%
\pgfpathcurveto{\pgfqpoint{0.430179in}{0.870726in}}{\pgfqpoint{0.426907in}{0.862826in}}{\pgfqpoint{0.426907in}{0.854590in}}%
\pgfpathcurveto{\pgfqpoint{0.426907in}{0.846354in}}{\pgfqpoint{0.430179in}{0.838454in}}{\pgfqpoint{0.436003in}{0.832630in}}%
\pgfpathcurveto{\pgfqpoint{0.441827in}{0.826806in}}{\pgfqpoint{0.449727in}{0.823534in}}{\pgfqpoint{0.457963in}{0.823534in}}%
\pgfpathclose%
\pgfusepath{stroke,fill}%
\end{pgfscope}%
\begin{pgfscope}%
\pgfpathrectangle{\pgfqpoint{0.457963in}{0.528059in}}{\pgfqpoint{6.200000in}{2.285714in}} %
\pgfusepath{clip}%
\pgfsetbuttcap%
\pgfsetroundjoin%
\definecolor{currentfill}{rgb}{0.833333,0.833333,1.000000}%
\pgfsetfillcolor{currentfill}%
\pgfsetlinewidth{1.003750pt}%
\definecolor{currentstroke}{rgb}{0.833333,0.833333,1.000000}%
\pgfsetstrokecolor{currentstroke}%
\pgfsetdash{}{0pt}%
\pgfpathmoveto{\pgfqpoint{0.457963in}{0.823534in}}%
\pgfpathcurveto{\pgfqpoint{0.466200in}{0.823534in}}{\pgfqpoint{0.474100in}{0.826806in}}{\pgfqpoint{0.479924in}{0.832630in}}%
\pgfpathcurveto{\pgfqpoint{0.485748in}{0.838454in}}{\pgfqpoint{0.489020in}{0.846354in}}{\pgfqpoint{0.489020in}{0.854590in}}%
\pgfpathcurveto{\pgfqpoint{0.489020in}{0.862826in}}{\pgfqpoint{0.485748in}{0.870726in}}{\pgfqpoint{0.479924in}{0.876550in}}%
\pgfpathcurveto{\pgfqpoint{0.474100in}{0.882374in}}{\pgfqpoint{0.466200in}{0.885647in}}{\pgfqpoint{0.457963in}{0.885647in}}%
\pgfpathcurveto{\pgfqpoint{0.449727in}{0.885647in}}{\pgfqpoint{0.441827in}{0.882374in}}{\pgfqpoint{0.436003in}{0.876550in}}%
\pgfpathcurveto{\pgfqpoint{0.430179in}{0.870726in}}{\pgfqpoint{0.426907in}{0.862826in}}{\pgfqpoint{0.426907in}{0.854590in}}%
\pgfpathcurveto{\pgfqpoint{0.426907in}{0.846354in}}{\pgfqpoint{0.430179in}{0.838454in}}{\pgfqpoint{0.436003in}{0.832630in}}%
\pgfpathcurveto{\pgfqpoint{0.441827in}{0.826806in}}{\pgfqpoint{0.449727in}{0.823534in}}{\pgfqpoint{0.457963in}{0.823534in}}%
\pgfpathclose%
\pgfusepath{stroke,fill}%
\end{pgfscope}%
\begin{pgfscope}%
\pgfpathrectangle{\pgfqpoint{0.457963in}{0.528059in}}{\pgfqpoint{6.200000in}{2.285714in}} %
\pgfusepath{clip}%
\pgfsetbuttcap%
\pgfsetroundjoin%
\definecolor{currentfill}{rgb}{0.833333,0.833333,1.000000}%
\pgfsetfillcolor{currentfill}%
\pgfsetlinewidth{1.003750pt}%
\definecolor{currentstroke}{rgb}{0.833333,0.833333,1.000000}%
\pgfsetstrokecolor{currentstroke}%
\pgfsetdash{}{0pt}%
\pgfpathmoveto{\pgfqpoint{0.468297in}{0.823534in}}%
\pgfpathcurveto{\pgfqpoint{0.476533in}{0.823534in}}{\pgfqpoint{0.484433in}{0.826806in}}{\pgfqpoint{0.490257in}{0.832630in}}%
\pgfpathcurveto{\pgfqpoint{0.496081in}{0.838454in}}{\pgfqpoint{0.499353in}{0.846354in}}{\pgfqpoint{0.499353in}{0.854590in}}%
\pgfpathcurveto{\pgfqpoint{0.499353in}{0.862826in}}{\pgfqpoint{0.496081in}{0.870726in}}{\pgfqpoint{0.490257in}{0.876550in}}%
\pgfpathcurveto{\pgfqpoint{0.484433in}{0.882374in}}{\pgfqpoint{0.476533in}{0.885647in}}{\pgfqpoint{0.468297in}{0.885647in}}%
\pgfpathcurveto{\pgfqpoint{0.460060in}{0.885647in}}{\pgfqpoint{0.452160in}{0.882374in}}{\pgfqpoint{0.446336in}{0.876550in}}%
\pgfpathcurveto{\pgfqpoint{0.440512in}{0.870726in}}{\pgfqpoint{0.437240in}{0.862826in}}{\pgfqpoint{0.437240in}{0.854590in}}%
\pgfpathcurveto{\pgfqpoint{0.437240in}{0.846354in}}{\pgfqpoint{0.440512in}{0.838454in}}{\pgfqpoint{0.446336in}{0.832630in}}%
\pgfpathcurveto{\pgfqpoint{0.452160in}{0.826806in}}{\pgfqpoint{0.460060in}{0.823534in}}{\pgfqpoint{0.468297in}{0.823534in}}%
\pgfpathclose%
\pgfusepath{stroke,fill}%
\end{pgfscope}%
\begin{pgfscope}%
\pgfpathrectangle{\pgfqpoint{0.457963in}{0.528059in}}{\pgfqpoint{6.200000in}{2.285714in}} %
\pgfusepath{clip}%
\pgfsetbuttcap%
\pgfsetroundjoin%
\definecolor{currentfill}{rgb}{0.833333,0.833333,1.000000}%
\pgfsetfillcolor{currentfill}%
\pgfsetlinewidth{1.003750pt}%
\definecolor{currentstroke}{rgb}{0.833333,0.833333,1.000000}%
\pgfsetstrokecolor{currentstroke}%
\pgfsetdash{}{0pt}%
\pgfpathmoveto{\pgfqpoint{0.468297in}{0.823534in}}%
\pgfpathcurveto{\pgfqpoint{0.476533in}{0.823534in}}{\pgfqpoint{0.484433in}{0.826806in}}{\pgfqpoint{0.490257in}{0.832630in}}%
\pgfpathcurveto{\pgfqpoint{0.496081in}{0.838454in}}{\pgfqpoint{0.499353in}{0.846354in}}{\pgfqpoint{0.499353in}{0.854590in}}%
\pgfpathcurveto{\pgfqpoint{0.499353in}{0.862826in}}{\pgfqpoint{0.496081in}{0.870726in}}{\pgfqpoint{0.490257in}{0.876550in}}%
\pgfpathcurveto{\pgfqpoint{0.484433in}{0.882374in}}{\pgfqpoint{0.476533in}{0.885647in}}{\pgfqpoint{0.468297in}{0.885647in}}%
\pgfpathcurveto{\pgfqpoint{0.460060in}{0.885647in}}{\pgfqpoint{0.452160in}{0.882374in}}{\pgfqpoint{0.446336in}{0.876550in}}%
\pgfpathcurveto{\pgfqpoint{0.440512in}{0.870726in}}{\pgfqpoint{0.437240in}{0.862826in}}{\pgfqpoint{0.437240in}{0.854590in}}%
\pgfpathcurveto{\pgfqpoint{0.437240in}{0.846354in}}{\pgfqpoint{0.440512in}{0.838454in}}{\pgfqpoint{0.446336in}{0.832630in}}%
\pgfpathcurveto{\pgfqpoint{0.452160in}{0.826806in}}{\pgfqpoint{0.460060in}{0.823534in}}{\pgfqpoint{0.468297in}{0.823534in}}%
\pgfpathclose%
\pgfusepath{stroke,fill}%
\end{pgfscope}%
\begin{pgfscope}%
\pgfpathrectangle{\pgfqpoint{0.457963in}{0.528059in}}{\pgfqpoint{6.200000in}{2.285714in}} %
\pgfusepath{clip}%
\pgfsetbuttcap%
\pgfsetroundjoin%
\definecolor{currentfill}{rgb}{0.833333,0.833333,1.000000}%
\pgfsetfillcolor{currentfill}%
\pgfsetlinewidth{1.003750pt}%
\definecolor{currentstroke}{rgb}{0.833333,0.833333,1.000000}%
\pgfsetstrokecolor{currentstroke}%
\pgfsetdash{}{0pt}%
\pgfpathmoveto{\pgfqpoint{0.499297in}{0.823534in}}%
\pgfpathcurveto{\pgfqpoint{0.507533in}{0.823534in}}{\pgfqpoint{0.515433in}{0.826806in}}{\pgfqpoint{0.521257in}{0.832630in}}%
\pgfpathcurveto{\pgfqpoint{0.527081in}{0.838454in}}{\pgfqpoint{0.530353in}{0.846354in}}{\pgfqpoint{0.530353in}{0.854590in}}%
\pgfpathcurveto{\pgfqpoint{0.530353in}{0.862826in}}{\pgfqpoint{0.527081in}{0.870726in}}{\pgfqpoint{0.521257in}{0.876550in}}%
\pgfpathcurveto{\pgfqpoint{0.515433in}{0.882374in}}{\pgfqpoint{0.507533in}{0.885647in}}{\pgfqpoint{0.499297in}{0.885647in}}%
\pgfpathcurveto{\pgfqpoint{0.491060in}{0.885647in}}{\pgfqpoint{0.483160in}{0.882374in}}{\pgfqpoint{0.477336in}{0.876550in}}%
\pgfpathcurveto{\pgfqpoint{0.471512in}{0.870726in}}{\pgfqpoint{0.468240in}{0.862826in}}{\pgfqpoint{0.468240in}{0.854590in}}%
\pgfpathcurveto{\pgfqpoint{0.468240in}{0.846354in}}{\pgfqpoint{0.471512in}{0.838454in}}{\pgfqpoint{0.477336in}{0.832630in}}%
\pgfpathcurveto{\pgfqpoint{0.483160in}{0.826806in}}{\pgfqpoint{0.491060in}{0.823534in}}{\pgfqpoint{0.499297in}{0.823534in}}%
\pgfpathclose%
\pgfusepath{stroke,fill}%
\end{pgfscope}%
\begin{pgfscope}%
\pgfpathrectangle{\pgfqpoint{0.457963in}{0.528059in}}{\pgfqpoint{6.200000in}{2.285714in}} %
\pgfusepath{clip}%
\pgfsetbuttcap%
\pgfsetroundjoin%
\definecolor{currentfill}{rgb}{0.833333,0.833333,1.000000}%
\pgfsetfillcolor{currentfill}%
\pgfsetlinewidth{1.003750pt}%
\definecolor{currentstroke}{rgb}{0.833333,0.833333,1.000000}%
\pgfsetstrokecolor{currentstroke}%
\pgfsetdash{}{0pt}%
\pgfpathmoveto{\pgfqpoint{0.499297in}{0.823534in}}%
\pgfpathcurveto{\pgfqpoint{0.507533in}{0.823534in}}{\pgfqpoint{0.515433in}{0.826806in}}{\pgfqpoint{0.521257in}{0.832630in}}%
\pgfpathcurveto{\pgfqpoint{0.527081in}{0.838454in}}{\pgfqpoint{0.530353in}{0.846354in}}{\pgfqpoint{0.530353in}{0.854590in}}%
\pgfpathcurveto{\pgfqpoint{0.530353in}{0.862826in}}{\pgfqpoint{0.527081in}{0.870726in}}{\pgfqpoint{0.521257in}{0.876550in}}%
\pgfpathcurveto{\pgfqpoint{0.515433in}{0.882374in}}{\pgfqpoint{0.507533in}{0.885647in}}{\pgfqpoint{0.499297in}{0.885647in}}%
\pgfpathcurveto{\pgfqpoint{0.491060in}{0.885647in}}{\pgfqpoint{0.483160in}{0.882374in}}{\pgfqpoint{0.477336in}{0.876550in}}%
\pgfpathcurveto{\pgfqpoint{0.471512in}{0.870726in}}{\pgfqpoint{0.468240in}{0.862826in}}{\pgfqpoint{0.468240in}{0.854590in}}%
\pgfpathcurveto{\pgfqpoint{0.468240in}{0.846354in}}{\pgfqpoint{0.471512in}{0.838454in}}{\pgfqpoint{0.477336in}{0.832630in}}%
\pgfpathcurveto{\pgfqpoint{0.483160in}{0.826806in}}{\pgfqpoint{0.491060in}{0.823534in}}{\pgfqpoint{0.499297in}{0.823534in}}%
\pgfpathclose%
\pgfusepath{stroke,fill}%
\end{pgfscope}%
\begin{pgfscope}%
\pgfpathrectangle{\pgfqpoint{0.457963in}{0.528059in}}{\pgfqpoint{6.200000in}{2.285714in}} %
\pgfusepath{clip}%
\pgfsetbuttcap%
\pgfsetroundjoin%
\definecolor{currentfill}{rgb}{0.833333,0.833333,1.000000}%
\pgfsetfillcolor{currentfill}%
\pgfsetlinewidth{1.003750pt}%
\definecolor{currentstroke}{rgb}{0.833333,0.833333,1.000000}%
\pgfsetstrokecolor{currentstroke}%
\pgfsetdash{}{0pt}%
\pgfpathmoveto{\pgfqpoint{0.530297in}{0.823534in}}%
\pgfpathcurveto{\pgfqpoint{0.538533in}{0.823534in}}{\pgfqpoint{0.546433in}{0.826806in}}{\pgfqpoint{0.552257in}{0.832630in}}%
\pgfpathcurveto{\pgfqpoint{0.558081in}{0.838454in}}{\pgfqpoint{0.561353in}{0.846354in}}{\pgfqpoint{0.561353in}{0.854590in}}%
\pgfpathcurveto{\pgfqpoint{0.561353in}{0.862826in}}{\pgfqpoint{0.558081in}{0.870726in}}{\pgfqpoint{0.552257in}{0.876550in}}%
\pgfpathcurveto{\pgfqpoint{0.546433in}{0.882374in}}{\pgfqpoint{0.538533in}{0.885647in}}{\pgfqpoint{0.530297in}{0.885647in}}%
\pgfpathcurveto{\pgfqpoint{0.522060in}{0.885647in}}{\pgfqpoint{0.514160in}{0.882374in}}{\pgfqpoint{0.508336in}{0.876550in}}%
\pgfpathcurveto{\pgfqpoint{0.502512in}{0.870726in}}{\pgfqpoint{0.499240in}{0.862826in}}{\pgfqpoint{0.499240in}{0.854590in}}%
\pgfpathcurveto{\pgfqpoint{0.499240in}{0.846354in}}{\pgfqpoint{0.502512in}{0.838454in}}{\pgfqpoint{0.508336in}{0.832630in}}%
\pgfpathcurveto{\pgfqpoint{0.514160in}{0.826806in}}{\pgfqpoint{0.522060in}{0.823534in}}{\pgfqpoint{0.530297in}{0.823534in}}%
\pgfpathclose%
\pgfusepath{stroke,fill}%
\end{pgfscope}%
\begin{pgfscope}%
\pgfpathrectangle{\pgfqpoint{0.457963in}{0.528059in}}{\pgfqpoint{6.200000in}{2.285714in}} %
\pgfusepath{clip}%
\pgfsetbuttcap%
\pgfsetroundjoin%
\definecolor{currentfill}{rgb}{0.833333,0.833333,1.000000}%
\pgfsetfillcolor{currentfill}%
\pgfsetlinewidth{1.003750pt}%
\definecolor{currentstroke}{rgb}{0.833333,0.833333,1.000000}%
\pgfsetstrokecolor{currentstroke}%
\pgfsetdash{}{0pt}%
\pgfpathmoveto{\pgfqpoint{0.561297in}{0.810472in}}%
\pgfpathcurveto{\pgfqpoint{0.569533in}{0.810472in}}{\pgfqpoint{0.577433in}{0.813745in}}{\pgfqpoint{0.583257in}{0.819569in}}%
\pgfpathcurveto{\pgfqpoint{0.589081in}{0.825393in}}{\pgfqpoint{0.592353in}{0.833293in}}{\pgfqpoint{0.592353in}{0.841529in}}%
\pgfpathcurveto{\pgfqpoint{0.592353in}{0.849765in}}{\pgfqpoint{0.589081in}{0.857665in}}{\pgfqpoint{0.583257in}{0.863489in}}%
\pgfpathcurveto{\pgfqpoint{0.577433in}{0.869313in}}{\pgfqpoint{0.569533in}{0.872585in}}{\pgfqpoint{0.561297in}{0.872585in}}%
\pgfpathcurveto{\pgfqpoint{0.553060in}{0.872585in}}{\pgfqpoint{0.545160in}{0.869313in}}{\pgfqpoint{0.539336in}{0.863489in}}%
\pgfpathcurveto{\pgfqpoint{0.533512in}{0.857665in}}{\pgfqpoint{0.530240in}{0.849765in}}{\pgfqpoint{0.530240in}{0.841529in}}%
\pgfpathcurveto{\pgfqpoint{0.530240in}{0.833293in}}{\pgfqpoint{0.533512in}{0.825393in}}{\pgfqpoint{0.539336in}{0.819569in}}%
\pgfpathcurveto{\pgfqpoint{0.545160in}{0.813745in}}{\pgfqpoint{0.553060in}{0.810472in}}{\pgfqpoint{0.561297in}{0.810472in}}%
\pgfpathclose%
\pgfusepath{stroke,fill}%
\end{pgfscope}%
\begin{pgfscope}%
\pgfpathrectangle{\pgfqpoint{0.457963in}{0.528059in}}{\pgfqpoint{6.200000in}{2.285714in}} %
\pgfusepath{clip}%
\pgfsetbuttcap%
\pgfsetroundjoin%
\definecolor{currentfill}{rgb}{0.833333,0.833333,1.000000}%
\pgfsetfillcolor{currentfill}%
\pgfsetlinewidth{1.003750pt}%
\definecolor{currentstroke}{rgb}{0.833333,0.833333,1.000000}%
\pgfsetstrokecolor{currentstroke}%
\pgfsetdash{}{0pt}%
\pgfpathmoveto{\pgfqpoint{0.643963in}{0.784350in}}%
\pgfpathcurveto{\pgfqpoint{0.652200in}{0.784350in}}{\pgfqpoint{0.660100in}{0.787622in}}{\pgfqpoint{0.665924in}{0.793446in}}%
\pgfpathcurveto{\pgfqpoint{0.671748in}{0.799270in}}{\pgfqpoint{0.675020in}{0.807170in}}{\pgfqpoint{0.675020in}{0.815406in}}%
\pgfpathcurveto{\pgfqpoint{0.675020in}{0.823643in}}{\pgfqpoint{0.671748in}{0.831543in}}{\pgfqpoint{0.665924in}{0.837367in}}%
\pgfpathcurveto{\pgfqpoint{0.660100in}{0.843191in}}{\pgfqpoint{0.652200in}{0.846463in}}{\pgfqpoint{0.643963in}{0.846463in}}%
\pgfpathcurveto{\pgfqpoint{0.635727in}{0.846463in}}{\pgfqpoint{0.627827in}{0.843191in}}{\pgfqpoint{0.622003in}{0.837367in}}%
\pgfpathcurveto{\pgfqpoint{0.616179in}{0.831543in}}{\pgfqpoint{0.612907in}{0.823643in}}{\pgfqpoint{0.612907in}{0.815406in}}%
\pgfpathcurveto{\pgfqpoint{0.612907in}{0.807170in}}{\pgfqpoint{0.616179in}{0.799270in}}{\pgfqpoint{0.622003in}{0.793446in}}%
\pgfpathcurveto{\pgfqpoint{0.627827in}{0.787622in}}{\pgfqpoint{0.635727in}{0.784350in}}{\pgfqpoint{0.643963in}{0.784350in}}%
\pgfpathclose%
\pgfusepath{stroke,fill}%
\end{pgfscope}%
\begin{pgfscope}%
\pgfpathrectangle{\pgfqpoint{0.457963in}{0.528059in}}{\pgfqpoint{6.200000in}{2.285714in}} %
\pgfusepath{clip}%
\pgfsetbuttcap%
\pgfsetroundjoin%
\definecolor{currentfill}{rgb}{0.833333,0.833333,1.000000}%
\pgfsetfillcolor{currentfill}%
\pgfsetlinewidth{1.003750pt}%
\definecolor{currentstroke}{rgb}{0.833333,0.833333,1.000000}%
\pgfsetstrokecolor{currentstroke}%
\pgfsetdash{}{0pt}%
\pgfpathmoveto{\pgfqpoint{0.767963in}{0.784350in}}%
\pgfpathcurveto{\pgfqpoint{0.776200in}{0.784350in}}{\pgfqpoint{0.784100in}{0.787622in}}{\pgfqpoint{0.789924in}{0.793446in}}%
\pgfpathcurveto{\pgfqpoint{0.795748in}{0.799270in}}{\pgfqpoint{0.799020in}{0.807170in}}{\pgfqpoint{0.799020in}{0.815406in}}%
\pgfpathcurveto{\pgfqpoint{0.799020in}{0.823643in}}{\pgfqpoint{0.795748in}{0.831543in}}{\pgfqpoint{0.789924in}{0.837367in}}%
\pgfpathcurveto{\pgfqpoint{0.784100in}{0.843191in}}{\pgfqpoint{0.776200in}{0.846463in}}{\pgfqpoint{0.767963in}{0.846463in}}%
\pgfpathcurveto{\pgfqpoint{0.759727in}{0.846463in}}{\pgfqpoint{0.751827in}{0.843191in}}{\pgfqpoint{0.746003in}{0.837367in}}%
\pgfpathcurveto{\pgfqpoint{0.740179in}{0.831543in}}{\pgfqpoint{0.736907in}{0.823643in}}{\pgfqpoint{0.736907in}{0.815406in}}%
\pgfpathcurveto{\pgfqpoint{0.736907in}{0.807170in}}{\pgfqpoint{0.740179in}{0.799270in}}{\pgfqpoint{0.746003in}{0.793446in}}%
\pgfpathcurveto{\pgfqpoint{0.751827in}{0.787622in}}{\pgfqpoint{0.759727in}{0.784350in}}{\pgfqpoint{0.767963in}{0.784350in}}%
\pgfpathclose%
\pgfusepath{stroke,fill}%
\end{pgfscope}%
\begin{pgfscope}%
\pgfpathrectangle{\pgfqpoint{0.457963in}{0.528059in}}{\pgfqpoint{6.200000in}{2.285714in}} %
\pgfusepath{clip}%
\pgfsetbuttcap%
\pgfsetroundjoin%
\definecolor{currentfill}{rgb}{0.833333,0.833333,1.000000}%
\pgfsetfillcolor{currentfill}%
\pgfsetlinewidth{1.003750pt}%
\definecolor{currentstroke}{rgb}{0.833333,0.833333,1.000000}%
\pgfsetstrokecolor{currentstroke}%
\pgfsetdash{}{0pt}%
\pgfpathmoveto{\pgfqpoint{0.860963in}{0.810472in}}%
\pgfpathcurveto{\pgfqpoint{0.869200in}{0.810472in}}{\pgfqpoint{0.877100in}{0.813745in}}{\pgfqpoint{0.882924in}{0.819569in}}%
\pgfpathcurveto{\pgfqpoint{0.888748in}{0.825393in}}{\pgfqpoint{0.892020in}{0.833293in}}{\pgfqpoint{0.892020in}{0.841529in}}%
\pgfpathcurveto{\pgfqpoint{0.892020in}{0.849765in}}{\pgfqpoint{0.888748in}{0.857665in}}{\pgfqpoint{0.882924in}{0.863489in}}%
\pgfpathcurveto{\pgfqpoint{0.877100in}{0.869313in}}{\pgfqpoint{0.869200in}{0.872585in}}{\pgfqpoint{0.860963in}{0.872585in}}%
\pgfpathcurveto{\pgfqpoint{0.852727in}{0.872585in}}{\pgfqpoint{0.844827in}{0.869313in}}{\pgfqpoint{0.839003in}{0.863489in}}%
\pgfpathcurveto{\pgfqpoint{0.833179in}{0.857665in}}{\pgfqpoint{0.829907in}{0.849765in}}{\pgfqpoint{0.829907in}{0.841529in}}%
\pgfpathcurveto{\pgfqpoint{0.829907in}{0.833293in}}{\pgfqpoint{0.833179in}{0.825393in}}{\pgfqpoint{0.839003in}{0.819569in}}%
\pgfpathcurveto{\pgfqpoint{0.844827in}{0.813745in}}{\pgfqpoint{0.852727in}{0.810472in}}{\pgfqpoint{0.860963in}{0.810472in}}%
\pgfpathclose%
\pgfusepath{stroke,fill}%
\end{pgfscope}%
\begin{pgfscope}%
\pgfpathrectangle{\pgfqpoint{0.457963in}{0.528059in}}{\pgfqpoint{6.200000in}{2.285714in}} %
\pgfusepath{clip}%
\pgfsetbuttcap%
\pgfsetroundjoin%
\definecolor{currentfill}{rgb}{0.833333,0.833333,1.000000}%
\pgfsetfillcolor{currentfill}%
\pgfsetlinewidth{1.003750pt}%
\definecolor{currentstroke}{rgb}{0.833333,0.833333,1.000000}%
\pgfsetstrokecolor{currentstroke}%
\pgfsetdash{}{0pt}%
\pgfpathmoveto{\pgfqpoint{0.943630in}{0.758227in}}%
\pgfpathcurveto{\pgfqpoint{0.951866in}{0.758227in}}{\pgfqpoint{0.959766in}{0.761500in}}{\pgfqpoint{0.965590in}{0.767324in}}%
\pgfpathcurveto{\pgfqpoint{0.971414in}{0.773148in}}{\pgfqpoint{0.974686in}{0.781048in}}{\pgfqpoint{0.974686in}{0.789284in}}%
\pgfpathcurveto{\pgfqpoint{0.974686in}{0.797520in}}{\pgfqpoint{0.971414in}{0.805420in}}{\pgfqpoint{0.965590in}{0.811244in}}%
\pgfpathcurveto{\pgfqpoint{0.959766in}{0.817068in}}{\pgfqpoint{0.951866in}{0.820340in}}{\pgfqpoint{0.943630in}{0.820340in}}%
\pgfpathcurveto{\pgfqpoint{0.935394in}{0.820340in}}{\pgfqpoint{0.927494in}{0.817068in}}{\pgfqpoint{0.921670in}{0.811244in}}%
\pgfpathcurveto{\pgfqpoint{0.915846in}{0.805420in}}{\pgfqpoint{0.912574in}{0.797520in}}{\pgfqpoint{0.912574in}{0.789284in}}%
\pgfpathcurveto{\pgfqpoint{0.912574in}{0.781048in}}{\pgfqpoint{0.915846in}{0.773148in}}{\pgfqpoint{0.921670in}{0.767324in}}%
\pgfpathcurveto{\pgfqpoint{0.927494in}{0.761500in}}{\pgfqpoint{0.935394in}{0.758227in}}{\pgfqpoint{0.943630in}{0.758227in}}%
\pgfpathclose%
\pgfusepath{stroke,fill}%
\end{pgfscope}%
\begin{pgfscope}%
\pgfpathrectangle{\pgfqpoint{0.457963in}{0.528059in}}{\pgfqpoint{6.200000in}{2.285714in}} %
\pgfusepath{clip}%
\pgfsetbuttcap%
\pgfsetroundjoin%
\definecolor{currentfill}{rgb}{0.833333,0.833333,1.000000}%
\pgfsetfillcolor{currentfill}%
\pgfsetlinewidth{1.003750pt}%
\definecolor{currentstroke}{rgb}{0.833333,0.833333,1.000000}%
\pgfsetstrokecolor{currentstroke}%
\pgfsetdash{}{0pt}%
\pgfpathmoveto{\pgfqpoint{0.953963in}{0.823534in}}%
\pgfpathcurveto{\pgfqpoint{0.962200in}{0.823534in}}{\pgfqpoint{0.970100in}{0.826806in}}{\pgfqpoint{0.975924in}{0.832630in}}%
\pgfpathcurveto{\pgfqpoint{0.981748in}{0.838454in}}{\pgfqpoint{0.985020in}{0.846354in}}{\pgfqpoint{0.985020in}{0.854590in}}%
\pgfpathcurveto{\pgfqpoint{0.985020in}{0.862826in}}{\pgfqpoint{0.981748in}{0.870726in}}{\pgfqpoint{0.975924in}{0.876550in}}%
\pgfpathcurveto{\pgfqpoint{0.970100in}{0.882374in}}{\pgfqpoint{0.962200in}{0.885647in}}{\pgfqpoint{0.953963in}{0.885647in}}%
\pgfpathcurveto{\pgfqpoint{0.945727in}{0.885647in}}{\pgfqpoint{0.937827in}{0.882374in}}{\pgfqpoint{0.932003in}{0.876550in}}%
\pgfpathcurveto{\pgfqpoint{0.926179in}{0.870726in}}{\pgfqpoint{0.922907in}{0.862826in}}{\pgfqpoint{0.922907in}{0.854590in}}%
\pgfpathcurveto{\pgfqpoint{0.922907in}{0.846354in}}{\pgfqpoint{0.926179in}{0.838454in}}{\pgfqpoint{0.932003in}{0.832630in}}%
\pgfpathcurveto{\pgfqpoint{0.937827in}{0.826806in}}{\pgfqpoint{0.945727in}{0.823534in}}{\pgfqpoint{0.953963in}{0.823534in}}%
\pgfpathclose%
\pgfusepath{stroke,fill}%
\end{pgfscope}%
\begin{pgfscope}%
\pgfpathrectangle{\pgfqpoint{0.457963in}{0.528059in}}{\pgfqpoint{6.200000in}{2.285714in}} %
\pgfusepath{clip}%
\pgfsetbuttcap%
\pgfsetroundjoin%
\definecolor{currentfill}{rgb}{0.833333,0.833333,1.000000}%
\pgfsetfillcolor{currentfill}%
\pgfsetlinewidth{1.003750pt}%
\definecolor{currentstroke}{rgb}{0.833333,0.833333,1.000000}%
\pgfsetstrokecolor{currentstroke}%
\pgfsetdash{}{0pt}%
\pgfpathmoveto{\pgfqpoint{1.098630in}{0.745166in}}%
\pgfpathcurveto{\pgfqpoint{1.106866in}{0.745166in}}{\pgfqpoint{1.114766in}{0.748439in}}{\pgfqpoint{1.120590in}{0.754262in}}%
\pgfpathcurveto{\pgfqpoint{1.126414in}{0.760086in}}{\pgfqpoint{1.129686in}{0.767986in}}{\pgfqpoint{1.129686in}{0.776223in}}%
\pgfpathcurveto{\pgfqpoint{1.129686in}{0.784459in}}{\pgfqpoint{1.126414in}{0.792359in}}{\pgfqpoint{1.120590in}{0.798183in}}%
\pgfpathcurveto{\pgfqpoint{1.114766in}{0.804007in}}{\pgfqpoint{1.106866in}{0.807279in}}{\pgfqpoint{1.098630in}{0.807279in}}%
\pgfpathcurveto{\pgfqpoint{1.090394in}{0.807279in}}{\pgfqpoint{1.082494in}{0.804007in}}{\pgfqpoint{1.076670in}{0.798183in}}%
\pgfpathcurveto{\pgfqpoint{1.070846in}{0.792359in}}{\pgfqpoint{1.067574in}{0.784459in}}{\pgfqpoint{1.067574in}{0.776223in}}%
\pgfpathcurveto{\pgfqpoint{1.067574in}{0.767986in}}{\pgfqpoint{1.070846in}{0.760086in}}{\pgfqpoint{1.076670in}{0.754262in}}%
\pgfpathcurveto{\pgfqpoint{1.082494in}{0.748439in}}{\pgfqpoint{1.090394in}{0.745166in}}{\pgfqpoint{1.098630in}{0.745166in}}%
\pgfpathclose%
\pgfusepath{stroke,fill}%
\end{pgfscope}%
\begin{pgfscope}%
\pgfpathrectangle{\pgfqpoint{0.457963in}{0.528059in}}{\pgfqpoint{6.200000in}{2.285714in}} %
\pgfusepath{clip}%
\pgfsetbuttcap%
\pgfsetroundjoin%
\definecolor{currentfill}{rgb}{0.833333,0.833333,1.000000}%
\pgfsetfillcolor{currentfill}%
\pgfsetlinewidth{1.003750pt}%
\definecolor{currentstroke}{rgb}{0.833333,0.833333,1.000000}%
\pgfsetstrokecolor{currentstroke}%
\pgfsetdash{}{0pt}%
\pgfpathmoveto{\pgfqpoint{1.222630in}{0.823534in}}%
\pgfpathcurveto{\pgfqpoint{1.230866in}{0.823534in}}{\pgfqpoint{1.238766in}{0.826806in}}{\pgfqpoint{1.244590in}{0.832630in}}%
\pgfpathcurveto{\pgfqpoint{1.250414in}{0.838454in}}{\pgfqpoint{1.253686in}{0.846354in}}{\pgfqpoint{1.253686in}{0.854590in}}%
\pgfpathcurveto{\pgfqpoint{1.253686in}{0.862826in}}{\pgfqpoint{1.250414in}{0.870726in}}{\pgfqpoint{1.244590in}{0.876550in}}%
\pgfpathcurveto{\pgfqpoint{1.238766in}{0.882374in}}{\pgfqpoint{1.230866in}{0.885647in}}{\pgfqpoint{1.222630in}{0.885647in}}%
\pgfpathcurveto{\pgfqpoint{1.214394in}{0.885647in}}{\pgfqpoint{1.206494in}{0.882374in}}{\pgfqpoint{1.200670in}{0.876550in}}%
\pgfpathcurveto{\pgfqpoint{1.194846in}{0.870726in}}{\pgfqpoint{1.191574in}{0.862826in}}{\pgfqpoint{1.191574in}{0.854590in}}%
\pgfpathcurveto{\pgfqpoint{1.191574in}{0.846354in}}{\pgfqpoint{1.194846in}{0.838454in}}{\pgfqpoint{1.200670in}{0.832630in}}%
\pgfpathcurveto{\pgfqpoint{1.206494in}{0.826806in}}{\pgfqpoint{1.214394in}{0.823534in}}{\pgfqpoint{1.222630in}{0.823534in}}%
\pgfpathclose%
\pgfusepath{stroke,fill}%
\end{pgfscope}%
\begin{pgfscope}%
\pgfpathrectangle{\pgfqpoint{0.457963in}{0.528059in}}{\pgfqpoint{6.200000in}{2.285714in}} %
\pgfusepath{clip}%
\pgfsetbuttcap%
\pgfsetroundjoin%
\definecolor{currentfill}{rgb}{0.833333,0.833333,1.000000}%
\pgfsetfillcolor{currentfill}%
\pgfsetlinewidth{1.003750pt}%
\definecolor{currentstroke}{rgb}{0.833333,0.833333,1.000000}%
\pgfsetstrokecolor{currentstroke}%
\pgfsetdash{}{0pt}%
\pgfpathmoveto{\pgfqpoint{1.356963in}{0.784350in}}%
\pgfpathcurveto{\pgfqpoint{1.365200in}{0.784350in}}{\pgfqpoint{1.373100in}{0.787622in}}{\pgfqpoint{1.378924in}{0.793446in}}%
\pgfpathcurveto{\pgfqpoint{1.384748in}{0.799270in}}{\pgfqpoint{1.388020in}{0.807170in}}{\pgfqpoint{1.388020in}{0.815406in}}%
\pgfpathcurveto{\pgfqpoint{1.388020in}{0.823643in}}{\pgfqpoint{1.384748in}{0.831543in}}{\pgfqpoint{1.378924in}{0.837367in}}%
\pgfpathcurveto{\pgfqpoint{1.373100in}{0.843191in}}{\pgfqpoint{1.365200in}{0.846463in}}{\pgfqpoint{1.356963in}{0.846463in}}%
\pgfpathcurveto{\pgfqpoint{1.348727in}{0.846463in}}{\pgfqpoint{1.340827in}{0.843191in}}{\pgfqpoint{1.335003in}{0.837367in}}%
\pgfpathcurveto{\pgfqpoint{1.329179in}{0.831543in}}{\pgfqpoint{1.325907in}{0.823643in}}{\pgfqpoint{1.325907in}{0.815406in}}%
\pgfpathcurveto{\pgfqpoint{1.325907in}{0.807170in}}{\pgfqpoint{1.329179in}{0.799270in}}{\pgfqpoint{1.335003in}{0.793446in}}%
\pgfpathcurveto{\pgfqpoint{1.340827in}{0.787622in}}{\pgfqpoint{1.348727in}{0.784350in}}{\pgfqpoint{1.356963in}{0.784350in}}%
\pgfpathclose%
\pgfusepath{stroke,fill}%
\end{pgfscope}%
\begin{pgfscope}%
\pgfpathrectangle{\pgfqpoint{0.457963in}{0.528059in}}{\pgfqpoint{6.200000in}{2.285714in}} %
\pgfusepath{clip}%
\pgfsetbuttcap%
\pgfsetroundjoin%
\definecolor{currentfill}{rgb}{0.833333,0.833333,1.000000}%
\pgfsetfillcolor{currentfill}%
\pgfsetlinewidth{1.003750pt}%
\definecolor{currentstroke}{rgb}{0.833333,0.833333,1.000000}%
\pgfsetstrokecolor{currentstroke}%
\pgfsetdash{}{0pt}%
\pgfpathmoveto{\pgfqpoint{1.449963in}{0.797411in}}%
\pgfpathcurveto{\pgfqpoint{1.458200in}{0.797411in}}{\pgfqpoint{1.466100in}{0.800683in}}{\pgfqpoint{1.471924in}{0.806507in}}%
\pgfpathcurveto{\pgfqpoint{1.477748in}{0.812331in}}{\pgfqpoint{1.481020in}{0.820231in}}{\pgfqpoint{1.481020in}{0.828468in}}%
\pgfpathcurveto{\pgfqpoint{1.481020in}{0.836704in}}{\pgfqpoint{1.477748in}{0.844604in}}{\pgfqpoint{1.471924in}{0.850428in}}%
\pgfpathcurveto{\pgfqpoint{1.466100in}{0.856252in}}{\pgfqpoint{1.458200in}{0.859524in}}{\pgfqpoint{1.449963in}{0.859524in}}%
\pgfpathcurveto{\pgfqpoint{1.441727in}{0.859524in}}{\pgfqpoint{1.433827in}{0.856252in}}{\pgfqpoint{1.428003in}{0.850428in}}%
\pgfpathcurveto{\pgfqpoint{1.422179in}{0.844604in}}{\pgfqpoint{1.418907in}{0.836704in}}{\pgfqpoint{1.418907in}{0.828468in}}%
\pgfpathcurveto{\pgfqpoint{1.418907in}{0.820231in}}{\pgfqpoint{1.422179in}{0.812331in}}{\pgfqpoint{1.428003in}{0.806507in}}%
\pgfpathcurveto{\pgfqpoint{1.433827in}{0.800683in}}{\pgfqpoint{1.441727in}{0.797411in}}{\pgfqpoint{1.449963in}{0.797411in}}%
\pgfpathclose%
\pgfusepath{stroke,fill}%
\end{pgfscope}%
\begin{pgfscope}%
\pgfpathrectangle{\pgfqpoint{0.457963in}{0.528059in}}{\pgfqpoint{6.200000in}{2.285714in}} %
\pgfusepath{clip}%
\pgfsetbuttcap%
\pgfsetroundjoin%
\definecolor{currentfill}{rgb}{0.833333,0.833333,1.000000}%
\pgfsetfillcolor{currentfill}%
\pgfsetlinewidth{1.003750pt}%
\definecolor{currentstroke}{rgb}{0.833333,0.833333,1.000000}%
\pgfsetstrokecolor{currentstroke}%
\pgfsetdash{}{0pt}%
\pgfpathmoveto{\pgfqpoint{1.573963in}{0.745166in}}%
\pgfpathcurveto{\pgfqpoint{1.582200in}{0.745166in}}{\pgfqpoint{1.590100in}{0.748439in}}{\pgfqpoint{1.595924in}{0.754262in}}%
\pgfpathcurveto{\pgfqpoint{1.601748in}{0.760086in}}{\pgfqpoint{1.605020in}{0.767986in}}{\pgfqpoint{1.605020in}{0.776223in}}%
\pgfpathcurveto{\pgfqpoint{1.605020in}{0.784459in}}{\pgfqpoint{1.601748in}{0.792359in}}{\pgfqpoint{1.595924in}{0.798183in}}%
\pgfpathcurveto{\pgfqpoint{1.590100in}{0.804007in}}{\pgfqpoint{1.582200in}{0.807279in}}{\pgfqpoint{1.573963in}{0.807279in}}%
\pgfpathcurveto{\pgfqpoint{1.565727in}{0.807279in}}{\pgfqpoint{1.557827in}{0.804007in}}{\pgfqpoint{1.552003in}{0.798183in}}%
\pgfpathcurveto{\pgfqpoint{1.546179in}{0.792359in}}{\pgfqpoint{1.542907in}{0.784459in}}{\pgfqpoint{1.542907in}{0.776223in}}%
\pgfpathcurveto{\pgfqpoint{1.542907in}{0.767986in}}{\pgfqpoint{1.546179in}{0.760086in}}{\pgfqpoint{1.552003in}{0.754262in}}%
\pgfpathcurveto{\pgfqpoint{1.557827in}{0.748439in}}{\pgfqpoint{1.565727in}{0.745166in}}{\pgfqpoint{1.573963in}{0.745166in}}%
\pgfpathclose%
\pgfusepath{stroke,fill}%
\end{pgfscope}%
\begin{pgfscope}%
\pgfpathrectangle{\pgfqpoint{0.457963in}{0.528059in}}{\pgfqpoint{6.200000in}{2.285714in}} %
\pgfusepath{clip}%
\pgfsetbuttcap%
\pgfsetroundjoin%
\definecolor{currentfill}{rgb}{0.833333,0.833333,1.000000}%
\pgfsetfillcolor{currentfill}%
\pgfsetlinewidth{1.003750pt}%
\definecolor{currentstroke}{rgb}{0.833333,0.833333,1.000000}%
\pgfsetstrokecolor{currentstroke}%
\pgfsetdash{}{0pt}%
\pgfpathmoveto{\pgfqpoint{2.193963in}{0.627615in}}%
\pgfpathcurveto{\pgfqpoint{2.202200in}{0.627615in}}{\pgfqpoint{2.210100in}{0.630887in}}{\pgfqpoint{2.215924in}{0.636711in}}%
\pgfpathcurveto{\pgfqpoint{2.221748in}{0.642535in}}{\pgfqpoint{2.225020in}{0.650435in}}{\pgfqpoint{2.225020in}{0.658672in}}%
\pgfpathcurveto{\pgfqpoint{2.225020in}{0.666908in}}{\pgfqpoint{2.221748in}{0.674808in}}{\pgfqpoint{2.215924in}{0.680632in}}%
\pgfpathcurveto{\pgfqpoint{2.210100in}{0.686456in}}{\pgfqpoint{2.202200in}{0.689728in}}{\pgfqpoint{2.193963in}{0.689728in}}%
\pgfpathcurveto{\pgfqpoint{2.185727in}{0.689728in}}{\pgfqpoint{2.177827in}{0.686456in}}{\pgfqpoint{2.172003in}{0.680632in}}%
\pgfpathcurveto{\pgfqpoint{2.166179in}{0.674808in}}{\pgfqpoint{2.162907in}{0.666908in}}{\pgfqpoint{2.162907in}{0.658672in}}%
\pgfpathcurveto{\pgfqpoint{2.162907in}{0.650435in}}{\pgfqpoint{2.166179in}{0.642535in}}{\pgfqpoint{2.172003in}{0.636711in}}%
\pgfpathcurveto{\pgfqpoint{2.177827in}{0.630887in}}{\pgfqpoint{2.185727in}{0.627615in}}{\pgfqpoint{2.193963in}{0.627615in}}%
\pgfpathclose%
\pgfusepath{stroke,fill}%
\end{pgfscope}%
\begin{pgfscope}%
\pgfpathrectangle{\pgfqpoint{0.457963in}{0.528059in}}{\pgfqpoint{6.200000in}{2.285714in}} %
\pgfusepath{clip}%
\pgfsetbuttcap%
\pgfsetroundjoin%
\definecolor{currentfill}{rgb}{0.666667,0.666667,1.000000}%
\pgfsetfillcolor{currentfill}%
\pgfsetlinewidth{1.003750pt}%
\definecolor{currentstroke}{rgb}{0.666667,0.666667,1.000000}%
\pgfsetstrokecolor{currentstroke}%
\pgfsetdash{}{0pt}%
\pgfpathmoveto{\pgfqpoint{0.457963in}{1.150064in}}%
\pgfpathcurveto{\pgfqpoint{0.466200in}{1.150064in}}{\pgfqpoint{0.474100in}{1.153336in}}{\pgfqpoint{0.479924in}{1.159160in}}%
\pgfpathcurveto{\pgfqpoint{0.485748in}{1.164984in}}{\pgfqpoint{0.489020in}{1.172884in}}{\pgfqpoint{0.489020in}{1.181121in}}%
\pgfpathcurveto{\pgfqpoint{0.489020in}{1.189357in}}{\pgfqpoint{0.485748in}{1.197257in}}{\pgfqpoint{0.479924in}{1.203081in}}%
\pgfpathcurveto{\pgfqpoint{0.474100in}{1.208905in}}{\pgfqpoint{0.466200in}{1.212177in}}{\pgfqpoint{0.457963in}{1.212177in}}%
\pgfpathcurveto{\pgfqpoint{0.449727in}{1.212177in}}{\pgfqpoint{0.441827in}{1.208905in}}{\pgfqpoint{0.436003in}{1.203081in}}%
\pgfpathcurveto{\pgfqpoint{0.430179in}{1.197257in}}{\pgfqpoint{0.426907in}{1.189357in}}{\pgfqpoint{0.426907in}{1.181121in}}%
\pgfpathcurveto{\pgfqpoint{0.426907in}{1.172884in}}{\pgfqpoint{0.430179in}{1.164984in}}{\pgfqpoint{0.436003in}{1.159160in}}%
\pgfpathcurveto{\pgfqpoint{0.441827in}{1.153336in}}{\pgfqpoint{0.449727in}{1.150064in}}{\pgfqpoint{0.457963in}{1.150064in}}%
\pgfpathclose%
\pgfusepath{stroke,fill}%
\end{pgfscope}%
\begin{pgfscope}%
\pgfpathrectangle{\pgfqpoint{0.457963in}{0.528059in}}{\pgfqpoint{6.200000in}{2.285714in}} %
\pgfusepath{clip}%
\pgfsetbuttcap%
\pgfsetroundjoin%
\definecolor{currentfill}{rgb}{0.666667,0.666667,1.000000}%
\pgfsetfillcolor{currentfill}%
\pgfsetlinewidth{1.003750pt}%
\definecolor{currentstroke}{rgb}{0.666667,0.666667,1.000000}%
\pgfsetstrokecolor{currentstroke}%
\pgfsetdash{}{0pt}%
\pgfpathmoveto{\pgfqpoint{0.457963in}{1.150064in}}%
\pgfpathcurveto{\pgfqpoint{0.466200in}{1.150064in}}{\pgfqpoint{0.474100in}{1.153336in}}{\pgfqpoint{0.479924in}{1.159160in}}%
\pgfpathcurveto{\pgfqpoint{0.485748in}{1.164984in}}{\pgfqpoint{0.489020in}{1.172884in}}{\pgfqpoint{0.489020in}{1.181121in}}%
\pgfpathcurveto{\pgfqpoint{0.489020in}{1.189357in}}{\pgfqpoint{0.485748in}{1.197257in}}{\pgfqpoint{0.479924in}{1.203081in}}%
\pgfpathcurveto{\pgfqpoint{0.474100in}{1.208905in}}{\pgfqpoint{0.466200in}{1.212177in}}{\pgfqpoint{0.457963in}{1.212177in}}%
\pgfpathcurveto{\pgfqpoint{0.449727in}{1.212177in}}{\pgfqpoint{0.441827in}{1.208905in}}{\pgfqpoint{0.436003in}{1.203081in}}%
\pgfpathcurveto{\pgfqpoint{0.430179in}{1.197257in}}{\pgfqpoint{0.426907in}{1.189357in}}{\pgfqpoint{0.426907in}{1.181121in}}%
\pgfpathcurveto{\pgfqpoint{0.426907in}{1.172884in}}{\pgfqpoint{0.430179in}{1.164984in}}{\pgfqpoint{0.436003in}{1.159160in}}%
\pgfpathcurveto{\pgfqpoint{0.441827in}{1.153336in}}{\pgfqpoint{0.449727in}{1.150064in}}{\pgfqpoint{0.457963in}{1.150064in}}%
\pgfpathclose%
\pgfusepath{stroke,fill}%
\end{pgfscope}%
\begin{pgfscope}%
\pgfpathrectangle{\pgfqpoint{0.457963in}{0.528059in}}{\pgfqpoint{6.200000in}{2.285714in}} %
\pgfusepath{clip}%
\pgfsetbuttcap%
\pgfsetroundjoin%
\definecolor{currentfill}{rgb}{0.666667,0.666667,1.000000}%
\pgfsetfillcolor{currentfill}%
\pgfsetlinewidth{1.003750pt}%
\definecolor{currentstroke}{rgb}{0.666667,0.666667,1.000000}%
\pgfsetstrokecolor{currentstroke}%
\pgfsetdash{}{0pt}%
\pgfpathmoveto{\pgfqpoint{0.457963in}{1.150064in}}%
\pgfpathcurveto{\pgfqpoint{0.466200in}{1.150064in}}{\pgfqpoint{0.474100in}{1.153336in}}{\pgfqpoint{0.479924in}{1.159160in}}%
\pgfpathcurveto{\pgfqpoint{0.485748in}{1.164984in}}{\pgfqpoint{0.489020in}{1.172884in}}{\pgfqpoint{0.489020in}{1.181121in}}%
\pgfpathcurveto{\pgfqpoint{0.489020in}{1.189357in}}{\pgfqpoint{0.485748in}{1.197257in}}{\pgfqpoint{0.479924in}{1.203081in}}%
\pgfpathcurveto{\pgfqpoint{0.474100in}{1.208905in}}{\pgfqpoint{0.466200in}{1.212177in}}{\pgfqpoint{0.457963in}{1.212177in}}%
\pgfpathcurveto{\pgfqpoint{0.449727in}{1.212177in}}{\pgfqpoint{0.441827in}{1.208905in}}{\pgfqpoint{0.436003in}{1.203081in}}%
\pgfpathcurveto{\pgfqpoint{0.430179in}{1.197257in}}{\pgfqpoint{0.426907in}{1.189357in}}{\pgfqpoint{0.426907in}{1.181121in}}%
\pgfpathcurveto{\pgfqpoint{0.426907in}{1.172884in}}{\pgfqpoint{0.430179in}{1.164984in}}{\pgfqpoint{0.436003in}{1.159160in}}%
\pgfpathcurveto{\pgfqpoint{0.441827in}{1.153336in}}{\pgfqpoint{0.449727in}{1.150064in}}{\pgfqpoint{0.457963in}{1.150064in}}%
\pgfpathclose%
\pgfusepath{stroke,fill}%
\end{pgfscope}%
\begin{pgfscope}%
\pgfpathrectangle{\pgfqpoint{0.457963in}{0.528059in}}{\pgfqpoint{6.200000in}{2.285714in}} %
\pgfusepath{clip}%
\pgfsetbuttcap%
\pgfsetroundjoin%
\definecolor{currentfill}{rgb}{0.666667,0.666667,1.000000}%
\pgfsetfillcolor{currentfill}%
\pgfsetlinewidth{1.003750pt}%
\definecolor{currentstroke}{rgb}{0.666667,0.666667,1.000000}%
\pgfsetstrokecolor{currentstroke}%
\pgfsetdash{}{0pt}%
\pgfpathmoveto{\pgfqpoint{0.457963in}{1.150064in}}%
\pgfpathcurveto{\pgfqpoint{0.466200in}{1.150064in}}{\pgfqpoint{0.474100in}{1.153336in}}{\pgfqpoint{0.479924in}{1.159160in}}%
\pgfpathcurveto{\pgfqpoint{0.485748in}{1.164984in}}{\pgfqpoint{0.489020in}{1.172884in}}{\pgfqpoint{0.489020in}{1.181121in}}%
\pgfpathcurveto{\pgfqpoint{0.489020in}{1.189357in}}{\pgfqpoint{0.485748in}{1.197257in}}{\pgfqpoint{0.479924in}{1.203081in}}%
\pgfpathcurveto{\pgfqpoint{0.474100in}{1.208905in}}{\pgfqpoint{0.466200in}{1.212177in}}{\pgfqpoint{0.457963in}{1.212177in}}%
\pgfpathcurveto{\pgfqpoint{0.449727in}{1.212177in}}{\pgfqpoint{0.441827in}{1.208905in}}{\pgfqpoint{0.436003in}{1.203081in}}%
\pgfpathcurveto{\pgfqpoint{0.430179in}{1.197257in}}{\pgfqpoint{0.426907in}{1.189357in}}{\pgfqpoint{0.426907in}{1.181121in}}%
\pgfpathcurveto{\pgfqpoint{0.426907in}{1.172884in}}{\pgfqpoint{0.430179in}{1.164984in}}{\pgfqpoint{0.436003in}{1.159160in}}%
\pgfpathcurveto{\pgfqpoint{0.441827in}{1.153336in}}{\pgfqpoint{0.449727in}{1.150064in}}{\pgfqpoint{0.457963in}{1.150064in}}%
\pgfpathclose%
\pgfusepath{stroke,fill}%
\end{pgfscope}%
\begin{pgfscope}%
\pgfpathrectangle{\pgfqpoint{0.457963in}{0.528059in}}{\pgfqpoint{6.200000in}{2.285714in}} %
\pgfusepath{clip}%
\pgfsetbuttcap%
\pgfsetroundjoin%
\definecolor{currentfill}{rgb}{0.666667,0.666667,1.000000}%
\pgfsetfillcolor{currentfill}%
\pgfsetlinewidth{1.003750pt}%
\definecolor{currentstroke}{rgb}{0.666667,0.666667,1.000000}%
\pgfsetstrokecolor{currentstroke}%
\pgfsetdash{}{0pt}%
\pgfpathmoveto{\pgfqpoint{0.468297in}{1.150064in}}%
\pgfpathcurveto{\pgfqpoint{0.476533in}{1.150064in}}{\pgfqpoint{0.484433in}{1.153336in}}{\pgfqpoint{0.490257in}{1.159160in}}%
\pgfpathcurveto{\pgfqpoint{0.496081in}{1.164984in}}{\pgfqpoint{0.499353in}{1.172884in}}{\pgfqpoint{0.499353in}{1.181121in}}%
\pgfpathcurveto{\pgfqpoint{0.499353in}{1.189357in}}{\pgfqpoint{0.496081in}{1.197257in}}{\pgfqpoint{0.490257in}{1.203081in}}%
\pgfpathcurveto{\pgfqpoint{0.484433in}{1.208905in}}{\pgfqpoint{0.476533in}{1.212177in}}{\pgfqpoint{0.468297in}{1.212177in}}%
\pgfpathcurveto{\pgfqpoint{0.460060in}{1.212177in}}{\pgfqpoint{0.452160in}{1.208905in}}{\pgfqpoint{0.446336in}{1.203081in}}%
\pgfpathcurveto{\pgfqpoint{0.440512in}{1.197257in}}{\pgfqpoint{0.437240in}{1.189357in}}{\pgfqpoint{0.437240in}{1.181121in}}%
\pgfpathcurveto{\pgfqpoint{0.437240in}{1.172884in}}{\pgfqpoint{0.440512in}{1.164984in}}{\pgfqpoint{0.446336in}{1.159160in}}%
\pgfpathcurveto{\pgfqpoint{0.452160in}{1.153336in}}{\pgfqpoint{0.460060in}{1.150064in}}{\pgfqpoint{0.468297in}{1.150064in}}%
\pgfpathclose%
\pgfusepath{stroke,fill}%
\end{pgfscope}%
\begin{pgfscope}%
\pgfpathrectangle{\pgfqpoint{0.457963in}{0.528059in}}{\pgfqpoint{6.200000in}{2.285714in}} %
\pgfusepath{clip}%
\pgfsetbuttcap%
\pgfsetroundjoin%
\definecolor{currentfill}{rgb}{0.666667,0.666667,1.000000}%
\pgfsetfillcolor{currentfill}%
\pgfsetlinewidth{1.003750pt}%
\definecolor{currentstroke}{rgb}{0.666667,0.666667,1.000000}%
\pgfsetstrokecolor{currentstroke}%
\pgfsetdash{}{0pt}%
\pgfpathmoveto{\pgfqpoint{0.488963in}{1.150064in}}%
\pgfpathcurveto{\pgfqpoint{0.497200in}{1.150064in}}{\pgfqpoint{0.505100in}{1.153336in}}{\pgfqpoint{0.510924in}{1.159160in}}%
\pgfpathcurveto{\pgfqpoint{0.516748in}{1.164984in}}{\pgfqpoint{0.520020in}{1.172884in}}{\pgfqpoint{0.520020in}{1.181121in}}%
\pgfpathcurveto{\pgfqpoint{0.520020in}{1.189357in}}{\pgfqpoint{0.516748in}{1.197257in}}{\pgfqpoint{0.510924in}{1.203081in}}%
\pgfpathcurveto{\pgfqpoint{0.505100in}{1.208905in}}{\pgfqpoint{0.497200in}{1.212177in}}{\pgfqpoint{0.488963in}{1.212177in}}%
\pgfpathcurveto{\pgfqpoint{0.480727in}{1.212177in}}{\pgfqpoint{0.472827in}{1.208905in}}{\pgfqpoint{0.467003in}{1.203081in}}%
\pgfpathcurveto{\pgfqpoint{0.461179in}{1.197257in}}{\pgfqpoint{0.457907in}{1.189357in}}{\pgfqpoint{0.457907in}{1.181121in}}%
\pgfpathcurveto{\pgfqpoint{0.457907in}{1.172884in}}{\pgfqpoint{0.461179in}{1.164984in}}{\pgfqpoint{0.467003in}{1.159160in}}%
\pgfpathcurveto{\pgfqpoint{0.472827in}{1.153336in}}{\pgfqpoint{0.480727in}{1.150064in}}{\pgfqpoint{0.488963in}{1.150064in}}%
\pgfpathclose%
\pgfusepath{stroke,fill}%
\end{pgfscope}%
\begin{pgfscope}%
\pgfpathrectangle{\pgfqpoint{0.457963in}{0.528059in}}{\pgfqpoint{6.200000in}{2.285714in}} %
\pgfusepath{clip}%
\pgfsetbuttcap%
\pgfsetroundjoin%
\definecolor{currentfill}{rgb}{0.666667,0.666667,1.000000}%
\pgfsetfillcolor{currentfill}%
\pgfsetlinewidth{1.003750pt}%
\definecolor{currentstroke}{rgb}{0.666667,0.666667,1.000000}%
\pgfsetstrokecolor{currentstroke}%
\pgfsetdash{}{0pt}%
\pgfpathmoveto{\pgfqpoint{0.519963in}{1.150064in}}%
\pgfpathcurveto{\pgfqpoint{0.528200in}{1.150064in}}{\pgfqpoint{0.536100in}{1.153336in}}{\pgfqpoint{0.541924in}{1.159160in}}%
\pgfpathcurveto{\pgfqpoint{0.547748in}{1.164984in}}{\pgfqpoint{0.551020in}{1.172884in}}{\pgfqpoint{0.551020in}{1.181121in}}%
\pgfpathcurveto{\pgfqpoint{0.551020in}{1.189357in}}{\pgfqpoint{0.547748in}{1.197257in}}{\pgfqpoint{0.541924in}{1.203081in}}%
\pgfpathcurveto{\pgfqpoint{0.536100in}{1.208905in}}{\pgfqpoint{0.528200in}{1.212177in}}{\pgfqpoint{0.519963in}{1.212177in}}%
\pgfpathcurveto{\pgfqpoint{0.511727in}{1.212177in}}{\pgfqpoint{0.503827in}{1.208905in}}{\pgfqpoint{0.498003in}{1.203081in}}%
\pgfpathcurveto{\pgfqpoint{0.492179in}{1.197257in}}{\pgfqpoint{0.488907in}{1.189357in}}{\pgfqpoint{0.488907in}{1.181121in}}%
\pgfpathcurveto{\pgfqpoint{0.488907in}{1.172884in}}{\pgfqpoint{0.492179in}{1.164984in}}{\pgfqpoint{0.498003in}{1.159160in}}%
\pgfpathcurveto{\pgfqpoint{0.503827in}{1.153336in}}{\pgfqpoint{0.511727in}{1.150064in}}{\pgfqpoint{0.519963in}{1.150064in}}%
\pgfpathclose%
\pgfusepath{stroke,fill}%
\end{pgfscope}%
\begin{pgfscope}%
\pgfpathrectangle{\pgfqpoint{0.457963in}{0.528059in}}{\pgfqpoint{6.200000in}{2.285714in}} %
\pgfusepath{clip}%
\pgfsetbuttcap%
\pgfsetroundjoin%
\definecolor{currentfill}{rgb}{0.666667,0.666667,1.000000}%
\pgfsetfillcolor{currentfill}%
\pgfsetlinewidth{1.003750pt}%
\definecolor{currentstroke}{rgb}{0.666667,0.666667,1.000000}%
\pgfsetstrokecolor{currentstroke}%
\pgfsetdash{}{0pt}%
\pgfpathmoveto{\pgfqpoint{0.540630in}{1.150064in}}%
\pgfpathcurveto{\pgfqpoint{0.548866in}{1.150064in}}{\pgfqpoint{0.556766in}{1.153336in}}{\pgfqpoint{0.562590in}{1.159160in}}%
\pgfpathcurveto{\pgfqpoint{0.568414in}{1.164984in}}{\pgfqpoint{0.571686in}{1.172884in}}{\pgfqpoint{0.571686in}{1.181121in}}%
\pgfpathcurveto{\pgfqpoint{0.571686in}{1.189357in}}{\pgfqpoint{0.568414in}{1.197257in}}{\pgfqpoint{0.562590in}{1.203081in}}%
\pgfpathcurveto{\pgfqpoint{0.556766in}{1.208905in}}{\pgfqpoint{0.548866in}{1.212177in}}{\pgfqpoint{0.540630in}{1.212177in}}%
\pgfpathcurveto{\pgfqpoint{0.532394in}{1.212177in}}{\pgfqpoint{0.524494in}{1.208905in}}{\pgfqpoint{0.518670in}{1.203081in}}%
\pgfpathcurveto{\pgfqpoint{0.512846in}{1.197257in}}{\pgfqpoint{0.509574in}{1.189357in}}{\pgfqpoint{0.509574in}{1.181121in}}%
\pgfpathcurveto{\pgfqpoint{0.509574in}{1.172884in}}{\pgfqpoint{0.512846in}{1.164984in}}{\pgfqpoint{0.518670in}{1.159160in}}%
\pgfpathcurveto{\pgfqpoint{0.524494in}{1.153336in}}{\pgfqpoint{0.532394in}{1.150064in}}{\pgfqpoint{0.540630in}{1.150064in}}%
\pgfpathclose%
\pgfusepath{stroke,fill}%
\end{pgfscope}%
\begin{pgfscope}%
\pgfpathrectangle{\pgfqpoint{0.457963in}{0.528059in}}{\pgfqpoint{6.200000in}{2.285714in}} %
\pgfusepath{clip}%
\pgfsetbuttcap%
\pgfsetroundjoin%
\definecolor{currentfill}{rgb}{0.666667,0.666667,1.000000}%
\pgfsetfillcolor{currentfill}%
\pgfsetlinewidth{1.003750pt}%
\definecolor{currentstroke}{rgb}{0.666667,0.666667,1.000000}%
\pgfsetstrokecolor{currentstroke}%
\pgfsetdash{}{0pt}%
\pgfpathmoveto{\pgfqpoint{0.561297in}{1.150064in}}%
\pgfpathcurveto{\pgfqpoint{0.569533in}{1.150064in}}{\pgfqpoint{0.577433in}{1.153336in}}{\pgfqpoint{0.583257in}{1.159160in}}%
\pgfpathcurveto{\pgfqpoint{0.589081in}{1.164984in}}{\pgfqpoint{0.592353in}{1.172884in}}{\pgfqpoint{0.592353in}{1.181121in}}%
\pgfpathcurveto{\pgfqpoint{0.592353in}{1.189357in}}{\pgfqpoint{0.589081in}{1.197257in}}{\pgfqpoint{0.583257in}{1.203081in}}%
\pgfpathcurveto{\pgfqpoint{0.577433in}{1.208905in}}{\pgfqpoint{0.569533in}{1.212177in}}{\pgfqpoint{0.561297in}{1.212177in}}%
\pgfpathcurveto{\pgfqpoint{0.553060in}{1.212177in}}{\pgfqpoint{0.545160in}{1.208905in}}{\pgfqpoint{0.539336in}{1.203081in}}%
\pgfpathcurveto{\pgfqpoint{0.533512in}{1.197257in}}{\pgfqpoint{0.530240in}{1.189357in}}{\pgfqpoint{0.530240in}{1.181121in}}%
\pgfpathcurveto{\pgfqpoint{0.530240in}{1.172884in}}{\pgfqpoint{0.533512in}{1.164984in}}{\pgfqpoint{0.539336in}{1.159160in}}%
\pgfpathcurveto{\pgfqpoint{0.545160in}{1.153336in}}{\pgfqpoint{0.553060in}{1.150064in}}{\pgfqpoint{0.561297in}{1.150064in}}%
\pgfpathclose%
\pgfusepath{stroke,fill}%
\end{pgfscope}%
\begin{pgfscope}%
\pgfpathrectangle{\pgfqpoint{0.457963in}{0.528059in}}{\pgfqpoint{6.200000in}{2.285714in}} %
\pgfusepath{clip}%
\pgfsetbuttcap%
\pgfsetroundjoin%
\definecolor{currentfill}{rgb}{0.666667,0.666667,1.000000}%
\pgfsetfillcolor{currentfill}%
\pgfsetlinewidth{1.003750pt}%
\definecolor{currentstroke}{rgb}{0.666667,0.666667,1.000000}%
\pgfsetstrokecolor{currentstroke}%
\pgfsetdash{}{0pt}%
\pgfpathmoveto{\pgfqpoint{0.674963in}{1.150064in}}%
\pgfpathcurveto{\pgfqpoint{0.683200in}{1.150064in}}{\pgfqpoint{0.691100in}{1.153336in}}{\pgfqpoint{0.696924in}{1.159160in}}%
\pgfpathcurveto{\pgfqpoint{0.702748in}{1.164984in}}{\pgfqpoint{0.706020in}{1.172884in}}{\pgfqpoint{0.706020in}{1.181121in}}%
\pgfpathcurveto{\pgfqpoint{0.706020in}{1.189357in}}{\pgfqpoint{0.702748in}{1.197257in}}{\pgfqpoint{0.696924in}{1.203081in}}%
\pgfpathcurveto{\pgfqpoint{0.691100in}{1.208905in}}{\pgfqpoint{0.683200in}{1.212177in}}{\pgfqpoint{0.674963in}{1.212177in}}%
\pgfpathcurveto{\pgfqpoint{0.666727in}{1.212177in}}{\pgfqpoint{0.658827in}{1.208905in}}{\pgfqpoint{0.653003in}{1.203081in}}%
\pgfpathcurveto{\pgfqpoint{0.647179in}{1.197257in}}{\pgfqpoint{0.643907in}{1.189357in}}{\pgfqpoint{0.643907in}{1.181121in}}%
\pgfpathcurveto{\pgfqpoint{0.643907in}{1.172884in}}{\pgfqpoint{0.647179in}{1.164984in}}{\pgfqpoint{0.653003in}{1.159160in}}%
\pgfpathcurveto{\pgfqpoint{0.658827in}{1.153336in}}{\pgfqpoint{0.666727in}{1.150064in}}{\pgfqpoint{0.674963in}{1.150064in}}%
\pgfpathclose%
\pgfusepath{stroke,fill}%
\end{pgfscope}%
\begin{pgfscope}%
\pgfpathrectangle{\pgfqpoint{0.457963in}{0.528059in}}{\pgfqpoint{6.200000in}{2.285714in}} %
\pgfusepath{clip}%
\pgfsetbuttcap%
\pgfsetroundjoin%
\definecolor{currentfill}{rgb}{0.666667,0.666667,1.000000}%
\pgfsetfillcolor{currentfill}%
\pgfsetlinewidth{1.003750pt}%
\definecolor{currentstroke}{rgb}{0.666667,0.666667,1.000000}%
\pgfsetstrokecolor{currentstroke}%
\pgfsetdash{}{0pt}%
\pgfpathmoveto{\pgfqpoint{0.695630in}{1.150064in}}%
\pgfpathcurveto{\pgfqpoint{0.703866in}{1.150064in}}{\pgfqpoint{0.711766in}{1.153336in}}{\pgfqpoint{0.717590in}{1.159160in}}%
\pgfpathcurveto{\pgfqpoint{0.723414in}{1.164984in}}{\pgfqpoint{0.726686in}{1.172884in}}{\pgfqpoint{0.726686in}{1.181121in}}%
\pgfpathcurveto{\pgfqpoint{0.726686in}{1.189357in}}{\pgfqpoint{0.723414in}{1.197257in}}{\pgfqpoint{0.717590in}{1.203081in}}%
\pgfpathcurveto{\pgfqpoint{0.711766in}{1.208905in}}{\pgfqpoint{0.703866in}{1.212177in}}{\pgfqpoint{0.695630in}{1.212177in}}%
\pgfpathcurveto{\pgfqpoint{0.687394in}{1.212177in}}{\pgfqpoint{0.679494in}{1.208905in}}{\pgfqpoint{0.673670in}{1.203081in}}%
\pgfpathcurveto{\pgfqpoint{0.667846in}{1.197257in}}{\pgfqpoint{0.664574in}{1.189357in}}{\pgfqpoint{0.664574in}{1.181121in}}%
\pgfpathcurveto{\pgfqpoint{0.664574in}{1.172884in}}{\pgfqpoint{0.667846in}{1.164984in}}{\pgfqpoint{0.673670in}{1.159160in}}%
\pgfpathcurveto{\pgfqpoint{0.679494in}{1.153336in}}{\pgfqpoint{0.687394in}{1.150064in}}{\pgfqpoint{0.695630in}{1.150064in}}%
\pgfpathclose%
\pgfusepath{stroke,fill}%
\end{pgfscope}%
\begin{pgfscope}%
\pgfpathrectangle{\pgfqpoint{0.457963in}{0.528059in}}{\pgfqpoint{6.200000in}{2.285714in}} %
\pgfusepath{clip}%
\pgfsetbuttcap%
\pgfsetroundjoin%
\definecolor{currentfill}{rgb}{0.666667,0.666667,1.000000}%
\pgfsetfillcolor{currentfill}%
\pgfsetlinewidth{1.003750pt}%
\definecolor{currentstroke}{rgb}{0.666667,0.666667,1.000000}%
\pgfsetstrokecolor{currentstroke}%
\pgfsetdash{}{0pt}%
\pgfpathmoveto{\pgfqpoint{0.726630in}{1.123942in}}%
\pgfpathcurveto{\pgfqpoint{0.734866in}{1.123942in}}{\pgfqpoint{0.742766in}{1.127214in}}{\pgfqpoint{0.748590in}{1.133038in}}%
\pgfpathcurveto{\pgfqpoint{0.754414in}{1.138862in}}{\pgfqpoint{0.757686in}{1.146762in}}{\pgfqpoint{0.757686in}{1.154998in}}%
\pgfpathcurveto{\pgfqpoint{0.757686in}{1.163234in}}{\pgfqpoint{0.754414in}{1.171135in}}{\pgfqpoint{0.748590in}{1.176958in}}%
\pgfpathcurveto{\pgfqpoint{0.742766in}{1.182782in}}{\pgfqpoint{0.734866in}{1.186055in}}{\pgfqpoint{0.726630in}{1.186055in}}%
\pgfpathcurveto{\pgfqpoint{0.718394in}{1.186055in}}{\pgfqpoint{0.710494in}{1.182782in}}{\pgfqpoint{0.704670in}{1.176958in}}%
\pgfpathcurveto{\pgfqpoint{0.698846in}{1.171135in}}{\pgfqpoint{0.695574in}{1.163234in}}{\pgfqpoint{0.695574in}{1.154998in}}%
\pgfpathcurveto{\pgfqpoint{0.695574in}{1.146762in}}{\pgfqpoint{0.698846in}{1.138862in}}{\pgfqpoint{0.704670in}{1.133038in}}%
\pgfpathcurveto{\pgfqpoint{0.710494in}{1.127214in}}{\pgfqpoint{0.718394in}{1.123942in}}{\pgfqpoint{0.726630in}{1.123942in}}%
\pgfpathclose%
\pgfusepath{stroke,fill}%
\end{pgfscope}%
\begin{pgfscope}%
\pgfpathrectangle{\pgfqpoint{0.457963in}{0.528059in}}{\pgfqpoint{6.200000in}{2.285714in}} %
\pgfusepath{clip}%
\pgfsetbuttcap%
\pgfsetroundjoin%
\definecolor{currentfill}{rgb}{0.666667,0.666667,1.000000}%
\pgfsetfillcolor{currentfill}%
\pgfsetlinewidth{1.003750pt}%
\definecolor{currentstroke}{rgb}{0.666667,0.666667,1.000000}%
\pgfsetstrokecolor{currentstroke}%
\pgfsetdash{}{0pt}%
\pgfpathmoveto{\pgfqpoint{1.098630in}{1.071697in}}%
\pgfpathcurveto{\pgfqpoint{1.106866in}{1.071697in}}{\pgfqpoint{1.114766in}{1.074969in}}{\pgfqpoint{1.120590in}{1.080793in}}%
\pgfpathcurveto{\pgfqpoint{1.126414in}{1.086617in}}{\pgfqpoint{1.129686in}{1.094517in}}{\pgfqpoint{1.129686in}{1.102753in}}%
\pgfpathcurveto{\pgfqpoint{1.129686in}{1.110990in}}{\pgfqpoint{1.126414in}{1.118890in}}{\pgfqpoint{1.120590in}{1.124714in}}%
\pgfpathcurveto{\pgfqpoint{1.114766in}{1.130538in}}{\pgfqpoint{1.106866in}{1.133810in}}{\pgfqpoint{1.098630in}{1.133810in}}%
\pgfpathcurveto{\pgfqpoint{1.090394in}{1.133810in}}{\pgfqpoint{1.082494in}{1.130538in}}{\pgfqpoint{1.076670in}{1.124714in}}%
\pgfpathcurveto{\pgfqpoint{1.070846in}{1.118890in}}{\pgfqpoint{1.067574in}{1.110990in}}{\pgfqpoint{1.067574in}{1.102753in}}%
\pgfpathcurveto{\pgfqpoint{1.067574in}{1.094517in}}{\pgfqpoint{1.070846in}{1.086617in}}{\pgfqpoint{1.076670in}{1.080793in}}%
\pgfpathcurveto{\pgfqpoint{1.082494in}{1.074969in}}{\pgfqpoint{1.090394in}{1.071697in}}{\pgfqpoint{1.098630in}{1.071697in}}%
\pgfpathclose%
\pgfusepath{stroke,fill}%
\end{pgfscope}%
\begin{pgfscope}%
\pgfpathrectangle{\pgfqpoint{0.457963in}{0.528059in}}{\pgfqpoint{6.200000in}{2.285714in}} %
\pgfusepath{clip}%
\pgfsetbuttcap%
\pgfsetroundjoin%
\definecolor{currentfill}{rgb}{0.666667,0.666667,1.000000}%
\pgfsetfillcolor{currentfill}%
\pgfsetlinewidth{1.003750pt}%
\definecolor{currentstroke}{rgb}{0.666667,0.666667,1.000000}%
\pgfsetstrokecolor{currentstroke}%
\pgfsetdash{}{0pt}%
\pgfpathmoveto{\pgfqpoint{1.325963in}{1.137003in}}%
\pgfpathcurveto{\pgfqpoint{1.334200in}{1.137003in}}{\pgfqpoint{1.342100in}{1.140275in}}{\pgfqpoint{1.347924in}{1.146099in}}%
\pgfpathcurveto{\pgfqpoint{1.353748in}{1.151923in}}{\pgfqpoint{1.357020in}{1.159823in}}{\pgfqpoint{1.357020in}{1.168059in}}%
\pgfpathcurveto{\pgfqpoint{1.357020in}{1.176296in}}{\pgfqpoint{1.353748in}{1.184196in}}{\pgfqpoint{1.347924in}{1.190020in}}%
\pgfpathcurveto{\pgfqpoint{1.342100in}{1.195844in}}{\pgfqpoint{1.334200in}{1.199116in}}{\pgfqpoint{1.325963in}{1.199116in}}%
\pgfpathcurveto{\pgfqpoint{1.317727in}{1.199116in}}{\pgfqpoint{1.309827in}{1.195844in}}{\pgfqpoint{1.304003in}{1.190020in}}%
\pgfpathcurveto{\pgfqpoint{1.298179in}{1.184196in}}{\pgfqpoint{1.294907in}{1.176296in}}{\pgfqpoint{1.294907in}{1.168059in}}%
\pgfpathcurveto{\pgfqpoint{1.294907in}{1.159823in}}{\pgfqpoint{1.298179in}{1.151923in}}{\pgfqpoint{1.304003in}{1.146099in}}%
\pgfpathcurveto{\pgfqpoint{1.309827in}{1.140275in}}{\pgfqpoint{1.317727in}{1.137003in}}{\pgfqpoint{1.325963in}{1.137003in}}%
\pgfpathclose%
\pgfusepath{stroke,fill}%
\end{pgfscope}%
\begin{pgfscope}%
\pgfpathrectangle{\pgfqpoint{0.457963in}{0.528059in}}{\pgfqpoint{6.200000in}{2.285714in}} %
\pgfusepath{clip}%
\pgfsetbuttcap%
\pgfsetroundjoin%
\definecolor{currentfill}{rgb}{0.666667,0.666667,1.000000}%
\pgfsetfillcolor{currentfill}%
\pgfsetlinewidth{1.003750pt}%
\definecolor{currentstroke}{rgb}{0.666667,0.666667,1.000000}%
\pgfsetstrokecolor{currentstroke}%
\pgfsetdash{}{0pt}%
\pgfpathmoveto{\pgfqpoint{1.925297in}{1.150064in}}%
\pgfpathcurveto{\pgfqpoint{1.933533in}{1.150064in}}{\pgfqpoint{1.941433in}{1.153336in}}{\pgfqpoint{1.947257in}{1.159160in}}%
\pgfpathcurveto{\pgfqpoint{1.953081in}{1.164984in}}{\pgfqpoint{1.956353in}{1.172884in}}{\pgfqpoint{1.956353in}{1.181121in}}%
\pgfpathcurveto{\pgfqpoint{1.956353in}{1.189357in}}{\pgfqpoint{1.953081in}{1.197257in}}{\pgfqpoint{1.947257in}{1.203081in}}%
\pgfpathcurveto{\pgfqpoint{1.941433in}{1.208905in}}{\pgfqpoint{1.933533in}{1.212177in}}{\pgfqpoint{1.925297in}{1.212177in}}%
\pgfpathcurveto{\pgfqpoint{1.917060in}{1.212177in}}{\pgfqpoint{1.909160in}{1.208905in}}{\pgfqpoint{1.903336in}{1.203081in}}%
\pgfpathcurveto{\pgfqpoint{1.897512in}{1.197257in}}{\pgfqpoint{1.894240in}{1.189357in}}{\pgfqpoint{1.894240in}{1.181121in}}%
\pgfpathcurveto{\pgfqpoint{1.894240in}{1.172884in}}{\pgfqpoint{1.897512in}{1.164984in}}{\pgfqpoint{1.903336in}{1.159160in}}%
\pgfpathcurveto{\pgfqpoint{1.909160in}{1.153336in}}{\pgfqpoint{1.917060in}{1.150064in}}{\pgfqpoint{1.925297in}{1.150064in}}%
\pgfpathclose%
\pgfusepath{stroke,fill}%
\end{pgfscope}%
\begin{pgfscope}%
\pgfpathrectangle{\pgfqpoint{0.457963in}{0.528059in}}{\pgfqpoint{6.200000in}{2.285714in}} %
\pgfusepath{clip}%
\pgfsetbuttcap%
\pgfsetroundjoin%
\definecolor{currentfill}{rgb}{0.666667,0.666667,1.000000}%
\pgfsetfillcolor{currentfill}%
\pgfsetlinewidth{1.003750pt}%
\definecolor{currentstroke}{rgb}{0.666667,0.666667,1.000000}%
\pgfsetstrokecolor{currentstroke}%
\pgfsetdash{}{0pt}%
\pgfpathmoveto{\pgfqpoint{1.935630in}{0.914962in}}%
\pgfpathcurveto{\pgfqpoint{1.943866in}{0.914962in}}{\pgfqpoint{1.951766in}{0.918234in}}{\pgfqpoint{1.957590in}{0.924058in}}%
\pgfpathcurveto{\pgfqpoint{1.963414in}{0.929882in}}{\pgfqpoint{1.966686in}{0.937782in}}{\pgfqpoint{1.966686in}{0.946019in}}%
\pgfpathcurveto{\pgfqpoint{1.966686in}{0.954255in}}{\pgfqpoint{1.963414in}{0.962155in}}{\pgfqpoint{1.957590in}{0.967979in}}%
\pgfpathcurveto{\pgfqpoint{1.951766in}{0.973803in}}{\pgfqpoint{1.943866in}{0.977075in}}{\pgfqpoint{1.935630in}{0.977075in}}%
\pgfpathcurveto{\pgfqpoint{1.927394in}{0.977075in}}{\pgfqpoint{1.919494in}{0.973803in}}{\pgfqpoint{1.913670in}{0.967979in}}%
\pgfpathcurveto{\pgfqpoint{1.907846in}{0.962155in}}{\pgfqpoint{1.904574in}{0.954255in}}{\pgfqpoint{1.904574in}{0.946019in}}%
\pgfpathcurveto{\pgfqpoint{1.904574in}{0.937782in}}{\pgfqpoint{1.907846in}{0.929882in}}{\pgfqpoint{1.913670in}{0.924058in}}%
\pgfpathcurveto{\pgfqpoint{1.919494in}{0.918234in}}{\pgfqpoint{1.927394in}{0.914962in}}{\pgfqpoint{1.935630in}{0.914962in}}%
\pgfpathclose%
\pgfusepath{stroke,fill}%
\end{pgfscope}%
\begin{pgfscope}%
\pgfpathrectangle{\pgfqpoint{0.457963in}{0.528059in}}{\pgfqpoint{6.200000in}{2.285714in}} %
\pgfusepath{clip}%
\pgfsetbuttcap%
\pgfsetroundjoin%
\definecolor{currentfill}{rgb}{0.666667,0.666667,1.000000}%
\pgfsetfillcolor{currentfill}%
\pgfsetlinewidth{1.003750pt}%
\definecolor{currentstroke}{rgb}{0.666667,0.666667,1.000000}%
\pgfsetstrokecolor{currentstroke}%
\pgfsetdash{}{0pt}%
\pgfpathmoveto{\pgfqpoint{2.772630in}{1.123942in}}%
\pgfpathcurveto{\pgfqpoint{2.780866in}{1.123942in}}{\pgfqpoint{2.788766in}{1.127214in}}{\pgfqpoint{2.794590in}{1.133038in}}%
\pgfpathcurveto{\pgfqpoint{2.800414in}{1.138862in}}{\pgfqpoint{2.803686in}{1.146762in}}{\pgfqpoint{2.803686in}{1.154998in}}%
\pgfpathcurveto{\pgfqpoint{2.803686in}{1.163234in}}{\pgfqpoint{2.800414in}{1.171135in}}{\pgfqpoint{2.794590in}{1.176958in}}%
\pgfpathcurveto{\pgfqpoint{2.788766in}{1.182782in}}{\pgfqpoint{2.780866in}{1.186055in}}{\pgfqpoint{2.772630in}{1.186055in}}%
\pgfpathcurveto{\pgfqpoint{2.764394in}{1.186055in}}{\pgfqpoint{2.756494in}{1.182782in}}{\pgfqpoint{2.750670in}{1.176958in}}%
\pgfpathcurveto{\pgfqpoint{2.744846in}{1.171135in}}{\pgfqpoint{2.741574in}{1.163234in}}{\pgfqpoint{2.741574in}{1.154998in}}%
\pgfpathcurveto{\pgfqpoint{2.741574in}{1.146762in}}{\pgfqpoint{2.744846in}{1.138862in}}{\pgfqpoint{2.750670in}{1.133038in}}%
\pgfpathcurveto{\pgfqpoint{2.756494in}{1.127214in}}{\pgfqpoint{2.764394in}{1.123942in}}{\pgfqpoint{2.772630in}{1.123942in}}%
\pgfpathclose%
\pgfusepath{stroke,fill}%
\end{pgfscope}%
\begin{pgfscope}%
\pgfpathrectangle{\pgfqpoint{0.457963in}{0.528059in}}{\pgfqpoint{6.200000in}{2.285714in}} %
\pgfusepath{clip}%
\pgfsetbuttcap%
\pgfsetroundjoin%
\definecolor{currentfill}{rgb}{0.666667,0.666667,1.000000}%
\pgfsetfillcolor{currentfill}%
\pgfsetlinewidth{1.003750pt}%
\definecolor{currentstroke}{rgb}{0.666667,0.666667,1.000000}%
\pgfsetstrokecolor{currentstroke}%
\pgfsetdash{}{0pt}%
\pgfpathmoveto{\pgfqpoint{2.782963in}{1.071697in}}%
\pgfpathcurveto{\pgfqpoint{2.791200in}{1.071697in}}{\pgfqpoint{2.799100in}{1.074969in}}{\pgfqpoint{2.804924in}{1.080793in}}%
\pgfpathcurveto{\pgfqpoint{2.810748in}{1.086617in}}{\pgfqpoint{2.814020in}{1.094517in}}{\pgfqpoint{2.814020in}{1.102753in}}%
\pgfpathcurveto{\pgfqpoint{2.814020in}{1.110990in}}{\pgfqpoint{2.810748in}{1.118890in}}{\pgfqpoint{2.804924in}{1.124714in}}%
\pgfpathcurveto{\pgfqpoint{2.799100in}{1.130538in}}{\pgfqpoint{2.791200in}{1.133810in}}{\pgfqpoint{2.782963in}{1.133810in}}%
\pgfpathcurveto{\pgfqpoint{2.774727in}{1.133810in}}{\pgfqpoint{2.766827in}{1.130538in}}{\pgfqpoint{2.761003in}{1.124714in}}%
\pgfpathcurveto{\pgfqpoint{2.755179in}{1.118890in}}{\pgfqpoint{2.751907in}{1.110990in}}{\pgfqpoint{2.751907in}{1.102753in}}%
\pgfpathcurveto{\pgfqpoint{2.751907in}{1.094517in}}{\pgfqpoint{2.755179in}{1.086617in}}{\pgfqpoint{2.761003in}{1.080793in}}%
\pgfpathcurveto{\pgfqpoint{2.766827in}{1.074969in}}{\pgfqpoint{2.774727in}{1.071697in}}{\pgfqpoint{2.782963in}{1.071697in}}%
\pgfpathclose%
\pgfusepath{stroke,fill}%
\end{pgfscope}%
\begin{pgfscope}%
\pgfpathrectangle{\pgfqpoint{0.457963in}{0.528059in}}{\pgfqpoint{6.200000in}{2.285714in}} %
\pgfusepath{clip}%
\pgfsetbuttcap%
\pgfsetroundjoin%
\definecolor{currentfill}{rgb}{0.666667,0.666667,1.000000}%
\pgfsetfillcolor{currentfill}%
\pgfsetlinewidth{1.003750pt}%
\definecolor{currentstroke}{rgb}{0.666667,0.666667,1.000000}%
\pgfsetstrokecolor{currentstroke}%
\pgfsetdash{}{0pt}%
\pgfpathmoveto{\pgfqpoint{3.237630in}{1.032513in}}%
\pgfpathcurveto{\pgfqpoint{3.245866in}{1.032513in}}{\pgfqpoint{3.253766in}{1.035785in}}{\pgfqpoint{3.259590in}{1.041609in}}%
\pgfpathcurveto{\pgfqpoint{3.265414in}{1.047433in}}{\pgfqpoint{3.268686in}{1.055333in}}{\pgfqpoint{3.268686in}{1.063570in}}%
\pgfpathcurveto{\pgfqpoint{3.268686in}{1.071806in}}{\pgfqpoint{3.265414in}{1.079706in}}{\pgfqpoint{3.259590in}{1.085530in}}%
\pgfpathcurveto{\pgfqpoint{3.253766in}{1.091354in}}{\pgfqpoint{3.245866in}{1.094626in}}{\pgfqpoint{3.237630in}{1.094626in}}%
\pgfpathcurveto{\pgfqpoint{3.229394in}{1.094626in}}{\pgfqpoint{3.221494in}{1.091354in}}{\pgfqpoint{3.215670in}{1.085530in}}%
\pgfpathcurveto{\pgfqpoint{3.209846in}{1.079706in}}{\pgfqpoint{3.206574in}{1.071806in}}{\pgfqpoint{3.206574in}{1.063570in}}%
\pgfpathcurveto{\pgfqpoint{3.206574in}{1.055333in}}{\pgfqpoint{3.209846in}{1.047433in}}{\pgfqpoint{3.215670in}{1.041609in}}%
\pgfpathcurveto{\pgfqpoint{3.221494in}{1.035785in}}{\pgfqpoint{3.229394in}{1.032513in}}{\pgfqpoint{3.237630in}{1.032513in}}%
\pgfpathclose%
\pgfusepath{stroke,fill}%
\end{pgfscope}%
\begin{pgfscope}%
\pgfpathrectangle{\pgfqpoint{0.457963in}{0.528059in}}{\pgfqpoint{6.200000in}{2.285714in}} %
\pgfusepath{clip}%
\pgfsetbuttcap%
\pgfsetroundjoin%
\definecolor{currentfill}{rgb}{0.666667,0.666667,1.000000}%
\pgfsetfillcolor{currentfill}%
\pgfsetlinewidth{1.003750pt}%
\definecolor{currentstroke}{rgb}{0.666667,0.666667,1.000000}%
\pgfsetstrokecolor{currentstroke}%
\pgfsetdash{}{0pt}%
\pgfpathmoveto{\pgfqpoint{3.537297in}{0.888840in}}%
\pgfpathcurveto{\pgfqpoint{3.545533in}{0.888840in}}{\pgfqpoint{3.553433in}{0.892112in}}{\pgfqpoint{3.559257in}{0.897936in}}%
\pgfpathcurveto{\pgfqpoint{3.565081in}{0.903760in}}{\pgfqpoint{3.568353in}{0.911660in}}{\pgfqpoint{3.568353in}{0.919896in}}%
\pgfpathcurveto{\pgfqpoint{3.568353in}{0.928132in}}{\pgfqpoint{3.565081in}{0.936033in}}{\pgfqpoint{3.559257in}{0.941856in}}%
\pgfpathcurveto{\pgfqpoint{3.553433in}{0.947680in}}{\pgfqpoint{3.545533in}{0.950953in}}{\pgfqpoint{3.537297in}{0.950953in}}%
\pgfpathcurveto{\pgfqpoint{3.529060in}{0.950953in}}{\pgfqpoint{3.521160in}{0.947680in}}{\pgfqpoint{3.515336in}{0.941856in}}%
\pgfpathcurveto{\pgfqpoint{3.509512in}{0.936033in}}{\pgfqpoint{3.506240in}{0.928132in}}{\pgfqpoint{3.506240in}{0.919896in}}%
\pgfpathcurveto{\pgfqpoint{3.506240in}{0.911660in}}{\pgfqpoint{3.509512in}{0.903760in}}{\pgfqpoint{3.515336in}{0.897936in}}%
\pgfpathcurveto{\pgfqpoint{3.521160in}{0.892112in}}{\pgfqpoint{3.529060in}{0.888840in}}{\pgfqpoint{3.537297in}{0.888840in}}%
\pgfpathclose%
\pgfusepath{stroke,fill}%
\end{pgfscope}%
\begin{pgfscope}%
\pgfpathrectangle{\pgfqpoint{0.457963in}{0.528059in}}{\pgfqpoint{6.200000in}{2.285714in}} %
\pgfusepath{clip}%
\pgfsetbuttcap%
\pgfsetroundjoin%
\definecolor{currentfill}{rgb}{0.500000,0.500000,1.000000}%
\pgfsetfillcolor{currentfill}%
\pgfsetlinewidth{1.003750pt}%
\definecolor{currentstroke}{rgb}{0.500000,0.500000,1.000000}%
\pgfsetstrokecolor{currentstroke}%
\pgfsetdash{}{0pt}%
\pgfpathmoveto{\pgfqpoint{0.457963in}{1.476595in}}%
\pgfpathcurveto{\pgfqpoint{0.466200in}{1.476595in}}{\pgfqpoint{0.474100in}{1.479867in}}{\pgfqpoint{0.479924in}{1.485691in}}%
\pgfpathcurveto{\pgfqpoint{0.485748in}{1.491515in}}{\pgfqpoint{0.489020in}{1.499415in}}{\pgfqpoint{0.489020in}{1.507651in}}%
\pgfpathcurveto{\pgfqpoint{0.489020in}{1.515888in}}{\pgfqpoint{0.485748in}{1.523788in}}{\pgfqpoint{0.479924in}{1.529612in}}%
\pgfpathcurveto{\pgfqpoint{0.474100in}{1.535435in}}{\pgfqpoint{0.466200in}{1.538708in}}{\pgfqpoint{0.457963in}{1.538708in}}%
\pgfpathcurveto{\pgfqpoint{0.449727in}{1.538708in}}{\pgfqpoint{0.441827in}{1.535435in}}{\pgfqpoint{0.436003in}{1.529612in}}%
\pgfpathcurveto{\pgfqpoint{0.430179in}{1.523788in}}{\pgfqpoint{0.426907in}{1.515888in}}{\pgfqpoint{0.426907in}{1.507651in}}%
\pgfpathcurveto{\pgfqpoint{0.426907in}{1.499415in}}{\pgfqpoint{0.430179in}{1.491515in}}{\pgfqpoint{0.436003in}{1.485691in}}%
\pgfpathcurveto{\pgfqpoint{0.441827in}{1.479867in}}{\pgfqpoint{0.449727in}{1.476595in}}{\pgfqpoint{0.457963in}{1.476595in}}%
\pgfpathclose%
\pgfusepath{stroke,fill}%
\end{pgfscope}%
\begin{pgfscope}%
\pgfpathrectangle{\pgfqpoint{0.457963in}{0.528059in}}{\pgfqpoint{6.200000in}{2.285714in}} %
\pgfusepath{clip}%
\pgfsetbuttcap%
\pgfsetroundjoin%
\definecolor{currentfill}{rgb}{0.500000,0.500000,1.000000}%
\pgfsetfillcolor{currentfill}%
\pgfsetlinewidth{1.003750pt}%
\definecolor{currentstroke}{rgb}{0.500000,0.500000,1.000000}%
\pgfsetstrokecolor{currentstroke}%
\pgfsetdash{}{0pt}%
\pgfpathmoveto{\pgfqpoint{0.457963in}{1.476595in}}%
\pgfpathcurveto{\pgfqpoint{0.466200in}{1.476595in}}{\pgfqpoint{0.474100in}{1.479867in}}{\pgfqpoint{0.479924in}{1.485691in}}%
\pgfpathcurveto{\pgfqpoint{0.485748in}{1.491515in}}{\pgfqpoint{0.489020in}{1.499415in}}{\pgfqpoint{0.489020in}{1.507651in}}%
\pgfpathcurveto{\pgfqpoint{0.489020in}{1.515888in}}{\pgfqpoint{0.485748in}{1.523788in}}{\pgfqpoint{0.479924in}{1.529612in}}%
\pgfpathcurveto{\pgfqpoint{0.474100in}{1.535435in}}{\pgfqpoint{0.466200in}{1.538708in}}{\pgfqpoint{0.457963in}{1.538708in}}%
\pgfpathcurveto{\pgfqpoint{0.449727in}{1.538708in}}{\pgfqpoint{0.441827in}{1.535435in}}{\pgfqpoint{0.436003in}{1.529612in}}%
\pgfpathcurveto{\pgfqpoint{0.430179in}{1.523788in}}{\pgfqpoint{0.426907in}{1.515888in}}{\pgfqpoint{0.426907in}{1.507651in}}%
\pgfpathcurveto{\pgfqpoint{0.426907in}{1.499415in}}{\pgfqpoint{0.430179in}{1.491515in}}{\pgfqpoint{0.436003in}{1.485691in}}%
\pgfpathcurveto{\pgfqpoint{0.441827in}{1.479867in}}{\pgfqpoint{0.449727in}{1.476595in}}{\pgfqpoint{0.457963in}{1.476595in}}%
\pgfpathclose%
\pgfusepath{stroke,fill}%
\end{pgfscope}%
\begin{pgfscope}%
\pgfpathrectangle{\pgfqpoint{0.457963in}{0.528059in}}{\pgfqpoint{6.200000in}{2.285714in}} %
\pgfusepath{clip}%
\pgfsetbuttcap%
\pgfsetroundjoin%
\definecolor{currentfill}{rgb}{0.500000,0.500000,1.000000}%
\pgfsetfillcolor{currentfill}%
\pgfsetlinewidth{1.003750pt}%
\definecolor{currentstroke}{rgb}{0.500000,0.500000,1.000000}%
\pgfsetstrokecolor{currentstroke}%
\pgfsetdash{}{0pt}%
\pgfpathmoveto{\pgfqpoint{0.478630in}{1.476595in}}%
\pgfpathcurveto{\pgfqpoint{0.486866in}{1.476595in}}{\pgfqpoint{0.494766in}{1.479867in}}{\pgfqpoint{0.500590in}{1.485691in}}%
\pgfpathcurveto{\pgfqpoint{0.506414in}{1.491515in}}{\pgfqpoint{0.509686in}{1.499415in}}{\pgfqpoint{0.509686in}{1.507651in}}%
\pgfpathcurveto{\pgfqpoint{0.509686in}{1.515888in}}{\pgfqpoint{0.506414in}{1.523788in}}{\pgfqpoint{0.500590in}{1.529612in}}%
\pgfpathcurveto{\pgfqpoint{0.494766in}{1.535435in}}{\pgfqpoint{0.486866in}{1.538708in}}{\pgfqpoint{0.478630in}{1.538708in}}%
\pgfpathcurveto{\pgfqpoint{0.470394in}{1.538708in}}{\pgfqpoint{0.462494in}{1.535435in}}{\pgfqpoint{0.456670in}{1.529612in}}%
\pgfpathcurveto{\pgfqpoint{0.450846in}{1.523788in}}{\pgfqpoint{0.447574in}{1.515888in}}{\pgfqpoint{0.447574in}{1.507651in}}%
\pgfpathcurveto{\pgfqpoint{0.447574in}{1.499415in}}{\pgfqpoint{0.450846in}{1.491515in}}{\pgfqpoint{0.456670in}{1.485691in}}%
\pgfpathcurveto{\pgfqpoint{0.462494in}{1.479867in}}{\pgfqpoint{0.470394in}{1.476595in}}{\pgfqpoint{0.478630in}{1.476595in}}%
\pgfpathclose%
\pgfusepath{stroke,fill}%
\end{pgfscope}%
\begin{pgfscope}%
\pgfpathrectangle{\pgfqpoint{0.457963in}{0.528059in}}{\pgfqpoint{6.200000in}{2.285714in}} %
\pgfusepath{clip}%
\pgfsetbuttcap%
\pgfsetroundjoin%
\definecolor{currentfill}{rgb}{0.500000,0.500000,1.000000}%
\pgfsetfillcolor{currentfill}%
\pgfsetlinewidth{1.003750pt}%
\definecolor{currentstroke}{rgb}{0.500000,0.500000,1.000000}%
\pgfsetstrokecolor{currentstroke}%
\pgfsetdash{}{0pt}%
\pgfpathmoveto{\pgfqpoint{0.519963in}{1.463534in}}%
\pgfpathcurveto{\pgfqpoint{0.528200in}{1.463534in}}{\pgfqpoint{0.536100in}{1.466806in}}{\pgfqpoint{0.541924in}{1.472630in}}%
\pgfpathcurveto{\pgfqpoint{0.547748in}{1.478454in}}{\pgfqpoint{0.551020in}{1.486354in}}{\pgfqpoint{0.551020in}{1.494590in}}%
\pgfpathcurveto{\pgfqpoint{0.551020in}{1.502826in}}{\pgfqpoint{0.547748in}{1.510726in}}{\pgfqpoint{0.541924in}{1.516550in}}%
\pgfpathcurveto{\pgfqpoint{0.536100in}{1.522374in}}{\pgfqpoint{0.528200in}{1.525647in}}{\pgfqpoint{0.519963in}{1.525647in}}%
\pgfpathcurveto{\pgfqpoint{0.511727in}{1.525647in}}{\pgfqpoint{0.503827in}{1.522374in}}{\pgfqpoint{0.498003in}{1.516550in}}%
\pgfpathcurveto{\pgfqpoint{0.492179in}{1.510726in}}{\pgfqpoint{0.488907in}{1.502826in}}{\pgfqpoint{0.488907in}{1.494590in}}%
\pgfpathcurveto{\pgfqpoint{0.488907in}{1.486354in}}{\pgfqpoint{0.492179in}{1.478454in}}{\pgfqpoint{0.498003in}{1.472630in}}%
\pgfpathcurveto{\pgfqpoint{0.503827in}{1.466806in}}{\pgfqpoint{0.511727in}{1.463534in}}{\pgfqpoint{0.519963in}{1.463534in}}%
\pgfpathclose%
\pgfusepath{stroke,fill}%
\end{pgfscope}%
\begin{pgfscope}%
\pgfpathrectangle{\pgfqpoint{0.457963in}{0.528059in}}{\pgfqpoint{6.200000in}{2.285714in}} %
\pgfusepath{clip}%
\pgfsetbuttcap%
\pgfsetroundjoin%
\definecolor{currentfill}{rgb}{0.500000,0.500000,1.000000}%
\pgfsetfillcolor{currentfill}%
\pgfsetlinewidth{1.003750pt}%
\definecolor{currentstroke}{rgb}{0.500000,0.500000,1.000000}%
\pgfsetstrokecolor{currentstroke}%
\pgfsetdash{}{0pt}%
\pgfpathmoveto{\pgfqpoint{0.519963in}{1.463534in}}%
\pgfpathcurveto{\pgfqpoint{0.528200in}{1.463534in}}{\pgfqpoint{0.536100in}{1.466806in}}{\pgfqpoint{0.541924in}{1.472630in}}%
\pgfpathcurveto{\pgfqpoint{0.547748in}{1.478454in}}{\pgfqpoint{0.551020in}{1.486354in}}{\pgfqpoint{0.551020in}{1.494590in}}%
\pgfpathcurveto{\pgfqpoint{0.551020in}{1.502826in}}{\pgfqpoint{0.547748in}{1.510726in}}{\pgfqpoint{0.541924in}{1.516550in}}%
\pgfpathcurveto{\pgfqpoint{0.536100in}{1.522374in}}{\pgfqpoint{0.528200in}{1.525647in}}{\pgfqpoint{0.519963in}{1.525647in}}%
\pgfpathcurveto{\pgfqpoint{0.511727in}{1.525647in}}{\pgfqpoint{0.503827in}{1.522374in}}{\pgfqpoint{0.498003in}{1.516550in}}%
\pgfpathcurveto{\pgfqpoint{0.492179in}{1.510726in}}{\pgfqpoint{0.488907in}{1.502826in}}{\pgfqpoint{0.488907in}{1.494590in}}%
\pgfpathcurveto{\pgfqpoint{0.488907in}{1.486354in}}{\pgfqpoint{0.492179in}{1.478454in}}{\pgfqpoint{0.498003in}{1.472630in}}%
\pgfpathcurveto{\pgfqpoint{0.503827in}{1.466806in}}{\pgfqpoint{0.511727in}{1.463534in}}{\pgfqpoint{0.519963in}{1.463534in}}%
\pgfpathclose%
\pgfusepath{stroke,fill}%
\end{pgfscope}%
\begin{pgfscope}%
\pgfpathrectangle{\pgfqpoint{0.457963in}{0.528059in}}{\pgfqpoint{6.200000in}{2.285714in}} %
\pgfusepath{clip}%
\pgfsetbuttcap%
\pgfsetroundjoin%
\definecolor{currentfill}{rgb}{0.500000,0.500000,1.000000}%
\pgfsetfillcolor{currentfill}%
\pgfsetlinewidth{1.003750pt}%
\definecolor{currentstroke}{rgb}{0.500000,0.500000,1.000000}%
\pgfsetstrokecolor{currentstroke}%
\pgfsetdash{}{0pt}%
\pgfpathmoveto{\pgfqpoint{0.561297in}{1.463534in}}%
\pgfpathcurveto{\pgfqpoint{0.569533in}{1.463534in}}{\pgfqpoint{0.577433in}{1.466806in}}{\pgfqpoint{0.583257in}{1.472630in}}%
\pgfpathcurveto{\pgfqpoint{0.589081in}{1.478454in}}{\pgfqpoint{0.592353in}{1.486354in}}{\pgfqpoint{0.592353in}{1.494590in}}%
\pgfpathcurveto{\pgfqpoint{0.592353in}{1.502826in}}{\pgfqpoint{0.589081in}{1.510726in}}{\pgfqpoint{0.583257in}{1.516550in}}%
\pgfpathcurveto{\pgfqpoint{0.577433in}{1.522374in}}{\pgfqpoint{0.569533in}{1.525647in}}{\pgfqpoint{0.561297in}{1.525647in}}%
\pgfpathcurveto{\pgfqpoint{0.553060in}{1.525647in}}{\pgfqpoint{0.545160in}{1.522374in}}{\pgfqpoint{0.539336in}{1.516550in}}%
\pgfpathcurveto{\pgfqpoint{0.533512in}{1.510726in}}{\pgfqpoint{0.530240in}{1.502826in}}{\pgfqpoint{0.530240in}{1.494590in}}%
\pgfpathcurveto{\pgfqpoint{0.530240in}{1.486354in}}{\pgfqpoint{0.533512in}{1.478454in}}{\pgfqpoint{0.539336in}{1.472630in}}%
\pgfpathcurveto{\pgfqpoint{0.545160in}{1.466806in}}{\pgfqpoint{0.553060in}{1.463534in}}{\pgfqpoint{0.561297in}{1.463534in}}%
\pgfpathclose%
\pgfusepath{stroke,fill}%
\end{pgfscope}%
\begin{pgfscope}%
\pgfpathrectangle{\pgfqpoint{0.457963in}{0.528059in}}{\pgfqpoint{6.200000in}{2.285714in}} %
\pgfusepath{clip}%
\pgfsetbuttcap%
\pgfsetroundjoin%
\definecolor{currentfill}{rgb}{0.500000,0.500000,1.000000}%
\pgfsetfillcolor{currentfill}%
\pgfsetlinewidth{1.003750pt}%
\definecolor{currentstroke}{rgb}{0.500000,0.500000,1.000000}%
\pgfsetstrokecolor{currentstroke}%
\pgfsetdash{}{0pt}%
\pgfpathmoveto{\pgfqpoint{0.581963in}{1.476595in}}%
\pgfpathcurveto{\pgfqpoint{0.590200in}{1.476595in}}{\pgfqpoint{0.598100in}{1.479867in}}{\pgfqpoint{0.603924in}{1.485691in}}%
\pgfpathcurveto{\pgfqpoint{0.609748in}{1.491515in}}{\pgfqpoint{0.613020in}{1.499415in}}{\pgfqpoint{0.613020in}{1.507651in}}%
\pgfpathcurveto{\pgfqpoint{0.613020in}{1.515888in}}{\pgfqpoint{0.609748in}{1.523788in}}{\pgfqpoint{0.603924in}{1.529612in}}%
\pgfpathcurveto{\pgfqpoint{0.598100in}{1.535435in}}{\pgfqpoint{0.590200in}{1.538708in}}{\pgfqpoint{0.581963in}{1.538708in}}%
\pgfpathcurveto{\pgfqpoint{0.573727in}{1.538708in}}{\pgfqpoint{0.565827in}{1.535435in}}{\pgfqpoint{0.560003in}{1.529612in}}%
\pgfpathcurveto{\pgfqpoint{0.554179in}{1.523788in}}{\pgfqpoint{0.550907in}{1.515888in}}{\pgfqpoint{0.550907in}{1.507651in}}%
\pgfpathcurveto{\pgfqpoint{0.550907in}{1.499415in}}{\pgfqpoint{0.554179in}{1.491515in}}{\pgfqpoint{0.560003in}{1.485691in}}%
\pgfpathcurveto{\pgfqpoint{0.565827in}{1.479867in}}{\pgfqpoint{0.573727in}{1.476595in}}{\pgfqpoint{0.581963in}{1.476595in}}%
\pgfpathclose%
\pgfusepath{stroke,fill}%
\end{pgfscope}%
\begin{pgfscope}%
\pgfpathrectangle{\pgfqpoint{0.457963in}{0.528059in}}{\pgfqpoint{6.200000in}{2.285714in}} %
\pgfusepath{clip}%
\pgfsetbuttcap%
\pgfsetroundjoin%
\definecolor{currentfill}{rgb}{0.500000,0.500000,1.000000}%
\pgfsetfillcolor{currentfill}%
\pgfsetlinewidth{1.003750pt}%
\definecolor{currentstroke}{rgb}{0.500000,0.500000,1.000000}%
\pgfsetstrokecolor{currentstroke}%
\pgfsetdash{}{0pt}%
\pgfpathmoveto{\pgfqpoint{0.643963in}{1.476595in}}%
\pgfpathcurveto{\pgfqpoint{0.652200in}{1.476595in}}{\pgfqpoint{0.660100in}{1.479867in}}{\pgfqpoint{0.665924in}{1.485691in}}%
\pgfpathcurveto{\pgfqpoint{0.671748in}{1.491515in}}{\pgfqpoint{0.675020in}{1.499415in}}{\pgfqpoint{0.675020in}{1.507651in}}%
\pgfpathcurveto{\pgfqpoint{0.675020in}{1.515888in}}{\pgfqpoint{0.671748in}{1.523788in}}{\pgfqpoint{0.665924in}{1.529612in}}%
\pgfpathcurveto{\pgfqpoint{0.660100in}{1.535435in}}{\pgfqpoint{0.652200in}{1.538708in}}{\pgfqpoint{0.643963in}{1.538708in}}%
\pgfpathcurveto{\pgfqpoint{0.635727in}{1.538708in}}{\pgfqpoint{0.627827in}{1.535435in}}{\pgfqpoint{0.622003in}{1.529612in}}%
\pgfpathcurveto{\pgfqpoint{0.616179in}{1.523788in}}{\pgfqpoint{0.612907in}{1.515888in}}{\pgfqpoint{0.612907in}{1.507651in}}%
\pgfpathcurveto{\pgfqpoint{0.612907in}{1.499415in}}{\pgfqpoint{0.616179in}{1.491515in}}{\pgfqpoint{0.622003in}{1.485691in}}%
\pgfpathcurveto{\pgfqpoint{0.627827in}{1.479867in}}{\pgfqpoint{0.635727in}{1.476595in}}{\pgfqpoint{0.643963in}{1.476595in}}%
\pgfpathclose%
\pgfusepath{stroke,fill}%
\end{pgfscope}%
\begin{pgfscope}%
\pgfpathrectangle{\pgfqpoint{0.457963in}{0.528059in}}{\pgfqpoint{6.200000in}{2.285714in}} %
\pgfusepath{clip}%
\pgfsetbuttcap%
\pgfsetroundjoin%
\definecolor{currentfill}{rgb}{0.500000,0.500000,1.000000}%
\pgfsetfillcolor{currentfill}%
\pgfsetlinewidth{1.003750pt}%
\definecolor{currentstroke}{rgb}{0.500000,0.500000,1.000000}%
\pgfsetstrokecolor{currentstroke}%
\pgfsetdash{}{0pt}%
\pgfpathmoveto{\pgfqpoint{0.767963in}{1.450472in}}%
\pgfpathcurveto{\pgfqpoint{0.776200in}{1.450472in}}{\pgfqpoint{0.784100in}{1.453745in}}{\pgfqpoint{0.789924in}{1.459569in}}%
\pgfpathcurveto{\pgfqpoint{0.795748in}{1.465393in}}{\pgfqpoint{0.799020in}{1.473293in}}{\pgfqpoint{0.799020in}{1.481529in}}%
\pgfpathcurveto{\pgfqpoint{0.799020in}{1.489765in}}{\pgfqpoint{0.795748in}{1.497665in}}{\pgfqpoint{0.789924in}{1.503489in}}%
\pgfpathcurveto{\pgfqpoint{0.784100in}{1.509313in}}{\pgfqpoint{0.776200in}{1.512585in}}{\pgfqpoint{0.767963in}{1.512585in}}%
\pgfpathcurveto{\pgfqpoint{0.759727in}{1.512585in}}{\pgfqpoint{0.751827in}{1.509313in}}{\pgfqpoint{0.746003in}{1.503489in}}%
\pgfpathcurveto{\pgfqpoint{0.740179in}{1.497665in}}{\pgfqpoint{0.736907in}{1.489765in}}{\pgfqpoint{0.736907in}{1.481529in}}%
\pgfpathcurveto{\pgfqpoint{0.736907in}{1.473293in}}{\pgfqpoint{0.740179in}{1.465393in}}{\pgfqpoint{0.746003in}{1.459569in}}%
\pgfpathcurveto{\pgfqpoint{0.751827in}{1.453745in}}{\pgfqpoint{0.759727in}{1.450472in}}{\pgfqpoint{0.767963in}{1.450472in}}%
\pgfpathclose%
\pgfusepath{stroke,fill}%
\end{pgfscope}%
\begin{pgfscope}%
\pgfpathrectangle{\pgfqpoint{0.457963in}{0.528059in}}{\pgfqpoint{6.200000in}{2.285714in}} %
\pgfusepath{clip}%
\pgfsetbuttcap%
\pgfsetroundjoin%
\definecolor{currentfill}{rgb}{0.500000,0.500000,1.000000}%
\pgfsetfillcolor{currentfill}%
\pgfsetlinewidth{1.003750pt}%
\definecolor{currentstroke}{rgb}{0.500000,0.500000,1.000000}%
\pgfsetstrokecolor{currentstroke}%
\pgfsetdash{}{0pt}%
\pgfpathmoveto{\pgfqpoint{0.995297in}{1.398227in}}%
\pgfpathcurveto{\pgfqpoint{1.003533in}{1.398227in}}{\pgfqpoint{1.011433in}{1.401500in}}{\pgfqpoint{1.017257in}{1.407324in}}%
\pgfpathcurveto{\pgfqpoint{1.023081in}{1.413148in}}{\pgfqpoint{1.026353in}{1.421048in}}{\pgfqpoint{1.026353in}{1.429284in}}%
\pgfpathcurveto{\pgfqpoint{1.026353in}{1.437520in}}{\pgfqpoint{1.023081in}{1.445420in}}{\pgfqpoint{1.017257in}{1.451244in}}%
\pgfpathcurveto{\pgfqpoint{1.011433in}{1.457068in}}{\pgfqpoint{1.003533in}{1.460340in}}{\pgfqpoint{0.995297in}{1.460340in}}%
\pgfpathcurveto{\pgfqpoint{0.987060in}{1.460340in}}{\pgfqpoint{0.979160in}{1.457068in}}{\pgfqpoint{0.973336in}{1.451244in}}%
\pgfpathcurveto{\pgfqpoint{0.967512in}{1.445420in}}{\pgfqpoint{0.964240in}{1.437520in}}{\pgfqpoint{0.964240in}{1.429284in}}%
\pgfpathcurveto{\pgfqpoint{0.964240in}{1.421048in}}{\pgfqpoint{0.967512in}{1.413148in}}{\pgfqpoint{0.973336in}{1.407324in}}%
\pgfpathcurveto{\pgfqpoint{0.979160in}{1.401500in}}{\pgfqpoint{0.987060in}{1.398227in}}{\pgfqpoint{0.995297in}{1.398227in}}%
\pgfpathclose%
\pgfusepath{stroke,fill}%
\end{pgfscope}%
\begin{pgfscope}%
\pgfpathrectangle{\pgfqpoint{0.457963in}{0.528059in}}{\pgfqpoint{6.200000in}{2.285714in}} %
\pgfusepath{clip}%
\pgfsetbuttcap%
\pgfsetroundjoin%
\definecolor{currentfill}{rgb}{0.500000,0.500000,1.000000}%
\pgfsetfillcolor{currentfill}%
\pgfsetlinewidth{1.003750pt}%
\definecolor{currentstroke}{rgb}{0.500000,0.500000,1.000000}%
\pgfsetstrokecolor{currentstroke}%
\pgfsetdash{}{0pt}%
\pgfpathmoveto{\pgfqpoint{1.026297in}{1.476595in}}%
\pgfpathcurveto{\pgfqpoint{1.034533in}{1.476595in}}{\pgfqpoint{1.042433in}{1.479867in}}{\pgfqpoint{1.048257in}{1.485691in}}%
\pgfpathcurveto{\pgfqpoint{1.054081in}{1.491515in}}{\pgfqpoint{1.057353in}{1.499415in}}{\pgfqpoint{1.057353in}{1.507651in}}%
\pgfpathcurveto{\pgfqpoint{1.057353in}{1.515888in}}{\pgfqpoint{1.054081in}{1.523788in}}{\pgfqpoint{1.048257in}{1.529612in}}%
\pgfpathcurveto{\pgfqpoint{1.042433in}{1.535435in}}{\pgfqpoint{1.034533in}{1.538708in}}{\pgfqpoint{1.026297in}{1.538708in}}%
\pgfpathcurveto{\pgfqpoint{1.018060in}{1.538708in}}{\pgfqpoint{1.010160in}{1.535435in}}{\pgfqpoint{1.004336in}{1.529612in}}%
\pgfpathcurveto{\pgfqpoint{0.998512in}{1.523788in}}{\pgfqpoint{0.995240in}{1.515888in}}{\pgfqpoint{0.995240in}{1.507651in}}%
\pgfpathcurveto{\pgfqpoint{0.995240in}{1.499415in}}{\pgfqpoint{0.998512in}{1.491515in}}{\pgfqpoint{1.004336in}{1.485691in}}%
\pgfpathcurveto{\pgfqpoint{1.010160in}{1.479867in}}{\pgfqpoint{1.018060in}{1.476595in}}{\pgfqpoint{1.026297in}{1.476595in}}%
\pgfpathclose%
\pgfusepath{stroke,fill}%
\end{pgfscope}%
\begin{pgfscope}%
\pgfpathrectangle{\pgfqpoint{0.457963in}{0.528059in}}{\pgfqpoint{6.200000in}{2.285714in}} %
\pgfusepath{clip}%
\pgfsetbuttcap%
\pgfsetroundjoin%
\definecolor{currentfill}{rgb}{0.500000,0.500000,1.000000}%
\pgfsetfillcolor{currentfill}%
\pgfsetlinewidth{1.003750pt}%
\definecolor{currentstroke}{rgb}{0.500000,0.500000,1.000000}%
\pgfsetstrokecolor{currentstroke}%
\pgfsetdash{}{0pt}%
\pgfpathmoveto{\pgfqpoint{1.325963in}{1.476595in}}%
\pgfpathcurveto{\pgfqpoint{1.334200in}{1.476595in}}{\pgfqpoint{1.342100in}{1.479867in}}{\pgfqpoint{1.347924in}{1.485691in}}%
\pgfpathcurveto{\pgfqpoint{1.353748in}{1.491515in}}{\pgfqpoint{1.357020in}{1.499415in}}{\pgfqpoint{1.357020in}{1.507651in}}%
\pgfpathcurveto{\pgfqpoint{1.357020in}{1.515888in}}{\pgfqpoint{1.353748in}{1.523788in}}{\pgfqpoint{1.347924in}{1.529612in}}%
\pgfpathcurveto{\pgfqpoint{1.342100in}{1.535435in}}{\pgfqpoint{1.334200in}{1.538708in}}{\pgfqpoint{1.325963in}{1.538708in}}%
\pgfpathcurveto{\pgfqpoint{1.317727in}{1.538708in}}{\pgfqpoint{1.309827in}{1.535435in}}{\pgfqpoint{1.304003in}{1.529612in}}%
\pgfpathcurveto{\pgfqpoint{1.298179in}{1.523788in}}{\pgfqpoint{1.294907in}{1.515888in}}{\pgfqpoint{1.294907in}{1.507651in}}%
\pgfpathcurveto{\pgfqpoint{1.294907in}{1.499415in}}{\pgfqpoint{1.298179in}{1.491515in}}{\pgfqpoint{1.304003in}{1.485691in}}%
\pgfpathcurveto{\pgfqpoint{1.309827in}{1.479867in}}{\pgfqpoint{1.317727in}{1.476595in}}{\pgfqpoint{1.325963in}{1.476595in}}%
\pgfpathclose%
\pgfusepath{stroke,fill}%
\end{pgfscope}%
\begin{pgfscope}%
\pgfpathrectangle{\pgfqpoint{0.457963in}{0.528059in}}{\pgfqpoint{6.200000in}{2.285714in}} %
\pgfusepath{clip}%
\pgfsetbuttcap%
\pgfsetroundjoin%
\definecolor{currentfill}{rgb}{0.500000,0.500000,1.000000}%
\pgfsetfillcolor{currentfill}%
\pgfsetlinewidth{1.003750pt}%
\definecolor{currentstroke}{rgb}{0.500000,0.500000,1.000000}%
\pgfsetstrokecolor{currentstroke}%
\pgfsetdash{}{0pt}%
\pgfpathmoveto{\pgfqpoint{1.387963in}{1.476595in}}%
\pgfpathcurveto{\pgfqpoint{1.396200in}{1.476595in}}{\pgfqpoint{1.404100in}{1.479867in}}{\pgfqpoint{1.409924in}{1.485691in}}%
\pgfpathcurveto{\pgfqpoint{1.415748in}{1.491515in}}{\pgfqpoint{1.419020in}{1.499415in}}{\pgfqpoint{1.419020in}{1.507651in}}%
\pgfpathcurveto{\pgfqpoint{1.419020in}{1.515888in}}{\pgfqpoint{1.415748in}{1.523788in}}{\pgfqpoint{1.409924in}{1.529612in}}%
\pgfpathcurveto{\pgfqpoint{1.404100in}{1.535435in}}{\pgfqpoint{1.396200in}{1.538708in}}{\pgfqpoint{1.387963in}{1.538708in}}%
\pgfpathcurveto{\pgfqpoint{1.379727in}{1.538708in}}{\pgfqpoint{1.371827in}{1.535435in}}{\pgfqpoint{1.366003in}{1.529612in}}%
\pgfpathcurveto{\pgfqpoint{1.360179in}{1.523788in}}{\pgfqpoint{1.356907in}{1.515888in}}{\pgfqpoint{1.356907in}{1.507651in}}%
\pgfpathcurveto{\pgfqpoint{1.356907in}{1.499415in}}{\pgfqpoint{1.360179in}{1.491515in}}{\pgfqpoint{1.366003in}{1.485691in}}%
\pgfpathcurveto{\pgfqpoint{1.371827in}{1.479867in}}{\pgfqpoint{1.379727in}{1.476595in}}{\pgfqpoint{1.387963in}{1.476595in}}%
\pgfpathclose%
\pgfusepath{stroke,fill}%
\end{pgfscope}%
\begin{pgfscope}%
\pgfpathrectangle{\pgfqpoint{0.457963in}{0.528059in}}{\pgfqpoint{6.200000in}{2.285714in}} %
\pgfusepath{clip}%
\pgfsetbuttcap%
\pgfsetroundjoin%
\definecolor{currentfill}{rgb}{0.500000,0.500000,1.000000}%
\pgfsetfillcolor{currentfill}%
\pgfsetlinewidth{1.003750pt}%
\definecolor{currentstroke}{rgb}{0.500000,0.500000,1.000000}%
\pgfsetstrokecolor{currentstroke}%
\pgfsetdash{}{0pt}%
\pgfpathmoveto{\pgfqpoint{1.697963in}{1.345983in}}%
\pgfpathcurveto{\pgfqpoint{1.706200in}{1.345983in}}{\pgfqpoint{1.714100in}{1.349255in}}{\pgfqpoint{1.719924in}{1.355079in}}%
\pgfpathcurveto{\pgfqpoint{1.725748in}{1.360903in}}{\pgfqpoint{1.729020in}{1.368803in}}{\pgfqpoint{1.729020in}{1.377039in}}%
\pgfpathcurveto{\pgfqpoint{1.729020in}{1.385275in}}{\pgfqpoint{1.725748in}{1.393175in}}{\pgfqpoint{1.719924in}{1.398999in}}%
\pgfpathcurveto{\pgfqpoint{1.714100in}{1.404823in}}{\pgfqpoint{1.706200in}{1.408096in}}{\pgfqpoint{1.697963in}{1.408096in}}%
\pgfpathcurveto{\pgfqpoint{1.689727in}{1.408096in}}{\pgfqpoint{1.681827in}{1.404823in}}{\pgfqpoint{1.676003in}{1.398999in}}%
\pgfpathcurveto{\pgfqpoint{1.670179in}{1.393175in}}{\pgfqpoint{1.666907in}{1.385275in}}{\pgfqpoint{1.666907in}{1.377039in}}%
\pgfpathcurveto{\pgfqpoint{1.666907in}{1.368803in}}{\pgfqpoint{1.670179in}{1.360903in}}{\pgfqpoint{1.676003in}{1.355079in}}%
\pgfpathcurveto{\pgfqpoint{1.681827in}{1.349255in}}{\pgfqpoint{1.689727in}{1.345983in}}{\pgfqpoint{1.697963in}{1.345983in}}%
\pgfpathclose%
\pgfusepath{stroke,fill}%
\end{pgfscope}%
\begin{pgfscope}%
\pgfpathrectangle{\pgfqpoint{0.457963in}{0.528059in}}{\pgfqpoint{6.200000in}{2.285714in}} %
\pgfusepath{clip}%
\pgfsetbuttcap%
\pgfsetroundjoin%
\definecolor{currentfill}{rgb}{0.500000,0.500000,1.000000}%
\pgfsetfillcolor{currentfill}%
\pgfsetlinewidth{1.003750pt}%
\definecolor{currentstroke}{rgb}{0.500000,0.500000,1.000000}%
\pgfsetstrokecolor{currentstroke}%
\pgfsetdash{}{0pt}%
\pgfpathmoveto{\pgfqpoint{2.565963in}{1.411289in}}%
\pgfpathcurveto{\pgfqpoint{2.574200in}{1.411289in}}{\pgfqpoint{2.582100in}{1.414561in}}{\pgfqpoint{2.587924in}{1.420385in}}%
\pgfpathcurveto{\pgfqpoint{2.593748in}{1.426209in}}{\pgfqpoint{2.597020in}{1.434109in}}{\pgfqpoint{2.597020in}{1.442345in}}%
\pgfpathcurveto{\pgfqpoint{2.597020in}{1.450581in}}{\pgfqpoint{2.593748in}{1.458481in}}{\pgfqpoint{2.587924in}{1.464305in}}%
\pgfpathcurveto{\pgfqpoint{2.582100in}{1.470129in}}{\pgfqpoint{2.574200in}{1.473402in}}{\pgfqpoint{2.565963in}{1.473402in}}%
\pgfpathcurveto{\pgfqpoint{2.557727in}{1.473402in}}{\pgfqpoint{2.549827in}{1.470129in}}{\pgfqpoint{2.544003in}{1.464305in}}%
\pgfpathcurveto{\pgfqpoint{2.538179in}{1.458481in}}{\pgfqpoint{2.534907in}{1.450581in}}{\pgfqpoint{2.534907in}{1.442345in}}%
\pgfpathcurveto{\pgfqpoint{2.534907in}{1.434109in}}{\pgfqpoint{2.538179in}{1.426209in}}{\pgfqpoint{2.544003in}{1.420385in}}%
\pgfpathcurveto{\pgfqpoint{2.549827in}{1.414561in}}{\pgfqpoint{2.557727in}{1.411289in}}{\pgfqpoint{2.565963in}{1.411289in}}%
\pgfpathclose%
\pgfusepath{stroke,fill}%
\end{pgfscope}%
\begin{pgfscope}%
\pgfpathrectangle{\pgfqpoint{0.457963in}{0.528059in}}{\pgfqpoint{6.200000in}{2.285714in}} %
\pgfusepath{clip}%
\pgfsetbuttcap%
\pgfsetroundjoin%
\definecolor{currentfill}{rgb}{0.500000,0.500000,1.000000}%
\pgfsetfillcolor{currentfill}%
\pgfsetlinewidth{1.003750pt}%
\definecolor{currentstroke}{rgb}{0.500000,0.500000,1.000000}%
\pgfsetstrokecolor{currentstroke}%
\pgfsetdash{}{0pt}%
\pgfpathmoveto{\pgfqpoint{2.937963in}{1.071697in}}%
\pgfpathcurveto{\pgfqpoint{2.946200in}{1.071697in}}{\pgfqpoint{2.954100in}{1.074969in}}{\pgfqpoint{2.959924in}{1.080793in}}%
\pgfpathcurveto{\pgfqpoint{2.965748in}{1.086617in}}{\pgfqpoint{2.969020in}{1.094517in}}{\pgfqpoint{2.969020in}{1.102753in}}%
\pgfpathcurveto{\pgfqpoint{2.969020in}{1.110990in}}{\pgfqpoint{2.965748in}{1.118890in}}{\pgfqpoint{2.959924in}{1.124714in}}%
\pgfpathcurveto{\pgfqpoint{2.954100in}{1.130538in}}{\pgfqpoint{2.946200in}{1.133810in}}{\pgfqpoint{2.937963in}{1.133810in}}%
\pgfpathcurveto{\pgfqpoint{2.929727in}{1.133810in}}{\pgfqpoint{2.921827in}{1.130538in}}{\pgfqpoint{2.916003in}{1.124714in}}%
\pgfpathcurveto{\pgfqpoint{2.910179in}{1.118890in}}{\pgfqpoint{2.906907in}{1.110990in}}{\pgfqpoint{2.906907in}{1.102753in}}%
\pgfpathcurveto{\pgfqpoint{2.906907in}{1.094517in}}{\pgfqpoint{2.910179in}{1.086617in}}{\pgfqpoint{2.916003in}{1.080793in}}%
\pgfpathcurveto{\pgfqpoint{2.921827in}{1.074969in}}{\pgfqpoint{2.929727in}{1.071697in}}{\pgfqpoint{2.937963in}{1.071697in}}%
\pgfpathclose%
\pgfusepath{stroke,fill}%
\end{pgfscope}%
\begin{pgfscope}%
\pgfpathrectangle{\pgfqpoint{0.457963in}{0.528059in}}{\pgfqpoint{6.200000in}{2.285714in}} %
\pgfusepath{clip}%
\pgfsetbuttcap%
\pgfsetroundjoin%
\definecolor{currentfill}{rgb}{0.500000,0.500000,1.000000}%
\pgfsetfillcolor{currentfill}%
\pgfsetlinewidth{1.003750pt}%
\definecolor{currentstroke}{rgb}{0.500000,0.500000,1.000000}%
\pgfsetstrokecolor{currentstroke}%
\pgfsetdash{}{0pt}%
\pgfpathmoveto{\pgfqpoint{3.278963in}{1.424350in}}%
\pgfpathcurveto{\pgfqpoint{3.287200in}{1.424350in}}{\pgfqpoint{3.295100in}{1.427622in}}{\pgfqpoint{3.300924in}{1.433446in}}%
\pgfpathcurveto{\pgfqpoint{3.306748in}{1.439270in}}{\pgfqpoint{3.310020in}{1.447170in}}{\pgfqpoint{3.310020in}{1.455406in}}%
\pgfpathcurveto{\pgfqpoint{3.310020in}{1.463643in}}{\pgfqpoint{3.306748in}{1.471543in}}{\pgfqpoint{3.300924in}{1.477367in}}%
\pgfpathcurveto{\pgfqpoint{3.295100in}{1.483191in}}{\pgfqpoint{3.287200in}{1.486463in}}{\pgfqpoint{3.278963in}{1.486463in}}%
\pgfpathcurveto{\pgfqpoint{3.270727in}{1.486463in}}{\pgfqpoint{3.262827in}{1.483191in}}{\pgfqpoint{3.257003in}{1.477367in}}%
\pgfpathcurveto{\pgfqpoint{3.251179in}{1.471543in}}{\pgfqpoint{3.247907in}{1.463643in}}{\pgfqpoint{3.247907in}{1.455406in}}%
\pgfpathcurveto{\pgfqpoint{3.247907in}{1.447170in}}{\pgfqpoint{3.251179in}{1.439270in}}{\pgfqpoint{3.257003in}{1.433446in}}%
\pgfpathcurveto{\pgfqpoint{3.262827in}{1.427622in}}{\pgfqpoint{3.270727in}{1.424350in}}{\pgfqpoint{3.278963in}{1.424350in}}%
\pgfpathclose%
\pgfusepath{stroke,fill}%
\end{pgfscope}%
\begin{pgfscope}%
\pgfpathrectangle{\pgfqpoint{0.457963in}{0.528059in}}{\pgfqpoint{6.200000in}{2.285714in}} %
\pgfusepath{clip}%
\pgfsetbuttcap%
\pgfsetroundjoin%
\definecolor{currentfill}{rgb}{0.500000,0.500000,1.000000}%
\pgfsetfillcolor{currentfill}%
\pgfsetlinewidth{1.003750pt}%
\definecolor{currentstroke}{rgb}{0.500000,0.500000,1.000000}%
\pgfsetstrokecolor{currentstroke}%
\pgfsetdash{}{0pt}%
\pgfpathmoveto{\pgfqpoint{3.671630in}{1.424350in}}%
\pgfpathcurveto{\pgfqpoint{3.679866in}{1.424350in}}{\pgfqpoint{3.687766in}{1.427622in}}{\pgfqpoint{3.693590in}{1.433446in}}%
\pgfpathcurveto{\pgfqpoint{3.699414in}{1.439270in}}{\pgfqpoint{3.702686in}{1.447170in}}{\pgfqpoint{3.702686in}{1.455406in}}%
\pgfpathcurveto{\pgfqpoint{3.702686in}{1.463643in}}{\pgfqpoint{3.699414in}{1.471543in}}{\pgfqpoint{3.693590in}{1.477367in}}%
\pgfpathcurveto{\pgfqpoint{3.687766in}{1.483191in}}{\pgfqpoint{3.679866in}{1.486463in}}{\pgfqpoint{3.671630in}{1.486463in}}%
\pgfpathcurveto{\pgfqpoint{3.663394in}{1.486463in}}{\pgfqpoint{3.655494in}{1.483191in}}{\pgfqpoint{3.649670in}{1.477367in}}%
\pgfpathcurveto{\pgfqpoint{3.643846in}{1.471543in}}{\pgfqpoint{3.640574in}{1.463643in}}{\pgfqpoint{3.640574in}{1.455406in}}%
\pgfpathcurveto{\pgfqpoint{3.640574in}{1.447170in}}{\pgfqpoint{3.643846in}{1.439270in}}{\pgfqpoint{3.649670in}{1.433446in}}%
\pgfpathcurveto{\pgfqpoint{3.655494in}{1.427622in}}{\pgfqpoint{3.663394in}{1.424350in}}{\pgfqpoint{3.671630in}{1.424350in}}%
\pgfpathclose%
\pgfusepath{stroke,fill}%
\end{pgfscope}%
\begin{pgfscope}%
\pgfpathrectangle{\pgfqpoint{0.457963in}{0.528059in}}{\pgfqpoint{6.200000in}{2.285714in}} %
\pgfusepath{clip}%
\pgfsetbuttcap%
\pgfsetroundjoin%
\definecolor{currentfill}{rgb}{0.500000,0.500000,1.000000}%
\pgfsetfillcolor{currentfill}%
\pgfsetlinewidth{1.003750pt}%
\definecolor{currentstroke}{rgb}{0.500000,0.500000,1.000000}%
\pgfsetstrokecolor{currentstroke}%
\pgfsetdash{}{0pt}%
\pgfpathmoveto{\pgfqpoint{4.570630in}{1.176187in}}%
\pgfpathcurveto{\pgfqpoint{4.578866in}{1.176187in}}{\pgfqpoint{4.586766in}{1.179459in}}{\pgfqpoint{4.592590in}{1.185283in}}%
\pgfpathcurveto{\pgfqpoint{4.598414in}{1.191107in}}{\pgfqpoint{4.601686in}{1.199007in}}{\pgfqpoint{4.601686in}{1.207243in}}%
\pgfpathcurveto{\pgfqpoint{4.601686in}{1.215479in}}{\pgfqpoint{4.598414in}{1.223379in}}{\pgfqpoint{4.592590in}{1.229203in}}%
\pgfpathcurveto{\pgfqpoint{4.586766in}{1.235027in}}{\pgfqpoint{4.578866in}{1.238300in}}{\pgfqpoint{4.570630in}{1.238300in}}%
\pgfpathcurveto{\pgfqpoint{4.562394in}{1.238300in}}{\pgfqpoint{4.554494in}{1.235027in}}{\pgfqpoint{4.548670in}{1.229203in}}%
\pgfpathcurveto{\pgfqpoint{4.542846in}{1.223379in}}{\pgfqpoint{4.539574in}{1.215479in}}{\pgfqpoint{4.539574in}{1.207243in}}%
\pgfpathcurveto{\pgfqpoint{4.539574in}{1.199007in}}{\pgfqpoint{4.542846in}{1.191107in}}{\pgfqpoint{4.548670in}{1.185283in}}%
\pgfpathcurveto{\pgfqpoint{4.554494in}{1.179459in}}{\pgfqpoint{4.562394in}{1.176187in}}{\pgfqpoint{4.570630in}{1.176187in}}%
\pgfpathclose%
\pgfusepath{stroke,fill}%
\end{pgfscope}%
\begin{pgfscope}%
\pgfpathrectangle{\pgfqpoint{0.457963in}{0.528059in}}{\pgfqpoint{6.200000in}{2.285714in}} %
\pgfusepath{clip}%
\pgfsetbuttcap%
\pgfsetroundjoin%
\definecolor{currentfill}{rgb}{0.500000,0.500000,1.000000}%
\pgfsetfillcolor{currentfill}%
\pgfsetlinewidth{1.003750pt}%
\definecolor{currentstroke}{rgb}{0.500000,0.500000,1.000000}%
\pgfsetstrokecolor{currentstroke}%
\pgfsetdash{}{0pt}%
\pgfpathmoveto{\pgfqpoint{5.510963in}{0.980268in}}%
\pgfpathcurveto{\pgfqpoint{5.519200in}{0.980268in}}{\pgfqpoint{5.527100in}{0.983541in}}{\pgfqpoint{5.532924in}{0.989364in}}%
\pgfpathcurveto{\pgfqpoint{5.538748in}{0.995188in}}{\pgfqpoint{5.542020in}{1.003088in}}{\pgfqpoint{5.542020in}{1.011325in}}%
\pgfpathcurveto{\pgfqpoint{5.542020in}{1.019561in}}{\pgfqpoint{5.538748in}{1.027461in}}{\pgfqpoint{5.532924in}{1.033285in}}%
\pgfpathcurveto{\pgfqpoint{5.527100in}{1.039109in}}{\pgfqpoint{5.519200in}{1.042381in}}{\pgfqpoint{5.510963in}{1.042381in}}%
\pgfpathcurveto{\pgfqpoint{5.502727in}{1.042381in}}{\pgfqpoint{5.494827in}{1.039109in}}{\pgfqpoint{5.489003in}{1.033285in}}%
\pgfpathcurveto{\pgfqpoint{5.483179in}{1.027461in}}{\pgfqpoint{5.479907in}{1.019561in}}{\pgfqpoint{5.479907in}{1.011325in}}%
\pgfpathcurveto{\pgfqpoint{5.479907in}{1.003088in}}{\pgfqpoint{5.483179in}{0.995188in}}{\pgfqpoint{5.489003in}{0.989364in}}%
\pgfpathcurveto{\pgfqpoint{5.494827in}{0.983541in}}{\pgfqpoint{5.502727in}{0.980268in}}{\pgfqpoint{5.510963in}{0.980268in}}%
\pgfpathclose%
\pgfusepath{stroke,fill}%
\end{pgfscope}%
\begin{pgfscope}%
\pgfpathrectangle{\pgfqpoint{0.457963in}{0.528059in}}{\pgfqpoint{6.200000in}{2.285714in}} %
\pgfusepath{clip}%
\pgfsetbuttcap%
\pgfsetroundjoin%
\definecolor{currentfill}{rgb}{0.333333,0.333333,1.000000}%
\pgfsetfillcolor{currentfill}%
\pgfsetlinewidth{1.003750pt}%
\definecolor{currentstroke}{rgb}{0.333333,0.333333,1.000000}%
\pgfsetstrokecolor{currentstroke}%
\pgfsetdash{}{0pt}%
\pgfpathmoveto{\pgfqpoint{0.457963in}{1.803125in}}%
\pgfpathcurveto{\pgfqpoint{0.466200in}{1.803125in}}{\pgfqpoint{0.474100in}{1.806398in}}{\pgfqpoint{0.479924in}{1.812222in}}%
\pgfpathcurveto{\pgfqpoint{0.485748in}{1.818046in}}{\pgfqpoint{0.489020in}{1.825946in}}{\pgfqpoint{0.489020in}{1.834182in}}%
\pgfpathcurveto{\pgfqpoint{0.489020in}{1.842418in}}{\pgfqpoint{0.485748in}{1.850318in}}{\pgfqpoint{0.479924in}{1.856142in}}%
\pgfpathcurveto{\pgfqpoint{0.474100in}{1.861966in}}{\pgfqpoint{0.466200in}{1.865238in}}{\pgfqpoint{0.457963in}{1.865238in}}%
\pgfpathcurveto{\pgfqpoint{0.449727in}{1.865238in}}{\pgfqpoint{0.441827in}{1.861966in}}{\pgfqpoint{0.436003in}{1.856142in}}%
\pgfpathcurveto{\pgfqpoint{0.430179in}{1.850318in}}{\pgfqpoint{0.426907in}{1.842418in}}{\pgfqpoint{0.426907in}{1.834182in}}%
\pgfpathcurveto{\pgfqpoint{0.426907in}{1.825946in}}{\pgfqpoint{0.430179in}{1.818046in}}{\pgfqpoint{0.436003in}{1.812222in}}%
\pgfpathcurveto{\pgfqpoint{0.441827in}{1.806398in}}{\pgfqpoint{0.449727in}{1.803125in}}{\pgfqpoint{0.457963in}{1.803125in}}%
\pgfpathclose%
\pgfusepath{stroke,fill}%
\end{pgfscope}%
\begin{pgfscope}%
\pgfpathrectangle{\pgfqpoint{0.457963in}{0.528059in}}{\pgfqpoint{6.200000in}{2.285714in}} %
\pgfusepath{clip}%
\pgfsetbuttcap%
\pgfsetroundjoin%
\definecolor{currentfill}{rgb}{0.333333,0.333333,1.000000}%
\pgfsetfillcolor{currentfill}%
\pgfsetlinewidth{1.003750pt}%
\definecolor{currentstroke}{rgb}{0.333333,0.333333,1.000000}%
\pgfsetstrokecolor{currentstroke}%
\pgfsetdash{}{0pt}%
\pgfpathmoveto{\pgfqpoint{0.457963in}{1.803125in}}%
\pgfpathcurveto{\pgfqpoint{0.466200in}{1.803125in}}{\pgfqpoint{0.474100in}{1.806398in}}{\pgfqpoint{0.479924in}{1.812222in}}%
\pgfpathcurveto{\pgfqpoint{0.485748in}{1.818046in}}{\pgfqpoint{0.489020in}{1.825946in}}{\pgfqpoint{0.489020in}{1.834182in}}%
\pgfpathcurveto{\pgfqpoint{0.489020in}{1.842418in}}{\pgfqpoint{0.485748in}{1.850318in}}{\pgfqpoint{0.479924in}{1.856142in}}%
\pgfpathcurveto{\pgfqpoint{0.474100in}{1.861966in}}{\pgfqpoint{0.466200in}{1.865238in}}{\pgfqpoint{0.457963in}{1.865238in}}%
\pgfpathcurveto{\pgfqpoint{0.449727in}{1.865238in}}{\pgfqpoint{0.441827in}{1.861966in}}{\pgfqpoint{0.436003in}{1.856142in}}%
\pgfpathcurveto{\pgfqpoint{0.430179in}{1.850318in}}{\pgfqpoint{0.426907in}{1.842418in}}{\pgfqpoint{0.426907in}{1.834182in}}%
\pgfpathcurveto{\pgfqpoint{0.426907in}{1.825946in}}{\pgfqpoint{0.430179in}{1.818046in}}{\pgfqpoint{0.436003in}{1.812222in}}%
\pgfpathcurveto{\pgfqpoint{0.441827in}{1.806398in}}{\pgfqpoint{0.449727in}{1.803125in}}{\pgfqpoint{0.457963in}{1.803125in}}%
\pgfpathclose%
\pgfusepath{stroke,fill}%
\end{pgfscope}%
\begin{pgfscope}%
\pgfpathrectangle{\pgfqpoint{0.457963in}{0.528059in}}{\pgfqpoint{6.200000in}{2.285714in}} %
\pgfusepath{clip}%
\pgfsetbuttcap%
\pgfsetroundjoin%
\definecolor{currentfill}{rgb}{0.333333,0.333333,1.000000}%
\pgfsetfillcolor{currentfill}%
\pgfsetlinewidth{1.003750pt}%
\definecolor{currentstroke}{rgb}{0.333333,0.333333,1.000000}%
\pgfsetstrokecolor{currentstroke}%
\pgfsetdash{}{0pt}%
\pgfpathmoveto{\pgfqpoint{0.478630in}{1.803125in}}%
\pgfpathcurveto{\pgfqpoint{0.486866in}{1.803125in}}{\pgfqpoint{0.494766in}{1.806398in}}{\pgfqpoint{0.500590in}{1.812222in}}%
\pgfpathcurveto{\pgfqpoint{0.506414in}{1.818046in}}{\pgfqpoint{0.509686in}{1.825946in}}{\pgfqpoint{0.509686in}{1.834182in}}%
\pgfpathcurveto{\pgfqpoint{0.509686in}{1.842418in}}{\pgfqpoint{0.506414in}{1.850318in}}{\pgfqpoint{0.500590in}{1.856142in}}%
\pgfpathcurveto{\pgfqpoint{0.494766in}{1.861966in}}{\pgfqpoint{0.486866in}{1.865238in}}{\pgfqpoint{0.478630in}{1.865238in}}%
\pgfpathcurveto{\pgfqpoint{0.470394in}{1.865238in}}{\pgfqpoint{0.462494in}{1.861966in}}{\pgfqpoint{0.456670in}{1.856142in}}%
\pgfpathcurveto{\pgfqpoint{0.450846in}{1.850318in}}{\pgfqpoint{0.447574in}{1.842418in}}{\pgfqpoint{0.447574in}{1.834182in}}%
\pgfpathcurveto{\pgfqpoint{0.447574in}{1.825946in}}{\pgfqpoint{0.450846in}{1.818046in}}{\pgfqpoint{0.456670in}{1.812222in}}%
\pgfpathcurveto{\pgfqpoint{0.462494in}{1.806398in}}{\pgfqpoint{0.470394in}{1.803125in}}{\pgfqpoint{0.478630in}{1.803125in}}%
\pgfpathclose%
\pgfusepath{stroke,fill}%
\end{pgfscope}%
\begin{pgfscope}%
\pgfpathrectangle{\pgfqpoint{0.457963in}{0.528059in}}{\pgfqpoint{6.200000in}{2.285714in}} %
\pgfusepath{clip}%
\pgfsetbuttcap%
\pgfsetroundjoin%
\definecolor{currentfill}{rgb}{0.333333,0.333333,1.000000}%
\pgfsetfillcolor{currentfill}%
\pgfsetlinewidth{1.003750pt}%
\definecolor{currentstroke}{rgb}{0.333333,0.333333,1.000000}%
\pgfsetstrokecolor{currentstroke}%
\pgfsetdash{}{0pt}%
\pgfpathmoveto{\pgfqpoint{0.519963in}{1.790064in}}%
\pgfpathcurveto{\pgfqpoint{0.528200in}{1.790064in}}{\pgfqpoint{0.536100in}{1.793336in}}{\pgfqpoint{0.541924in}{1.799160in}}%
\pgfpathcurveto{\pgfqpoint{0.547748in}{1.804984in}}{\pgfqpoint{0.551020in}{1.812884in}}{\pgfqpoint{0.551020in}{1.821121in}}%
\pgfpathcurveto{\pgfqpoint{0.551020in}{1.829357in}}{\pgfqpoint{0.547748in}{1.837257in}}{\pgfqpoint{0.541924in}{1.843081in}}%
\pgfpathcurveto{\pgfqpoint{0.536100in}{1.848905in}}{\pgfqpoint{0.528200in}{1.852177in}}{\pgfqpoint{0.519963in}{1.852177in}}%
\pgfpathcurveto{\pgfqpoint{0.511727in}{1.852177in}}{\pgfqpoint{0.503827in}{1.848905in}}{\pgfqpoint{0.498003in}{1.843081in}}%
\pgfpathcurveto{\pgfqpoint{0.492179in}{1.837257in}}{\pgfqpoint{0.488907in}{1.829357in}}{\pgfqpoint{0.488907in}{1.821121in}}%
\pgfpathcurveto{\pgfqpoint{0.488907in}{1.812884in}}{\pgfqpoint{0.492179in}{1.804984in}}{\pgfqpoint{0.498003in}{1.799160in}}%
\pgfpathcurveto{\pgfqpoint{0.503827in}{1.793336in}}{\pgfqpoint{0.511727in}{1.790064in}}{\pgfqpoint{0.519963in}{1.790064in}}%
\pgfpathclose%
\pgfusepath{stroke,fill}%
\end{pgfscope}%
\begin{pgfscope}%
\pgfpathrectangle{\pgfqpoint{0.457963in}{0.528059in}}{\pgfqpoint{6.200000in}{2.285714in}} %
\pgfusepath{clip}%
\pgfsetbuttcap%
\pgfsetroundjoin%
\definecolor{currentfill}{rgb}{0.333333,0.333333,1.000000}%
\pgfsetfillcolor{currentfill}%
\pgfsetlinewidth{1.003750pt}%
\definecolor{currentstroke}{rgb}{0.333333,0.333333,1.000000}%
\pgfsetstrokecolor{currentstroke}%
\pgfsetdash{}{0pt}%
\pgfpathmoveto{\pgfqpoint{0.571630in}{1.803125in}}%
\pgfpathcurveto{\pgfqpoint{0.579866in}{1.803125in}}{\pgfqpoint{0.587766in}{1.806398in}}{\pgfqpoint{0.593590in}{1.812222in}}%
\pgfpathcurveto{\pgfqpoint{0.599414in}{1.818046in}}{\pgfqpoint{0.602686in}{1.825946in}}{\pgfqpoint{0.602686in}{1.834182in}}%
\pgfpathcurveto{\pgfqpoint{0.602686in}{1.842418in}}{\pgfqpoint{0.599414in}{1.850318in}}{\pgfqpoint{0.593590in}{1.856142in}}%
\pgfpathcurveto{\pgfqpoint{0.587766in}{1.861966in}}{\pgfqpoint{0.579866in}{1.865238in}}{\pgfqpoint{0.571630in}{1.865238in}}%
\pgfpathcurveto{\pgfqpoint{0.563394in}{1.865238in}}{\pgfqpoint{0.555494in}{1.861966in}}{\pgfqpoint{0.549670in}{1.856142in}}%
\pgfpathcurveto{\pgfqpoint{0.543846in}{1.850318in}}{\pgfqpoint{0.540574in}{1.842418in}}{\pgfqpoint{0.540574in}{1.834182in}}%
\pgfpathcurveto{\pgfqpoint{0.540574in}{1.825946in}}{\pgfqpoint{0.543846in}{1.818046in}}{\pgfqpoint{0.549670in}{1.812222in}}%
\pgfpathcurveto{\pgfqpoint{0.555494in}{1.806398in}}{\pgfqpoint{0.563394in}{1.803125in}}{\pgfqpoint{0.571630in}{1.803125in}}%
\pgfpathclose%
\pgfusepath{stroke,fill}%
\end{pgfscope}%
\begin{pgfscope}%
\pgfpathrectangle{\pgfqpoint{0.457963in}{0.528059in}}{\pgfqpoint{6.200000in}{2.285714in}} %
\pgfusepath{clip}%
\pgfsetbuttcap%
\pgfsetroundjoin%
\definecolor{currentfill}{rgb}{0.333333,0.333333,1.000000}%
\pgfsetfillcolor{currentfill}%
\pgfsetlinewidth{1.003750pt}%
\definecolor{currentstroke}{rgb}{0.333333,0.333333,1.000000}%
\pgfsetstrokecolor{currentstroke}%
\pgfsetdash{}{0pt}%
\pgfpathmoveto{\pgfqpoint{0.643963in}{1.790064in}}%
\pgfpathcurveto{\pgfqpoint{0.652200in}{1.790064in}}{\pgfqpoint{0.660100in}{1.793336in}}{\pgfqpoint{0.665924in}{1.799160in}}%
\pgfpathcurveto{\pgfqpoint{0.671748in}{1.804984in}}{\pgfqpoint{0.675020in}{1.812884in}}{\pgfqpoint{0.675020in}{1.821121in}}%
\pgfpathcurveto{\pgfqpoint{0.675020in}{1.829357in}}{\pgfqpoint{0.671748in}{1.837257in}}{\pgfqpoint{0.665924in}{1.843081in}}%
\pgfpathcurveto{\pgfqpoint{0.660100in}{1.848905in}}{\pgfqpoint{0.652200in}{1.852177in}}{\pgfqpoint{0.643963in}{1.852177in}}%
\pgfpathcurveto{\pgfqpoint{0.635727in}{1.852177in}}{\pgfqpoint{0.627827in}{1.848905in}}{\pgfqpoint{0.622003in}{1.843081in}}%
\pgfpathcurveto{\pgfqpoint{0.616179in}{1.837257in}}{\pgfqpoint{0.612907in}{1.829357in}}{\pgfqpoint{0.612907in}{1.821121in}}%
\pgfpathcurveto{\pgfqpoint{0.612907in}{1.812884in}}{\pgfqpoint{0.616179in}{1.804984in}}{\pgfqpoint{0.622003in}{1.799160in}}%
\pgfpathcurveto{\pgfqpoint{0.627827in}{1.793336in}}{\pgfqpoint{0.635727in}{1.790064in}}{\pgfqpoint{0.643963in}{1.790064in}}%
\pgfpathclose%
\pgfusepath{stroke,fill}%
\end{pgfscope}%
\begin{pgfscope}%
\pgfpathrectangle{\pgfqpoint{0.457963in}{0.528059in}}{\pgfqpoint{6.200000in}{2.285714in}} %
\pgfusepath{clip}%
\pgfsetbuttcap%
\pgfsetroundjoin%
\definecolor{currentfill}{rgb}{0.333333,0.333333,1.000000}%
\pgfsetfillcolor{currentfill}%
\pgfsetlinewidth{1.003750pt}%
\definecolor{currentstroke}{rgb}{0.333333,0.333333,1.000000}%
\pgfsetstrokecolor{currentstroke}%
\pgfsetdash{}{0pt}%
\pgfpathmoveto{\pgfqpoint{0.664630in}{1.777003in}}%
\pgfpathcurveto{\pgfqpoint{0.672866in}{1.777003in}}{\pgfqpoint{0.680766in}{1.780275in}}{\pgfqpoint{0.686590in}{1.786099in}}%
\pgfpathcurveto{\pgfqpoint{0.692414in}{1.791923in}}{\pgfqpoint{0.695686in}{1.799823in}}{\pgfqpoint{0.695686in}{1.808059in}}%
\pgfpathcurveto{\pgfqpoint{0.695686in}{1.816296in}}{\pgfqpoint{0.692414in}{1.824196in}}{\pgfqpoint{0.686590in}{1.830020in}}%
\pgfpathcurveto{\pgfqpoint{0.680766in}{1.835844in}}{\pgfqpoint{0.672866in}{1.839116in}}{\pgfqpoint{0.664630in}{1.839116in}}%
\pgfpathcurveto{\pgfqpoint{0.656394in}{1.839116in}}{\pgfqpoint{0.648494in}{1.835844in}}{\pgfqpoint{0.642670in}{1.830020in}}%
\pgfpathcurveto{\pgfqpoint{0.636846in}{1.824196in}}{\pgfqpoint{0.633574in}{1.816296in}}{\pgfqpoint{0.633574in}{1.808059in}}%
\pgfpathcurveto{\pgfqpoint{0.633574in}{1.799823in}}{\pgfqpoint{0.636846in}{1.791923in}}{\pgfqpoint{0.642670in}{1.786099in}}%
\pgfpathcurveto{\pgfqpoint{0.648494in}{1.780275in}}{\pgfqpoint{0.656394in}{1.777003in}}{\pgfqpoint{0.664630in}{1.777003in}}%
\pgfpathclose%
\pgfusepath{stroke,fill}%
\end{pgfscope}%
\begin{pgfscope}%
\pgfpathrectangle{\pgfqpoint{0.457963in}{0.528059in}}{\pgfqpoint{6.200000in}{2.285714in}} %
\pgfusepath{clip}%
\pgfsetbuttcap%
\pgfsetroundjoin%
\definecolor{currentfill}{rgb}{0.333333,0.333333,1.000000}%
\pgfsetfillcolor{currentfill}%
\pgfsetlinewidth{1.003750pt}%
\definecolor{currentstroke}{rgb}{0.333333,0.333333,1.000000}%
\pgfsetstrokecolor{currentstroke}%
\pgfsetdash{}{0pt}%
\pgfpathmoveto{\pgfqpoint{0.747297in}{1.803125in}}%
\pgfpathcurveto{\pgfqpoint{0.755533in}{1.803125in}}{\pgfqpoint{0.763433in}{1.806398in}}{\pgfqpoint{0.769257in}{1.812222in}}%
\pgfpathcurveto{\pgfqpoint{0.775081in}{1.818046in}}{\pgfqpoint{0.778353in}{1.825946in}}{\pgfqpoint{0.778353in}{1.834182in}}%
\pgfpathcurveto{\pgfqpoint{0.778353in}{1.842418in}}{\pgfqpoint{0.775081in}{1.850318in}}{\pgfqpoint{0.769257in}{1.856142in}}%
\pgfpathcurveto{\pgfqpoint{0.763433in}{1.861966in}}{\pgfqpoint{0.755533in}{1.865238in}}{\pgfqpoint{0.747297in}{1.865238in}}%
\pgfpathcurveto{\pgfqpoint{0.739060in}{1.865238in}}{\pgfqpoint{0.731160in}{1.861966in}}{\pgfqpoint{0.725336in}{1.856142in}}%
\pgfpathcurveto{\pgfqpoint{0.719512in}{1.850318in}}{\pgfqpoint{0.716240in}{1.842418in}}{\pgfqpoint{0.716240in}{1.834182in}}%
\pgfpathcurveto{\pgfqpoint{0.716240in}{1.825946in}}{\pgfqpoint{0.719512in}{1.818046in}}{\pgfqpoint{0.725336in}{1.812222in}}%
\pgfpathcurveto{\pgfqpoint{0.731160in}{1.806398in}}{\pgfqpoint{0.739060in}{1.803125in}}{\pgfqpoint{0.747297in}{1.803125in}}%
\pgfpathclose%
\pgfusepath{stroke,fill}%
\end{pgfscope}%
\begin{pgfscope}%
\pgfpathrectangle{\pgfqpoint{0.457963in}{0.528059in}}{\pgfqpoint{6.200000in}{2.285714in}} %
\pgfusepath{clip}%
\pgfsetbuttcap%
\pgfsetroundjoin%
\definecolor{currentfill}{rgb}{0.333333,0.333333,1.000000}%
\pgfsetfillcolor{currentfill}%
\pgfsetlinewidth{1.003750pt}%
\definecolor{currentstroke}{rgb}{0.333333,0.333333,1.000000}%
\pgfsetstrokecolor{currentstroke}%
\pgfsetdash{}{0pt}%
\pgfpathmoveto{\pgfqpoint{0.995297in}{1.724758in}}%
\pgfpathcurveto{\pgfqpoint{1.003533in}{1.724758in}}{\pgfqpoint{1.011433in}{1.728030in}}{\pgfqpoint{1.017257in}{1.733854in}}%
\pgfpathcurveto{\pgfqpoint{1.023081in}{1.739678in}}{\pgfqpoint{1.026353in}{1.747578in}}{\pgfqpoint{1.026353in}{1.755815in}}%
\pgfpathcurveto{\pgfqpoint{1.026353in}{1.764051in}}{\pgfqpoint{1.023081in}{1.771951in}}{\pgfqpoint{1.017257in}{1.777775in}}%
\pgfpathcurveto{\pgfqpoint{1.011433in}{1.783599in}}{\pgfqpoint{1.003533in}{1.786871in}}{\pgfqpoint{0.995297in}{1.786871in}}%
\pgfpathcurveto{\pgfqpoint{0.987060in}{1.786871in}}{\pgfqpoint{0.979160in}{1.783599in}}{\pgfqpoint{0.973336in}{1.777775in}}%
\pgfpathcurveto{\pgfqpoint{0.967512in}{1.771951in}}{\pgfqpoint{0.964240in}{1.764051in}}{\pgfqpoint{0.964240in}{1.755815in}}%
\pgfpathcurveto{\pgfqpoint{0.964240in}{1.747578in}}{\pgfqpoint{0.967512in}{1.739678in}}{\pgfqpoint{0.973336in}{1.733854in}}%
\pgfpathcurveto{\pgfqpoint{0.979160in}{1.728030in}}{\pgfqpoint{0.987060in}{1.724758in}}{\pgfqpoint{0.995297in}{1.724758in}}%
\pgfpathclose%
\pgfusepath{stroke,fill}%
\end{pgfscope}%
\begin{pgfscope}%
\pgfpathrectangle{\pgfqpoint{0.457963in}{0.528059in}}{\pgfqpoint{6.200000in}{2.285714in}} %
\pgfusepath{clip}%
\pgfsetbuttcap%
\pgfsetroundjoin%
\definecolor{currentfill}{rgb}{0.333333,0.333333,1.000000}%
\pgfsetfillcolor{currentfill}%
\pgfsetlinewidth{1.003750pt}%
\definecolor{currentstroke}{rgb}{0.333333,0.333333,1.000000}%
\pgfsetstrokecolor{currentstroke}%
\pgfsetdash{}{0pt}%
\pgfpathmoveto{\pgfqpoint{1.036630in}{1.777003in}}%
\pgfpathcurveto{\pgfqpoint{1.044866in}{1.777003in}}{\pgfqpoint{1.052766in}{1.780275in}}{\pgfqpoint{1.058590in}{1.786099in}}%
\pgfpathcurveto{\pgfqpoint{1.064414in}{1.791923in}}{\pgfqpoint{1.067686in}{1.799823in}}{\pgfqpoint{1.067686in}{1.808059in}}%
\pgfpathcurveto{\pgfqpoint{1.067686in}{1.816296in}}{\pgfqpoint{1.064414in}{1.824196in}}{\pgfqpoint{1.058590in}{1.830020in}}%
\pgfpathcurveto{\pgfqpoint{1.052766in}{1.835844in}}{\pgfqpoint{1.044866in}{1.839116in}}{\pgfqpoint{1.036630in}{1.839116in}}%
\pgfpathcurveto{\pgfqpoint{1.028394in}{1.839116in}}{\pgfqpoint{1.020494in}{1.835844in}}{\pgfqpoint{1.014670in}{1.830020in}}%
\pgfpathcurveto{\pgfqpoint{1.008846in}{1.824196in}}{\pgfqpoint{1.005574in}{1.816296in}}{\pgfqpoint{1.005574in}{1.808059in}}%
\pgfpathcurveto{\pgfqpoint{1.005574in}{1.799823in}}{\pgfqpoint{1.008846in}{1.791923in}}{\pgfqpoint{1.014670in}{1.786099in}}%
\pgfpathcurveto{\pgfqpoint{1.020494in}{1.780275in}}{\pgfqpoint{1.028394in}{1.777003in}}{\pgfqpoint{1.036630in}{1.777003in}}%
\pgfpathclose%
\pgfusepath{stroke,fill}%
\end{pgfscope}%
\begin{pgfscope}%
\pgfpathrectangle{\pgfqpoint{0.457963in}{0.528059in}}{\pgfqpoint{6.200000in}{2.285714in}} %
\pgfusepath{clip}%
\pgfsetbuttcap%
\pgfsetroundjoin%
\definecolor{currentfill}{rgb}{0.333333,0.333333,1.000000}%
\pgfsetfillcolor{currentfill}%
\pgfsetlinewidth{1.003750pt}%
\definecolor{currentstroke}{rgb}{0.333333,0.333333,1.000000}%
\pgfsetstrokecolor{currentstroke}%
\pgfsetdash{}{0pt}%
\pgfpathmoveto{\pgfqpoint{1.150297in}{1.790064in}}%
\pgfpathcurveto{\pgfqpoint{1.158533in}{1.790064in}}{\pgfqpoint{1.166433in}{1.793336in}}{\pgfqpoint{1.172257in}{1.799160in}}%
\pgfpathcurveto{\pgfqpoint{1.178081in}{1.804984in}}{\pgfqpoint{1.181353in}{1.812884in}}{\pgfqpoint{1.181353in}{1.821121in}}%
\pgfpathcurveto{\pgfqpoint{1.181353in}{1.829357in}}{\pgfqpoint{1.178081in}{1.837257in}}{\pgfqpoint{1.172257in}{1.843081in}}%
\pgfpathcurveto{\pgfqpoint{1.166433in}{1.848905in}}{\pgfqpoint{1.158533in}{1.852177in}}{\pgfqpoint{1.150297in}{1.852177in}}%
\pgfpathcurveto{\pgfqpoint{1.142060in}{1.852177in}}{\pgfqpoint{1.134160in}{1.848905in}}{\pgfqpoint{1.128336in}{1.843081in}}%
\pgfpathcurveto{\pgfqpoint{1.122512in}{1.837257in}}{\pgfqpoint{1.119240in}{1.829357in}}{\pgfqpoint{1.119240in}{1.821121in}}%
\pgfpathcurveto{\pgfqpoint{1.119240in}{1.812884in}}{\pgfqpoint{1.122512in}{1.804984in}}{\pgfqpoint{1.128336in}{1.799160in}}%
\pgfpathcurveto{\pgfqpoint{1.134160in}{1.793336in}}{\pgfqpoint{1.142060in}{1.790064in}}{\pgfqpoint{1.150297in}{1.790064in}}%
\pgfpathclose%
\pgfusepath{stroke,fill}%
\end{pgfscope}%
\begin{pgfscope}%
\pgfpathrectangle{\pgfqpoint{0.457963in}{0.528059in}}{\pgfqpoint{6.200000in}{2.285714in}} %
\pgfusepath{clip}%
\pgfsetbuttcap%
\pgfsetroundjoin%
\definecolor{currentfill}{rgb}{0.333333,0.333333,1.000000}%
\pgfsetfillcolor{currentfill}%
\pgfsetlinewidth{1.003750pt}%
\definecolor{currentstroke}{rgb}{0.333333,0.333333,1.000000}%
\pgfsetstrokecolor{currentstroke}%
\pgfsetdash{}{0pt}%
\pgfpathmoveto{\pgfqpoint{2.007963in}{1.594146in}}%
\pgfpathcurveto{\pgfqpoint{2.016200in}{1.594146in}}{\pgfqpoint{2.024100in}{1.597418in}}{\pgfqpoint{2.029924in}{1.603242in}}%
\pgfpathcurveto{\pgfqpoint{2.035748in}{1.609066in}}{\pgfqpoint{2.039020in}{1.616966in}}{\pgfqpoint{2.039020in}{1.625202in}}%
\pgfpathcurveto{\pgfqpoint{2.039020in}{1.633439in}}{\pgfqpoint{2.035748in}{1.641339in}}{\pgfqpoint{2.029924in}{1.647163in}}%
\pgfpathcurveto{\pgfqpoint{2.024100in}{1.652986in}}{\pgfqpoint{2.016200in}{1.656259in}}{\pgfqpoint{2.007963in}{1.656259in}}%
\pgfpathcurveto{\pgfqpoint{1.999727in}{1.656259in}}{\pgfqpoint{1.991827in}{1.652986in}}{\pgfqpoint{1.986003in}{1.647163in}}%
\pgfpathcurveto{\pgfqpoint{1.980179in}{1.641339in}}{\pgfqpoint{1.976907in}{1.633439in}}{\pgfqpoint{1.976907in}{1.625202in}}%
\pgfpathcurveto{\pgfqpoint{1.976907in}{1.616966in}}{\pgfqpoint{1.980179in}{1.609066in}}{\pgfqpoint{1.986003in}{1.603242in}}%
\pgfpathcurveto{\pgfqpoint{1.991827in}{1.597418in}}{\pgfqpoint{1.999727in}{1.594146in}}{\pgfqpoint{2.007963in}{1.594146in}}%
\pgfpathclose%
\pgfusepath{stroke,fill}%
\end{pgfscope}%
\begin{pgfscope}%
\pgfpathrectangle{\pgfqpoint{0.457963in}{0.528059in}}{\pgfqpoint{6.200000in}{2.285714in}} %
\pgfusepath{clip}%
\pgfsetbuttcap%
\pgfsetroundjoin%
\definecolor{currentfill}{rgb}{0.333333,0.333333,1.000000}%
\pgfsetfillcolor{currentfill}%
\pgfsetlinewidth{1.003750pt}%
\definecolor{currentstroke}{rgb}{0.333333,0.333333,1.000000}%
\pgfsetstrokecolor{currentstroke}%
\pgfsetdash{}{0pt}%
\pgfpathmoveto{\pgfqpoint{2.193963in}{1.803125in}}%
\pgfpathcurveto{\pgfqpoint{2.202200in}{1.803125in}}{\pgfqpoint{2.210100in}{1.806398in}}{\pgfqpoint{2.215924in}{1.812222in}}%
\pgfpathcurveto{\pgfqpoint{2.221748in}{1.818046in}}{\pgfqpoint{2.225020in}{1.825946in}}{\pgfqpoint{2.225020in}{1.834182in}}%
\pgfpathcurveto{\pgfqpoint{2.225020in}{1.842418in}}{\pgfqpoint{2.221748in}{1.850318in}}{\pgfqpoint{2.215924in}{1.856142in}}%
\pgfpathcurveto{\pgfqpoint{2.210100in}{1.861966in}}{\pgfqpoint{2.202200in}{1.865238in}}{\pgfqpoint{2.193963in}{1.865238in}}%
\pgfpathcurveto{\pgfqpoint{2.185727in}{1.865238in}}{\pgfqpoint{2.177827in}{1.861966in}}{\pgfqpoint{2.172003in}{1.856142in}}%
\pgfpathcurveto{\pgfqpoint{2.166179in}{1.850318in}}{\pgfqpoint{2.162907in}{1.842418in}}{\pgfqpoint{2.162907in}{1.834182in}}%
\pgfpathcurveto{\pgfqpoint{2.162907in}{1.825946in}}{\pgfqpoint{2.166179in}{1.818046in}}{\pgfqpoint{2.172003in}{1.812222in}}%
\pgfpathcurveto{\pgfqpoint{2.177827in}{1.806398in}}{\pgfqpoint{2.185727in}{1.803125in}}{\pgfqpoint{2.193963in}{1.803125in}}%
\pgfpathclose%
\pgfusepath{stroke,fill}%
\end{pgfscope}%
\begin{pgfscope}%
\pgfpathrectangle{\pgfqpoint{0.457963in}{0.528059in}}{\pgfqpoint{6.200000in}{2.285714in}} %
\pgfusepath{clip}%
\pgfsetbuttcap%
\pgfsetroundjoin%
\definecolor{currentfill}{rgb}{0.333333,0.333333,1.000000}%
\pgfsetfillcolor{currentfill}%
\pgfsetlinewidth{1.003750pt}%
\definecolor{currentstroke}{rgb}{0.333333,0.333333,1.000000}%
\pgfsetstrokecolor{currentstroke}%
\pgfsetdash{}{0pt}%
\pgfpathmoveto{\pgfqpoint{2.266297in}{1.803125in}}%
\pgfpathcurveto{\pgfqpoint{2.274533in}{1.803125in}}{\pgfqpoint{2.282433in}{1.806398in}}{\pgfqpoint{2.288257in}{1.812222in}}%
\pgfpathcurveto{\pgfqpoint{2.294081in}{1.818046in}}{\pgfqpoint{2.297353in}{1.825946in}}{\pgfqpoint{2.297353in}{1.834182in}}%
\pgfpathcurveto{\pgfqpoint{2.297353in}{1.842418in}}{\pgfqpoint{2.294081in}{1.850318in}}{\pgfqpoint{2.288257in}{1.856142in}}%
\pgfpathcurveto{\pgfqpoint{2.282433in}{1.861966in}}{\pgfqpoint{2.274533in}{1.865238in}}{\pgfqpoint{2.266297in}{1.865238in}}%
\pgfpathcurveto{\pgfqpoint{2.258060in}{1.865238in}}{\pgfqpoint{2.250160in}{1.861966in}}{\pgfqpoint{2.244336in}{1.856142in}}%
\pgfpathcurveto{\pgfqpoint{2.238512in}{1.850318in}}{\pgfqpoint{2.235240in}{1.842418in}}{\pgfqpoint{2.235240in}{1.834182in}}%
\pgfpathcurveto{\pgfqpoint{2.235240in}{1.825946in}}{\pgfqpoint{2.238512in}{1.818046in}}{\pgfqpoint{2.244336in}{1.812222in}}%
\pgfpathcurveto{\pgfqpoint{2.250160in}{1.806398in}}{\pgfqpoint{2.258060in}{1.803125in}}{\pgfqpoint{2.266297in}{1.803125in}}%
\pgfpathclose%
\pgfusepath{stroke,fill}%
\end{pgfscope}%
\begin{pgfscope}%
\pgfpathrectangle{\pgfqpoint{0.457963in}{0.528059in}}{\pgfqpoint{6.200000in}{2.285714in}} %
\pgfusepath{clip}%
\pgfsetbuttcap%
\pgfsetroundjoin%
\definecolor{currentfill}{rgb}{0.333333,0.333333,1.000000}%
\pgfsetfillcolor{currentfill}%
\pgfsetlinewidth{1.003750pt}%
\definecolor{currentstroke}{rgb}{0.333333,0.333333,1.000000}%
\pgfsetstrokecolor{currentstroke}%
\pgfsetdash{}{0pt}%
\pgfpathmoveto{\pgfqpoint{2.948297in}{1.437411in}}%
\pgfpathcurveto{\pgfqpoint{2.956533in}{1.437411in}}{\pgfqpoint{2.964433in}{1.440683in}}{\pgfqpoint{2.970257in}{1.446507in}}%
\pgfpathcurveto{\pgfqpoint{2.976081in}{1.452331in}}{\pgfqpoint{2.979353in}{1.460231in}}{\pgfqpoint{2.979353in}{1.468468in}}%
\pgfpathcurveto{\pgfqpoint{2.979353in}{1.476704in}}{\pgfqpoint{2.976081in}{1.484604in}}{\pgfqpoint{2.970257in}{1.490428in}}%
\pgfpathcurveto{\pgfqpoint{2.964433in}{1.496252in}}{\pgfqpoint{2.956533in}{1.499524in}}{\pgfqpoint{2.948297in}{1.499524in}}%
\pgfpathcurveto{\pgfqpoint{2.940060in}{1.499524in}}{\pgfqpoint{2.932160in}{1.496252in}}{\pgfqpoint{2.926336in}{1.490428in}}%
\pgfpathcurveto{\pgfqpoint{2.920512in}{1.484604in}}{\pgfqpoint{2.917240in}{1.476704in}}{\pgfqpoint{2.917240in}{1.468468in}}%
\pgfpathcurveto{\pgfqpoint{2.917240in}{1.460231in}}{\pgfqpoint{2.920512in}{1.452331in}}{\pgfqpoint{2.926336in}{1.446507in}}%
\pgfpathcurveto{\pgfqpoint{2.932160in}{1.440683in}}{\pgfqpoint{2.940060in}{1.437411in}}{\pgfqpoint{2.948297in}{1.437411in}}%
\pgfpathclose%
\pgfusepath{stroke,fill}%
\end{pgfscope}%
\begin{pgfscope}%
\pgfpathrectangle{\pgfqpoint{0.457963in}{0.528059in}}{\pgfqpoint{6.200000in}{2.285714in}} %
\pgfusepath{clip}%
\pgfsetbuttcap%
\pgfsetroundjoin%
\definecolor{currentfill}{rgb}{0.333333,0.333333,1.000000}%
\pgfsetfillcolor{currentfill}%
\pgfsetlinewidth{1.003750pt}%
\definecolor{currentstroke}{rgb}{0.333333,0.333333,1.000000}%
\pgfsetstrokecolor{currentstroke}%
\pgfsetdash{}{0pt}%
\pgfpathmoveto{\pgfqpoint{3.299630in}{1.724758in}}%
\pgfpathcurveto{\pgfqpoint{3.307866in}{1.724758in}}{\pgfqpoint{3.315766in}{1.728030in}}{\pgfqpoint{3.321590in}{1.733854in}}%
\pgfpathcurveto{\pgfqpoint{3.327414in}{1.739678in}}{\pgfqpoint{3.330686in}{1.747578in}}{\pgfqpoint{3.330686in}{1.755815in}}%
\pgfpathcurveto{\pgfqpoint{3.330686in}{1.764051in}}{\pgfqpoint{3.327414in}{1.771951in}}{\pgfqpoint{3.321590in}{1.777775in}}%
\pgfpathcurveto{\pgfqpoint{3.315766in}{1.783599in}}{\pgfqpoint{3.307866in}{1.786871in}}{\pgfqpoint{3.299630in}{1.786871in}}%
\pgfpathcurveto{\pgfqpoint{3.291394in}{1.786871in}}{\pgfqpoint{3.283494in}{1.783599in}}{\pgfqpoint{3.277670in}{1.777775in}}%
\pgfpathcurveto{\pgfqpoint{3.271846in}{1.771951in}}{\pgfqpoint{3.268574in}{1.764051in}}{\pgfqpoint{3.268574in}{1.755815in}}%
\pgfpathcurveto{\pgfqpoint{3.268574in}{1.747578in}}{\pgfqpoint{3.271846in}{1.739678in}}{\pgfqpoint{3.277670in}{1.733854in}}%
\pgfpathcurveto{\pgfqpoint{3.283494in}{1.728030in}}{\pgfqpoint{3.291394in}{1.724758in}}{\pgfqpoint{3.299630in}{1.724758in}}%
\pgfpathclose%
\pgfusepath{stroke,fill}%
\end{pgfscope}%
\begin{pgfscope}%
\pgfpathrectangle{\pgfqpoint{0.457963in}{0.528059in}}{\pgfqpoint{6.200000in}{2.285714in}} %
\pgfusepath{clip}%
\pgfsetbuttcap%
\pgfsetroundjoin%
\definecolor{currentfill}{rgb}{0.333333,0.333333,1.000000}%
\pgfsetfillcolor{currentfill}%
\pgfsetlinewidth{1.003750pt}%
\definecolor{currentstroke}{rgb}{0.333333,0.333333,1.000000}%
\pgfsetstrokecolor{currentstroke}%
\pgfsetdash{}{0pt}%
\pgfpathmoveto{\pgfqpoint{3.702630in}{1.607207in}}%
\pgfpathcurveto{\pgfqpoint{3.710866in}{1.607207in}}{\pgfqpoint{3.718766in}{1.610479in}}{\pgfqpoint{3.724590in}{1.616303in}}%
\pgfpathcurveto{\pgfqpoint{3.730414in}{1.622127in}}{\pgfqpoint{3.733686in}{1.630027in}}{\pgfqpoint{3.733686in}{1.638264in}}%
\pgfpathcurveto{\pgfqpoint{3.733686in}{1.646500in}}{\pgfqpoint{3.730414in}{1.654400in}}{\pgfqpoint{3.724590in}{1.660224in}}%
\pgfpathcurveto{\pgfqpoint{3.718766in}{1.666048in}}{\pgfqpoint{3.710866in}{1.669320in}}{\pgfqpoint{3.702630in}{1.669320in}}%
\pgfpathcurveto{\pgfqpoint{3.694394in}{1.669320in}}{\pgfqpoint{3.686494in}{1.666048in}}{\pgfqpoint{3.680670in}{1.660224in}}%
\pgfpathcurveto{\pgfqpoint{3.674846in}{1.654400in}}{\pgfqpoint{3.671574in}{1.646500in}}{\pgfqpoint{3.671574in}{1.638264in}}%
\pgfpathcurveto{\pgfqpoint{3.671574in}{1.630027in}}{\pgfqpoint{3.674846in}{1.622127in}}{\pgfqpoint{3.680670in}{1.616303in}}%
\pgfpathcurveto{\pgfqpoint{3.686494in}{1.610479in}}{\pgfqpoint{3.694394in}{1.607207in}}{\pgfqpoint{3.702630in}{1.607207in}}%
\pgfpathclose%
\pgfusepath{stroke,fill}%
\end{pgfscope}%
\begin{pgfscope}%
\pgfpathrectangle{\pgfqpoint{0.457963in}{0.528059in}}{\pgfqpoint{6.200000in}{2.285714in}} %
\pgfusepath{clip}%
\pgfsetbuttcap%
\pgfsetroundjoin%
\definecolor{currentfill}{rgb}{0.333333,0.333333,1.000000}%
\pgfsetfillcolor{currentfill}%
\pgfsetlinewidth{1.003750pt}%
\definecolor{currentstroke}{rgb}{0.333333,0.333333,1.000000}%
\pgfsetstrokecolor{currentstroke}%
\pgfsetdash{}{0pt}%
\pgfpathmoveto{\pgfqpoint{4.115963in}{1.489656in}}%
\pgfpathcurveto{\pgfqpoint{4.124200in}{1.489656in}}{\pgfqpoint{4.132100in}{1.492928in}}{\pgfqpoint{4.137924in}{1.498752in}}%
\pgfpathcurveto{\pgfqpoint{4.143748in}{1.504576in}}{\pgfqpoint{4.147020in}{1.512476in}}{\pgfqpoint{4.147020in}{1.520713in}}%
\pgfpathcurveto{\pgfqpoint{4.147020in}{1.528949in}}{\pgfqpoint{4.143748in}{1.536849in}}{\pgfqpoint{4.137924in}{1.542673in}}%
\pgfpathcurveto{\pgfqpoint{4.132100in}{1.548497in}}{\pgfqpoint{4.124200in}{1.551769in}}{\pgfqpoint{4.115963in}{1.551769in}}%
\pgfpathcurveto{\pgfqpoint{4.107727in}{1.551769in}}{\pgfqpoint{4.099827in}{1.548497in}}{\pgfqpoint{4.094003in}{1.542673in}}%
\pgfpathcurveto{\pgfqpoint{4.088179in}{1.536849in}}{\pgfqpoint{4.084907in}{1.528949in}}{\pgfqpoint{4.084907in}{1.520713in}}%
\pgfpathcurveto{\pgfqpoint{4.084907in}{1.512476in}}{\pgfqpoint{4.088179in}{1.504576in}}{\pgfqpoint{4.094003in}{1.498752in}}%
\pgfpathcurveto{\pgfqpoint{4.099827in}{1.492928in}}{\pgfqpoint{4.107727in}{1.489656in}}{\pgfqpoint{4.115963in}{1.489656in}}%
\pgfpathclose%
\pgfusepath{stroke,fill}%
\end{pgfscope}%
\begin{pgfscope}%
\pgfpathrectangle{\pgfqpoint{0.457963in}{0.528059in}}{\pgfqpoint{6.200000in}{2.285714in}} %
\pgfusepath{clip}%
\pgfsetbuttcap%
\pgfsetroundjoin%
\definecolor{currentfill}{rgb}{0.333333,0.333333,1.000000}%
\pgfsetfillcolor{currentfill}%
\pgfsetlinewidth{1.003750pt}%
\definecolor{currentstroke}{rgb}{0.333333,0.333333,1.000000}%
\pgfsetstrokecolor{currentstroke}%
\pgfsetdash{}{0pt}%
\pgfpathmoveto{\pgfqpoint{4.673963in}{1.293738in}}%
\pgfpathcurveto{\pgfqpoint{4.682200in}{1.293738in}}{\pgfqpoint{4.690100in}{1.297010in}}{\pgfqpoint{4.695924in}{1.302834in}}%
\pgfpathcurveto{\pgfqpoint{4.701748in}{1.308658in}}{\pgfqpoint{4.705020in}{1.316558in}}{\pgfqpoint{4.705020in}{1.324794in}}%
\pgfpathcurveto{\pgfqpoint{4.705020in}{1.333030in}}{\pgfqpoint{4.701748in}{1.340930in}}{\pgfqpoint{4.695924in}{1.346754in}}%
\pgfpathcurveto{\pgfqpoint{4.690100in}{1.352578in}}{\pgfqpoint{4.682200in}{1.355851in}}{\pgfqpoint{4.673963in}{1.355851in}}%
\pgfpathcurveto{\pgfqpoint{4.665727in}{1.355851in}}{\pgfqpoint{4.657827in}{1.352578in}}{\pgfqpoint{4.652003in}{1.346754in}}%
\pgfpathcurveto{\pgfqpoint{4.646179in}{1.340930in}}{\pgfqpoint{4.642907in}{1.333030in}}{\pgfqpoint{4.642907in}{1.324794in}}%
\pgfpathcurveto{\pgfqpoint{4.642907in}{1.316558in}}{\pgfqpoint{4.646179in}{1.308658in}}{\pgfqpoint{4.652003in}{1.302834in}}%
\pgfpathcurveto{\pgfqpoint{4.657827in}{1.297010in}}{\pgfqpoint{4.665727in}{1.293738in}}{\pgfqpoint{4.673963in}{1.293738in}}%
\pgfpathclose%
\pgfusepath{stroke,fill}%
\end{pgfscope}%
\begin{pgfscope}%
\pgfpathrectangle{\pgfqpoint{0.457963in}{0.528059in}}{\pgfqpoint{6.200000in}{2.285714in}} %
\pgfusepath{clip}%
\pgfsetbuttcap%
\pgfsetroundjoin%
\definecolor{currentfill}{rgb}{0.333333,0.333333,1.000000}%
\pgfsetfillcolor{currentfill}%
\pgfsetlinewidth{1.003750pt}%
\definecolor{currentstroke}{rgb}{0.333333,0.333333,1.000000}%
\pgfsetstrokecolor{currentstroke}%
\pgfsetdash{}{0pt}%
\pgfpathmoveto{\pgfqpoint{5.986297in}{1.058636in}}%
\pgfpathcurveto{\pgfqpoint{5.994533in}{1.058636in}}{\pgfqpoint{6.002433in}{1.061908in}}{\pgfqpoint{6.008257in}{1.067732in}}%
\pgfpathcurveto{\pgfqpoint{6.014081in}{1.073556in}}{\pgfqpoint{6.017353in}{1.081456in}}{\pgfqpoint{6.017353in}{1.089692in}}%
\pgfpathcurveto{\pgfqpoint{6.017353in}{1.097928in}}{\pgfqpoint{6.014081in}{1.105828in}}{\pgfqpoint{6.008257in}{1.111652in}}%
\pgfpathcurveto{\pgfqpoint{6.002433in}{1.117476in}}{\pgfqpoint{5.994533in}{1.120749in}}{\pgfqpoint{5.986297in}{1.120749in}}%
\pgfpathcurveto{\pgfqpoint{5.978060in}{1.120749in}}{\pgfqpoint{5.970160in}{1.117476in}}{\pgfqpoint{5.964336in}{1.111652in}}%
\pgfpathcurveto{\pgfqpoint{5.958512in}{1.105828in}}{\pgfqpoint{5.955240in}{1.097928in}}{\pgfqpoint{5.955240in}{1.089692in}}%
\pgfpathcurveto{\pgfqpoint{5.955240in}{1.081456in}}{\pgfqpoint{5.958512in}{1.073556in}}{\pgfqpoint{5.964336in}{1.067732in}}%
\pgfpathcurveto{\pgfqpoint{5.970160in}{1.061908in}}{\pgfqpoint{5.978060in}{1.058636in}}{\pgfqpoint{5.986297in}{1.058636in}}%
\pgfpathclose%
\pgfusepath{stroke,fill}%
\end{pgfscope}%
\begin{pgfscope}%
\pgfpathrectangle{\pgfqpoint{0.457963in}{0.528059in}}{\pgfqpoint{6.200000in}{2.285714in}} %
\pgfusepath{clip}%
\pgfsetbuttcap%
\pgfsetroundjoin%
\definecolor{currentfill}{rgb}{0.166667,0.166667,1.000000}%
\pgfsetfillcolor{currentfill}%
\pgfsetlinewidth{1.003750pt}%
\definecolor{currentstroke}{rgb}{0.166667,0.166667,1.000000}%
\pgfsetstrokecolor{currentstroke}%
\pgfsetdash{}{0pt}%
\pgfpathmoveto{\pgfqpoint{0.457963in}{2.129656in}}%
\pgfpathcurveto{\pgfqpoint{0.466200in}{2.129656in}}{\pgfqpoint{0.474100in}{2.132928in}}{\pgfqpoint{0.479924in}{2.138752in}}%
\pgfpathcurveto{\pgfqpoint{0.485748in}{2.144576in}}{\pgfqpoint{0.489020in}{2.152476in}}{\pgfqpoint{0.489020in}{2.160713in}}%
\pgfpathcurveto{\pgfqpoint{0.489020in}{2.168949in}}{\pgfqpoint{0.485748in}{2.176849in}}{\pgfqpoint{0.479924in}{2.182673in}}%
\pgfpathcurveto{\pgfqpoint{0.474100in}{2.188497in}}{\pgfqpoint{0.466200in}{2.191769in}}{\pgfqpoint{0.457963in}{2.191769in}}%
\pgfpathcurveto{\pgfqpoint{0.449727in}{2.191769in}}{\pgfqpoint{0.441827in}{2.188497in}}{\pgfqpoint{0.436003in}{2.182673in}}%
\pgfpathcurveto{\pgfqpoint{0.430179in}{2.176849in}}{\pgfqpoint{0.426907in}{2.168949in}}{\pgfqpoint{0.426907in}{2.160713in}}%
\pgfpathcurveto{\pgfqpoint{0.426907in}{2.152476in}}{\pgfqpoint{0.430179in}{2.144576in}}{\pgfqpoint{0.436003in}{2.138752in}}%
\pgfpathcurveto{\pgfqpoint{0.441827in}{2.132928in}}{\pgfqpoint{0.449727in}{2.129656in}}{\pgfqpoint{0.457963in}{2.129656in}}%
\pgfpathclose%
\pgfusepath{stroke,fill}%
\end{pgfscope}%
\begin{pgfscope}%
\pgfpathrectangle{\pgfqpoint{0.457963in}{0.528059in}}{\pgfqpoint{6.200000in}{2.285714in}} %
\pgfusepath{clip}%
\pgfsetbuttcap%
\pgfsetroundjoin%
\definecolor{currentfill}{rgb}{0.166667,0.166667,1.000000}%
\pgfsetfillcolor{currentfill}%
\pgfsetlinewidth{1.003750pt}%
\definecolor{currentstroke}{rgb}{0.166667,0.166667,1.000000}%
\pgfsetstrokecolor{currentstroke}%
\pgfsetdash{}{0pt}%
\pgfpathmoveto{\pgfqpoint{0.457963in}{2.129656in}}%
\pgfpathcurveto{\pgfqpoint{0.466200in}{2.129656in}}{\pgfqpoint{0.474100in}{2.132928in}}{\pgfqpoint{0.479924in}{2.138752in}}%
\pgfpathcurveto{\pgfqpoint{0.485748in}{2.144576in}}{\pgfqpoint{0.489020in}{2.152476in}}{\pgfqpoint{0.489020in}{2.160713in}}%
\pgfpathcurveto{\pgfqpoint{0.489020in}{2.168949in}}{\pgfqpoint{0.485748in}{2.176849in}}{\pgfqpoint{0.479924in}{2.182673in}}%
\pgfpathcurveto{\pgfqpoint{0.474100in}{2.188497in}}{\pgfqpoint{0.466200in}{2.191769in}}{\pgfqpoint{0.457963in}{2.191769in}}%
\pgfpathcurveto{\pgfqpoint{0.449727in}{2.191769in}}{\pgfqpoint{0.441827in}{2.188497in}}{\pgfqpoint{0.436003in}{2.182673in}}%
\pgfpathcurveto{\pgfqpoint{0.430179in}{2.176849in}}{\pgfqpoint{0.426907in}{2.168949in}}{\pgfqpoint{0.426907in}{2.160713in}}%
\pgfpathcurveto{\pgfqpoint{0.426907in}{2.152476in}}{\pgfqpoint{0.430179in}{2.144576in}}{\pgfqpoint{0.436003in}{2.138752in}}%
\pgfpathcurveto{\pgfqpoint{0.441827in}{2.132928in}}{\pgfqpoint{0.449727in}{2.129656in}}{\pgfqpoint{0.457963in}{2.129656in}}%
\pgfpathclose%
\pgfusepath{stroke,fill}%
\end{pgfscope}%
\begin{pgfscope}%
\pgfpathrectangle{\pgfqpoint{0.457963in}{0.528059in}}{\pgfqpoint{6.200000in}{2.285714in}} %
\pgfusepath{clip}%
\pgfsetbuttcap%
\pgfsetroundjoin%
\definecolor{currentfill}{rgb}{0.166667,0.166667,1.000000}%
\pgfsetfillcolor{currentfill}%
\pgfsetlinewidth{1.003750pt}%
\definecolor{currentstroke}{rgb}{0.166667,0.166667,1.000000}%
\pgfsetstrokecolor{currentstroke}%
\pgfsetdash{}{0pt}%
\pgfpathmoveto{\pgfqpoint{0.457963in}{2.129656in}}%
\pgfpathcurveto{\pgfqpoint{0.466200in}{2.129656in}}{\pgfqpoint{0.474100in}{2.132928in}}{\pgfqpoint{0.479924in}{2.138752in}}%
\pgfpathcurveto{\pgfqpoint{0.485748in}{2.144576in}}{\pgfqpoint{0.489020in}{2.152476in}}{\pgfqpoint{0.489020in}{2.160713in}}%
\pgfpathcurveto{\pgfqpoint{0.489020in}{2.168949in}}{\pgfqpoint{0.485748in}{2.176849in}}{\pgfqpoint{0.479924in}{2.182673in}}%
\pgfpathcurveto{\pgfqpoint{0.474100in}{2.188497in}}{\pgfqpoint{0.466200in}{2.191769in}}{\pgfqpoint{0.457963in}{2.191769in}}%
\pgfpathcurveto{\pgfqpoint{0.449727in}{2.191769in}}{\pgfqpoint{0.441827in}{2.188497in}}{\pgfqpoint{0.436003in}{2.182673in}}%
\pgfpathcurveto{\pgfqpoint{0.430179in}{2.176849in}}{\pgfqpoint{0.426907in}{2.168949in}}{\pgfqpoint{0.426907in}{2.160713in}}%
\pgfpathcurveto{\pgfqpoint{0.426907in}{2.152476in}}{\pgfqpoint{0.430179in}{2.144576in}}{\pgfqpoint{0.436003in}{2.138752in}}%
\pgfpathcurveto{\pgfqpoint{0.441827in}{2.132928in}}{\pgfqpoint{0.449727in}{2.129656in}}{\pgfqpoint{0.457963in}{2.129656in}}%
\pgfpathclose%
\pgfusepath{stroke,fill}%
\end{pgfscope}%
\begin{pgfscope}%
\pgfpathrectangle{\pgfqpoint{0.457963in}{0.528059in}}{\pgfqpoint{6.200000in}{2.285714in}} %
\pgfusepath{clip}%
\pgfsetbuttcap%
\pgfsetroundjoin%
\definecolor{currentfill}{rgb}{0.166667,0.166667,1.000000}%
\pgfsetfillcolor{currentfill}%
\pgfsetlinewidth{1.003750pt}%
\definecolor{currentstroke}{rgb}{0.166667,0.166667,1.000000}%
\pgfsetstrokecolor{currentstroke}%
\pgfsetdash{}{0pt}%
\pgfpathmoveto{\pgfqpoint{0.488963in}{2.129656in}}%
\pgfpathcurveto{\pgfqpoint{0.497200in}{2.129656in}}{\pgfqpoint{0.505100in}{2.132928in}}{\pgfqpoint{0.510924in}{2.138752in}}%
\pgfpathcurveto{\pgfqpoint{0.516748in}{2.144576in}}{\pgfqpoint{0.520020in}{2.152476in}}{\pgfqpoint{0.520020in}{2.160713in}}%
\pgfpathcurveto{\pgfqpoint{0.520020in}{2.168949in}}{\pgfqpoint{0.516748in}{2.176849in}}{\pgfqpoint{0.510924in}{2.182673in}}%
\pgfpathcurveto{\pgfqpoint{0.505100in}{2.188497in}}{\pgfqpoint{0.497200in}{2.191769in}}{\pgfqpoint{0.488963in}{2.191769in}}%
\pgfpathcurveto{\pgfqpoint{0.480727in}{2.191769in}}{\pgfqpoint{0.472827in}{2.188497in}}{\pgfqpoint{0.467003in}{2.182673in}}%
\pgfpathcurveto{\pgfqpoint{0.461179in}{2.176849in}}{\pgfqpoint{0.457907in}{2.168949in}}{\pgfqpoint{0.457907in}{2.160713in}}%
\pgfpathcurveto{\pgfqpoint{0.457907in}{2.152476in}}{\pgfqpoint{0.461179in}{2.144576in}}{\pgfqpoint{0.467003in}{2.138752in}}%
\pgfpathcurveto{\pgfqpoint{0.472827in}{2.132928in}}{\pgfqpoint{0.480727in}{2.129656in}}{\pgfqpoint{0.488963in}{2.129656in}}%
\pgfpathclose%
\pgfusepath{stroke,fill}%
\end{pgfscope}%
\begin{pgfscope}%
\pgfpathrectangle{\pgfqpoint{0.457963in}{0.528059in}}{\pgfqpoint{6.200000in}{2.285714in}} %
\pgfusepath{clip}%
\pgfsetbuttcap%
\pgfsetroundjoin%
\definecolor{currentfill}{rgb}{0.166667,0.166667,1.000000}%
\pgfsetfillcolor{currentfill}%
\pgfsetlinewidth{1.003750pt}%
\definecolor{currentstroke}{rgb}{0.166667,0.166667,1.000000}%
\pgfsetstrokecolor{currentstroke}%
\pgfsetdash{}{0pt}%
\pgfpathmoveto{\pgfqpoint{0.499297in}{2.129656in}}%
\pgfpathcurveto{\pgfqpoint{0.507533in}{2.129656in}}{\pgfqpoint{0.515433in}{2.132928in}}{\pgfqpoint{0.521257in}{2.138752in}}%
\pgfpathcurveto{\pgfqpoint{0.527081in}{2.144576in}}{\pgfqpoint{0.530353in}{2.152476in}}{\pgfqpoint{0.530353in}{2.160713in}}%
\pgfpathcurveto{\pgfqpoint{0.530353in}{2.168949in}}{\pgfqpoint{0.527081in}{2.176849in}}{\pgfqpoint{0.521257in}{2.182673in}}%
\pgfpathcurveto{\pgfqpoint{0.515433in}{2.188497in}}{\pgfqpoint{0.507533in}{2.191769in}}{\pgfqpoint{0.499297in}{2.191769in}}%
\pgfpathcurveto{\pgfqpoint{0.491060in}{2.191769in}}{\pgfqpoint{0.483160in}{2.188497in}}{\pgfqpoint{0.477336in}{2.182673in}}%
\pgfpathcurveto{\pgfqpoint{0.471512in}{2.176849in}}{\pgfqpoint{0.468240in}{2.168949in}}{\pgfqpoint{0.468240in}{2.160713in}}%
\pgfpathcurveto{\pgfqpoint{0.468240in}{2.152476in}}{\pgfqpoint{0.471512in}{2.144576in}}{\pgfqpoint{0.477336in}{2.138752in}}%
\pgfpathcurveto{\pgfqpoint{0.483160in}{2.132928in}}{\pgfqpoint{0.491060in}{2.129656in}}{\pgfqpoint{0.499297in}{2.129656in}}%
\pgfpathclose%
\pgfusepath{stroke,fill}%
\end{pgfscope}%
\begin{pgfscope}%
\pgfpathrectangle{\pgfqpoint{0.457963in}{0.528059in}}{\pgfqpoint{6.200000in}{2.285714in}} %
\pgfusepath{clip}%
\pgfsetbuttcap%
\pgfsetroundjoin%
\definecolor{currentfill}{rgb}{0.166667,0.166667,1.000000}%
\pgfsetfillcolor{currentfill}%
\pgfsetlinewidth{1.003750pt}%
\definecolor{currentstroke}{rgb}{0.166667,0.166667,1.000000}%
\pgfsetstrokecolor{currentstroke}%
\pgfsetdash{}{0pt}%
\pgfpathmoveto{\pgfqpoint{0.643963in}{2.129656in}}%
\pgfpathcurveto{\pgfqpoint{0.652200in}{2.129656in}}{\pgfqpoint{0.660100in}{2.132928in}}{\pgfqpoint{0.665924in}{2.138752in}}%
\pgfpathcurveto{\pgfqpoint{0.671748in}{2.144576in}}{\pgfqpoint{0.675020in}{2.152476in}}{\pgfqpoint{0.675020in}{2.160713in}}%
\pgfpathcurveto{\pgfqpoint{0.675020in}{2.168949in}}{\pgfqpoint{0.671748in}{2.176849in}}{\pgfqpoint{0.665924in}{2.182673in}}%
\pgfpathcurveto{\pgfqpoint{0.660100in}{2.188497in}}{\pgfqpoint{0.652200in}{2.191769in}}{\pgfqpoint{0.643963in}{2.191769in}}%
\pgfpathcurveto{\pgfqpoint{0.635727in}{2.191769in}}{\pgfqpoint{0.627827in}{2.188497in}}{\pgfqpoint{0.622003in}{2.182673in}}%
\pgfpathcurveto{\pgfqpoint{0.616179in}{2.176849in}}{\pgfqpoint{0.612907in}{2.168949in}}{\pgfqpoint{0.612907in}{2.160713in}}%
\pgfpathcurveto{\pgfqpoint{0.612907in}{2.152476in}}{\pgfqpoint{0.616179in}{2.144576in}}{\pgfqpoint{0.622003in}{2.138752in}}%
\pgfpathcurveto{\pgfqpoint{0.627827in}{2.132928in}}{\pgfqpoint{0.635727in}{2.129656in}}{\pgfqpoint{0.643963in}{2.129656in}}%
\pgfpathclose%
\pgfusepath{stroke,fill}%
\end{pgfscope}%
\begin{pgfscope}%
\pgfpathrectangle{\pgfqpoint{0.457963in}{0.528059in}}{\pgfqpoint{6.200000in}{2.285714in}} %
\pgfusepath{clip}%
\pgfsetbuttcap%
\pgfsetroundjoin%
\definecolor{currentfill}{rgb}{0.166667,0.166667,1.000000}%
\pgfsetfillcolor{currentfill}%
\pgfsetlinewidth{1.003750pt}%
\definecolor{currentstroke}{rgb}{0.166667,0.166667,1.000000}%
\pgfsetstrokecolor{currentstroke}%
\pgfsetdash{}{0pt}%
\pgfpathmoveto{\pgfqpoint{0.664630in}{2.116595in}}%
\pgfpathcurveto{\pgfqpoint{0.672866in}{2.116595in}}{\pgfqpoint{0.680766in}{2.119867in}}{\pgfqpoint{0.686590in}{2.125691in}}%
\pgfpathcurveto{\pgfqpoint{0.692414in}{2.131515in}}{\pgfqpoint{0.695686in}{2.139415in}}{\pgfqpoint{0.695686in}{2.147651in}}%
\pgfpathcurveto{\pgfqpoint{0.695686in}{2.155888in}}{\pgfqpoint{0.692414in}{2.163788in}}{\pgfqpoint{0.686590in}{2.169612in}}%
\pgfpathcurveto{\pgfqpoint{0.680766in}{2.175435in}}{\pgfqpoint{0.672866in}{2.178708in}}{\pgfqpoint{0.664630in}{2.178708in}}%
\pgfpathcurveto{\pgfqpoint{0.656394in}{2.178708in}}{\pgfqpoint{0.648494in}{2.175435in}}{\pgfqpoint{0.642670in}{2.169612in}}%
\pgfpathcurveto{\pgfqpoint{0.636846in}{2.163788in}}{\pgfqpoint{0.633574in}{2.155888in}}{\pgfqpoint{0.633574in}{2.147651in}}%
\pgfpathcurveto{\pgfqpoint{0.633574in}{2.139415in}}{\pgfqpoint{0.636846in}{2.131515in}}{\pgfqpoint{0.642670in}{2.125691in}}%
\pgfpathcurveto{\pgfqpoint{0.648494in}{2.119867in}}{\pgfqpoint{0.656394in}{2.116595in}}{\pgfqpoint{0.664630in}{2.116595in}}%
\pgfpathclose%
\pgfusepath{stroke,fill}%
\end{pgfscope}%
\begin{pgfscope}%
\pgfpathrectangle{\pgfqpoint{0.457963in}{0.528059in}}{\pgfqpoint{6.200000in}{2.285714in}} %
\pgfusepath{clip}%
\pgfsetbuttcap%
\pgfsetroundjoin%
\definecolor{currentfill}{rgb}{0.166667,0.166667,1.000000}%
\pgfsetfillcolor{currentfill}%
\pgfsetlinewidth{1.003750pt}%
\definecolor{currentstroke}{rgb}{0.166667,0.166667,1.000000}%
\pgfsetstrokecolor{currentstroke}%
\pgfsetdash{}{0pt}%
\pgfpathmoveto{\pgfqpoint{0.788630in}{2.129656in}}%
\pgfpathcurveto{\pgfqpoint{0.796866in}{2.129656in}}{\pgfqpoint{0.804766in}{2.132928in}}{\pgfqpoint{0.810590in}{2.138752in}}%
\pgfpathcurveto{\pgfqpoint{0.816414in}{2.144576in}}{\pgfqpoint{0.819686in}{2.152476in}}{\pgfqpoint{0.819686in}{2.160713in}}%
\pgfpathcurveto{\pgfqpoint{0.819686in}{2.168949in}}{\pgfqpoint{0.816414in}{2.176849in}}{\pgfqpoint{0.810590in}{2.182673in}}%
\pgfpathcurveto{\pgfqpoint{0.804766in}{2.188497in}}{\pgfqpoint{0.796866in}{2.191769in}}{\pgfqpoint{0.788630in}{2.191769in}}%
\pgfpathcurveto{\pgfqpoint{0.780394in}{2.191769in}}{\pgfqpoint{0.772494in}{2.188497in}}{\pgfqpoint{0.766670in}{2.182673in}}%
\pgfpathcurveto{\pgfqpoint{0.760846in}{2.176849in}}{\pgfqpoint{0.757574in}{2.168949in}}{\pgfqpoint{0.757574in}{2.160713in}}%
\pgfpathcurveto{\pgfqpoint{0.757574in}{2.152476in}}{\pgfqpoint{0.760846in}{2.144576in}}{\pgfqpoint{0.766670in}{2.138752in}}%
\pgfpathcurveto{\pgfqpoint{0.772494in}{2.132928in}}{\pgfqpoint{0.780394in}{2.129656in}}{\pgfqpoint{0.788630in}{2.129656in}}%
\pgfpathclose%
\pgfusepath{stroke,fill}%
\end{pgfscope}%
\begin{pgfscope}%
\pgfpathrectangle{\pgfqpoint{0.457963in}{0.528059in}}{\pgfqpoint{6.200000in}{2.285714in}} %
\pgfusepath{clip}%
\pgfsetbuttcap%
\pgfsetroundjoin%
\definecolor{currentfill}{rgb}{0.166667,0.166667,1.000000}%
\pgfsetfillcolor{currentfill}%
\pgfsetlinewidth{1.003750pt}%
\definecolor{currentstroke}{rgb}{0.166667,0.166667,1.000000}%
\pgfsetstrokecolor{currentstroke}%
\pgfsetdash{}{0pt}%
\pgfpathmoveto{\pgfqpoint{1.119297in}{2.051289in}}%
\pgfpathcurveto{\pgfqpoint{1.127533in}{2.051289in}}{\pgfqpoint{1.135433in}{2.054561in}}{\pgfqpoint{1.141257in}{2.060385in}}%
\pgfpathcurveto{\pgfqpoint{1.147081in}{2.066209in}}{\pgfqpoint{1.150353in}{2.074109in}}{\pgfqpoint{1.150353in}{2.082345in}}%
\pgfpathcurveto{\pgfqpoint{1.150353in}{2.090581in}}{\pgfqpoint{1.147081in}{2.098481in}}{\pgfqpoint{1.141257in}{2.104305in}}%
\pgfpathcurveto{\pgfqpoint{1.135433in}{2.110129in}}{\pgfqpoint{1.127533in}{2.113402in}}{\pgfqpoint{1.119297in}{2.113402in}}%
\pgfpathcurveto{\pgfqpoint{1.111060in}{2.113402in}}{\pgfqpoint{1.103160in}{2.110129in}}{\pgfqpoint{1.097336in}{2.104305in}}%
\pgfpathcurveto{\pgfqpoint{1.091512in}{2.098481in}}{\pgfqpoint{1.088240in}{2.090581in}}{\pgfqpoint{1.088240in}{2.082345in}}%
\pgfpathcurveto{\pgfqpoint{1.088240in}{2.074109in}}{\pgfqpoint{1.091512in}{2.066209in}}{\pgfqpoint{1.097336in}{2.060385in}}%
\pgfpathcurveto{\pgfqpoint{1.103160in}{2.054561in}}{\pgfqpoint{1.111060in}{2.051289in}}{\pgfqpoint{1.119297in}{2.051289in}}%
\pgfpathclose%
\pgfusepath{stroke,fill}%
\end{pgfscope}%
\begin{pgfscope}%
\pgfpathrectangle{\pgfqpoint{0.457963in}{0.528059in}}{\pgfqpoint{6.200000in}{2.285714in}} %
\pgfusepath{clip}%
\pgfsetbuttcap%
\pgfsetroundjoin%
\definecolor{currentfill}{rgb}{0.166667,0.166667,1.000000}%
\pgfsetfillcolor{currentfill}%
\pgfsetlinewidth{1.003750pt}%
\definecolor{currentstroke}{rgb}{0.166667,0.166667,1.000000}%
\pgfsetstrokecolor{currentstroke}%
\pgfsetdash{}{0pt}%
\pgfpathmoveto{\pgfqpoint{1.522297in}{2.116595in}}%
\pgfpathcurveto{\pgfqpoint{1.530533in}{2.116595in}}{\pgfqpoint{1.538433in}{2.119867in}}{\pgfqpoint{1.544257in}{2.125691in}}%
\pgfpathcurveto{\pgfqpoint{1.550081in}{2.131515in}}{\pgfqpoint{1.553353in}{2.139415in}}{\pgfqpoint{1.553353in}{2.147651in}}%
\pgfpathcurveto{\pgfqpoint{1.553353in}{2.155888in}}{\pgfqpoint{1.550081in}{2.163788in}}{\pgfqpoint{1.544257in}{2.169612in}}%
\pgfpathcurveto{\pgfqpoint{1.538433in}{2.175435in}}{\pgfqpoint{1.530533in}{2.178708in}}{\pgfqpoint{1.522297in}{2.178708in}}%
\pgfpathcurveto{\pgfqpoint{1.514060in}{2.178708in}}{\pgfqpoint{1.506160in}{2.175435in}}{\pgfqpoint{1.500336in}{2.169612in}}%
\pgfpathcurveto{\pgfqpoint{1.494512in}{2.163788in}}{\pgfqpoint{1.491240in}{2.155888in}}{\pgfqpoint{1.491240in}{2.147651in}}%
\pgfpathcurveto{\pgfqpoint{1.491240in}{2.139415in}}{\pgfqpoint{1.494512in}{2.131515in}}{\pgfqpoint{1.500336in}{2.125691in}}%
\pgfpathcurveto{\pgfqpoint{1.506160in}{2.119867in}}{\pgfqpoint{1.514060in}{2.116595in}}{\pgfqpoint{1.522297in}{2.116595in}}%
\pgfpathclose%
\pgfusepath{stroke,fill}%
\end{pgfscope}%
\begin{pgfscope}%
\pgfpathrectangle{\pgfqpoint{0.457963in}{0.528059in}}{\pgfqpoint{6.200000in}{2.285714in}} %
\pgfusepath{clip}%
\pgfsetbuttcap%
\pgfsetroundjoin%
\definecolor{currentfill}{rgb}{0.166667,0.166667,1.000000}%
\pgfsetfillcolor{currentfill}%
\pgfsetlinewidth{1.003750pt}%
\definecolor{currentstroke}{rgb}{0.166667,0.166667,1.000000}%
\pgfsetstrokecolor{currentstroke}%
\pgfsetdash{}{0pt}%
\pgfpathmoveto{\pgfqpoint{1.625630in}{1.959860in}}%
\pgfpathcurveto{\pgfqpoint{1.633866in}{1.959860in}}{\pgfqpoint{1.641766in}{1.963132in}}{\pgfqpoint{1.647590in}{1.968956in}}%
\pgfpathcurveto{\pgfqpoint{1.653414in}{1.974780in}}{\pgfqpoint{1.656686in}{1.982680in}}{\pgfqpoint{1.656686in}{1.990917in}}%
\pgfpathcurveto{\pgfqpoint{1.656686in}{1.999153in}}{\pgfqpoint{1.653414in}{2.007053in}}{\pgfqpoint{1.647590in}{2.012877in}}%
\pgfpathcurveto{\pgfqpoint{1.641766in}{2.018701in}}{\pgfqpoint{1.633866in}{2.021973in}}{\pgfqpoint{1.625630in}{2.021973in}}%
\pgfpathcurveto{\pgfqpoint{1.617394in}{2.021973in}}{\pgfqpoint{1.609494in}{2.018701in}}{\pgfqpoint{1.603670in}{2.012877in}}%
\pgfpathcurveto{\pgfqpoint{1.597846in}{2.007053in}}{\pgfqpoint{1.594574in}{1.999153in}}{\pgfqpoint{1.594574in}{1.990917in}}%
\pgfpathcurveto{\pgfqpoint{1.594574in}{1.982680in}}{\pgfqpoint{1.597846in}{1.974780in}}{\pgfqpoint{1.603670in}{1.968956in}}%
\pgfpathcurveto{\pgfqpoint{1.609494in}{1.963132in}}{\pgfqpoint{1.617394in}{1.959860in}}{\pgfqpoint{1.625630in}{1.959860in}}%
\pgfpathclose%
\pgfusepath{stroke,fill}%
\end{pgfscope}%
\begin{pgfscope}%
\pgfpathrectangle{\pgfqpoint{0.457963in}{0.528059in}}{\pgfqpoint{6.200000in}{2.285714in}} %
\pgfusepath{clip}%
\pgfsetbuttcap%
\pgfsetroundjoin%
\definecolor{currentfill}{rgb}{0.166667,0.166667,1.000000}%
\pgfsetfillcolor{currentfill}%
\pgfsetlinewidth{1.003750pt}%
\definecolor{currentstroke}{rgb}{0.166667,0.166667,1.000000}%
\pgfsetstrokecolor{currentstroke}%
\pgfsetdash{}{0pt}%
\pgfpathmoveto{\pgfqpoint{2.152630in}{2.103534in}}%
\pgfpathcurveto{\pgfqpoint{2.160866in}{2.103534in}}{\pgfqpoint{2.168766in}{2.106806in}}{\pgfqpoint{2.174590in}{2.112630in}}%
\pgfpathcurveto{\pgfqpoint{2.180414in}{2.118454in}}{\pgfqpoint{2.183686in}{2.126354in}}{\pgfqpoint{2.183686in}{2.134590in}}%
\pgfpathcurveto{\pgfqpoint{2.183686in}{2.142826in}}{\pgfqpoint{2.180414in}{2.150726in}}{\pgfqpoint{2.174590in}{2.156550in}}%
\pgfpathcurveto{\pgfqpoint{2.168766in}{2.162374in}}{\pgfqpoint{2.160866in}{2.165647in}}{\pgfqpoint{2.152630in}{2.165647in}}%
\pgfpathcurveto{\pgfqpoint{2.144394in}{2.165647in}}{\pgfqpoint{2.136494in}{2.162374in}}{\pgfqpoint{2.130670in}{2.156550in}}%
\pgfpathcurveto{\pgfqpoint{2.124846in}{2.150726in}}{\pgfqpoint{2.121574in}{2.142826in}}{\pgfqpoint{2.121574in}{2.134590in}}%
\pgfpathcurveto{\pgfqpoint{2.121574in}{2.126354in}}{\pgfqpoint{2.124846in}{2.118454in}}{\pgfqpoint{2.130670in}{2.112630in}}%
\pgfpathcurveto{\pgfqpoint{2.136494in}{2.106806in}}{\pgfqpoint{2.144394in}{2.103534in}}{\pgfqpoint{2.152630in}{2.103534in}}%
\pgfpathclose%
\pgfusepath{stroke,fill}%
\end{pgfscope}%
\begin{pgfscope}%
\pgfpathrectangle{\pgfqpoint{0.457963in}{0.528059in}}{\pgfqpoint{6.200000in}{2.285714in}} %
\pgfusepath{clip}%
\pgfsetbuttcap%
\pgfsetroundjoin%
\definecolor{currentfill}{rgb}{0.166667,0.166667,1.000000}%
\pgfsetfillcolor{currentfill}%
\pgfsetlinewidth{1.003750pt}%
\definecolor{currentstroke}{rgb}{0.166667,0.166667,1.000000}%
\pgfsetstrokecolor{currentstroke}%
\pgfsetdash{}{0pt}%
\pgfpathmoveto{\pgfqpoint{2.751963in}{1.777003in}}%
\pgfpathcurveto{\pgfqpoint{2.760200in}{1.777003in}}{\pgfqpoint{2.768100in}{1.780275in}}{\pgfqpoint{2.773924in}{1.786099in}}%
\pgfpathcurveto{\pgfqpoint{2.779748in}{1.791923in}}{\pgfqpoint{2.783020in}{1.799823in}}{\pgfqpoint{2.783020in}{1.808059in}}%
\pgfpathcurveto{\pgfqpoint{2.783020in}{1.816296in}}{\pgfqpoint{2.779748in}{1.824196in}}{\pgfqpoint{2.773924in}{1.830020in}}%
\pgfpathcurveto{\pgfqpoint{2.768100in}{1.835844in}}{\pgfqpoint{2.760200in}{1.839116in}}{\pgfqpoint{2.751963in}{1.839116in}}%
\pgfpathcurveto{\pgfqpoint{2.743727in}{1.839116in}}{\pgfqpoint{2.735827in}{1.835844in}}{\pgfqpoint{2.730003in}{1.830020in}}%
\pgfpathcurveto{\pgfqpoint{2.724179in}{1.824196in}}{\pgfqpoint{2.720907in}{1.816296in}}{\pgfqpoint{2.720907in}{1.808059in}}%
\pgfpathcurveto{\pgfqpoint{2.720907in}{1.799823in}}{\pgfqpoint{2.724179in}{1.791923in}}{\pgfqpoint{2.730003in}{1.786099in}}%
\pgfpathcurveto{\pgfqpoint{2.735827in}{1.780275in}}{\pgfqpoint{2.743727in}{1.777003in}}{\pgfqpoint{2.751963in}{1.777003in}}%
\pgfpathclose%
\pgfusepath{stroke,fill}%
\end{pgfscope}%
\begin{pgfscope}%
\pgfpathrectangle{\pgfqpoint{0.457963in}{0.528059in}}{\pgfqpoint{6.200000in}{2.285714in}} %
\pgfusepath{clip}%
\pgfsetbuttcap%
\pgfsetroundjoin%
\definecolor{currentfill}{rgb}{0.166667,0.166667,1.000000}%
\pgfsetfillcolor{currentfill}%
\pgfsetlinewidth{1.003750pt}%
\definecolor{currentstroke}{rgb}{0.166667,0.166667,1.000000}%
\pgfsetstrokecolor{currentstroke}%
\pgfsetdash{}{0pt}%
\pgfpathmoveto{\pgfqpoint{2.782963in}{1.959860in}}%
\pgfpathcurveto{\pgfqpoint{2.791200in}{1.959860in}}{\pgfqpoint{2.799100in}{1.963132in}}{\pgfqpoint{2.804924in}{1.968956in}}%
\pgfpathcurveto{\pgfqpoint{2.810748in}{1.974780in}}{\pgfqpoint{2.814020in}{1.982680in}}{\pgfqpoint{2.814020in}{1.990917in}}%
\pgfpathcurveto{\pgfqpoint{2.814020in}{1.999153in}}{\pgfqpoint{2.810748in}{2.007053in}}{\pgfqpoint{2.804924in}{2.012877in}}%
\pgfpathcurveto{\pgfqpoint{2.799100in}{2.018701in}}{\pgfqpoint{2.791200in}{2.021973in}}{\pgfqpoint{2.782963in}{2.021973in}}%
\pgfpathcurveto{\pgfqpoint{2.774727in}{2.021973in}}{\pgfqpoint{2.766827in}{2.018701in}}{\pgfqpoint{2.761003in}{2.012877in}}%
\pgfpathcurveto{\pgfqpoint{2.755179in}{2.007053in}}{\pgfqpoint{2.751907in}{1.999153in}}{\pgfqpoint{2.751907in}{1.990917in}}%
\pgfpathcurveto{\pgfqpoint{2.751907in}{1.982680in}}{\pgfqpoint{2.755179in}{1.974780in}}{\pgfqpoint{2.761003in}{1.968956in}}%
\pgfpathcurveto{\pgfqpoint{2.766827in}{1.963132in}}{\pgfqpoint{2.774727in}{1.959860in}}{\pgfqpoint{2.782963in}{1.959860in}}%
\pgfpathclose%
\pgfusepath{stroke,fill}%
\end{pgfscope}%
\begin{pgfscope}%
\pgfpathrectangle{\pgfqpoint{0.457963in}{0.528059in}}{\pgfqpoint{6.200000in}{2.285714in}} %
\pgfusepath{clip}%
\pgfsetbuttcap%
\pgfsetroundjoin%
\definecolor{currentfill}{rgb}{0.166667,0.166667,1.000000}%
\pgfsetfillcolor{currentfill}%
\pgfsetlinewidth{1.003750pt}%
\definecolor{currentstroke}{rgb}{0.166667,0.166667,1.000000}%
\pgfsetstrokecolor{currentstroke}%
\pgfsetdash{}{0pt}%
\pgfpathmoveto{\pgfqpoint{3.847297in}{1.737819in}}%
\pgfpathcurveto{\pgfqpoint{3.855533in}{1.737819in}}{\pgfqpoint{3.863433in}{1.741092in}}{\pgfqpoint{3.869257in}{1.746916in}}%
\pgfpathcurveto{\pgfqpoint{3.875081in}{1.752739in}}{\pgfqpoint{3.878353in}{1.760639in}}{\pgfqpoint{3.878353in}{1.768876in}}%
\pgfpathcurveto{\pgfqpoint{3.878353in}{1.777112in}}{\pgfqpoint{3.875081in}{1.785012in}}{\pgfqpoint{3.869257in}{1.790836in}}%
\pgfpathcurveto{\pgfqpoint{3.863433in}{1.796660in}}{\pgfqpoint{3.855533in}{1.799932in}}{\pgfqpoint{3.847297in}{1.799932in}}%
\pgfpathcurveto{\pgfqpoint{3.839060in}{1.799932in}}{\pgfqpoint{3.831160in}{1.796660in}}{\pgfqpoint{3.825336in}{1.790836in}}%
\pgfpathcurveto{\pgfqpoint{3.819512in}{1.785012in}}{\pgfqpoint{3.816240in}{1.777112in}}{\pgfqpoint{3.816240in}{1.768876in}}%
\pgfpathcurveto{\pgfqpoint{3.816240in}{1.760639in}}{\pgfqpoint{3.819512in}{1.752739in}}{\pgfqpoint{3.825336in}{1.746916in}}%
\pgfpathcurveto{\pgfqpoint{3.831160in}{1.741092in}}{\pgfqpoint{3.839060in}{1.737819in}}{\pgfqpoint{3.847297in}{1.737819in}}%
\pgfpathclose%
\pgfusepath{stroke,fill}%
\end{pgfscope}%
\begin{pgfscope}%
\pgfpathrectangle{\pgfqpoint{0.457963in}{0.528059in}}{\pgfqpoint{6.200000in}{2.285714in}} %
\pgfusepath{clip}%
\pgfsetbuttcap%
\pgfsetroundjoin%
\definecolor{currentfill}{rgb}{0.166667,0.166667,1.000000}%
\pgfsetfillcolor{currentfill}%
\pgfsetlinewidth{1.003750pt}%
\definecolor{currentstroke}{rgb}{0.166667,0.166667,1.000000}%
\pgfsetstrokecolor{currentstroke}%
\pgfsetdash{}{0pt}%
\pgfpathmoveto{\pgfqpoint{3.867963in}{1.646391in}}%
\pgfpathcurveto{\pgfqpoint{3.876200in}{1.646391in}}{\pgfqpoint{3.884100in}{1.649663in}}{\pgfqpoint{3.889924in}{1.655487in}}%
\pgfpathcurveto{\pgfqpoint{3.895748in}{1.661311in}}{\pgfqpoint{3.899020in}{1.669211in}}{\pgfqpoint{3.899020in}{1.677447in}}%
\pgfpathcurveto{\pgfqpoint{3.899020in}{1.685683in}}{\pgfqpoint{3.895748in}{1.693584in}}{\pgfqpoint{3.889924in}{1.699407in}}%
\pgfpathcurveto{\pgfqpoint{3.884100in}{1.705231in}}{\pgfqpoint{3.876200in}{1.708504in}}{\pgfqpoint{3.867963in}{1.708504in}}%
\pgfpathcurveto{\pgfqpoint{3.859727in}{1.708504in}}{\pgfqpoint{3.851827in}{1.705231in}}{\pgfqpoint{3.846003in}{1.699407in}}%
\pgfpathcurveto{\pgfqpoint{3.840179in}{1.693584in}}{\pgfqpoint{3.836907in}{1.685683in}}{\pgfqpoint{3.836907in}{1.677447in}}%
\pgfpathcurveto{\pgfqpoint{3.836907in}{1.669211in}}{\pgfqpoint{3.840179in}{1.661311in}}{\pgfqpoint{3.846003in}{1.655487in}}%
\pgfpathcurveto{\pgfqpoint{3.851827in}{1.649663in}}{\pgfqpoint{3.859727in}{1.646391in}}{\pgfqpoint{3.867963in}{1.646391in}}%
\pgfpathclose%
\pgfusepath{stroke,fill}%
\end{pgfscope}%
\begin{pgfscope}%
\pgfpathrectangle{\pgfqpoint{0.457963in}{0.528059in}}{\pgfqpoint{6.200000in}{2.285714in}} %
\pgfusepath{clip}%
\pgfsetbuttcap%
\pgfsetroundjoin%
\definecolor{currentfill}{rgb}{0.166667,0.166667,1.000000}%
\pgfsetfillcolor{currentfill}%
\pgfsetlinewidth{1.003750pt}%
\definecolor{currentstroke}{rgb}{0.166667,0.166667,1.000000}%
\pgfsetstrokecolor{currentstroke}%
\pgfsetdash{}{0pt}%
\pgfpathmoveto{\pgfqpoint{4.095297in}{1.646391in}}%
\pgfpathcurveto{\pgfqpoint{4.103533in}{1.646391in}}{\pgfqpoint{4.111433in}{1.649663in}}{\pgfqpoint{4.117257in}{1.655487in}}%
\pgfpathcurveto{\pgfqpoint{4.123081in}{1.661311in}}{\pgfqpoint{4.126353in}{1.669211in}}{\pgfqpoint{4.126353in}{1.677447in}}%
\pgfpathcurveto{\pgfqpoint{4.126353in}{1.685683in}}{\pgfqpoint{4.123081in}{1.693584in}}{\pgfqpoint{4.117257in}{1.699407in}}%
\pgfpathcurveto{\pgfqpoint{4.111433in}{1.705231in}}{\pgfqpoint{4.103533in}{1.708504in}}{\pgfqpoint{4.095297in}{1.708504in}}%
\pgfpathcurveto{\pgfqpoint{4.087060in}{1.708504in}}{\pgfqpoint{4.079160in}{1.705231in}}{\pgfqpoint{4.073336in}{1.699407in}}%
\pgfpathcurveto{\pgfqpoint{4.067512in}{1.693584in}}{\pgfqpoint{4.064240in}{1.685683in}}{\pgfqpoint{4.064240in}{1.677447in}}%
\pgfpathcurveto{\pgfqpoint{4.064240in}{1.669211in}}{\pgfqpoint{4.067512in}{1.661311in}}{\pgfqpoint{4.073336in}{1.655487in}}%
\pgfpathcurveto{\pgfqpoint{4.079160in}{1.649663in}}{\pgfqpoint{4.087060in}{1.646391in}}{\pgfqpoint{4.095297in}{1.646391in}}%
\pgfpathclose%
\pgfusepath{stroke,fill}%
\end{pgfscope}%
\begin{pgfscope}%
\pgfpathrectangle{\pgfqpoint{0.457963in}{0.528059in}}{\pgfqpoint{6.200000in}{2.285714in}} %
\pgfusepath{clip}%
\pgfsetbuttcap%
\pgfsetroundjoin%
\definecolor{currentfill}{rgb}{0.166667,0.166667,1.000000}%
\pgfsetfillcolor{currentfill}%
\pgfsetlinewidth{1.003750pt}%
\definecolor{currentstroke}{rgb}{0.166667,0.166667,1.000000}%
\pgfsetstrokecolor{currentstroke}%
\pgfsetdash{}{0pt}%
\pgfpathmoveto{\pgfqpoint{4.353630in}{1.581085in}}%
\pgfpathcurveto{\pgfqpoint{4.361866in}{1.581085in}}{\pgfqpoint{4.369766in}{1.584357in}}{\pgfqpoint{4.375590in}{1.590181in}}%
\pgfpathcurveto{\pgfqpoint{4.381414in}{1.596005in}}{\pgfqpoint{4.384686in}{1.603905in}}{\pgfqpoint{4.384686in}{1.612141in}}%
\pgfpathcurveto{\pgfqpoint{4.384686in}{1.620377in}}{\pgfqpoint{4.381414in}{1.628277in}}{\pgfqpoint{4.375590in}{1.634101in}}%
\pgfpathcurveto{\pgfqpoint{4.369766in}{1.639925in}}{\pgfqpoint{4.361866in}{1.643198in}}{\pgfqpoint{4.353630in}{1.643198in}}%
\pgfpathcurveto{\pgfqpoint{4.345394in}{1.643198in}}{\pgfqpoint{4.337494in}{1.639925in}}{\pgfqpoint{4.331670in}{1.634101in}}%
\pgfpathcurveto{\pgfqpoint{4.325846in}{1.628277in}}{\pgfqpoint{4.322574in}{1.620377in}}{\pgfqpoint{4.322574in}{1.612141in}}%
\pgfpathcurveto{\pgfqpoint{4.322574in}{1.603905in}}{\pgfqpoint{4.325846in}{1.596005in}}{\pgfqpoint{4.331670in}{1.590181in}}%
\pgfpathcurveto{\pgfqpoint{4.337494in}{1.584357in}}{\pgfqpoint{4.345394in}{1.581085in}}{\pgfqpoint{4.353630in}{1.581085in}}%
\pgfpathclose%
\pgfusepath{stroke,fill}%
\end{pgfscope}%
\begin{pgfscope}%
\pgfpathrectangle{\pgfqpoint{0.457963in}{0.528059in}}{\pgfqpoint{6.200000in}{2.285714in}} %
\pgfusepath{clip}%
\pgfsetbuttcap%
\pgfsetroundjoin%
\definecolor{currentfill}{rgb}{0.166667,0.166667,1.000000}%
\pgfsetfillcolor{currentfill}%
\pgfsetlinewidth{1.003750pt}%
\definecolor{currentstroke}{rgb}{0.166667,0.166667,1.000000}%
\pgfsetstrokecolor{currentstroke}%
\pgfsetdash{}{0pt}%
\pgfpathmoveto{\pgfqpoint{4.673963in}{1.293738in}}%
\pgfpathcurveto{\pgfqpoint{4.682200in}{1.293738in}}{\pgfqpoint{4.690100in}{1.297010in}}{\pgfqpoint{4.695924in}{1.302834in}}%
\pgfpathcurveto{\pgfqpoint{4.701748in}{1.308658in}}{\pgfqpoint{4.705020in}{1.316558in}}{\pgfqpoint{4.705020in}{1.324794in}}%
\pgfpathcurveto{\pgfqpoint{4.705020in}{1.333030in}}{\pgfqpoint{4.701748in}{1.340930in}}{\pgfqpoint{4.695924in}{1.346754in}}%
\pgfpathcurveto{\pgfqpoint{4.690100in}{1.352578in}}{\pgfqpoint{4.682200in}{1.355851in}}{\pgfqpoint{4.673963in}{1.355851in}}%
\pgfpathcurveto{\pgfqpoint{4.665727in}{1.355851in}}{\pgfqpoint{4.657827in}{1.352578in}}{\pgfqpoint{4.652003in}{1.346754in}}%
\pgfpathcurveto{\pgfqpoint{4.646179in}{1.340930in}}{\pgfqpoint{4.642907in}{1.333030in}}{\pgfqpoint{4.642907in}{1.324794in}}%
\pgfpathcurveto{\pgfqpoint{4.642907in}{1.316558in}}{\pgfqpoint{4.646179in}{1.308658in}}{\pgfqpoint{4.652003in}{1.302834in}}%
\pgfpathcurveto{\pgfqpoint{4.657827in}{1.297010in}}{\pgfqpoint{4.665727in}{1.293738in}}{\pgfqpoint{4.673963in}{1.293738in}}%
\pgfpathclose%
\pgfusepath{stroke,fill}%
\end{pgfscope}%
\begin{pgfscope}%
\pgfpathrectangle{\pgfqpoint{0.457963in}{0.528059in}}{\pgfqpoint{6.200000in}{2.285714in}} %
\pgfusepath{clip}%
\pgfsetbuttcap%
\pgfsetroundjoin%
\definecolor{currentfill}{rgb}{0.166667,0.166667,1.000000}%
\pgfsetfillcolor{currentfill}%
\pgfsetlinewidth{1.003750pt}%
\definecolor{currentstroke}{rgb}{0.166667,0.166667,1.000000}%
\pgfsetstrokecolor{currentstroke}%
\pgfsetdash{}{0pt}%
\pgfpathmoveto{\pgfqpoint{5.562630in}{1.097819in}}%
\pgfpathcurveto{\pgfqpoint{5.570866in}{1.097819in}}{\pgfqpoint{5.578766in}{1.101092in}}{\pgfqpoint{5.584590in}{1.106916in}}%
\pgfpathcurveto{\pgfqpoint{5.590414in}{1.112739in}}{\pgfqpoint{5.593686in}{1.120639in}}{\pgfqpoint{5.593686in}{1.128876in}}%
\pgfpathcurveto{\pgfqpoint{5.593686in}{1.137112in}}{\pgfqpoint{5.590414in}{1.145012in}}{\pgfqpoint{5.584590in}{1.150836in}}%
\pgfpathcurveto{\pgfqpoint{5.578766in}{1.156660in}}{\pgfqpoint{5.570866in}{1.159932in}}{\pgfqpoint{5.562630in}{1.159932in}}%
\pgfpathcurveto{\pgfqpoint{5.554394in}{1.159932in}}{\pgfqpoint{5.546494in}{1.156660in}}{\pgfqpoint{5.540670in}{1.150836in}}%
\pgfpathcurveto{\pgfqpoint{5.534846in}{1.145012in}}{\pgfqpoint{5.531574in}{1.137112in}}{\pgfqpoint{5.531574in}{1.128876in}}%
\pgfpathcurveto{\pgfqpoint{5.531574in}{1.120639in}}{\pgfqpoint{5.534846in}{1.112739in}}{\pgfqpoint{5.540670in}{1.106916in}}%
\pgfpathcurveto{\pgfqpoint{5.546494in}{1.101092in}}{\pgfqpoint{5.554394in}{1.097819in}}{\pgfqpoint{5.562630in}{1.097819in}}%
\pgfpathclose%
\pgfusepath{stroke,fill}%
\end{pgfscope}%
\begin{pgfscope}%
\pgfpathrectangle{\pgfqpoint{0.457963in}{0.528059in}}{\pgfqpoint{6.200000in}{2.285714in}} %
\pgfusepath{clip}%
\pgfsetbuttcap%
\pgfsetroundjoin%
\definecolor{currentfill}{rgb}{0.000000,0.000000,1.000000}%
\pgfsetfillcolor{currentfill}%
\pgfsetlinewidth{1.003750pt}%
\definecolor{currentstroke}{rgb}{0.000000,0.000000,1.000000}%
\pgfsetstrokecolor{currentstroke}%
\pgfsetdash{}{0pt}%
\pgfpathmoveto{\pgfqpoint{0.457963in}{2.456187in}}%
\pgfpathcurveto{\pgfqpoint{0.466200in}{2.456187in}}{\pgfqpoint{0.474100in}{2.459459in}}{\pgfqpoint{0.479924in}{2.465283in}}%
\pgfpathcurveto{\pgfqpoint{0.485748in}{2.471107in}}{\pgfqpoint{0.489020in}{2.479007in}}{\pgfqpoint{0.489020in}{2.487243in}}%
\pgfpathcurveto{\pgfqpoint{0.489020in}{2.495479in}}{\pgfqpoint{0.485748in}{2.503379in}}{\pgfqpoint{0.479924in}{2.509203in}}%
\pgfpathcurveto{\pgfqpoint{0.474100in}{2.515027in}}{\pgfqpoint{0.466200in}{2.518300in}}{\pgfqpoint{0.457963in}{2.518300in}}%
\pgfpathcurveto{\pgfqpoint{0.449727in}{2.518300in}}{\pgfqpoint{0.441827in}{2.515027in}}{\pgfqpoint{0.436003in}{2.509203in}}%
\pgfpathcurveto{\pgfqpoint{0.430179in}{2.503379in}}{\pgfqpoint{0.426907in}{2.495479in}}{\pgfqpoint{0.426907in}{2.487243in}}%
\pgfpathcurveto{\pgfqpoint{0.426907in}{2.479007in}}{\pgfqpoint{0.430179in}{2.471107in}}{\pgfqpoint{0.436003in}{2.465283in}}%
\pgfpathcurveto{\pgfqpoint{0.441827in}{2.459459in}}{\pgfqpoint{0.449727in}{2.456187in}}{\pgfqpoint{0.457963in}{2.456187in}}%
\pgfpathclose%
\pgfusepath{stroke,fill}%
\end{pgfscope}%
\begin{pgfscope}%
\pgfpathrectangle{\pgfqpoint{0.457963in}{0.528059in}}{\pgfqpoint{6.200000in}{2.285714in}} %
\pgfusepath{clip}%
\pgfsetbuttcap%
\pgfsetroundjoin%
\definecolor{currentfill}{rgb}{0.000000,0.000000,1.000000}%
\pgfsetfillcolor{currentfill}%
\pgfsetlinewidth{1.003750pt}%
\definecolor{currentstroke}{rgb}{0.000000,0.000000,1.000000}%
\pgfsetstrokecolor{currentstroke}%
\pgfsetdash{}{0pt}%
\pgfpathmoveto{\pgfqpoint{0.457963in}{2.456187in}}%
\pgfpathcurveto{\pgfqpoint{0.466200in}{2.456187in}}{\pgfqpoint{0.474100in}{2.459459in}}{\pgfqpoint{0.479924in}{2.465283in}}%
\pgfpathcurveto{\pgfqpoint{0.485748in}{2.471107in}}{\pgfqpoint{0.489020in}{2.479007in}}{\pgfqpoint{0.489020in}{2.487243in}}%
\pgfpathcurveto{\pgfqpoint{0.489020in}{2.495479in}}{\pgfqpoint{0.485748in}{2.503379in}}{\pgfqpoint{0.479924in}{2.509203in}}%
\pgfpathcurveto{\pgfqpoint{0.474100in}{2.515027in}}{\pgfqpoint{0.466200in}{2.518300in}}{\pgfqpoint{0.457963in}{2.518300in}}%
\pgfpathcurveto{\pgfqpoint{0.449727in}{2.518300in}}{\pgfqpoint{0.441827in}{2.515027in}}{\pgfqpoint{0.436003in}{2.509203in}}%
\pgfpathcurveto{\pgfqpoint{0.430179in}{2.503379in}}{\pgfqpoint{0.426907in}{2.495479in}}{\pgfqpoint{0.426907in}{2.487243in}}%
\pgfpathcurveto{\pgfqpoint{0.426907in}{2.479007in}}{\pgfqpoint{0.430179in}{2.471107in}}{\pgfqpoint{0.436003in}{2.465283in}}%
\pgfpathcurveto{\pgfqpoint{0.441827in}{2.459459in}}{\pgfqpoint{0.449727in}{2.456187in}}{\pgfqpoint{0.457963in}{2.456187in}}%
\pgfpathclose%
\pgfusepath{stroke,fill}%
\end{pgfscope}%
\begin{pgfscope}%
\pgfpathrectangle{\pgfqpoint{0.457963in}{0.528059in}}{\pgfqpoint{6.200000in}{2.285714in}} %
\pgfusepath{clip}%
\pgfsetbuttcap%
\pgfsetroundjoin%
\definecolor{currentfill}{rgb}{0.000000,0.000000,1.000000}%
\pgfsetfillcolor{currentfill}%
\pgfsetlinewidth{1.003750pt}%
\definecolor{currentstroke}{rgb}{0.000000,0.000000,1.000000}%
\pgfsetstrokecolor{currentstroke}%
\pgfsetdash{}{0pt}%
\pgfpathmoveto{\pgfqpoint{0.468297in}{2.456187in}}%
\pgfpathcurveto{\pgfqpoint{0.476533in}{2.456187in}}{\pgfqpoint{0.484433in}{2.459459in}}{\pgfqpoint{0.490257in}{2.465283in}}%
\pgfpathcurveto{\pgfqpoint{0.496081in}{2.471107in}}{\pgfqpoint{0.499353in}{2.479007in}}{\pgfqpoint{0.499353in}{2.487243in}}%
\pgfpathcurveto{\pgfqpoint{0.499353in}{2.495479in}}{\pgfqpoint{0.496081in}{2.503379in}}{\pgfqpoint{0.490257in}{2.509203in}}%
\pgfpathcurveto{\pgfqpoint{0.484433in}{2.515027in}}{\pgfqpoint{0.476533in}{2.518300in}}{\pgfqpoint{0.468297in}{2.518300in}}%
\pgfpathcurveto{\pgfqpoint{0.460060in}{2.518300in}}{\pgfqpoint{0.452160in}{2.515027in}}{\pgfqpoint{0.446336in}{2.509203in}}%
\pgfpathcurveto{\pgfqpoint{0.440512in}{2.503379in}}{\pgfqpoint{0.437240in}{2.495479in}}{\pgfqpoint{0.437240in}{2.487243in}}%
\pgfpathcurveto{\pgfqpoint{0.437240in}{2.479007in}}{\pgfqpoint{0.440512in}{2.471107in}}{\pgfqpoint{0.446336in}{2.465283in}}%
\pgfpathcurveto{\pgfqpoint{0.452160in}{2.459459in}}{\pgfqpoint{0.460060in}{2.456187in}}{\pgfqpoint{0.468297in}{2.456187in}}%
\pgfpathclose%
\pgfusepath{stroke,fill}%
\end{pgfscope}%
\begin{pgfscope}%
\pgfpathrectangle{\pgfqpoint{0.457963in}{0.528059in}}{\pgfqpoint{6.200000in}{2.285714in}} %
\pgfusepath{clip}%
\pgfsetbuttcap%
\pgfsetroundjoin%
\definecolor{currentfill}{rgb}{0.000000,0.000000,1.000000}%
\pgfsetfillcolor{currentfill}%
\pgfsetlinewidth{1.003750pt}%
\definecolor{currentstroke}{rgb}{0.000000,0.000000,1.000000}%
\pgfsetstrokecolor{currentstroke}%
\pgfsetdash{}{0pt}%
\pgfpathmoveto{\pgfqpoint{0.468297in}{2.456187in}}%
\pgfpathcurveto{\pgfqpoint{0.476533in}{2.456187in}}{\pgfqpoint{0.484433in}{2.459459in}}{\pgfqpoint{0.490257in}{2.465283in}}%
\pgfpathcurveto{\pgfqpoint{0.496081in}{2.471107in}}{\pgfqpoint{0.499353in}{2.479007in}}{\pgfqpoint{0.499353in}{2.487243in}}%
\pgfpathcurveto{\pgfqpoint{0.499353in}{2.495479in}}{\pgfqpoint{0.496081in}{2.503379in}}{\pgfqpoint{0.490257in}{2.509203in}}%
\pgfpathcurveto{\pgfqpoint{0.484433in}{2.515027in}}{\pgfqpoint{0.476533in}{2.518300in}}{\pgfqpoint{0.468297in}{2.518300in}}%
\pgfpathcurveto{\pgfqpoint{0.460060in}{2.518300in}}{\pgfqpoint{0.452160in}{2.515027in}}{\pgfqpoint{0.446336in}{2.509203in}}%
\pgfpathcurveto{\pgfqpoint{0.440512in}{2.503379in}}{\pgfqpoint{0.437240in}{2.495479in}}{\pgfqpoint{0.437240in}{2.487243in}}%
\pgfpathcurveto{\pgfqpoint{0.437240in}{2.479007in}}{\pgfqpoint{0.440512in}{2.471107in}}{\pgfqpoint{0.446336in}{2.465283in}}%
\pgfpathcurveto{\pgfqpoint{0.452160in}{2.459459in}}{\pgfqpoint{0.460060in}{2.456187in}}{\pgfqpoint{0.468297in}{2.456187in}}%
\pgfpathclose%
\pgfusepath{stroke,fill}%
\end{pgfscope}%
\begin{pgfscope}%
\pgfpathrectangle{\pgfqpoint{0.457963in}{0.528059in}}{\pgfqpoint{6.200000in}{2.285714in}} %
\pgfusepath{clip}%
\pgfsetbuttcap%
\pgfsetroundjoin%
\definecolor{currentfill}{rgb}{0.000000,0.000000,1.000000}%
\pgfsetfillcolor{currentfill}%
\pgfsetlinewidth{1.003750pt}%
\definecolor{currentstroke}{rgb}{0.000000,0.000000,1.000000}%
\pgfsetstrokecolor{currentstroke}%
\pgfsetdash{}{0pt}%
\pgfpathmoveto{\pgfqpoint{0.499297in}{2.456187in}}%
\pgfpathcurveto{\pgfqpoint{0.507533in}{2.456187in}}{\pgfqpoint{0.515433in}{2.459459in}}{\pgfqpoint{0.521257in}{2.465283in}}%
\pgfpathcurveto{\pgfqpoint{0.527081in}{2.471107in}}{\pgfqpoint{0.530353in}{2.479007in}}{\pgfqpoint{0.530353in}{2.487243in}}%
\pgfpathcurveto{\pgfqpoint{0.530353in}{2.495479in}}{\pgfqpoint{0.527081in}{2.503379in}}{\pgfqpoint{0.521257in}{2.509203in}}%
\pgfpathcurveto{\pgfqpoint{0.515433in}{2.515027in}}{\pgfqpoint{0.507533in}{2.518300in}}{\pgfqpoint{0.499297in}{2.518300in}}%
\pgfpathcurveto{\pgfqpoint{0.491060in}{2.518300in}}{\pgfqpoint{0.483160in}{2.515027in}}{\pgfqpoint{0.477336in}{2.509203in}}%
\pgfpathcurveto{\pgfqpoint{0.471512in}{2.503379in}}{\pgfqpoint{0.468240in}{2.495479in}}{\pgfqpoint{0.468240in}{2.487243in}}%
\pgfpathcurveto{\pgfqpoint{0.468240in}{2.479007in}}{\pgfqpoint{0.471512in}{2.471107in}}{\pgfqpoint{0.477336in}{2.465283in}}%
\pgfpathcurveto{\pgfqpoint{0.483160in}{2.459459in}}{\pgfqpoint{0.491060in}{2.456187in}}{\pgfqpoint{0.499297in}{2.456187in}}%
\pgfpathclose%
\pgfusepath{stroke,fill}%
\end{pgfscope}%
\begin{pgfscope}%
\pgfpathrectangle{\pgfqpoint{0.457963in}{0.528059in}}{\pgfqpoint{6.200000in}{2.285714in}} %
\pgfusepath{clip}%
\pgfsetbuttcap%
\pgfsetroundjoin%
\definecolor{currentfill}{rgb}{0.000000,0.000000,1.000000}%
\pgfsetfillcolor{currentfill}%
\pgfsetlinewidth{1.003750pt}%
\definecolor{currentstroke}{rgb}{0.000000,0.000000,1.000000}%
\pgfsetstrokecolor{currentstroke}%
\pgfsetdash{}{0pt}%
\pgfpathmoveto{\pgfqpoint{0.643963in}{2.430064in}}%
\pgfpathcurveto{\pgfqpoint{0.652200in}{2.430064in}}{\pgfqpoint{0.660100in}{2.433336in}}{\pgfqpoint{0.665924in}{2.439160in}}%
\pgfpathcurveto{\pgfqpoint{0.671748in}{2.444984in}}{\pgfqpoint{0.675020in}{2.452884in}}{\pgfqpoint{0.675020in}{2.461121in}}%
\pgfpathcurveto{\pgfqpoint{0.675020in}{2.469357in}}{\pgfqpoint{0.671748in}{2.477257in}}{\pgfqpoint{0.665924in}{2.483081in}}%
\pgfpathcurveto{\pgfqpoint{0.660100in}{2.488905in}}{\pgfqpoint{0.652200in}{2.492177in}}{\pgfqpoint{0.643963in}{2.492177in}}%
\pgfpathcurveto{\pgfqpoint{0.635727in}{2.492177in}}{\pgfqpoint{0.627827in}{2.488905in}}{\pgfqpoint{0.622003in}{2.483081in}}%
\pgfpathcurveto{\pgfqpoint{0.616179in}{2.477257in}}{\pgfqpoint{0.612907in}{2.469357in}}{\pgfqpoint{0.612907in}{2.461121in}}%
\pgfpathcurveto{\pgfqpoint{0.612907in}{2.452884in}}{\pgfqpoint{0.616179in}{2.444984in}}{\pgfqpoint{0.622003in}{2.439160in}}%
\pgfpathcurveto{\pgfqpoint{0.627827in}{2.433336in}}{\pgfqpoint{0.635727in}{2.430064in}}{\pgfqpoint{0.643963in}{2.430064in}}%
\pgfpathclose%
\pgfusepath{stroke,fill}%
\end{pgfscope}%
\begin{pgfscope}%
\pgfpathrectangle{\pgfqpoint{0.457963in}{0.528059in}}{\pgfqpoint{6.200000in}{2.285714in}} %
\pgfusepath{clip}%
\pgfsetbuttcap%
\pgfsetroundjoin%
\definecolor{currentfill}{rgb}{0.000000,0.000000,1.000000}%
\pgfsetfillcolor{currentfill}%
\pgfsetlinewidth{1.003750pt}%
\definecolor{currentstroke}{rgb}{0.000000,0.000000,1.000000}%
\pgfsetstrokecolor{currentstroke}%
\pgfsetdash{}{0pt}%
\pgfpathmoveto{\pgfqpoint{0.664630in}{2.456187in}}%
\pgfpathcurveto{\pgfqpoint{0.672866in}{2.456187in}}{\pgfqpoint{0.680766in}{2.459459in}}{\pgfqpoint{0.686590in}{2.465283in}}%
\pgfpathcurveto{\pgfqpoint{0.692414in}{2.471107in}}{\pgfqpoint{0.695686in}{2.479007in}}{\pgfqpoint{0.695686in}{2.487243in}}%
\pgfpathcurveto{\pgfqpoint{0.695686in}{2.495479in}}{\pgfqpoint{0.692414in}{2.503379in}}{\pgfqpoint{0.686590in}{2.509203in}}%
\pgfpathcurveto{\pgfqpoint{0.680766in}{2.515027in}}{\pgfqpoint{0.672866in}{2.518300in}}{\pgfqpoint{0.664630in}{2.518300in}}%
\pgfpathcurveto{\pgfqpoint{0.656394in}{2.518300in}}{\pgfqpoint{0.648494in}{2.515027in}}{\pgfqpoint{0.642670in}{2.509203in}}%
\pgfpathcurveto{\pgfqpoint{0.636846in}{2.503379in}}{\pgfqpoint{0.633574in}{2.495479in}}{\pgfqpoint{0.633574in}{2.487243in}}%
\pgfpathcurveto{\pgfqpoint{0.633574in}{2.479007in}}{\pgfqpoint{0.636846in}{2.471107in}}{\pgfqpoint{0.642670in}{2.465283in}}%
\pgfpathcurveto{\pgfqpoint{0.648494in}{2.459459in}}{\pgfqpoint{0.656394in}{2.456187in}}{\pgfqpoint{0.664630in}{2.456187in}}%
\pgfpathclose%
\pgfusepath{stroke,fill}%
\end{pgfscope}%
\begin{pgfscope}%
\pgfpathrectangle{\pgfqpoint{0.457963in}{0.528059in}}{\pgfqpoint{6.200000in}{2.285714in}} %
\pgfusepath{clip}%
\pgfsetbuttcap%
\pgfsetroundjoin%
\definecolor{currentfill}{rgb}{0.000000,0.000000,1.000000}%
\pgfsetfillcolor{currentfill}%
\pgfsetlinewidth{1.003750pt}%
\definecolor{currentstroke}{rgb}{0.000000,0.000000,1.000000}%
\pgfsetstrokecolor{currentstroke}%
\pgfsetdash{}{0pt}%
\pgfpathmoveto{\pgfqpoint{1.201963in}{2.338636in}}%
\pgfpathcurveto{\pgfqpoint{1.210200in}{2.338636in}}{\pgfqpoint{1.218100in}{2.341908in}}{\pgfqpoint{1.223924in}{2.347732in}}%
\pgfpathcurveto{\pgfqpoint{1.229748in}{2.353556in}}{\pgfqpoint{1.233020in}{2.361456in}}{\pgfqpoint{1.233020in}{2.369692in}}%
\pgfpathcurveto{\pgfqpoint{1.233020in}{2.377928in}}{\pgfqpoint{1.229748in}{2.385828in}}{\pgfqpoint{1.223924in}{2.391652in}}%
\pgfpathcurveto{\pgfqpoint{1.218100in}{2.397476in}}{\pgfqpoint{1.210200in}{2.400749in}}{\pgfqpoint{1.201963in}{2.400749in}}%
\pgfpathcurveto{\pgfqpoint{1.193727in}{2.400749in}}{\pgfqpoint{1.185827in}{2.397476in}}{\pgfqpoint{1.180003in}{2.391652in}}%
\pgfpathcurveto{\pgfqpoint{1.174179in}{2.385828in}}{\pgfqpoint{1.170907in}{2.377928in}}{\pgfqpoint{1.170907in}{2.369692in}}%
\pgfpathcurveto{\pgfqpoint{1.170907in}{2.361456in}}{\pgfqpoint{1.174179in}{2.353556in}}{\pgfqpoint{1.180003in}{2.347732in}}%
\pgfpathcurveto{\pgfqpoint{1.185827in}{2.341908in}}{\pgfqpoint{1.193727in}{2.338636in}}{\pgfqpoint{1.201963in}{2.338636in}}%
\pgfpathclose%
\pgfusepath{stroke,fill}%
\end{pgfscope}%
\begin{pgfscope}%
\pgfpathrectangle{\pgfqpoint{0.457963in}{0.528059in}}{\pgfqpoint{6.200000in}{2.285714in}} %
\pgfusepath{clip}%
\pgfsetbuttcap%
\pgfsetroundjoin%
\definecolor{currentfill}{rgb}{0.000000,0.000000,1.000000}%
\pgfsetfillcolor{currentfill}%
\pgfsetlinewidth{1.003750pt}%
\definecolor{currentstroke}{rgb}{0.000000,0.000000,1.000000}%
\pgfsetstrokecolor{currentstroke}%
\pgfsetdash{}{0pt}%
\pgfpathmoveto{\pgfqpoint{1.594630in}{2.443125in}}%
\pgfpathcurveto{\pgfqpoint{1.602866in}{2.443125in}}{\pgfqpoint{1.610766in}{2.446398in}}{\pgfqpoint{1.616590in}{2.452222in}}%
\pgfpathcurveto{\pgfqpoint{1.622414in}{2.458046in}}{\pgfqpoint{1.625686in}{2.465946in}}{\pgfqpoint{1.625686in}{2.474182in}}%
\pgfpathcurveto{\pgfqpoint{1.625686in}{2.482418in}}{\pgfqpoint{1.622414in}{2.490318in}}{\pgfqpoint{1.616590in}{2.496142in}}%
\pgfpathcurveto{\pgfqpoint{1.610766in}{2.501966in}}{\pgfqpoint{1.602866in}{2.505238in}}{\pgfqpoint{1.594630in}{2.505238in}}%
\pgfpathcurveto{\pgfqpoint{1.586394in}{2.505238in}}{\pgfqpoint{1.578494in}{2.501966in}}{\pgfqpoint{1.572670in}{2.496142in}}%
\pgfpathcurveto{\pgfqpoint{1.566846in}{2.490318in}}{\pgfqpoint{1.563574in}{2.482418in}}{\pgfqpoint{1.563574in}{2.474182in}}%
\pgfpathcurveto{\pgfqpoint{1.563574in}{2.465946in}}{\pgfqpoint{1.566846in}{2.458046in}}{\pgfqpoint{1.572670in}{2.452222in}}%
\pgfpathcurveto{\pgfqpoint{1.578494in}{2.446398in}}{\pgfqpoint{1.586394in}{2.443125in}}{\pgfqpoint{1.594630in}{2.443125in}}%
\pgfpathclose%
\pgfusepath{stroke,fill}%
\end{pgfscope}%
\begin{pgfscope}%
\pgfpathrectangle{\pgfqpoint{0.457963in}{0.528059in}}{\pgfqpoint{6.200000in}{2.285714in}} %
\pgfusepath{clip}%
\pgfsetbuttcap%
\pgfsetroundjoin%
\definecolor{currentfill}{rgb}{0.000000,0.000000,1.000000}%
\pgfsetfillcolor{currentfill}%
\pgfsetlinewidth{1.003750pt}%
\definecolor{currentstroke}{rgb}{0.000000,0.000000,1.000000}%
\pgfsetstrokecolor{currentstroke}%
\pgfsetdash{}{0pt}%
\pgfpathmoveto{\pgfqpoint{1.770297in}{2.299452in}}%
\pgfpathcurveto{\pgfqpoint{1.778533in}{2.299452in}}{\pgfqpoint{1.786433in}{2.302724in}}{\pgfqpoint{1.792257in}{2.308548in}}%
\pgfpathcurveto{\pgfqpoint{1.798081in}{2.314372in}}{\pgfqpoint{1.801353in}{2.322272in}}{\pgfqpoint{1.801353in}{2.330508in}}%
\pgfpathcurveto{\pgfqpoint{1.801353in}{2.338745in}}{\pgfqpoint{1.798081in}{2.346645in}}{\pgfqpoint{1.792257in}{2.352469in}}%
\pgfpathcurveto{\pgfqpoint{1.786433in}{2.358293in}}{\pgfqpoint{1.778533in}{2.361565in}}{\pgfqpoint{1.770297in}{2.361565in}}%
\pgfpathcurveto{\pgfqpoint{1.762060in}{2.361565in}}{\pgfqpoint{1.754160in}{2.358293in}}{\pgfqpoint{1.748336in}{2.352469in}}%
\pgfpathcurveto{\pgfqpoint{1.742512in}{2.346645in}}{\pgfqpoint{1.739240in}{2.338745in}}{\pgfqpoint{1.739240in}{2.330508in}}%
\pgfpathcurveto{\pgfqpoint{1.739240in}{2.322272in}}{\pgfqpoint{1.742512in}{2.314372in}}{\pgfqpoint{1.748336in}{2.308548in}}%
\pgfpathcurveto{\pgfqpoint{1.754160in}{2.302724in}}{\pgfqpoint{1.762060in}{2.299452in}}{\pgfqpoint{1.770297in}{2.299452in}}%
\pgfpathclose%
\pgfusepath{stroke,fill}%
\end{pgfscope}%
\begin{pgfscope}%
\pgfpathrectangle{\pgfqpoint{0.457963in}{0.528059in}}{\pgfqpoint{6.200000in}{2.285714in}} %
\pgfusepath{clip}%
\pgfsetbuttcap%
\pgfsetroundjoin%
\definecolor{currentfill}{rgb}{0.000000,0.000000,1.000000}%
\pgfsetfillcolor{currentfill}%
\pgfsetlinewidth{1.003750pt}%
\definecolor{currentstroke}{rgb}{0.000000,0.000000,1.000000}%
\pgfsetstrokecolor{currentstroke}%
\pgfsetdash{}{0pt}%
\pgfpathmoveto{\pgfqpoint{1.966630in}{2.443125in}}%
\pgfpathcurveto{\pgfqpoint{1.974866in}{2.443125in}}{\pgfqpoint{1.982766in}{2.446398in}}{\pgfqpoint{1.988590in}{2.452222in}}%
\pgfpathcurveto{\pgfqpoint{1.994414in}{2.458046in}}{\pgfqpoint{1.997686in}{2.465946in}}{\pgfqpoint{1.997686in}{2.474182in}}%
\pgfpathcurveto{\pgfqpoint{1.997686in}{2.482418in}}{\pgfqpoint{1.994414in}{2.490318in}}{\pgfqpoint{1.988590in}{2.496142in}}%
\pgfpathcurveto{\pgfqpoint{1.982766in}{2.501966in}}{\pgfqpoint{1.974866in}{2.505238in}}{\pgfqpoint{1.966630in}{2.505238in}}%
\pgfpathcurveto{\pgfqpoint{1.958394in}{2.505238in}}{\pgfqpoint{1.950494in}{2.501966in}}{\pgfqpoint{1.944670in}{2.496142in}}%
\pgfpathcurveto{\pgfqpoint{1.938846in}{2.490318in}}{\pgfqpoint{1.935574in}{2.482418in}}{\pgfqpoint{1.935574in}{2.474182in}}%
\pgfpathcurveto{\pgfqpoint{1.935574in}{2.465946in}}{\pgfqpoint{1.938846in}{2.458046in}}{\pgfqpoint{1.944670in}{2.452222in}}%
\pgfpathcurveto{\pgfqpoint{1.950494in}{2.446398in}}{\pgfqpoint{1.958394in}{2.443125in}}{\pgfqpoint{1.966630in}{2.443125in}}%
\pgfpathclose%
\pgfusepath{stroke,fill}%
\end{pgfscope}%
\begin{pgfscope}%
\pgfpathrectangle{\pgfqpoint{0.457963in}{0.528059in}}{\pgfqpoint{6.200000in}{2.285714in}} %
\pgfusepath{clip}%
\pgfsetbuttcap%
\pgfsetroundjoin%
\definecolor{currentfill}{rgb}{0.000000,0.000000,1.000000}%
\pgfsetfillcolor{currentfill}%
\pgfsetlinewidth{1.003750pt}%
\definecolor{currentstroke}{rgb}{0.000000,0.000000,1.000000}%
\pgfsetstrokecolor{currentstroke}%
\pgfsetdash{}{0pt}%
\pgfpathmoveto{\pgfqpoint{2.658963in}{2.129656in}}%
\pgfpathcurveto{\pgfqpoint{2.667200in}{2.129656in}}{\pgfqpoint{2.675100in}{2.132928in}}{\pgfqpoint{2.680924in}{2.138752in}}%
\pgfpathcurveto{\pgfqpoint{2.686748in}{2.144576in}}{\pgfqpoint{2.690020in}{2.152476in}}{\pgfqpoint{2.690020in}{2.160713in}}%
\pgfpathcurveto{\pgfqpoint{2.690020in}{2.168949in}}{\pgfqpoint{2.686748in}{2.176849in}}{\pgfqpoint{2.680924in}{2.182673in}}%
\pgfpathcurveto{\pgfqpoint{2.675100in}{2.188497in}}{\pgfqpoint{2.667200in}{2.191769in}}{\pgfqpoint{2.658963in}{2.191769in}}%
\pgfpathcurveto{\pgfqpoint{2.650727in}{2.191769in}}{\pgfqpoint{2.642827in}{2.188497in}}{\pgfqpoint{2.637003in}{2.182673in}}%
\pgfpathcurveto{\pgfqpoint{2.631179in}{2.176849in}}{\pgfqpoint{2.627907in}{2.168949in}}{\pgfqpoint{2.627907in}{2.160713in}}%
\pgfpathcurveto{\pgfqpoint{2.627907in}{2.152476in}}{\pgfqpoint{2.631179in}{2.144576in}}{\pgfqpoint{2.637003in}{2.138752in}}%
\pgfpathcurveto{\pgfqpoint{2.642827in}{2.132928in}}{\pgfqpoint{2.650727in}{2.129656in}}{\pgfqpoint{2.658963in}{2.129656in}}%
\pgfpathclose%
\pgfusepath{stroke,fill}%
\end{pgfscope}%
\begin{pgfscope}%
\pgfpathrectangle{\pgfqpoint{0.457963in}{0.528059in}}{\pgfqpoint{6.200000in}{2.285714in}} %
\pgfusepath{clip}%
\pgfsetbuttcap%
\pgfsetroundjoin%
\definecolor{currentfill}{rgb}{0.000000,0.000000,1.000000}%
\pgfsetfillcolor{currentfill}%
\pgfsetlinewidth{1.003750pt}%
\definecolor{currentstroke}{rgb}{0.000000,0.000000,1.000000}%
\pgfsetstrokecolor{currentstroke}%
\pgfsetdash{}{0pt}%
\pgfpathmoveto{\pgfqpoint{2.844963in}{2.338636in}}%
\pgfpathcurveto{\pgfqpoint{2.853200in}{2.338636in}}{\pgfqpoint{2.861100in}{2.341908in}}{\pgfqpoint{2.866924in}{2.347732in}}%
\pgfpathcurveto{\pgfqpoint{2.872748in}{2.353556in}}{\pgfqpoint{2.876020in}{2.361456in}}{\pgfqpoint{2.876020in}{2.369692in}}%
\pgfpathcurveto{\pgfqpoint{2.876020in}{2.377928in}}{\pgfqpoint{2.872748in}{2.385828in}}{\pgfqpoint{2.866924in}{2.391652in}}%
\pgfpathcurveto{\pgfqpoint{2.861100in}{2.397476in}}{\pgfqpoint{2.853200in}{2.400749in}}{\pgfqpoint{2.844963in}{2.400749in}}%
\pgfpathcurveto{\pgfqpoint{2.836727in}{2.400749in}}{\pgfqpoint{2.828827in}{2.397476in}}{\pgfqpoint{2.823003in}{2.391652in}}%
\pgfpathcurveto{\pgfqpoint{2.817179in}{2.385828in}}{\pgfqpoint{2.813907in}{2.377928in}}{\pgfqpoint{2.813907in}{2.369692in}}%
\pgfpathcurveto{\pgfqpoint{2.813907in}{2.361456in}}{\pgfqpoint{2.817179in}{2.353556in}}{\pgfqpoint{2.823003in}{2.347732in}}%
\pgfpathcurveto{\pgfqpoint{2.828827in}{2.341908in}}{\pgfqpoint{2.836727in}{2.338636in}}{\pgfqpoint{2.844963in}{2.338636in}}%
\pgfpathclose%
\pgfusepath{stroke,fill}%
\end{pgfscope}%
\begin{pgfscope}%
\pgfpathrectangle{\pgfqpoint{0.457963in}{0.528059in}}{\pgfqpoint{6.200000in}{2.285714in}} %
\pgfusepath{clip}%
\pgfsetbuttcap%
\pgfsetroundjoin%
\definecolor{currentfill}{rgb}{0.000000,0.000000,1.000000}%
\pgfsetfillcolor{currentfill}%
\pgfsetlinewidth{1.003750pt}%
\definecolor{currentstroke}{rgb}{0.000000,0.000000,1.000000}%
\pgfsetstrokecolor{currentstroke}%
\pgfsetdash{}{0pt}%
\pgfpathmoveto{\pgfqpoint{3.723297in}{2.221085in}}%
\pgfpathcurveto{\pgfqpoint{3.731533in}{2.221085in}}{\pgfqpoint{3.739433in}{2.224357in}}{\pgfqpoint{3.745257in}{2.230181in}}%
\pgfpathcurveto{\pgfqpoint{3.751081in}{2.236005in}}{\pgfqpoint{3.754353in}{2.243905in}}{\pgfqpoint{3.754353in}{2.252141in}}%
\pgfpathcurveto{\pgfqpoint{3.754353in}{2.260377in}}{\pgfqpoint{3.751081in}{2.268277in}}{\pgfqpoint{3.745257in}{2.274101in}}%
\pgfpathcurveto{\pgfqpoint{3.739433in}{2.279925in}}{\pgfqpoint{3.731533in}{2.283198in}}{\pgfqpoint{3.723297in}{2.283198in}}%
\pgfpathcurveto{\pgfqpoint{3.715060in}{2.283198in}}{\pgfqpoint{3.707160in}{2.279925in}}{\pgfqpoint{3.701336in}{2.274101in}}%
\pgfpathcurveto{\pgfqpoint{3.695512in}{2.268277in}}{\pgfqpoint{3.692240in}{2.260377in}}{\pgfqpoint{3.692240in}{2.252141in}}%
\pgfpathcurveto{\pgfqpoint{3.692240in}{2.243905in}}{\pgfqpoint{3.695512in}{2.236005in}}{\pgfqpoint{3.701336in}{2.230181in}}%
\pgfpathcurveto{\pgfqpoint{3.707160in}{2.224357in}}{\pgfqpoint{3.715060in}{2.221085in}}{\pgfqpoint{3.723297in}{2.221085in}}%
\pgfpathclose%
\pgfusepath{stroke,fill}%
\end{pgfscope}%
\begin{pgfscope}%
\pgfpathrectangle{\pgfqpoint{0.457963in}{0.528059in}}{\pgfqpoint{6.200000in}{2.285714in}} %
\pgfusepath{clip}%
\pgfsetbuttcap%
\pgfsetroundjoin%
\definecolor{currentfill}{rgb}{0.000000,0.000000,1.000000}%
\pgfsetfillcolor{currentfill}%
\pgfsetlinewidth{1.003750pt}%
\definecolor{currentstroke}{rgb}{0.000000,0.000000,1.000000}%
\pgfsetstrokecolor{currentstroke}%
\pgfsetdash{}{0pt}%
\pgfpathmoveto{\pgfqpoint{3.733630in}{1.672513in}}%
\pgfpathcurveto{\pgfqpoint{3.741866in}{1.672513in}}{\pgfqpoint{3.749766in}{1.675785in}}{\pgfqpoint{3.755590in}{1.681609in}}%
\pgfpathcurveto{\pgfqpoint{3.761414in}{1.687433in}}{\pgfqpoint{3.764686in}{1.695333in}}{\pgfqpoint{3.764686in}{1.703570in}}%
\pgfpathcurveto{\pgfqpoint{3.764686in}{1.711806in}}{\pgfqpoint{3.761414in}{1.719706in}}{\pgfqpoint{3.755590in}{1.725530in}}%
\pgfpathcurveto{\pgfqpoint{3.749766in}{1.731354in}}{\pgfqpoint{3.741866in}{1.734626in}}{\pgfqpoint{3.733630in}{1.734626in}}%
\pgfpathcurveto{\pgfqpoint{3.725394in}{1.734626in}}{\pgfqpoint{3.717494in}{1.731354in}}{\pgfqpoint{3.711670in}{1.725530in}}%
\pgfpathcurveto{\pgfqpoint{3.705846in}{1.719706in}}{\pgfqpoint{3.702574in}{1.711806in}}{\pgfqpoint{3.702574in}{1.703570in}}%
\pgfpathcurveto{\pgfqpoint{3.702574in}{1.695333in}}{\pgfqpoint{3.705846in}{1.687433in}}{\pgfqpoint{3.711670in}{1.681609in}}%
\pgfpathcurveto{\pgfqpoint{3.717494in}{1.675785in}}{\pgfqpoint{3.725394in}{1.672513in}}{\pgfqpoint{3.733630in}{1.672513in}}%
\pgfpathclose%
\pgfusepath{stroke,fill}%
\end{pgfscope}%
\begin{pgfscope}%
\pgfpathrectangle{\pgfqpoint{0.457963in}{0.528059in}}{\pgfqpoint{6.200000in}{2.285714in}} %
\pgfusepath{clip}%
\pgfsetbuttcap%
\pgfsetroundjoin%
\definecolor{currentfill}{rgb}{0.000000,0.000000,1.000000}%
\pgfsetfillcolor{currentfill}%
\pgfsetlinewidth{1.003750pt}%
\definecolor{currentstroke}{rgb}{0.000000,0.000000,1.000000}%
\pgfsetstrokecolor{currentstroke}%
\pgfsetdash{}{0pt}%
\pgfpathmoveto{\pgfqpoint{3.816297in}{1.829248in}}%
\pgfpathcurveto{\pgfqpoint{3.824533in}{1.829248in}}{\pgfqpoint{3.832433in}{1.832520in}}{\pgfqpoint{3.838257in}{1.838344in}}%
\pgfpathcurveto{\pgfqpoint{3.844081in}{1.844168in}}{\pgfqpoint{3.847353in}{1.852068in}}{\pgfqpoint{3.847353in}{1.860304in}}%
\pgfpathcurveto{\pgfqpoint{3.847353in}{1.868541in}}{\pgfqpoint{3.844081in}{1.876441in}}{\pgfqpoint{3.838257in}{1.882265in}}%
\pgfpathcurveto{\pgfqpoint{3.832433in}{1.888089in}}{\pgfqpoint{3.824533in}{1.891361in}}{\pgfqpoint{3.816297in}{1.891361in}}%
\pgfpathcurveto{\pgfqpoint{3.808060in}{1.891361in}}{\pgfqpoint{3.800160in}{1.888089in}}{\pgfqpoint{3.794336in}{1.882265in}}%
\pgfpathcurveto{\pgfqpoint{3.788512in}{1.876441in}}{\pgfqpoint{3.785240in}{1.868541in}}{\pgfqpoint{3.785240in}{1.860304in}}%
\pgfpathcurveto{\pgfqpoint{3.785240in}{1.852068in}}{\pgfqpoint{3.788512in}{1.844168in}}{\pgfqpoint{3.794336in}{1.838344in}}%
\pgfpathcurveto{\pgfqpoint{3.800160in}{1.832520in}}{\pgfqpoint{3.808060in}{1.829248in}}{\pgfqpoint{3.816297in}{1.829248in}}%
\pgfpathclose%
\pgfusepath{stroke,fill}%
\end{pgfscope}%
\begin{pgfscope}%
\pgfpathrectangle{\pgfqpoint{0.457963in}{0.528059in}}{\pgfqpoint{6.200000in}{2.285714in}} %
\pgfusepath{clip}%
\pgfsetbuttcap%
\pgfsetroundjoin%
\definecolor{currentfill}{rgb}{0.000000,0.000000,1.000000}%
\pgfsetfillcolor{currentfill}%
\pgfsetlinewidth{1.003750pt}%
\definecolor{currentstroke}{rgb}{0.000000,0.000000,1.000000}%
\pgfsetstrokecolor{currentstroke}%
\pgfsetdash{}{0pt}%
\pgfpathmoveto{\pgfqpoint{4.157297in}{1.554962in}}%
\pgfpathcurveto{\pgfqpoint{4.165533in}{1.554962in}}{\pgfqpoint{4.173433in}{1.558234in}}{\pgfqpoint{4.179257in}{1.564058in}}%
\pgfpathcurveto{\pgfqpoint{4.185081in}{1.569882in}}{\pgfqpoint{4.188353in}{1.577782in}}{\pgfqpoint{4.188353in}{1.586019in}}%
\pgfpathcurveto{\pgfqpoint{4.188353in}{1.594255in}}{\pgfqpoint{4.185081in}{1.602155in}}{\pgfqpoint{4.179257in}{1.607979in}}%
\pgfpathcurveto{\pgfqpoint{4.173433in}{1.613803in}}{\pgfqpoint{4.165533in}{1.617075in}}{\pgfqpoint{4.157297in}{1.617075in}}%
\pgfpathcurveto{\pgfqpoint{4.149060in}{1.617075in}}{\pgfqpoint{4.141160in}{1.613803in}}{\pgfqpoint{4.135336in}{1.607979in}}%
\pgfpathcurveto{\pgfqpoint{4.129512in}{1.602155in}}{\pgfqpoint{4.126240in}{1.594255in}}{\pgfqpoint{4.126240in}{1.586019in}}%
\pgfpathcurveto{\pgfqpoint{4.126240in}{1.577782in}}{\pgfqpoint{4.129512in}{1.569882in}}{\pgfqpoint{4.135336in}{1.564058in}}%
\pgfpathcurveto{\pgfqpoint{4.141160in}{1.558234in}}{\pgfqpoint{4.149060in}{1.554962in}}{\pgfqpoint{4.157297in}{1.554962in}}%
\pgfpathclose%
\pgfusepath{stroke,fill}%
\end{pgfscope}%
\begin{pgfscope}%
\pgfpathrectangle{\pgfqpoint{0.457963in}{0.528059in}}{\pgfqpoint{6.200000in}{2.285714in}} %
\pgfusepath{clip}%
\pgfsetbuttcap%
\pgfsetroundjoin%
\definecolor{currentfill}{rgb}{0.000000,0.000000,1.000000}%
\pgfsetfillcolor{currentfill}%
\pgfsetlinewidth{1.003750pt}%
\definecolor{currentstroke}{rgb}{0.000000,0.000000,1.000000}%
\pgfsetstrokecolor{currentstroke}%
\pgfsetdash{}{0pt}%
\pgfpathmoveto{\pgfqpoint{4.673963in}{1.293738in}}%
\pgfpathcurveto{\pgfqpoint{4.682200in}{1.293738in}}{\pgfqpoint{4.690100in}{1.297010in}}{\pgfqpoint{4.695924in}{1.302834in}}%
\pgfpathcurveto{\pgfqpoint{4.701748in}{1.308658in}}{\pgfqpoint{4.705020in}{1.316558in}}{\pgfqpoint{4.705020in}{1.324794in}}%
\pgfpathcurveto{\pgfqpoint{4.705020in}{1.333030in}}{\pgfqpoint{4.701748in}{1.340930in}}{\pgfqpoint{4.695924in}{1.346754in}}%
\pgfpathcurveto{\pgfqpoint{4.690100in}{1.352578in}}{\pgfqpoint{4.682200in}{1.355851in}}{\pgfqpoint{4.673963in}{1.355851in}}%
\pgfpathcurveto{\pgfqpoint{4.665727in}{1.355851in}}{\pgfqpoint{4.657827in}{1.352578in}}{\pgfqpoint{4.652003in}{1.346754in}}%
\pgfpathcurveto{\pgfqpoint{4.646179in}{1.340930in}}{\pgfqpoint{4.642907in}{1.333030in}}{\pgfqpoint{4.642907in}{1.324794in}}%
\pgfpathcurveto{\pgfqpoint{4.642907in}{1.316558in}}{\pgfqpoint{4.646179in}{1.308658in}}{\pgfqpoint{4.652003in}{1.302834in}}%
\pgfpathcurveto{\pgfqpoint{4.657827in}{1.297010in}}{\pgfqpoint{4.665727in}{1.293738in}}{\pgfqpoint{4.673963in}{1.293738in}}%
\pgfpathclose%
\pgfusepath{stroke,fill}%
\end{pgfscope}%
\begin{pgfscope}%
\pgfpathrectangle{\pgfqpoint{0.457963in}{0.528059in}}{\pgfqpoint{6.200000in}{2.285714in}} %
\pgfusepath{clip}%
\pgfsetbuttcap%
\pgfsetroundjoin%
\definecolor{currentfill}{rgb}{0.000000,0.000000,1.000000}%
\pgfsetfillcolor{currentfill}%
\pgfsetlinewidth{1.003750pt}%
\definecolor{currentstroke}{rgb}{0.000000,0.000000,1.000000}%
\pgfsetstrokecolor{currentstroke}%
\pgfsetdash{}{0pt}%
\pgfpathmoveto{\pgfqpoint{4.694630in}{1.829248in}}%
\pgfpathcurveto{\pgfqpoint{4.702866in}{1.829248in}}{\pgfqpoint{4.710766in}{1.832520in}}{\pgfqpoint{4.716590in}{1.838344in}}%
\pgfpathcurveto{\pgfqpoint{4.722414in}{1.844168in}}{\pgfqpoint{4.725686in}{1.852068in}}{\pgfqpoint{4.725686in}{1.860304in}}%
\pgfpathcurveto{\pgfqpoint{4.725686in}{1.868541in}}{\pgfqpoint{4.722414in}{1.876441in}}{\pgfqpoint{4.716590in}{1.882265in}}%
\pgfpathcurveto{\pgfqpoint{4.710766in}{1.888089in}}{\pgfqpoint{4.702866in}{1.891361in}}{\pgfqpoint{4.694630in}{1.891361in}}%
\pgfpathcurveto{\pgfqpoint{4.686394in}{1.891361in}}{\pgfqpoint{4.678494in}{1.888089in}}{\pgfqpoint{4.672670in}{1.882265in}}%
\pgfpathcurveto{\pgfqpoint{4.666846in}{1.876441in}}{\pgfqpoint{4.663574in}{1.868541in}}{\pgfqpoint{4.663574in}{1.860304in}}%
\pgfpathcurveto{\pgfqpoint{4.663574in}{1.852068in}}{\pgfqpoint{4.666846in}{1.844168in}}{\pgfqpoint{4.672670in}{1.838344in}}%
\pgfpathcurveto{\pgfqpoint{4.678494in}{1.832520in}}{\pgfqpoint{4.686394in}{1.829248in}}{\pgfqpoint{4.694630in}{1.829248in}}%
\pgfpathclose%
\pgfusepath{stroke,fill}%
\end{pgfscope}%
\begin{pgfscope}%
\pgfpathrectangle{\pgfqpoint{0.457963in}{0.528059in}}{\pgfqpoint{6.200000in}{2.285714in}} %
\pgfusepath{clip}%
\pgfsetbuttcap%
\pgfsetroundjoin%
\definecolor{currentfill}{rgb}{0.000000,0.000000,1.000000}%
\pgfsetfillcolor{currentfill}%
\pgfsetlinewidth{1.003750pt}%
\definecolor{currentstroke}{rgb}{0.000000,0.000000,1.000000}%
\pgfsetstrokecolor{currentstroke}%
\pgfsetdash{}{0pt}%
\pgfpathmoveto{\pgfqpoint{5.562630in}{1.084758in}}%
\pgfpathcurveto{\pgfqpoint{5.570866in}{1.084758in}}{\pgfqpoint{5.578766in}{1.088030in}}{\pgfqpoint{5.584590in}{1.093854in}}%
\pgfpathcurveto{\pgfqpoint{5.590414in}{1.099678in}}{\pgfqpoint{5.593686in}{1.107578in}}{\pgfqpoint{5.593686in}{1.115815in}}%
\pgfpathcurveto{\pgfqpoint{5.593686in}{1.124051in}}{\pgfqpoint{5.590414in}{1.131951in}}{\pgfqpoint{5.584590in}{1.137775in}}%
\pgfpathcurveto{\pgfqpoint{5.578766in}{1.143599in}}{\pgfqpoint{5.570866in}{1.146871in}}{\pgfqpoint{5.562630in}{1.146871in}}%
\pgfpathcurveto{\pgfqpoint{5.554394in}{1.146871in}}{\pgfqpoint{5.546494in}{1.143599in}}{\pgfqpoint{5.540670in}{1.137775in}}%
\pgfpathcurveto{\pgfqpoint{5.534846in}{1.131951in}}{\pgfqpoint{5.531574in}{1.124051in}}{\pgfqpoint{5.531574in}{1.115815in}}%
\pgfpathcurveto{\pgfqpoint{5.531574in}{1.107578in}}{\pgfqpoint{5.534846in}{1.099678in}}{\pgfqpoint{5.540670in}{1.093854in}}%
\pgfpathcurveto{\pgfqpoint{5.546494in}{1.088030in}}{\pgfqpoint{5.554394in}{1.084758in}}{\pgfqpoint{5.562630in}{1.084758in}}%
\pgfpathclose%
\pgfusepath{stroke,fill}%
\end{pgfscope}%
\begin{pgfscope}%
\pgfpathrectangle{\pgfqpoint{0.457963in}{0.528059in}}{\pgfqpoint{6.200000in}{2.285714in}} %
\pgfusepath{clip}%
\pgfsetbuttcap%
\pgfsetroundjoin%
\definecolor{currentfill}{rgb}{1.000000,0.833333,0.833333}%
\pgfsetfillcolor{currentfill}%
\pgfsetlinewidth{1.003750pt}%
\definecolor{currentstroke}{rgb}{1.000000,0.833333,0.833333}%
\pgfsetstrokecolor{currentstroke}%
\pgfsetdash{}{0pt}%
\pgfpathmoveto{\pgfqpoint{0.457963in}{0.823534in}}%
\pgfpathcurveto{\pgfqpoint{0.466200in}{0.823534in}}{\pgfqpoint{0.474100in}{0.826806in}}{\pgfqpoint{0.479924in}{0.832630in}}%
\pgfpathcurveto{\pgfqpoint{0.485748in}{0.838454in}}{\pgfqpoint{0.489020in}{0.846354in}}{\pgfqpoint{0.489020in}{0.854590in}}%
\pgfpathcurveto{\pgfqpoint{0.489020in}{0.862826in}}{\pgfqpoint{0.485748in}{0.870726in}}{\pgfqpoint{0.479924in}{0.876550in}}%
\pgfpathcurveto{\pgfqpoint{0.474100in}{0.882374in}}{\pgfqpoint{0.466200in}{0.885647in}}{\pgfqpoint{0.457963in}{0.885647in}}%
\pgfpathcurveto{\pgfqpoint{0.449727in}{0.885647in}}{\pgfqpoint{0.441827in}{0.882374in}}{\pgfqpoint{0.436003in}{0.876550in}}%
\pgfpathcurveto{\pgfqpoint{0.430179in}{0.870726in}}{\pgfqpoint{0.426907in}{0.862826in}}{\pgfqpoint{0.426907in}{0.854590in}}%
\pgfpathcurveto{\pgfqpoint{0.426907in}{0.846354in}}{\pgfqpoint{0.430179in}{0.838454in}}{\pgfqpoint{0.436003in}{0.832630in}}%
\pgfpathcurveto{\pgfqpoint{0.441827in}{0.826806in}}{\pgfqpoint{0.449727in}{0.823534in}}{\pgfqpoint{0.457963in}{0.823534in}}%
\pgfpathclose%
\pgfusepath{stroke,fill}%
\end{pgfscope}%
\begin{pgfscope}%
\pgfpathrectangle{\pgfqpoint{0.457963in}{0.528059in}}{\pgfqpoint{6.200000in}{2.285714in}} %
\pgfusepath{clip}%
\pgfsetbuttcap%
\pgfsetroundjoin%
\definecolor{currentfill}{rgb}{1.000000,0.833333,0.833333}%
\pgfsetfillcolor{currentfill}%
\pgfsetlinewidth{1.003750pt}%
\definecolor{currentstroke}{rgb}{1.000000,0.833333,0.833333}%
\pgfsetstrokecolor{currentstroke}%
\pgfsetdash{}{0pt}%
\pgfpathmoveto{\pgfqpoint{0.457963in}{0.823534in}}%
\pgfpathcurveto{\pgfqpoint{0.466200in}{0.823534in}}{\pgfqpoint{0.474100in}{0.826806in}}{\pgfqpoint{0.479924in}{0.832630in}}%
\pgfpathcurveto{\pgfqpoint{0.485748in}{0.838454in}}{\pgfqpoint{0.489020in}{0.846354in}}{\pgfqpoint{0.489020in}{0.854590in}}%
\pgfpathcurveto{\pgfqpoint{0.489020in}{0.862826in}}{\pgfqpoint{0.485748in}{0.870726in}}{\pgfqpoint{0.479924in}{0.876550in}}%
\pgfpathcurveto{\pgfqpoint{0.474100in}{0.882374in}}{\pgfqpoint{0.466200in}{0.885647in}}{\pgfqpoint{0.457963in}{0.885647in}}%
\pgfpathcurveto{\pgfqpoint{0.449727in}{0.885647in}}{\pgfqpoint{0.441827in}{0.882374in}}{\pgfqpoint{0.436003in}{0.876550in}}%
\pgfpathcurveto{\pgfqpoint{0.430179in}{0.870726in}}{\pgfqpoint{0.426907in}{0.862826in}}{\pgfqpoint{0.426907in}{0.854590in}}%
\pgfpathcurveto{\pgfqpoint{0.426907in}{0.846354in}}{\pgfqpoint{0.430179in}{0.838454in}}{\pgfqpoint{0.436003in}{0.832630in}}%
\pgfpathcurveto{\pgfqpoint{0.441827in}{0.826806in}}{\pgfqpoint{0.449727in}{0.823534in}}{\pgfqpoint{0.457963in}{0.823534in}}%
\pgfpathclose%
\pgfusepath{stroke,fill}%
\end{pgfscope}%
\begin{pgfscope}%
\pgfpathrectangle{\pgfqpoint{0.457963in}{0.528059in}}{\pgfqpoint{6.200000in}{2.285714in}} %
\pgfusepath{clip}%
\pgfsetbuttcap%
\pgfsetroundjoin%
\definecolor{currentfill}{rgb}{1.000000,0.833333,0.833333}%
\pgfsetfillcolor{currentfill}%
\pgfsetlinewidth{1.003750pt}%
\definecolor{currentstroke}{rgb}{1.000000,0.833333,0.833333}%
\pgfsetstrokecolor{currentstroke}%
\pgfsetdash{}{0pt}%
\pgfpathmoveto{\pgfqpoint{0.457963in}{0.823534in}}%
\pgfpathcurveto{\pgfqpoint{0.466200in}{0.823534in}}{\pgfqpoint{0.474100in}{0.826806in}}{\pgfqpoint{0.479924in}{0.832630in}}%
\pgfpathcurveto{\pgfqpoint{0.485748in}{0.838454in}}{\pgfqpoint{0.489020in}{0.846354in}}{\pgfqpoint{0.489020in}{0.854590in}}%
\pgfpathcurveto{\pgfqpoint{0.489020in}{0.862826in}}{\pgfqpoint{0.485748in}{0.870726in}}{\pgfqpoint{0.479924in}{0.876550in}}%
\pgfpathcurveto{\pgfqpoint{0.474100in}{0.882374in}}{\pgfqpoint{0.466200in}{0.885647in}}{\pgfqpoint{0.457963in}{0.885647in}}%
\pgfpathcurveto{\pgfqpoint{0.449727in}{0.885647in}}{\pgfqpoint{0.441827in}{0.882374in}}{\pgfqpoint{0.436003in}{0.876550in}}%
\pgfpathcurveto{\pgfqpoint{0.430179in}{0.870726in}}{\pgfqpoint{0.426907in}{0.862826in}}{\pgfqpoint{0.426907in}{0.854590in}}%
\pgfpathcurveto{\pgfqpoint{0.426907in}{0.846354in}}{\pgfqpoint{0.430179in}{0.838454in}}{\pgfqpoint{0.436003in}{0.832630in}}%
\pgfpathcurveto{\pgfqpoint{0.441827in}{0.826806in}}{\pgfqpoint{0.449727in}{0.823534in}}{\pgfqpoint{0.457963in}{0.823534in}}%
\pgfpathclose%
\pgfusepath{stroke,fill}%
\end{pgfscope}%
\begin{pgfscope}%
\pgfpathrectangle{\pgfqpoint{0.457963in}{0.528059in}}{\pgfqpoint{6.200000in}{2.285714in}} %
\pgfusepath{clip}%
\pgfsetbuttcap%
\pgfsetroundjoin%
\definecolor{currentfill}{rgb}{1.000000,0.833333,0.833333}%
\pgfsetfillcolor{currentfill}%
\pgfsetlinewidth{1.003750pt}%
\definecolor{currentstroke}{rgb}{1.000000,0.833333,0.833333}%
\pgfsetstrokecolor{currentstroke}%
\pgfsetdash{}{0pt}%
\pgfpathmoveto{\pgfqpoint{0.457963in}{0.823534in}}%
\pgfpathcurveto{\pgfqpoint{0.466200in}{0.823534in}}{\pgfqpoint{0.474100in}{0.826806in}}{\pgfqpoint{0.479924in}{0.832630in}}%
\pgfpathcurveto{\pgfqpoint{0.485748in}{0.838454in}}{\pgfqpoint{0.489020in}{0.846354in}}{\pgfqpoint{0.489020in}{0.854590in}}%
\pgfpathcurveto{\pgfqpoint{0.489020in}{0.862826in}}{\pgfqpoint{0.485748in}{0.870726in}}{\pgfqpoint{0.479924in}{0.876550in}}%
\pgfpathcurveto{\pgfqpoint{0.474100in}{0.882374in}}{\pgfqpoint{0.466200in}{0.885647in}}{\pgfqpoint{0.457963in}{0.885647in}}%
\pgfpathcurveto{\pgfqpoint{0.449727in}{0.885647in}}{\pgfqpoint{0.441827in}{0.882374in}}{\pgfqpoint{0.436003in}{0.876550in}}%
\pgfpathcurveto{\pgfqpoint{0.430179in}{0.870726in}}{\pgfqpoint{0.426907in}{0.862826in}}{\pgfqpoint{0.426907in}{0.854590in}}%
\pgfpathcurveto{\pgfqpoint{0.426907in}{0.846354in}}{\pgfqpoint{0.430179in}{0.838454in}}{\pgfqpoint{0.436003in}{0.832630in}}%
\pgfpathcurveto{\pgfqpoint{0.441827in}{0.826806in}}{\pgfqpoint{0.449727in}{0.823534in}}{\pgfqpoint{0.457963in}{0.823534in}}%
\pgfpathclose%
\pgfusepath{stroke,fill}%
\end{pgfscope}%
\begin{pgfscope}%
\pgfpathrectangle{\pgfqpoint{0.457963in}{0.528059in}}{\pgfqpoint{6.200000in}{2.285714in}} %
\pgfusepath{clip}%
\pgfsetbuttcap%
\pgfsetroundjoin%
\definecolor{currentfill}{rgb}{1.000000,0.833333,0.833333}%
\pgfsetfillcolor{currentfill}%
\pgfsetlinewidth{1.003750pt}%
\definecolor{currentstroke}{rgb}{1.000000,0.833333,0.833333}%
\pgfsetstrokecolor{currentstroke}%
\pgfsetdash{}{0pt}%
\pgfpathmoveto{\pgfqpoint{0.468297in}{0.810472in}}%
\pgfpathcurveto{\pgfqpoint{0.476533in}{0.810472in}}{\pgfqpoint{0.484433in}{0.813745in}}{\pgfqpoint{0.490257in}{0.819569in}}%
\pgfpathcurveto{\pgfqpoint{0.496081in}{0.825393in}}{\pgfqpoint{0.499353in}{0.833293in}}{\pgfqpoint{0.499353in}{0.841529in}}%
\pgfpathcurveto{\pgfqpoint{0.499353in}{0.849765in}}{\pgfqpoint{0.496081in}{0.857665in}}{\pgfqpoint{0.490257in}{0.863489in}}%
\pgfpathcurveto{\pgfqpoint{0.484433in}{0.869313in}}{\pgfqpoint{0.476533in}{0.872585in}}{\pgfqpoint{0.468297in}{0.872585in}}%
\pgfpathcurveto{\pgfqpoint{0.460060in}{0.872585in}}{\pgfqpoint{0.452160in}{0.869313in}}{\pgfqpoint{0.446336in}{0.863489in}}%
\pgfpathcurveto{\pgfqpoint{0.440512in}{0.857665in}}{\pgfqpoint{0.437240in}{0.849765in}}{\pgfqpoint{0.437240in}{0.841529in}}%
\pgfpathcurveto{\pgfqpoint{0.437240in}{0.833293in}}{\pgfqpoint{0.440512in}{0.825393in}}{\pgfqpoint{0.446336in}{0.819569in}}%
\pgfpathcurveto{\pgfqpoint{0.452160in}{0.813745in}}{\pgfqpoint{0.460060in}{0.810472in}}{\pgfqpoint{0.468297in}{0.810472in}}%
\pgfpathclose%
\pgfusepath{stroke,fill}%
\end{pgfscope}%
\begin{pgfscope}%
\pgfpathrectangle{\pgfqpoint{0.457963in}{0.528059in}}{\pgfqpoint{6.200000in}{2.285714in}} %
\pgfusepath{clip}%
\pgfsetbuttcap%
\pgfsetroundjoin%
\definecolor{currentfill}{rgb}{1.000000,0.833333,0.833333}%
\pgfsetfillcolor{currentfill}%
\pgfsetlinewidth{1.003750pt}%
\definecolor{currentstroke}{rgb}{1.000000,0.833333,0.833333}%
\pgfsetstrokecolor{currentstroke}%
\pgfsetdash{}{0pt}%
\pgfpathmoveto{\pgfqpoint{0.468297in}{0.810472in}}%
\pgfpathcurveto{\pgfqpoint{0.476533in}{0.810472in}}{\pgfqpoint{0.484433in}{0.813745in}}{\pgfqpoint{0.490257in}{0.819569in}}%
\pgfpathcurveto{\pgfqpoint{0.496081in}{0.825393in}}{\pgfqpoint{0.499353in}{0.833293in}}{\pgfqpoint{0.499353in}{0.841529in}}%
\pgfpathcurveto{\pgfqpoint{0.499353in}{0.849765in}}{\pgfqpoint{0.496081in}{0.857665in}}{\pgfqpoint{0.490257in}{0.863489in}}%
\pgfpathcurveto{\pgfqpoint{0.484433in}{0.869313in}}{\pgfqpoint{0.476533in}{0.872585in}}{\pgfqpoint{0.468297in}{0.872585in}}%
\pgfpathcurveto{\pgfqpoint{0.460060in}{0.872585in}}{\pgfqpoint{0.452160in}{0.869313in}}{\pgfqpoint{0.446336in}{0.863489in}}%
\pgfpathcurveto{\pgfqpoint{0.440512in}{0.857665in}}{\pgfqpoint{0.437240in}{0.849765in}}{\pgfqpoint{0.437240in}{0.841529in}}%
\pgfpathcurveto{\pgfqpoint{0.437240in}{0.833293in}}{\pgfqpoint{0.440512in}{0.825393in}}{\pgfqpoint{0.446336in}{0.819569in}}%
\pgfpathcurveto{\pgfqpoint{0.452160in}{0.813745in}}{\pgfqpoint{0.460060in}{0.810472in}}{\pgfqpoint{0.468297in}{0.810472in}}%
\pgfpathclose%
\pgfusepath{stroke,fill}%
\end{pgfscope}%
\begin{pgfscope}%
\pgfpathrectangle{\pgfqpoint{0.457963in}{0.528059in}}{\pgfqpoint{6.200000in}{2.285714in}} %
\pgfusepath{clip}%
\pgfsetbuttcap%
\pgfsetroundjoin%
\definecolor{currentfill}{rgb}{1.000000,0.833333,0.833333}%
\pgfsetfillcolor{currentfill}%
\pgfsetlinewidth{1.003750pt}%
\definecolor{currentstroke}{rgb}{1.000000,0.833333,0.833333}%
\pgfsetstrokecolor{currentstroke}%
\pgfsetdash{}{0pt}%
\pgfpathmoveto{\pgfqpoint{0.468297in}{0.823534in}}%
\pgfpathcurveto{\pgfqpoint{0.476533in}{0.823534in}}{\pgfqpoint{0.484433in}{0.826806in}}{\pgfqpoint{0.490257in}{0.832630in}}%
\pgfpathcurveto{\pgfqpoint{0.496081in}{0.838454in}}{\pgfqpoint{0.499353in}{0.846354in}}{\pgfqpoint{0.499353in}{0.854590in}}%
\pgfpathcurveto{\pgfqpoint{0.499353in}{0.862826in}}{\pgfqpoint{0.496081in}{0.870726in}}{\pgfqpoint{0.490257in}{0.876550in}}%
\pgfpathcurveto{\pgfqpoint{0.484433in}{0.882374in}}{\pgfqpoint{0.476533in}{0.885647in}}{\pgfqpoint{0.468297in}{0.885647in}}%
\pgfpathcurveto{\pgfqpoint{0.460060in}{0.885647in}}{\pgfqpoint{0.452160in}{0.882374in}}{\pgfqpoint{0.446336in}{0.876550in}}%
\pgfpathcurveto{\pgfqpoint{0.440512in}{0.870726in}}{\pgfqpoint{0.437240in}{0.862826in}}{\pgfqpoint{0.437240in}{0.854590in}}%
\pgfpathcurveto{\pgfqpoint{0.437240in}{0.846354in}}{\pgfqpoint{0.440512in}{0.838454in}}{\pgfqpoint{0.446336in}{0.832630in}}%
\pgfpathcurveto{\pgfqpoint{0.452160in}{0.826806in}}{\pgfqpoint{0.460060in}{0.823534in}}{\pgfqpoint{0.468297in}{0.823534in}}%
\pgfpathclose%
\pgfusepath{stroke,fill}%
\end{pgfscope}%
\begin{pgfscope}%
\pgfpathrectangle{\pgfqpoint{0.457963in}{0.528059in}}{\pgfqpoint{6.200000in}{2.285714in}} %
\pgfusepath{clip}%
\pgfsetbuttcap%
\pgfsetroundjoin%
\definecolor{currentfill}{rgb}{1.000000,0.833333,0.833333}%
\pgfsetfillcolor{currentfill}%
\pgfsetlinewidth{1.003750pt}%
\definecolor{currentstroke}{rgb}{1.000000,0.833333,0.833333}%
\pgfsetstrokecolor{currentstroke}%
\pgfsetdash{}{0pt}%
\pgfpathmoveto{\pgfqpoint{0.540630in}{0.823534in}}%
\pgfpathcurveto{\pgfqpoint{0.548866in}{0.823534in}}{\pgfqpoint{0.556766in}{0.826806in}}{\pgfqpoint{0.562590in}{0.832630in}}%
\pgfpathcurveto{\pgfqpoint{0.568414in}{0.838454in}}{\pgfqpoint{0.571686in}{0.846354in}}{\pgfqpoint{0.571686in}{0.854590in}}%
\pgfpathcurveto{\pgfqpoint{0.571686in}{0.862826in}}{\pgfqpoint{0.568414in}{0.870726in}}{\pgfqpoint{0.562590in}{0.876550in}}%
\pgfpathcurveto{\pgfqpoint{0.556766in}{0.882374in}}{\pgfqpoint{0.548866in}{0.885647in}}{\pgfqpoint{0.540630in}{0.885647in}}%
\pgfpathcurveto{\pgfqpoint{0.532394in}{0.885647in}}{\pgfqpoint{0.524494in}{0.882374in}}{\pgfqpoint{0.518670in}{0.876550in}}%
\pgfpathcurveto{\pgfqpoint{0.512846in}{0.870726in}}{\pgfqpoint{0.509574in}{0.862826in}}{\pgfqpoint{0.509574in}{0.854590in}}%
\pgfpathcurveto{\pgfqpoint{0.509574in}{0.846354in}}{\pgfqpoint{0.512846in}{0.838454in}}{\pgfqpoint{0.518670in}{0.832630in}}%
\pgfpathcurveto{\pgfqpoint{0.524494in}{0.826806in}}{\pgfqpoint{0.532394in}{0.823534in}}{\pgfqpoint{0.540630in}{0.823534in}}%
\pgfpathclose%
\pgfusepath{stroke,fill}%
\end{pgfscope}%
\begin{pgfscope}%
\pgfpathrectangle{\pgfqpoint{0.457963in}{0.528059in}}{\pgfqpoint{6.200000in}{2.285714in}} %
\pgfusepath{clip}%
\pgfsetbuttcap%
\pgfsetroundjoin%
\definecolor{currentfill}{rgb}{1.000000,0.833333,0.833333}%
\pgfsetfillcolor{currentfill}%
\pgfsetlinewidth{1.003750pt}%
\definecolor{currentstroke}{rgb}{1.000000,0.833333,0.833333}%
\pgfsetstrokecolor{currentstroke}%
\pgfsetdash{}{0pt}%
\pgfpathmoveto{\pgfqpoint{0.550963in}{0.810472in}}%
\pgfpathcurveto{\pgfqpoint{0.559200in}{0.810472in}}{\pgfqpoint{0.567100in}{0.813745in}}{\pgfqpoint{0.572924in}{0.819569in}}%
\pgfpathcurveto{\pgfqpoint{0.578748in}{0.825393in}}{\pgfqpoint{0.582020in}{0.833293in}}{\pgfqpoint{0.582020in}{0.841529in}}%
\pgfpathcurveto{\pgfqpoint{0.582020in}{0.849765in}}{\pgfqpoint{0.578748in}{0.857665in}}{\pgfqpoint{0.572924in}{0.863489in}}%
\pgfpathcurveto{\pgfqpoint{0.567100in}{0.869313in}}{\pgfqpoint{0.559200in}{0.872585in}}{\pgfqpoint{0.550963in}{0.872585in}}%
\pgfpathcurveto{\pgfqpoint{0.542727in}{0.872585in}}{\pgfqpoint{0.534827in}{0.869313in}}{\pgfqpoint{0.529003in}{0.863489in}}%
\pgfpathcurveto{\pgfqpoint{0.523179in}{0.857665in}}{\pgfqpoint{0.519907in}{0.849765in}}{\pgfqpoint{0.519907in}{0.841529in}}%
\pgfpathcurveto{\pgfqpoint{0.519907in}{0.833293in}}{\pgfqpoint{0.523179in}{0.825393in}}{\pgfqpoint{0.529003in}{0.819569in}}%
\pgfpathcurveto{\pgfqpoint{0.534827in}{0.813745in}}{\pgfqpoint{0.542727in}{0.810472in}}{\pgfqpoint{0.550963in}{0.810472in}}%
\pgfpathclose%
\pgfusepath{stroke,fill}%
\end{pgfscope}%
\begin{pgfscope}%
\pgfpathrectangle{\pgfqpoint{0.457963in}{0.528059in}}{\pgfqpoint{6.200000in}{2.285714in}} %
\pgfusepath{clip}%
\pgfsetbuttcap%
\pgfsetroundjoin%
\definecolor{currentfill}{rgb}{1.000000,0.833333,0.833333}%
\pgfsetfillcolor{currentfill}%
\pgfsetlinewidth{1.003750pt}%
\definecolor{currentstroke}{rgb}{1.000000,0.833333,0.833333}%
\pgfsetstrokecolor{currentstroke}%
\pgfsetdash{}{0pt}%
\pgfpathmoveto{\pgfqpoint{0.674963in}{0.810472in}}%
\pgfpathcurveto{\pgfqpoint{0.683200in}{0.810472in}}{\pgfqpoint{0.691100in}{0.813745in}}{\pgfqpoint{0.696924in}{0.819569in}}%
\pgfpathcurveto{\pgfqpoint{0.702748in}{0.825393in}}{\pgfqpoint{0.706020in}{0.833293in}}{\pgfqpoint{0.706020in}{0.841529in}}%
\pgfpathcurveto{\pgfqpoint{0.706020in}{0.849765in}}{\pgfqpoint{0.702748in}{0.857665in}}{\pgfqpoint{0.696924in}{0.863489in}}%
\pgfpathcurveto{\pgfqpoint{0.691100in}{0.869313in}}{\pgfqpoint{0.683200in}{0.872585in}}{\pgfqpoint{0.674963in}{0.872585in}}%
\pgfpathcurveto{\pgfqpoint{0.666727in}{0.872585in}}{\pgfqpoint{0.658827in}{0.869313in}}{\pgfqpoint{0.653003in}{0.863489in}}%
\pgfpathcurveto{\pgfqpoint{0.647179in}{0.857665in}}{\pgfqpoint{0.643907in}{0.849765in}}{\pgfqpoint{0.643907in}{0.841529in}}%
\pgfpathcurveto{\pgfqpoint{0.643907in}{0.833293in}}{\pgfqpoint{0.647179in}{0.825393in}}{\pgfqpoint{0.653003in}{0.819569in}}%
\pgfpathcurveto{\pgfqpoint{0.658827in}{0.813745in}}{\pgfqpoint{0.666727in}{0.810472in}}{\pgfqpoint{0.674963in}{0.810472in}}%
\pgfpathclose%
\pgfusepath{stroke,fill}%
\end{pgfscope}%
\begin{pgfscope}%
\pgfpathrectangle{\pgfqpoint{0.457963in}{0.528059in}}{\pgfqpoint{6.200000in}{2.285714in}} %
\pgfusepath{clip}%
\pgfsetbuttcap%
\pgfsetroundjoin%
\definecolor{currentfill}{rgb}{1.000000,0.833333,0.833333}%
\pgfsetfillcolor{currentfill}%
\pgfsetlinewidth{1.003750pt}%
\definecolor{currentstroke}{rgb}{1.000000,0.833333,0.833333}%
\pgfsetstrokecolor{currentstroke}%
\pgfsetdash{}{0pt}%
\pgfpathmoveto{\pgfqpoint{0.685297in}{0.784350in}}%
\pgfpathcurveto{\pgfqpoint{0.693533in}{0.784350in}}{\pgfqpoint{0.701433in}{0.787622in}}{\pgfqpoint{0.707257in}{0.793446in}}%
\pgfpathcurveto{\pgfqpoint{0.713081in}{0.799270in}}{\pgfqpoint{0.716353in}{0.807170in}}{\pgfqpoint{0.716353in}{0.815406in}}%
\pgfpathcurveto{\pgfqpoint{0.716353in}{0.823643in}}{\pgfqpoint{0.713081in}{0.831543in}}{\pgfqpoint{0.707257in}{0.837367in}}%
\pgfpathcurveto{\pgfqpoint{0.701433in}{0.843191in}}{\pgfqpoint{0.693533in}{0.846463in}}{\pgfqpoint{0.685297in}{0.846463in}}%
\pgfpathcurveto{\pgfqpoint{0.677060in}{0.846463in}}{\pgfqpoint{0.669160in}{0.843191in}}{\pgfqpoint{0.663336in}{0.837367in}}%
\pgfpathcurveto{\pgfqpoint{0.657512in}{0.831543in}}{\pgfqpoint{0.654240in}{0.823643in}}{\pgfqpoint{0.654240in}{0.815406in}}%
\pgfpathcurveto{\pgfqpoint{0.654240in}{0.807170in}}{\pgfqpoint{0.657512in}{0.799270in}}{\pgfqpoint{0.663336in}{0.793446in}}%
\pgfpathcurveto{\pgfqpoint{0.669160in}{0.787622in}}{\pgfqpoint{0.677060in}{0.784350in}}{\pgfqpoint{0.685297in}{0.784350in}}%
\pgfpathclose%
\pgfusepath{stroke,fill}%
\end{pgfscope}%
\begin{pgfscope}%
\pgfpathrectangle{\pgfqpoint{0.457963in}{0.528059in}}{\pgfqpoint{6.200000in}{2.285714in}} %
\pgfusepath{clip}%
\pgfsetbuttcap%
\pgfsetroundjoin%
\definecolor{currentfill}{rgb}{1.000000,0.833333,0.833333}%
\pgfsetfillcolor{currentfill}%
\pgfsetlinewidth{1.003750pt}%
\definecolor{currentstroke}{rgb}{1.000000,0.833333,0.833333}%
\pgfsetstrokecolor{currentstroke}%
\pgfsetdash{}{0pt}%
\pgfpathmoveto{\pgfqpoint{0.788630in}{0.745166in}}%
\pgfpathcurveto{\pgfqpoint{0.796866in}{0.745166in}}{\pgfqpoint{0.804766in}{0.748439in}}{\pgfqpoint{0.810590in}{0.754262in}}%
\pgfpathcurveto{\pgfqpoint{0.816414in}{0.760086in}}{\pgfqpoint{0.819686in}{0.767986in}}{\pgfqpoint{0.819686in}{0.776223in}}%
\pgfpathcurveto{\pgfqpoint{0.819686in}{0.784459in}}{\pgfqpoint{0.816414in}{0.792359in}}{\pgfqpoint{0.810590in}{0.798183in}}%
\pgfpathcurveto{\pgfqpoint{0.804766in}{0.804007in}}{\pgfqpoint{0.796866in}{0.807279in}}{\pgfqpoint{0.788630in}{0.807279in}}%
\pgfpathcurveto{\pgfqpoint{0.780394in}{0.807279in}}{\pgfqpoint{0.772494in}{0.804007in}}{\pgfqpoint{0.766670in}{0.798183in}}%
\pgfpathcurveto{\pgfqpoint{0.760846in}{0.792359in}}{\pgfqpoint{0.757574in}{0.784459in}}{\pgfqpoint{0.757574in}{0.776223in}}%
\pgfpathcurveto{\pgfqpoint{0.757574in}{0.767986in}}{\pgfqpoint{0.760846in}{0.760086in}}{\pgfqpoint{0.766670in}{0.754262in}}%
\pgfpathcurveto{\pgfqpoint{0.772494in}{0.748439in}}{\pgfqpoint{0.780394in}{0.745166in}}{\pgfqpoint{0.788630in}{0.745166in}}%
\pgfpathclose%
\pgfusepath{stroke,fill}%
\end{pgfscope}%
\begin{pgfscope}%
\pgfpathrectangle{\pgfqpoint{0.457963in}{0.528059in}}{\pgfqpoint{6.200000in}{2.285714in}} %
\pgfusepath{clip}%
\pgfsetbuttcap%
\pgfsetroundjoin%
\definecolor{currentfill}{rgb}{1.000000,0.833333,0.833333}%
\pgfsetfillcolor{currentfill}%
\pgfsetlinewidth{1.003750pt}%
\definecolor{currentstroke}{rgb}{1.000000,0.833333,0.833333}%
\pgfsetstrokecolor{currentstroke}%
\pgfsetdash{}{0pt}%
\pgfpathmoveto{\pgfqpoint{0.943630in}{0.732105in}}%
\pgfpathcurveto{\pgfqpoint{0.951866in}{0.732105in}}{\pgfqpoint{0.959766in}{0.735377in}}{\pgfqpoint{0.965590in}{0.741201in}}%
\pgfpathcurveto{\pgfqpoint{0.971414in}{0.747025in}}{\pgfqpoint{0.974686in}{0.754925in}}{\pgfqpoint{0.974686in}{0.763161in}}%
\pgfpathcurveto{\pgfqpoint{0.974686in}{0.771398in}}{\pgfqpoint{0.971414in}{0.779298in}}{\pgfqpoint{0.965590in}{0.785122in}}%
\pgfpathcurveto{\pgfqpoint{0.959766in}{0.790946in}}{\pgfqpoint{0.951866in}{0.794218in}}{\pgfqpoint{0.943630in}{0.794218in}}%
\pgfpathcurveto{\pgfqpoint{0.935394in}{0.794218in}}{\pgfqpoint{0.927494in}{0.790946in}}{\pgfqpoint{0.921670in}{0.785122in}}%
\pgfpathcurveto{\pgfqpoint{0.915846in}{0.779298in}}{\pgfqpoint{0.912574in}{0.771398in}}{\pgfqpoint{0.912574in}{0.763161in}}%
\pgfpathcurveto{\pgfqpoint{0.912574in}{0.754925in}}{\pgfqpoint{0.915846in}{0.747025in}}{\pgfqpoint{0.921670in}{0.741201in}}%
\pgfpathcurveto{\pgfqpoint{0.927494in}{0.735377in}}{\pgfqpoint{0.935394in}{0.732105in}}{\pgfqpoint{0.943630in}{0.732105in}}%
\pgfpathclose%
\pgfusepath{stroke,fill}%
\end{pgfscope}%
\begin{pgfscope}%
\pgfpathrectangle{\pgfqpoint{0.457963in}{0.528059in}}{\pgfqpoint{6.200000in}{2.285714in}} %
\pgfusepath{clip}%
\pgfsetbuttcap%
\pgfsetroundjoin%
\definecolor{currentfill}{rgb}{1.000000,0.833333,0.833333}%
\pgfsetfillcolor{currentfill}%
\pgfsetlinewidth{1.003750pt}%
\definecolor{currentstroke}{rgb}{1.000000,0.833333,0.833333}%
\pgfsetstrokecolor{currentstroke}%
\pgfsetdash{}{0pt}%
\pgfpathmoveto{\pgfqpoint{1.026297in}{0.679860in}}%
\pgfpathcurveto{\pgfqpoint{1.034533in}{0.679860in}}{\pgfqpoint{1.042433in}{0.683132in}}{\pgfqpoint{1.048257in}{0.688956in}}%
\pgfpathcurveto{\pgfqpoint{1.054081in}{0.694780in}}{\pgfqpoint{1.057353in}{0.702680in}}{\pgfqpoint{1.057353in}{0.710917in}}%
\pgfpathcurveto{\pgfqpoint{1.057353in}{0.719153in}}{\pgfqpoint{1.054081in}{0.727053in}}{\pgfqpoint{1.048257in}{0.732877in}}%
\pgfpathcurveto{\pgfqpoint{1.042433in}{0.738701in}}{\pgfqpoint{1.034533in}{0.741973in}}{\pgfqpoint{1.026297in}{0.741973in}}%
\pgfpathcurveto{\pgfqpoint{1.018060in}{0.741973in}}{\pgfqpoint{1.010160in}{0.738701in}}{\pgfqpoint{1.004336in}{0.732877in}}%
\pgfpathcurveto{\pgfqpoint{0.998512in}{0.727053in}}{\pgfqpoint{0.995240in}{0.719153in}}{\pgfqpoint{0.995240in}{0.710917in}}%
\pgfpathcurveto{\pgfqpoint{0.995240in}{0.702680in}}{\pgfqpoint{0.998512in}{0.694780in}}{\pgfqpoint{1.004336in}{0.688956in}}%
\pgfpathcurveto{\pgfqpoint{1.010160in}{0.683132in}}{\pgfqpoint{1.018060in}{0.679860in}}{\pgfqpoint{1.026297in}{0.679860in}}%
\pgfpathclose%
\pgfusepath{stroke,fill}%
\end{pgfscope}%
\begin{pgfscope}%
\pgfpathrectangle{\pgfqpoint{0.457963in}{0.528059in}}{\pgfqpoint{6.200000in}{2.285714in}} %
\pgfusepath{clip}%
\pgfsetbuttcap%
\pgfsetroundjoin%
\definecolor{currentfill}{rgb}{1.000000,0.833333,0.833333}%
\pgfsetfillcolor{currentfill}%
\pgfsetlinewidth{1.003750pt}%
\definecolor{currentstroke}{rgb}{1.000000,0.833333,0.833333}%
\pgfsetstrokecolor{currentstroke}%
\pgfsetdash{}{0pt}%
\pgfpathmoveto{\pgfqpoint{1.222630in}{0.810472in}}%
\pgfpathcurveto{\pgfqpoint{1.230866in}{0.810472in}}{\pgfqpoint{1.238766in}{0.813745in}}{\pgfqpoint{1.244590in}{0.819569in}}%
\pgfpathcurveto{\pgfqpoint{1.250414in}{0.825393in}}{\pgfqpoint{1.253686in}{0.833293in}}{\pgfqpoint{1.253686in}{0.841529in}}%
\pgfpathcurveto{\pgfqpoint{1.253686in}{0.849765in}}{\pgfqpoint{1.250414in}{0.857665in}}{\pgfqpoint{1.244590in}{0.863489in}}%
\pgfpathcurveto{\pgfqpoint{1.238766in}{0.869313in}}{\pgfqpoint{1.230866in}{0.872585in}}{\pgfqpoint{1.222630in}{0.872585in}}%
\pgfpathcurveto{\pgfqpoint{1.214394in}{0.872585in}}{\pgfqpoint{1.206494in}{0.869313in}}{\pgfqpoint{1.200670in}{0.863489in}}%
\pgfpathcurveto{\pgfqpoint{1.194846in}{0.857665in}}{\pgfqpoint{1.191574in}{0.849765in}}{\pgfqpoint{1.191574in}{0.841529in}}%
\pgfpathcurveto{\pgfqpoint{1.191574in}{0.833293in}}{\pgfqpoint{1.194846in}{0.825393in}}{\pgfqpoint{1.200670in}{0.819569in}}%
\pgfpathcurveto{\pgfqpoint{1.206494in}{0.813745in}}{\pgfqpoint{1.214394in}{0.810472in}}{\pgfqpoint{1.222630in}{0.810472in}}%
\pgfpathclose%
\pgfusepath{stroke,fill}%
\end{pgfscope}%
\begin{pgfscope}%
\pgfpathrectangle{\pgfqpoint{0.457963in}{0.528059in}}{\pgfqpoint{6.200000in}{2.285714in}} %
\pgfusepath{clip}%
\pgfsetbuttcap%
\pgfsetroundjoin%
\definecolor{currentfill}{rgb}{1.000000,0.833333,0.833333}%
\pgfsetfillcolor{currentfill}%
\pgfsetlinewidth{1.003750pt}%
\definecolor{currentstroke}{rgb}{1.000000,0.833333,0.833333}%
\pgfsetstrokecolor{currentstroke}%
\pgfsetdash{}{0pt}%
\pgfpathmoveto{\pgfqpoint{1.232963in}{0.823534in}}%
\pgfpathcurveto{\pgfqpoint{1.241200in}{0.823534in}}{\pgfqpoint{1.249100in}{0.826806in}}{\pgfqpoint{1.254924in}{0.832630in}}%
\pgfpathcurveto{\pgfqpoint{1.260748in}{0.838454in}}{\pgfqpoint{1.264020in}{0.846354in}}{\pgfqpoint{1.264020in}{0.854590in}}%
\pgfpathcurveto{\pgfqpoint{1.264020in}{0.862826in}}{\pgfqpoint{1.260748in}{0.870726in}}{\pgfqpoint{1.254924in}{0.876550in}}%
\pgfpathcurveto{\pgfqpoint{1.249100in}{0.882374in}}{\pgfqpoint{1.241200in}{0.885647in}}{\pgfqpoint{1.232963in}{0.885647in}}%
\pgfpathcurveto{\pgfqpoint{1.224727in}{0.885647in}}{\pgfqpoint{1.216827in}{0.882374in}}{\pgfqpoint{1.211003in}{0.876550in}}%
\pgfpathcurveto{\pgfqpoint{1.205179in}{0.870726in}}{\pgfqpoint{1.201907in}{0.862826in}}{\pgfqpoint{1.201907in}{0.854590in}}%
\pgfpathcurveto{\pgfqpoint{1.201907in}{0.846354in}}{\pgfqpoint{1.205179in}{0.838454in}}{\pgfqpoint{1.211003in}{0.832630in}}%
\pgfpathcurveto{\pgfqpoint{1.216827in}{0.826806in}}{\pgfqpoint{1.224727in}{0.823534in}}{\pgfqpoint{1.232963in}{0.823534in}}%
\pgfpathclose%
\pgfusepath{stroke,fill}%
\end{pgfscope}%
\begin{pgfscope}%
\pgfpathrectangle{\pgfqpoint{0.457963in}{0.528059in}}{\pgfqpoint{6.200000in}{2.285714in}} %
\pgfusepath{clip}%
\pgfsetbuttcap%
\pgfsetroundjoin%
\definecolor{currentfill}{rgb}{1.000000,0.833333,0.833333}%
\pgfsetfillcolor{currentfill}%
\pgfsetlinewidth{1.003750pt}%
\definecolor{currentstroke}{rgb}{1.000000,0.833333,0.833333}%
\pgfsetstrokecolor{currentstroke}%
\pgfsetdash{}{0pt}%
\pgfpathmoveto{\pgfqpoint{1.460297in}{0.797411in}}%
\pgfpathcurveto{\pgfqpoint{1.468533in}{0.797411in}}{\pgfqpoint{1.476433in}{0.800683in}}{\pgfqpoint{1.482257in}{0.806507in}}%
\pgfpathcurveto{\pgfqpoint{1.488081in}{0.812331in}}{\pgfqpoint{1.491353in}{0.820231in}}{\pgfqpoint{1.491353in}{0.828468in}}%
\pgfpathcurveto{\pgfqpoint{1.491353in}{0.836704in}}{\pgfqpoint{1.488081in}{0.844604in}}{\pgfqpoint{1.482257in}{0.850428in}}%
\pgfpathcurveto{\pgfqpoint{1.476433in}{0.856252in}}{\pgfqpoint{1.468533in}{0.859524in}}{\pgfqpoint{1.460297in}{0.859524in}}%
\pgfpathcurveto{\pgfqpoint{1.452060in}{0.859524in}}{\pgfqpoint{1.444160in}{0.856252in}}{\pgfqpoint{1.438336in}{0.850428in}}%
\pgfpathcurveto{\pgfqpoint{1.432512in}{0.844604in}}{\pgfqpoint{1.429240in}{0.836704in}}{\pgfqpoint{1.429240in}{0.828468in}}%
\pgfpathcurveto{\pgfqpoint{1.429240in}{0.820231in}}{\pgfqpoint{1.432512in}{0.812331in}}{\pgfqpoint{1.438336in}{0.806507in}}%
\pgfpathcurveto{\pgfqpoint{1.444160in}{0.800683in}}{\pgfqpoint{1.452060in}{0.797411in}}{\pgfqpoint{1.460297in}{0.797411in}}%
\pgfpathclose%
\pgfusepath{stroke,fill}%
\end{pgfscope}%
\begin{pgfscope}%
\pgfpathrectangle{\pgfqpoint{0.457963in}{0.528059in}}{\pgfqpoint{6.200000in}{2.285714in}} %
\pgfusepath{clip}%
\pgfsetbuttcap%
\pgfsetroundjoin%
\definecolor{currentfill}{rgb}{1.000000,0.833333,0.833333}%
\pgfsetfillcolor{currentfill}%
\pgfsetlinewidth{1.003750pt}%
\definecolor{currentstroke}{rgb}{1.000000,0.833333,0.833333}%
\pgfsetstrokecolor{currentstroke}%
\pgfsetdash{}{0pt}%
\pgfpathmoveto{\pgfqpoint{1.677297in}{0.705983in}}%
\pgfpathcurveto{\pgfqpoint{1.685533in}{0.705983in}}{\pgfqpoint{1.693433in}{0.709255in}}{\pgfqpoint{1.699257in}{0.715079in}}%
\pgfpathcurveto{\pgfqpoint{1.705081in}{0.720903in}}{\pgfqpoint{1.708353in}{0.728803in}}{\pgfqpoint{1.708353in}{0.737039in}}%
\pgfpathcurveto{\pgfqpoint{1.708353in}{0.745275in}}{\pgfqpoint{1.705081in}{0.753175in}}{\pgfqpoint{1.699257in}{0.758999in}}%
\pgfpathcurveto{\pgfqpoint{1.693433in}{0.764823in}}{\pgfqpoint{1.685533in}{0.768096in}}{\pgfqpoint{1.677297in}{0.768096in}}%
\pgfpathcurveto{\pgfqpoint{1.669060in}{0.768096in}}{\pgfqpoint{1.661160in}{0.764823in}}{\pgfqpoint{1.655336in}{0.758999in}}%
\pgfpathcurveto{\pgfqpoint{1.649512in}{0.753175in}}{\pgfqpoint{1.646240in}{0.745275in}}{\pgfqpoint{1.646240in}{0.737039in}}%
\pgfpathcurveto{\pgfqpoint{1.646240in}{0.728803in}}{\pgfqpoint{1.649512in}{0.720903in}}{\pgfqpoint{1.655336in}{0.715079in}}%
\pgfpathcurveto{\pgfqpoint{1.661160in}{0.709255in}}{\pgfqpoint{1.669060in}{0.705983in}}{\pgfqpoint{1.677297in}{0.705983in}}%
\pgfpathclose%
\pgfusepath{stroke,fill}%
\end{pgfscope}%
\begin{pgfscope}%
\pgfpathrectangle{\pgfqpoint{0.457963in}{0.528059in}}{\pgfqpoint{6.200000in}{2.285714in}} %
\pgfusepath{clip}%
\pgfsetbuttcap%
\pgfsetroundjoin%
\definecolor{currentfill}{rgb}{1.000000,0.833333,0.833333}%
\pgfsetfillcolor{currentfill}%
\pgfsetlinewidth{1.003750pt}%
\definecolor{currentstroke}{rgb}{1.000000,0.833333,0.833333}%
\pgfsetstrokecolor{currentstroke}%
\pgfsetdash{}{0pt}%
\pgfpathmoveto{\pgfqpoint{1.739297in}{0.745166in}}%
\pgfpathcurveto{\pgfqpoint{1.747533in}{0.745166in}}{\pgfqpoint{1.755433in}{0.748439in}}{\pgfqpoint{1.761257in}{0.754262in}}%
\pgfpathcurveto{\pgfqpoint{1.767081in}{0.760086in}}{\pgfqpoint{1.770353in}{0.767986in}}{\pgfqpoint{1.770353in}{0.776223in}}%
\pgfpathcurveto{\pgfqpoint{1.770353in}{0.784459in}}{\pgfqpoint{1.767081in}{0.792359in}}{\pgfqpoint{1.761257in}{0.798183in}}%
\pgfpathcurveto{\pgfqpoint{1.755433in}{0.804007in}}{\pgfqpoint{1.747533in}{0.807279in}}{\pgfqpoint{1.739297in}{0.807279in}}%
\pgfpathcurveto{\pgfqpoint{1.731060in}{0.807279in}}{\pgfqpoint{1.723160in}{0.804007in}}{\pgfqpoint{1.717336in}{0.798183in}}%
\pgfpathcurveto{\pgfqpoint{1.711512in}{0.792359in}}{\pgfqpoint{1.708240in}{0.784459in}}{\pgfqpoint{1.708240in}{0.776223in}}%
\pgfpathcurveto{\pgfqpoint{1.708240in}{0.767986in}}{\pgfqpoint{1.711512in}{0.760086in}}{\pgfqpoint{1.717336in}{0.754262in}}%
\pgfpathcurveto{\pgfqpoint{1.723160in}{0.748439in}}{\pgfqpoint{1.731060in}{0.745166in}}{\pgfqpoint{1.739297in}{0.745166in}}%
\pgfpathclose%
\pgfusepath{stroke,fill}%
\end{pgfscope}%
\begin{pgfscope}%
\pgfpathrectangle{\pgfqpoint{0.457963in}{0.528059in}}{\pgfqpoint{6.200000in}{2.285714in}} %
\pgfusepath{clip}%
\pgfsetbuttcap%
\pgfsetroundjoin%
\definecolor{currentfill}{rgb}{1.000000,0.833333,0.833333}%
\pgfsetfillcolor{currentfill}%
\pgfsetlinewidth{1.003750pt}%
\definecolor{currentstroke}{rgb}{1.000000,0.833333,0.833333}%
\pgfsetstrokecolor{currentstroke}%
\pgfsetdash{}{0pt}%
\pgfpathmoveto{\pgfqpoint{1.997630in}{0.549248in}}%
\pgfpathcurveto{\pgfqpoint{2.005866in}{0.549248in}}{\pgfqpoint{2.013766in}{0.552520in}}{\pgfqpoint{2.019590in}{0.558344in}}%
\pgfpathcurveto{\pgfqpoint{2.025414in}{0.564168in}}{\pgfqpoint{2.028686in}{0.572068in}}{\pgfqpoint{2.028686in}{0.580304in}}%
\pgfpathcurveto{\pgfqpoint{2.028686in}{0.588541in}}{\pgfqpoint{2.025414in}{0.596441in}}{\pgfqpoint{2.019590in}{0.602265in}}%
\pgfpathcurveto{\pgfqpoint{2.013766in}{0.608089in}}{\pgfqpoint{2.005866in}{0.611361in}}{\pgfqpoint{1.997630in}{0.611361in}}%
\pgfpathcurveto{\pgfqpoint{1.989394in}{0.611361in}}{\pgfqpoint{1.981494in}{0.608089in}}{\pgfqpoint{1.975670in}{0.602265in}}%
\pgfpathcurveto{\pgfqpoint{1.969846in}{0.596441in}}{\pgfqpoint{1.966574in}{0.588541in}}{\pgfqpoint{1.966574in}{0.580304in}}%
\pgfpathcurveto{\pgfqpoint{1.966574in}{0.572068in}}{\pgfqpoint{1.969846in}{0.564168in}}{\pgfqpoint{1.975670in}{0.558344in}}%
\pgfpathcurveto{\pgfqpoint{1.981494in}{0.552520in}}{\pgfqpoint{1.989394in}{0.549248in}}{\pgfqpoint{1.997630in}{0.549248in}}%
\pgfpathclose%
\pgfusepath{stroke,fill}%
\end{pgfscope}%
\begin{pgfscope}%
\pgfpathrectangle{\pgfqpoint{0.457963in}{0.528059in}}{\pgfqpoint{6.200000in}{2.285714in}} %
\pgfusepath{clip}%
\pgfsetbuttcap%
\pgfsetroundjoin%
\definecolor{currentfill}{rgb}{1.000000,0.666667,0.666667}%
\pgfsetfillcolor{currentfill}%
\pgfsetlinewidth{1.003750pt}%
\definecolor{currentstroke}{rgb}{1.000000,0.666667,0.666667}%
\pgfsetstrokecolor{currentstroke}%
\pgfsetdash{}{0pt}%
\pgfpathmoveto{\pgfqpoint{0.457963in}{1.150064in}}%
\pgfpathcurveto{\pgfqpoint{0.466200in}{1.150064in}}{\pgfqpoint{0.474100in}{1.153336in}}{\pgfqpoint{0.479924in}{1.159160in}}%
\pgfpathcurveto{\pgfqpoint{0.485748in}{1.164984in}}{\pgfqpoint{0.489020in}{1.172884in}}{\pgfqpoint{0.489020in}{1.181121in}}%
\pgfpathcurveto{\pgfqpoint{0.489020in}{1.189357in}}{\pgfqpoint{0.485748in}{1.197257in}}{\pgfqpoint{0.479924in}{1.203081in}}%
\pgfpathcurveto{\pgfqpoint{0.474100in}{1.208905in}}{\pgfqpoint{0.466200in}{1.212177in}}{\pgfqpoint{0.457963in}{1.212177in}}%
\pgfpathcurveto{\pgfqpoint{0.449727in}{1.212177in}}{\pgfqpoint{0.441827in}{1.208905in}}{\pgfqpoint{0.436003in}{1.203081in}}%
\pgfpathcurveto{\pgfqpoint{0.430179in}{1.197257in}}{\pgfqpoint{0.426907in}{1.189357in}}{\pgfqpoint{0.426907in}{1.181121in}}%
\pgfpathcurveto{\pgfqpoint{0.426907in}{1.172884in}}{\pgfqpoint{0.430179in}{1.164984in}}{\pgfqpoint{0.436003in}{1.159160in}}%
\pgfpathcurveto{\pgfqpoint{0.441827in}{1.153336in}}{\pgfqpoint{0.449727in}{1.150064in}}{\pgfqpoint{0.457963in}{1.150064in}}%
\pgfpathclose%
\pgfusepath{stroke,fill}%
\end{pgfscope}%
\begin{pgfscope}%
\pgfpathrectangle{\pgfqpoint{0.457963in}{0.528059in}}{\pgfqpoint{6.200000in}{2.285714in}} %
\pgfusepath{clip}%
\pgfsetbuttcap%
\pgfsetroundjoin%
\definecolor{currentfill}{rgb}{1.000000,0.666667,0.666667}%
\pgfsetfillcolor{currentfill}%
\pgfsetlinewidth{1.003750pt}%
\definecolor{currentstroke}{rgb}{1.000000,0.666667,0.666667}%
\pgfsetstrokecolor{currentstroke}%
\pgfsetdash{}{0pt}%
\pgfpathmoveto{\pgfqpoint{0.457963in}{1.150064in}}%
\pgfpathcurveto{\pgfqpoint{0.466200in}{1.150064in}}{\pgfqpoint{0.474100in}{1.153336in}}{\pgfqpoint{0.479924in}{1.159160in}}%
\pgfpathcurveto{\pgfqpoint{0.485748in}{1.164984in}}{\pgfqpoint{0.489020in}{1.172884in}}{\pgfqpoint{0.489020in}{1.181121in}}%
\pgfpathcurveto{\pgfqpoint{0.489020in}{1.189357in}}{\pgfqpoint{0.485748in}{1.197257in}}{\pgfqpoint{0.479924in}{1.203081in}}%
\pgfpathcurveto{\pgfqpoint{0.474100in}{1.208905in}}{\pgfqpoint{0.466200in}{1.212177in}}{\pgfqpoint{0.457963in}{1.212177in}}%
\pgfpathcurveto{\pgfqpoint{0.449727in}{1.212177in}}{\pgfqpoint{0.441827in}{1.208905in}}{\pgfqpoint{0.436003in}{1.203081in}}%
\pgfpathcurveto{\pgfqpoint{0.430179in}{1.197257in}}{\pgfqpoint{0.426907in}{1.189357in}}{\pgfqpoint{0.426907in}{1.181121in}}%
\pgfpathcurveto{\pgfqpoint{0.426907in}{1.172884in}}{\pgfqpoint{0.430179in}{1.164984in}}{\pgfqpoint{0.436003in}{1.159160in}}%
\pgfpathcurveto{\pgfqpoint{0.441827in}{1.153336in}}{\pgfqpoint{0.449727in}{1.150064in}}{\pgfqpoint{0.457963in}{1.150064in}}%
\pgfpathclose%
\pgfusepath{stroke,fill}%
\end{pgfscope}%
\begin{pgfscope}%
\pgfpathrectangle{\pgfqpoint{0.457963in}{0.528059in}}{\pgfqpoint{6.200000in}{2.285714in}} %
\pgfusepath{clip}%
\pgfsetbuttcap%
\pgfsetroundjoin%
\definecolor{currentfill}{rgb}{1.000000,0.666667,0.666667}%
\pgfsetfillcolor{currentfill}%
\pgfsetlinewidth{1.003750pt}%
\definecolor{currentstroke}{rgb}{1.000000,0.666667,0.666667}%
\pgfsetstrokecolor{currentstroke}%
\pgfsetdash{}{0pt}%
\pgfpathmoveto{\pgfqpoint{0.457963in}{1.150064in}}%
\pgfpathcurveto{\pgfqpoint{0.466200in}{1.150064in}}{\pgfqpoint{0.474100in}{1.153336in}}{\pgfqpoint{0.479924in}{1.159160in}}%
\pgfpathcurveto{\pgfqpoint{0.485748in}{1.164984in}}{\pgfqpoint{0.489020in}{1.172884in}}{\pgfqpoint{0.489020in}{1.181121in}}%
\pgfpathcurveto{\pgfqpoint{0.489020in}{1.189357in}}{\pgfqpoint{0.485748in}{1.197257in}}{\pgfqpoint{0.479924in}{1.203081in}}%
\pgfpathcurveto{\pgfqpoint{0.474100in}{1.208905in}}{\pgfqpoint{0.466200in}{1.212177in}}{\pgfqpoint{0.457963in}{1.212177in}}%
\pgfpathcurveto{\pgfqpoint{0.449727in}{1.212177in}}{\pgfqpoint{0.441827in}{1.208905in}}{\pgfqpoint{0.436003in}{1.203081in}}%
\pgfpathcurveto{\pgfqpoint{0.430179in}{1.197257in}}{\pgfqpoint{0.426907in}{1.189357in}}{\pgfqpoint{0.426907in}{1.181121in}}%
\pgfpathcurveto{\pgfqpoint{0.426907in}{1.172884in}}{\pgfqpoint{0.430179in}{1.164984in}}{\pgfqpoint{0.436003in}{1.159160in}}%
\pgfpathcurveto{\pgfqpoint{0.441827in}{1.153336in}}{\pgfqpoint{0.449727in}{1.150064in}}{\pgfqpoint{0.457963in}{1.150064in}}%
\pgfpathclose%
\pgfusepath{stroke,fill}%
\end{pgfscope}%
\begin{pgfscope}%
\pgfpathrectangle{\pgfqpoint{0.457963in}{0.528059in}}{\pgfqpoint{6.200000in}{2.285714in}} %
\pgfusepath{clip}%
\pgfsetbuttcap%
\pgfsetroundjoin%
\definecolor{currentfill}{rgb}{1.000000,0.666667,0.666667}%
\pgfsetfillcolor{currentfill}%
\pgfsetlinewidth{1.003750pt}%
\definecolor{currentstroke}{rgb}{1.000000,0.666667,0.666667}%
\pgfsetstrokecolor{currentstroke}%
\pgfsetdash{}{0pt}%
\pgfpathmoveto{\pgfqpoint{0.457963in}{1.150064in}}%
\pgfpathcurveto{\pgfqpoint{0.466200in}{1.150064in}}{\pgfqpoint{0.474100in}{1.153336in}}{\pgfqpoint{0.479924in}{1.159160in}}%
\pgfpathcurveto{\pgfqpoint{0.485748in}{1.164984in}}{\pgfqpoint{0.489020in}{1.172884in}}{\pgfqpoint{0.489020in}{1.181121in}}%
\pgfpathcurveto{\pgfqpoint{0.489020in}{1.189357in}}{\pgfqpoint{0.485748in}{1.197257in}}{\pgfqpoint{0.479924in}{1.203081in}}%
\pgfpathcurveto{\pgfqpoint{0.474100in}{1.208905in}}{\pgfqpoint{0.466200in}{1.212177in}}{\pgfqpoint{0.457963in}{1.212177in}}%
\pgfpathcurveto{\pgfqpoint{0.449727in}{1.212177in}}{\pgfqpoint{0.441827in}{1.208905in}}{\pgfqpoint{0.436003in}{1.203081in}}%
\pgfpathcurveto{\pgfqpoint{0.430179in}{1.197257in}}{\pgfqpoint{0.426907in}{1.189357in}}{\pgfqpoint{0.426907in}{1.181121in}}%
\pgfpathcurveto{\pgfqpoint{0.426907in}{1.172884in}}{\pgfqpoint{0.430179in}{1.164984in}}{\pgfqpoint{0.436003in}{1.159160in}}%
\pgfpathcurveto{\pgfqpoint{0.441827in}{1.153336in}}{\pgfqpoint{0.449727in}{1.150064in}}{\pgfqpoint{0.457963in}{1.150064in}}%
\pgfpathclose%
\pgfusepath{stroke,fill}%
\end{pgfscope}%
\begin{pgfscope}%
\pgfpathrectangle{\pgfqpoint{0.457963in}{0.528059in}}{\pgfqpoint{6.200000in}{2.285714in}} %
\pgfusepath{clip}%
\pgfsetbuttcap%
\pgfsetroundjoin%
\definecolor{currentfill}{rgb}{1.000000,0.666667,0.666667}%
\pgfsetfillcolor{currentfill}%
\pgfsetlinewidth{1.003750pt}%
\definecolor{currentstroke}{rgb}{1.000000,0.666667,0.666667}%
\pgfsetstrokecolor{currentstroke}%
\pgfsetdash{}{0pt}%
\pgfpathmoveto{\pgfqpoint{0.468297in}{1.123942in}}%
\pgfpathcurveto{\pgfqpoint{0.476533in}{1.123942in}}{\pgfqpoint{0.484433in}{1.127214in}}{\pgfqpoint{0.490257in}{1.133038in}}%
\pgfpathcurveto{\pgfqpoint{0.496081in}{1.138862in}}{\pgfqpoint{0.499353in}{1.146762in}}{\pgfqpoint{0.499353in}{1.154998in}}%
\pgfpathcurveto{\pgfqpoint{0.499353in}{1.163234in}}{\pgfqpoint{0.496081in}{1.171135in}}{\pgfqpoint{0.490257in}{1.176958in}}%
\pgfpathcurveto{\pgfqpoint{0.484433in}{1.182782in}}{\pgfqpoint{0.476533in}{1.186055in}}{\pgfqpoint{0.468297in}{1.186055in}}%
\pgfpathcurveto{\pgfqpoint{0.460060in}{1.186055in}}{\pgfqpoint{0.452160in}{1.182782in}}{\pgfqpoint{0.446336in}{1.176958in}}%
\pgfpathcurveto{\pgfqpoint{0.440512in}{1.171135in}}{\pgfqpoint{0.437240in}{1.163234in}}{\pgfqpoint{0.437240in}{1.154998in}}%
\pgfpathcurveto{\pgfqpoint{0.437240in}{1.146762in}}{\pgfqpoint{0.440512in}{1.138862in}}{\pgfqpoint{0.446336in}{1.133038in}}%
\pgfpathcurveto{\pgfqpoint{0.452160in}{1.127214in}}{\pgfqpoint{0.460060in}{1.123942in}}{\pgfqpoint{0.468297in}{1.123942in}}%
\pgfpathclose%
\pgfusepath{stroke,fill}%
\end{pgfscope}%
\begin{pgfscope}%
\pgfpathrectangle{\pgfqpoint{0.457963in}{0.528059in}}{\pgfqpoint{6.200000in}{2.285714in}} %
\pgfusepath{clip}%
\pgfsetbuttcap%
\pgfsetroundjoin%
\definecolor{currentfill}{rgb}{1.000000,0.666667,0.666667}%
\pgfsetfillcolor{currentfill}%
\pgfsetlinewidth{1.003750pt}%
\definecolor{currentstroke}{rgb}{1.000000,0.666667,0.666667}%
\pgfsetstrokecolor{currentstroke}%
\pgfsetdash{}{0pt}%
\pgfpathmoveto{\pgfqpoint{0.499297in}{1.071697in}}%
\pgfpathcurveto{\pgfqpoint{0.507533in}{1.071697in}}{\pgfqpoint{0.515433in}{1.074969in}}{\pgfqpoint{0.521257in}{1.080793in}}%
\pgfpathcurveto{\pgfqpoint{0.527081in}{1.086617in}}{\pgfqpoint{0.530353in}{1.094517in}}{\pgfqpoint{0.530353in}{1.102753in}}%
\pgfpathcurveto{\pgfqpoint{0.530353in}{1.110990in}}{\pgfqpoint{0.527081in}{1.118890in}}{\pgfqpoint{0.521257in}{1.124714in}}%
\pgfpathcurveto{\pgfqpoint{0.515433in}{1.130538in}}{\pgfqpoint{0.507533in}{1.133810in}}{\pgfqpoint{0.499297in}{1.133810in}}%
\pgfpathcurveto{\pgfqpoint{0.491060in}{1.133810in}}{\pgfqpoint{0.483160in}{1.130538in}}{\pgfqpoint{0.477336in}{1.124714in}}%
\pgfpathcurveto{\pgfqpoint{0.471512in}{1.118890in}}{\pgfqpoint{0.468240in}{1.110990in}}{\pgfqpoint{0.468240in}{1.102753in}}%
\pgfpathcurveto{\pgfqpoint{0.468240in}{1.094517in}}{\pgfqpoint{0.471512in}{1.086617in}}{\pgfqpoint{0.477336in}{1.080793in}}%
\pgfpathcurveto{\pgfqpoint{0.483160in}{1.074969in}}{\pgfqpoint{0.491060in}{1.071697in}}{\pgfqpoint{0.499297in}{1.071697in}}%
\pgfpathclose%
\pgfusepath{stroke,fill}%
\end{pgfscope}%
\begin{pgfscope}%
\pgfpathrectangle{\pgfqpoint{0.457963in}{0.528059in}}{\pgfqpoint{6.200000in}{2.285714in}} %
\pgfusepath{clip}%
\pgfsetbuttcap%
\pgfsetroundjoin%
\definecolor{currentfill}{rgb}{1.000000,0.666667,0.666667}%
\pgfsetfillcolor{currentfill}%
\pgfsetlinewidth{1.003750pt}%
\definecolor{currentstroke}{rgb}{1.000000,0.666667,0.666667}%
\pgfsetstrokecolor{currentstroke}%
\pgfsetdash{}{0pt}%
\pgfpathmoveto{\pgfqpoint{0.509630in}{1.150064in}}%
\pgfpathcurveto{\pgfqpoint{0.517866in}{1.150064in}}{\pgfqpoint{0.525766in}{1.153336in}}{\pgfqpoint{0.531590in}{1.159160in}}%
\pgfpathcurveto{\pgfqpoint{0.537414in}{1.164984in}}{\pgfqpoint{0.540686in}{1.172884in}}{\pgfqpoint{0.540686in}{1.181121in}}%
\pgfpathcurveto{\pgfqpoint{0.540686in}{1.189357in}}{\pgfqpoint{0.537414in}{1.197257in}}{\pgfqpoint{0.531590in}{1.203081in}}%
\pgfpathcurveto{\pgfqpoint{0.525766in}{1.208905in}}{\pgfqpoint{0.517866in}{1.212177in}}{\pgfqpoint{0.509630in}{1.212177in}}%
\pgfpathcurveto{\pgfqpoint{0.501394in}{1.212177in}}{\pgfqpoint{0.493494in}{1.208905in}}{\pgfqpoint{0.487670in}{1.203081in}}%
\pgfpathcurveto{\pgfqpoint{0.481846in}{1.197257in}}{\pgfqpoint{0.478574in}{1.189357in}}{\pgfqpoint{0.478574in}{1.181121in}}%
\pgfpathcurveto{\pgfqpoint{0.478574in}{1.172884in}}{\pgfqpoint{0.481846in}{1.164984in}}{\pgfqpoint{0.487670in}{1.159160in}}%
\pgfpathcurveto{\pgfqpoint{0.493494in}{1.153336in}}{\pgfqpoint{0.501394in}{1.150064in}}{\pgfqpoint{0.509630in}{1.150064in}}%
\pgfpathclose%
\pgfusepath{stroke,fill}%
\end{pgfscope}%
\begin{pgfscope}%
\pgfpathrectangle{\pgfqpoint{0.457963in}{0.528059in}}{\pgfqpoint{6.200000in}{2.285714in}} %
\pgfusepath{clip}%
\pgfsetbuttcap%
\pgfsetroundjoin%
\definecolor{currentfill}{rgb}{1.000000,0.666667,0.666667}%
\pgfsetfillcolor{currentfill}%
\pgfsetlinewidth{1.003750pt}%
\definecolor{currentstroke}{rgb}{1.000000,0.666667,0.666667}%
\pgfsetstrokecolor{currentstroke}%
\pgfsetdash{}{0pt}%
\pgfpathmoveto{\pgfqpoint{0.519963in}{1.019452in}}%
\pgfpathcurveto{\pgfqpoint{0.528200in}{1.019452in}}{\pgfqpoint{0.536100in}{1.022724in}}{\pgfqpoint{0.541924in}{1.028548in}}%
\pgfpathcurveto{\pgfqpoint{0.547748in}{1.034372in}}{\pgfqpoint{0.551020in}{1.042272in}}{\pgfqpoint{0.551020in}{1.050508in}}%
\pgfpathcurveto{\pgfqpoint{0.551020in}{1.058745in}}{\pgfqpoint{0.547748in}{1.066645in}}{\pgfqpoint{0.541924in}{1.072469in}}%
\pgfpathcurveto{\pgfqpoint{0.536100in}{1.078293in}}{\pgfqpoint{0.528200in}{1.081565in}}{\pgfqpoint{0.519963in}{1.081565in}}%
\pgfpathcurveto{\pgfqpoint{0.511727in}{1.081565in}}{\pgfqpoint{0.503827in}{1.078293in}}{\pgfqpoint{0.498003in}{1.072469in}}%
\pgfpathcurveto{\pgfqpoint{0.492179in}{1.066645in}}{\pgfqpoint{0.488907in}{1.058745in}}{\pgfqpoint{0.488907in}{1.050508in}}%
\pgfpathcurveto{\pgfqpoint{0.488907in}{1.042272in}}{\pgfqpoint{0.492179in}{1.034372in}}{\pgfqpoint{0.498003in}{1.028548in}}%
\pgfpathcurveto{\pgfqpoint{0.503827in}{1.022724in}}{\pgfqpoint{0.511727in}{1.019452in}}{\pgfqpoint{0.519963in}{1.019452in}}%
\pgfpathclose%
\pgfusepath{stroke,fill}%
\end{pgfscope}%
\begin{pgfscope}%
\pgfpathrectangle{\pgfqpoint{0.457963in}{0.528059in}}{\pgfqpoint{6.200000in}{2.285714in}} %
\pgfusepath{clip}%
\pgfsetbuttcap%
\pgfsetroundjoin%
\definecolor{currentfill}{rgb}{1.000000,0.666667,0.666667}%
\pgfsetfillcolor{currentfill}%
\pgfsetlinewidth{1.003750pt}%
\definecolor{currentstroke}{rgb}{1.000000,0.666667,0.666667}%
\pgfsetstrokecolor{currentstroke}%
\pgfsetdash{}{0pt}%
\pgfpathmoveto{\pgfqpoint{0.519963in}{1.097819in}}%
\pgfpathcurveto{\pgfqpoint{0.528200in}{1.097819in}}{\pgfqpoint{0.536100in}{1.101092in}}{\pgfqpoint{0.541924in}{1.106916in}}%
\pgfpathcurveto{\pgfqpoint{0.547748in}{1.112739in}}{\pgfqpoint{0.551020in}{1.120639in}}{\pgfqpoint{0.551020in}{1.128876in}}%
\pgfpathcurveto{\pgfqpoint{0.551020in}{1.137112in}}{\pgfqpoint{0.547748in}{1.145012in}}{\pgfqpoint{0.541924in}{1.150836in}}%
\pgfpathcurveto{\pgfqpoint{0.536100in}{1.156660in}}{\pgfqpoint{0.528200in}{1.159932in}}{\pgfqpoint{0.519963in}{1.159932in}}%
\pgfpathcurveto{\pgfqpoint{0.511727in}{1.159932in}}{\pgfqpoint{0.503827in}{1.156660in}}{\pgfqpoint{0.498003in}{1.150836in}}%
\pgfpathcurveto{\pgfqpoint{0.492179in}{1.145012in}}{\pgfqpoint{0.488907in}{1.137112in}}{\pgfqpoint{0.488907in}{1.128876in}}%
\pgfpathcurveto{\pgfqpoint{0.488907in}{1.120639in}}{\pgfqpoint{0.492179in}{1.112739in}}{\pgfqpoint{0.498003in}{1.106916in}}%
\pgfpathcurveto{\pgfqpoint{0.503827in}{1.101092in}}{\pgfqpoint{0.511727in}{1.097819in}}{\pgfqpoint{0.519963in}{1.097819in}}%
\pgfpathclose%
\pgfusepath{stroke,fill}%
\end{pgfscope}%
\begin{pgfscope}%
\pgfpathrectangle{\pgfqpoint{0.457963in}{0.528059in}}{\pgfqpoint{6.200000in}{2.285714in}} %
\pgfusepath{clip}%
\pgfsetbuttcap%
\pgfsetroundjoin%
\definecolor{currentfill}{rgb}{1.000000,0.666667,0.666667}%
\pgfsetfillcolor{currentfill}%
\pgfsetlinewidth{1.003750pt}%
\definecolor{currentstroke}{rgb}{1.000000,0.666667,0.666667}%
\pgfsetstrokecolor{currentstroke}%
\pgfsetdash{}{0pt}%
\pgfpathmoveto{\pgfqpoint{0.571630in}{1.150064in}}%
\pgfpathcurveto{\pgfqpoint{0.579866in}{1.150064in}}{\pgfqpoint{0.587766in}{1.153336in}}{\pgfqpoint{0.593590in}{1.159160in}}%
\pgfpathcurveto{\pgfqpoint{0.599414in}{1.164984in}}{\pgfqpoint{0.602686in}{1.172884in}}{\pgfqpoint{0.602686in}{1.181121in}}%
\pgfpathcurveto{\pgfqpoint{0.602686in}{1.189357in}}{\pgfqpoint{0.599414in}{1.197257in}}{\pgfqpoint{0.593590in}{1.203081in}}%
\pgfpathcurveto{\pgfqpoint{0.587766in}{1.208905in}}{\pgfqpoint{0.579866in}{1.212177in}}{\pgfqpoint{0.571630in}{1.212177in}}%
\pgfpathcurveto{\pgfqpoint{0.563394in}{1.212177in}}{\pgfqpoint{0.555494in}{1.208905in}}{\pgfqpoint{0.549670in}{1.203081in}}%
\pgfpathcurveto{\pgfqpoint{0.543846in}{1.197257in}}{\pgfqpoint{0.540574in}{1.189357in}}{\pgfqpoint{0.540574in}{1.181121in}}%
\pgfpathcurveto{\pgfqpoint{0.540574in}{1.172884in}}{\pgfqpoint{0.543846in}{1.164984in}}{\pgfqpoint{0.549670in}{1.159160in}}%
\pgfpathcurveto{\pgfqpoint{0.555494in}{1.153336in}}{\pgfqpoint{0.563394in}{1.150064in}}{\pgfqpoint{0.571630in}{1.150064in}}%
\pgfpathclose%
\pgfusepath{stroke,fill}%
\end{pgfscope}%
\begin{pgfscope}%
\pgfpathrectangle{\pgfqpoint{0.457963in}{0.528059in}}{\pgfqpoint{6.200000in}{2.285714in}} %
\pgfusepath{clip}%
\pgfsetbuttcap%
\pgfsetroundjoin%
\definecolor{currentfill}{rgb}{1.000000,0.666667,0.666667}%
\pgfsetfillcolor{currentfill}%
\pgfsetlinewidth{1.003750pt}%
\definecolor{currentstroke}{rgb}{1.000000,0.666667,0.666667}%
\pgfsetstrokecolor{currentstroke}%
\pgfsetdash{}{0pt}%
\pgfpathmoveto{\pgfqpoint{0.798963in}{1.150064in}}%
\pgfpathcurveto{\pgfqpoint{0.807200in}{1.150064in}}{\pgfqpoint{0.815100in}{1.153336in}}{\pgfqpoint{0.820924in}{1.159160in}}%
\pgfpathcurveto{\pgfqpoint{0.826748in}{1.164984in}}{\pgfqpoint{0.830020in}{1.172884in}}{\pgfqpoint{0.830020in}{1.181121in}}%
\pgfpathcurveto{\pgfqpoint{0.830020in}{1.189357in}}{\pgfqpoint{0.826748in}{1.197257in}}{\pgfqpoint{0.820924in}{1.203081in}}%
\pgfpathcurveto{\pgfqpoint{0.815100in}{1.208905in}}{\pgfqpoint{0.807200in}{1.212177in}}{\pgfqpoint{0.798963in}{1.212177in}}%
\pgfpathcurveto{\pgfqpoint{0.790727in}{1.212177in}}{\pgfqpoint{0.782827in}{1.208905in}}{\pgfqpoint{0.777003in}{1.203081in}}%
\pgfpathcurveto{\pgfqpoint{0.771179in}{1.197257in}}{\pgfqpoint{0.767907in}{1.189357in}}{\pgfqpoint{0.767907in}{1.181121in}}%
\pgfpathcurveto{\pgfqpoint{0.767907in}{1.172884in}}{\pgfqpoint{0.771179in}{1.164984in}}{\pgfqpoint{0.777003in}{1.159160in}}%
\pgfpathcurveto{\pgfqpoint{0.782827in}{1.153336in}}{\pgfqpoint{0.790727in}{1.150064in}}{\pgfqpoint{0.798963in}{1.150064in}}%
\pgfpathclose%
\pgfusepath{stroke,fill}%
\end{pgfscope}%
\begin{pgfscope}%
\pgfpathrectangle{\pgfqpoint{0.457963in}{0.528059in}}{\pgfqpoint{6.200000in}{2.285714in}} %
\pgfusepath{clip}%
\pgfsetbuttcap%
\pgfsetroundjoin%
\definecolor{currentfill}{rgb}{1.000000,0.666667,0.666667}%
\pgfsetfillcolor{currentfill}%
\pgfsetlinewidth{1.003750pt}%
\definecolor{currentstroke}{rgb}{1.000000,0.666667,0.666667}%
\pgfsetstrokecolor{currentstroke}%
\pgfsetdash{}{0pt}%
\pgfpathmoveto{\pgfqpoint{0.809297in}{0.836595in}}%
\pgfpathcurveto{\pgfqpoint{0.817533in}{0.836595in}}{\pgfqpoint{0.825433in}{0.839867in}}{\pgfqpoint{0.831257in}{0.845691in}}%
\pgfpathcurveto{\pgfqpoint{0.837081in}{0.851515in}}{\pgfqpoint{0.840353in}{0.859415in}}{\pgfqpoint{0.840353in}{0.867651in}}%
\pgfpathcurveto{\pgfqpoint{0.840353in}{0.875888in}}{\pgfqpoint{0.837081in}{0.883788in}}{\pgfqpoint{0.831257in}{0.889612in}}%
\pgfpathcurveto{\pgfqpoint{0.825433in}{0.895435in}}{\pgfqpoint{0.817533in}{0.898708in}}{\pgfqpoint{0.809297in}{0.898708in}}%
\pgfpathcurveto{\pgfqpoint{0.801060in}{0.898708in}}{\pgfqpoint{0.793160in}{0.895435in}}{\pgfqpoint{0.787336in}{0.889612in}}%
\pgfpathcurveto{\pgfqpoint{0.781512in}{0.883788in}}{\pgfqpoint{0.778240in}{0.875888in}}{\pgfqpoint{0.778240in}{0.867651in}}%
\pgfpathcurveto{\pgfqpoint{0.778240in}{0.859415in}}{\pgfqpoint{0.781512in}{0.851515in}}{\pgfqpoint{0.787336in}{0.845691in}}%
\pgfpathcurveto{\pgfqpoint{0.793160in}{0.839867in}}{\pgfqpoint{0.801060in}{0.836595in}}{\pgfqpoint{0.809297in}{0.836595in}}%
\pgfpathclose%
\pgfusepath{stroke,fill}%
\end{pgfscope}%
\begin{pgfscope}%
\pgfpathrectangle{\pgfqpoint{0.457963in}{0.528059in}}{\pgfqpoint{6.200000in}{2.285714in}} %
\pgfusepath{clip}%
\pgfsetbuttcap%
\pgfsetroundjoin%
\definecolor{currentfill}{rgb}{1.000000,0.666667,0.666667}%
\pgfsetfillcolor{currentfill}%
\pgfsetlinewidth{1.003750pt}%
\definecolor{currentstroke}{rgb}{1.000000,0.666667,0.666667}%
\pgfsetstrokecolor{currentstroke}%
\pgfsetdash{}{0pt}%
\pgfpathmoveto{\pgfqpoint{1.098630in}{1.150064in}}%
\pgfpathcurveto{\pgfqpoint{1.106866in}{1.150064in}}{\pgfqpoint{1.114766in}{1.153336in}}{\pgfqpoint{1.120590in}{1.159160in}}%
\pgfpathcurveto{\pgfqpoint{1.126414in}{1.164984in}}{\pgfqpoint{1.129686in}{1.172884in}}{\pgfqpoint{1.129686in}{1.181121in}}%
\pgfpathcurveto{\pgfqpoint{1.129686in}{1.189357in}}{\pgfqpoint{1.126414in}{1.197257in}}{\pgfqpoint{1.120590in}{1.203081in}}%
\pgfpathcurveto{\pgfqpoint{1.114766in}{1.208905in}}{\pgfqpoint{1.106866in}{1.212177in}}{\pgfqpoint{1.098630in}{1.212177in}}%
\pgfpathcurveto{\pgfqpoint{1.090394in}{1.212177in}}{\pgfqpoint{1.082494in}{1.208905in}}{\pgfqpoint{1.076670in}{1.203081in}}%
\pgfpathcurveto{\pgfqpoint{1.070846in}{1.197257in}}{\pgfqpoint{1.067574in}{1.189357in}}{\pgfqpoint{1.067574in}{1.181121in}}%
\pgfpathcurveto{\pgfqpoint{1.067574in}{1.172884in}}{\pgfqpoint{1.070846in}{1.164984in}}{\pgfqpoint{1.076670in}{1.159160in}}%
\pgfpathcurveto{\pgfqpoint{1.082494in}{1.153336in}}{\pgfqpoint{1.090394in}{1.150064in}}{\pgfqpoint{1.098630in}{1.150064in}}%
\pgfpathclose%
\pgfusepath{stroke,fill}%
\end{pgfscope}%
\begin{pgfscope}%
\pgfpathrectangle{\pgfqpoint{0.457963in}{0.528059in}}{\pgfqpoint{6.200000in}{2.285714in}} %
\pgfusepath{clip}%
\pgfsetbuttcap%
\pgfsetroundjoin%
\definecolor{currentfill}{rgb}{1.000000,0.666667,0.666667}%
\pgfsetfillcolor{currentfill}%
\pgfsetlinewidth{1.003750pt}%
\definecolor{currentstroke}{rgb}{1.000000,0.666667,0.666667}%
\pgfsetstrokecolor{currentstroke}%
\pgfsetdash{}{0pt}%
\pgfpathmoveto{\pgfqpoint{1.201963in}{1.137003in}}%
\pgfpathcurveto{\pgfqpoint{1.210200in}{1.137003in}}{\pgfqpoint{1.218100in}{1.140275in}}{\pgfqpoint{1.223924in}{1.146099in}}%
\pgfpathcurveto{\pgfqpoint{1.229748in}{1.151923in}}{\pgfqpoint{1.233020in}{1.159823in}}{\pgfqpoint{1.233020in}{1.168059in}}%
\pgfpathcurveto{\pgfqpoint{1.233020in}{1.176296in}}{\pgfqpoint{1.229748in}{1.184196in}}{\pgfqpoint{1.223924in}{1.190020in}}%
\pgfpathcurveto{\pgfqpoint{1.218100in}{1.195844in}}{\pgfqpoint{1.210200in}{1.199116in}}{\pgfqpoint{1.201963in}{1.199116in}}%
\pgfpathcurveto{\pgfqpoint{1.193727in}{1.199116in}}{\pgfqpoint{1.185827in}{1.195844in}}{\pgfqpoint{1.180003in}{1.190020in}}%
\pgfpathcurveto{\pgfqpoint{1.174179in}{1.184196in}}{\pgfqpoint{1.170907in}{1.176296in}}{\pgfqpoint{1.170907in}{1.168059in}}%
\pgfpathcurveto{\pgfqpoint{1.170907in}{1.159823in}}{\pgfqpoint{1.174179in}{1.151923in}}{\pgfqpoint{1.180003in}{1.146099in}}%
\pgfpathcurveto{\pgfqpoint{1.185827in}{1.140275in}}{\pgfqpoint{1.193727in}{1.137003in}}{\pgfqpoint{1.201963in}{1.137003in}}%
\pgfpathclose%
\pgfusepath{stroke,fill}%
\end{pgfscope}%
\begin{pgfscope}%
\pgfpathrectangle{\pgfqpoint{0.457963in}{0.528059in}}{\pgfqpoint{6.200000in}{2.285714in}} %
\pgfusepath{clip}%
\pgfsetbuttcap%
\pgfsetroundjoin%
\definecolor{currentfill}{rgb}{1.000000,0.666667,0.666667}%
\pgfsetfillcolor{currentfill}%
\pgfsetlinewidth{1.003750pt}%
\definecolor{currentstroke}{rgb}{1.000000,0.666667,0.666667}%
\pgfsetstrokecolor{currentstroke}%
\pgfsetdash{}{0pt}%
\pgfpathmoveto{\pgfqpoint{1.367297in}{1.150064in}}%
\pgfpathcurveto{\pgfqpoint{1.375533in}{1.150064in}}{\pgfqpoint{1.383433in}{1.153336in}}{\pgfqpoint{1.389257in}{1.159160in}}%
\pgfpathcurveto{\pgfqpoint{1.395081in}{1.164984in}}{\pgfqpoint{1.398353in}{1.172884in}}{\pgfqpoint{1.398353in}{1.181121in}}%
\pgfpathcurveto{\pgfqpoint{1.398353in}{1.189357in}}{\pgfqpoint{1.395081in}{1.197257in}}{\pgfqpoint{1.389257in}{1.203081in}}%
\pgfpathcurveto{\pgfqpoint{1.383433in}{1.208905in}}{\pgfqpoint{1.375533in}{1.212177in}}{\pgfqpoint{1.367297in}{1.212177in}}%
\pgfpathcurveto{\pgfqpoint{1.359060in}{1.212177in}}{\pgfqpoint{1.351160in}{1.208905in}}{\pgfqpoint{1.345336in}{1.203081in}}%
\pgfpathcurveto{\pgfqpoint{1.339512in}{1.197257in}}{\pgfqpoint{1.336240in}{1.189357in}}{\pgfqpoint{1.336240in}{1.181121in}}%
\pgfpathcurveto{\pgfqpoint{1.336240in}{1.172884in}}{\pgfqpoint{1.339512in}{1.164984in}}{\pgfqpoint{1.345336in}{1.159160in}}%
\pgfpathcurveto{\pgfqpoint{1.351160in}{1.153336in}}{\pgfqpoint{1.359060in}{1.150064in}}{\pgfqpoint{1.367297in}{1.150064in}}%
\pgfpathclose%
\pgfusepath{stroke,fill}%
\end{pgfscope}%
\begin{pgfscope}%
\pgfpathrectangle{\pgfqpoint{0.457963in}{0.528059in}}{\pgfqpoint{6.200000in}{2.285714in}} %
\pgfusepath{clip}%
\pgfsetbuttcap%
\pgfsetroundjoin%
\definecolor{currentfill}{rgb}{1.000000,0.666667,0.666667}%
\pgfsetfillcolor{currentfill}%
\pgfsetlinewidth{1.003750pt}%
\definecolor{currentstroke}{rgb}{1.000000,0.666667,0.666667}%
\pgfsetstrokecolor{currentstroke}%
\pgfsetdash{}{0pt}%
\pgfpathmoveto{\pgfqpoint{1.491297in}{0.627615in}}%
\pgfpathcurveto{\pgfqpoint{1.499533in}{0.627615in}}{\pgfqpoint{1.507433in}{0.630887in}}{\pgfqpoint{1.513257in}{0.636711in}}%
\pgfpathcurveto{\pgfqpoint{1.519081in}{0.642535in}}{\pgfqpoint{1.522353in}{0.650435in}}{\pgfqpoint{1.522353in}{0.658672in}}%
\pgfpathcurveto{\pgfqpoint{1.522353in}{0.666908in}}{\pgfqpoint{1.519081in}{0.674808in}}{\pgfqpoint{1.513257in}{0.680632in}}%
\pgfpathcurveto{\pgfqpoint{1.507433in}{0.686456in}}{\pgfqpoint{1.499533in}{0.689728in}}{\pgfqpoint{1.491297in}{0.689728in}}%
\pgfpathcurveto{\pgfqpoint{1.483060in}{0.689728in}}{\pgfqpoint{1.475160in}{0.686456in}}{\pgfqpoint{1.469336in}{0.680632in}}%
\pgfpathcurveto{\pgfqpoint{1.463512in}{0.674808in}}{\pgfqpoint{1.460240in}{0.666908in}}{\pgfqpoint{1.460240in}{0.658672in}}%
\pgfpathcurveto{\pgfqpoint{1.460240in}{0.650435in}}{\pgfqpoint{1.463512in}{0.642535in}}{\pgfqpoint{1.469336in}{0.636711in}}%
\pgfpathcurveto{\pgfqpoint{1.475160in}{0.630887in}}{\pgfqpoint{1.483060in}{0.627615in}}{\pgfqpoint{1.491297in}{0.627615in}}%
\pgfpathclose%
\pgfusepath{stroke,fill}%
\end{pgfscope}%
\begin{pgfscope}%
\pgfpathrectangle{\pgfqpoint{0.457963in}{0.528059in}}{\pgfqpoint{6.200000in}{2.285714in}} %
\pgfusepath{clip}%
\pgfsetbuttcap%
\pgfsetroundjoin%
\definecolor{currentfill}{rgb}{1.000000,0.666667,0.666667}%
\pgfsetfillcolor{currentfill}%
\pgfsetlinewidth{1.003750pt}%
\definecolor{currentstroke}{rgb}{1.000000,0.666667,0.666667}%
\pgfsetstrokecolor{currentstroke}%
\pgfsetdash{}{0pt}%
\pgfpathmoveto{\pgfqpoint{1.532630in}{1.150064in}}%
\pgfpathcurveto{\pgfqpoint{1.540866in}{1.150064in}}{\pgfqpoint{1.548766in}{1.153336in}}{\pgfqpoint{1.554590in}{1.159160in}}%
\pgfpathcurveto{\pgfqpoint{1.560414in}{1.164984in}}{\pgfqpoint{1.563686in}{1.172884in}}{\pgfqpoint{1.563686in}{1.181121in}}%
\pgfpathcurveto{\pgfqpoint{1.563686in}{1.189357in}}{\pgfqpoint{1.560414in}{1.197257in}}{\pgfqpoint{1.554590in}{1.203081in}}%
\pgfpathcurveto{\pgfqpoint{1.548766in}{1.208905in}}{\pgfqpoint{1.540866in}{1.212177in}}{\pgfqpoint{1.532630in}{1.212177in}}%
\pgfpathcurveto{\pgfqpoint{1.524394in}{1.212177in}}{\pgfqpoint{1.516494in}{1.208905in}}{\pgfqpoint{1.510670in}{1.203081in}}%
\pgfpathcurveto{\pgfqpoint{1.504846in}{1.197257in}}{\pgfqpoint{1.501574in}{1.189357in}}{\pgfqpoint{1.501574in}{1.181121in}}%
\pgfpathcurveto{\pgfqpoint{1.501574in}{1.172884in}}{\pgfqpoint{1.504846in}{1.164984in}}{\pgfqpoint{1.510670in}{1.159160in}}%
\pgfpathcurveto{\pgfqpoint{1.516494in}{1.153336in}}{\pgfqpoint{1.524394in}{1.150064in}}{\pgfqpoint{1.532630in}{1.150064in}}%
\pgfpathclose%
\pgfusepath{stroke,fill}%
\end{pgfscope}%
\begin{pgfscope}%
\pgfpathrectangle{\pgfqpoint{0.457963in}{0.528059in}}{\pgfqpoint{6.200000in}{2.285714in}} %
\pgfusepath{clip}%
\pgfsetbuttcap%
\pgfsetroundjoin%
\definecolor{currentfill}{rgb}{1.000000,0.666667,0.666667}%
\pgfsetfillcolor{currentfill}%
\pgfsetlinewidth{1.003750pt}%
\definecolor{currentstroke}{rgb}{1.000000,0.666667,0.666667}%
\pgfsetstrokecolor{currentstroke}%
\pgfsetdash{}{0pt}%
\pgfpathmoveto{\pgfqpoint{1.759963in}{1.110880in}}%
\pgfpathcurveto{\pgfqpoint{1.768200in}{1.110880in}}{\pgfqpoint{1.776100in}{1.114153in}}{\pgfqpoint{1.781924in}{1.119977in}}%
\pgfpathcurveto{\pgfqpoint{1.787748in}{1.125801in}}{\pgfqpoint{1.791020in}{1.133701in}}{\pgfqpoint{1.791020in}{1.141937in}}%
\pgfpathcurveto{\pgfqpoint{1.791020in}{1.150173in}}{\pgfqpoint{1.787748in}{1.158073in}}{\pgfqpoint{1.781924in}{1.163897in}}%
\pgfpathcurveto{\pgfqpoint{1.776100in}{1.169721in}}{\pgfqpoint{1.768200in}{1.172993in}}{\pgfqpoint{1.759963in}{1.172993in}}%
\pgfpathcurveto{\pgfqpoint{1.751727in}{1.172993in}}{\pgfqpoint{1.743827in}{1.169721in}}{\pgfqpoint{1.738003in}{1.163897in}}%
\pgfpathcurveto{\pgfqpoint{1.732179in}{1.158073in}}{\pgfqpoint{1.728907in}{1.150173in}}{\pgfqpoint{1.728907in}{1.141937in}}%
\pgfpathcurveto{\pgfqpoint{1.728907in}{1.133701in}}{\pgfqpoint{1.732179in}{1.125801in}}{\pgfqpoint{1.738003in}{1.119977in}}%
\pgfpathcurveto{\pgfqpoint{1.743827in}{1.114153in}}{\pgfqpoint{1.751727in}{1.110880in}}{\pgfqpoint{1.759963in}{1.110880in}}%
\pgfpathclose%
\pgfusepath{stroke,fill}%
\end{pgfscope}%
\begin{pgfscope}%
\pgfpathrectangle{\pgfqpoint{0.457963in}{0.528059in}}{\pgfqpoint{6.200000in}{2.285714in}} %
\pgfusepath{clip}%
\pgfsetbuttcap%
\pgfsetroundjoin%
\definecolor{currentfill}{rgb}{1.000000,0.666667,0.666667}%
\pgfsetfillcolor{currentfill}%
\pgfsetlinewidth{1.003750pt}%
\definecolor{currentstroke}{rgb}{1.000000,0.666667,0.666667}%
\pgfsetstrokecolor{currentstroke}%
\pgfsetdash{}{0pt}%
\pgfpathmoveto{\pgfqpoint{2.514297in}{0.928023in}}%
\pgfpathcurveto{\pgfqpoint{2.522533in}{0.928023in}}{\pgfqpoint{2.530433in}{0.931296in}}{\pgfqpoint{2.536257in}{0.937120in}}%
\pgfpathcurveto{\pgfqpoint{2.542081in}{0.942944in}}{\pgfqpoint{2.545353in}{0.950844in}}{\pgfqpoint{2.545353in}{0.959080in}}%
\pgfpathcurveto{\pgfqpoint{2.545353in}{0.967316in}}{\pgfqpoint{2.542081in}{0.975216in}}{\pgfqpoint{2.536257in}{0.981040in}}%
\pgfpathcurveto{\pgfqpoint{2.530433in}{0.986864in}}{\pgfqpoint{2.522533in}{0.990136in}}{\pgfqpoint{2.514297in}{0.990136in}}%
\pgfpathcurveto{\pgfqpoint{2.506060in}{0.990136in}}{\pgfqpoint{2.498160in}{0.986864in}}{\pgfqpoint{2.492336in}{0.981040in}}%
\pgfpathcurveto{\pgfqpoint{2.486512in}{0.975216in}}{\pgfqpoint{2.483240in}{0.967316in}}{\pgfqpoint{2.483240in}{0.959080in}}%
\pgfpathcurveto{\pgfqpoint{2.483240in}{0.950844in}}{\pgfqpoint{2.486512in}{0.942944in}}{\pgfqpoint{2.492336in}{0.937120in}}%
\pgfpathcurveto{\pgfqpoint{2.498160in}{0.931296in}}{\pgfqpoint{2.506060in}{0.928023in}}{\pgfqpoint{2.514297in}{0.928023in}}%
\pgfpathclose%
\pgfusepath{stroke,fill}%
\end{pgfscope}%
\begin{pgfscope}%
\pgfpathrectangle{\pgfqpoint{0.457963in}{0.528059in}}{\pgfqpoint{6.200000in}{2.285714in}} %
\pgfusepath{clip}%
\pgfsetbuttcap%
\pgfsetroundjoin%
\definecolor{currentfill}{rgb}{1.000000,0.666667,0.666667}%
\pgfsetfillcolor{currentfill}%
\pgfsetlinewidth{1.003750pt}%
\definecolor{currentstroke}{rgb}{1.000000,0.666667,0.666667}%
\pgfsetstrokecolor{currentstroke}%
\pgfsetdash{}{0pt}%
\pgfpathmoveto{\pgfqpoint{2.762297in}{0.836595in}}%
\pgfpathcurveto{\pgfqpoint{2.770533in}{0.836595in}}{\pgfqpoint{2.778433in}{0.839867in}}{\pgfqpoint{2.784257in}{0.845691in}}%
\pgfpathcurveto{\pgfqpoint{2.790081in}{0.851515in}}{\pgfqpoint{2.793353in}{0.859415in}}{\pgfqpoint{2.793353in}{0.867651in}}%
\pgfpathcurveto{\pgfqpoint{2.793353in}{0.875888in}}{\pgfqpoint{2.790081in}{0.883788in}}{\pgfqpoint{2.784257in}{0.889612in}}%
\pgfpathcurveto{\pgfqpoint{2.778433in}{0.895435in}}{\pgfqpoint{2.770533in}{0.898708in}}{\pgfqpoint{2.762297in}{0.898708in}}%
\pgfpathcurveto{\pgfqpoint{2.754060in}{0.898708in}}{\pgfqpoint{2.746160in}{0.895435in}}{\pgfqpoint{2.740336in}{0.889612in}}%
\pgfpathcurveto{\pgfqpoint{2.734512in}{0.883788in}}{\pgfqpoint{2.731240in}{0.875888in}}{\pgfqpoint{2.731240in}{0.867651in}}%
\pgfpathcurveto{\pgfqpoint{2.731240in}{0.859415in}}{\pgfqpoint{2.734512in}{0.851515in}}{\pgfqpoint{2.740336in}{0.845691in}}%
\pgfpathcurveto{\pgfqpoint{2.746160in}{0.839867in}}{\pgfqpoint{2.754060in}{0.836595in}}{\pgfqpoint{2.762297in}{0.836595in}}%
\pgfpathclose%
\pgfusepath{stroke,fill}%
\end{pgfscope}%
\begin{pgfscope}%
\pgfpathrectangle{\pgfqpoint{0.457963in}{0.528059in}}{\pgfqpoint{6.200000in}{2.285714in}} %
\pgfusepath{clip}%
\pgfsetbuttcap%
\pgfsetroundjoin%
\definecolor{currentfill}{rgb}{1.000000,0.500000,0.500000}%
\pgfsetfillcolor{currentfill}%
\pgfsetlinewidth{1.003750pt}%
\definecolor{currentstroke}{rgb}{1.000000,0.500000,0.500000}%
\pgfsetstrokecolor{currentstroke}%
\pgfsetdash{}{0pt}%
\pgfpathmoveto{\pgfqpoint{0.457963in}{1.476595in}}%
\pgfpathcurveto{\pgfqpoint{0.466200in}{1.476595in}}{\pgfqpoint{0.474100in}{1.479867in}}{\pgfqpoint{0.479924in}{1.485691in}}%
\pgfpathcurveto{\pgfqpoint{0.485748in}{1.491515in}}{\pgfqpoint{0.489020in}{1.499415in}}{\pgfqpoint{0.489020in}{1.507651in}}%
\pgfpathcurveto{\pgfqpoint{0.489020in}{1.515888in}}{\pgfqpoint{0.485748in}{1.523788in}}{\pgfqpoint{0.479924in}{1.529612in}}%
\pgfpathcurveto{\pgfqpoint{0.474100in}{1.535435in}}{\pgfqpoint{0.466200in}{1.538708in}}{\pgfqpoint{0.457963in}{1.538708in}}%
\pgfpathcurveto{\pgfqpoint{0.449727in}{1.538708in}}{\pgfqpoint{0.441827in}{1.535435in}}{\pgfqpoint{0.436003in}{1.529612in}}%
\pgfpathcurveto{\pgfqpoint{0.430179in}{1.523788in}}{\pgfqpoint{0.426907in}{1.515888in}}{\pgfqpoint{0.426907in}{1.507651in}}%
\pgfpathcurveto{\pgfqpoint{0.426907in}{1.499415in}}{\pgfqpoint{0.430179in}{1.491515in}}{\pgfqpoint{0.436003in}{1.485691in}}%
\pgfpathcurveto{\pgfqpoint{0.441827in}{1.479867in}}{\pgfqpoint{0.449727in}{1.476595in}}{\pgfqpoint{0.457963in}{1.476595in}}%
\pgfpathclose%
\pgfusepath{stroke,fill}%
\end{pgfscope}%
\begin{pgfscope}%
\pgfpathrectangle{\pgfqpoint{0.457963in}{0.528059in}}{\pgfqpoint{6.200000in}{2.285714in}} %
\pgfusepath{clip}%
\pgfsetbuttcap%
\pgfsetroundjoin%
\definecolor{currentfill}{rgb}{1.000000,0.500000,0.500000}%
\pgfsetfillcolor{currentfill}%
\pgfsetlinewidth{1.003750pt}%
\definecolor{currentstroke}{rgb}{1.000000,0.500000,0.500000}%
\pgfsetstrokecolor{currentstroke}%
\pgfsetdash{}{0pt}%
\pgfpathmoveto{\pgfqpoint{0.457963in}{1.476595in}}%
\pgfpathcurveto{\pgfqpoint{0.466200in}{1.476595in}}{\pgfqpoint{0.474100in}{1.479867in}}{\pgfqpoint{0.479924in}{1.485691in}}%
\pgfpathcurveto{\pgfqpoint{0.485748in}{1.491515in}}{\pgfqpoint{0.489020in}{1.499415in}}{\pgfqpoint{0.489020in}{1.507651in}}%
\pgfpathcurveto{\pgfqpoint{0.489020in}{1.515888in}}{\pgfqpoint{0.485748in}{1.523788in}}{\pgfqpoint{0.479924in}{1.529612in}}%
\pgfpathcurveto{\pgfqpoint{0.474100in}{1.535435in}}{\pgfqpoint{0.466200in}{1.538708in}}{\pgfqpoint{0.457963in}{1.538708in}}%
\pgfpathcurveto{\pgfqpoint{0.449727in}{1.538708in}}{\pgfqpoint{0.441827in}{1.535435in}}{\pgfqpoint{0.436003in}{1.529612in}}%
\pgfpathcurveto{\pgfqpoint{0.430179in}{1.523788in}}{\pgfqpoint{0.426907in}{1.515888in}}{\pgfqpoint{0.426907in}{1.507651in}}%
\pgfpathcurveto{\pgfqpoint{0.426907in}{1.499415in}}{\pgfqpoint{0.430179in}{1.491515in}}{\pgfqpoint{0.436003in}{1.485691in}}%
\pgfpathcurveto{\pgfqpoint{0.441827in}{1.479867in}}{\pgfqpoint{0.449727in}{1.476595in}}{\pgfqpoint{0.457963in}{1.476595in}}%
\pgfpathclose%
\pgfusepath{stroke,fill}%
\end{pgfscope}%
\begin{pgfscope}%
\pgfpathrectangle{\pgfqpoint{0.457963in}{0.528059in}}{\pgfqpoint{6.200000in}{2.285714in}} %
\pgfusepath{clip}%
\pgfsetbuttcap%
\pgfsetroundjoin%
\definecolor{currentfill}{rgb}{1.000000,0.500000,0.500000}%
\pgfsetfillcolor{currentfill}%
\pgfsetlinewidth{1.003750pt}%
\definecolor{currentstroke}{rgb}{1.000000,0.500000,0.500000}%
\pgfsetstrokecolor{currentstroke}%
\pgfsetdash{}{0pt}%
\pgfpathmoveto{\pgfqpoint{0.457963in}{1.476595in}}%
\pgfpathcurveto{\pgfqpoint{0.466200in}{1.476595in}}{\pgfqpoint{0.474100in}{1.479867in}}{\pgfqpoint{0.479924in}{1.485691in}}%
\pgfpathcurveto{\pgfqpoint{0.485748in}{1.491515in}}{\pgfqpoint{0.489020in}{1.499415in}}{\pgfqpoint{0.489020in}{1.507651in}}%
\pgfpathcurveto{\pgfqpoint{0.489020in}{1.515888in}}{\pgfqpoint{0.485748in}{1.523788in}}{\pgfqpoint{0.479924in}{1.529612in}}%
\pgfpathcurveto{\pgfqpoint{0.474100in}{1.535435in}}{\pgfqpoint{0.466200in}{1.538708in}}{\pgfqpoint{0.457963in}{1.538708in}}%
\pgfpathcurveto{\pgfqpoint{0.449727in}{1.538708in}}{\pgfqpoint{0.441827in}{1.535435in}}{\pgfqpoint{0.436003in}{1.529612in}}%
\pgfpathcurveto{\pgfqpoint{0.430179in}{1.523788in}}{\pgfqpoint{0.426907in}{1.515888in}}{\pgfqpoint{0.426907in}{1.507651in}}%
\pgfpathcurveto{\pgfqpoint{0.426907in}{1.499415in}}{\pgfqpoint{0.430179in}{1.491515in}}{\pgfqpoint{0.436003in}{1.485691in}}%
\pgfpathcurveto{\pgfqpoint{0.441827in}{1.479867in}}{\pgfqpoint{0.449727in}{1.476595in}}{\pgfqpoint{0.457963in}{1.476595in}}%
\pgfpathclose%
\pgfusepath{stroke,fill}%
\end{pgfscope}%
\begin{pgfscope}%
\pgfpathrectangle{\pgfqpoint{0.457963in}{0.528059in}}{\pgfqpoint{6.200000in}{2.285714in}} %
\pgfusepath{clip}%
\pgfsetbuttcap%
\pgfsetroundjoin%
\definecolor{currentfill}{rgb}{1.000000,0.500000,0.500000}%
\pgfsetfillcolor{currentfill}%
\pgfsetlinewidth{1.003750pt}%
\definecolor{currentstroke}{rgb}{1.000000,0.500000,0.500000}%
\pgfsetstrokecolor{currentstroke}%
\pgfsetdash{}{0pt}%
\pgfpathmoveto{\pgfqpoint{0.457963in}{1.476595in}}%
\pgfpathcurveto{\pgfqpoint{0.466200in}{1.476595in}}{\pgfqpoint{0.474100in}{1.479867in}}{\pgfqpoint{0.479924in}{1.485691in}}%
\pgfpathcurveto{\pgfqpoint{0.485748in}{1.491515in}}{\pgfqpoint{0.489020in}{1.499415in}}{\pgfqpoint{0.489020in}{1.507651in}}%
\pgfpathcurveto{\pgfqpoint{0.489020in}{1.515888in}}{\pgfqpoint{0.485748in}{1.523788in}}{\pgfqpoint{0.479924in}{1.529612in}}%
\pgfpathcurveto{\pgfqpoint{0.474100in}{1.535435in}}{\pgfqpoint{0.466200in}{1.538708in}}{\pgfqpoint{0.457963in}{1.538708in}}%
\pgfpathcurveto{\pgfqpoint{0.449727in}{1.538708in}}{\pgfqpoint{0.441827in}{1.535435in}}{\pgfqpoint{0.436003in}{1.529612in}}%
\pgfpathcurveto{\pgfqpoint{0.430179in}{1.523788in}}{\pgfqpoint{0.426907in}{1.515888in}}{\pgfqpoint{0.426907in}{1.507651in}}%
\pgfpathcurveto{\pgfqpoint{0.426907in}{1.499415in}}{\pgfqpoint{0.430179in}{1.491515in}}{\pgfqpoint{0.436003in}{1.485691in}}%
\pgfpathcurveto{\pgfqpoint{0.441827in}{1.479867in}}{\pgfqpoint{0.449727in}{1.476595in}}{\pgfqpoint{0.457963in}{1.476595in}}%
\pgfpathclose%
\pgfusepath{stroke,fill}%
\end{pgfscope}%
\begin{pgfscope}%
\pgfpathrectangle{\pgfqpoint{0.457963in}{0.528059in}}{\pgfqpoint{6.200000in}{2.285714in}} %
\pgfusepath{clip}%
\pgfsetbuttcap%
\pgfsetroundjoin%
\definecolor{currentfill}{rgb}{1.000000,0.500000,0.500000}%
\pgfsetfillcolor{currentfill}%
\pgfsetlinewidth{1.003750pt}%
\definecolor{currentstroke}{rgb}{1.000000,0.500000,0.500000}%
\pgfsetstrokecolor{currentstroke}%
\pgfsetdash{}{0pt}%
\pgfpathmoveto{\pgfqpoint{0.457963in}{1.476595in}}%
\pgfpathcurveto{\pgfqpoint{0.466200in}{1.476595in}}{\pgfqpoint{0.474100in}{1.479867in}}{\pgfqpoint{0.479924in}{1.485691in}}%
\pgfpathcurveto{\pgfqpoint{0.485748in}{1.491515in}}{\pgfqpoint{0.489020in}{1.499415in}}{\pgfqpoint{0.489020in}{1.507651in}}%
\pgfpathcurveto{\pgfqpoint{0.489020in}{1.515888in}}{\pgfqpoint{0.485748in}{1.523788in}}{\pgfqpoint{0.479924in}{1.529612in}}%
\pgfpathcurveto{\pgfqpoint{0.474100in}{1.535435in}}{\pgfqpoint{0.466200in}{1.538708in}}{\pgfqpoint{0.457963in}{1.538708in}}%
\pgfpathcurveto{\pgfqpoint{0.449727in}{1.538708in}}{\pgfqpoint{0.441827in}{1.535435in}}{\pgfqpoint{0.436003in}{1.529612in}}%
\pgfpathcurveto{\pgfqpoint{0.430179in}{1.523788in}}{\pgfqpoint{0.426907in}{1.515888in}}{\pgfqpoint{0.426907in}{1.507651in}}%
\pgfpathcurveto{\pgfqpoint{0.426907in}{1.499415in}}{\pgfqpoint{0.430179in}{1.491515in}}{\pgfqpoint{0.436003in}{1.485691in}}%
\pgfpathcurveto{\pgfqpoint{0.441827in}{1.479867in}}{\pgfqpoint{0.449727in}{1.476595in}}{\pgfqpoint{0.457963in}{1.476595in}}%
\pgfpathclose%
\pgfusepath{stroke,fill}%
\end{pgfscope}%
\begin{pgfscope}%
\pgfpathrectangle{\pgfqpoint{0.457963in}{0.528059in}}{\pgfqpoint{6.200000in}{2.285714in}} %
\pgfusepath{clip}%
\pgfsetbuttcap%
\pgfsetroundjoin%
\definecolor{currentfill}{rgb}{1.000000,0.500000,0.500000}%
\pgfsetfillcolor{currentfill}%
\pgfsetlinewidth{1.003750pt}%
\definecolor{currentstroke}{rgb}{1.000000,0.500000,0.500000}%
\pgfsetstrokecolor{currentstroke}%
\pgfsetdash{}{0pt}%
\pgfpathmoveto{\pgfqpoint{0.468297in}{1.437411in}}%
\pgfpathcurveto{\pgfqpoint{0.476533in}{1.437411in}}{\pgfqpoint{0.484433in}{1.440683in}}{\pgfqpoint{0.490257in}{1.446507in}}%
\pgfpathcurveto{\pgfqpoint{0.496081in}{1.452331in}}{\pgfqpoint{0.499353in}{1.460231in}}{\pgfqpoint{0.499353in}{1.468468in}}%
\pgfpathcurveto{\pgfqpoint{0.499353in}{1.476704in}}{\pgfqpoint{0.496081in}{1.484604in}}{\pgfqpoint{0.490257in}{1.490428in}}%
\pgfpathcurveto{\pgfqpoint{0.484433in}{1.496252in}}{\pgfqpoint{0.476533in}{1.499524in}}{\pgfqpoint{0.468297in}{1.499524in}}%
\pgfpathcurveto{\pgfqpoint{0.460060in}{1.499524in}}{\pgfqpoint{0.452160in}{1.496252in}}{\pgfqpoint{0.446336in}{1.490428in}}%
\pgfpathcurveto{\pgfqpoint{0.440512in}{1.484604in}}{\pgfqpoint{0.437240in}{1.476704in}}{\pgfqpoint{0.437240in}{1.468468in}}%
\pgfpathcurveto{\pgfqpoint{0.437240in}{1.460231in}}{\pgfqpoint{0.440512in}{1.452331in}}{\pgfqpoint{0.446336in}{1.446507in}}%
\pgfpathcurveto{\pgfqpoint{0.452160in}{1.440683in}}{\pgfqpoint{0.460060in}{1.437411in}}{\pgfqpoint{0.468297in}{1.437411in}}%
\pgfpathclose%
\pgfusepath{stroke,fill}%
\end{pgfscope}%
\begin{pgfscope}%
\pgfpathrectangle{\pgfqpoint{0.457963in}{0.528059in}}{\pgfqpoint{6.200000in}{2.285714in}} %
\pgfusepath{clip}%
\pgfsetbuttcap%
\pgfsetroundjoin%
\definecolor{currentfill}{rgb}{1.000000,0.500000,0.500000}%
\pgfsetfillcolor{currentfill}%
\pgfsetlinewidth{1.003750pt}%
\definecolor{currentstroke}{rgb}{1.000000,0.500000,0.500000}%
\pgfsetstrokecolor{currentstroke}%
\pgfsetdash{}{0pt}%
\pgfpathmoveto{\pgfqpoint{0.468297in}{1.476595in}}%
\pgfpathcurveto{\pgfqpoint{0.476533in}{1.476595in}}{\pgfqpoint{0.484433in}{1.479867in}}{\pgfqpoint{0.490257in}{1.485691in}}%
\pgfpathcurveto{\pgfqpoint{0.496081in}{1.491515in}}{\pgfqpoint{0.499353in}{1.499415in}}{\pgfqpoint{0.499353in}{1.507651in}}%
\pgfpathcurveto{\pgfqpoint{0.499353in}{1.515888in}}{\pgfqpoint{0.496081in}{1.523788in}}{\pgfqpoint{0.490257in}{1.529612in}}%
\pgfpathcurveto{\pgfqpoint{0.484433in}{1.535435in}}{\pgfqpoint{0.476533in}{1.538708in}}{\pgfqpoint{0.468297in}{1.538708in}}%
\pgfpathcurveto{\pgfqpoint{0.460060in}{1.538708in}}{\pgfqpoint{0.452160in}{1.535435in}}{\pgfqpoint{0.446336in}{1.529612in}}%
\pgfpathcurveto{\pgfqpoint{0.440512in}{1.523788in}}{\pgfqpoint{0.437240in}{1.515888in}}{\pgfqpoint{0.437240in}{1.507651in}}%
\pgfpathcurveto{\pgfqpoint{0.437240in}{1.499415in}}{\pgfqpoint{0.440512in}{1.491515in}}{\pgfqpoint{0.446336in}{1.485691in}}%
\pgfpathcurveto{\pgfqpoint{0.452160in}{1.479867in}}{\pgfqpoint{0.460060in}{1.476595in}}{\pgfqpoint{0.468297in}{1.476595in}}%
\pgfpathclose%
\pgfusepath{stroke,fill}%
\end{pgfscope}%
\begin{pgfscope}%
\pgfpathrectangle{\pgfqpoint{0.457963in}{0.528059in}}{\pgfqpoint{6.200000in}{2.285714in}} %
\pgfusepath{clip}%
\pgfsetbuttcap%
\pgfsetroundjoin%
\definecolor{currentfill}{rgb}{1.000000,0.500000,0.500000}%
\pgfsetfillcolor{currentfill}%
\pgfsetlinewidth{1.003750pt}%
\definecolor{currentstroke}{rgb}{1.000000,0.500000,0.500000}%
\pgfsetstrokecolor{currentstroke}%
\pgfsetdash{}{0pt}%
\pgfpathmoveto{\pgfqpoint{0.561297in}{1.359044in}}%
\pgfpathcurveto{\pgfqpoint{0.569533in}{1.359044in}}{\pgfqpoint{0.577433in}{1.362316in}}{\pgfqpoint{0.583257in}{1.368140in}}%
\pgfpathcurveto{\pgfqpoint{0.589081in}{1.373964in}}{\pgfqpoint{0.592353in}{1.381864in}}{\pgfqpoint{0.592353in}{1.390100in}}%
\pgfpathcurveto{\pgfqpoint{0.592353in}{1.398337in}}{\pgfqpoint{0.589081in}{1.406237in}}{\pgfqpoint{0.583257in}{1.412061in}}%
\pgfpathcurveto{\pgfqpoint{0.577433in}{1.417884in}}{\pgfqpoint{0.569533in}{1.421157in}}{\pgfqpoint{0.561297in}{1.421157in}}%
\pgfpathcurveto{\pgfqpoint{0.553060in}{1.421157in}}{\pgfqpoint{0.545160in}{1.417884in}}{\pgfqpoint{0.539336in}{1.412061in}}%
\pgfpathcurveto{\pgfqpoint{0.533512in}{1.406237in}}{\pgfqpoint{0.530240in}{1.398337in}}{\pgfqpoint{0.530240in}{1.390100in}}%
\pgfpathcurveto{\pgfqpoint{0.530240in}{1.381864in}}{\pgfqpoint{0.533512in}{1.373964in}}{\pgfqpoint{0.539336in}{1.368140in}}%
\pgfpathcurveto{\pgfqpoint{0.545160in}{1.362316in}}{\pgfqpoint{0.553060in}{1.359044in}}{\pgfqpoint{0.561297in}{1.359044in}}%
\pgfpathclose%
\pgfusepath{stroke,fill}%
\end{pgfscope}%
\begin{pgfscope}%
\pgfpathrectangle{\pgfqpoint{0.457963in}{0.528059in}}{\pgfqpoint{6.200000in}{2.285714in}} %
\pgfusepath{clip}%
\pgfsetbuttcap%
\pgfsetroundjoin%
\definecolor{currentfill}{rgb}{1.000000,0.500000,0.500000}%
\pgfsetfillcolor{currentfill}%
\pgfsetlinewidth{1.003750pt}%
\definecolor{currentstroke}{rgb}{1.000000,0.500000,0.500000}%
\pgfsetstrokecolor{currentstroke}%
\pgfsetdash{}{0pt}%
\pgfpathmoveto{\pgfqpoint{0.685297in}{1.319860in}}%
\pgfpathcurveto{\pgfqpoint{0.693533in}{1.319860in}}{\pgfqpoint{0.701433in}{1.323132in}}{\pgfqpoint{0.707257in}{1.328956in}}%
\pgfpathcurveto{\pgfqpoint{0.713081in}{1.334780in}}{\pgfqpoint{0.716353in}{1.342680in}}{\pgfqpoint{0.716353in}{1.350917in}}%
\pgfpathcurveto{\pgfqpoint{0.716353in}{1.359153in}}{\pgfqpoint{0.713081in}{1.367053in}}{\pgfqpoint{0.707257in}{1.372877in}}%
\pgfpathcurveto{\pgfqpoint{0.701433in}{1.378701in}}{\pgfqpoint{0.693533in}{1.381973in}}{\pgfqpoint{0.685297in}{1.381973in}}%
\pgfpathcurveto{\pgfqpoint{0.677060in}{1.381973in}}{\pgfqpoint{0.669160in}{1.378701in}}{\pgfqpoint{0.663336in}{1.372877in}}%
\pgfpathcurveto{\pgfqpoint{0.657512in}{1.367053in}}{\pgfqpoint{0.654240in}{1.359153in}}{\pgfqpoint{0.654240in}{1.350917in}}%
\pgfpathcurveto{\pgfqpoint{0.654240in}{1.342680in}}{\pgfqpoint{0.657512in}{1.334780in}}{\pgfqpoint{0.663336in}{1.328956in}}%
\pgfpathcurveto{\pgfqpoint{0.669160in}{1.323132in}}{\pgfqpoint{0.677060in}{1.319860in}}{\pgfqpoint{0.685297in}{1.319860in}}%
\pgfpathclose%
\pgfusepath{stroke,fill}%
\end{pgfscope}%
\begin{pgfscope}%
\pgfpathrectangle{\pgfqpoint{0.457963in}{0.528059in}}{\pgfqpoint{6.200000in}{2.285714in}} %
\pgfusepath{clip}%
\pgfsetbuttcap%
\pgfsetroundjoin%
\definecolor{currentfill}{rgb}{1.000000,0.500000,0.500000}%
\pgfsetfillcolor{currentfill}%
\pgfsetlinewidth{1.003750pt}%
\definecolor{currentstroke}{rgb}{1.000000,0.500000,0.500000}%
\pgfsetstrokecolor{currentstroke}%
\pgfsetdash{}{0pt}%
\pgfpathmoveto{\pgfqpoint{0.736963in}{1.476595in}}%
\pgfpathcurveto{\pgfqpoint{0.745200in}{1.476595in}}{\pgfqpoint{0.753100in}{1.479867in}}{\pgfqpoint{0.758924in}{1.485691in}}%
\pgfpathcurveto{\pgfqpoint{0.764748in}{1.491515in}}{\pgfqpoint{0.768020in}{1.499415in}}{\pgfqpoint{0.768020in}{1.507651in}}%
\pgfpathcurveto{\pgfqpoint{0.768020in}{1.515888in}}{\pgfqpoint{0.764748in}{1.523788in}}{\pgfqpoint{0.758924in}{1.529612in}}%
\pgfpathcurveto{\pgfqpoint{0.753100in}{1.535435in}}{\pgfqpoint{0.745200in}{1.538708in}}{\pgfqpoint{0.736963in}{1.538708in}}%
\pgfpathcurveto{\pgfqpoint{0.728727in}{1.538708in}}{\pgfqpoint{0.720827in}{1.535435in}}{\pgfqpoint{0.715003in}{1.529612in}}%
\pgfpathcurveto{\pgfqpoint{0.709179in}{1.523788in}}{\pgfqpoint{0.705907in}{1.515888in}}{\pgfqpoint{0.705907in}{1.507651in}}%
\pgfpathcurveto{\pgfqpoint{0.705907in}{1.499415in}}{\pgfqpoint{0.709179in}{1.491515in}}{\pgfqpoint{0.715003in}{1.485691in}}%
\pgfpathcurveto{\pgfqpoint{0.720827in}{1.479867in}}{\pgfqpoint{0.728727in}{1.476595in}}{\pgfqpoint{0.736963in}{1.476595in}}%
\pgfpathclose%
\pgfusepath{stroke,fill}%
\end{pgfscope}%
\begin{pgfscope}%
\pgfpathrectangle{\pgfqpoint{0.457963in}{0.528059in}}{\pgfqpoint{6.200000in}{2.285714in}} %
\pgfusepath{clip}%
\pgfsetbuttcap%
\pgfsetroundjoin%
\definecolor{currentfill}{rgb}{1.000000,0.500000,0.500000}%
\pgfsetfillcolor{currentfill}%
\pgfsetlinewidth{1.003750pt}%
\definecolor{currentstroke}{rgb}{1.000000,0.500000,0.500000}%
\pgfsetstrokecolor{currentstroke}%
\pgfsetdash{}{0pt}%
\pgfpathmoveto{\pgfqpoint{0.860963in}{1.476595in}}%
\pgfpathcurveto{\pgfqpoint{0.869200in}{1.476595in}}{\pgfqpoint{0.877100in}{1.479867in}}{\pgfqpoint{0.882924in}{1.485691in}}%
\pgfpathcurveto{\pgfqpoint{0.888748in}{1.491515in}}{\pgfqpoint{0.892020in}{1.499415in}}{\pgfqpoint{0.892020in}{1.507651in}}%
\pgfpathcurveto{\pgfqpoint{0.892020in}{1.515888in}}{\pgfqpoint{0.888748in}{1.523788in}}{\pgfqpoint{0.882924in}{1.529612in}}%
\pgfpathcurveto{\pgfqpoint{0.877100in}{1.535435in}}{\pgfqpoint{0.869200in}{1.538708in}}{\pgfqpoint{0.860963in}{1.538708in}}%
\pgfpathcurveto{\pgfqpoint{0.852727in}{1.538708in}}{\pgfqpoint{0.844827in}{1.535435in}}{\pgfqpoint{0.839003in}{1.529612in}}%
\pgfpathcurveto{\pgfqpoint{0.833179in}{1.523788in}}{\pgfqpoint{0.829907in}{1.515888in}}{\pgfqpoint{0.829907in}{1.507651in}}%
\pgfpathcurveto{\pgfqpoint{0.829907in}{1.499415in}}{\pgfqpoint{0.833179in}{1.491515in}}{\pgfqpoint{0.839003in}{1.485691in}}%
\pgfpathcurveto{\pgfqpoint{0.844827in}{1.479867in}}{\pgfqpoint{0.852727in}{1.476595in}}{\pgfqpoint{0.860963in}{1.476595in}}%
\pgfpathclose%
\pgfusepath{stroke,fill}%
\end{pgfscope}%
\begin{pgfscope}%
\pgfpathrectangle{\pgfqpoint{0.457963in}{0.528059in}}{\pgfqpoint{6.200000in}{2.285714in}} %
\pgfusepath{clip}%
\pgfsetbuttcap%
\pgfsetroundjoin%
\definecolor{currentfill}{rgb}{1.000000,0.500000,0.500000}%
\pgfsetfillcolor{currentfill}%
\pgfsetlinewidth{1.003750pt}%
\definecolor{currentstroke}{rgb}{1.000000,0.500000,0.500000}%
\pgfsetstrokecolor{currentstroke}%
\pgfsetdash{}{0pt}%
\pgfpathmoveto{\pgfqpoint{0.933297in}{1.084758in}}%
\pgfpathcurveto{\pgfqpoint{0.941533in}{1.084758in}}{\pgfqpoint{0.949433in}{1.088030in}}{\pgfqpoint{0.955257in}{1.093854in}}%
\pgfpathcurveto{\pgfqpoint{0.961081in}{1.099678in}}{\pgfqpoint{0.964353in}{1.107578in}}{\pgfqpoint{0.964353in}{1.115815in}}%
\pgfpathcurveto{\pgfqpoint{0.964353in}{1.124051in}}{\pgfqpoint{0.961081in}{1.131951in}}{\pgfqpoint{0.955257in}{1.137775in}}%
\pgfpathcurveto{\pgfqpoint{0.949433in}{1.143599in}}{\pgfqpoint{0.941533in}{1.146871in}}{\pgfqpoint{0.933297in}{1.146871in}}%
\pgfpathcurveto{\pgfqpoint{0.925060in}{1.146871in}}{\pgfqpoint{0.917160in}{1.143599in}}{\pgfqpoint{0.911336in}{1.137775in}}%
\pgfpathcurveto{\pgfqpoint{0.905512in}{1.131951in}}{\pgfqpoint{0.902240in}{1.124051in}}{\pgfqpoint{0.902240in}{1.115815in}}%
\pgfpathcurveto{\pgfqpoint{0.902240in}{1.107578in}}{\pgfqpoint{0.905512in}{1.099678in}}{\pgfqpoint{0.911336in}{1.093854in}}%
\pgfpathcurveto{\pgfqpoint{0.917160in}{1.088030in}}{\pgfqpoint{0.925060in}{1.084758in}}{\pgfqpoint{0.933297in}{1.084758in}}%
\pgfpathclose%
\pgfusepath{stroke,fill}%
\end{pgfscope}%
\begin{pgfscope}%
\pgfpathrectangle{\pgfqpoint{0.457963in}{0.528059in}}{\pgfqpoint{6.200000in}{2.285714in}} %
\pgfusepath{clip}%
\pgfsetbuttcap%
\pgfsetroundjoin%
\definecolor{currentfill}{rgb}{1.000000,0.500000,0.500000}%
\pgfsetfillcolor{currentfill}%
\pgfsetlinewidth{1.003750pt}%
\definecolor{currentstroke}{rgb}{1.000000,0.500000,0.500000}%
\pgfsetstrokecolor{currentstroke}%
\pgfsetdash{}{0pt}%
\pgfpathmoveto{\pgfqpoint{1.470630in}{1.476595in}}%
\pgfpathcurveto{\pgfqpoint{1.478866in}{1.476595in}}{\pgfqpoint{1.486766in}{1.479867in}}{\pgfqpoint{1.492590in}{1.485691in}}%
\pgfpathcurveto{\pgfqpoint{1.498414in}{1.491515in}}{\pgfqpoint{1.501686in}{1.499415in}}{\pgfqpoint{1.501686in}{1.507651in}}%
\pgfpathcurveto{\pgfqpoint{1.501686in}{1.515888in}}{\pgfqpoint{1.498414in}{1.523788in}}{\pgfqpoint{1.492590in}{1.529612in}}%
\pgfpathcurveto{\pgfqpoint{1.486766in}{1.535435in}}{\pgfqpoint{1.478866in}{1.538708in}}{\pgfqpoint{1.470630in}{1.538708in}}%
\pgfpathcurveto{\pgfqpoint{1.462394in}{1.538708in}}{\pgfqpoint{1.454494in}{1.535435in}}{\pgfqpoint{1.448670in}{1.529612in}}%
\pgfpathcurveto{\pgfqpoint{1.442846in}{1.523788in}}{\pgfqpoint{1.439574in}{1.515888in}}{\pgfqpoint{1.439574in}{1.507651in}}%
\pgfpathcurveto{\pgfqpoint{1.439574in}{1.499415in}}{\pgfqpoint{1.442846in}{1.491515in}}{\pgfqpoint{1.448670in}{1.485691in}}%
\pgfpathcurveto{\pgfqpoint{1.454494in}{1.479867in}}{\pgfqpoint{1.462394in}{1.476595in}}{\pgfqpoint{1.470630in}{1.476595in}}%
\pgfpathclose%
\pgfusepath{stroke,fill}%
\end{pgfscope}%
\begin{pgfscope}%
\pgfpathrectangle{\pgfqpoint{0.457963in}{0.528059in}}{\pgfqpoint{6.200000in}{2.285714in}} %
\pgfusepath{clip}%
\pgfsetbuttcap%
\pgfsetroundjoin%
\definecolor{currentfill}{rgb}{1.000000,0.500000,0.500000}%
\pgfsetfillcolor{currentfill}%
\pgfsetlinewidth{1.003750pt}%
\definecolor{currentstroke}{rgb}{1.000000,0.500000,0.500000}%
\pgfsetstrokecolor{currentstroke}%
\pgfsetdash{}{0pt}%
\pgfpathmoveto{\pgfqpoint{1.480963in}{1.424350in}}%
\pgfpathcurveto{\pgfqpoint{1.489200in}{1.424350in}}{\pgfqpoint{1.497100in}{1.427622in}}{\pgfqpoint{1.502924in}{1.433446in}}%
\pgfpathcurveto{\pgfqpoint{1.508748in}{1.439270in}}{\pgfqpoint{1.512020in}{1.447170in}}{\pgfqpoint{1.512020in}{1.455406in}}%
\pgfpathcurveto{\pgfqpoint{1.512020in}{1.463643in}}{\pgfqpoint{1.508748in}{1.471543in}}{\pgfqpoint{1.502924in}{1.477367in}}%
\pgfpathcurveto{\pgfqpoint{1.497100in}{1.483191in}}{\pgfqpoint{1.489200in}{1.486463in}}{\pgfqpoint{1.480963in}{1.486463in}}%
\pgfpathcurveto{\pgfqpoint{1.472727in}{1.486463in}}{\pgfqpoint{1.464827in}{1.483191in}}{\pgfqpoint{1.459003in}{1.477367in}}%
\pgfpathcurveto{\pgfqpoint{1.453179in}{1.471543in}}{\pgfqpoint{1.449907in}{1.463643in}}{\pgfqpoint{1.449907in}{1.455406in}}%
\pgfpathcurveto{\pgfqpoint{1.449907in}{1.447170in}}{\pgfqpoint{1.453179in}{1.439270in}}{\pgfqpoint{1.459003in}{1.433446in}}%
\pgfpathcurveto{\pgfqpoint{1.464827in}{1.427622in}}{\pgfqpoint{1.472727in}{1.424350in}}{\pgfqpoint{1.480963in}{1.424350in}}%
\pgfpathclose%
\pgfusepath{stroke,fill}%
\end{pgfscope}%
\begin{pgfscope}%
\pgfpathrectangle{\pgfqpoint{0.457963in}{0.528059in}}{\pgfqpoint{6.200000in}{2.285714in}} %
\pgfusepath{clip}%
\pgfsetbuttcap%
\pgfsetroundjoin%
\definecolor{currentfill}{rgb}{1.000000,0.500000,0.500000}%
\pgfsetfillcolor{currentfill}%
\pgfsetlinewidth{1.003750pt}%
\definecolor{currentstroke}{rgb}{1.000000,0.500000,0.500000}%
\pgfsetstrokecolor{currentstroke}%
\pgfsetdash{}{0pt}%
\pgfpathmoveto{\pgfqpoint{1.511963in}{0.692921in}}%
\pgfpathcurveto{\pgfqpoint{1.520200in}{0.692921in}}{\pgfqpoint{1.528100in}{0.696194in}}{\pgfqpoint{1.533924in}{0.702018in}}%
\pgfpathcurveto{\pgfqpoint{1.539748in}{0.707841in}}{\pgfqpoint{1.543020in}{0.715742in}}{\pgfqpoint{1.543020in}{0.723978in}}%
\pgfpathcurveto{\pgfqpoint{1.543020in}{0.732214in}}{\pgfqpoint{1.539748in}{0.740114in}}{\pgfqpoint{1.533924in}{0.745938in}}%
\pgfpathcurveto{\pgfqpoint{1.528100in}{0.751762in}}{\pgfqpoint{1.520200in}{0.755034in}}{\pgfqpoint{1.511963in}{0.755034in}}%
\pgfpathcurveto{\pgfqpoint{1.503727in}{0.755034in}}{\pgfqpoint{1.495827in}{0.751762in}}{\pgfqpoint{1.490003in}{0.745938in}}%
\pgfpathcurveto{\pgfqpoint{1.484179in}{0.740114in}}{\pgfqpoint{1.480907in}{0.732214in}}{\pgfqpoint{1.480907in}{0.723978in}}%
\pgfpathcurveto{\pgfqpoint{1.480907in}{0.715742in}}{\pgfqpoint{1.484179in}{0.707841in}}{\pgfqpoint{1.490003in}{0.702018in}}%
\pgfpathcurveto{\pgfqpoint{1.495827in}{0.696194in}}{\pgfqpoint{1.503727in}{0.692921in}}{\pgfqpoint{1.511963in}{0.692921in}}%
\pgfpathclose%
\pgfusepath{stroke,fill}%
\end{pgfscope}%
\begin{pgfscope}%
\pgfpathrectangle{\pgfqpoint{0.457963in}{0.528059in}}{\pgfqpoint{6.200000in}{2.285714in}} %
\pgfusepath{clip}%
\pgfsetbuttcap%
\pgfsetroundjoin%
\definecolor{currentfill}{rgb}{1.000000,0.500000,0.500000}%
\pgfsetfillcolor{currentfill}%
\pgfsetlinewidth{1.003750pt}%
\definecolor{currentstroke}{rgb}{1.000000,0.500000,0.500000}%
\pgfsetstrokecolor{currentstroke}%
\pgfsetdash{}{0pt}%
\pgfpathmoveto{\pgfqpoint{1.666963in}{1.424350in}}%
\pgfpathcurveto{\pgfqpoint{1.675200in}{1.424350in}}{\pgfqpoint{1.683100in}{1.427622in}}{\pgfqpoint{1.688924in}{1.433446in}}%
\pgfpathcurveto{\pgfqpoint{1.694748in}{1.439270in}}{\pgfqpoint{1.698020in}{1.447170in}}{\pgfqpoint{1.698020in}{1.455406in}}%
\pgfpathcurveto{\pgfqpoint{1.698020in}{1.463643in}}{\pgfqpoint{1.694748in}{1.471543in}}{\pgfqpoint{1.688924in}{1.477367in}}%
\pgfpathcurveto{\pgfqpoint{1.683100in}{1.483191in}}{\pgfqpoint{1.675200in}{1.486463in}}{\pgfqpoint{1.666963in}{1.486463in}}%
\pgfpathcurveto{\pgfqpoint{1.658727in}{1.486463in}}{\pgfqpoint{1.650827in}{1.483191in}}{\pgfqpoint{1.645003in}{1.477367in}}%
\pgfpathcurveto{\pgfqpoint{1.639179in}{1.471543in}}{\pgfqpoint{1.635907in}{1.463643in}}{\pgfqpoint{1.635907in}{1.455406in}}%
\pgfpathcurveto{\pgfqpoint{1.635907in}{1.447170in}}{\pgfqpoint{1.639179in}{1.439270in}}{\pgfqpoint{1.645003in}{1.433446in}}%
\pgfpathcurveto{\pgfqpoint{1.650827in}{1.427622in}}{\pgfqpoint{1.658727in}{1.424350in}}{\pgfqpoint{1.666963in}{1.424350in}}%
\pgfpathclose%
\pgfusepath{stroke,fill}%
\end{pgfscope}%
\begin{pgfscope}%
\pgfpathrectangle{\pgfqpoint{0.457963in}{0.528059in}}{\pgfqpoint{6.200000in}{2.285714in}} %
\pgfusepath{clip}%
\pgfsetbuttcap%
\pgfsetroundjoin%
\definecolor{currentfill}{rgb}{1.000000,0.500000,0.500000}%
\pgfsetfillcolor{currentfill}%
\pgfsetlinewidth{1.003750pt}%
\definecolor{currentstroke}{rgb}{1.000000,0.500000,0.500000}%
\pgfsetstrokecolor{currentstroke}%
\pgfsetdash{}{0pt}%
\pgfpathmoveto{\pgfqpoint{1.790963in}{1.437411in}}%
\pgfpathcurveto{\pgfqpoint{1.799200in}{1.437411in}}{\pgfqpoint{1.807100in}{1.440683in}}{\pgfqpoint{1.812924in}{1.446507in}}%
\pgfpathcurveto{\pgfqpoint{1.818748in}{1.452331in}}{\pgfqpoint{1.822020in}{1.460231in}}{\pgfqpoint{1.822020in}{1.468468in}}%
\pgfpathcurveto{\pgfqpoint{1.822020in}{1.476704in}}{\pgfqpoint{1.818748in}{1.484604in}}{\pgfqpoint{1.812924in}{1.490428in}}%
\pgfpathcurveto{\pgfqpoint{1.807100in}{1.496252in}}{\pgfqpoint{1.799200in}{1.499524in}}{\pgfqpoint{1.790963in}{1.499524in}}%
\pgfpathcurveto{\pgfqpoint{1.782727in}{1.499524in}}{\pgfqpoint{1.774827in}{1.496252in}}{\pgfqpoint{1.769003in}{1.490428in}}%
\pgfpathcurveto{\pgfqpoint{1.763179in}{1.484604in}}{\pgfqpoint{1.759907in}{1.476704in}}{\pgfqpoint{1.759907in}{1.468468in}}%
\pgfpathcurveto{\pgfqpoint{1.759907in}{1.460231in}}{\pgfqpoint{1.763179in}{1.452331in}}{\pgfqpoint{1.769003in}{1.446507in}}%
\pgfpathcurveto{\pgfqpoint{1.774827in}{1.440683in}}{\pgfqpoint{1.782727in}{1.437411in}}{\pgfqpoint{1.790963in}{1.437411in}}%
\pgfpathclose%
\pgfusepath{stroke,fill}%
\end{pgfscope}%
\begin{pgfscope}%
\pgfpathrectangle{\pgfqpoint{0.457963in}{0.528059in}}{\pgfqpoint{6.200000in}{2.285714in}} %
\pgfusepath{clip}%
\pgfsetbuttcap%
\pgfsetroundjoin%
\definecolor{currentfill}{rgb}{1.000000,0.500000,0.500000}%
\pgfsetfillcolor{currentfill}%
\pgfsetlinewidth{1.003750pt}%
\definecolor{currentstroke}{rgb}{1.000000,0.500000,0.500000}%
\pgfsetstrokecolor{currentstroke}%
\pgfsetdash{}{0pt}%
\pgfpathmoveto{\pgfqpoint{2.080297in}{1.398227in}}%
\pgfpathcurveto{\pgfqpoint{2.088533in}{1.398227in}}{\pgfqpoint{2.096433in}{1.401500in}}{\pgfqpoint{2.102257in}{1.407324in}}%
\pgfpathcurveto{\pgfqpoint{2.108081in}{1.413148in}}{\pgfqpoint{2.111353in}{1.421048in}}{\pgfqpoint{2.111353in}{1.429284in}}%
\pgfpathcurveto{\pgfqpoint{2.111353in}{1.437520in}}{\pgfqpoint{2.108081in}{1.445420in}}{\pgfqpoint{2.102257in}{1.451244in}}%
\pgfpathcurveto{\pgfqpoint{2.096433in}{1.457068in}}{\pgfqpoint{2.088533in}{1.460340in}}{\pgfqpoint{2.080297in}{1.460340in}}%
\pgfpathcurveto{\pgfqpoint{2.072060in}{1.460340in}}{\pgfqpoint{2.064160in}{1.457068in}}{\pgfqpoint{2.058336in}{1.451244in}}%
\pgfpathcurveto{\pgfqpoint{2.052512in}{1.445420in}}{\pgfqpoint{2.049240in}{1.437520in}}{\pgfqpoint{2.049240in}{1.429284in}}%
\pgfpathcurveto{\pgfqpoint{2.049240in}{1.421048in}}{\pgfqpoint{2.052512in}{1.413148in}}{\pgfqpoint{2.058336in}{1.407324in}}%
\pgfpathcurveto{\pgfqpoint{2.064160in}{1.401500in}}{\pgfqpoint{2.072060in}{1.398227in}}{\pgfqpoint{2.080297in}{1.398227in}}%
\pgfpathclose%
\pgfusepath{stroke,fill}%
\end{pgfscope}%
\begin{pgfscope}%
\pgfpathrectangle{\pgfqpoint{0.457963in}{0.528059in}}{\pgfqpoint{6.200000in}{2.285714in}} %
\pgfusepath{clip}%
\pgfsetbuttcap%
\pgfsetroundjoin%
\definecolor{currentfill}{rgb}{1.000000,0.500000,0.500000}%
\pgfsetfillcolor{currentfill}%
\pgfsetlinewidth{1.003750pt}%
\definecolor{currentstroke}{rgb}{1.000000,0.500000,0.500000}%
\pgfsetstrokecolor{currentstroke}%
\pgfsetdash{}{0pt}%
\pgfpathmoveto{\pgfqpoint{3.051630in}{1.097819in}}%
\pgfpathcurveto{\pgfqpoint{3.059866in}{1.097819in}}{\pgfqpoint{3.067766in}{1.101092in}}{\pgfqpoint{3.073590in}{1.106916in}}%
\pgfpathcurveto{\pgfqpoint{3.079414in}{1.112739in}}{\pgfqpoint{3.082686in}{1.120639in}}{\pgfqpoint{3.082686in}{1.128876in}}%
\pgfpathcurveto{\pgfqpoint{3.082686in}{1.137112in}}{\pgfqpoint{3.079414in}{1.145012in}}{\pgfqpoint{3.073590in}{1.150836in}}%
\pgfpathcurveto{\pgfqpoint{3.067766in}{1.156660in}}{\pgfqpoint{3.059866in}{1.159932in}}{\pgfqpoint{3.051630in}{1.159932in}}%
\pgfpathcurveto{\pgfqpoint{3.043394in}{1.159932in}}{\pgfqpoint{3.035494in}{1.156660in}}{\pgfqpoint{3.029670in}{1.150836in}}%
\pgfpathcurveto{\pgfqpoint{3.023846in}{1.145012in}}{\pgfqpoint{3.020574in}{1.137112in}}{\pgfqpoint{3.020574in}{1.128876in}}%
\pgfpathcurveto{\pgfqpoint{3.020574in}{1.120639in}}{\pgfqpoint{3.023846in}{1.112739in}}{\pgfqpoint{3.029670in}{1.106916in}}%
\pgfpathcurveto{\pgfqpoint{3.035494in}{1.101092in}}{\pgfqpoint{3.043394in}{1.097819in}}{\pgfqpoint{3.051630in}{1.097819in}}%
\pgfpathclose%
\pgfusepath{stroke,fill}%
\end{pgfscope}%
\begin{pgfscope}%
\pgfpathrectangle{\pgfqpoint{0.457963in}{0.528059in}}{\pgfqpoint{6.200000in}{2.285714in}} %
\pgfusepath{clip}%
\pgfsetbuttcap%
\pgfsetroundjoin%
\definecolor{currentfill}{rgb}{1.000000,0.500000,0.500000}%
\pgfsetfillcolor{currentfill}%
\pgfsetlinewidth{1.003750pt}%
\definecolor{currentstroke}{rgb}{1.000000,0.500000,0.500000}%
\pgfsetstrokecolor{currentstroke}%
\pgfsetdash{}{0pt}%
\pgfpathmoveto{\pgfqpoint{3.371963in}{0.980268in}}%
\pgfpathcurveto{\pgfqpoint{3.380200in}{0.980268in}}{\pgfqpoint{3.388100in}{0.983541in}}{\pgfqpoint{3.393924in}{0.989364in}}%
\pgfpathcurveto{\pgfqpoint{3.399748in}{0.995188in}}{\pgfqpoint{3.403020in}{1.003088in}}{\pgfqpoint{3.403020in}{1.011325in}}%
\pgfpathcurveto{\pgfqpoint{3.403020in}{1.019561in}}{\pgfqpoint{3.399748in}{1.027461in}}{\pgfqpoint{3.393924in}{1.033285in}}%
\pgfpathcurveto{\pgfqpoint{3.388100in}{1.039109in}}{\pgfqpoint{3.380200in}{1.042381in}}{\pgfqpoint{3.371963in}{1.042381in}}%
\pgfpathcurveto{\pgfqpoint{3.363727in}{1.042381in}}{\pgfqpoint{3.355827in}{1.039109in}}{\pgfqpoint{3.350003in}{1.033285in}}%
\pgfpathcurveto{\pgfqpoint{3.344179in}{1.027461in}}{\pgfqpoint{3.340907in}{1.019561in}}{\pgfqpoint{3.340907in}{1.011325in}}%
\pgfpathcurveto{\pgfqpoint{3.340907in}{1.003088in}}{\pgfqpoint{3.344179in}{0.995188in}}{\pgfqpoint{3.350003in}{0.989364in}}%
\pgfpathcurveto{\pgfqpoint{3.355827in}{0.983541in}}{\pgfqpoint{3.363727in}{0.980268in}}{\pgfqpoint{3.371963in}{0.980268in}}%
\pgfpathclose%
\pgfusepath{stroke,fill}%
\end{pgfscope}%
\begin{pgfscope}%
\pgfpathrectangle{\pgfqpoint{0.457963in}{0.528059in}}{\pgfqpoint{6.200000in}{2.285714in}} %
\pgfusepath{clip}%
\pgfsetbuttcap%
\pgfsetroundjoin%
\definecolor{currentfill}{rgb}{1.000000,0.333333,0.333333}%
\pgfsetfillcolor{currentfill}%
\pgfsetlinewidth{1.003750pt}%
\definecolor{currentstroke}{rgb}{1.000000,0.333333,0.333333}%
\pgfsetstrokecolor{currentstroke}%
\pgfsetdash{}{0pt}%
\pgfpathmoveto{\pgfqpoint{0.457963in}{1.803125in}}%
\pgfpathcurveto{\pgfqpoint{0.466200in}{1.803125in}}{\pgfqpoint{0.474100in}{1.806398in}}{\pgfqpoint{0.479924in}{1.812222in}}%
\pgfpathcurveto{\pgfqpoint{0.485748in}{1.818046in}}{\pgfqpoint{0.489020in}{1.825946in}}{\pgfqpoint{0.489020in}{1.834182in}}%
\pgfpathcurveto{\pgfqpoint{0.489020in}{1.842418in}}{\pgfqpoint{0.485748in}{1.850318in}}{\pgfqpoint{0.479924in}{1.856142in}}%
\pgfpathcurveto{\pgfqpoint{0.474100in}{1.861966in}}{\pgfqpoint{0.466200in}{1.865238in}}{\pgfqpoint{0.457963in}{1.865238in}}%
\pgfpathcurveto{\pgfqpoint{0.449727in}{1.865238in}}{\pgfqpoint{0.441827in}{1.861966in}}{\pgfqpoint{0.436003in}{1.856142in}}%
\pgfpathcurveto{\pgfqpoint{0.430179in}{1.850318in}}{\pgfqpoint{0.426907in}{1.842418in}}{\pgfqpoint{0.426907in}{1.834182in}}%
\pgfpathcurveto{\pgfqpoint{0.426907in}{1.825946in}}{\pgfqpoint{0.430179in}{1.818046in}}{\pgfqpoint{0.436003in}{1.812222in}}%
\pgfpathcurveto{\pgfqpoint{0.441827in}{1.806398in}}{\pgfqpoint{0.449727in}{1.803125in}}{\pgfqpoint{0.457963in}{1.803125in}}%
\pgfpathclose%
\pgfusepath{stroke,fill}%
\end{pgfscope}%
\begin{pgfscope}%
\pgfpathrectangle{\pgfqpoint{0.457963in}{0.528059in}}{\pgfqpoint{6.200000in}{2.285714in}} %
\pgfusepath{clip}%
\pgfsetbuttcap%
\pgfsetroundjoin%
\definecolor{currentfill}{rgb}{1.000000,0.333333,0.333333}%
\pgfsetfillcolor{currentfill}%
\pgfsetlinewidth{1.003750pt}%
\definecolor{currentstroke}{rgb}{1.000000,0.333333,0.333333}%
\pgfsetstrokecolor{currentstroke}%
\pgfsetdash{}{0pt}%
\pgfpathmoveto{\pgfqpoint{0.457963in}{1.803125in}}%
\pgfpathcurveto{\pgfqpoint{0.466200in}{1.803125in}}{\pgfqpoint{0.474100in}{1.806398in}}{\pgfqpoint{0.479924in}{1.812222in}}%
\pgfpathcurveto{\pgfqpoint{0.485748in}{1.818046in}}{\pgfqpoint{0.489020in}{1.825946in}}{\pgfqpoint{0.489020in}{1.834182in}}%
\pgfpathcurveto{\pgfqpoint{0.489020in}{1.842418in}}{\pgfqpoint{0.485748in}{1.850318in}}{\pgfqpoint{0.479924in}{1.856142in}}%
\pgfpathcurveto{\pgfqpoint{0.474100in}{1.861966in}}{\pgfqpoint{0.466200in}{1.865238in}}{\pgfqpoint{0.457963in}{1.865238in}}%
\pgfpathcurveto{\pgfqpoint{0.449727in}{1.865238in}}{\pgfqpoint{0.441827in}{1.861966in}}{\pgfqpoint{0.436003in}{1.856142in}}%
\pgfpathcurveto{\pgfqpoint{0.430179in}{1.850318in}}{\pgfqpoint{0.426907in}{1.842418in}}{\pgfqpoint{0.426907in}{1.834182in}}%
\pgfpathcurveto{\pgfqpoint{0.426907in}{1.825946in}}{\pgfqpoint{0.430179in}{1.818046in}}{\pgfqpoint{0.436003in}{1.812222in}}%
\pgfpathcurveto{\pgfqpoint{0.441827in}{1.806398in}}{\pgfqpoint{0.449727in}{1.803125in}}{\pgfqpoint{0.457963in}{1.803125in}}%
\pgfpathclose%
\pgfusepath{stroke,fill}%
\end{pgfscope}%
\begin{pgfscope}%
\pgfpathrectangle{\pgfqpoint{0.457963in}{0.528059in}}{\pgfqpoint{6.200000in}{2.285714in}} %
\pgfusepath{clip}%
\pgfsetbuttcap%
\pgfsetroundjoin%
\definecolor{currentfill}{rgb}{1.000000,0.333333,0.333333}%
\pgfsetfillcolor{currentfill}%
\pgfsetlinewidth{1.003750pt}%
\definecolor{currentstroke}{rgb}{1.000000,0.333333,0.333333}%
\pgfsetstrokecolor{currentstroke}%
\pgfsetdash{}{0pt}%
\pgfpathmoveto{\pgfqpoint{0.457963in}{1.803125in}}%
\pgfpathcurveto{\pgfqpoint{0.466200in}{1.803125in}}{\pgfqpoint{0.474100in}{1.806398in}}{\pgfqpoint{0.479924in}{1.812222in}}%
\pgfpathcurveto{\pgfqpoint{0.485748in}{1.818046in}}{\pgfqpoint{0.489020in}{1.825946in}}{\pgfqpoint{0.489020in}{1.834182in}}%
\pgfpathcurveto{\pgfqpoint{0.489020in}{1.842418in}}{\pgfqpoint{0.485748in}{1.850318in}}{\pgfqpoint{0.479924in}{1.856142in}}%
\pgfpathcurveto{\pgfqpoint{0.474100in}{1.861966in}}{\pgfqpoint{0.466200in}{1.865238in}}{\pgfqpoint{0.457963in}{1.865238in}}%
\pgfpathcurveto{\pgfqpoint{0.449727in}{1.865238in}}{\pgfqpoint{0.441827in}{1.861966in}}{\pgfqpoint{0.436003in}{1.856142in}}%
\pgfpathcurveto{\pgfqpoint{0.430179in}{1.850318in}}{\pgfqpoint{0.426907in}{1.842418in}}{\pgfqpoint{0.426907in}{1.834182in}}%
\pgfpathcurveto{\pgfqpoint{0.426907in}{1.825946in}}{\pgfqpoint{0.430179in}{1.818046in}}{\pgfqpoint{0.436003in}{1.812222in}}%
\pgfpathcurveto{\pgfqpoint{0.441827in}{1.806398in}}{\pgfqpoint{0.449727in}{1.803125in}}{\pgfqpoint{0.457963in}{1.803125in}}%
\pgfpathclose%
\pgfusepath{stroke,fill}%
\end{pgfscope}%
\begin{pgfscope}%
\pgfpathrectangle{\pgfqpoint{0.457963in}{0.528059in}}{\pgfqpoint{6.200000in}{2.285714in}} %
\pgfusepath{clip}%
\pgfsetbuttcap%
\pgfsetroundjoin%
\definecolor{currentfill}{rgb}{1.000000,0.333333,0.333333}%
\pgfsetfillcolor{currentfill}%
\pgfsetlinewidth{1.003750pt}%
\definecolor{currentstroke}{rgb}{1.000000,0.333333,0.333333}%
\pgfsetstrokecolor{currentstroke}%
\pgfsetdash{}{0pt}%
\pgfpathmoveto{\pgfqpoint{0.457963in}{1.803125in}}%
\pgfpathcurveto{\pgfqpoint{0.466200in}{1.803125in}}{\pgfqpoint{0.474100in}{1.806398in}}{\pgfqpoint{0.479924in}{1.812222in}}%
\pgfpathcurveto{\pgfqpoint{0.485748in}{1.818046in}}{\pgfqpoint{0.489020in}{1.825946in}}{\pgfqpoint{0.489020in}{1.834182in}}%
\pgfpathcurveto{\pgfqpoint{0.489020in}{1.842418in}}{\pgfqpoint{0.485748in}{1.850318in}}{\pgfqpoint{0.479924in}{1.856142in}}%
\pgfpathcurveto{\pgfqpoint{0.474100in}{1.861966in}}{\pgfqpoint{0.466200in}{1.865238in}}{\pgfqpoint{0.457963in}{1.865238in}}%
\pgfpathcurveto{\pgfqpoint{0.449727in}{1.865238in}}{\pgfqpoint{0.441827in}{1.861966in}}{\pgfqpoint{0.436003in}{1.856142in}}%
\pgfpathcurveto{\pgfqpoint{0.430179in}{1.850318in}}{\pgfqpoint{0.426907in}{1.842418in}}{\pgfqpoint{0.426907in}{1.834182in}}%
\pgfpathcurveto{\pgfqpoint{0.426907in}{1.825946in}}{\pgfqpoint{0.430179in}{1.818046in}}{\pgfqpoint{0.436003in}{1.812222in}}%
\pgfpathcurveto{\pgfqpoint{0.441827in}{1.806398in}}{\pgfqpoint{0.449727in}{1.803125in}}{\pgfqpoint{0.457963in}{1.803125in}}%
\pgfpathclose%
\pgfusepath{stroke,fill}%
\end{pgfscope}%
\begin{pgfscope}%
\pgfpathrectangle{\pgfqpoint{0.457963in}{0.528059in}}{\pgfqpoint{6.200000in}{2.285714in}} %
\pgfusepath{clip}%
\pgfsetbuttcap%
\pgfsetroundjoin%
\definecolor{currentfill}{rgb}{1.000000,0.333333,0.333333}%
\pgfsetfillcolor{currentfill}%
\pgfsetlinewidth{1.003750pt}%
\definecolor{currentstroke}{rgb}{1.000000,0.333333,0.333333}%
\pgfsetstrokecolor{currentstroke}%
\pgfsetdash{}{0pt}%
\pgfpathmoveto{\pgfqpoint{0.457963in}{1.803125in}}%
\pgfpathcurveto{\pgfqpoint{0.466200in}{1.803125in}}{\pgfqpoint{0.474100in}{1.806398in}}{\pgfqpoint{0.479924in}{1.812222in}}%
\pgfpathcurveto{\pgfqpoint{0.485748in}{1.818046in}}{\pgfqpoint{0.489020in}{1.825946in}}{\pgfqpoint{0.489020in}{1.834182in}}%
\pgfpathcurveto{\pgfqpoint{0.489020in}{1.842418in}}{\pgfqpoint{0.485748in}{1.850318in}}{\pgfqpoint{0.479924in}{1.856142in}}%
\pgfpathcurveto{\pgfqpoint{0.474100in}{1.861966in}}{\pgfqpoint{0.466200in}{1.865238in}}{\pgfqpoint{0.457963in}{1.865238in}}%
\pgfpathcurveto{\pgfqpoint{0.449727in}{1.865238in}}{\pgfqpoint{0.441827in}{1.861966in}}{\pgfqpoint{0.436003in}{1.856142in}}%
\pgfpathcurveto{\pgfqpoint{0.430179in}{1.850318in}}{\pgfqpoint{0.426907in}{1.842418in}}{\pgfqpoint{0.426907in}{1.834182in}}%
\pgfpathcurveto{\pgfqpoint{0.426907in}{1.825946in}}{\pgfqpoint{0.430179in}{1.818046in}}{\pgfqpoint{0.436003in}{1.812222in}}%
\pgfpathcurveto{\pgfqpoint{0.441827in}{1.806398in}}{\pgfqpoint{0.449727in}{1.803125in}}{\pgfqpoint{0.457963in}{1.803125in}}%
\pgfpathclose%
\pgfusepath{stroke,fill}%
\end{pgfscope}%
\begin{pgfscope}%
\pgfpathrectangle{\pgfqpoint{0.457963in}{0.528059in}}{\pgfqpoint{6.200000in}{2.285714in}} %
\pgfusepath{clip}%
\pgfsetbuttcap%
\pgfsetroundjoin%
\definecolor{currentfill}{rgb}{1.000000,0.333333,0.333333}%
\pgfsetfillcolor{currentfill}%
\pgfsetlinewidth{1.003750pt}%
\definecolor{currentstroke}{rgb}{1.000000,0.333333,0.333333}%
\pgfsetstrokecolor{currentstroke}%
\pgfsetdash{}{0pt}%
\pgfpathmoveto{\pgfqpoint{0.478630in}{1.750880in}}%
\pgfpathcurveto{\pgfqpoint{0.486866in}{1.750880in}}{\pgfqpoint{0.494766in}{1.754153in}}{\pgfqpoint{0.500590in}{1.759977in}}%
\pgfpathcurveto{\pgfqpoint{0.506414in}{1.765801in}}{\pgfqpoint{0.509686in}{1.773701in}}{\pgfqpoint{0.509686in}{1.781937in}}%
\pgfpathcurveto{\pgfqpoint{0.509686in}{1.790173in}}{\pgfqpoint{0.506414in}{1.798073in}}{\pgfqpoint{0.500590in}{1.803897in}}%
\pgfpathcurveto{\pgfqpoint{0.494766in}{1.809721in}}{\pgfqpoint{0.486866in}{1.812993in}}{\pgfqpoint{0.478630in}{1.812993in}}%
\pgfpathcurveto{\pgfqpoint{0.470394in}{1.812993in}}{\pgfqpoint{0.462494in}{1.809721in}}{\pgfqpoint{0.456670in}{1.803897in}}%
\pgfpathcurveto{\pgfqpoint{0.450846in}{1.798073in}}{\pgfqpoint{0.447574in}{1.790173in}}{\pgfqpoint{0.447574in}{1.781937in}}%
\pgfpathcurveto{\pgfqpoint{0.447574in}{1.773701in}}{\pgfqpoint{0.450846in}{1.765801in}}{\pgfqpoint{0.456670in}{1.759977in}}%
\pgfpathcurveto{\pgfqpoint{0.462494in}{1.754153in}}{\pgfqpoint{0.470394in}{1.750880in}}{\pgfqpoint{0.478630in}{1.750880in}}%
\pgfpathclose%
\pgfusepath{stroke,fill}%
\end{pgfscope}%
\begin{pgfscope}%
\pgfpathrectangle{\pgfqpoint{0.457963in}{0.528059in}}{\pgfqpoint{6.200000in}{2.285714in}} %
\pgfusepath{clip}%
\pgfsetbuttcap%
\pgfsetroundjoin%
\definecolor{currentfill}{rgb}{1.000000,0.333333,0.333333}%
\pgfsetfillcolor{currentfill}%
\pgfsetlinewidth{1.003750pt}%
\definecolor{currentstroke}{rgb}{1.000000,0.333333,0.333333}%
\pgfsetstrokecolor{currentstroke}%
\pgfsetdash{}{0pt}%
\pgfpathmoveto{\pgfqpoint{0.530297in}{1.750880in}}%
\pgfpathcurveto{\pgfqpoint{0.538533in}{1.750880in}}{\pgfqpoint{0.546433in}{1.754153in}}{\pgfqpoint{0.552257in}{1.759977in}}%
\pgfpathcurveto{\pgfqpoint{0.558081in}{1.765801in}}{\pgfqpoint{0.561353in}{1.773701in}}{\pgfqpoint{0.561353in}{1.781937in}}%
\pgfpathcurveto{\pgfqpoint{0.561353in}{1.790173in}}{\pgfqpoint{0.558081in}{1.798073in}}{\pgfqpoint{0.552257in}{1.803897in}}%
\pgfpathcurveto{\pgfqpoint{0.546433in}{1.809721in}}{\pgfqpoint{0.538533in}{1.812993in}}{\pgfqpoint{0.530297in}{1.812993in}}%
\pgfpathcurveto{\pgfqpoint{0.522060in}{1.812993in}}{\pgfqpoint{0.514160in}{1.809721in}}{\pgfqpoint{0.508336in}{1.803897in}}%
\pgfpathcurveto{\pgfqpoint{0.502512in}{1.798073in}}{\pgfqpoint{0.499240in}{1.790173in}}{\pgfqpoint{0.499240in}{1.781937in}}%
\pgfpathcurveto{\pgfqpoint{0.499240in}{1.773701in}}{\pgfqpoint{0.502512in}{1.765801in}}{\pgfqpoint{0.508336in}{1.759977in}}%
\pgfpathcurveto{\pgfqpoint{0.514160in}{1.754153in}}{\pgfqpoint{0.522060in}{1.750880in}}{\pgfqpoint{0.530297in}{1.750880in}}%
\pgfpathclose%
\pgfusepath{stroke,fill}%
\end{pgfscope}%
\begin{pgfscope}%
\pgfpathrectangle{\pgfqpoint{0.457963in}{0.528059in}}{\pgfqpoint{6.200000in}{2.285714in}} %
\pgfusepath{clip}%
\pgfsetbuttcap%
\pgfsetroundjoin%
\definecolor{currentfill}{rgb}{1.000000,0.333333,0.333333}%
\pgfsetfillcolor{currentfill}%
\pgfsetlinewidth{1.003750pt}%
\definecolor{currentstroke}{rgb}{1.000000,0.333333,0.333333}%
\pgfsetstrokecolor{currentstroke}%
\pgfsetdash{}{0pt}%
\pgfpathmoveto{\pgfqpoint{0.530297in}{1.803125in}}%
\pgfpathcurveto{\pgfqpoint{0.538533in}{1.803125in}}{\pgfqpoint{0.546433in}{1.806398in}}{\pgfqpoint{0.552257in}{1.812222in}}%
\pgfpathcurveto{\pgfqpoint{0.558081in}{1.818046in}}{\pgfqpoint{0.561353in}{1.825946in}}{\pgfqpoint{0.561353in}{1.834182in}}%
\pgfpathcurveto{\pgfqpoint{0.561353in}{1.842418in}}{\pgfqpoint{0.558081in}{1.850318in}}{\pgfqpoint{0.552257in}{1.856142in}}%
\pgfpathcurveto{\pgfqpoint{0.546433in}{1.861966in}}{\pgfqpoint{0.538533in}{1.865238in}}{\pgfqpoint{0.530297in}{1.865238in}}%
\pgfpathcurveto{\pgfqpoint{0.522060in}{1.865238in}}{\pgfqpoint{0.514160in}{1.861966in}}{\pgfqpoint{0.508336in}{1.856142in}}%
\pgfpathcurveto{\pgfqpoint{0.502512in}{1.850318in}}{\pgfqpoint{0.499240in}{1.842418in}}{\pgfqpoint{0.499240in}{1.834182in}}%
\pgfpathcurveto{\pgfqpoint{0.499240in}{1.825946in}}{\pgfqpoint{0.502512in}{1.818046in}}{\pgfqpoint{0.508336in}{1.812222in}}%
\pgfpathcurveto{\pgfqpoint{0.514160in}{1.806398in}}{\pgfqpoint{0.522060in}{1.803125in}}{\pgfqpoint{0.530297in}{1.803125in}}%
\pgfpathclose%
\pgfusepath{stroke,fill}%
\end{pgfscope}%
\begin{pgfscope}%
\pgfpathrectangle{\pgfqpoint{0.457963in}{0.528059in}}{\pgfqpoint{6.200000in}{2.285714in}} %
\pgfusepath{clip}%
\pgfsetbuttcap%
\pgfsetroundjoin%
\definecolor{currentfill}{rgb}{1.000000,0.333333,0.333333}%
\pgfsetfillcolor{currentfill}%
\pgfsetlinewidth{1.003750pt}%
\definecolor{currentstroke}{rgb}{1.000000,0.333333,0.333333}%
\pgfsetstrokecolor{currentstroke}%
\pgfsetdash{}{0pt}%
\pgfpathmoveto{\pgfqpoint{0.664630in}{1.489656in}}%
\pgfpathcurveto{\pgfqpoint{0.672866in}{1.489656in}}{\pgfqpoint{0.680766in}{1.492928in}}{\pgfqpoint{0.686590in}{1.498752in}}%
\pgfpathcurveto{\pgfqpoint{0.692414in}{1.504576in}}{\pgfqpoint{0.695686in}{1.512476in}}{\pgfqpoint{0.695686in}{1.520713in}}%
\pgfpathcurveto{\pgfqpoint{0.695686in}{1.528949in}}{\pgfqpoint{0.692414in}{1.536849in}}{\pgfqpoint{0.686590in}{1.542673in}}%
\pgfpathcurveto{\pgfqpoint{0.680766in}{1.548497in}}{\pgfqpoint{0.672866in}{1.551769in}}{\pgfqpoint{0.664630in}{1.551769in}}%
\pgfpathcurveto{\pgfqpoint{0.656394in}{1.551769in}}{\pgfqpoint{0.648494in}{1.548497in}}{\pgfqpoint{0.642670in}{1.542673in}}%
\pgfpathcurveto{\pgfqpoint{0.636846in}{1.536849in}}{\pgfqpoint{0.633574in}{1.528949in}}{\pgfqpoint{0.633574in}{1.520713in}}%
\pgfpathcurveto{\pgfqpoint{0.633574in}{1.512476in}}{\pgfqpoint{0.636846in}{1.504576in}}{\pgfqpoint{0.642670in}{1.498752in}}%
\pgfpathcurveto{\pgfqpoint{0.648494in}{1.492928in}}{\pgfqpoint{0.656394in}{1.489656in}}{\pgfqpoint{0.664630in}{1.489656in}}%
\pgfpathclose%
\pgfusepath{stroke,fill}%
\end{pgfscope}%
\begin{pgfscope}%
\pgfpathrectangle{\pgfqpoint{0.457963in}{0.528059in}}{\pgfqpoint{6.200000in}{2.285714in}} %
\pgfusepath{clip}%
\pgfsetbuttcap%
\pgfsetroundjoin%
\definecolor{currentfill}{rgb}{1.000000,0.333333,0.333333}%
\pgfsetfillcolor{currentfill}%
\pgfsetlinewidth{1.003750pt}%
\definecolor{currentstroke}{rgb}{1.000000,0.333333,0.333333}%
\pgfsetstrokecolor{currentstroke}%
\pgfsetdash{}{0pt}%
\pgfpathmoveto{\pgfqpoint{0.860963in}{1.803125in}}%
\pgfpathcurveto{\pgfqpoint{0.869200in}{1.803125in}}{\pgfqpoint{0.877100in}{1.806398in}}{\pgfqpoint{0.882924in}{1.812222in}}%
\pgfpathcurveto{\pgfqpoint{0.888748in}{1.818046in}}{\pgfqpoint{0.892020in}{1.825946in}}{\pgfqpoint{0.892020in}{1.834182in}}%
\pgfpathcurveto{\pgfqpoint{0.892020in}{1.842418in}}{\pgfqpoint{0.888748in}{1.850318in}}{\pgfqpoint{0.882924in}{1.856142in}}%
\pgfpathcurveto{\pgfqpoint{0.877100in}{1.861966in}}{\pgfqpoint{0.869200in}{1.865238in}}{\pgfqpoint{0.860963in}{1.865238in}}%
\pgfpathcurveto{\pgfqpoint{0.852727in}{1.865238in}}{\pgfqpoint{0.844827in}{1.861966in}}{\pgfqpoint{0.839003in}{1.856142in}}%
\pgfpathcurveto{\pgfqpoint{0.833179in}{1.850318in}}{\pgfqpoint{0.829907in}{1.842418in}}{\pgfqpoint{0.829907in}{1.834182in}}%
\pgfpathcurveto{\pgfqpoint{0.829907in}{1.825946in}}{\pgfqpoint{0.833179in}{1.818046in}}{\pgfqpoint{0.839003in}{1.812222in}}%
\pgfpathcurveto{\pgfqpoint{0.844827in}{1.806398in}}{\pgfqpoint{0.852727in}{1.803125in}}{\pgfqpoint{0.860963in}{1.803125in}}%
\pgfpathclose%
\pgfusepath{stroke,fill}%
\end{pgfscope}%
\begin{pgfscope}%
\pgfpathrectangle{\pgfqpoint{0.457963in}{0.528059in}}{\pgfqpoint{6.200000in}{2.285714in}} %
\pgfusepath{clip}%
\pgfsetbuttcap%
\pgfsetroundjoin%
\definecolor{currentfill}{rgb}{1.000000,0.333333,0.333333}%
\pgfsetfillcolor{currentfill}%
\pgfsetlinewidth{1.003750pt}%
\definecolor{currentstroke}{rgb}{1.000000,0.333333,0.333333}%
\pgfsetstrokecolor{currentstroke}%
\pgfsetdash{}{0pt}%
\pgfpathmoveto{\pgfqpoint{1.253630in}{1.777003in}}%
\pgfpathcurveto{\pgfqpoint{1.261866in}{1.777003in}}{\pgfqpoint{1.269766in}{1.780275in}}{\pgfqpoint{1.275590in}{1.786099in}}%
\pgfpathcurveto{\pgfqpoint{1.281414in}{1.791923in}}{\pgfqpoint{1.284686in}{1.799823in}}{\pgfqpoint{1.284686in}{1.808059in}}%
\pgfpathcurveto{\pgfqpoint{1.284686in}{1.816296in}}{\pgfqpoint{1.281414in}{1.824196in}}{\pgfqpoint{1.275590in}{1.830020in}}%
\pgfpathcurveto{\pgfqpoint{1.269766in}{1.835844in}}{\pgfqpoint{1.261866in}{1.839116in}}{\pgfqpoint{1.253630in}{1.839116in}}%
\pgfpathcurveto{\pgfqpoint{1.245394in}{1.839116in}}{\pgfqpoint{1.237494in}{1.835844in}}{\pgfqpoint{1.231670in}{1.830020in}}%
\pgfpathcurveto{\pgfqpoint{1.225846in}{1.824196in}}{\pgfqpoint{1.222574in}{1.816296in}}{\pgfqpoint{1.222574in}{1.808059in}}%
\pgfpathcurveto{\pgfqpoint{1.222574in}{1.799823in}}{\pgfqpoint{1.225846in}{1.791923in}}{\pgfqpoint{1.231670in}{1.786099in}}%
\pgfpathcurveto{\pgfqpoint{1.237494in}{1.780275in}}{\pgfqpoint{1.245394in}{1.777003in}}{\pgfqpoint{1.253630in}{1.777003in}}%
\pgfpathclose%
\pgfusepath{stroke,fill}%
\end{pgfscope}%
\begin{pgfscope}%
\pgfpathrectangle{\pgfqpoint{0.457963in}{0.528059in}}{\pgfqpoint{6.200000in}{2.285714in}} %
\pgfusepath{clip}%
\pgfsetbuttcap%
\pgfsetroundjoin%
\definecolor{currentfill}{rgb}{1.000000,0.333333,0.333333}%
\pgfsetfillcolor{currentfill}%
\pgfsetlinewidth{1.003750pt}%
\definecolor{currentstroke}{rgb}{1.000000,0.333333,0.333333}%
\pgfsetstrokecolor{currentstroke}%
\pgfsetdash{}{0pt}%
\pgfpathmoveto{\pgfqpoint{1.325963in}{1.123942in}}%
\pgfpathcurveto{\pgfqpoint{1.334200in}{1.123942in}}{\pgfqpoint{1.342100in}{1.127214in}}{\pgfqpoint{1.347924in}{1.133038in}}%
\pgfpathcurveto{\pgfqpoint{1.353748in}{1.138862in}}{\pgfqpoint{1.357020in}{1.146762in}}{\pgfqpoint{1.357020in}{1.154998in}}%
\pgfpathcurveto{\pgfqpoint{1.357020in}{1.163234in}}{\pgfqpoint{1.353748in}{1.171135in}}{\pgfqpoint{1.347924in}{1.176958in}}%
\pgfpathcurveto{\pgfqpoint{1.342100in}{1.182782in}}{\pgfqpoint{1.334200in}{1.186055in}}{\pgfqpoint{1.325963in}{1.186055in}}%
\pgfpathcurveto{\pgfqpoint{1.317727in}{1.186055in}}{\pgfqpoint{1.309827in}{1.182782in}}{\pgfqpoint{1.304003in}{1.176958in}}%
\pgfpathcurveto{\pgfqpoint{1.298179in}{1.171135in}}{\pgfqpoint{1.294907in}{1.163234in}}{\pgfqpoint{1.294907in}{1.154998in}}%
\pgfpathcurveto{\pgfqpoint{1.294907in}{1.146762in}}{\pgfqpoint{1.298179in}{1.138862in}}{\pgfqpoint{1.304003in}{1.133038in}}%
\pgfpathcurveto{\pgfqpoint{1.309827in}{1.127214in}}{\pgfqpoint{1.317727in}{1.123942in}}{\pgfqpoint{1.325963in}{1.123942in}}%
\pgfpathclose%
\pgfusepath{stroke,fill}%
\end{pgfscope}%
\begin{pgfscope}%
\pgfpathrectangle{\pgfqpoint{0.457963in}{0.528059in}}{\pgfqpoint{6.200000in}{2.285714in}} %
\pgfusepath{clip}%
\pgfsetbuttcap%
\pgfsetroundjoin%
\definecolor{currentfill}{rgb}{1.000000,0.333333,0.333333}%
\pgfsetfillcolor{currentfill}%
\pgfsetlinewidth{1.003750pt}%
\definecolor{currentstroke}{rgb}{1.000000,0.333333,0.333333}%
\pgfsetstrokecolor{currentstroke}%
\pgfsetdash{}{0pt}%
\pgfpathmoveto{\pgfqpoint{1.573963in}{0.914962in}}%
\pgfpathcurveto{\pgfqpoint{1.582200in}{0.914962in}}{\pgfqpoint{1.590100in}{0.918234in}}{\pgfqpoint{1.595924in}{0.924058in}}%
\pgfpathcurveto{\pgfqpoint{1.601748in}{0.929882in}}{\pgfqpoint{1.605020in}{0.937782in}}{\pgfqpoint{1.605020in}{0.946019in}}%
\pgfpathcurveto{\pgfqpoint{1.605020in}{0.954255in}}{\pgfqpoint{1.601748in}{0.962155in}}{\pgfqpoint{1.595924in}{0.967979in}}%
\pgfpathcurveto{\pgfqpoint{1.590100in}{0.973803in}}{\pgfqpoint{1.582200in}{0.977075in}}{\pgfqpoint{1.573963in}{0.977075in}}%
\pgfpathcurveto{\pgfqpoint{1.565727in}{0.977075in}}{\pgfqpoint{1.557827in}{0.973803in}}{\pgfqpoint{1.552003in}{0.967979in}}%
\pgfpathcurveto{\pgfqpoint{1.546179in}{0.962155in}}{\pgfqpoint{1.542907in}{0.954255in}}{\pgfqpoint{1.542907in}{0.946019in}}%
\pgfpathcurveto{\pgfqpoint{1.542907in}{0.937782in}}{\pgfqpoint{1.546179in}{0.929882in}}{\pgfqpoint{1.552003in}{0.924058in}}%
\pgfpathcurveto{\pgfqpoint{1.557827in}{0.918234in}}{\pgfqpoint{1.565727in}{0.914962in}}{\pgfqpoint{1.573963in}{0.914962in}}%
\pgfpathclose%
\pgfusepath{stroke,fill}%
\end{pgfscope}%
\begin{pgfscope}%
\pgfpathrectangle{\pgfqpoint{0.457963in}{0.528059in}}{\pgfqpoint{6.200000in}{2.285714in}} %
\pgfusepath{clip}%
\pgfsetbuttcap%
\pgfsetroundjoin%
\definecolor{currentfill}{rgb}{1.000000,0.333333,0.333333}%
\pgfsetfillcolor{currentfill}%
\pgfsetlinewidth{1.003750pt}%
\definecolor{currentstroke}{rgb}{1.000000,0.333333,0.333333}%
\pgfsetstrokecolor{currentstroke}%
\pgfsetdash{}{0pt}%
\pgfpathmoveto{\pgfqpoint{1.573963in}{1.803125in}}%
\pgfpathcurveto{\pgfqpoint{1.582200in}{1.803125in}}{\pgfqpoint{1.590100in}{1.806398in}}{\pgfqpoint{1.595924in}{1.812222in}}%
\pgfpathcurveto{\pgfqpoint{1.601748in}{1.818046in}}{\pgfqpoint{1.605020in}{1.825946in}}{\pgfqpoint{1.605020in}{1.834182in}}%
\pgfpathcurveto{\pgfqpoint{1.605020in}{1.842418in}}{\pgfqpoint{1.601748in}{1.850318in}}{\pgfqpoint{1.595924in}{1.856142in}}%
\pgfpathcurveto{\pgfqpoint{1.590100in}{1.861966in}}{\pgfqpoint{1.582200in}{1.865238in}}{\pgfqpoint{1.573963in}{1.865238in}}%
\pgfpathcurveto{\pgfqpoint{1.565727in}{1.865238in}}{\pgfqpoint{1.557827in}{1.861966in}}{\pgfqpoint{1.552003in}{1.856142in}}%
\pgfpathcurveto{\pgfqpoint{1.546179in}{1.850318in}}{\pgfqpoint{1.542907in}{1.842418in}}{\pgfqpoint{1.542907in}{1.834182in}}%
\pgfpathcurveto{\pgfqpoint{1.542907in}{1.825946in}}{\pgfqpoint{1.546179in}{1.818046in}}{\pgfqpoint{1.552003in}{1.812222in}}%
\pgfpathcurveto{\pgfqpoint{1.557827in}{1.806398in}}{\pgfqpoint{1.565727in}{1.803125in}}{\pgfqpoint{1.573963in}{1.803125in}}%
\pgfpathclose%
\pgfusepath{stroke,fill}%
\end{pgfscope}%
\begin{pgfscope}%
\pgfpathrectangle{\pgfqpoint{0.457963in}{0.528059in}}{\pgfqpoint{6.200000in}{2.285714in}} %
\pgfusepath{clip}%
\pgfsetbuttcap%
\pgfsetroundjoin%
\definecolor{currentfill}{rgb}{1.000000,0.333333,0.333333}%
\pgfsetfillcolor{currentfill}%
\pgfsetlinewidth{1.003750pt}%
\definecolor{currentstroke}{rgb}{1.000000,0.333333,0.333333}%
\pgfsetstrokecolor{currentstroke}%
\pgfsetdash{}{0pt}%
\pgfpathmoveto{\pgfqpoint{1.904630in}{1.763942in}}%
\pgfpathcurveto{\pgfqpoint{1.912866in}{1.763942in}}{\pgfqpoint{1.920766in}{1.767214in}}{\pgfqpoint{1.926590in}{1.773038in}}%
\pgfpathcurveto{\pgfqpoint{1.932414in}{1.778862in}}{\pgfqpoint{1.935686in}{1.786762in}}{\pgfqpoint{1.935686in}{1.794998in}}%
\pgfpathcurveto{\pgfqpoint{1.935686in}{1.803234in}}{\pgfqpoint{1.932414in}{1.811135in}}{\pgfqpoint{1.926590in}{1.816958in}}%
\pgfpathcurveto{\pgfqpoint{1.920766in}{1.822782in}}{\pgfqpoint{1.912866in}{1.826055in}}{\pgfqpoint{1.904630in}{1.826055in}}%
\pgfpathcurveto{\pgfqpoint{1.896394in}{1.826055in}}{\pgfqpoint{1.888494in}{1.822782in}}{\pgfqpoint{1.882670in}{1.816958in}}%
\pgfpathcurveto{\pgfqpoint{1.876846in}{1.811135in}}{\pgfqpoint{1.873574in}{1.803234in}}{\pgfqpoint{1.873574in}{1.794998in}}%
\pgfpathcurveto{\pgfqpoint{1.873574in}{1.786762in}}{\pgfqpoint{1.876846in}{1.778862in}}{\pgfqpoint{1.882670in}{1.773038in}}%
\pgfpathcurveto{\pgfqpoint{1.888494in}{1.767214in}}{\pgfqpoint{1.896394in}{1.763942in}}{\pgfqpoint{1.904630in}{1.763942in}}%
\pgfpathclose%
\pgfusepath{stroke,fill}%
\end{pgfscope}%
\begin{pgfscope}%
\pgfpathrectangle{\pgfqpoint{0.457963in}{0.528059in}}{\pgfqpoint{6.200000in}{2.285714in}} %
\pgfusepath{clip}%
\pgfsetbuttcap%
\pgfsetroundjoin%
\definecolor{currentfill}{rgb}{1.000000,0.333333,0.333333}%
\pgfsetfillcolor{currentfill}%
\pgfsetlinewidth{1.003750pt}%
\definecolor{currentstroke}{rgb}{1.000000,0.333333,0.333333}%
\pgfsetstrokecolor{currentstroke}%
\pgfsetdash{}{0pt}%
\pgfpathmoveto{\pgfqpoint{2.152630in}{1.698636in}}%
\pgfpathcurveto{\pgfqpoint{2.160866in}{1.698636in}}{\pgfqpoint{2.168766in}{1.701908in}}{\pgfqpoint{2.174590in}{1.707732in}}%
\pgfpathcurveto{\pgfqpoint{2.180414in}{1.713556in}}{\pgfqpoint{2.183686in}{1.721456in}}{\pgfqpoint{2.183686in}{1.729692in}}%
\pgfpathcurveto{\pgfqpoint{2.183686in}{1.737928in}}{\pgfqpoint{2.180414in}{1.745828in}}{\pgfqpoint{2.174590in}{1.751652in}}%
\pgfpathcurveto{\pgfqpoint{2.168766in}{1.757476in}}{\pgfqpoint{2.160866in}{1.760749in}}{\pgfqpoint{2.152630in}{1.760749in}}%
\pgfpathcurveto{\pgfqpoint{2.144394in}{1.760749in}}{\pgfqpoint{2.136494in}{1.757476in}}{\pgfqpoint{2.130670in}{1.751652in}}%
\pgfpathcurveto{\pgfqpoint{2.124846in}{1.745828in}}{\pgfqpoint{2.121574in}{1.737928in}}{\pgfqpoint{2.121574in}{1.729692in}}%
\pgfpathcurveto{\pgfqpoint{2.121574in}{1.721456in}}{\pgfqpoint{2.124846in}{1.713556in}}{\pgfqpoint{2.130670in}{1.707732in}}%
\pgfpathcurveto{\pgfqpoint{2.136494in}{1.701908in}}{\pgfqpoint{2.144394in}{1.698636in}}{\pgfqpoint{2.152630in}{1.698636in}}%
\pgfpathclose%
\pgfusepath{stroke,fill}%
\end{pgfscope}%
\begin{pgfscope}%
\pgfpathrectangle{\pgfqpoint{0.457963in}{0.528059in}}{\pgfqpoint{6.200000in}{2.285714in}} %
\pgfusepath{clip}%
\pgfsetbuttcap%
\pgfsetroundjoin%
\definecolor{currentfill}{rgb}{1.000000,0.333333,0.333333}%
\pgfsetfillcolor{currentfill}%
\pgfsetlinewidth{1.003750pt}%
\definecolor{currentstroke}{rgb}{1.000000,0.333333,0.333333}%
\pgfsetstrokecolor{currentstroke}%
\pgfsetdash{}{0pt}%
\pgfpathmoveto{\pgfqpoint{2.152630in}{1.711697in}}%
\pgfpathcurveto{\pgfqpoint{2.160866in}{1.711697in}}{\pgfqpoint{2.168766in}{1.714969in}}{\pgfqpoint{2.174590in}{1.720793in}}%
\pgfpathcurveto{\pgfqpoint{2.180414in}{1.726617in}}{\pgfqpoint{2.183686in}{1.734517in}}{\pgfqpoint{2.183686in}{1.742753in}}%
\pgfpathcurveto{\pgfqpoint{2.183686in}{1.750990in}}{\pgfqpoint{2.180414in}{1.758890in}}{\pgfqpoint{2.174590in}{1.764714in}}%
\pgfpathcurveto{\pgfqpoint{2.168766in}{1.770538in}}{\pgfqpoint{2.160866in}{1.773810in}}{\pgfqpoint{2.152630in}{1.773810in}}%
\pgfpathcurveto{\pgfqpoint{2.144394in}{1.773810in}}{\pgfqpoint{2.136494in}{1.770538in}}{\pgfqpoint{2.130670in}{1.764714in}}%
\pgfpathcurveto{\pgfqpoint{2.124846in}{1.758890in}}{\pgfqpoint{2.121574in}{1.750990in}}{\pgfqpoint{2.121574in}{1.742753in}}%
\pgfpathcurveto{\pgfqpoint{2.121574in}{1.734517in}}{\pgfqpoint{2.124846in}{1.726617in}}{\pgfqpoint{2.130670in}{1.720793in}}%
\pgfpathcurveto{\pgfqpoint{2.136494in}{1.714969in}}{\pgfqpoint{2.144394in}{1.711697in}}{\pgfqpoint{2.152630in}{1.711697in}}%
\pgfpathclose%
\pgfusepath{stroke,fill}%
\end{pgfscope}%
\begin{pgfscope}%
\pgfpathrectangle{\pgfqpoint{0.457963in}{0.528059in}}{\pgfqpoint{6.200000in}{2.285714in}} %
\pgfusepath{clip}%
\pgfsetbuttcap%
\pgfsetroundjoin%
\definecolor{currentfill}{rgb}{1.000000,0.333333,0.333333}%
\pgfsetfillcolor{currentfill}%
\pgfsetlinewidth{1.003750pt}%
\definecolor{currentstroke}{rgb}{1.000000,0.333333,0.333333}%
\pgfsetstrokecolor{currentstroke}%
\pgfsetdash{}{0pt}%
\pgfpathmoveto{\pgfqpoint{2.400630in}{1.581085in}}%
\pgfpathcurveto{\pgfqpoint{2.408866in}{1.581085in}}{\pgfqpoint{2.416766in}{1.584357in}}{\pgfqpoint{2.422590in}{1.590181in}}%
\pgfpathcurveto{\pgfqpoint{2.428414in}{1.596005in}}{\pgfqpoint{2.431686in}{1.603905in}}{\pgfqpoint{2.431686in}{1.612141in}}%
\pgfpathcurveto{\pgfqpoint{2.431686in}{1.620377in}}{\pgfqpoint{2.428414in}{1.628277in}}{\pgfqpoint{2.422590in}{1.634101in}}%
\pgfpathcurveto{\pgfqpoint{2.416766in}{1.639925in}}{\pgfqpoint{2.408866in}{1.643198in}}{\pgfqpoint{2.400630in}{1.643198in}}%
\pgfpathcurveto{\pgfqpoint{2.392394in}{1.643198in}}{\pgfqpoint{2.384494in}{1.639925in}}{\pgfqpoint{2.378670in}{1.634101in}}%
\pgfpathcurveto{\pgfqpoint{2.372846in}{1.628277in}}{\pgfqpoint{2.369574in}{1.620377in}}{\pgfqpoint{2.369574in}{1.612141in}}%
\pgfpathcurveto{\pgfqpoint{2.369574in}{1.603905in}}{\pgfqpoint{2.372846in}{1.596005in}}{\pgfqpoint{2.378670in}{1.590181in}}%
\pgfpathcurveto{\pgfqpoint{2.384494in}{1.584357in}}{\pgfqpoint{2.392394in}{1.581085in}}{\pgfqpoint{2.400630in}{1.581085in}}%
\pgfpathclose%
\pgfusepath{stroke,fill}%
\end{pgfscope}%
\begin{pgfscope}%
\pgfpathrectangle{\pgfqpoint{0.457963in}{0.528059in}}{\pgfqpoint{6.200000in}{2.285714in}} %
\pgfusepath{clip}%
\pgfsetbuttcap%
\pgfsetroundjoin%
\definecolor{currentfill}{rgb}{1.000000,0.333333,0.333333}%
\pgfsetfillcolor{currentfill}%
\pgfsetlinewidth{1.003750pt}%
\definecolor{currentstroke}{rgb}{1.000000,0.333333,0.333333}%
\pgfsetstrokecolor{currentstroke}%
\pgfsetdash{}{0pt}%
\pgfpathmoveto{\pgfqpoint{2.514297in}{1.163125in}}%
\pgfpathcurveto{\pgfqpoint{2.522533in}{1.163125in}}{\pgfqpoint{2.530433in}{1.166398in}}{\pgfqpoint{2.536257in}{1.172222in}}%
\pgfpathcurveto{\pgfqpoint{2.542081in}{1.178046in}}{\pgfqpoint{2.545353in}{1.185946in}}{\pgfqpoint{2.545353in}{1.194182in}}%
\pgfpathcurveto{\pgfqpoint{2.545353in}{1.202418in}}{\pgfqpoint{2.542081in}{1.210318in}}{\pgfqpoint{2.536257in}{1.216142in}}%
\pgfpathcurveto{\pgfqpoint{2.530433in}{1.221966in}}{\pgfqpoint{2.522533in}{1.225238in}}{\pgfqpoint{2.514297in}{1.225238in}}%
\pgfpathcurveto{\pgfqpoint{2.506060in}{1.225238in}}{\pgfqpoint{2.498160in}{1.221966in}}{\pgfqpoint{2.492336in}{1.216142in}}%
\pgfpathcurveto{\pgfqpoint{2.486512in}{1.210318in}}{\pgfqpoint{2.483240in}{1.202418in}}{\pgfqpoint{2.483240in}{1.194182in}}%
\pgfpathcurveto{\pgfqpoint{2.483240in}{1.185946in}}{\pgfqpoint{2.486512in}{1.178046in}}{\pgfqpoint{2.492336in}{1.172222in}}%
\pgfpathcurveto{\pgfqpoint{2.498160in}{1.166398in}}{\pgfqpoint{2.506060in}{1.163125in}}{\pgfqpoint{2.514297in}{1.163125in}}%
\pgfpathclose%
\pgfusepath{stroke,fill}%
\end{pgfscope}%
\begin{pgfscope}%
\pgfpathrectangle{\pgfqpoint{0.457963in}{0.528059in}}{\pgfqpoint{6.200000in}{2.285714in}} %
\pgfusepath{clip}%
\pgfsetbuttcap%
\pgfsetroundjoin%
\definecolor{currentfill}{rgb}{1.000000,0.333333,0.333333}%
\pgfsetfillcolor{currentfill}%
\pgfsetlinewidth{1.003750pt}%
\definecolor{currentstroke}{rgb}{1.000000,0.333333,0.333333}%
\pgfsetstrokecolor{currentstroke}%
\pgfsetdash{}{0pt}%
\pgfpathmoveto{\pgfqpoint{3.630297in}{0.993329in}}%
\pgfpathcurveto{\pgfqpoint{3.638533in}{0.993329in}}{\pgfqpoint{3.646433in}{0.996602in}}{\pgfqpoint{3.652257in}{1.002426in}}%
\pgfpathcurveto{\pgfqpoint{3.658081in}{1.008250in}}{\pgfqpoint{3.661353in}{1.016150in}}{\pgfqpoint{3.661353in}{1.024386in}}%
\pgfpathcurveto{\pgfqpoint{3.661353in}{1.032622in}}{\pgfqpoint{3.658081in}{1.040522in}}{\pgfqpoint{3.652257in}{1.046346in}}%
\pgfpathcurveto{\pgfqpoint{3.646433in}{1.052170in}}{\pgfqpoint{3.638533in}{1.055442in}}{\pgfqpoint{3.630297in}{1.055442in}}%
\pgfpathcurveto{\pgfqpoint{3.622060in}{1.055442in}}{\pgfqpoint{3.614160in}{1.052170in}}{\pgfqpoint{3.608336in}{1.046346in}}%
\pgfpathcurveto{\pgfqpoint{3.602512in}{1.040522in}}{\pgfqpoint{3.599240in}{1.032622in}}{\pgfqpoint{3.599240in}{1.024386in}}%
\pgfpathcurveto{\pgfqpoint{3.599240in}{1.016150in}}{\pgfqpoint{3.602512in}{1.008250in}}{\pgfqpoint{3.608336in}{1.002426in}}%
\pgfpathcurveto{\pgfqpoint{3.614160in}{0.996602in}}{\pgfqpoint{3.622060in}{0.993329in}}{\pgfqpoint{3.630297in}{0.993329in}}%
\pgfpathclose%
\pgfusepath{stroke,fill}%
\end{pgfscope}%
\begin{pgfscope}%
\pgfpathrectangle{\pgfqpoint{0.457963in}{0.528059in}}{\pgfqpoint{6.200000in}{2.285714in}} %
\pgfusepath{clip}%
\pgfsetbuttcap%
\pgfsetroundjoin%
\definecolor{currentfill}{rgb}{1.000000,0.166667,0.166667}%
\pgfsetfillcolor{currentfill}%
\pgfsetlinewidth{1.003750pt}%
\definecolor{currentstroke}{rgb}{1.000000,0.166667,0.166667}%
\pgfsetstrokecolor{currentstroke}%
\pgfsetdash{}{0pt}%
\pgfpathmoveto{\pgfqpoint{0.457963in}{2.129656in}}%
\pgfpathcurveto{\pgfqpoint{0.466200in}{2.129656in}}{\pgfqpoint{0.474100in}{2.132928in}}{\pgfqpoint{0.479924in}{2.138752in}}%
\pgfpathcurveto{\pgfqpoint{0.485748in}{2.144576in}}{\pgfqpoint{0.489020in}{2.152476in}}{\pgfqpoint{0.489020in}{2.160713in}}%
\pgfpathcurveto{\pgfqpoint{0.489020in}{2.168949in}}{\pgfqpoint{0.485748in}{2.176849in}}{\pgfqpoint{0.479924in}{2.182673in}}%
\pgfpathcurveto{\pgfqpoint{0.474100in}{2.188497in}}{\pgfqpoint{0.466200in}{2.191769in}}{\pgfqpoint{0.457963in}{2.191769in}}%
\pgfpathcurveto{\pgfqpoint{0.449727in}{2.191769in}}{\pgfqpoint{0.441827in}{2.188497in}}{\pgfqpoint{0.436003in}{2.182673in}}%
\pgfpathcurveto{\pgfqpoint{0.430179in}{2.176849in}}{\pgfqpoint{0.426907in}{2.168949in}}{\pgfqpoint{0.426907in}{2.160713in}}%
\pgfpathcurveto{\pgfqpoint{0.426907in}{2.152476in}}{\pgfqpoint{0.430179in}{2.144576in}}{\pgfqpoint{0.436003in}{2.138752in}}%
\pgfpathcurveto{\pgfqpoint{0.441827in}{2.132928in}}{\pgfqpoint{0.449727in}{2.129656in}}{\pgfqpoint{0.457963in}{2.129656in}}%
\pgfpathclose%
\pgfusepath{stroke,fill}%
\end{pgfscope}%
\begin{pgfscope}%
\pgfpathrectangle{\pgfqpoint{0.457963in}{0.528059in}}{\pgfqpoint{6.200000in}{2.285714in}} %
\pgfusepath{clip}%
\pgfsetbuttcap%
\pgfsetroundjoin%
\definecolor{currentfill}{rgb}{1.000000,0.166667,0.166667}%
\pgfsetfillcolor{currentfill}%
\pgfsetlinewidth{1.003750pt}%
\definecolor{currentstroke}{rgb}{1.000000,0.166667,0.166667}%
\pgfsetstrokecolor{currentstroke}%
\pgfsetdash{}{0pt}%
\pgfpathmoveto{\pgfqpoint{0.457963in}{2.129656in}}%
\pgfpathcurveto{\pgfqpoint{0.466200in}{2.129656in}}{\pgfqpoint{0.474100in}{2.132928in}}{\pgfqpoint{0.479924in}{2.138752in}}%
\pgfpathcurveto{\pgfqpoint{0.485748in}{2.144576in}}{\pgfqpoint{0.489020in}{2.152476in}}{\pgfqpoint{0.489020in}{2.160713in}}%
\pgfpathcurveto{\pgfqpoint{0.489020in}{2.168949in}}{\pgfqpoint{0.485748in}{2.176849in}}{\pgfqpoint{0.479924in}{2.182673in}}%
\pgfpathcurveto{\pgfqpoint{0.474100in}{2.188497in}}{\pgfqpoint{0.466200in}{2.191769in}}{\pgfqpoint{0.457963in}{2.191769in}}%
\pgfpathcurveto{\pgfqpoint{0.449727in}{2.191769in}}{\pgfqpoint{0.441827in}{2.188497in}}{\pgfqpoint{0.436003in}{2.182673in}}%
\pgfpathcurveto{\pgfqpoint{0.430179in}{2.176849in}}{\pgfqpoint{0.426907in}{2.168949in}}{\pgfqpoint{0.426907in}{2.160713in}}%
\pgfpathcurveto{\pgfqpoint{0.426907in}{2.152476in}}{\pgfqpoint{0.430179in}{2.144576in}}{\pgfqpoint{0.436003in}{2.138752in}}%
\pgfpathcurveto{\pgfqpoint{0.441827in}{2.132928in}}{\pgfqpoint{0.449727in}{2.129656in}}{\pgfqpoint{0.457963in}{2.129656in}}%
\pgfpathclose%
\pgfusepath{stroke,fill}%
\end{pgfscope}%
\begin{pgfscope}%
\pgfpathrectangle{\pgfqpoint{0.457963in}{0.528059in}}{\pgfqpoint{6.200000in}{2.285714in}} %
\pgfusepath{clip}%
\pgfsetbuttcap%
\pgfsetroundjoin%
\definecolor{currentfill}{rgb}{1.000000,0.166667,0.166667}%
\pgfsetfillcolor{currentfill}%
\pgfsetlinewidth{1.003750pt}%
\definecolor{currentstroke}{rgb}{1.000000,0.166667,0.166667}%
\pgfsetstrokecolor{currentstroke}%
\pgfsetdash{}{0pt}%
\pgfpathmoveto{\pgfqpoint{0.457963in}{2.129656in}}%
\pgfpathcurveto{\pgfqpoint{0.466200in}{2.129656in}}{\pgfqpoint{0.474100in}{2.132928in}}{\pgfqpoint{0.479924in}{2.138752in}}%
\pgfpathcurveto{\pgfqpoint{0.485748in}{2.144576in}}{\pgfqpoint{0.489020in}{2.152476in}}{\pgfqpoint{0.489020in}{2.160713in}}%
\pgfpathcurveto{\pgfqpoint{0.489020in}{2.168949in}}{\pgfqpoint{0.485748in}{2.176849in}}{\pgfqpoint{0.479924in}{2.182673in}}%
\pgfpathcurveto{\pgfqpoint{0.474100in}{2.188497in}}{\pgfqpoint{0.466200in}{2.191769in}}{\pgfqpoint{0.457963in}{2.191769in}}%
\pgfpathcurveto{\pgfqpoint{0.449727in}{2.191769in}}{\pgfqpoint{0.441827in}{2.188497in}}{\pgfqpoint{0.436003in}{2.182673in}}%
\pgfpathcurveto{\pgfqpoint{0.430179in}{2.176849in}}{\pgfqpoint{0.426907in}{2.168949in}}{\pgfqpoint{0.426907in}{2.160713in}}%
\pgfpathcurveto{\pgfqpoint{0.426907in}{2.152476in}}{\pgfqpoint{0.430179in}{2.144576in}}{\pgfqpoint{0.436003in}{2.138752in}}%
\pgfpathcurveto{\pgfqpoint{0.441827in}{2.132928in}}{\pgfqpoint{0.449727in}{2.129656in}}{\pgfqpoint{0.457963in}{2.129656in}}%
\pgfpathclose%
\pgfusepath{stroke,fill}%
\end{pgfscope}%
\begin{pgfscope}%
\pgfpathrectangle{\pgfqpoint{0.457963in}{0.528059in}}{\pgfqpoint{6.200000in}{2.285714in}} %
\pgfusepath{clip}%
\pgfsetbuttcap%
\pgfsetroundjoin%
\definecolor{currentfill}{rgb}{1.000000,0.166667,0.166667}%
\pgfsetfillcolor{currentfill}%
\pgfsetlinewidth{1.003750pt}%
\definecolor{currentstroke}{rgb}{1.000000,0.166667,0.166667}%
\pgfsetstrokecolor{currentstroke}%
\pgfsetdash{}{0pt}%
\pgfpathmoveto{\pgfqpoint{0.457963in}{2.129656in}}%
\pgfpathcurveto{\pgfqpoint{0.466200in}{2.129656in}}{\pgfqpoint{0.474100in}{2.132928in}}{\pgfqpoint{0.479924in}{2.138752in}}%
\pgfpathcurveto{\pgfqpoint{0.485748in}{2.144576in}}{\pgfqpoint{0.489020in}{2.152476in}}{\pgfqpoint{0.489020in}{2.160713in}}%
\pgfpathcurveto{\pgfqpoint{0.489020in}{2.168949in}}{\pgfqpoint{0.485748in}{2.176849in}}{\pgfqpoint{0.479924in}{2.182673in}}%
\pgfpathcurveto{\pgfqpoint{0.474100in}{2.188497in}}{\pgfqpoint{0.466200in}{2.191769in}}{\pgfqpoint{0.457963in}{2.191769in}}%
\pgfpathcurveto{\pgfqpoint{0.449727in}{2.191769in}}{\pgfqpoint{0.441827in}{2.188497in}}{\pgfqpoint{0.436003in}{2.182673in}}%
\pgfpathcurveto{\pgfqpoint{0.430179in}{2.176849in}}{\pgfqpoint{0.426907in}{2.168949in}}{\pgfqpoint{0.426907in}{2.160713in}}%
\pgfpathcurveto{\pgfqpoint{0.426907in}{2.152476in}}{\pgfqpoint{0.430179in}{2.144576in}}{\pgfqpoint{0.436003in}{2.138752in}}%
\pgfpathcurveto{\pgfqpoint{0.441827in}{2.132928in}}{\pgfqpoint{0.449727in}{2.129656in}}{\pgfqpoint{0.457963in}{2.129656in}}%
\pgfpathclose%
\pgfusepath{stroke,fill}%
\end{pgfscope}%
\begin{pgfscope}%
\pgfpathrectangle{\pgfqpoint{0.457963in}{0.528059in}}{\pgfqpoint{6.200000in}{2.285714in}} %
\pgfusepath{clip}%
\pgfsetbuttcap%
\pgfsetroundjoin%
\definecolor{currentfill}{rgb}{1.000000,0.166667,0.166667}%
\pgfsetfillcolor{currentfill}%
\pgfsetlinewidth{1.003750pt}%
\definecolor{currentstroke}{rgb}{1.000000,0.166667,0.166667}%
\pgfsetstrokecolor{currentstroke}%
\pgfsetdash{}{0pt}%
\pgfpathmoveto{\pgfqpoint{0.468297in}{2.064350in}}%
\pgfpathcurveto{\pgfqpoint{0.476533in}{2.064350in}}{\pgfqpoint{0.484433in}{2.067622in}}{\pgfqpoint{0.490257in}{2.073446in}}%
\pgfpathcurveto{\pgfqpoint{0.496081in}{2.079270in}}{\pgfqpoint{0.499353in}{2.087170in}}{\pgfqpoint{0.499353in}{2.095406in}}%
\pgfpathcurveto{\pgfqpoint{0.499353in}{2.103643in}}{\pgfqpoint{0.496081in}{2.111543in}}{\pgfqpoint{0.490257in}{2.117367in}}%
\pgfpathcurveto{\pgfqpoint{0.484433in}{2.123191in}}{\pgfqpoint{0.476533in}{2.126463in}}{\pgfqpoint{0.468297in}{2.126463in}}%
\pgfpathcurveto{\pgfqpoint{0.460060in}{2.126463in}}{\pgfqpoint{0.452160in}{2.123191in}}{\pgfqpoint{0.446336in}{2.117367in}}%
\pgfpathcurveto{\pgfqpoint{0.440512in}{2.111543in}}{\pgfqpoint{0.437240in}{2.103643in}}{\pgfqpoint{0.437240in}{2.095406in}}%
\pgfpathcurveto{\pgfqpoint{0.437240in}{2.087170in}}{\pgfqpoint{0.440512in}{2.079270in}}{\pgfqpoint{0.446336in}{2.073446in}}%
\pgfpathcurveto{\pgfqpoint{0.452160in}{2.067622in}}{\pgfqpoint{0.460060in}{2.064350in}}{\pgfqpoint{0.468297in}{2.064350in}}%
\pgfpathclose%
\pgfusepath{stroke,fill}%
\end{pgfscope}%
\begin{pgfscope}%
\pgfpathrectangle{\pgfqpoint{0.457963in}{0.528059in}}{\pgfqpoint{6.200000in}{2.285714in}} %
\pgfusepath{clip}%
\pgfsetbuttcap%
\pgfsetroundjoin%
\definecolor{currentfill}{rgb}{1.000000,0.166667,0.166667}%
\pgfsetfillcolor{currentfill}%
\pgfsetlinewidth{1.003750pt}%
\definecolor{currentstroke}{rgb}{1.000000,0.166667,0.166667}%
\pgfsetstrokecolor{currentstroke}%
\pgfsetdash{}{0pt}%
\pgfpathmoveto{\pgfqpoint{0.509630in}{1.999044in}}%
\pgfpathcurveto{\pgfqpoint{0.517866in}{1.999044in}}{\pgfqpoint{0.525766in}{2.002316in}}{\pgfqpoint{0.531590in}{2.008140in}}%
\pgfpathcurveto{\pgfqpoint{0.537414in}{2.013964in}}{\pgfqpoint{0.540686in}{2.021864in}}{\pgfqpoint{0.540686in}{2.030100in}}%
\pgfpathcurveto{\pgfqpoint{0.540686in}{2.038337in}}{\pgfqpoint{0.537414in}{2.046237in}}{\pgfqpoint{0.531590in}{2.052061in}}%
\pgfpathcurveto{\pgfqpoint{0.525766in}{2.057884in}}{\pgfqpoint{0.517866in}{2.061157in}}{\pgfqpoint{0.509630in}{2.061157in}}%
\pgfpathcurveto{\pgfqpoint{0.501394in}{2.061157in}}{\pgfqpoint{0.493494in}{2.057884in}}{\pgfqpoint{0.487670in}{2.052061in}}%
\pgfpathcurveto{\pgfqpoint{0.481846in}{2.046237in}}{\pgfqpoint{0.478574in}{2.038337in}}{\pgfqpoint{0.478574in}{2.030100in}}%
\pgfpathcurveto{\pgfqpoint{0.478574in}{2.021864in}}{\pgfqpoint{0.481846in}{2.013964in}}{\pgfqpoint{0.487670in}{2.008140in}}%
\pgfpathcurveto{\pgfqpoint{0.493494in}{2.002316in}}{\pgfqpoint{0.501394in}{1.999044in}}{\pgfqpoint{0.509630in}{1.999044in}}%
\pgfpathclose%
\pgfusepath{stroke,fill}%
\end{pgfscope}%
\begin{pgfscope}%
\pgfpathrectangle{\pgfqpoint{0.457963in}{0.528059in}}{\pgfqpoint{6.200000in}{2.285714in}} %
\pgfusepath{clip}%
\pgfsetbuttcap%
\pgfsetroundjoin%
\definecolor{currentfill}{rgb}{1.000000,0.166667,0.166667}%
\pgfsetfillcolor{currentfill}%
\pgfsetlinewidth{1.003750pt}%
\definecolor{currentstroke}{rgb}{1.000000,0.166667,0.166667}%
\pgfsetstrokecolor{currentstroke}%
\pgfsetdash{}{0pt}%
\pgfpathmoveto{\pgfqpoint{0.643963in}{2.129656in}}%
\pgfpathcurveto{\pgfqpoint{0.652200in}{2.129656in}}{\pgfqpoint{0.660100in}{2.132928in}}{\pgfqpoint{0.665924in}{2.138752in}}%
\pgfpathcurveto{\pgfqpoint{0.671748in}{2.144576in}}{\pgfqpoint{0.675020in}{2.152476in}}{\pgfqpoint{0.675020in}{2.160713in}}%
\pgfpathcurveto{\pgfqpoint{0.675020in}{2.168949in}}{\pgfqpoint{0.671748in}{2.176849in}}{\pgfqpoint{0.665924in}{2.182673in}}%
\pgfpathcurveto{\pgfqpoint{0.660100in}{2.188497in}}{\pgfqpoint{0.652200in}{2.191769in}}{\pgfqpoint{0.643963in}{2.191769in}}%
\pgfpathcurveto{\pgfqpoint{0.635727in}{2.191769in}}{\pgfqpoint{0.627827in}{2.188497in}}{\pgfqpoint{0.622003in}{2.182673in}}%
\pgfpathcurveto{\pgfqpoint{0.616179in}{2.176849in}}{\pgfqpoint{0.612907in}{2.168949in}}{\pgfqpoint{0.612907in}{2.160713in}}%
\pgfpathcurveto{\pgfqpoint{0.612907in}{2.152476in}}{\pgfqpoint{0.616179in}{2.144576in}}{\pgfqpoint{0.622003in}{2.138752in}}%
\pgfpathcurveto{\pgfqpoint{0.627827in}{2.132928in}}{\pgfqpoint{0.635727in}{2.129656in}}{\pgfqpoint{0.643963in}{2.129656in}}%
\pgfpathclose%
\pgfusepath{stroke,fill}%
\end{pgfscope}%
\begin{pgfscope}%
\pgfpathrectangle{\pgfqpoint{0.457963in}{0.528059in}}{\pgfqpoint{6.200000in}{2.285714in}} %
\pgfusepath{clip}%
\pgfsetbuttcap%
\pgfsetroundjoin%
\definecolor{currentfill}{rgb}{1.000000,0.166667,0.166667}%
\pgfsetfillcolor{currentfill}%
\pgfsetlinewidth{1.003750pt}%
\definecolor{currentstroke}{rgb}{1.000000,0.166667,0.166667}%
\pgfsetstrokecolor{currentstroke}%
\pgfsetdash{}{0pt}%
\pgfpathmoveto{\pgfqpoint{0.695630in}{1.933738in}}%
\pgfpathcurveto{\pgfqpoint{0.703866in}{1.933738in}}{\pgfqpoint{0.711766in}{1.937010in}}{\pgfqpoint{0.717590in}{1.942834in}}%
\pgfpathcurveto{\pgfqpoint{0.723414in}{1.948658in}}{\pgfqpoint{0.726686in}{1.956558in}}{\pgfqpoint{0.726686in}{1.964794in}}%
\pgfpathcurveto{\pgfqpoint{0.726686in}{1.973030in}}{\pgfqpoint{0.723414in}{1.980930in}}{\pgfqpoint{0.717590in}{1.986754in}}%
\pgfpathcurveto{\pgfqpoint{0.711766in}{1.992578in}}{\pgfqpoint{0.703866in}{1.995851in}}{\pgfqpoint{0.695630in}{1.995851in}}%
\pgfpathcurveto{\pgfqpoint{0.687394in}{1.995851in}}{\pgfqpoint{0.679494in}{1.992578in}}{\pgfqpoint{0.673670in}{1.986754in}}%
\pgfpathcurveto{\pgfqpoint{0.667846in}{1.980930in}}{\pgfqpoint{0.664574in}{1.973030in}}{\pgfqpoint{0.664574in}{1.964794in}}%
\pgfpathcurveto{\pgfqpoint{0.664574in}{1.956558in}}{\pgfqpoint{0.667846in}{1.948658in}}{\pgfqpoint{0.673670in}{1.942834in}}%
\pgfpathcurveto{\pgfqpoint{0.679494in}{1.937010in}}{\pgfqpoint{0.687394in}{1.933738in}}{\pgfqpoint{0.695630in}{1.933738in}}%
\pgfpathclose%
\pgfusepath{stroke,fill}%
\end{pgfscope}%
\begin{pgfscope}%
\pgfpathrectangle{\pgfqpoint{0.457963in}{0.528059in}}{\pgfqpoint{6.200000in}{2.285714in}} %
\pgfusepath{clip}%
\pgfsetbuttcap%
\pgfsetroundjoin%
\definecolor{currentfill}{rgb}{1.000000,0.166667,0.166667}%
\pgfsetfillcolor{currentfill}%
\pgfsetlinewidth{1.003750pt}%
\definecolor{currentstroke}{rgb}{1.000000,0.166667,0.166667}%
\pgfsetstrokecolor{currentstroke}%
\pgfsetdash{}{0pt}%
\pgfpathmoveto{\pgfqpoint{0.840297in}{1.672513in}}%
\pgfpathcurveto{\pgfqpoint{0.848533in}{1.672513in}}{\pgfqpoint{0.856433in}{1.675785in}}{\pgfqpoint{0.862257in}{1.681609in}}%
\pgfpathcurveto{\pgfqpoint{0.868081in}{1.687433in}}{\pgfqpoint{0.871353in}{1.695333in}}{\pgfqpoint{0.871353in}{1.703570in}}%
\pgfpathcurveto{\pgfqpoint{0.871353in}{1.711806in}}{\pgfqpoint{0.868081in}{1.719706in}}{\pgfqpoint{0.862257in}{1.725530in}}%
\pgfpathcurveto{\pgfqpoint{0.856433in}{1.731354in}}{\pgfqpoint{0.848533in}{1.734626in}}{\pgfqpoint{0.840297in}{1.734626in}}%
\pgfpathcurveto{\pgfqpoint{0.832060in}{1.734626in}}{\pgfqpoint{0.824160in}{1.731354in}}{\pgfqpoint{0.818336in}{1.725530in}}%
\pgfpathcurveto{\pgfqpoint{0.812512in}{1.719706in}}{\pgfqpoint{0.809240in}{1.711806in}}{\pgfqpoint{0.809240in}{1.703570in}}%
\pgfpathcurveto{\pgfqpoint{0.809240in}{1.695333in}}{\pgfqpoint{0.812512in}{1.687433in}}{\pgfqpoint{0.818336in}{1.681609in}}%
\pgfpathcurveto{\pgfqpoint{0.824160in}{1.675785in}}{\pgfqpoint{0.832060in}{1.672513in}}{\pgfqpoint{0.840297in}{1.672513in}}%
\pgfpathclose%
\pgfusepath{stroke,fill}%
\end{pgfscope}%
\begin{pgfscope}%
\pgfpathrectangle{\pgfqpoint{0.457963in}{0.528059in}}{\pgfqpoint{6.200000in}{2.285714in}} %
\pgfusepath{clip}%
\pgfsetbuttcap%
\pgfsetroundjoin%
\definecolor{currentfill}{rgb}{1.000000,0.166667,0.166667}%
\pgfsetfillcolor{currentfill}%
\pgfsetlinewidth{1.003750pt}%
\definecolor{currentstroke}{rgb}{1.000000,0.166667,0.166667}%
\pgfsetstrokecolor{currentstroke}%
\pgfsetdash{}{0pt}%
\pgfpathmoveto{\pgfqpoint{1.057297in}{2.129656in}}%
\pgfpathcurveto{\pgfqpoint{1.065533in}{2.129656in}}{\pgfqpoint{1.073433in}{2.132928in}}{\pgfqpoint{1.079257in}{2.138752in}}%
\pgfpathcurveto{\pgfqpoint{1.085081in}{2.144576in}}{\pgfqpoint{1.088353in}{2.152476in}}{\pgfqpoint{1.088353in}{2.160713in}}%
\pgfpathcurveto{\pgfqpoint{1.088353in}{2.168949in}}{\pgfqpoint{1.085081in}{2.176849in}}{\pgfqpoint{1.079257in}{2.182673in}}%
\pgfpathcurveto{\pgfqpoint{1.073433in}{2.188497in}}{\pgfqpoint{1.065533in}{2.191769in}}{\pgfqpoint{1.057297in}{2.191769in}}%
\pgfpathcurveto{\pgfqpoint{1.049060in}{2.191769in}}{\pgfqpoint{1.041160in}{2.188497in}}{\pgfqpoint{1.035336in}{2.182673in}}%
\pgfpathcurveto{\pgfqpoint{1.029512in}{2.176849in}}{\pgfqpoint{1.026240in}{2.168949in}}{\pgfqpoint{1.026240in}{2.160713in}}%
\pgfpathcurveto{\pgfqpoint{1.026240in}{2.152476in}}{\pgfqpoint{1.029512in}{2.144576in}}{\pgfqpoint{1.035336in}{2.138752in}}%
\pgfpathcurveto{\pgfqpoint{1.041160in}{2.132928in}}{\pgfqpoint{1.049060in}{2.129656in}}{\pgfqpoint{1.057297in}{2.129656in}}%
\pgfpathclose%
\pgfusepath{stroke,fill}%
\end{pgfscope}%
\begin{pgfscope}%
\pgfpathrectangle{\pgfqpoint{0.457963in}{0.528059in}}{\pgfqpoint{6.200000in}{2.285714in}} %
\pgfusepath{clip}%
\pgfsetbuttcap%
\pgfsetroundjoin%
\definecolor{currentfill}{rgb}{1.000000,0.166667,0.166667}%
\pgfsetfillcolor{currentfill}%
\pgfsetlinewidth{1.003750pt}%
\definecolor{currentstroke}{rgb}{1.000000,0.166667,0.166667}%
\pgfsetstrokecolor{currentstroke}%
\pgfsetdash{}{0pt}%
\pgfpathmoveto{\pgfqpoint{1.243297in}{1.345983in}}%
\pgfpathcurveto{\pgfqpoint{1.251533in}{1.345983in}}{\pgfqpoint{1.259433in}{1.349255in}}{\pgfqpoint{1.265257in}{1.355079in}}%
\pgfpathcurveto{\pgfqpoint{1.271081in}{1.360903in}}{\pgfqpoint{1.274353in}{1.368803in}}{\pgfqpoint{1.274353in}{1.377039in}}%
\pgfpathcurveto{\pgfqpoint{1.274353in}{1.385275in}}{\pgfqpoint{1.271081in}{1.393175in}}{\pgfqpoint{1.265257in}{1.398999in}}%
\pgfpathcurveto{\pgfqpoint{1.259433in}{1.404823in}}{\pgfqpoint{1.251533in}{1.408096in}}{\pgfqpoint{1.243297in}{1.408096in}}%
\pgfpathcurveto{\pgfqpoint{1.235060in}{1.408096in}}{\pgfqpoint{1.227160in}{1.404823in}}{\pgfqpoint{1.221336in}{1.398999in}}%
\pgfpathcurveto{\pgfqpoint{1.215512in}{1.393175in}}{\pgfqpoint{1.212240in}{1.385275in}}{\pgfqpoint{1.212240in}{1.377039in}}%
\pgfpathcurveto{\pgfqpoint{1.212240in}{1.368803in}}{\pgfqpoint{1.215512in}{1.360903in}}{\pgfqpoint{1.221336in}{1.355079in}}%
\pgfpathcurveto{\pgfqpoint{1.227160in}{1.349255in}}{\pgfqpoint{1.235060in}{1.345983in}}{\pgfqpoint{1.243297in}{1.345983in}}%
\pgfpathclose%
\pgfusepath{stroke,fill}%
\end{pgfscope}%
\begin{pgfscope}%
\pgfpathrectangle{\pgfqpoint{0.457963in}{0.528059in}}{\pgfqpoint{6.200000in}{2.285714in}} %
\pgfusepath{clip}%
\pgfsetbuttcap%
\pgfsetroundjoin%
\definecolor{currentfill}{rgb}{1.000000,0.166667,0.166667}%
\pgfsetfillcolor{currentfill}%
\pgfsetlinewidth{1.003750pt}%
\definecolor{currentstroke}{rgb}{1.000000,0.166667,0.166667}%
\pgfsetstrokecolor{currentstroke}%
\pgfsetdash{}{0pt}%
\pgfpathmoveto{\pgfqpoint{1.439630in}{2.103534in}}%
\pgfpathcurveto{\pgfqpoint{1.447866in}{2.103534in}}{\pgfqpoint{1.455766in}{2.106806in}}{\pgfqpoint{1.461590in}{2.112630in}}%
\pgfpathcurveto{\pgfqpoint{1.467414in}{2.118454in}}{\pgfqpoint{1.470686in}{2.126354in}}{\pgfqpoint{1.470686in}{2.134590in}}%
\pgfpathcurveto{\pgfqpoint{1.470686in}{2.142826in}}{\pgfqpoint{1.467414in}{2.150726in}}{\pgfqpoint{1.461590in}{2.156550in}}%
\pgfpathcurveto{\pgfqpoint{1.455766in}{2.162374in}}{\pgfqpoint{1.447866in}{2.165647in}}{\pgfqpoint{1.439630in}{2.165647in}}%
\pgfpathcurveto{\pgfqpoint{1.431394in}{2.165647in}}{\pgfqpoint{1.423494in}{2.162374in}}{\pgfqpoint{1.417670in}{2.156550in}}%
\pgfpathcurveto{\pgfqpoint{1.411846in}{2.150726in}}{\pgfqpoint{1.408574in}{2.142826in}}{\pgfqpoint{1.408574in}{2.134590in}}%
\pgfpathcurveto{\pgfqpoint{1.408574in}{2.126354in}}{\pgfqpoint{1.411846in}{2.118454in}}{\pgfqpoint{1.417670in}{2.112630in}}%
\pgfpathcurveto{\pgfqpoint{1.423494in}{2.106806in}}{\pgfqpoint{1.431394in}{2.103534in}}{\pgfqpoint{1.439630in}{2.103534in}}%
\pgfpathclose%
\pgfusepath{stroke,fill}%
\end{pgfscope}%
\begin{pgfscope}%
\pgfpathrectangle{\pgfqpoint{0.457963in}{0.528059in}}{\pgfqpoint{6.200000in}{2.285714in}} %
\pgfusepath{clip}%
\pgfsetbuttcap%
\pgfsetroundjoin%
\definecolor{currentfill}{rgb}{1.000000,0.166667,0.166667}%
\pgfsetfillcolor{currentfill}%
\pgfsetlinewidth{1.003750pt}%
\definecolor{currentstroke}{rgb}{1.000000,0.166667,0.166667}%
\pgfsetstrokecolor{currentstroke}%
\pgfsetdash{}{0pt}%
\pgfpathmoveto{\pgfqpoint{1.460297in}{1.019452in}}%
\pgfpathcurveto{\pgfqpoint{1.468533in}{1.019452in}}{\pgfqpoint{1.476433in}{1.022724in}}{\pgfqpoint{1.482257in}{1.028548in}}%
\pgfpathcurveto{\pgfqpoint{1.488081in}{1.034372in}}{\pgfqpoint{1.491353in}{1.042272in}}{\pgfqpoint{1.491353in}{1.050508in}}%
\pgfpathcurveto{\pgfqpoint{1.491353in}{1.058745in}}{\pgfqpoint{1.488081in}{1.066645in}}{\pgfqpoint{1.482257in}{1.072469in}}%
\pgfpathcurveto{\pgfqpoint{1.476433in}{1.078293in}}{\pgfqpoint{1.468533in}{1.081565in}}{\pgfqpoint{1.460297in}{1.081565in}}%
\pgfpathcurveto{\pgfqpoint{1.452060in}{1.081565in}}{\pgfqpoint{1.444160in}{1.078293in}}{\pgfqpoint{1.438336in}{1.072469in}}%
\pgfpathcurveto{\pgfqpoint{1.432512in}{1.066645in}}{\pgfqpoint{1.429240in}{1.058745in}}{\pgfqpoint{1.429240in}{1.050508in}}%
\pgfpathcurveto{\pgfqpoint{1.429240in}{1.042272in}}{\pgfqpoint{1.432512in}{1.034372in}}{\pgfqpoint{1.438336in}{1.028548in}}%
\pgfpathcurveto{\pgfqpoint{1.444160in}{1.022724in}}{\pgfqpoint{1.452060in}{1.019452in}}{\pgfqpoint{1.460297in}{1.019452in}}%
\pgfpathclose%
\pgfusepath{stroke,fill}%
\end{pgfscope}%
\begin{pgfscope}%
\pgfpathrectangle{\pgfqpoint{0.457963in}{0.528059in}}{\pgfqpoint{6.200000in}{2.285714in}} %
\pgfusepath{clip}%
\pgfsetbuttcap%
\pgfsetroundjoin%
\definecolor{currentfill}{rgb}{1.000000,0.166667,0.166667}%
\pgfsetfillcolor{currentfill}%
\pgfsetlinewidth{1.003750pt}%
\definecolor{currentstroke}{rgb}{1.000000,0.166667,0.166667}%
\pgfsetstrokecolor{currentstroke}%
\pgfsetdash{}{0pt}%
\pgfpathmoveto{\pgfqpoint{1.553297in}{2.129656in}}%
\pgfpathcurveto{\pgfqpoint{1.561533in}{2.129656in}}{\pgfqpoint{1.569433in}{2.132928in}}{\pgfqpoint{1.575257in}{2.138752in}}%
\pgfpathcurveto{\pgfqpoint{1.581081in}{2.144576in}}{\pgfqpoint{1.584353in}{2.152476in}}{\pgfqpoint{1.584353in}{2.160713in}}%
\pgfpathcurveto{\pgfqpoint{1.584353in}{2.168949in}}{\pgfqpoint{1.581081in}{2.176849in}}{\pgfqpoint{1.575257in}{2.182673in}}%
\pgfpathcurveto{\pgfqpoint{1.569433in}{2.188497in}}{\pgfqpoint{1.561533in}{2.191769in}}{\pgfqpoint{1.553297in}{2.191769in}}%
\pgfpathcurveto{\pgfqpoint{1.545060in}{2.191769in}}{\pgfqpoint{1.537160in}{2.188497in}}{\pgfqpoint{1.531336in}{2.182673in}}%
\pgfpathcurveto{\pgfqpoint{1.525512in}{2.176849in}}{\pgfqpoint{1.522240in}{2.168949in}}{\pgfqpoint{1.522240in}{2.160713in}}%
\pgfpathcurveto{\pgfqpoint{1.522240in}{2.152476in}}{\pgfqpoint{1.525512in}{2.144576in}}{\pgfqpoint{1.531336in}{2.138752in}}%
\pgfpathcurveto{\pgfqpoint{1.537160in}{2.132928in}}{\pgfqpoint{1.545060in}{2.129656in}}{\pgfqpoint{1.553297in}{2.129656in}}%
\pgfpathclose%
\pgfusepath{stroke,fill}%
\end{pgfscope}%
\begin{pgfscope}%
\pgfpathrectangle{\pgfqpoint{0.457963in}{0.528059in}}{\pgfqpoint{6.200000in}{2.285714in}} %
\pgfusepath{clip}%
\pgfsetbuttcap%
\pgfsetroundjoin%
\definecolor{currentfill}{rgb}{1.000000,0.166667,0.166667}%
\pgfsetfillcolor{currentfill}%
\pgfsetlinewidth{1.003750pt}%
\definecolor{currentstroke}{rgb}{1.000000,0.166667,0.166667}%
\pgfsetstrokecolor{currentstroke}%
\pgfsetdash{}{0pt}%
\pgfpathmoveto{\pgfqpoint{2.018297in}{1.985983in}}%
\pgfpathcurveto{\pgfqpoint{2.026533in}{1.985983in}}{\pgfqpoint{2.034433in}{1.989255in}}{\pgfqpoint{2.040257in}{1.995079in}}%
\pgfpathcurveto{\pgfqpoint{2.046081in}{2.000903in}}{\pgfqpoint{2.049353in}{2.008803in}}{\pgfqpoint{2.049353in}{2.017039in}}%
\pgfpathcurveto{\pgfqpoint{2.049353in}{2.025275in}}{\pgfqpoint{2.046081in}{2.033175in}}{\pgfqpoint{2.040257in}{2.038999in}}%
\pgfpathcurveto{\pgfqpoint{2.034433in}{2.044823in}}{\pgfqpoint{2.026533in}{2.048096in}}{\pgfqpoint{2.018297in}{2.048096in}}%
\pgfpathcurveto{\pgfqpoint{2.010060in}{2.048096in}}{\pgfqpoint{2.002160in}{2.044823in}}{\pgfqpoint{1.996336in}{2.038999in}}%
\pgfpathcurveto{\pgfqpoint{1.990512in}{2.033175in}}{\pgfqpoint{1.987240in}{2.025275in}}{\pgfqpoint{1.987240in}{2.017039in}}%
\pgfpathcurveto{\pgfqpoint{1.987240in}{2.008803in}}{\pgfqpoint{1.990512in}{2.000903in}}{\pgfqpoint{1.996336in}{1.995079in}}%
\pgfpathcurveto{\pgfqpoint{2.002160in}{1.989255in}}{\pgfqpoint{2.010060in}{1.985983in}}{\pgfqpoint{2.018297in}{1.985983in}}%
\pgfpathclose%
\pgfusepath{stroke,fill}%
\end{pgfscope}%
\begin{pgfscope}%
\pgfpathrectangle{\pgfqpoint{0.457963in}{0.528059in}}{\pgfqpoint{6.200000in}{2.285714in}} %
\pgfusepath{clip}%
\pgfsetbuttcap%
\pgfsetroundjoin%
\definecolor{currentfill}{rgb}{1.000000,0.166667,0.166667}%
\pgfsetfillcolor{currentfill}%
\pgfsetlinewidth{1.003750pt}%
\definecolor{currentstroke}{rgb}{1.000000,0.166667,0.166667}%
\pgfsetstrokecolor{currentstroke}%
\pgfsetdash{}{0pt}%
\pgfpathmoveto{\pgfqpoint{2.038963in}{2.103534in}}%
\pgfpathcurveto{\pgfqpoint{2.047200in}{2.103534in}}{\pgfqpoint{2.055100in}{2.106806in}}{\pgfqpoint{2.060924in}{2.112630in}}%
\pgfpathcurveto{\pgfqpoint{2.066748in}{2.118454in}}{\pgfqpoint{2.070020in}{2.126354in}}{\pgfqpoint{2.070020in}{2.134590in}}%
\pgfpathcurveto{\pgfqpoint{2.070020in}{2.142826in}}{\pgfqpoint{2.066748in}{2.150726in}}{\pgfqpoint{2.060924in}{2.156550in}}%
\pgfpathcurveto{\pgfqpoint{2.055100in}{2.162374in}}{\pgfqpoint{2.047200in}{2.165647in}}{\pgfqpoint{2.038963in}{2.165647in}}%
\pgfpathcurveto{\pgfqpoint{2.030727in}{2.165647in}}{\pgfqpoint{2.022827in}{2.162374in}}{\pgfqpoint{2.017003in}{2.156550in}}%
\pgfpathcurveto{\pgfqpoint{2.011179in}{2.150726in}}{\pgfqpoint{2.007907in}{2.142826in}}{\pgfqpoint{2.007907in}{2.134590in}}%
\pgfpathcurveto{\pgfqpoint{2.007907in}{2.126354in}}{\pgfqpoint{2.011179in}{2.118454in}}{\pgfqpoint{2.017003in}{2.112630in}}%
\pgfpathcurveto{\pgfqpoint{2.022827in}{2.106806in}}{\pgfqpoint{2.030727in}{2.103534in}}{\pgfqpoint{2.038963in}{2.103534in}}%
\pgfpathclose%
\pgfusepath{stroke,fill}%
\end{pgfscope}%
\begin{pgfscope}%
\pgfpathrectangle{\pgfqpoint{0.457963in}{0.528059in}}{\pgfqpoint{6.200000in}{2.285714in}} %
\pgfusepath{clip}%
\pgfsetbuttcap%
\pgfsetroundjoin%
\definecolor{currentfill}{rgb}{1.000000,0.166667,0.166667}%
\pgfsetfillcolor{currentfill}%
\pgfsetlinewidth{1.003750pt}%
\definecolor{currentstroke}{rgb}{1.000000,0.166667,0.166667}%
\pgfsetstrokecolor{currentstroke}%
\pgfsetdash{}{0pt}%
\pgfpathmoveto{\pgfqpoint{2.049297in}{1.698636in}}%
\pgfpathcurveto{\pgfqpoint{2.057533in}{1.698636in}}{\pgfqpoint{2.065433in}{1.701908in}}{\pgfqpoint{2.071257in}{1.707732in}}%
\pgfpathcurveto{\pgfqpoint{2.077081in}{1.713556in}}{\pgfqpoint{2.080353in}{1.721456in}}{\pgfqpoint{2.080353in}{1.729692in}}%
\pgfpathcurveto{\pgfqpoint{2.080353in}{1.737928in}}{\pgfqpoint{2.077081in}{1.745828in}}{\pgfqpoint{2.071257in}{1.751652in}}%
\pgfpathcurveto{\pgfqpoint{2.065433in}{1.757476in}}{\pgfqpoint{2.057533in}{1.760749in}}{\pgfqpoint{2.049297in}{1.760749in}}%
\pgfpathcurveto{\pgfqpoint{2.041060in}{1.760749in}}{\pgfqpoint{2.033160in}{1.757476in}}{\pgfqpoint{2.027336in}{1.751652in}}%
\pgfpathcurveto{\pgfqpoint{2.021512in}{1.745828in}}{\pgfqpoint{2.018240in}{1.737928in}}{\pgfqpoint{2.018240in}{1.729692in}}%
\pgfpathcurveto{\pgfqpoint{2.018240in}{1.721456in}}{\pgfqpoint{2.021512in}{1.713556in}}{\pgfqpoint{2.027336in}{1.707732in}}%
\pgfpathcurveto{\pgfqpoint{2.033160in}{1.701908in}}{\pgfqpoint{2.041060in}{1.698636in}}{\pgfqpoint{2.049297in}{1.698636in}}%
\pgfpathclose%
\pgfusepath{stroke,fill}%
\end{pgfscope}%
\begin{pgfscope}%
\pgfpathrectangle{\pgfqpoint{0.457963in}{0.528059in}}{\pgfqpoint{6.200000in}{2.285714in}} %
\pgfusepath{clip}%
\pgfsetbuttcap%
\pgfsetroundjoin%
\definecolor{currentfill}{rgb}{1.000000,0.166667,0.166667}%
\pgfsetfillcolor{currentfill}%
\pgfsetlinewidth{1.003750pt}%
\definecolor{currentstroke}{rgb}{1.000000,0.166667,0.166667}%
\pgfsetstrokecolor{currentstroke}%
\pgfsetdash{}{0pt}%
\pgfpathmoveto{\pgfqpoint{2.317963in}{2.077411in}}%
\pgfpathcurveto{\pgfqpoint{2.326200in}{2.077411in}}{\pgfqpoint{2.334100in}{2.080683in}}{\pgfqpoint{2.339924in}{2.086507in}}%
\pgfpathcurveto{\pgfqpoint{2.345748in}{2.092331in}}{\pgfqpoint{2.349020in}{2.100231in}}{\pgfqpoint{2.349020in}{2.108468in}}%
\pgfpathcurveto{\pgfqpoint{2.349020in}{2.116704in}}{\pgfqpoint{2.345748in}{2.124604in}}{\pgfqpoint{2.339924in}{2.130428in}}%
\pgfpathcurveto{\pgfqpoint{2.334100in}{2.136252in}}{\pgfqpoint{2.326200in}{2.139524in}}{\pgfqpoint{2.317963in}{2.139524in}}%
\pgfpathcurveto{\pgfqpoint{2.309727in}{2.139524in}}{\pgfqpoint{2.301827in}{2.136252in}}{\pgfqpoint{2.296003in}{2.130428in}}%
\pgfpathcurveto{\pgfqpoint{2.290179in}{2.124604in}}{\pgfqpoint{2.286907in}{2.116704in}}{\pgfqpoint{2.286907in}{2.108468in}}%
\pgfpathcurveto{\pgfqpoint{2.286907in}{2.100231in}}{\pgfqpoint{2.290179in}{2.092331in}}{\pgfqpoint{2.296003in}{2.086507in}}%
\pgfpathcurveto{\pgfqpoint{2.301827in}{2.080683in}}{\pgfqpoint{2.309727in}{2.077411in}}{\pgfqpoint{2.317963in}{2.077411in}}%
\pgfpathclose%
\pgfusepath{stroke,fill}%
\end{pgfscope}%
\begin{pgfscope}%
\pgfpathrectangle{\pgfqpoint{0.457963in}{0.528059in}}{\pgfqpoint{6.200000in}{2.285714in}} %
\pgfusepath{clip}%
\pgfsetbuttcap%
\pgfsetroundjoin%
\definecolor{currentfill}{rgb}{1.000000,0.166667,0.166667}%
\pgfsetfillcolor{currentfill}%
\pgfsetlinewidth{1.003750pt}%
\definecolor{currentstroke}{rgb}{1.000000,0.166667,0.166667}%
\pgfsetstrokecolor{currentstroke}%
\pgfsetdash{}{0pt}%
\pgfpathmoveto{\pgfqpoint{2.545297in}{1.189248in}}%
\pgfpathcurveto{\pgfqpoint{2.553533in}{1.189248in}}{\pgfqpoint{2.561433in}{1.192520in}}{\pgfqpoint{2.567257in}{1.198344in}}%
\pgfpathcurveto{\pgfqpoint{2.573081in}{1.204168in}}{\pgfqpoint{2.576353in}{1.212068in}}{\pgfqpoint{2.576353in}{1.220304in}}%
\pgfpathcurveto{\pgfqpoint{2.576353in}{1.228541in}}{\pgfqpoint{2.573081in}{1.236441in}}{\pgfqpoint{2.567257in}{1.242265in}}%
\pgfpathcurveto{\pgfqpoint{2.561433in}{1.248089in}}{\pgfqpoint{2.553533in}{1.251361in}}{\pgfqpoint{2.545297in}{1.251361in}}%
\pgfpathcurveto{\pgfqpoint{2.537060in}{1.251361in}}{\pgfqpoint{2.529160in}{1.248089in}}{\pgfqpoint{2.523336in}{1.242265in}}%
\pgfpathcurveto{\pgfqpoint{2.517512in}{1.236441in}}{\pgfqpoint{2.514240in}{1.228541in}}{\pgfqpoint{2.514240in}{1.220304in}}%
\pgfpathcurveto{\pgfqpoint{2.514240in}{1.212068in}}{\pgfqpoint{2.517512in}{1.204168in}}{\pgfqpoint{2.523336in}{1.198344in}}%
\pgfpathcurveto{\pgfqpoint{2.529160in}{1.192520in}}{\pgfqpoint{2.537060in}{1.189248in}}{\pgfqpoint{2.545297in}{1.189248in}}%
\pgfpathclose%
\pgfusepath{stroke,fill}%
\end{pgfscope}%
\begin{pgfscope}%
\pgfpathrectangle{\pgfqpoint{0.457963in}{0.528059in}}{\pgfqpoint{6.200000in}{2.285714in}} %
\pgfusepath{clip}%
\pgfsetbuttcap%
\pgfsetroundjoin%
\definecolor{currentfill}{rgb}{1.000000,0.166667,0.166667}%
\pgfsetfillcolor{currentfill}%
\pgfsetlinewidth{1.003750pt}%
\definecolor{currentstroke}{rgb}{1.000000,0.166667,0.166667}%
\pgfsetstrokecolor{currentstroke}%
\pgfsetdash{}{0pt}%
\pgfpathmoveto{\pgfqpoint{3.185963in}{1.071697in}}%
\pgfpathcurveto{\pgfqpoint{3.194200in}{1.071697in}}{\pgfqpoint{3.202100in}{1.074969in}}{\pgfqpoint{3.207924in}{1.080793in}}%
\pgfpathcurveto{\pgfqpoint{3.213748in}{1.086617in}}{\pgfqpoint{3.217020in}{1.094517in}}{\pgfqpoint{3.217020in}{1.102753in}}%
\pgfpathcurveto{\pgfqpoint{3.217020in}{1.110990in}}{\pgfqpoint{3.213748in}{1.118890in}}{\pgfqpoint{3.207924in}{1.124714in}}%
\pgfpathcurveto{\pgfqpoint{3.202100in}{1.130538in}}{\pgfqpoint{3.194200in}{1.133810in}}{\pgfqpoint{3.185963in}{1.133810in}}%
\pgfpathcurveto{\pgfqpoint{3.177727in}{1.133810in}}{\pgfqpoint{3.169827in}{1.130538in}}{\pgfqpoint{3.164003in}{1.124714in}}%
\pgfpathcurveto{\pgfqpoint{3.158179in}{1.118890in}}{\pgfqpoint{3.154907in}{1.110990in}}{\pgfqpoint{3.154907in}{1.102753in}}%
\pgfpathcurveto{\pgfqpoint{3.154907in}{1.094517in}}{\pgfqpoint{3.158179in}{1.086617in}}{\pgfqpoint{3.164003in}{1.080793in}}%
\pgfpathcurveto{\pgfqpoint{3.169827in}{1.074969in}}{\pgfqpoint{3.177727in}{1.071697in}}{\pgfqpoint{3.185963in}{1.071697in}}%
\pgfpathclose%
\pgfusepath{stroke,fill}%
\end{pgfscope}%
\begin{pgfscope}%
\pgfpathrectangle{\pgfqpoint{0.457963in}{0.528059in}}{\pgfqpoint{6.200000in}{2.285714in}} %
\pgfusepath{clip}%
\pgfsetbuttcap%
\pgfsetroundjoin%
\definecolor{currentfill}{rgb}{1.000000,0.000000,0.000000}%
\pgfsetfillcolor{currentfill}%
\pgfsetlinewidth{1.003750pt}%
\definecolor{currentstroke}{rgb}{1.000000,0.000000,0.000000}%
\pgfsetstrokecolor{currentstroke}%
\pgfsetdash{}{0pt}%
\pgfpathmoveto{\pgfqpoint{0.457963in}{2.456187in}}%
\pgfpathcurveto{\pgfqpoint{0.466200in}{2.456187in}}{\pgfqpoint{0.474100in}{2.459459in}}{\pgfqpoint{0.479924in}{2.465283in}}%
\pgfpathcurveto{\pgfqpoint{0.485748in}{2.471107in}}{\pgfqpoint{0.489020in}{2.479007in}}{\pgfqpoint{0.489020in}{2.487243in}}%
\pgfpathcurveto{\pgfqpoint{0.489020in}{2.495479in}}{\pgfqpoint{0.485748in}{2.503379in}}{\pgfqpoint{0.479924in}{2.509203in}}%
\pgfpathcurveto{\pgfqpoint{0.474100in}{2.515027in}}{\pgfqpoint{0.466200in}{2.518300in}}{\pgfqpoint{0.457963in}{2.518300in}}%
\pgfpathcurveto{\pgfqpoint{0.449727in}{2.518300in}}{\pgfqpoint{0.441827in}{2.515027in}}{\pgfqpoint{0.436003in}{2.509203in}}%
\pgfpathcurveto{\pgfqpoint{0.430179in}{2.503379in}}{\pgfqpoint{0.426907in}{2.495479in}}{\pgfqpoint{0.426907in}{2.487243in}}%
\pgfpathcurveto{\pgfqpoint{0.426907in}{2.479007in}}{\pgfqpoint{0.430179in}{2.471107in}}{\pgfqpoint{0.436003in}{2.465283in}}%
\pgfpathcurveto{\pgfqpoint{0.441827in}{2.459459in}}{\pgfqpoint{0.449727in}{2.456187in}}{\pgfqpoint{0.457963in}{2.456187in}}%
\pgfpathclose%
\pgfusepath{stroke,fill}%
\end{pgfscope}%
\begin{pgfscope}%
\pgfpathrectangle{\pgfqpoint{0.457963in}{0.528059in}}{\pgfqpoint{6.200000in}{2.285714in}} %
\pgfusepath{clip}%
\pgfsetbuttcap%
\pgfsetroundjoin%
\definecolor{currentfill}{rgb}{1.000000,0.000000,0.000000}%
\pgfsetfillcolor{currentfill}%
\pgfsetlinewidth{1.003750pt}%
\definecolor{currentstroke}{rgb}{1.000000,0.000000,0.000000}%
\pgfsetstrokecolor{currentstroke}%
\pgfsetdash{}{0pt}%
\pgfpathmoveto{\pgfqpoint{0.457963in}{2.456187in}}%
\pgfpathcurveto{\pgfqpoint{0.466200in}{2.456187in}}{\pgfqpoint{0.474100in}{2.459459in}}{\pgfqpoint{0.479924in}{2.465283in}}%
\pgfpathcurveto{\pgfqpoint{0.485748in}{2.471107in}}{\pgfqpoint{0.489020in}{2.479007in}}{\pgfqpoint{0.489020in}{2.487243in}}%
\pgfpathcurveto{\pgfqpoint{0.489020in}{2.495479in}}{\pgfqpoint{0.485748in}{2.503379in}}{\pgfqpoint{0.479924in}{2.509203in}}%
\pgfpathcurveto{\pgfqpoint{0.474100in}{2.515027in}}{\pgfqpoint{0.466200in}{2.518300in}}{\pgfqpoint{0.457963in}{2.518300in}}%
\pgfpathcurveto{\pgfqpoint{0.449727in}{2.518300in}}{\pgfqpoint{0.441827in}{2.515027in}}{\pgfqpoint{0.436003in}{2.509203in}}%
\pgfpathcurveto{\pgfqpoint{0.430179in}{2.503379in}}{\pgfqpoint{0.426907in}{2.495479in}}{\pgfqpoint{0.426907in}{2.487243in}}%
\pgfpathcurveto{\pgfqpoint{0.426907in}{2.479007in}}{\pgfqpoint{0.430179in}{2.471107in}}{\pgfqpoint{0.436003in}{2.465283in}}%
\pgfpathcurveto{\pgfqpoint{0.441827in}{2.459459in}}{\pgfqpoint{0.449727in}{2.456187in}}{\pgfqpoint{0.457963in}{2.456187in}}%
\pgfpathclose%
\pgfusepath{stroke,fill}%
\end{pgfscope}%
\begin{pgfscope}%
\pgfpathrectangle{\pgfqpoint{0.457963in}{0.528059in}}{\pgfqpoint{6.200000in}{2.285714in}} %
\pgfusepath{clip}%
\pgfsetbuttcap%
\pgfsetroundjoin%
\definecolor{currentfill}{rgb}{1.000000,0.000000,0.000000}%
\pgfsetfillcolor{currentfill}%
\pgfsetlinewidth{1.003750pt}%
\definecolor{currentstroke}{rgb}{1.000000,0.000000,0.000000}%
\pgfsetstrokecolor{currentstroke}%
\pgfsetdash{}{0pt}%
\pgfpathmoveto{\pgfqpoint{0.457963in}{2.456187in}}%
\pgfpathcurveto{\pgfqpoint{0.466200in}{2.456187in}}{\pgfqpoint{0.474100in}{2.459459in}}{\pgfqpoint{0.479924in}{2.465283in}}%
\pgfpathcurveto{\pgfqpoint{0.485748in}{2.471107in}}{\pgfqpoint{0.489020in}{2.479007in}}{\pgfqpoint{0.489020in}{2.487243in}}%
\pgfpathcurveto{\pgfqpoint{0.489020in}{2.495479in}}{\pgfqpoint{0.485748in}{2.503379in}}{\pgfqpoint{0.479924in}{2.509203in}}%
\pgfpathcurveto{\pgfqpoint{0.474100in}{2.515027in}}{\pgfqpoint{0.466200in}{2.518300in}}{\pgfqpoint{0.457963in}{2.518300in}}%
\pgfpathcurveto{\pgfqpoint{0.449727in}{2.518300in}}{\pgfqpoint{0.441827in}{2.515027in}}{\pgfqpoint{0.436003in}{2.509203in}}%
\pgfpathcurveto{\pgfqpoint{0.430179in}{2.503379in}}{\pgfqpoint{0.426907in}{2.495479in}}{\pgfqpoint{0.426907in}{2.487243in}}%
\pgfpathcurveto{\pgfqpoint{0.426907in}{2.479007in}}{\pgfqpoint{0.430179in}{2.471107in}}{\pgfqpoint{0.436003in}{2.465283in}}%
\pgfpathcurveto{\pgfqpoint{0.441827in}{2.459459in}}{\pgfqpoint{0.449727in}{2.456187in}}{\pgfqpoint{0.457963in}{2.456187in}}%
\pgfpathclose%
\pgfusepath{stroke,fill}%
\end{pgfscope}%
\begin{pgfscope}%
\pgfpathrectangle{\pgfqpoint{0.457963in}{0.528059in}}{\pgfqpoint{6.200000in}{2.285714in}} %
\pgfusepath{clip}%
\pgfsetbuttcap%
\pgfsetroundjoin%
\definecolor{currentfill}{rgb}{1.000000,0.000000,0.000000}%
\pgfsetfillcolor{currentfill}%
\pgfsetlinewidth{1.003750pt}%
\definecolor{currentstroke}{rgb}{1.000000,0.000000,0.000000}%
\pgfsetstrokecolor{currentstroke}%
\pgfsetdash{}{0pt}%
\pgfpathmoveto{\pgfqpoint{0.478630in}{2.377819in}}%
\pgfpathcurveto{\pgfqpoint{0.486866in}{2.377819in}}{\pgfqpoint{0.494766in}{2.381092in}}{\pgfqpoint{0.500590in}{2.386916in}}%
\pgfpathcurveto{\pgfqpoint{0.506414in}{2.392739in}}{\pgfqpoint{0.509686in}{2.400639in}}{\pgfqpoint{0.509686in}{2.408876in}}%
\pgfpathcurveto{\pgfqpoint{0.509686in}{2.417112in}}{\pgfqpoint{0.506414in}{2.425012in}}{\pgfqpoint{0.500590in}{2.430836in}}%
\pgfpathcurveto{\pgfqpoint{0.494766in}{2.436660in}}{\pgfqpoint{0.486866in}{2.439932in}}{\pgfqpoint{0.478630in}{2.439932in}}%
\pgfpathcurveto{\pgfqpoint{0.470394in}{2.439932in}}{\pgfqpoint{0.462494in}{2.436660in}}{\pgfqpoint{0.456670in}{2.430836in}}%
\pgfpathcurveto{\pgfqpoint{0.450846in}{2.425012in}}{\pgfqpoint{0.447574in}{2.417112in}}{\pgfqpoint{0.447574in}{2.408876in}}%
\pgfpathcurveto{\pgfqpoint{0.447574in}{2.400639in}}{\pgfqpoint{0.450846in}{2.392739in}}{\pgfqpoint{0.456670in}{2.386916in}}%
\pgfpathcurveto{\pgfqpoint{0.462494in}{2.381092in}}{\pgfqpoint{0.470394in}{2.377819in}}{\pgfqpoint{0.478630in}{2.377819in}}%
\pgfpathclose%
\pgfusepath{stroke,fill}%
\end{pgfscope}%
\begin{pgfscope}%
\pgfpathrectangle{\pgfqpoint{0.457963in}{0.528059in}}{\pgfqpoint{6.200000in}{2.285714in}} %
\pgfusepath{clip}%
\pgfsetbuttcap%
\pgfsetroundjoin%
\definecolor{currentfill}{rgb}{1.000000,0.000000,0.000000}%
\pgfsetfillcolor{currentfill}%
\pgfsetlinewidth{1.003750pt}%
\definecolor{currentstroke}{rgb}{1.000000,0.000000,0.000000}%
\pgfsetstrokecolor{currentstroke}%
\pgfsetdash{}{0pt}%
\pgfpathmoveto{\pgfqpoint{0.499297in}{2.377819in}}%
\pgfpathcurveto{\pgfqpoint{0.507533in}{2.377819in}}{\pgfqpoint{0.515433in}{2.381092in}}{\pgfqpoint{0.521257in}{2.386916in}}%
\pgfpathcurveto{\pgfqpoint{0.527081in}{2.392739in}}{\pgfqpoint{0.530353in}{2.400639in}}{\pgfqpoint{0.530353in}{2.408876in}}%
\pgfpathcurveto{\pgfqpoint{0.530353in}{2.417112in}}{\pgfqpoint{0.527081in}{2.425012in}}{\pgfqpoint{0.521257in}{2.430836in}}%
\pgfpathcurveto{\pgfqpoint{0.515433in}{2.436660in}}{\pgfqpoint{0.507533in}{2.439932in}}{\pgfqpoint{0.499297in}{2.439932in}}%
\pgfpathcurveto{\pgfqpoint{0.491060in}{2.439932in}}{\pgfqpoint{0.483160in}{2.436660in}}{\pgfqpoint{0.477336in}{2.430836in}}%
\pgfpathcurveto{\pgfqpoint{0.471512in}{2.425012in}}{\pgfqpoint{0.468240in}{2.417112in}}{\pgfqpoint{0.468240in}{2.408876in}}%
\pgfpathcurveto{\pgfqpoint{0.468240in}{2.400639in}}{\pgfqpoint{0.471512in}{2.392739in}}{\pgfqpoint{0.477336in}{2.386916in}}%
\pgfpathcurveto{\pgfqpoint{0.483160in}{2.381092in}}{\pgfqpoint{0.491060in}{2.377819in}}{\pgfqpoint{0.499297in}{2.377819in}}%
\pgfpathclose%
\pgfusepath{stroke,fill}%
\end{pgfscope}%
\begin{pgfscope}%
\pgfpathrectangle{\pgfqpoint{0.457963in}{0.528059in}}{\pgfqpoint{6.200000in}{2.285714in}} %
\pgfusepath{clip}%
\pgfsetbuttcap%
\pgfsetroundjoin%
\definecolor{currentfill}{rgb}{1.000000,0.000000,0.000000}%
\pgfsetfillcolor{currentfill}%
\pgfsetlinewidth{1.003750pt}%
\definecolor{currentstroke}{rgb}{1.000000,0.000000,0.000000}%
\pgfsetstrokecolor{currentstroke}%
\pgfsetdash{}{0pt}%
\pgfpathmoveto{\pgfqpoint{0.654297in}{2.430064in}}%
\pgfpathcurveto{\pgfqpoint{0.662533in}{2.430064in}}{\pgfqpoint{0.670433in}{2.433336in}}{\pgfqpoint{0.676257in}{2.439160in}}%
\pgfpathcurveto{\pgfqpoint{0.682081in}{2.444984in}}{\pgfqpoint{0.685353in}{2.452884in}}{\pgfqpoint{0.685353in}{2.461121in}}%
\pgfpathcurveto{\pgfqpoint{0.685353in}{2.469357in}}{\pgfqpoint{0.682081in}{2.477257in}}{\pgfqpoint{0.676257in}{2.483081in}}%
\pgfpathcurveto{\pgfqpoint{0.670433in}{2.488905in}}{\pgfqpoint{0.662533in}{2.492177in}}{\pgfqpoint{0.654297in}{2.492177in}}%
\pgfpathcurveto{\pgfqpoint{0.646060in}{2.492177in}}{\pgfqpoint{0.638160in}{2.488905in}}{\pgfqpoint{0.632336in}{2.483081in}}%
\pgfpathcurveto{\pgfqpoint{0.626512in}{2.477257in}}{\pgfqpoint{0.623240in}{2.469357in}}{\pgfqpoint{0.623240in}{2.461121in}}%
\pgfpathcurveto{\pgfqpoint{0.623240in}{2.452884in}}{\pgfqpoint{0.626512in}{2.444984in}}{\pgfqpoint{0.632336in}{2.439160in}}%
\pgfpathcurveto{\pgfqpoint{0.638160in}{2.433336in}}{\pgfqpoint{0.646060in}{2.430064in}}{\pgfqpoint{0.654297in}{2.430064in}}%
\pgfpathclose%
\pgfusepath{stroke,fill}%
\end{pgfscope}%
\begin{pgfscope}%
\pgfpathrectangle{\pgfqpoint{0.457963in}{0.528059in}}{\pgfqpoint{6.200000in}{2.285714in}} %
\pgfusepath{clip}%
\pgfsetbuttcap%
\pgfsetroundjoin%
\definecolor{currentfill}{rgb}{1.000000,0.000000,0.000000}%
\pgfsetfillcolor{currentfill}%
\pgfsetlinewidth{1.003750pt}%
\definecolor{currentstroke}{rgb}{1.000000,0.000000,0.000000}%
\pgfsetstrokecolor{currentstroke}%
\pgfsetdash{}{0pt}%
\pgfpathmoveto{\pgfqpoint{0.685297in}{2.064350in}}%
\pgfpathcurveto{\pgfqpoint{0.693533in}{2.064350in}}{\pgfqpoint{0.701433in}{2.067622in}}{\pgfqpoint{0.707257in}{2.073446in}}%
\pgfpathcurveto{\pgfqpoint{0.713081in}{2.079270in}}{\pgfqpoint{0.716353in}{2.087170in}}{\pgfqpoint{0.716353in}{2.095406in}}%
\pgfpathcurveto{\pgfqpoint{0.716353in}{2.103643in}}{\pgfqpoint{0.713081in}{2.111543in}}{\pgfqpoint{0.707257in}{2.117367in}}%
\pgfpathcurveto{\pgfqpoint{0.701433in}{2.123191in}}{\pgfqpoint{0.693533in}{2.126463in}}{\pgfqpoint{0.685297in}{2.126463in}}%
\pgfpathcurveto{\pgfqpoint{0.677060in}{2.126463in}}{\pgfqpoint{0.669160in}{2.123191in}}{\pgfqpoint{0.663336in}{2.117367in}}%
\pgfpathcurveto{\pgfqpoint{0.657512in}{2.111543in}}{\pgfqpoint{0.654240in}{2.103643in}}{\pgfqpoint{0.654240in}{2.095406in}}%
\pgfpathcurveto{\pgfqpoint{0.654240in}{2.087170in}}{\pgfqpoint{0.657512in}{2.079270in}}{\pgfqpoint{0.663336in}{2.073446in}}%
\pgfpathcurveto{\pgfqpoint{0.669160in}{2.067622in}}{\pgfqpoint{0.677060in}{2.064350in}}{\pgfqpoint{0.685297in}{2.064350in}}%
\pgfpathclose%
\pgfusepath{stroke,fill}%
\end{pgfscope}%
\begin{pgfscope}%
\pgfpathrectangle{\pgfqpoint{0.457963in}{0.528059in}}{\pgfqpoint{6.200000in}{2.285714in}} %
\pgfusepath{clip}%
\pgfsetbuttcap%
\pgfsetroundjoin%
\definecolor{currentfill}{rgb}{1.000000,0.000000,0.000000}%
\pgfsetfillcolor{currentfill}%
\pgfsetlinewidth{1.003750pt}%
\definecolor{currentstroke}{rgb}{1.000000,0.000000,0.000000}%
\pgfsetstrokecolor{currentstroke}%
\pgfsetdash{}{0pt}%
\pgfpathmoveto{\pgfqpoint{0.685297in}{2.221085in}}%
\pgfpathcurveto{\pgfqpoint{0.693533in}{2.221085in}}{\pgfqpoint{0.701433in}{2.224357in}}{\pgfqpoint{0.707257in}{2.230181in}}%
\pgfpathcurveto{\pgfqpoint{0.713081in}{2.236005in}}{\pgfqpoint{0.716353in}{2.243905in}}{\pgfqpoint{0.716353in}{2.252141in}}%
\pgfpathcurveto{\pgfqpoint{0.716353in}{2.260377in}}{\pgfqpoint{0.713081in}{2.268277in}}{\pgfqpoint{0.707257in}{2.274101in}}%
\pgfpathcurveto{\pgfqpoint{0.701433in}{2.279925in}}{\pgfqpoint{0.693533in}{2.283198in}}{\pgfqpoint{0.685297in}{2.283198in}}%
\pgfpathcurveto{\pgfqpoint{0.677060in}{2.283198in}}{\pgfqpoint{0.669160in}{2.279925in}}{\pgfqpoint{0.663336in}{2.274101in}}%
\pgfpathcurveto{\pgfqpoint{0.657512in}{2.268277in}}{\pgfqpoint{0.654240in}{2.260377in}}{\pgfqpoint{0.654240in}{2.252141in}}%
\pgfpathcurveto{\pgfqpoint{0.654240in}{2.243905in}}{\pgfqpoint{0.657512in}{2.236005in}}{\pgfqpoint{0.663336in}{2.230181in}}%
\pgfpathcurveto{\pgfqpoint{0.669160in}{2.224357in}}{\pgfqpoint{0.677060in}{2.221085in}}{\pgfqpoint{0.685297in}{2.221085in}}%
\pgfpathclose%
\pgfusepath{stroke,fill}%
\end{pgfscope}%
\begin{pgfscope}%
\pgfpathrectangle{\pgfqpoint{0.457963in}{0.528059in}}{\pgfqpoint{6.200000in}{2.285714in}} %
\pgfusepath{clip}%
\pgfsetbuttcap%
\pgfsetroundjoin%
\definecolor{currentfill}{rgb}{1.000000,0.000000,0.000000}%
\pgfsetfillcolor{currentfill}%
\pgfsetlinewidth{1.003750pt}%
\definecolor{currentstroke}{rgb}{1.000000,0.000000,0.000000}%
\pgfsetstrokecolor{currentstroke}%
\pgfsetdash{}{0pt}%
\pgfpathmoveto{\pgfqpoint{0.943630in}{1.672513in}}%
\pgfpathcurveto{\pgfqpoint{0.951866in}{1.672513in}}{\pgfqpoint{0.959766in}{1.675785in}}{\pgfqpoint{0.965590in}{1.681609in}}%
\pgfpathcurveto{\pgfqpoint{0.971414in}{1.687433in}}{\pgfqpoint{0.974686in}{1.695333in}}{\pgfqpoint{0.974686in}{1.703570in}}%
\pgfpathcurveto{\pgfqpoint{0.974686in}{1.711806in}}{\pgfqpoint{0.971414in}{1.719706in}}{\pgfqpoint{0.965590in}{1.725530in}}%
\pgfpathcurveto{\pgfqpoint{0.959766in}{1.731354in}}{\pgfqpoint{0.951866in}{1.734626in}}{\pgfqpoint{0.943630in}{1.734626in}}%
\pgfpathcurveto{\pgfqpoint{0.935394in}{1.734626in}}{\pgfqpoint{0.927494in}{1.731354in}}{\pgfqpoint{0.921670in}{1.725530in}}%
\pgfpathcurveto{\pgfqpoint{0.915846in}{1.719706in}}{\pgfqpoint{0.912574in}{1.711806in}}{\pgfqpoint{0.912574in}{1.703570in}}%
\pgfpathcurveto{\pgfqpoint{0.912574in}{1.695333in}}{\pgfqpoint{0.915846in}{1.687433in}}{\pgfqpoint{0.921670in}{1.681609in}}%
\pgfpathcurveto{\pgfqpoint{0.927494in}{1.675785in}}{\pgfqpoint{0.935394in}{1.672513in}}{\pgfqpoint{0.943630in}{1.672513in}}%
\pgfpathclose%
\pgfusepath{stroke,fill}%
\end{pgfscope}%
\begin{pgfscope}%
\pgfpathrectangle{\pgfqpoint{0.457963in}{0.528059in}}{\pgfqpoint{6.200000in}{2.285714in}} %
\pgfusepath{clip}%
\pgfsetbuttcap%
\pgfsetroundjoin%
\definecolor{currentfill}{rgb}{1.000000,0.000000,0.000000}%
\pgfsetfillcolor{currentfill}%
\pgfsetlinewidth{1.003750pt}%
\definecolor{currentstroke}{rgb}{1.000000,0.000000,0.000000}%
\pgfsetstrokecolor{currentstroke}%
\pgfsetdash{}{0pt}%
\pgfpathmoveto{\pgfqpoint{0.953963in}{1.907615in}}%
\pgfpathcurveto{\pgfqpoint{0.962200in}{1.907615in}}{\pgfqpoint{0.970100in}{1.910887in}}{\pgfqpoint{0.975924in}{1.916711in}}%
\pgfpathcurveto{\pgfqpoint{0.981748in}{1.922535in}}{\pgfqpoint{0.985020in}{1.930435in}}{\pgfqpoint{0.985020in}{1.938672in}}%
\pgfpathcurveto{\pgfqpoint{0.985020in}{1.946908in}}{\pgfqpoint{0.981748in}{1.954808in}}{\pgfqpoint{0.975924in}{1.960632in}}%
\pgfpathcurveto{\pgfqpoint{0.970100in}{1.966456in}}{\pgfqpoint{0.962200in}{1.969728in}}{\pgfqpoint{0.953963in}{1.969728in}}%
\pgfpathcurveto{\pgfqpoint{0.945727in}{1.969728in}}{\pgfqpoint{0.937827in}{1.966456in}}{\pgfqpoint{0.932003in}{1.960632in}}%
\pgfpathcurveto{\pgfqpoint{0.926179in}{1.954808in}}{\pgfqpoint{0.922907in}{1.946908in}}{\pgfqpoint{0.922907in}{1.938672in}}%
\pgfpathcurveto{\pgfqpoint{0.922907in}{1.930435in}}{\pgfqpoint{0.926179in}{1.922535in}}{\pgfqpoint{0.932003in}{1.916711in}}%
\pgfpathcurveto{\pgfqpoint{0.937827in}{1.910887in}}{\pgfqpoint{0.945727in}{1.907615in}}{\pgfqpoint{0.953963in}{1.907615in}}%
\pgfpathclose%
\pgfusepath{stroke,fill}%
\end{pgfscope}%
\begin{pgfscope}%
\pgfpathrectangle{\pgfqpoint{0.457963in}{0.528059in}}{\pgfqpoint{6.200000in}{2.285714in}} %
\pgfusepath{clip}%
\pgfsetbuttcap%
\pgfsetroundjoin%
\definecolor{currentfill}{rgb}{1.000000,0.000000,0.000000}%
\pgfsetfillcolor{currentfill}%
\pgfsetlinewidth{1.003750pt}%
\definecolor{currentstroke}{rgb}{1.000000,0.000000,0.000000}%
\pgfsetstrokecolor{currentstroke}%
\pgfsetdash{}{0pt}%
\pgfpathmoveto{\pgfqpoint{1.108963in}{2.456187in}}%
\pgfpathcurveto{\pgfqpoint{1.117200in}{2.456187in}}{\pgfqpoint{1.125100in}{2.459459in}}{\pgfqpoint{1.130924in}{2.465283in}}%
\pgfpathcurveto{\pgfqpoint{1.136748in}{2.471107in}}{\pgfqpoint{1.140020in}{2.479007in}}{\pgfqpoint{1.140020in}{2.487243in}}%
\pgfpathcurveto{\pgfqpoint{1.140020in}{2.495479in}}{\pgfqpoint{1.136748in}{2.503379in}}{\pgfqpoint{1.130924in}{2.509203in}}%
\pgfpathcurveto{\pgfqpoint{1.125100in}{2.515027in}}{\pgfqpoint{1.117200in}{2.518300in}}{\pgfqpoint{1.108963in}{2.518300in}}%
\pgfpathcurveto{\pgfqpoint{1.100727in}{2.518300in}}{\pgfqpoint{1.092827in}{2.515027in}}{\pgfqpoint{1.087003in}{2.509203in}}%
\pgfpathcurveto{\pgfqpoint{1.081179in}{2.503379in}}{\pgfqpoint{1.077907in}{2.495479in}}{\pgfqpoint{1.077907in}{2.487243in}}%
\pgfpathcurveto{\pgfqpoint{1.077907in}{2.479007in}}{\pgfqpoint{1.081179in}{2.471107in}}{\pgfqpoint{1.087003in}{2.465283in}}%
\pgfpathcurveto{\pgfqpoint{1.092827in}{2.459459in}}{\pgfqpoint{1.100727in}{2.456187in}}{\pgfqpoint{1.108963in}{2.456187in}}%
\pgfpathclose%
\pgfusepath{stroke,fill}%
\end{pgfscope}%
\begin{pgfscope}%
\pgfpathrectangle{\pgfqpoint{0.457963in}{0.528059in}}{\pgfqpoint{6.200000in}{2.285714in}} %
\pgfusepath{clip}%
\pgfsetbuttcap%
\pgfsetroundjoin%
\definecolor{currentfill}{rgb}{1.000000,0.000000,0.000000}%
\pgfsetfillcolor{currentfill}%
\pgfsetlinewidth{1.003750pt}%
\definecolor{currentstroke}{rgb}{1.000000,0.000000,0.000000}%
\pgfsetstrokecolor{currentstroke}%
\pgfsetdash{}{0pt}%
\pgfpathmoveto{\pgfqpoint{1.501630in}{2.456187in}}%
\pgfpathcurveto{\pgfqpoint{1.509866in}{2.456187in}}{\pgfqpoint{1.517766in}{2.459459in}}{\pgfqpoint{1.523590in}{2.465283in}}%
\pgfpathcurveto{\pgfqpoint{1.529414in}{2.471107in}}{\pgfqpoint{1.532686in}{2.479007in}}{\pgfqpoint{1.532686in}{2.487243in}}%
\pgfpathcurveto{\pgfqpoint{1.532686in}{2.495479in}}{\pgfqpoint{1.529414in}{2.503379in}}{\pgfqpoint{1.523590in}{2.509203in}}%
\pgfpathcurveto{\pgfqpoint{1.517766in}{2.515027in}}{\pgfqpoint{1.509866in}{2.518300in}}{\pgfqpoint{1.501630in}{2.518300in}}%
\pgfpathcurveto{\pgfqpoint{1.493394in}{2.518300in}}{\pgfqpoint{1.485494in}{2.515027in}}{\pgfqpoint{1.479670in}{2.509203in}}%
\pgfpathcurveto{\pgfqpoint{1.473846in}{2.503379in}}{\pgfqpoint{1.470574in}{2.495479in}}{\pgfqpoint{1.470574in}{2.487243in}}%
\pgfpathcurveto{\pgfqpoint{1.470574in}{2.479007in}}{\pgfqpoint{1.473846in}{2.471107in}}{\pgfqpoint{1.479670in}{2.465283in}}%
\pgfpathcurveto{\pgfqpoint{1.485494in}{2.459459in}}{\pgfqpoint{1.493394in}{2.456187in}}{\pgfqpoint{1.501630in}{2.456187in}}%
\pgfpathclose%
\pgfusepath{stroke,fill}%
\end{pgfscope}%
\begin{pgfscope}%
\pgfpathrectangle{\pgfqpoint{0.457963in}{0.528059in}}{\pgfqpoint{6.200000in}{2.285714in}} %
\pgfusepath{clip}%
\pgfsetbuttcap%
\pgfsetroundjoin%
\definecolor{currentfill}{rgb}{1.000000,0.000000,0.000000}%
\pgfsetfillcolor{currentfill}%
\pgfsetlinewidth{1.003750pt}%
\definecolor{currentstroke}{rgb}{1.000000,0.000000,0.000000}%
\pgfsetstrokecolor{currentstroke}%
\pgfsetdash{}{0pt}%
\pgfpathmoveto{\pgfqpoint{1.553297in}{0.888840in}}%
\pgfpathcurveto{\pgfqpoint{1.561533in}{0.888840in}}{\pgfqpoint{1.569433in}{0.892112in}}{\pgfqpoint{1.575257in}{0.897936in}}%
\pgfpathcurveto{\pgfqpoint{1.581081in}{0.903760in}}{\pgfqpoint{1.584353in}{0.911660in}}{\pgfqpoint{1.584353in}{0.919896in}}%
\pgfpathcurveto{\pgfqpoint{1.584353in}{0.928132in}}{\pgfqpoint{1.581081in}{0.936033in}}{\pgfqpoint{1.575257in}{0.941856in}}%
\pgfpathcurveto{\pgfqpoint{1.569433in}{0.947680in}}{\pgfqpoint{1.561533in}{0.950953in}}{\pgfqpoint{1.553297in}{0.950953in}}%
\pgfpathcurveto{\pgfqpoint{1.545060in}{0.950953in}}{\pgfqpoint{1.537160in}{0.947680in}}{\pgfqpoint{1.531336in}{0.941856in}}%
\pgfpathcurveto{\pgfqpoint{1.525512in}{0.936033in}}{\pgfqpoint{1.522240in}{0.928132in}}{\pgfqpoint{1.522240in}{0.919896in}}%
\pgfpathcurveto{\pgfqpoint{1.522240in}{0.911660in}}{\pgfqpoint{1.525512in}{0.903760in}}{\pgfqpoint{1.531336in}{0.897936in}}%
\pgfpathcurveto{\pgfqpoint{1.537160in}{0.892112in}}{\pgfqpoint{1.545060in}{0.888840in}}{\pgfqpoint{1.553297in}{0.888840in}}%
\pgfpathclose%
\pgfusepath{stroke,fill}%
\end{pgfscope}%
\begin{pgfscope}%
\pgfpathrectangle{\pgfqpoint{0.457963in}{0.528059in}}{\pgfqpoint{6.200000in}{2.285714in}} %
\pgfusepath{clip}%
\pgfsetbuttcap%
\pgfsetroundjoin%
\definecolor{currentfill}{rgb}{1.000000,0.000000,0.000000}%
\pgfsetfillcolor{currentfill}%
\pgfsetlinewidth{1.003750pt}%
\definecolor{currentstroke}{rgb}{1.000000,0.000000,0.000000}%
\pgfsetstrokecolor{currentstroke}%
\pgfsetdash{}{0pt}%
\pgfpathmoveto{\pgfqpoint{1.801297in}{1.972921in}}%
\pgfpathcurveto{\pgfqpoint{1.809533in}{1.972921in}}{\pgfqpoint{1.817433in}{1.976194in}}{\pgfqpoint{1.823257in}{1.982018in}}%
\pgfpathcurveto{\pgfqpoint{1.829081in}{1.987841in}}{\pgfqpoint{1.832353in}{1.995742in}}{\pgfqpoint{1.832353in}{2.003978in}}%
\pgfpathcurveto{\pgfqpoint{1.832353in}{2.012214in}}{\pgfqpoint{1.829081in}{2.020114in}}{\pgfqpoint{1.823257in}{2.025938in}}%
\pgfpathcurveto{\pgfqpoint{1.817433in}{2.031762in}}{\pgfqpoint{1.809533in}{2.035034in}}{\pgfqpoint{1.801297in}{2.035034in}}%
\pgfpathcurveto{\pgfqpoint{1.793060in}{2.035034in}}{\pgfqpoint{1.785160in}{2.031762in}}{\pgfqpoint{1.779336in}{2.025938in}}%
\pgfpathcurveto{\pgfqpoint{1.773512in}{2.020114in}}{\pgfqpoint{1.770240in}{2.012214in}}{\pgfqpoint{1.770240in}{2.003978in}}%
\pgfpathcurveto{\pgfqpoint{1.770240in}{1.995742in}}{\pgfqpoint{1.773512in}{1.987841in}}{\pgfqpoint{1.779336in}{1.982018in}}%
\pgfpathcurveto{\pgfqpoint{1.785160in}{1.976194in}}{\pgfqpoint{1.793060in}{1.972921in}}{\pgfqpoint{1.801297in}{1.972921in}}%
\pgfpathclose%
\pgfusepath{stroke,fill}%
\end{pgfscope}%
\begin{pgfscope}%
\pgfpathrectangle{\pgfqpoint{0.457963in}{0.528059in}}{\pgfqpoint{6.200000in}{2.285714in}} %
\pgfusepath{clip}%
\pgfsetbuttcap%
\pgfsetroundjoin%
\definecolor{currentfill}{rgb}{1.000000,0.000000,0.000000}%
\pgfsetfillcolor{currentfill}%
\pgfsetlinewidth{1.003750pt}%
\definecolor{currentstroke}{rgb}{1.000000,0.000000,0.000000}%
\pgfsetstrokecolor{currentstroke}%
\pgfsetdash{}{0pt}%
\pgfpathmoveto{\pgfqpoint{1.904630in}{2.364758in}}%
\pgfpathcurveto{\pgfqpoint{1.912866in}{2.364758in}}{\pgfqpoint{1.920766in}{2.368030in}}{\pgfqpoint{1.926590in}{2.373854in}}%
\pgfpathcurveto{\pgfqpoint{1.932414in}{2.379678in}}{\pgfqpoint{1.935686in}{2.387578in}}{\pgfqpoint{1.935686in}{2.395815in}}%
\pgfpathcurveto{\pgfqpoint{1.935686in}{2.404051in}}{\pgfqpoint{1.932414in}{2.411951in}}{\pgfqpoint{1.926590in}{2.417775in}}%
\pgfpathcurveto{\pgfqpoint{1.920766in}{2.423599in}}{\pgfqpoint{1.912866in}{2.426871in}}{\pgfqpoint{1.904630in}{2.426871in}}%
\pgfpathcurveto{\pgfqpoint{1.896394in}{2.426871in}}{\pgfqpoint{1.888494in}{2.423599in}}{\pgfqpoint{1.882670in}{2.417775in}}%
\pgfpathcurveto{\pgfqpoint{1.876846in}{2.411951in}}{\pgfqpoint{1.873574in}{2.404051in}}{\pgfqpoint{1.873574in}{2.395815in}}%
\pgfpathcurveto{\pgfqpoint{1.873574in}{2.387578in}}{\pgfqpoint{1.876846in}{2.379678in}}{\pgfqpoint{1.882670in}{2.373854in}}%
\pgfpathcurveto{\pgfqpoint{1.888494in}{2.368030in}}{\pgfqpoint{1.896394in}{2.364758in}}{\pgfqpoint{1.904630in}{2.364758in}}%
\pgfpathclose%
\pgfusepath{stroke,fill}%
\end{pgfscope}%
\begin{pgfscope}%
\pgfpathrectangle{\pgfqpoint{0.457963in}{0.528059in}}{\pgfqpoint{6.200000in}{2.285714in}} %
\pgfusepath{clip}%
\pgfsetbuttcap%
\pgfsetroundjoin%
\definecolor{currentfill}{rgb}{1.000000,0.000000,0.000000}%
\pgfsetfillcolor{currentfill}%
\pgfsetlinewidth{1.003750pt}%
\definecolor{currentstroke}{rgb}{1.000000,0.000000,0.000000}%
\pgfsetstrokecolor{currentstroke}%
\pgfsetdash{}{0pt}%
\pgfpathmoveto{\pgfqpoint{1.935630in}{2.443125in}}%
\pgfpathcurveto{\pgfqpoint{1.943866in}{2.443125in}}{\pgfqpoint{1.951766in}{2.446398in}}{\pgfqpoint{1.957590in}{2.452222in}}%
\pgfpathcurveto{\pgfqpoint{1.963414in}{2.458046in}}{\pgfqpoint{1.966686in}{2.465946in}}{\pgfqpoint{1.966686in}{2.474182in}}%
\pgfpathcurveto{\pgfqpoint{1.966686in}{2.482418in}}{\pgfqpoint{1.963414in}{2.490318in}}{\pgfqpoint{1.957590in}{2.496142in}}%
\pgfpathcurveto{\pgfqpoint{1.951766in}{2.501966in}}{\pgfqpoint{1.943866in}{2.505238in}}{\pgfqpoint{1.935630in}{2.505238in}}%
\pgfpathcurveto{\pgfqpoint{1.927394in}{2.505238in}}{\pgfqpoint{1.919494in}{2.501966in}}{\pgfqpoint{1.913670in}{2.496142in}}%
\pgfpathcurveto{\pgfqpoint{1.907846in}{2.490318in}}{\pgfqpoint{1.904574in}{2.482418in}}{\pgfqpoint{1.904574in}{2.474182in}}%
\pgfpathcurveto{\pgfqpoint{1.904574in}{2.465946in}}{\pgfqpoint{1.907846in}{2.458046in}}{\pgfqpoint{1.913670in}{2.452222in}}%
\pgfpathcurveto{\pgfqpoint{1.919494in}{2.446398in}}{\pgfqpoint{1.927394in}{2.443125in}}{\pgfqpoint{1.935630in}{2.443125in}}%
\pgfpathclose%
\pgfusepath{stroke,fill}%
\end{pgfscope}%
\begin{pgfscope}%
\pgfpathrectangle{\pgfqpoint{0.457963in}{0.528059in}}{\pgfqpoint{6.200000in}{2.285714in}} %
\pgfusepath{clip}%
\pgfsetbuttcap%
\pgfsetroundjoin%
\definecolor{currentfill}{rgb}{1.000000,0.000000,0.000000}%
\pgfsetfillcolor{currentfill}%
\pgfsetlinewidth{1.003750pt}%
\definecolor{currentstroke}{rgb}{1.000000,0.000000,0.000000}%
\pgfsetstrokecolor{currentstroke}%
\pgfsetdash{}{0pt}%
\pgfpathmoveto{\pgfqpoint{2.018297in}{2.142717in}}%
\pgfpathcurveto{\pgfqpoint{2.026533in}{2.142717in}}{\pgfqpoint{2.034433in}{2.145990in}}{\pgfqpoint{2.040257in}{2.151813in}}%
\pgfpathcurveto{\pgfqpoint{2.046081in}{2.157637in}}{\pgfqpoint{2.049353in}{2.165537in}}{\pgfqpoint{2.049353in}{2.173774in}}%
\pgfpathcurveto{\pgfqpoint{2.049353in}{2.182010in}}{\pgfqpoint{2.046081in}{2.189910in}}{\pgfqpoint{2.040257in}{2.195734in}}%
\pgfpathcurveto{\pgfqpoint{2.034433in}{2.201558in}}{\pgfqpoint{2.026533in}{2.204830in}}{\pgfqpoint{2.018297in}{2.204830in}}%
\pgfpathcurveto{\pgfqpoint{2.010060in}{2.204830in}}{\pgfqpoint{2.002160in}{2.201558in}}{\pgfqpoint{1.996336in}{2.195734in}}%
\pgfpathcurveto{\pgfqpoint{1.990512in}{2.189910in}}{\pgfqpoint{1.987240in}{2.182010in}}{\pgfqpoint{1.987240in}{2.173774in}}%
\pgfpathcurveto{\pgfqpoint{1.987240in}{2.165537in}}{\pgfqpoint{1.990512in}{2.157637in}}{\pgfqpoint{1.996336in}{2.151813in}}%
\pgfpathcurveto{\pgfqpoint{2.002160in}{2.145990in}}{\pgfqpoint{2.010060in}{2.142717in}}{\pgfqpoint{2.018297in}{2.142717in}}%
\pgfpathclose%
\pgfusepath{stroke,fill}%
\end{pgfscope}%
\begin{pgfscope}%
\pgfpathrectangle{\pgfqpoint{0.457963in}{0.528059in}}{\pgfqpoint{6.200000in}{2.285714in}} %
\pgfusepath{clip}%
\pgfsetbuttcap%
\pgfsetroundjoin%
\definecolor{currentfill}{rgb}{1.000000,0.000000,0.000000}%
\pgfsetfillcolor{currentfill}%
\pgfsetlinewidth{1.003750pt}%
\definecolor{currentstroke}{rgb}{1.000000,0.000000,0.000000}%
\pgfsetstrokecolor{currentstroke}%
\pgfsetdash{}{0pt}%
\pgfpathmoveto{\pgfqpoint{2.090630in}{2.325574in}}%
\pgfpathcurveto{\pgfqpoint{2.098866in}{2.325574in}}{\pgfqpoint{2.106766in}{2.328847in}}{\pgfqpoint{2.112590in}{2.334671in}}%
\pgfpathcurveto{\pgfqpoint{2.118414in}{2.340495in}}{\pgfqpoint{2.121686in}{2.348395in}}{\pgfqpoint{2.121686in}{2.356631in}}%
\pgfpathcurveto{\pgfqpoint{2.121686in}{2.364867in}}{\pgfqpoint{2.118414in}{2.372767in}}{\pgfqpoint{2.112590in}{2.378591in}}%
\pgfpathcurveto{\pgfqpoint{2.106766in}{2.384415in}}{\pgfqpoint{2.098866in}{2.387687in}}{\pgfqpoint{2.090630in}{2.387687in}}%
\pgfpathcurveto{\pgfqpoint{2.082394in}{2.387687in}}{\pgfqpoint{2.074494in}{2.384415in}}{\pgfqpoint{2.068670in}{2.378591in}}%
\pgfpathcurveto{\pgfqpoint{2.062846in}{2.372767in}}{\pgfqpoint{2.059574in}{2.364867in}}{\pgfqpoint{2.059574in}{2.356631in}}%
\pgfpathcurveto{\pgfqpoint{2.059574in}{2.348395in}}{\pgfqpoint{2.062846in}{2.340495in}}{\pgfqpoint{2.068670in}{2.334671in}}%
\pgfpathcurveto{\pgfqpoint{2.074494in}{2.328847in}}{\pgfqpoint{2.082394in}{2.325574in}}{\pgfqpoint{2.090630in}{2.325574in}}%
\pgfpathclose%
\pgfusepath{stroke,fill}%
\end{pgfscope}%
\begin{pgfscope}%
\pgfpathrectangle{\pgfqpoint{0.457963in}{0.528059in}}{\pgfqpoint{6.200000in}{2.285714in}} %
\pgfusepath{clip}%
\pgfsetbuttcap%
\pgfsetroundjoin%
\definecolor{currentfill}{rgb}{1.000000,0.000000,0.000000}%
\pgfsetfillcolor{currentfill}%
\pgfsetlinewidth{1.003750pt}%
\definecolor{currentstroke}{rgb}{1.000000,0.000000,0.000000}%
\pgfsetstrokecolor{currentstroke}%
\pgfsetdash{}{0pt}%
\pgfpathmoveto{\pgfqpoint{2.410963in}{1.254554in}}%
\pgfpathcurveto{\pgfqpoint{2.419200in}{1.254554in}}{\pgfqpoint{2.427100in}{1.257826in}}{\pgfqpoint{2.432924in}{1.263650in}}%
\pgfpathcurveto{\pgfqpoint{2.438748in}{1.269474in}}{\pgfqpoint{2.442020in}{1.277374in}}{\pgfqpoint{2.442020in}{1.285610in}}%
\pgfpathcurveto{\pgfqpoint{2.442020in}{1.293847in}}{\pgfqpoint{2.438748in}{1.301747in}}{\pgfqpoint{2.432924in}{1.307571in}}%
\pgfpathcurveto{\pgfqpoint{2.427100in}{1.313395in}}{\pgfqpoint{2.419200in}{1.316667in}}{\pgfqpoint{2.410963in}{1.316667in}}%
\pgfpathcurveto{\pgfqpoint{2.402727in}{1.316667in}}{\pgfqpoint{2.394827in}{1.313395in}}{\pgfqpoint{2.389003in}{1.307571in}}%
\pgfpathcurveto{\pgfqpoint{2.383179in}{1.301747in}}{\pgfqpoint{2.379907in}{1.293847in}}{\pgfqpoint{2.379907in}{1.285610in}}%
\pgfpathcurveto{\pgfqpoint{2.379907in}{1.277374in}}{\pgfqpoint{2.383179in}{1.269474in}}{\pgfqpoint{2.389003in}{1.263650in}}%
\pgfpathcurveto{\pgfqpoint{2.394827in}{1.257826in}}{\pgfqpoint{2.402727in}{1.254554in}}{\pgfqpoint{2.410963in}{1.254554in}}%
\pgfpathclose%
\pgfusepath{stroke,fill}%
\end{pgfscope}%
\begin{pgfscope}%
\pgfpathrectangle{\pgfqpoint{0.457963in}{0.528059in}}{\pgfqpoint{6.200000in}{2.285714in}} %
\pgfusepath{clip}%
\pgfsetbuttcap%
\pgfsetroundjoin%
\definecolor{currentfill}{rgb}{1.000000,0.000000,0.000000}%
\pgfsetfillcolor{currentfill}%
\pgfsetlinewidth{1.003750pt}%
\definecolor{currentstroke}{rgb}{1.000000,0.000000,0.000000}%
\pgfsetstrokecolor{currentstroke}%
\pgfsetdash{}{0pt}%
\pgfpathmoveto{\pgfqpoint{3.258297in}{1.084758in}}%
\pgfpathcurveto{\pgfqpoint{3.266533in}{1.084758in}}{\pgfqpoint{3.274433in}{1.088030in}}{\pgfqpoint{3.280257in}{1.093854in}}%
\pgfpathcurveto{\pgfqpoint{3.286081in}{1.099678in}}{\pgfqpoint{3.289353in}{1.107578in}}{\pgfqpoint{3.289353in}{1.115815in}}%
\pgfpathcurveto{\pgfqpoint{3.289353in}{1.124051in}}{\pgfqpoint{3.286081in}{1.131951in}}{\pgfqpoint{3.280257in}{1.137775in}}%
\pgfpathcurveto{\pgfqpoint{3.274433in}{1.143599in}}{\pgfqpoint{3.266533in}{1.146871in}}{\pgfqpoint{3.258297in}{1.146871in}}%
\pgfpathcurveto{\pgfqpoint{3.250060in}{1.146871in}}{\pgfqpoint{3.242160in}{1.143599in}}{\pgfqpoint{3.236336in}{1.137775in}}%
\pgfpathcurveto{\pgfqpoint{3.230512in}{1.131951in}}{\pgfqpoint{3.227240in}{1.124051in}}{\pgfqpoint{3.227240in}{1.115815in}}%
\pgfpathcurveto{\pgfqpoint{3.227240in}{1.107578in}}{\pgfqpoint{3.230512in}{1.099678in}}{\pgfqpoint{3.236336in}{1.093854in}}%
\pgfpathcurveto{\pgfqpoint{3.242160in}{1.088030in}}{\pgfqpoint{3.250060in}{1.084758in}}{\pgfqpoint{3.258297in}{1.084758in}}%
\pgfpathclose%
\pgfusepath{stroke,fill}%
\end{pgfscope}%
\begin{pgfscope}%
\pgfpathrectangle{\pgfqpoint{0.457963in}{0.528059in}}{\pgfqpoint{6.200000in}{2.285714in}} %
\pgfusepath{clip}%
\pgfsetrectcap%
\pgfsetroundjoin%
\pgfsetlinewidth{1.003750pt}%
\definecolor{currentstroke}{rgb}{0.833333,0.833333,1.000000}%
\pgfsetstrokecolor{currentstroke}%
\pgfsetdash{}{0pt}%
\pgfpathmoveto{\pgfqpoint{0.457963in}{0.849333in}}%
\pgfpathlineto{\pgfqpoint{0.457963in}{0.849333in}}%
\pgfpathlineto{\pgfqpoint{0.457963in}{0.849333in}}%
\pgfpathlineto{\pgfqpoint{0.457963in}{0.849333in}}%
\pgfpathlineto{\pgfqpoint{0.457963in}{0.849333in}}%
\pgfpathlineto{\pgfqpoint{0.457963in}{0.849333in}}%
\pgfpathlineto{\pgfqpoint{0.468297in}{0.849442in}}%
\pgfpathlineto{\pgfqpoint{0.468297in}{0.849442in}}%
\pgfpathlineto{\pgfqpoint{0.519963in}{0.849812in}}%
\pgfpathlineto{\pgfqpoint{0.561297in}{0.849896in}}%
\pgfpathlineto{\pgfqpoint{0.602630in}{0.849792in}}%
\pgfpathlineto{\pgfqpoint{0.612963in}{0.849736in}}%
\pgfpathlineto{\pgfqpoint{0.612963in}{0.849736in}}%
\pgfpathlineto{\pgfqpoint{0.850630in}{0.845210in}}%
\pgfpathlineto{\pgfqpoint{0.922963in}{0.842596in}}%
\pgfpathlineto{\pgfqpoint{1.139963in}{0.831292in}}%
\pgfpathlineto{\pgfqpoint{1.170963in}{0.829253in}}%
\pgfpathlineto{\pgfqpoint{1.470630in}{0.804083in}}%
\pgfpathlineto{\pgfqpoint{1.821963in}{0.761963in}}%
\pgfpathlineto{\pgfqpoint{2.193963in}{0.702530in}}%
\pgfusepath{stroke}%
\end{pgfscope}%
\begin{pgfscope}%
\pgfpathrectangle{\pgfqpoint{0.457963in}{0.528059in}}{\pgfqpoint{6.200000in}{2.285714in}} %
\pgfusepath{clip}%
\pgfsetrectcap%
\pgfsetroundjoin%
\pgfsetlinewidth{1.003750pt}%
\definecolor{currentstroke}{rgb}{0.666667,0.666667,1.000000}%
\pgfsetstrokecolor{currentstroke}%
\pgfsetdash{}{0pt}%
\pgfpathmoveto{\pgfqpoint{0.457963in}{1.175970in}}%
\pgfpathlineto{\pgfqpoint{0.457963in}{1.175970in}}%
\pgfpathlineto{\pgfqpoint{0.457963in}{1.175970in}}%
\pgfpathlineto{\pgfqpoint{0.457963in}{1.175970in}}%
\pgfpathlineto{\pgfqpoint{0.457963in}{1.175970in}}%
\pgfpathlineto{\pgfqpoint{0.468297in}{1.175411in}}%
\pgfpathlineto{\pgfqpoint{0.530297in}{1.172023in}}%
\pgfpathlineto{\pgfqpoint{0.530297in}{1.172023in}}%
\pgfpathlineto{\pgfqpoint{0.540630in}{1.171453in}}%
\pgfpathlineto{\pgfqpoint{0.643963in}{1.165656in}}%
\pgfpathlineto{\pgfqpoint{0.695630in}{1.162694in}}%
\pgfpathlineto{\pgfqpoint{0.757630in}{1.159086in}}%
\pgfpathlineto{\pgfqpoint{0.933297in}{1.148534in}}%
\pgfpathlineto{\pgfqpoint{0.995297in}{1.144694in}}%
\pgfpathlineto{\pgfqpoint{1.232963in}{1.129417in}}%
\pgfpathlineto{\pgfqpoint{1.336297in}{1.122499in}}%
\pgfpathlineto{\pgfqpoint{1.728963in}{1.094685in}}%
\pgfpathlineto{\pgfqpoint{2.297297in}{1.050148in}}%
\pgfpathlineto{\pgfqpoint{3.516630in}{0.937518in}}%
\pgfpathlineto{\pgfqpoint{4.301963in}{0.852642in}}%
\pgfusepath{stroke}%
\end{pgfscope}%
\begin{pgfscope}%
\pgfpathrectangle{\pgfqpoint{0.457963in}{0.528059in}}{\pgfqpoint{6.200000in}{2.285714in}} %
\pgfusepath{clip}%
\pgfsetrectcap%
\pgfsetroundjoin%
\pgfsetlinewidth{1.003750pt}%
\definecolor{currentstroke}{rgb}{0.500000,0.500000,1.000000}%
\pgfsetstrokecolor{currentstroke}%
\pgfsetdash{}{0pt}%
\pgfpathmoveto{\pgfqpoint{0.457963in}{1.505083in}}%
\pgfpathlineto{\pgfqpoint{0.457963in}{1.505083in}}%
\pgfpathlineto{\pgfqpoint{0.468297in}{1.504467in}}%
\pgfpathlineto{\pgfqpoint{0.468297in}{1.504467in}}%
\pgfpathlineto{\pgfqpoint{0.478630in}{1.503850in}}%
\pgfpathlineto{\pgfqpoint{0.488963in}{1.503233in}}%
\pgfpathlineto{\pgfqpoint{0.509630in}{1.501996in}}%
\pgfpathlineto{\pgfqpoint{0.540630in}{1.500136in}}%
\pgfpathlineto{\pgfqpoint{0.623297in}{1.495149in}}%
\pgfpathlineto{\pgfqpoint{0.633630in}{1.494523in}}%
\pgfpathlineto{\pgfqpoint{0.664630in}{1.492641in}}%
\pgfpathlineto{\pgfqpoint{0.943630in}{1.475452in}}%
\pgfpathlineto{\pgfqpoint{0.953963in}{1.474807in}}%
\pgfpathlineto{\pgfqpoint{1.584297in}{1.434277in}}%
\pgfpathlineto{\pgfqpoint{1.759963in}{1.422572in}}%
\pgfpathlineto{\pgfqpoint{2.710630in}{1.356132in}}%
\pgfpathlineto{\pgfqpoint{2.751963in}{1.353125in}}%
\pgfpathlineto{\pgfqpoint{3.041297in}{1.331797in}}%
\pgfpathlineto{\pgfqpoint{4.384630in}{1.226432in}}%
\pgfpathlineto{\pgfqpoint{5.934630in}{1.091885in}}%
\pgfusepath{stroke}%
\end{pgfscope}%
\begin{pgfscope}%
\pgfpathrectangle{\pgfqpoint{0.457963in}{0.528059in}}{\pgfqpoint{6.200000in}{2.285714in}} %
\pgfusepath{clip}%
\pgfsetrectcap%
\pgfsetroundjoin%
\pgfsetlinewidth{1.003750pt}%
\definecolor{currentstroke}{rgb}{0.333333,0.333333,1.000000}%
\pgfsetstrokecolor{currentstroke}%
\pgfsetdash{}{0pt}%
\pgfpathmoveto{\pgfqpoint{0.457963in}{1.825344in}}%
\pgfpathlineto{\pgfqpoint{0.457963in}{1.825344in}}%
\pgfpathlineto{\pgfqpoint{0.457963in}{1.825344in}}%
\pgfpathlineto{\pgfqpoint{0.457963in}{1.825344in}}%
\pgfpathlineto{\pgfqpoint{0.468297in}{1.825184in}}%
\pgfpathlineto{\pgfqpoint{0.488963in}{1.824855in}}%
\pgfpathlineto{\pgfqpoint{0.550963in}{1.823786in}}%
\pgfpathlineto{\pgfqpoint{0.581963in}{1.823205in}}%
\pgfpathlineto{\pgfqpoint{0.685297in}{1.821046in}}%
\pgfpathlineto{\pgfqpoint{0.747297in}{1.819587in}}%
\pgfpathlineto{\pgfqpoint{1.077963in}{1.809725in}}%
\pgfpathlineto{\pgfqpoint{1.315630in}{1.800473in}}%
\pgfpathlineto{\pgfqpoint{1.532630in}{1.790447in}}%
\pgfpathlineto{\pgfqpoint{1.832297in}{1.774121in}}%
\pgfpathlineto{\pgfqpoint{2.565963in}{1.722011in}}%
\pgfpathlineto{\pgfqpoint{3.320297in}{1.650458in}}%
\pgfpathlineto{\pgfqpoint{4.281297in}{1.532905in}}%
\pgfpathlineto{\pgfqpoint{4.343297in}{1.524306in}}%
\pgfpathlineto{\pgfqpoint{4.890963in}{1.442995in}}%
\pgfpathlineto{\pgfqpoint{6.234297in}{1.202875in}}%
\pgfusepath{stroke}%
\end{pgfscope}%
\begin{pgfscope}%
\pgfpathrectangle{\pgfqpoint{0.457963in}{0.528059in}}{\pgfqpoint{6.200000in}{2.285714in}} %
\pgfusepath{clip}%
\pgfsetrectcap%
\pgfsetroundjoin%
\pgfsetlinewidth{1.003750pt}%
\definecolor{currentstroke}{rgb}{0.166667,0.166667,1.000000}%
\pgfsetstrokecolor{currentstroke}%
\pgfsetdash{}{0pt}%
\pgfpathmoveto{\pgfqpoint{0.457963in}{2.146310in}}%
\pgfpathlineto{\pgfqpoint{0.457963in}{2.146310in}}%
\pgfpathlineto{\pgfqpoint{0.457963in}{2.146310in}}%
\pgfpathlineto{\pgfqpoint{0.457963in}{2.146310in}}%
\pgfpathlineto{\pgfqpoint{0.478630in}{2.146970in}}%
\pgfpathlineto{\pgfqpoint{0.488963in}{2.147288in}}%
\pgfpathlineto{\pgfqpoint{0.581963in}{2.149762in}}%
\pgfpathlineto{\pgfqpoint{0.581963in}{2.149762in}}%
\pgfpathlineto{\pgfqpoint{0.685297in}{2.151705in}}%
\pgfpathlineto{\pgfqpoint{0.891963in}{2.153045in}}%
\pgfpathlineto{\pgfqpoint{1.212297in}{2.148409in}}%
\pgfpathlineto{\pgfqpoint{1.666963in}{2.127820in}}%
\pgfpathlineto{\pgfqpoint{1.677297in}{2.127161in}}%
\pgfpathlineto{\pgfqpoint{2.782963in}{2.007596in}}%
\pgfpathlineto{\pgfqpoint{3.030963in}{1.967431in}}%
\pgfpathlineto{\pgfqpoint{3.919630in}{1.783348in}}%
\pgfpathlineto{\pgfqpoint{4.064297in}{1.747438in}}%
\pgfpathlineto{\pgfqpoint{4.673963in}{1.577818in}}%
\pgfpathlineto{\pgfqpoint{4.963297in}{1.486979in}}%
\pgfpathlineto{\pgfqpoint{5.789963in}{1.190762in}}%
\pgfusepath{stroke}%
\end{pgfscope}%
\begin{pgfscope}%
\pgfpathrectangle{\pgfqpoint{0.457963in}{0.528059in}}{\pgfqpoint{6.200000in}{2.285714in}} %
\pgfusepath{clip}%
\pgfsetrectcap%
\pgfsetroundjoin%
\pgfsetlinewidth{1.003750pt}%
\definecolor{currentstroke}{rgb}{0.000000,0.000000,1.000000}%
\pgfsetstrokecolor{currentstroke}%
\pgfsetdash{}{0pt}%
\pgfpathmoveto{\pgfqpoint{0.457963in}{2.463057in}}%
\pgfpathlineto{\pgfqpoint{0.457963in}{2.463057in}}%
\pgfpathlineto{\pgfqpoint{0.457963in}{2.463057in}}%
\pgfpathlineto{\pgfqpoint{0.478630in}{2.464529in}}%
\pgfpathlineto{\pgfqpoint{0.519963in}{2.467317in}}%
\pgfpathlineto{\pgfqpoint{0.623297in}{2.473379in}}%
\pgfpathlineto{\pgfqpoint{0.716297in}{2.477726in}}%
\pgfpathlineto{\pgfqpoint{0.747297in}{2.478942in}}%
\pgfpathlineto{\pgfqpoint{0.902297in}{2.483269in}}%
\pgfpathlineto{\pgfqpoint{1.181297in}{2.483702in}}%
\pgfpathlineto{\pgfqpoint{1.429297in}{2.476148in}}%
\pgfpathlineto{\pgfqpoint{1.429297in}{2.476148in}}%
\pgfpathlineto{\pgfqpoint{1.687630in}{2.460334in}}%
\pgfpathlineto{\pgfqpoint{3.113630in}{2.227153in}}%
\pgfpathlineto{\pgfqpoint{3.526963in}{2.113385in}}%
\pgfpathlineto{\pgfqpoint{3.712963in}{2.055419in}}%
\pgfpathlineto{\pgfqpoint{4.818630in}{1.624093in}}%
\pgfpathlineto{\pgfqpoint{4.921963in}{1.576194in}}%
\pgfpathlineto{\pgfqpoint{5.014963in}{1.531976in}}%
\pgfpathlineto{\pgfqpoint{5.469630in}{1.300672in}}%
\pgfusepath{stroke}%
\end{pgfscope}%
\begin{pgfscope}%
\pgfpathrectangle{\pgfqpoint{0.457963in}{0.528059in}}{\pgfqpoint{6.200000in}{2.285714in}} %
\pgfusepath{clip}%
\pgfsetrectcap%
\pgfsetroundjoin%
\pgfsetlinewidth{1.003750pt}%
\definecolor{currentstroke}{rgb}{1.000000,0.833333,0.833333}%
\pgfsetstrokecolor{currentstroke}%
\pgfsetdash{}{0pt}%
\pgfpathmoveto{\pgfqpoint{0.457963in}{0.848447in}}%
\pgfpathlineto{\pgfqpoint{0.457963in}{0.848447in}}%
\pgfpathlineto{\pgfqpoint{0.457963in}{0.848447in}}%
\pgfpathlineto{\pgfqpoint{0.457963in}{0.848447in}}%
\pgfpathlineto{\pgfqpoint{0.457963in}{0.848447in}}%
\pgfpathlineto{\pgfqpoint{0.457963in}{0.848447in}}%
\pgfpathlineto{\pgfqpoint{0.468297in}{0.847203in}}%
\pgfpathlineto{\pgfqpoint{0.478630in}{0.845968in}}%
\pgfpathlineto{\pgfqpoint{0.499297in}{0.843521in}}%
\pgfpathlineto{\pgfqpoint{0.519963in}{0.841105in}}%
\pgfpathlineto{\pgfqpoint{0.519963in}{0.841105in}}%
\pgfpathlineto{\pgfqpoint{0.550963in}{0.837542in}}%
\pgfpathlineto{\pgfqpoint{0.612963in}{0.830631in}}%
\pgfpathlineto{\pgfqpoint{0.984963in}{0.795177in}}%
\pgfpathlineto{\pgfqpoint{1.098630in}{0.786401in}}%
\pgfpathlineto{\pgfqpoint{1.305297in}{0.772912in}}%
\pgfpathlineto{\pgfqpoint{1.584297in}{0.759750in}}%
\pgfpathlineto{\pgfqpoint{1.656630in}{0.757285in}}%
\pgfpathlineto{\pgfqpoint{1.677297in}{0.756652in}}%
\pgfpathlineto{\pgfqpoint{1.759963in}{0.754439in}}%
\pgfusepath{stroke}%
\end{pgfscope}%
\begin{pgfscope}%
\pgfpathrectangle{\pgfqpoint{0.457963in}{0.528059in}}{\pgfqpoint{6.200000in}{2.285714in}} %
\pgfusepath{clip}%
\pgfsetrectcap%
\pgfsetroundjoin%
\pgfsetlinewidth{1.003750pt}%
\definecolor{currentstroke}{rgb}{1.000000,0.666667,0.666667}%
\pgfsetstrokecolor{currentstroke}%
\pgfsetdash{}{0pt}%
\pgfpathmoveto{\pgfqpoint{0.457963in}{1.145949in}}%
\pgfpathlineto{\pgfqpoint{0.457963in}{1.145949in}}%
\pgfpathlineto{\pgfqpoint{0.457963in}{1.145949in}}%
\pgfpathlineto{\pgfqpoint{0.457963in}{1.145949in}}%
\pgfpathlineto{\pgfqpoint{0.457963in}{1.145949in}}%
\pgfpathlineto{\pgfqpoint{0.468297in}{1.145881in}}%
\pgfpathlineto{\pgfqpoint{0.561297in}{1.144625in}}%
\pgfpathlineto{\pgfqpoint{0.664630in}{1.141870in}}%
\pgfpathlineto{\pgfqpoint{0.664630in}{1.141870in}}%
\pgfpathlineto{\pgfqpoint{0.695630in}{1.140764in}}%
\pgfpathlineto{\pgfqpoint{0.695630in}{1.140764in}}%
\pgfpathlineto{\pgfqpoint{0.695630in}{1.140764in}}%
\pgfpathlineto{\pgfqpoint{0.736963in}{1.139089in}}%
\pgfpathlineto{\pgfqpoint{0.953963in}{1.126541in}}%
\pgfpathlineto{\pgfqpoint{1.222630in}{1.102260in}}%
\pgfpathlineto{\pgfqpoint{1.615297in}{1.049371in}}%
\pgfpathlineto{\pgfqpoint{1.708297in}{1.033818in}}%
\pgfpathlineto{\pgfqpoint{2.100963in}{0.955368in}}%
\pgfpathlineto{\pgfqpoint{2.142297in}{0.945908in}}%
\pgfpathlineto{\pgfqpoint{2.576297in}{0.832753in}}%
\pgfusepath{stroke}%
\end{pgfscope}%
\begin{pgfscope}%
\pgfpathrectangle{\pgfqpoint{0.457963in}{0.528059in}}{\pgfqpoint{6.200000in}{2.285714in}} %
\pgfusepath{clip}%
\pgfsetrectcap%
\pgfsetroundjoin%
\pgfsetlinewidth{1.003750pt}%
\definecolor{currentstroke}{rgb}{1.000000,0.500000,0.500000}%
\pgfsetstrokecolor{currentstroke}%
\pgfsetdash{}{0pt}%
\pgfpathmoveto{\pgfqpoint{0.457963in}{1.470850in}}%
\pgfpathlineto{\pgfqpoint{0.457963in}{1.470850in}}%
\pgfpathlineto{\pgfqpoint{0.457963in}{1.470850in}}%
\pgfpathlineto{\pgfqpoint{0.468297in}{1.468501in}}%
\pgfpathlineto{\pgfqpoint{0.468297in}{1.468501in}}%
\pgfpathlineto{\pgfqpoint{0.478630in}{1.466160in}}%
\pgfpathlineto{\pgfqpoint{0.488963in}{1.463827in}}%
\pgfpathlineto{\pgfqpoint{0.499297in}{1.461502in}}%
\pgfpathlineto{\pgfqpoint{0.550963in}{1.449998in}}%
\pgfpathlineto{\pgfqpoint{0.705963in}{1.416687in}}%
\pgfpathlineto{\pgfqpoint{0.788630in}{1.399660in}}%
\pgfpathlineto{\pgfqpoint{0.829963in}{1.391338in}}%
\pgfpathlineto{\pgfqpoint{1.036630in}{1.351657in}}%
\pgfpathlineto{\pgfqpoint{1.160630in}{1.329387in}}%
\pgfpathlineto{\pgfqpoint{1.294963in}{1.306565in}}%
\pgfpathlineto{\pgfqpoint{1.532630in}{1.269508in}}%
\pgfpathlineto{\pgfqpoint{1.821963in}{1.230122in}}%
\pgfpathlineto{\pgfqpoint{1.925297in}{1.217579in}}%
\pgfpathlineto{\pgfqpoint{2.565963in}{1.157713in}}%
\pgfpathlineto{\pgfqpoint{2.968963in}{1.135851in}}%
\pgfusepath{stroke}%
\end{pgfscope}%
\begin{pgfscope}%
\pgfpathrectangle{\pgfqpoint{0.457963in}{0.528059in}}{\pgfqpoint{6.200000in}{2.285714in}} %
\pgfusepath{clip}%
\pgfsetrectcap%
\pgfsetroundjoin%
\pgfsetlinewidth{1.003750pt}%
\definecolor{currentstroke}{rgb}{1.000000,0.333333,0.333333}%
\pgfsetstrokecolor{currentstroke}%
\pgfsetdash{}{0pt}%
\pgfpathmoveto{\pgfqpoint{0.457963in}{1.813779in}}%
\pgfpathlineto{\pgfqpoint{0.457963in}{1.813779in}}%
\pgfpathlineto{\pgfqpoint{0.457963in}{1.813779in}}%
\pgfpathlineto{\pgfqpoint{0.457963in}{1.813779in}}%
\pgfpathlineto{\pgfqpoint{0.457963in}{1.813779in}}%
\pgfpathlineto{\pgfqpoint{0.488963in}{1.804241in}}%
\pgfpathlineto{\pgfqpoint{0.530297in}{1.791581in}}%
\pgfpathlineto{\pgfqpoint{0.540630in}{1.788426in}}%
\pgfpathlineto{\pgfqpoint{0.581963in}{1.775847in}}%
\pgfpathlineto{\pgfqpoint{0.592297in}{1.772712in}}%
\pgfpathlineto{\pgfqpoint{0.643963in}{1.757100in}}%
\pgfpathlineto{\pgfqpoint{0.881630in}{1.686598in}}%
\pgfpathlineto{\pgfqpoint{1.139963in}{1.612410in}}%
\pgfpathlineto{\pgfqpoint{1.305297in}{1.566267in}}%
\pgfpathlineto{\pgfqpoint{1.677297in}{1.466257in}}%
\pgfpathlineto{\pgfqpoint{1.945963in}{1.397312in}}%
\pgfpathlineto{\pgfqpoint{2.038963in}{1.374089in}}%
\pgfpathlineto{\pgfqpoint{2.152630in}{1.346152in}}%
\pgfpathlineto{\pgfqpoint{2.204297in}{1.333617in}}%
\pgfpathlineto{\pgfqpoint{3.630297in}{1.027847in}}%
\pgfusepath{stroke}%
\end{pgfscope}%
\begin{pgfscope}%
\pgfpathrectangle{\pgfqpoint{0.457963in}{0.528059in}}{\pgfqpoint{6.200000in}{2.285714in}} %
\pgfusepath{clip}%
\pgfsetrectcap%
\pgfsetroundjoin%
\pgfsetlinewidth{1.003750pt}%
\definecolor{currentstroke}{rgb}{1.000000,0.166667,0.166667}%
\pgfsetstrokecolor{currentstroke}%
\pgfsetdash{}{0pt}%
\pgfpathmoveto{\pgfqpoint{0.457963in}{2.089192in}}%
\pgfpathlineto{\pgfqpoint{0.457963in}{2.089192in}}%
\pgfpathlineto{\pgfqpoint{0.457963in}{2.089192in}}%
\pgfpathlineto{\pgfqpoint{0.457963in}{2.089192in}}%
\pgfpathlineto{\pgfqpoint{0.457963in}{2.089192in}}%
\pgfpathlineto{\pgfqpoint{0.468297in}{2.087638in}}%
\pgfpathlineto{\pgfqpoint{0.519963in}{2.079683in}}%
\pgfpathlineto{\pgfqpoint{0.530297in}{2.078056in}}%
\pgfpathlineto{\pgfqpoint{0.571630in}{2.071423in}}%
\pgfpathlineto{\pgfqpoint{0.571630in}{2.071423in}}%
\pgfpathlineto{\pgfqpoint{0.674963in}{2.053987in}}%
\pgfpathlineto{\pgfqpoint{0.860963in}{2.019523in}}%
\pgfpathlineto{\pgfqpoint{1.222630in}{1.941178in}}%
\pgfpathlineto{\pgfqpoint{1.387963in}{1.900378in}}%
\pgfpathlineto{\pgfqpoint{1.460297in}{1.881545in}}%
\pgfpathlineto{\pgfqpoint{1.790963in}{1.787827in}}%
\pgfpathlineto{\pgfqpoint{2.028630in}{1.712740in}}%
\pgfpathlineto{\pgfqpoint{2.090630in}{1.692089in}}%
\pgfpathlineto{\pgfqpoint{2.400630in}{1.582238in}}%
\pgfpathlineto{\pgfqpoint{3.361630in}{1.171825in}}%
\pgfusepath{stroke}%
\end{pgfscope}%
\begin{pgfscope}%
\pgfpathrectangle{\pgfqpoint{0.457963in}{0.528059in}}{\pgfqpoint{6.200000in}{2.285714in}} %
\pgfusepath{clip}%
\pgfsetrectcap%
\pgfsetroundjoin%
\pgfsetlinewidth{1.003750pt}%
\definecolor{currentstroke}{rgb}{1.000000,0.000000,0.000000}%
\pgfsetstrokecolor{currentstroke}%
\pgfsetdash{}{0pt}%
\pgfpathmoveto{\pgfqpoint{0.457963in}{2.380139in}}%
\pgfpathlineto{\pgfqpoint{0.457963in}{2.380139in}}%
\pgfpathlineto{\pgfqpoint{0.457963in}{2.380139in}}%
\pgfpathlineto{\pgfqpoint{0.457963in}{2.380139in}}%
\pgfpathlineto{\pgfqpoint{0.468297in}{2.377291in}}%
\pgfpathlineto{\pgfqpoint{0.509630in}{2.365804in}}%
\pgfpathlineto{\pgfqpoint{0.540630in}{2.357087in}}%
\pgfpathlineto{\pgfqpoint{0.747297in}{2.296743in}}%
\pgfpathlineto{\pgfqpoint{0.809297in}{2.277883in}}%
\pgfpathlineto{\pgfqpoint{0.891963in}{2.252194in}}%
\pgfpathlineto{\pgfqpoint{0.984963in}{2.222551in}}%
\pgfpathlineto{\pgfqpoint{1.057297in}{2.198953in}}%
\pgfpathlineto{\pgfqpoint{1.315630in}{2.110792in}}%
\pgfpathlineto{\pgfqpoint{1.449963in}{2.062553in}}%
\pgfpathlineto{\pgfqpoint{1.460297in}{2.058775in}}%
\pgfpathlineto{\pgfqpoint{1.821963in}{1.920413in}}%
\pgfpathlineto{\pgfqpoint{2.204297in}{1.761226in}}%
\pgfpathlineto{\pgfqpoint{2.214630in}{1.756740in}}%
\pgfpathlineto{\pgfqpoint{2.865630in}{1.454528in}}%
\pgfpathlineto{\pgfqpoint{3.258297in}{1.253629in}}%
\pgfusepath{stroke}%
\end{pgfscope}%
\begin{pgfscope}%
\pgfpathrectangle{\pgfqpoint{0.457963in}{0.528059in}}{\pgfqpoint{6.200000in}{2.285714in}} %
\pgfusepath{clip}%
\pgfsetrectcap%
\pgfsetroundjoin%
\pgfsetlinewidth{1.003750pt}%
\definecolor{currentstroke}{rgb}{0.833333,0.833333,1.000000}%
\pgfsetstrokecolor{currentstroke}%
\pgfsetdash{}{0pt}%
\pgfpathmoveto{\pgfqpoint{0.457963in}{0.847427in}}%
\pgfpathlineto{\pgfqpoint{0.457963in}{0.847427in}}%
\pgfpathlineto{\pgfqpoint{0.457963in}{0.847427in}}%
\pgfpathlineto{\pgfqpoint{0.468297in}{0.847449in}}%
\pgfpathlineto{\pgfqpoint{0.468297in}{0.847449in}}%
\pgfpathlineto{\pgfqpoint{0.499297in}{0.847439in}}%
\pgfpathlineto{\pgfqpoint{0.499297in}{0.847439in}}%
\pgfpathlineto{\pgfqpoint{0.530297in}{0.847314in}}%
\pgfpathlineto{\pgfqpoint{0.561297in}{0.847075in}}%
\pgfpathlineto{\pgfqpoint{0.643963in}{0.845877in}}%
\pgfpathlineto{\pgfqpoint{0.767963in}{0.842553in}}%
\pgfpathlineto{\pgfqpoint{0.860963in}{0.838856in}}%
\pgfpathlineto{\pgfqpoint{0.943630in}{0.834705in}}%
\pgfpathlineto{\pgfqpoint{0.953963in}{0.834129in}}%
\pgfpathlineto{\pgfqpoint{1.098630in}{0.824726in}}%
\pgfpathlineto{\pgfqpoint{1.222630in}{0.814680in}}%
\pgfpathlineto{\pgfqpoint{1.356963in}{0.801729in}}%
\pgfpathlineto{\pgfqpoint{1.449963in}{0.791503in}}%
\pgfpathlineto{\pgfqpoint{1.573963in}{0.776264in}}%
\pgfpathlineto{\pgfqpoint{2.193963in}{0.672575in}}%
\pgfusepath{stroke}%
\end{pgfscope}%
\begin{pgfscope}%
\pgfpathrectangle{\pgfqpoint{0.457963in}{0.528059in}}{\pgfqpoint{6.200000in}{2.285714in}} %
\pgfusepath{clip}%
\pgfsetrectcap%
\pgfsetroundjoin%
\pgfsetlinewidth{1.003750pt}%
\definecolor{currentstroke}{rgb}{0.666667,0.666667,1.000000}%
\pgfsetstrokecolor{currentstroke}%
\pgfsetdash{}{0pt}%
\pgfpathmoveto{\pgfqpoint{0.457963in}{1.178291in}}%
\pgfpathlineto{\pgfqpoint{0.457963in}{1.178291in}}%
\pgfpathlineto{\pgfqpoint{0.457963in}{1.178291in}}%
\pgfpathlineto{\pgfqpoint{0.457963in}{1.178291in}}%
\pgfpathlineto{\pgfqpoint{0.468297in}{1.177969in}}%
\pgfpathlineto{\pgfqpoint{0.488963in}{1.177321in}}%
\pgfpathlineto{\pgfqpoint{0.519963in}{1.176334in}}%
\pgfpathlineto{\pgfqpoint{0.540630in}{1.175668in}}%
\pgfpathlineto{\pgfqpoint{0.561297in}{1.174994in}}%
\pgfpathlineto{\pgfqpoint{0.674963in}{1.171159in}}%
\pgfpathlineto{\pgfqpoint{0.695630in}{1.170438in}}%
\pgfpathlineto{\pgfqpoint{0.726630in}{1.169344in}}%
\pgfpathlineto{\pgfqpoint{1.098630in}{1.154947in}}%
\pgfpathlineto{\pgfqpoint{1.325963in}{1.145000in}}%
\pgfpathlineto{\pgfqpoint{1.925297in}{1.114600in}}%
\pgfpathlineto{\pgfqpoint{1.935630in}{1.114023in}}%
\pgfpathlineto{\pgfqpoint{2.772630in}{1.061290in}}%
\pgfpathlineto{\pgfqpoint{2.782963in}{1.060565in}}%
\pgfpathlineto{\pgfqpoint{3.237630in}{1.026890in}}%
\pgfpathlineto{\pgfqpoint{3.537297in}{1.002790in}}%
\pgfusepath{stroke}%
\end{pgfscope}%
\begin{pgfscope}%
\pgfpathrectangle{\pgfqpoint{0.457963in}{0.528059in}}{\pgfqpoint{6.200000in}{2.285714in}} %
\pgfusepath{clip}%
\pgfsetrectcap%
\pgfsetroundjoin%
\pgfsetlinewidth{1.003750pt}%
\definecolor{currentstroke}{rgb}{0.500000,0.500000,1.000000}%
\pgfsetstrokecolor{currentstroke}%
\pgfsetdash{}{0pt}%
\pgfpathmoveto{\pgfqpoint{0.457963in}{1.497677in}}%
\pgfpathlineto{\pgfqpoint{0.457963in}{1.497677in}}%
\pgfpathlineto{\pgfqpoint{0.478630in}{1.497368in}}%
\pgfpathlineto{\pgfqpoint{0.519963in}{1.496712in}}%
\pgfpathlineto{\pgfqpoint{0.519963in}{1.496712in}}%
\pgfpathlineto{\pgfqpoint{0.561297in}{1.496008in}}%
\pgfpathlineto{\pgfqpoint{0.581963in}{1.495637in}}%
\pgfpathlineto{\pgfqpoint{0.643963in}{1.494451in}}%
\pgfpathlineto{\pgfqpoint{0.767963in}{1.491747in}}%
\pgfpathlineto{\pgfqpoint{0.995297in}{1.485642in}}%
\pgfpathlineto{\pgfqpoint{1.026297in}{1.484694in}}%
\pgfpathlineto{\pgfqpoint{1.325963in}{1.474107in}}%
\pgfpathlineto{\pgfqpoint{1.387963in}{1.471594in}}%
\pgfpathlineto{\pgfqpoint{1.697963in}{1.457371in}}%
\pgfpathlineto{\pgfqpoint{2.565963in}{1.402841in}}%
\pgfpathlineto{\pgfqpoint{2.937963in}{1.372837in}}%
\pgfpathlineto{\pgfqpoint{3.278963in}{1.341836in}}%
\pgfpathlineto{\pgfqpoint{3.671630in}{1.301995in}}%
\pgfpathlineto{\pgfqpoint{4.570630in}{1.194078in}}%
\pgfpathlineto{\pgfqpoint{5.510963in}{1.056325in}}%
\pgfusepath{stroke}%
\end{pgfscope}%
\begin{pgfscope}%
\pgfpathrectangle{\pgfqpoint{0.457963in}{0.528059in}}{\pgfqpoint{6.200000in}{2.285714in}} %
\pgfusepath{clip}%
\pgfsetrectcap%
\pgfsetroundjoin%
\pgfsetlinewidth{1.003750pt}%
\definecolor{currentstroke}{rgb}{0.333333,0.333333,1.000000}%
\pgfsetstrokecolor{currentstroke}%
\pgfsetdash{}{0pt}%
\pgfpathmoveto{\pgfqpoint{0.457963in}{1.821764in}}%
\pgfpathlineto{\pgfqpoint{0.457963in}{1.821764in}}%
\pgfpathlineto{\pgfqpoint{0.478630in}{1.821732in}}%
\pgfpathlineto{\pgfqpoint{0.519963in}{1.821607in}}%
\pgfpathlineto{\pgfqpoint{0.571630in}{1.821335in}}%
\pgfpathlineto{\pgfqpoint{0.643963in}{1.820741in}}%
\pgfpathlineto{\pgfqpoint{0.664630in}{1.820526in}}%
\pgfpathlineto{\pgfqpoint{0.747297in}{1.819459in}}%
\pgfpathlineto{\pgfqpoint{0.995297in}{1.814297in}}%
\pgfpathlineto{\pgfqpoint{1.036630in}{1.813151in}}%
\pgfpathlineto{\pgfqpoint{1.150297in}{1.809578in}}%
\pgfpathlineto{\pgfqpoint{2.007963in}{1.762706in}}%
\pgfpathlineto{\pgfqpoint{2.193963in}{1.747900in}}%
\pgfpathlineto{\pgfqpoint{2.266297in}{1.741696in}}%
\pgfpathlineto{\pgfqpoint{2.948297in}{1.670901in}}%
\pgfpathlineto{\pgfqpoint{3.299630in}{1.625752in}}%
\pgfpathlineto{\pgfqpoint{3.702630in}{1.566698in}}%
\pgfpathlineto{\pgfqpoint{4.115963in}{1.498063in}}%
\pgfpathlineto{\pgfqpoint{4.673963in}{1.392451in}}%
\pgfpathlineto{\pgfqpoint{5.986297in}{1.085395in}}%
\pgfusepath{stroke}%
\end{pgfscope}%
\begin{pgfscope}%
\pgfpathrectangle{\pgfqpoint{0.457963in}{0.528059in}}{\pgfqpoint{6.200000in}{2.285714in}} %
\pgfusepath{clip}%
\pgfsetrectcap%
\pgfsetroundjoin%
\pgfsetlinewidth{1.003750pt}%
\definecolor{currentstroke}{rgb}{0.166667,0.166667,1.000000}%
\pgfsetstrokecolor{currentstroke}%
\pgfsetdash{}{0pt}%
\pgfpathmoveto{\pgfqpoint{0.457963in}{2.153451in}}%
\pgfpathlineto{\pgfqpoint{0.457963in}{2.153451in}}%
\pgfpathlineto{\pgfqpoint{0.457963in}{2.153451in}}%
\pgfpathlineto{\pgfqpoint{0.488963in}{2.153381in}}%
\pgfpathlineto{\pgfqpoint{0.499297in}{2.153341in}}%
\pgfpathlineto{\pgfqpoint{0.643963in}{2.151895in}}%
\pgfpathlineto{\pgfqpoint{0.664630in}{2.151554in}}%
\pgfpathlineto{\pgfqpoint{0.788630in}{2.148798in}}%
\pgfpathlineto{\pgfqpoint{1.119297in}{2.135520in}}%
\pgfpathlineto{\pgfqpoint{1.522297in}{2.107678in}}%
\pgfpathlineto{\pgfqpoint{1.625630in}{2.098476in}}%
\pgfpathlineto{\pgfqpoint{2.152630in}{2.038444in}}%
\pgfpathlineto{\pgfqpoint{2.751963in}{1.943550in}}%
\pgfpathlineto{\pgfqpoint{2.782963in}{1.937871in}}%
\pgfpathlineto{\pgfqpoint{3.847297in}{1.696921in}}%
\pgfpathlineto{\pgfqpoint{3.867963in}{1.691358in}}%
\pgfpathlineto{\pgfqpoint{4.095297in}{1.627942in}}%
\pgfpathlineto{\pgfqpoint{4.353630in}{1.550931in}}%
\pgfpathlineto{\pgfqpoint{4.673963in}{1.448126in}}%
\pgfpathlineto{\pgfqpoint{5.562630in}{1.120557in}}%
\pgfusepath{stroke}%
\end{pgfscope}%
\begin{pgfscope}%
\pgfpathrectangle{\pgfqpoint{0.457963in}{0.528059in}}{\pgfqpoint{6.200000in}{2.285714in}} %
\pgfusepath{clip}%
\pgfsetrectcap%
\pgfsetroundjoin%
\pgfsetlinewidth{1.003750pt}%
\definecolor{currentstroke}{rgb}{0.000000,0.000000,1.000000}%
\pgfsetstrokecolor{currentstroke}%
\pgfsetdash{}{0pt}%
\pgfpathmoveto{\pgfqpoint{0.457963in}{2.479245in}}%
\pgfpathlineto{\pgfqpoint{0.457963in}{2.479245in}}%
\pgfpathlineto{\pgfqpoint{0.468297in}{2.479445in}}%
\pgfpathlineto{\pgfqpoint{0.468297in}{2.479445in}}%
\pgfpathlineto{\pgfqpoint{0.499297in}{2.479973in}}%
\pgfpathlineto{\pgfqpoint{0.643963in}{2.480999in}}%
\pgfpathlineto{\pgfqpoint{0.664630in}{2.480952in}}%
\pgfpathlineto{\pgfqpoint{1.201963in}{2.462755in}}%
\pgfpathlineto{\pgfqpoint{1.594630in}{2.428780in}}%
\pgfpathlineto{\pgfqpoint{1.770297in}{2.407929in}}%
\pgfpathlineto{\pgfqpoint{1.966630in}{2.380488in}}%
\pgfpathlineto{\pgfqpoint{2.658963in}{2.248890in}}%
\pgfpathlineto{\pgfqpoint{2.844963in}{2.204286in}}%
\pgfpathlineto{\pgfqpoint{3.723297in}{1.940722in}}%
\pgfpathlineto{\pgfqpoint{3.733630in}{1.937101in}}%
\pgfpathlineto{\pgfqpoint{3.816297in}{1.907701in}}%
\pgfpathlineto{\pgfqpoint{4.157297in}{1.778243in}}%
\pgfpathlineto{\pgfqpoint{4.673963in}{1.557005in}}%
\pgfpathlineto{\pgfqpoint{4.694630in}{1.547527in}}%
\pgfpathlineto{\pgfqpoint{5.562630in}{1.105765in}}%
\pgfusepath{stroke}%
\end{pgfscope}%
\begin{pgfscope}%
\pgfpathrectangle{\pgfqpoint{0.457963in}{0.528059in}}{\pgfqpoint{6.200000in}{2.285714in}} %
\pgfusepath{clip}%
\pgfsetrectcap%
\pgfsetroundjoin%
\pgfsetlinewidth{1.003750pt}%
\definecolor{currentstroke}{rgb}{1.000000,0.833333,0.833333}%
\pgfsetstrokecolor{currentstroke}%
\pgfsetdash{}{0pt}%
\pgfpathmoveto{\pgfqpoint{0.457963in}{0.840398in}}%
\pgfpathlineto{\pgfqpoint{0.457963in}{0.840398in}}%
\pgfpathlineto{\pgfqpoint{0.457963in}{0.840398in}}%
\pgfpathlineto{\pgfqpoint{0.457963in}{0.840398in}}%
\pgfpathlineto{\pgfqpoint{0.468297in}{0.840473in}}%
\pgfpathlineto{\pgfqpoint{0.468297in}{0.840473in}}%
\pgfpathlineto{\pgfqpoint{0.468297in}{0.840473in}}%
\pgfpathlineto{\pgfqpoint{0.540630in}{0.840508in}}%
\pgfpathlineto{\pgfqpoint{0.550963in}{0.840442in}}%
\pgfpathlineto{\pgfqpoint{0.674963in}{0.838276in}}%
\pgfpathlineto{\pgfqpoint{0.685297in}{0.837980in}}%
\pgfpathlineto{\pgfqpoint{0.788630in}{0.834055in}}%
\pgfpathlineto{\pgfqpoint{0.943630in}{0.824854in}}%
\pgfpathlineto{\pgfqpoint{1.026297in}{0.818321in}}%
\pgfpathlineto{\pgfqpoint{1.222630in}{0.798275in}}%
\pgfpathlineto{\pgfqpoint{1.232963in}{0.797043in}}%
\pgfpathlineto{\pgfqpoint{1.460297in}{0.765474in}}%
\pgfpathlineto{\pgfqpoint{1.677297in}{0.727364in}}%
\pgfpathlineto{\pgfqpoint{1.739297in}{0.715045in}}%
\pgfpathlineto{\pgfqpoint{1.997630in}{0.656867in}}%
\pgfusepath{stroke}%
\end{pgfscope}%
\begin{pgfscope}%
\pgfpathrectangle{\pgfqpoint{0.457963in}{0.528059in}}{\pgfqpoint{6.200000in}{2.285714in}} %
\pgfusepath{clip}%
\pgfsetrectcap%
\pgfsetroundjoin%
\pgfsetlinewidth{1.003750pt}%
\definecolor{currentstroke}{rgb}{1.000000,0.666667,0.666667}%
\pgfsetstrokecolor{currentstroke}%
\pgfsetdash{}{0pt}%
\pgfpathmoveto{\pgfqpoint{0.457963in}{1.147255in}}%
\pgfpathlineto{\pgfqpoint{0.457963in}{1.147255in}}%
\pgfpathlineto{\pgfqpoint{0.457963in}{1.147255in}}%
\pgfpathlineto{\pgfqpoint{0.457963in}{1.147255in}}%
\pgfpathlineto{\pgfqpoint{0.468297in}{1.146518in}}%
\pgfpathlineto{\pgfqpoint{0.499297in}{1.144289in}}%
\pgfpathlineto{\pgfqpoint{0.509630in}{1.143539in}}%
\pgfpathlineto{\pgfqpoint{0.519963in}{1.142786in}}%
\pgfpathlineto{\pgfqpoint{0.519963in}{1.142786in}}%
\pgfpathlineto{\pgfqpoint{0.571630in}{1.138970in}}%
\pgfpathlineto{\pgfqpoint{0.798963in}{1.121178in}}%
\pgfpathlineto{\pgfqpoint{0.809297in}{1.120331in}}%
\pgfpathlineto{\pgfqpoint{1.098630in}{1.095234in}}%
\pgfpathlineto{\pgfqpoint{1.201963in}{1.085630in}}%
\pgfpathlineto{\pgfqpoint{1.367297in}{1.069564in}}%
\pgfpathlineto{\pgfqpoint{1.491297in}{1.056948in}}%
\pgfpathlineto{\pgfqpoint{1.532630in}{1.052635in}}%
\pgfpathlineto{\pgfqpoint{1.759963in}{1.027949in}}%
\pgfpathlineto{\pgfqpoint{2.514297in}{0.934355in}}%
\pgfpathlineto{\pgfqpoint{2.762297in}{0.899662in}}%
\pgfusepath{stroke}%
\end{pgfscope}%
\begin{pgfscope}%
\pgfpathrectangle{\pgfqpoint{0.457963in}{0.528059in}}{\pgfqpoint{6.200000in}{2.285714in}} %
\pgfusepath{clip}%
\pgfsetrectcap%
\pgfsetroundjoin%
\pgfsetlinewidth{1.003750pt}%
\definecolor{currentstroke}{rgb}{1.000000,0.500000,0.500000}%
\pgfsetstrokecolor{currentstroke}%
\pgfsetdash{}{0pt}%
\pgfpathmoveto{\pgfqpoint{0.457963in}{1.468346in}}%
\pgfpathlineto{\pgfqpoint{0.457963in}{1.468346in}}%
\pgfpathlineto{\pgfqpoint{0.457963in}{1.468346in}}%
\pgfpathlineto{\pgfqpoint{0.457963in}{1.468346in}}%
\pgfpathlineto{\pgfqpoint{0.457963in}{1.468346in}}%
\pgfpathlineto{\pgfqpoint{0.468297in}{1.467199in}}%
\pgfpathlineto{\pgfqpoint{0.468297in}{1.467199in}}%
\pgfpathlineto{\pgfqpoint{0.561297in}{1.456787in}}%
\pgfpathlineto{\pgfqpoint{0.685297in}{1.442670in}}%
\pgfpathlineto{\pgfqpoint{0.736963in}{1.436709in}}%
\pgfpathlineto{\pgfqpoint{0.860963in}{1.422211in}}%
\pgfpathlineto{\pgfqpoint{0.933297in}{1.413631in}}%
\pgfpathlineto{\pgfqpoint{1.470630in}{1.347025in}}%
\pgfpathlineto{\pgfqpoint{1.480963in}{1.345695in}}%
\pgfpathlineto{\pgfqpoint{1.511963in}{1.341693in}}%
\pgfpathlineto{\pgfqpoint{1.666963in}{1.321431in}}%
\pgfpathlineto{\pgfqpoint{1.790963in}{1.304919in}}%
\pgfpathlineto{\pgfqpoint{2.080297in}{1.265347in}}%
\pgfpathlineto{\pgfqpoint{3.051630in}{1.121803in}}%
\pgfpathlineto{\pgfqpoint{3.371963in}{1.070851in}}%
\pgfusepath{stroke}%
\end{pgfscope}%
\begin{pgfscope}%
\pgfpathrectangle{\pgfqpoint{0.457963in}{0.528059in}}{\pgfqpoint{6.200000in}{2.285714in}} %
\pgfusepath{clip}%
\pgfsetrectcap%
\pgfsetroundjoin%
\pgfsetlinewidth{1.003750pt}%
\definecolor{currentstroke}{rgb}{1.000000,0.333333,0.333333}%
\pgfsetstrokecolor{currentstroke}%
\pgfsetdash{}{0pt}%
\pgfpathmoveto{\pgfqpoint{0.457963in}{1.783527in}}%
\pgfpathlineto{\pgfqpoint{0.457963in}{1.783527in}}%
\pgfpathlineto{\pgfqpoint{0.457963in}{1.783527in}}%
\pgfpathlineto{\pgfqpoint{0.457963in}{1.783527in}}%
\pgfpathlineto{\pgfqpoint{0.457963in}{1.783527in}}%
\pgfpathlineto{\pgfqpoint{0.478630in}{1.780971in}}%
\pgfpathlineto{\pgfqpoint{0.530297in}{1.774474in}}%
\pgfpathlineto{\pgfqpoint{0.530297in}{1.774474in}}%
\pgfpathlineto{\pgfqpoint{0.664630in}{1.756871in}}%
\pgfpathlineto{\pgfqpoint{0.860963in}{1.729300in}}%
\pgfpathlineto{\pgfqpoint{1.253630in}{1.667589in}}%
\pgfpathlineto{\pgfqpoint{1.325963in}{1.655266in}}%
\pgfpathlineto{\pgfqpoint{1.573963in}{1.610758in}}%
\pgfpathlineto{\pgfqpoint{1.573963in}{1.610758in}}%
\pgfpathlineto{\pgfqpoint{1.904630in}{1.545980in}}%
\pgfpathlineto{\pgfqpoint{2.152630in}{1.493320in}}%
\pgfpathlineto{\pgfqpoint{2.152630in}{1.493320in}}%
\pgfpathlineto{\pgfqpoint{2.400630in}{1.437166in}}%
\pgfpathlineto{\pgfqpoint{2.514297in}{1.410261in}}%
\pgfpathlineto{\pgfqpoint{3.630297in}{1.107126in}}%
\pgfusepath{stroke}%
\end{pgfscope}%
\begin{pgfscope}%
\pgfpathrectangle{\pgfqpoint{0.457963in}{0.528059in}}{\pgfqpoint{6.200000in}{2.285714in}} %
\pgfusepath{clip}%
\pgfsetrectcap%
\pgfsetroundjoin%
\pgfsetlinewidth{1.003750pt}%
\definecolor{currentstroke}{rgb}{1.000000,0.166667,0.166667}%
\pgfsetstrokecolor{currentstroke}%
\pgfsetdash{}{0pt}%
\pgfpathmoveto{\pgfqpoint{0.457963in}{2.065173in}}%
\pgfpathlineto{\pgfqpoint{0.457963in}{2.065173in}}%
\pgfpathlineto{\pgfqpoint{0.457963in}{2.065173in}}%
\pgfpathlineto{\pgfqpoint{0.457963in}{2.065173in}}%
\pgfpathlineto{\pgfqpoint{0.468297in}{2.064288in}}%
\pgfpathlineto{\pgfqpoint{0.509630in}{2.060582in}}%
\pgfpathlineto{\pgfqpoint{0.643963in}{2.046720in}}%
\pgfpathlineto{\pgfqpoint{0.695630in}{2.040648in}}%
\pgfpathlineto{\pgfqpoint{0.840297in}{2.021459in}}%
\pgfpathlineto{\pgfqpoint{1.057297in}{1.986627in}}%
\pgfpathlineto{\pgfqpoint{1.243297in}{1.950996in}}%
\pgfpathlineto{\pgfqpoint{1.439630in}{1.907601in}}%
\pgfpathlineto{\pgfqpoint{1.460297in}{1.902687in}}%
\pgfpathlineto{\pgfqpoint{1.553297in}{1.879763in}}%
\pgfpathlineto{\pgfqpoint{2.018297in}{1.745146in}}%
\pgfpathlineto{\pgfqpoint{2.038963in}{1.738390in}}%
\pgfpathlineto{\pgfqpoint{2.049297in}{1.734987in}}%
\pgfpathlineto{\pgfqpoint{2.317963in}{1.640738in}}%
\pgfpathlineto{\pgfqpoint{2.545297in}{1.552300in}}%
\pgfpathlineto{\pgfqpoint{3.185963in}{1.260219in}}%
\pgfusepath{stroke}%
\end{pgfscope}%
\begin{pgfscope}%
\pgfpathrectangle{\pgfqpoint{0.457963in}{0.528059in}}{\pgfqpoint{6.200000in}{2.285714in}} %
\pgfusepath{clip}%
\pgfsetrectcap%
\pgfsetroundjoin%
\pgfsetlinewidth{1.003750pt}%
\definecolor{currentstroke}{rgb}{1.000000,0.000000,0.000000}%
\pgfsetstrokecolor{currentstroke}%
\pgfsetdash{}{0pt}%
\pgfpathmoveto{\pgfqpoint{0.457963in}{2.337781in}}%
\pgfpathlineto{\pgfqpoint{0.457963in}{2.337781in}}%
\pgfpathlineto{\pgfqpoint{0.457963in}{2.337781in}}%
\pgfpathlineto{\pgfqpoint{0.478630in}{2.336080in}}%
\pgfpathlineto{\pgfqpoint{0.499297in}{2.334282in}}%
\pgfpathlineto{\pgfqpoint{0.654297in}{2.317684in}}%
\pgfpathlineto{\pgfqpoint{0.685297in}{2.313705in}}%
\pgfpathlineto{\pgfqpoint{0.685297in}{2.313705in}}%
\pgfpathlineto{\pgfqpoint{0.943630in}{2.272014in}}%
\pgfpathlineto{\pgfqpoint{0.953963in}{2.270029in}}%
\pgfpathlineto{\pgfqpoint{1.108963in}{2.237330in}}%
\pgfpathlineto{\pgfqpoint{1.501630in}{2.129925in}}%
\pgfpathlineto{\pgfqpoint{1.553297in}{2.113170in}}%
\pgfpathlineto{\pgfqpoint{1.801297in}{2.024258in}}%
\pgfpathlineto{\pgfqpoint{1.904630in}{1.983063in}}%
\pgfpathlineto{\pgfqpoint{1.935630in}{1.970230in}}%
\pgfpathlineto{\pgfqpoint{2.018297in}{1.934932in}}%
\pgfpathlineto{\pgfqpoint{2.090630in}{1.902766in}}%
\pgfpathlineto{\pgfqpoint{2.410963in}{1.745948in}}%
\pgfpathlineto{\pgfqpoint{3.258297in}{1.218112in}}%
\pgfusepath{stroke}%
\end{pgfscope}%
\begin{pgfscope}%
\pgfsetrectcap%
\pgfsetmiterjoin%
\pgfsetlinewidth{1.003750pt}%
\definecolor{currentstroke}{rgb}{0.000000,0.000000,0.000000}%
\pgfsetstrokecolor{currentstroke}%
\pgfsetdash{}{0pt}%
\pgfpathmoveto{\pgfqpoint{0.457963in}{2.813774in}}%
\pgfpathlineto{\pgfqpoint{6.657963in}{2.813774in}}%
\pgfusepath{stroke}%
\end{pgfscope}%
\begin{pgfscope}%
\pgfsetrectcap%
\pgfsetmiterjoin%
\pgfsetlinewidth{1.003750pt}%
\definecolor{currentstroke}{rgb}{0.000000,0.000000,0.000000}%
\pgfsetstrokecolor{currentstroke}%
\pgfsetdash{}{0pt}%
\pgfpathmoveto{\pgfqpoint{6.657963in}{0.528059in}}%
\pgfpathlineto{\pgfqpoint{6.657963in}{2.813774in}}%
\pgfusepath{stroke}%
\end{pgfscope}%
\begin{pgfscope}%
\pgfsetrectcap%
\pgfsetmiterjoin%
\pgfsetlinewidth{1.003750pt}%
\definecolor{currentstroke}{rgb}{0.000000,0.000000,0.000000}%
\pgfsetstrokecolor{currentstroke}%
\pgfsetdash{}{0pt}%
\pgfpathmoveto{\pgfqpoint{0.457963in}{0.528059in}}%
\pgfpathlineto{\pgfqpoint{6.657963in}{0.528059in}}%
\pgfusepath{stroke}%
\end{pgfscope}%
\begin{pgfscope}%
\pgfsetrectcap%
\pgfsetmiterjoin%
\pgfsetlinewidth{1.003750pt}%
\definecolor{currentstroke}{rgb}{0.000000,0.000000,0.000000}%
\pgfsetstrokecolor{currentstroke}%
\pgfsetdash{}{0pt}%
\pgfpathmoveto{\pgfqpoint{0.457963in}{0.528059in}}%
\pgfpathlineto{\pgfqpoint{0.457963in}{2.813774in}}%
\pgfusepath{stroke}%
\end{pgfscope}%
\begin{pgfscope}%
\pgfsetbuttcap%
\pgfsetroundjoin%
\definecolor{currentfill}{rgb}{0.000000,0.000000,0.000000}%
\pgfsetfillcolor{currentfill}%
\pgfsetlinewidth{0.501875pt}%
\definecolor{currentstroke}{rgb}{0.000000,0.000000,0.000000}%
\pgfsetstrokecolor{currentstroke}%
\pgfsetdash{}{0pt}%
\pgfsys@defobject{currentmarker}{\pgfqpoint{0.000000in}{0.000000in}}{\pgfqpoint{0.000000in}{0.055556in}}{%
\pgfpathmoveto{\pgfqpoint{0.000000in}{0.000000in}}%
\pgfpathlineto{\pgfqpoint{0.000000in}{0.055556in}}%
\pgfusepath{stroke,fill}%
}%
\begin{pgfscope}%
\pgfsys@transformshift{0.457963in}{0.528059in}%
\pgfsys@useobject{currentmarker}{}%
\end{pgfscope}%
\end{pgfscope}%
\begin{pgfscope}%
\pgfsetbuttcap%
\pgfsetroundjoin%
\definecolor{currentfill}{rgb}{0.000000,0.000000,0.000000}%
\pgfsetfillcolor{currentfill}%
\pgfsetlinewidth{0.501875pt}%
\definecolor{currentstroke}{rgb}{0.000000,0.000000,0.000000}%
\pgfsetstrokecolor{currentstroke}%
\pgfsetdash{}{0pt}%
\pgfsys@defobject{currentmarker}{\pgfqpoint{0.000000in}{-0.055556in}}{\pgfqpoint{0.000000in}{0.000000in}}{%
\pgfpathmoveto{\pgfqpoint{0.000000in}{0.000000in}}%
\pgfpathlineto{\pgfqpoint{0.000000in}{-0.055556in}}%
\pgfusepath{stroke,fill}%
}%
\begin{pgfscope}%
\pgfsys@transformshift{0.457963in}{2.813774in}%
\pgfsys@useobject{currentmarker}{}%
\end{pgfscope}%
\end{pgfscope}%
\begin{pgfscope}%
\pgftext[x=0.457963in,y=0.472504in,,top]{\rmfamily\fontsize{12.000000}{14.400000}\selectfont \(\displaystyle 0\)}%
\end{pgfscope}%
\begin{pgfscope}%
\pgfsetbuttcap%
\pgfsetroundjoin%
\definecolor{currentfill}{rgb}{0.000000,0.000000,0.000000}%
\pgfsetfillcolor{currentfill}%
\pgfsetlinewidth{0.501875pt}%
\definecolor{currentstroke}{rgb}{0.000000,0.000000,0.000000}%
\pgfsetstrokecolor{currentstroke}%
\pgfsetdash{}{0pt}%
\pgfsys@defobject{currentmarker}{\pgfqpoint{0.000000in}{0.000000in}}{\pgfqpoint{0.000000in}{0.055556in}}{%
\pgfpathmoveto{\pgfqpoint{0.000000in}{0.000000in}}%
\pgfpathlineto{\pgfqpoint{0.000000in}{0.055556in}}%
\pgfusepath{stroke,fill}%
}%
\begin{pgfscope}%
\pgfsys@transformshift{1.491297in}{0.528059in}%
\pgfsys@useobject{currentmarker}{}%
\end{pgfscope}%
\end{pgfscope}%
\begin{pgfscope}%
\pgfsetbuttcap%
\pgfsetroundjoin%
\definecolor{currentfill}{rgb}{0.000000,0.000000,0.000000}%
\pgfsetfillcolor{currentfill}%
\pgfsetlinewidth{0.501875pt}%
\definecolor{currentstroke}{rgb}{0.000000,0.000000,0.000000}%
\pgfsetstrokecolor{currentstroke}%
\pgfsetdash{}{0pt}%
\pgfsys@defobject{currentmarker}{\pgfqpoint{0.000000in}{-0.055556in}}{\pgfqpoint{0.000000in}{0.000000in}}{%
\pgfpathmoveto{\pgfqpoint{0.000000in}{0.000000in}}%
\pgfpathlineto{\pgfqpoint{0.000000in}{-0.055556in}}%
\pgfusepath{stroke,fill}%
}%
\begin{pgfscope}%
\pgfsys@transformshift{1.491297in}{2.813774in}%
\pgfsys@useobject{currentmarker}{}%
\end{pgfscope}%
\end{pgfscope}%
\begin{pgfscope}%
\pgftext[x=1.491297in,y=0.472504in,,top]{\rmfamily\fontsize{12.000000}{14.400000}\selectfont \(\displaystyle 100\)}%
\end{pgfscope}%
\begin{pgfscope}%
\pgfsetbuttcap%
\pgfsetroundjoin%
\definecolor{currentfill}{rgb}{0.000000,0.000000,0.000000}%
\pgfsetfillcolor{currentfill}%
\pgfsetlinewidth{0.501875pt}%
\definecolor{currentstroke}{rgb}{0.000000,0.000000,0.000000}%
\pgfsetstrokecolor{currentstroke}%
\pgfsetdash{}{0pt}%
\pgfsys@defobject{currentmarker}{\pgfqpoint{0.000000in}{0.000000in}}{\pgfqpoint{0.000000in}{0.055556in}}{%
\pgfpathmoveto{\pgfqpoint{0.000000in}{0.000000in}}%
\pgfpathlineto{\pgfqpoint{0.000000in}{0.055556in}}%
\pgfusepath{stroke,fill}%
}%
\begin{pgfscope}%
\pgfsys@transformshift{2.524630in}{0.528059in}%
\pgfsys@useobject{currentmarker}{}%
\end{pgfscope}%
\end{pgfscope}%
\begin{pgfscope}%
\pgfsetbuttcap%
\pgfsetroundjoin%
\definecolor{currentfill}{rgb}{0.000000,0.000000,0.000000}%
\pgfsetfillcolor{currentfill}%
\pgfsetlinewidth{0.501875pt}%
\definecolor{currentstroke}{rgb}{0.000000,0.000000,0.000000}%
\pgfsetstrokecolor{currentstroke}%
\pgfsetdash{}{0pt}%
\pgfsys@defobject{currentmarker}{\pgfqpoint{0.000000in}{-0.055556in}}{\pgfqpoint{0.000000in}{0.000000in}}{%
\pgfpathmoveto{\pgfqpoint{0.000000in}{0.000000in}}%
\pgfpathlineto{\pgfqpoint{0.000000in}{-0.055556in}}%
\pgfusepath{stroke,fill}%
}%
\begin{pgfscope}%
\pgfsys@transformshift{2.524630in}{2.813774in}%
\pgfsys@useobject{currentmarker}{}%
\end{pgfscope}%
\end{pgfscope}%
\begin{pgfscope}%
\pgftext[x=2.524630in,y=0.472504in,,top]{\rmfamily\fontsize{12.000000}{14.400000}\selectfont \(\displaystyle 200\)}%
\end{pgfscope}%
\begin{pgfscope}%
\pgfsetbuttcap%
\pgfsetroundjoin%
\definecolor{currentfill}{rgb}{0.000000,0.000000,0.000000}%
\pgfsetfillcolor{currentfill}%
\pgfsetlinewidth{0.501875pt}%
\definecolor{currentstroke}{rgb}{0.000000,0.000000,0.000000}%
\pgfsetstrokecolor{currentstroke}%
\pgfsetdash{}{0pt}%
\pgfsys@defobject{currentmarker}{\pgfqpoint{0.000000in}{0.000000in}}{\pgfqpoint{0.000000in}{0.055556in}}{%
\pgfpathmoveto{\pgfqpoint{0.000000in}{0.000000in}}%
\pgfpathlineto{\pgfqpoint{0.000000in}{0.055556in}}%
\pgfusepath{stroke,fill}%
}%
\begin{pgfscope}%
\pgfsys@transformshift{3.557963in}{0.528059in}%
\pgfsys@useobject{currentmarker}{}%
\end{pgfscope}%
\end{pgfscope}%
\begin{pgfscope}%
\pgfsetbuttcap%
\pgfsetroundjoin%
\definecolor{currentfill}{rgb}{0.000000,0.000000,0.000000}%
\pgfsetfillcolor{currentfill}%
\pgfsetlinewidth{0.501875pt}%
\definecolor{currentstroke}{rgb}{0.000000,0.000000,0.000000}%
\pgfsetstrokecolor{currentstroke}%
\pgfsetdash{}{0pt}%
\pgfsys@defobject{currentmarker}{\pgfqpoint{0.000000in}{-0.055556in}}{\pgfqpoint{0.000000in}{0.000000in}}{%
\pgfpathmoveto{\pgfqpoint{0.000000in}{0.000000in}}%
\pgfpathlineto{\pgfqpoint{0.000000in}{-0.055556in}}%
\pgfusepath{stroke,fill}%
}%
\begin{pgfscope}%
\pgfsys@transformshift{3.557963in}{2.813774in}%
\pgfsys@useobject{currentmarker}{}%
\end{pgfscope}%
\end{pgfscope}%
\begin{pgfscope}%
\pgftext[x=3.557963in,y=0.472504in,,top]{\rmfamily\fontsize{12.000000}{14.400000}\selectfont \(\displaystyle 300\)}%
\end{pgfscope}%
\begin{pgfscope}%
\pgfsetbuttcap%
\pgfsetroundjoin%
\definecolor{currentfill}{rgb}{0.000000,0.000000,0.000000}%
\pgfsetfillcolor{currentfill}%
\pgfsetlinewidth{0.501875pt}%
\definecolor{currentstroke}{rgb}{0.000000,0.000000,0.000000}%
\pgfsetstrokecolor{currentstroke}%
\pgfsetdash{}{0pt}%
\pgfsys@defobject{currentmarker}{\pgfqpoint{0.000000in}{0.000000in}}{\pgfqpoint{0.000000in}{0.055556in}}{%
\pgfpathmoveto{\pgfqpoint{0.000000in}{0.000000in}}%
\pgfpathlineto{\pgfqpoint{0.000000in}{0.055556in}}%
\pgfusepath{stroke,fill}%
}%
\begin{pgfscope}%
\pgfsys@transformshift{4.591297in}{0.528059in}%
\pgfsys@useobject{currentmarker}{}%
\end{pgfscope}%
\end{pgfscope}%
\begin{pgfscope}%
\pgfsetbuttcap%
\pgfsetroundjoin%
\definecolor{currentfill}{rgb}{0.000000,0.000000,0.000000}%
\pgfsetfillcolor{currentfill}%
\pgfsetlinewidth{0.501875pt}%
\definecolor{currentstroke}{rgb}{0.000000,0.000000,0.000000}%
\pgfsetstrokecolor{currentstroke}%
\pgfsetdash{}{0pt}%
\pgfsys@defobject{currentmarker}{\pgfqpoint{0.000000in}{-0.055556in}}{\pgfqpoint{0.000000in}{0.000000in}}{%
\pgfpathmoveto{\pgfqpoint{0.000000in}{0.000000in}}%
\pgfpathlineto{\pgfqpoint{0.000000in}{-0.055556in}}%
\pgfusepath{stroke,fill}%
}%
\begin{pgfscope}%
\pgfsys@transformshift{4.591297in}{2.813774in}%
\pgfsys@useobject{currentmarker}{}%
\end{pgfscope}%
\end{pgfscope}%
\begin{pgfscope}%
\pgftext[x=4.591297in,y=0.472504in,,top]{\rmfamily\fontsize{12.000000}{14.400000}\selectfont \(\displaystyle 400\)}%
\end{pgfscope}%
\begin{pgfscope}%
\pgfsetbuttcap%
\pgfsetroundjoin%
\definecolor{currentfill}{rgb}{0.000000,0.000000,0.000000}%
\pgfsetfillcolor{currentfill}%
\pgfsetlinewidth{0.501875pt}%
\definecolor{currentstroke}{rgb}{0.000000,0.000000,0.000000}%
\pgfsetstrokecolor{currentstroke}%
\pgfsetdash{}{0pt}%
\pgfsys@defobject{currentmarker}{\pgfqpoint{0.000000in}{0.000000in}}{\pgfqpoint{0.000000in}{0.055556in}}{%
\pgfpathmoveto{\pgfqpoint{0.000000in}{0.000000in}}%
\pgfpathlineto{\pgfqpoint{0.000000in}{0.055556in}}%
\pgfusepath{stroke,fill}%
}%
\begin{pgfscope}%
\pgfsys@transformshift{5.624630in}{0.528059in}%
\pgfsys@useobject{currentmarker}{}%
\end{pgfscope}%
\end{pgfscope}%
\begin{pgfscope}%
\pgfsetbuttcap%
\pgfsetroundjoin%
\definecolor{currentfill}{rgb}{0.000000,0.000000,0.000000}%
\pgfsetfillcolor{currentfill}%
\pgfsetlinewidth{0.501875pt}%
\definecolor{currentstroke}{rgb}{0.000000,0.000000,0.000000}%
\pgfsetstrokecolor{currentstroke}%
\pgfsetdash{}{0pt}%
\pgfsys@defobject{currentmarker}{\pgfqpoint{0.000000in}{-0.055556in}}{\pgfqpoint{0.000000in}{0.000000in}}{%
\pgfpathmoveto{\pgfqpoint{0.000000in}{0.000000in}}%
\pgfpathlineto{\pgfqpoint{0.000000in}{-0.055556in}}%
\pgfusepath{stroke,fill}%
}%
\begin{pgfscope}%
\pgfsys@transformshift{5.624630in}{2.813774in}%
\pgfsys@useobject{currentmarker}{}%
\end{pgfscope}%
\end{pgfscope}%
\begin{pgfscope}%
\pgftext[x=5.624630in,y=0.472504in,,top]{\rmfamily\fontsize{12.000000}{14.400000}\selectfont \(\displaystyle 500\)}%
\end{pgfscope}%
\begin{pgfscope}%
\pgfsetbuttcap%
\pgfsetroundjoin%
\definecolor{currentfill}{rgb}{0.000000,0.000000,0.000000}%
\pgfsetfillcolor{currentfill}%
\pgfsetlinewidth{0.501875pt}%
\definecolor{currentstroke}{rgb}{0.000000,0.000000,0.000000}%
\pgfsetstrokecolor{currentstroke}%
\pgfsetdash{}{0pt}%
\pgfsys@defobject{currentmarker}{\pgfqpoint{0.000000in}{0.000000in}}{\pgfqpoint{0.000000in}{0.055556in}}{%
\pgfpathmoveto{\pgfqpoint{0.000000in}{0.000000in}}%
\pgfpathlineto{\pgfqpoint{0.000000in}{0.055556in}}%
\pgfusepath{stroke,fill}%
}%
\begin{pgfscope}%
\pgfsys@transformshift{6.657963in}{0.528059in}%
\pgfsys@useobject{currentmarker}{}%
\end{pgfscope}%
\end{pgfscope}%
\begin{pgfscope}%
\pgfsetbuttcap%
\pgfsetroundjoin%
\definecolor{currentfill}{rgb}{0.000000,0.000000,0.000000}%
\pgfsetfillcolor{currentfill}%
\pgfsetlinewidth{0.501875pt}%
\definecolor{currentstroke}{rgb}{0.000000,0.000000,0.000000}%
\pgfsetstrokecolor{currentstroke}%
\pgfsetdash{}{0pt}%
\pgfsys@defobject{currentmarker}{\pgfqpoint{0.000000in}{-0.055556in}}{\pgfqpoint{0.000000in}{0.000000in}}{%
\pgfpathmoveto{\pgfqpoint{0.000000in}{0.000000in}}%
\pgfpathlineto{\pgfqpoint{0.000000in}{-0.055556in}}%
\pgfusepath{stroke,fill}%
}%
\begin{pgfscope}%
\pgfsys@transformshift{6.657963in}{2.813774in}%
\pgfsys@useobject{currentmarker}{}%
\end{pgfscope}%
\end{pgfscope}%
\begin{pgfscope}%
\pgftext[x=6.657963in,y=0.472504in,,top]{\rmfamily\fontsize{12.000000}{14.400000}\selectfont \(\displaystyle 600\)}%
\end{pgfscope}%
\begin{pgfscope}%
\pgftext[x=3.557963in,y=0.251692in,,top]{\rmfamily\fontsize{12.000000}{14.400000}\selectfont \#fuzzings}%
\end{pgfscope}%
\begin{pgfscope}%
\pgfsetbuttcap%
\pgfsetroundjoin%
\definecolor{currentfill}{rgb}{0.000000,0.000000,0.000000}%
\pgfsetfillcolor{currentfill}%
\pgfsetlinewidth{0.501875pt}%
\definecolor{currentstroke}{rgb}{0.000000,0.000000,0.000000}%
\pgfsetstrokecolor{currentstroke}%
\pgfsetdash{}{0pt}%
\pgfsys@defobject{currentmarker}{\pgfqpoint{0.000000in}{0.000000in}}{\pgfqpoint{0.055556in}{0.000000in}}{%
\pgfpathmoveto{\pgfqpoint{0.000000in}{0.000000in}}%
\pgfpathlineto{\pgfqpoint{0.055556in}{0.000000in}}%
\pgfusepath{stroke,fill}%
}%
\begin{pgfscope}%
\pgfsys@transformshift{0.457963in}{0.528059in}%
\pgfsys@useobject{currentmarker}{}%
\end{pgfscope}%
\end{pgfscope}%
\begin{pgfscope}%
\pgfsetbuttcap%
\pgfsetroundjoin%
\definecolor{currentfill}{rgb}{0.000000,0.000000,0.000000}%
\pgfsetfillcolor{currentfill}%
\pgfsetlinewidth{0.501875pt}%
\definecolor{currentstroke}{rgb}{0.000000,0.000000,0.000000}%
\pgfsetstrokecolor{currentstroke}%
\pgfsetdash{}{0pt}%
\pgfsys@defobject{currentmarker}{\pgfqpoint{-0.055556in}{0.000000in}}{\pgfqpoint{0.000000in}{0.000000in}}{%
\pgfpathmoveto{\pgfqpoint{0.000000in}{0.000000in}}%
\pgfpathlineto{\pgfqpoint{-0.055556in}{0.000000in}}%
\pgfusepath{stroke,fill}%
}%
\begin{pgfscope}%
\pgfsys@transformshift{6.657963in}{0.528059in}%
\pgfsys@useobject{currentmarker}{}%
\end{pgfscope}%
\end{pgfscope}%
\begin{pgfscope}%
\pgftext[x=0.402408in,y=0.528059in,right,]{\rmfamily\fontsize{12.000000}{14.400000}\selectfont \(\displaystyle 0\)}%
\end{pgfscope}%
\begin{pgfscope}%
\pgfsetbuttcap%
\pgfsetroundjoin%
\definecolor{currentfill}{rgb}{0.000000,0.000000,0.000000}%
\pgfsetfillcolor{currentfill}%
\pgfsetlinewidth{0.501875pt}%
\definecolor{currentstroke}{rgb}{0.000000,0.000000,0.000000}%
\pgfsetstrokecolor{currentstroke}%
\pgfsetdash{}{0pt}%
\pgfsys@defobject{currentmarker}{\pgfqpoint{0.000000in}{0.000000in}}{\pgfqpoint{0.055556in}{0.000000in}}{%
\pgfpathmoveto{\pgfqpoint{0.000000in}{0.000000in}}%
\pgfpathlineto{\pgfqpoint{0.055556in}{0.000000in}}%
\pgfusepath{stroke,fill}%
}%
\begin{pgfscope}%
\pgfsys@transformshift{0.457963in}{0.854590in}%
\pgfsys@useobject{currentmarker}{}%
\end{pgfscope}%
\end{pgfscope}%
\begin{pgfscope}%
\pgfsetbuttcap%
\pgfsetroundjoin%
\definecolor{currentfill}{rgb}{0.000000,0.000000,0.000000}%
\pgfsetfillcolor{currentfill}%
\pgfsetlinewidth{0.501875pt}%
\definecolor{currentstroke}{rgb}{0.000000,0.000000,0.000000}%
\pgfsetstrokecolor{currentstroke}%
\pgfsetdash{}{0pt}%
\pgfsys@defobject{currentmarker}{\pgfqpoint{-0.055556in}{0.000000in}}{\pgfqpoint{0.000000in}{0.000000in}}{%
\pgfpathmoveto{\pgfqpoint{0.000000in}{0.000000in}}%
\pgfpathlineto{\pgfqpoint{-0.055556in}{0.000000in}}%
\pgfusepath{stroke,fill}%
}%
\begin{pgfscope}%
\pgfsys@transformshift{6.657963in}{0.854590in}%
\pgfsys@useobject{currentmarker}{}%
\end{pgfscope}%
\end{pgfscope}%
\begin{pgfscope}%
\pgftext[x=0.402408in,y=0.854590in,right,]{\rmfamily\fontsize{12.000000}{14.400000}\selectfont \(\displaystyle 1\)}%
\end{pgfscope}%
\begin{pgfscope}%
\pgfsetbuttcap%
\pgfsetroundjoin%
\definecolor{currentfill}{rgb}{0.000000,0.000000,0.000000}%
\pgfsetfillcolor{currentfill}%
\pgfsetlinewidth{0.501875pt}%
\definecolor{currentstroke}{rgb}{0.000000,0.000000,0.000000}%
\pgfsetstrokecolor{currentstroke}%
\pgfsetdash{}{0pt}%
\pgfsys@defobject{currentmarker}{\pgfqpoint{0.000000in}{0.000000in}}{\pgfqpoint{0.055556in}{0.000000in}}{%
\pgfpathmoveto{\pgfqpoint{0.000000in}{0.000000in}}%
\pgfpathlineto{\pgfqpoint{0.055556in}{0.000000in}}%
\pgfusepath{stroke,fill}%
}%
\begin{pgfscope}%
\pgfsys@transformshift{0.457963in}{1.181121in}%
\pgfsys@useobject{currentmarker}{}%
\end{pgfscope}%
\end{pgfscope}%
\begin{pgfscope}%
\pgfsetbuttcap%
\pgfsetroundjoin%
\definecolor{currentfill}{rgb}{0.000000,0.000000,0.000000}%
\pgfsetfillcolor{currentfill}%
\pgfsetlinewidth{0.501875pt}%
\definecolor{currentstroke}{rgb}{0.000000,0.000000,0.000000}%
\pgfsetstrokecolor{currentstroke}%
\pgfsetdash{}{0pt}%
\pgfsys@defobject{currentmarker}{\pgfqpoint{-0.055556in}{0.000000in}}{\pgfqpoint{0.000000in}{0.000000in}}{%
\pgfpathmoveto{\pgfqpoint{0.000000in}{0.000000in}}%
\pgfpathlineto{\pgfqpoint{-0.055556in}{0.000000in}}%
\pgfusepath{stroke,fill}%
}%
\begin{pgfscope}%
\pgfsys@transformshift{6.657963in}{1.181121in}%
\pgfsys@useobject{currentmarker}{}%
\end{pgfscope}%
\end{pgfscope}%
\begin{pgfscope}%
\pgftext[x=0.402408in,y=1.181121in,right,]{\rmfamily\fontsize{12.000000}{14.400000}\selectfont \(\displaystyle 2\)}%
\end{pgfscope}%
\begin{pgfscope}%
\pgfsetbuttcap%
\pgfsetroundjoin%
\definecolor{currentfill}{rgb}{0.000000,0.000000,0.000000}%
\pgfsetfillcolor{currentfill}%
\pgfsetlinewidth{0.501875pt}%
\definecolor{currentstroke}{rgb}{0.000000,0.000000,0.000000}%
\pgfsetstrokecolor{currentstroke}%
\pgfsetdash{}{0pt}%
\pgfsys@defobject{currentmarker}{\pgfqpoint{0.000000in}{0.000000in}}{\pgfqpoint{0.055556in}{0.000000in}}{%
\pgfpathmoveto{\pgfqpoint{0.000000in}{0.000000in}}%
\pgfpathlineto{\pgfqpoint{0.055556in}{0.000000in}}%
\pgfusepath{stroke,fill}%
}%
\begin{pgfscope}%
\pgfsys@transformshift{0.457963in}{1.507651in}%
\pgfsys@useobject{currentmarker}{}%
\end{pgfscope}%
\end{pgfscope}%
\begin{pgfscope}%
\pgfsetbuttcap%
\pgfsetroundjoin%
\definecolor{currentfill}{rgb}{0.000000,0.000000,0.000000}%
\pgfsetfillcolor{currentfill}%
\pgfsetlinewidth{0.501875pt}%
\definecolor{currentstroke}{rgb}{0.000000,0.000000,0.000000}%
\pgfsetstrokecolor{currentstroke}%
\pgfsetdash{}{0pt}%
\pgfsys@defobject{currentmarker}{\pgfqpoint{-0.055556in}{0.000000in}}{\pgfqpoint{0.000000in}{0.000000in}}{%
\pgfpathmoveto{\pgfqpoint{0.000000in}{0.000000in}}%
\pgfpathlineto{\pgfqpoint{-0.055556in}{0.000000in}}%
\pgfusepath{stroke,fill}%
}%
\begin{pgfscope}%
\pgfsys@transformshift{6.657963in}{1.507651in}%
\pgfsys@useobject{currentmarker}{}%
\end{pgfscope}%
\end{pgfscope}%
\begin{pgfscope}%
\pgftext[x=0.402408in,y=1.507651in,right,]{\rmfamily\fontsize{12.000000}{14.400000}\selectfont \(\displaystyle 3\)}%
\end{pgfscope}%
\begin{pgfscope}%
\pgfsetbuttcap%
\pgfsetroundjoin%
\definecolor{currentfill}{rgb}{0.000000,0.000000,0.000000}%
\pgfsetfillcolor{currentfill}%
\pgfsetlinewidth{0.501875pt}%
\definecolor{currentstroke}{rgb}{0.000000,0.000000,0.000000}%
\pgfsetstrokecolor{currentstroke}%
\pgfsetdash{}{0pt}%
\pgfsys@defobject{currentmarker}{\pgfqpoint{0.000000in}{0.000000in}}{\pgfqpoint{0.055556in}{0.000000in}}{%
\pgfpathmoveto{\pgfqpoint{0.000000in}{0.000000in}}%
\pgfpathlineto{\pgfqpoint{0.055556in}{0.000000in}}%
\pgfusepath{stroke,fill}%
}%
\begin{pgfscope}%
\pgfsys@transformshift{0.457963in}{1.834182in}%
\pgfsys@useobject{currentmarker}{}%
\end{pgfscope}%
\end{pgfscope}%
\begin{pgfscope}%
\pgfsetbuttcap%
\pgfsetroundjoin%
\definecolor{currentfill}{rgb}{0.000000,0.000000,0.000000}%
\pgfsetfillcolor{currentfill}%
\pgfsetlinewidth{0.501875pt}%
\definecolor{currentstroke}{rgb}{0.000000,0.000000,0.000000}%
\pgfsetstrokecolor{currentstroke}%
\pgfsetdash{}{0pt}%
\pgfsys@defobject{currentmarker}{\pgfqpoint{-0.055556in}{0.000000in}}{\pgfqpoint{0.000000in}{0.000000in}}{%
\pgfpathmoveto{\pgfqpoint{0.000000in}{0.000000in}}%
\pgfpathlineto{\pgfqpoint{-0.055556in}{0.000000in}}%
\pgfusepath{stroke,fill}%
}%
\begin{pgfscope}%
\pgfsys@transformshift{6.657963in}{1.834182in}%
\pgfsys@useobject{currentmarker}{}%
\end{pgfscope}%
\end{pgfscope}%
\begin{pgfscope}%
\pgftext[x=0.402408in,y=1.834182in,right,]{\rmfamily\fontsize{12.000000}{14.400000}\selectfont \(\displaystyle 4\)}%
\end{pgfscope}%
\begin{pgfscope}%
\pgfsetbuttcap%
\pgfsetroundjoin%
\definecolor{currentfill}{rgb}{0.000000,0.000000,0.000000}%
\pgfsetfillcolor{currentfill}%
\pgfsetlinewidth{0.501875pt}%
\definecolor{currentstroke}{rgb}{0.000000,0.000000,0.000000}%
\pgfsetstrokecolor{currentstroke}%
\pgfsetdash{}{0pt}%
\pgfsys@defobject{currentmarker}{\pgfqpoint{0.000000in}{0.000000in}}{\pgfqpoint{0.055556in}{0.000000in}}{%
\pgfpathmoveto{\pgfqpoint{0.000000in}{0.000000in}}%
\pgfpathlineto{\pgfqpoint{0.055556in}{0.000000in}}%
\pgfusepath{stroke,fill}%
}%
\begin{pgfscope}%
\pgfsys@transformshift{0.457963in}{2.160713in}%
\pgfsys@useobject{currentmarker}{}%
\end{pgfscope}%
\end{pgfscope}%
\begin{pgfscope}%
\pgfsetbuttcap%
\pgfsetroundjoin%
\definecolor{currentfill}{rgb}{0.000000,0.000000,0.000000}%
\pgfsetfillcolor{currentfill}%
\pgfsetlinewidth{0.501875pt}%
\definecolor{currentstroke}{rgb}{0.000000,0.000000,0.000000}%
\pgfsetstrokecolor{currentstroke}%
\pgfsetdash{}{0pt}%
\pgfsys@defobject{currentmarker}{\pgfqpoint{-0.055556in}{0.000000in}}{\pgfqpoint{0.000000in}{0.000000in}}{%
\pgfpathmoveto{\pgfqpoint{0.000000in}{0.000000in}}%
\pgfpathlineto{\pgfqpoint{-0.055556in}{0.000000in}}%
\pgfusepath{stroke,fill}%
}%
\begin{pgfscope}%
\pgfsys@transformshift{6.657963in}{2.160713in}%
\pgfsys@useobject{currentmarker}{}%
\end{pgfscope}%
\end{pgfscope}%
\begin{pgfscope}%
\pgftext[x=0.402408in,y=2.160713in,right,]{\rmfamily\fontsize{12.000000}{14.400000}\selectfont \(\displaystyle 5\)}%
\end{pgfscope}%
\begin{pgfscope}%
\pgfsetbuttcap%
\pgfsetroundjoin%
\definecolor{currentfill}{rgb}{0.000000,0.000000,0.000000}%
\pgfsetfillcolor{currentfill}%
\pgfsetlinewidth{0.501875pt}%
\definecolor{currentstroke}{rgb}{0.000000,0.000000,0.000000}%
\pgfsetstrokecolor{currentstroke}%
\pgfsetdash{}{0pt}%
\pgfsys@defobject{currentmarker}{\pgfqpoint{0.000000in}{0.000000in}}{\pgfqpoint{0.055556in}{0.000000in}}{%
\pgfpathmoveto{\pgfqpoint{0.000000in}{0.000000in}}%
\pgfpathlineto{\pgfqpoint{0.055556in}{0.000000in}}%
\pgfusepath{stroke,fill}%
}%
\begin{pgfscope}%
\pgfsys@transformshift{0.457963in}{2.487243in}%
\pgfsys@useobject{currentmarker}{}%
\end{pgfscope}%
\end{pgfscope}%
\begin{pgfscope}%
\pgfsetbuttcap%
\pgfsetroundjoin%
\definecolor{currentfill}{rgb}{0.000000,0.000000,0.000000}%
\pgfsetfillcolor{currentfill}%
\pgfsetlinewidth{0.501875pt}%
\definecolor{currentstroke}{rgb}{0.000000,0.000000,0.000000}%
\pgfsetstrokecolor{currentstroke}%
\pgfsetdash{}{0pt}%
\pgfsys@defobject{currentmarker}{\pgfqpoint{-0.055556in}{0.000000in}}{\pgfqpoint{0.000000in}{0.000000in}}{%
\pgfpathmoveto{\pgfqpoint{0.000000in}{0.000000in}}%
\pgfpathlineto{\pgfqpoint{-0.055556in}{0.000000in}}%
\pgfusepath{stroke,fill}%
}%
\begin{pgfscope}%
\pgfsys@transformshift{6.657963in}{2.487243in}%
\pgfsys@useobject{currentmarker}{}%
\end{pgfscope}%
\end{pgfscope}%
\begin{pgfscope}%
\pgftext[x=0.402408in,y=2.487243in,right,]{\rmfamily\fontsize{12.000000}{14.400000}\selectfont \(\displaystyle 6\)}%
\end{pgfscope}%
\begin{pgfscope}%
\pgfsetbuttcap%
\pgfsetroundjoin%
\definecolor{currentfill}{rgb}{0.000000,0.000000,0.000000}%
\pgfsetfillcolor{currentfill}%
\pgfsetlinewidth{0.501875pt}%
\definecolor{currentstroke}{rgb}{0.000000,0.000000,0.000000}%
\pgfsetstrokecolor{currentstroke}%
\pgfsetdash{}{0pt}%
\pgfsys@defobject{currentmarker}{\pgfqpoint{0.000000in}{0.000000in}}{\pgfqpoint{0.055556in}{0.000000in}}{%
\pgfpathmoveto{\pgfqpoint{0.000000in}{0.000000in}}%
\pgfpathlineto{\pgfqpoint{0.055556in}{0.000000in}}%
\pgfusepath{stroke,fill}%
}%
\begin{pgfscope}%
\pgfsys@transformshift{0.457963in}{2.813774in}%
\pgfsys@useobject{currentmarker}{}%
\end{pgfscope}%
\end{pgfscope}%
\begin{pgfscope}%
\pgfsetbuttcap%
\pgfsetroundjoin%
\definecolor{currentfill}{rgb}{0.000000,0.000000,0.000000}%
\pgfsetfillcolor{currentfill}%
\pgfsetlinewidth{0.501875pt}%
\definecolor{currentstroke}{rgb}{0.000000,0.000000,0.000000}%
\pgfsetstrokecolor{currentstroke}%
\pgfsetdash{}{0pt}%
\pgfsys@defobject{currentmarker}{\pgfqpoint{-0.055556in}{0.000000in}}{\pgfqpoint{0.000000in}{0.000000in}}{%
\pgfpathmoveto{\pgfqpoint{0.000000in}{0.000000in}}%
\pgfpathlineto{\pgfqpoint{-0.055556in}{0.000000in}}%
\pgfusepath{stroke,fill}%
}%
\begin{pgfscope}%
\pgfsys@transformshift{6.657963in}{2.813774in}%
\pgfsys@useobject{currentmarker}{}%
\end{pgfscope}%
\end{pgfscope}%
\begin{pgfscope}%
\pgftext[x=0.402408in,y=2.813774in,right,]{\rmfamily\fontsize{12.000000}{14.400000}\selectfont \(\displaystyle 7\)}%
\end{pgfscope}%
\begin{pgfscope}%
\pgftext[x=0.251367in,y=1.670917in,,bottom,rotate=90.000000]{\rmfamily\fontsize{12.000000}{14.400000}\selectfont \#features}%
\end{pgfscope}%
\end{pgfpicture}%
\makeatother%
\endgroup%

  \caption{Scatter plot and trend lines of average features implemented versus total fuzzings during simulation for
    selected configurations of feature (2 and 4) and project size (1 - 6 features). Test driven development simulations
    are denoted in blue, Waterfall simulations are in red.}
\end{figure*}


%%%%%%%%%%%%%%%%%%%%%%%%%%%%%%%%%%%%%%%%%%%%%%%%%%%%%%%%%%%%%%%%%%%%%%%%%%%%%%%%%%%%%%%%%%%%%%%%%%%%%%%%%%%%%%%%%%%%%%%%

This section describes the results of a case study that was undertaken to evaluate the efficacy of executable workflow
fuzzing for modelling, simulating and predicting behaviours in software systems.  Software development was the problem
domain chosen for the case study, because the familiarity of the authors with the domain and associated workflows
mitigated the risk of infidelity in the domain model.  This allowed the evaluation to focus on the efficacy of the
fuzzing method, rather than the validity of the case study model. The problem domain model is illustrated in Figure
\ref{fig:full-class-diagram}.  The relationships between features, chunks and bugs have already been described in
Section \ref{sec:fuzzi-moss}.  Aspects of the model that have not yet been described are given below.

\begin{figure*}
  \centering
  \includegraphics{floats/full-class-diagram-1}
  \

  \

  \caption{Full class diagram of the software development problem domain.}
  \label{fig:full-class-diagram}
\end{figure*}


The \lstinline!Test! class represents unit tests that are developed against system specifications detailed by the
\lstinline!Feature! class.  The detection of bugs in a feature by tests is set probabilistically, but is deterministic
(once set, a test will either always or never reveal a bug when exercised).  By contrast the introduction of bugs into
chunks by modification; the relationship between bugs and chunk operation; and the introduction of dependencies between
chunks are stochastic and modelled by probabilistic functions.

Software systems are characterised by a collection of features and associated tests.  Like features, software systems
can be operated.  When this happens, features are selected randomly and their own \lstinline!operate()!  operation
invoked.  This feature selection and operation continues until either a maximum trace size is reached, an incomplete
feature is operated, or a bug is manifested.  Software systems also record logs of traces of feature operations and
other characteristics implemented as Python properties, including mean operations to failure of historic trace runs.
Developers perform the work on a software system in terms of development, debugging and refactoring features.  All of
these tasks consume available person time, measured in person time units (ptu), which is tracked as a
\lstinline!Developer! class attribute.  Developers are associated with software systems by the
\lstinline!SoftwareProject! class, which encapsulates a particular run of a software development workflow and subsequent
operation.  Finally, software projects for the same scenario are grouped together in the
\lstinline!SoftwareProjectGroup! class in order to obtain averaged metrics across multiple runs of the same simulation.

Two software development workflows were selected for evaluation in the case study: Waterfall
\citep{benington83production} and Test Driven Development (TDD) \citep{beck02test}.  Again, these workflows were
selected due to their familiarity to the authors.  To gain additional assurance, unit tests were developed for each of
the problem domain and workflow classes to document the expected behaviour of each class independently (using mocks to
replace dependencies) and to gain assurance that the exhibited behaviour without fuzzing was that intended by the
authors.

The implementation of the two workflows, decorated with fuzzers is illustrated in Figure \ref{fig:workflow-impl}.
Although the two workflows describe different behaviours, the implementation has been chosen to enable comparison
between the workflow structures and fuzzers chosen.  Fuzzers were applied based on the judgement of the authors
regarding the typical variance that occurs in software development activities, in terms of omitted or re-ordered steps.
Following implementation, it was hypothesised that:

\begin{itemize}

\item The waterfall workflow would achieve higher completion rates for features than test driven development, given the
  upfront implementation of functionality.

\item TDD would realise higher software quality (expressed as mean operations to failure) due to the emphasis on
  delivering working software.

\item Waterfall would consume less resources in the context of excess resources, due to the termination of the workflow
  as soon as planned quality objectives are met.

\item The behaviour of TDD would be less susceptible to socio-technical variance due to the application of workflow
  fuzzing, due to it's iterative nature.

\end{itemize}


\begin{figure*}
  \centering
  
  \begin{subfigure}[b]{.45\linewidth}
\begin{lstlisting}[basicstyle=\ttfamily\scriptsize]
@fuzz(choose_from( [(0.95, identity), (0.05, remove_random_step)] ) ) def work(system, developer, schedule):
_complete_specification(schedule, system) _implement_features(developer, system) _implement_test_suite(developer,
system) _debug_system(developer, system) _refactor_system(developer, system)

@fuzz(choose_from( [(0.95, identity), (0.05, remove_random_step)] ) ) def _complete_specification( schedule, system):
for feature_size in schedule: system.add_feature(feature_size)

@fuzz(choose_from( [(0.99, identity), (0.01, replace_condition_with(False))] ) ) def _implement_features( developer,
system): for feature in system.features: while not feature.is_implemented: developer.extend_feature(feature)

@fuzz(choose_from( [(0.95, identity), (0.05, replace_condition_with(False))] ) ) def _implement_test_suite( developer,
system): for feature in system.features: while feature.test_coverage < \ target_test_coverage_per_feature:
developer.add_test(feature)

@fuzz( choose_from( [(0.99, identity), (0.01, replace_condition_with(False))] ) ) def _debug_system(developer, system):
for test in system.tests: while True: try: test.exercise() break except BugEncounteredException as e:
developer.debug(test.feature, e.bug)

@fuzz( choose_from( [(0.99, identity), (0.01, replace_condition_with(False))] ) ) def _refactor_system(developer,
system): for feature in system.features: while len(feature.dependencies) > \ target_dependencies_per_feature:
developer.refactor(feature)






''''''
\end{lstlisting}
    \caption{Waterfall}
  \end{subfigure}
  \hfill
  \begin{subfigure}[b]{.45\linewidth}
\begin{lstlisting}[basicstyle=\ttfamily\scriptsize]
@fuzz( recurse_into_nested_steps( target_structures={ast.For, ast.TryExcept}, fuzzer=filter_steps(
exclude_control_structures(), fuzzer=choose_from( [(0.95, identity), (0.05, remove_random_step)]) ) ) ) def work(system,
developer, schedule):

for feature_size in schedule: try: feature = \ system.add_feature(feature_size) _ensure_sufficient_tests(developer,
feature) _complete_feature(developer, feature) _refactor_feature(developer, feature) except DeveloperExhaustedException:
system.features.remove(feature)

while True: try: feature = choice(system.features) _enhance_system_quality(feature, developer) except
DeveloperExhaustedException: break

@fuzz(choose_from( [(0.95, identity), (0.05, replace_condition_with(False))])) def _ensure_sufficient_tests(developer,
feature): while feature.test_coverage <\ target_test_coverage_per_feature: developer.add_test(feature)

@fuzz(choose_from( [(0.99, identity), (0.01, replace_condition_with(False))])) def _complete_feature(developer,
feature): while not feature.is_implemented: developer.extend_feature(feature) _debug_feature(developer, feature)

@fuzz(choose_from( [(0.95, identity), (0.05, remove_random_step)])) def _enhance_system_quality(feature, developer):
developer.add_test(feature) _debug_feature(developer, feature) _refactor_feature(developer, feature)

@fuzz(choose_from( [(0.99, identity), (0.01, replace_condition_with(False))] )) def _debug_feature(developer, feature):
while True: try: feature.exercise_tests() break except BugEncounteredException as e: developer.debug(feature, e.bug)

@fuzz( choose_from( [(0.99, identity), (0.01, replace_condition_with(False))] )) def _refactor_feature(developer,
feature): while len(feature.dependencies) >\ target_dependencies_per_feature: developer.refactor(feature)

''''''
\end{lstlisting}

    \caption{Test Driven Development}
  \end{subfigure}

  \caption{Workflow implementations in Python with fuzzers for Waterfall and Test Driven software development.}
  \label{fig:workflow-impl}
\end{figure*}

A default scenario was configured in which a single developer was tasked with building a software system with a schedule
comprising three features, of three, five and seven chunks in order of priority.  Both workflows aimed to achieve a
target test coverage of 100\% of code chunks and a inter-feature dependency rate of 0.  Dependencies between chunks
within features were unregulated.  Each system was constructed following the specified workflow behaviour, applying any
specified fuzzings.  Then, each resulting system was operated 50 times, allowing for a maximum trace of 750 features per
operation. Each overall build-operate sequence was executed 10 times, resulting in 500 system operation traces per
configuration.

The workflows scenarios were evaluated in twelve configurations, parameterised by workflow type, resource availability
and the presence of fuzzing.  Initially, the experiment was calibrated to determine an adequate resourcing level for the
Waterfall methodology to complete, by executing the Waterfall workflow and determining an average resource consumption.
This was found to be approximately 250ptu.  Then, two further resource parameters were specified: inadequate resources
(50ptu) and excess resources (500ptu).  Each of these three resource configurations were run with workflow fuzzing
disabled and enabled.  Average mean times to failure (feature operation trace length), remaining person time resources,
completed features and simulation run time were recorded and tabulated as shown in Table \ref{tab:results}.

\begin{table}
  \caption{%
    Results for twelve simulation runs of Fuzzi Moss on the software development
    problem domain.  The table shows workflow type, resources allocated (excess,
    adequate, inadequate), whether fuzzings were applied, mean feature operations
    to failure, remaining person unit time available at end of development phase,
    average features asserted as implemented, and simulation runtime.
  }
  \label{tab:results}

  \centering
  \begin{tabular}{|l|r|l|r|r|r|r|l|} \hline
    Workflow & \begin{tabular}{@{}c@{}}Res.\\ PTU\end{tabular}& Fuzz  & MOF & \begin{tabular}{@{}c@{}}Rem.\\ PTU\end{tabular} & Feat. & \begin{tabular}{@{}c@{}}RT\\PTU\end{tabular}\\ \hline

    Waterfall & 50 & No & 14 & 1 & 3.0 & 3 \\
    Waterfall & 250 & No & 72 & 93 & 3.0 & 5 \\
    Waterfall & 500 & No & 72 & 343 & 3.0 & 6 \\
    Waterfall & 50 & Yes & 10 & 4 & 3.0 & 2 \\
    Waterfall & 250 & Yes & 114 & 136 & 2.4 & 5 \\
    Waterfall & 500 & Yes & 114 & 386 & 2.4 & 5 \\
    TDD & 50 & No & 259 & 1 & 1.5 & 8 \\
    TDD & 250 & No & 680 & 1 & 3.0 & 28 \\
    TDD & 500 & No & 750 & 1 & 3.0 & 56 \\
    TDD & 50 & Yes & 259 & 1 & 1.5 & 10 \\
    TDD & 250 & Yes & 513 & 1 & 3.0 & 34 \\
    TDD & 500 & Yes & 750 & 1 & 3.0 & 70 \\ \hline
 

  \end{tabular}
\end{table}

As expected, without fuzzing, the TDD workflows consume all available resource, in contrast to the plan driven approach
followed by Waterfall.  These results confirm the hypothesis that TDD is a `greedy' workflow, absorbing available
resources allocated to a project (\citet{sommerville10software} has argued that this characteristic makes agile like
methodologies such as TDD difficult to develop contracts for).  Note that the similarity in results between the adequate
and excess resource requirements for Waterfall is likely explained by the workflow terminating once the minimum
resources required (250) are consumed, regardless of the excess availability.

The results also confirm the hypotheses that without fuzzing, the simulated TDD workflow achieves higher software
quality than the simulated Waterfall workflow, at the cost of a reduction of features incorporated in the system.  TDD
achieves a mean operation to failure that is higher than Waterfall even when comparing execution of constrained TDD
resources with unconstrained Waterfall (259 vs 114).  This difference is explained by the far greater presence of bugs
and greater complexity of systems built following the Waterfall driven approach, as well as the presence of incomplete
features in the systems built following the Waterfall workflow.  This distinction is particularly marked in the context
of constrained resources, in which test driven development achieves a far higher mean operations to failure (259 vs 14),
but fewer features are delivered (1.5 vs 3.0).

The result of applying fuzzing to the workflows reveals both expected and unexpected behaviours.  On the one hand, the
results support the hypothesis that TDD is relatively resistant to workflow fuzzing, since the measured characteristics
(MOF and Features implemented) are not dramatically affected by the application of fuzzing.  In contrast, the
measurements for the Waterfall model are changed considerably.  However, unexpectedly, fuzzing does not appear to have
reduced quality, but rather to have altered the characteristics to be closer to the behaviour of TDD, with reduced
features implemented and greater software quality.  This result may be explained by the relatively small number of
features to be implemented.  In this circumstance, forgetting to implement a feature (simulated by a fuzzer replacing
this step with a pass) will cause the number of dependencies to be reduced and hence the likelihood of manifesting a
bug.

Finally, the results show the extent to which Fuzzi Moss increases the runtime for a simulation.  This is most
noticeable for the excess resource simulations that operate systems to the specified trace limit.  In the TDD scenario,
Fuzzi Moss adds 20\% to the runtime of the simulation.  Much of this additional cost will be attributable to the dynamic
recompilation of workflow functions each time a fuzzer is applied.  Further investigation with more complex workflow
models and large scenarios is required to assess this cost and determine whether optimisations are necessary.


%%%%%%%%%%%%%%%%%%%%%%%%%%%%%%%%%%%%%%%%%%%%%%%%%%%%%%%%%%%%%%%%%%%%%%%%%%%%%%%%%%%%%%%%%%%%%%%%%%%%%%%%%%%%%%%%%%%%%%%%

\section{Conclusions}
\label{sec:conclusions}

%%%%%%%%%%%%%%%%%%%%%%%%%%%%%%%%%%%%%%%%%%%%%%%%%%%%%%%%%%%%%%%%%%%%%%%%%%%%%%%%%%%%%%%%%%%%%%%%%%%%%%%%%%%%%%%%%%%%%%%%

This paper has presented and evaluated the use of executable workflow fuzzing to the problem of modelling and simulating
variance in socio-technical system behaviours.  The paper described a proof-of-concept workflow fuzzing tool, Fuzzi
Moss, and applied it in a case study of software development workflows.  The workflow tool was demonstrated to introduce
realistic variance into idealised workflows in accordance with expectations of software development workflow in
practice.

The proof of concept has created a substantial research agenda in the application of fuzzing techniques to
socio-technical system modelling.  Within the scope of the Fuzzi Moss project, several immediate next steps are
proposed:

\begin{itemize}

\item Evaluate Fuzzi Moss further by extending the current case study to incorporate more complex aspects of software
  development (such as change management). Other case studies such as the e-counting system described in detail by
  \citet{lock07observations} will also be investigated.  A more complex case study will allow the identification of
  required new fuzzers, as well as areas in which the methodology can be optimised.

\item The implementation of support for concurrent workflows.  Fuzzi Moss simulations are currently executed as a single
  thread.  However, real world socio-technical systems are most conveniently modelling as a collection of concurrently
  interacting workflows.  The modular implementation of Fuzzi Moss should make this extension straight forward.

\item Experimentation with the Fuzzi Moss API (aspects) and extension of fuzz operator capability.  In some respects,
  the current specifications of fuzzers using function decorators is unsatisfactory, since it prevents flexible
  experimentation with different fuzzer configurations.  An aspect-oriented style approach \citep{filman01aspect}, with
  problem domain models oblivious to the application of fuzzers may be more appropriate for this purpose.

\item The development of more rigorous methods for identifying and validating suitable fuzzers for application to
  workflows.  One possibility is the development of recommendations based on the internal structure of a workflow.  An
  alternative option is methods that allow the interactive specification of workflows with project stakeholders.
  Workflows could be developed through the narration of a scenario by a stakeholder, similar to the act of literate
  programming \cite{knuth84literate}.

\end{itemize}

More widely, fuzzing techniques may be applicable for other socio-technical modelling formalisms.  For example, fuzzing
goal oriented models such as \emph{i*} could be used to simulate the shifting goals of actors over time as priorities
and focus varies.  Similarly, fuzzing enterprise modelling techniques such as OBASHI could provide means of assessing
the resilience of critical infrastructures when subject to unexpected behaviours, similar to HAZOPS like techniques that
have been applied manually to responsibility models \cite{lock09modelling}. The proof of concept in this research
demonstrates the potential for the development of realistic simulations of socio-technical systems with predictive
power.  The availability of such tools would do much to progress the current craft of large scale systems engineering.

%%%%%%%%%%%%%%%%%%%%%%%%%%%%%%%%%%%%%%%%%%%%%%%%%%%%%%%%%%%%%%%%%%%%%%%%%%%%%%%%%%%%%%%%%%%%%%%%%%%%%%%%%%%%%%%%%%%%%%%% 

\bibliographystyle{abbrvnat} \bibliography{lib}


%%%%%%%%%%%%%%%%%%%%%%%%%%%%%%%%%%%%%%%%%%%%%%%%%%%%%%%%%%%%%%%%%%%%%%%%%%%%%%%%%%%%%%%%%%%%%%%%%%%%%%%%%%%%%%%%%%%%%%%%


\end{document}
