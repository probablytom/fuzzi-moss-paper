%%%%%%%%%%%%%%%%%%%%%%%%%%%%%%%%%%%%%%%%%%%%%%%%%%%%%%%%%%%%%%%%%%%%%%%%%%%%%%%%

\documentclass{sig-alternate}

\usepackage[T1]{fontenc}

\usepackage[lighttt]{lmodern}

\usepackage[british]{babel}%
\usepackage{cleveref}%
\usepackage{csquotes}%
\usepackage[shortcuts]{extdash}%
\usepackage{natbib}
\usepackage{listings}
\usepackage{subcaption}

\lstset{language=python}
\lstset{basicstyle=\tt\small}
\lstset{keywordstyle=\ttfamily\bfseries}

\newenvironment{FunctionList}{%
\lstset{basicstyle=\ttfamily\bfseries\small}
\begin{list}{}{\leftmargin=5pt}
}{\end{list}\lstset{basicstyle=\ttfamily\small}}

\newcommand{\picalc}{\(\pi\)-calculus }

%%%%%%%%%%%%%%%%%%%%%%%%%%%%%%%%%%%%%%%%%%%%%%%%%%%%%%%%%%%%%%%%%%%%%%%%%%%%%%%%

\title{Simulating Variance in Socio-Technical Workflows using Process Fuzzing}

%\author{Tom Wallis \and Tim Storer}

% \institute{Tom Wallis
%   \at School of Computing Science, University of Glasgow, Glasgow \\
%   \email{twallisgm@gmail.com}
%   \and
%   Tim Storer
%   \at School of Computing Science, University of Glasgow, Glasgow \\
%   \email{timothy.storer@glasgow.ac.uk}
%   }  
  
% \date{Received: XXX / Accepted: XXX}

% \date{}    Do we want a date on the paper? Don't know how these things are done. 

%%%%%%%%%%%%%%%%%%%%%%%%%%%%%%%%%%%%%%%%%%%%%%%%%%%%%%%%%%%%%%%%%%%%%%%%%%%%%%%%

\begin{document}

%%%%%%%%%%%%%%%%%%%%%%%%%%%%%%%%%%%%%%%%%%%%%%%%%%%%%%%%%%%%%%%%%%%%%%%%%%%%%%%%

\maketitle

%%%%%%%%%%%%%%%%%%%%%%%%%%%%%%%%%%%%%%%%%%%%%%%%%%%%%%%%%%%%%%%%%%%%%%%%%%%%%%%%

\begin{abstract}
  The engineering of large scale, complex software based socio-technical systems
  is still very much a craft, dependent on methods based on trial, error and
  subsequent revision.  A particular difficulty is the lack of modelling tools
  that support the analysis of actor behaviours in and around socio-technical
  systems.  Existing notations such as activity diagrams, business process
  modelling languages or petri nets assume that behaviour can be described as
  idealised work flows.  Unfortunately, the behaviour in a socio-technical
  system is highly contingent and subject to considerable variability as actors
  react to changing conditions and identify potential optimisations to their
  practices.  As a consequence, existing techniques result in models that either
  lack sufficient detail to capture the effect of subtle contingencies; are too
  narrow to make useful assessments about the larger system; are unable to
  capture evolution in behaviours; or are so complex that analysis and
  interpretation becomes intractable.

  This paper presents Fuzzi Moss, a novel method for simulating the effect of
  contingent behaviour in socio-technical systems using software code fuzzing.
  A socio-technical system is defined as an object-oriented domain model
  compromising one or more classes.  Behaviours that operate on the system model
  state are described separately as a set of idealised executable work flows.
  These work flows are then annotated with code fuzzers, declarations of how the
  work flow could be dynamically altered (removing steps, duplicating steps,
  introducing new steps for example) when the work flow is executed.  A
  simulation is then configured by executing the fuzzed work flow on an instance
  of the domain model.  Results collected from the domain model state inform
  predictions as to how an idealised work flow will perform in the presence of
  contingent behaviour.
 
  We have implemented a proof of concept tool to evaluate our method. Fuzzi Moss
  is as a small Python library, comprising decorators for annotating workflows
  with desired fuzzers, a mechanism for dynamically inspecting the structure of
  a work flow and applying fuzzers during execution and a library of core
  fuzzers.  We evaluate our approach to simulating socio-technical behaviours by
  applying Fuzzi Moss to a socio-technical system case study.  The results of
  the case study demonstrate the feasibility of the fuzzing method.  Finally, We
  explore the wider applications of fuzzing in socio-technical systems modelling
  and discuss the next steps for the research.
\end{abstract}

%%%%%%%%%%%%%%%%%%%%%%%%%%%%%%%%%%%%%%%%%%%%%%%%%%%%%%%%%%%%%%%%%%%%%%%%%%%%%%%%

\section{Introduction}

%%%%%%%%%%%%%%%%%%%%%%%%%%%%%%%%%%%%%%%%%%%%%%%%%%%%%%%%%%%%%%%%%%%%%%%%%%%%%%%%

Modelling and simulating in large scale, complex socio-technical systems
engineering projects is an on-going research challenge.  Existing methods and
notations are largely derived from a classical engineering abstract, reduce, and
refine technique.  Unfortunately, these principles are confounded by
socio-technical systems problems, which resist the separation of concerns
necessary for abstraction and reduction.  As a consequence, modelling
socio-technical systems using these techniques will typically either result in a
model that is tractable, but lacks the necessary detail of the underlying system
to provide informative results; so narrow in scope as to be uninformative about
the behaviour of the wider system of interest; or so large and complex as to be
intractable for analysis.

A critical aspect of these challenges is the non-determinism and variance in
behaviour exhibited by socio-technical systems.

This variance may be caused by a variety of factors including natural
variability in human decision making, external pressures on the system, such as
loss of resources,...

This paper presents a novel approach to simulating and measuring the effect of
variance in socio-technical behaviours on overall system performance.  Systems
are modelled as collections of idealised workflows, constructed from a modular
toolkit of actions for a problem domain.  Properties for good outcomes.

  These workflows are then subjected to
`fuzzing' in order to simulate the effects of variance in the system.

The rest of this paper is structured as follows

%%%%%%%%%%%%%%%%%%%%%%%%%%%%%%%%%%%%%%%%%%%%%%%%%%%%%%%%%%%%%%%%%%%%%%%%%%%%%%%%

\section{Related Work}

%%%%%%%%%%%%%%%%%%%%%%%%%%%%%%%%%%%%%%%%%%%%%%%%%%%%%%%%%%%%%%%%%%%%%%%%%%%%%%%%

\label{related_head}
Herrmann \& Loser\cite{Herrmann1999} discuss sociotechnical modelling with inherent vagueness, and while this might be shown to be a useful construct, it doesn't address the problem that sociotechnical modelling with uncertainty of action is difficult to model. Instead, Herrmann \& Loser create sociotechnical models using pictorial representations -- but these may be difficult to fuzz programmatically. Therefore, a comparison of current methods for representing sociotechnical systems and introducing variance was necessary. \par

\subsection{Modelling System Requirements}\label{planning_modelling_requirements}
The modelling system desired had some properties that could not be found in the available alternatives. For example, the models had to be procedural, but also easy to write and maintain. For reasons laid out below, Python was the most suitable choice. \par

The language had to be powerful, because the system would need to be capable of creating potentially complex models. The fact that Python is a very powerful, very portable language that allows for readable codebases at scale was a strong influencing factor when deciding what language would suit the construction of a sociotechnical system. Another feature of the models that needed to be implemented was that they must be modular. Python allows for the creation of modules and packages that makes it easy to bring a workflow written for a subsection of a system and integrate it into a larger model of that system. \par

A lesson learned from research on i* (found at \cref{research_istar}) was that the interactions between different people makes a big difference to the end result of the sociotechnical simulation to be built. Therefore, in an effort to simplify the models needed to test the hypothesis, our modelling system need not be capable of modelling multiple concurrent actors. 

\subsubsection{Currently existing platforms}
As a result of the desire to create programmatic models to mutate, it was decided that any pictographic approach would add unnecessary complications. Then, if variance is shown to be modelled successfully using code fuzzing, this technique could then be applied to pictographic methods. \par

\subsubsection{KaOS} 
KaOS is a goal-oriented modelling technique for sociotechnical systems modelling\cite{Werneck2009}. This seemed like an appropriate place to begin, as KaOS is heavily used and cited in the academic community. The actors we model are attempting to achieve or fulfil some goal, so surely this is an appropriate place to start researching current modelling techniques.\par%Goal-oriented model
%Our actors are ultimately trying to fulfil some goal, so surely goal-orientation is what we want?
However, there are some issues with KaOS for the purposes of our models. For one, KaOS is useful if one must elicit software engineering requirements from a client, but in terms of modelling human actors, KaOS is less widely used. Requirements engineering techniques would be inappropriate for a model we intend to fuzz, even if a representation of the goals might also be mutatable. \par 
Ideally these models should be easily verifiable by a human, so either a graphical or human-readable textual format would be good to represent these, but KaOS lends itself well to graphical representations that seemed hard to execute as some simulation. Therefore, while widely taught and used, KaOS doesn't meet our requirements for modelling. \par
%Widely used and seemingly popular for academic purposes -- good if we wanted to do future work or get interest in the project!
%KaOS is often used in Requirements Engineering, and we're really interested in workflow modelling, so maybe best to look elsewhere.

\subsubsection{i*} %TODO
\label{research_istar}
i* is another technique that, while useful for gaining an insight into some sociotechnical system, is ill-suited to our needs as a result of its requirements engineering background\cite{Werneck2009}. In i*, actors rely on each other for goals and responsibilities to be fulfilled. i* therefore represents a better social model for a system than some of its competitors, and is widely cited and popular in the academic community, similar to KaOS. Both are used in situations as different as software engineering\cite{Almisned2010}, artificial intelligence\cite{VanDiggelen2010}, and even comes up in journals for infectious diseases\cite{Tutorial2007}. \par
%Actor intent model
%We're putting stress on actors, so this is interesting.
As a result of i*'s better social modelling than KaOS, we get a slightly different look a sociotechnical modelling and requirements engineering. However, as widely cited and interesting as i* might be, it suffers the same fundamental flaws that KaOS does, the largest of which being its lack of workflow modelling. It is clear to see that interactions between people is an important component of any sociotechnical modelling platform, however. Due consideration should be given as to the social implications in a sociotechnical model and whether any sociotechnical stresses that arise from the social interaction would be useful for this project. \par
%Used for requirements engineering?
%Not really workflow-oriented, which is what we want, but actor orientation is interesting, because we're modelling the actions of people, rather than their goals. Therefore, this tells us something about the system we ultimately want to find/create.

\subsubsection{YAWL} %TODO
YAWL was particularly interesting because of its focus on workflow. YAWL, which stands for Yet Another Workflow Language\cite{hofstede2010yawl}, is a workflow modelling platform with its own tools, which are sophisticated and multi-platform. The creators of YAWL also have interests in areas such as \picalc\cite{Aalst2004}, which makes YAWL interesting for the additional reason that it has some mathematical backing in its representation of workflows. With this said, the lead authors have also published work stating that workflows are much more than simple \picalc processes\cite{Aalst2004}. \par

While workflow is our main concern when searching for candidates to perform fuzzing on, YAWL is largely constructed with a graphical tool, making it difficult to find things to mutate. Ergo, YAWL would be inappropriate as a candidate for fuzzable systems. However, it does get us closer to the modelling system we require, and shows that there is some academic interest in workflow-oriented modelling systems. \par
%Actually a workflow language!
%Graphical
%has a multi-platform editor but maybe not best for our purposes

% Can probably be cut or be very reduced.
\subsubsection{OBASHI}
OBASHI\cite{ObashiMethodology} is a modelling system that focuses on business processes instead of sociotechnical models. However, OBASHI's business process modelling techniques focus on dataflow, meaning that systems have states that get changed as a simulation occurs. OBASHI also focuses on simplicity in its models, which is something that would be required by a sociotechnical modelling framework to ensure that the models created for this project would be at least moderately accurate representations of the systems as a whole. Given that dataflow and workflow have clear correlations, it might be prudent for a fuzzable sociotechnical platform to separate its own concerns by using layered models. \par 

Indeed, focusing away from the academic perspective, many modelling systems are used commercially that are not open to scientific analysis. Few pieces of academic literature exist at all for OBASHI, despite its 15 year existence. It may be prudent to learn from insights into what works in the practically-minded setting of modelling in the private sector. An appropriate way to prune the methods to research would be to observe those that have lasted in a changing technological landscape. Therefore, due attention should be paid to OBAHSI's layered dataflow models, as this would help separate concerns in the modelling system being created, as it does in the OBASHI setting. \par

%%%%%%%%%%%%%%%%%%%%%%%%%%%%%%%%%%%%%%%%%%%%%%%%%%%%%%%%%%%%%%%%%%%%%%%%%%%%%%%%%%%%%%%%%%%%%%%%%%%%%%%
%%%   TOM HAS EDITED TO HERE SO FAR AND HAS MORE TO GO, BUT MUST STUDY. MORE TOMORROW, HOPEFULLY.   %%%
%%%%%%%%%%%%%%%%%%%%%%%%%%%%%%%%%%%%%%%%%%%%%%%%%%%%%%%%%%%%%%%%%%%%%%%%%%%%%%%%%%%%%%%%%%%%%%%%%%%%%%%


\subsection{Code Fuzzing for Sociotechnical Variance}
Little academic research appears to have been done on programmatically inserting sociotechnical variance. A little work has been done into systems for \emph{Punctuated Socio-Technical Information System Change model}s\cite{Lyytinen2008}, but this appears to refer more to the way information systems alter over time and less to do with the study of uncertainty in information systems. \par
Other research has been done into the management of organisational uncertainty\cite{grote2004uncertainty}\cite{Herrmann1999}, which appears to be ideal for the task at hand. Unfortunately, no sociotechnical simulations with uncertainty injected into the models created could be found. \par
Therefore, while sociotechnical modelling is now a well-established field with competing methodologies, and sociotechnical uncertainty has also been a research subject since as early as 1976\cite{Susman1976}, no programmatic sociotechnical modelling has been documented which models sociotechnical stress as a component of that model. Therefore, the hypothesis of the project is clear; that is, to construct such a system and attempt to verify that sociotechnical stress can be introduced to the system by means of code fuzzing. \par

\subsubsection{Introducing Variance}
\label{Variance_research}
Currently, variance must be introduced to a programmatic model by hand. That is, as a construction of a model, we must build the sociotechnical stress into the model rather than applying it after the model has been created. \par

\subsubsection{Graphical representations and UML}
\label{pictorial}
Pictorial representations of sociotechnical systems can be very human-readable, which makes them suitable candidates for creating a sociotechnical model. Modelling systems such as OBASHI\cite{ObashiMethodology} use graphical representations of sociotechnical systems to build models of business and IT processes, but build these models using dataflow. To contrast, a UML Use Case diagram\cite{omg2010omguml} models an actor within a larger system to visualise a sociotechnical system. Both models use graphical representations of the systems they model to make the complexity of the model easy for a human to parse, but these models can be difficult for a computer to parse. \par
UML tries to overcome this by being equally readable by a computer: frameworks like the Eclipse Modelling Framework can generate Java code from a UML class diagram, for example\cite{EMFManual}. UML also succeeds in creating a standard for colloquial "flowcharts" by creating well-defined specifications such as the UML Activity Diagram. Therefore, using UML as a format for laying out a sociotechnical system that can be easily created and read by humans, but is also easily computer-parsable, seemed to be an appropriate way to create the programmatic models.

\subsection{Fuzzing Library Requirements}\label{planning_fuzzing}
Having researched what was already available in the academic and commercial spaces, it was decided after some deliberation that the best option was to define our own modelling system and create our own fuzzing library.  \par 
The fuzzing library desired also had unique properties that made it difficult to find an equivalent for in a commercial or academic setting. \par
The fuzzing library was of particular importance because the main study of the project was on variance in sociotechnical systems. Therefore, the fuzzing library had a particularly strict set of requirements. the fuzzing library needed to have fairly precise control over the activity of the mutations, because the mutations it introduced needed to be representative of sociotechnical stress. Ideally, it would be possible that the mutation testing would not be a black box, so that the exact nature of the sociotechnical stress could be discerned. Therefore, if a custom library was not to be made, then the chosen library would need to be open sourced. \par
It transpired that the library would need to be built to spec anyway, because the library needed to generate different mutations on every call to the functions it was fuzzing. This meant that not only would specific parts of the procedural model need to be targeted, but the library would either have to intercept function calls somehow or fuzz the function to randomly select one of many fuzzed versions of itself at execution time. This functionality was \emph{critical}, because human variance in a workflow is to change the behaviour in some random way every time that behaviour is executed. With flows and atoms acting as blueprints for behaviour, these blueprints would need to be altered when enacted, just as in real life, to make a meaningful simulation. No available code fuzzing library could be found that could do fuzzing during the execution of the fuzzed program, so a custom library needed to be built. \par
With the fuzzing library being such a critical part of the research, none of these points could be compromised on. In addition, a custom built library would allow fuzzing methods to adapt and grow as the modelling method matured. Therefore, a custom fuzzing library was made. \par

\subsubsection{Existing Fuzzing Platforms}
\label{research_fuzzing}
Because the systems we were interested in creating were to be self-documenting, we chose Python as a language to write our models in. Python's clarity combined with the ease of quickly writing working Python code meant that self-documenting models could be feasibly made. In addition, Python supplied language features such as decorators and a built in Abstract Syntax Tree library that made it an ideal candidate. \par
We therefore set about finding appropriate fuzzing libraries in Python, and found several:
\subsubsection{Sulley}
\label{fuzzing_sulley}
Sulley is a fuzzing library for Python which is popular for remote application and protocol fuzzing. Sulley alters the protocol by which it inserts data into an application or system, and observes how that application reacts to the altered input, to simulate problems with networks and human users. It is under active development, and is open sourced.\par
While using a popular, open-source library for inspiration when creating our own mutation system, Sulley was inappropriate for our needs.\par
Sulley fuzzes by way of changing protocols and inputs to some system it interacts with. We felt that altering inputs to that system lacked the degree of control that changing workflows directly would provide. Technically, it would be possible to get around this problem by creating models in some remote program that took instructions from some fuzzed input, but we felt that this additional degree of abstraction would lower the self-documenting nature of our models and would be an impractical modelling technique anyway. \par

\subsubsection{Fusil}
\label{fuzzing_fuzil}
Fusil was another fuzzing library that was investigated. Fusil allows for fuzzing the environment of a python program, which made it appealing to us. While environmental factors impact the sociotechnical model however, these factors are unrelated to the properties Fusil is capable of monitoring. \par
Fusil is able to create mangled files as fuzzed/invalid input to a program, monitors and limits CPU and memory usage, and checks logs for system errors and other signs that a fuzzed program was not functioning correctly under adverse circumstances. This seemed appealing, because monitoring performance under adverse circumstances is precisely our goal in this project. \par
Unfortunately Fusil also did not meet our requirements: it monitors the effect of a change in computational environment on a program, where we are interested in the effect of the change of a social environment on a workflow. While our workflows are being modelled as programs, this does not mean that the workflows themselves would be being altered, so stress on a sociotechnical system still wouldn't be being modelled. Fusil was a further worse choice when taking into account the fact that we would not be able to pinpoint the errors that were being introduced by observing memory and CPU usage, and other computational concerns. \par
Therefore, we could not use Fusil as our fuzzing system of choice. \par

\subsubsection{MutPy} 
\label{fuzzing_mutpy}
From the MutPy package page\cite{mutpy26}: 
\begin{displayquote}
MutPy is a mutation testing tool for Python 3.x source code. MutPy supports standard unittest module, generates YAML reports and has colorful output. It’s apply mutation on AST level. You could boost your mutation testing process with high order mutations (HOM) and code coverage analysis.
\end{displayquote}\par

It seemed that MutPy supplied \emph{everything} required. This would be beneficial, but unfortunately, MutPy had a relatively large and complex codebase\cite{khala8:online} that made it hard to modify for the specific requirements of the research. In addition, relying on a complex dependency that is designed for unit testing seemed to be a bad approach, as research interests might not align with MutPy's maintainers' in the future. As a result, it might have been necessary to fork their codebase at the time and maintain our own version for research purposes; this approach was not future-proof, and so an alternative approach was deemed necessary. \par

Working in MutPy's favour, however, was that it was a \emph{code fuzzing} rather than protocol fuzzing library. Particularly, it fuzzed source code to a similar end as something like Java's PiTest\cite{coles14pitest}. This was more in line with our goals, although MutPy still didn't deliver what was required from a code fuzzing library for this experiment. Therefore, it was decided to learn from MutPy's strong points and implement something similar, but from a fresh, maintainable codebase with interests more aligned with that of the research. This would also allow for potential improvements to be implemented in the future, which could not be determined before modelling began; as it happens, this decision turned out to pay off, as additional functionality was needed. \par

\subsubsection{PyMuTester}
\label{fuzzing_pymutester}
PyMuTester\cite{PyMutTester} proved to be the closest to the library required, though it still fell short in some areas. Particularly, PyMuTester did not appear to have many types of mutation it could enact. \par
PyMuTester did not have a large codebase\cite{PyMutTester}, nor did it have very much functionality to offer, but it offered mutation via Abstract Syntax Tree editing. The decision was made to study the PyMuTester source as a template for a code fuzzing library of a new design. PyMuTester seemed to be much too immature as a library to base the whole project around, and was last updated in October 2014. However, it did show promise, and gave a basis around which writing a code mutation library could be built, while also taking cues from the more advanced source of MutPy\cite{khala8:online}.  \par%\cite{} PyMuTester source at https://github.com/miketeo/PyMuTester       \cite{} MutPy's source at https://bitbucket.org/khalas/mutpy/src


%%%%%%%%%%%%%%%%%%%%%%%%%%%%%%%%%%%%%%%%%%%%%%%%%%%%%%%%%%%%%%%%%%%%%%%%%%%%%%%%

\section{Socio-Technical Workflow Fuzzing}

%%%%%%%%%%%%%%%%%%%%%%%%%%%%%%%%%%%%%%%%%%%%%%%%%%%%%%%%%%%%%%%%%%%%%%%%%%%%%%%%

In this section we introduce our approach to modelling idealised workflows in
socio-technical systems and then introducing variance in their execution through
fuzzing.  Iimplementation details of the the proof of concept library, Fuzzi
Moss, are also presented.  Source code for the Fuzzi Moss library is available
from the project's GitHub repository \citep{storer2016fuzzi-moss-scm-repo}.  The
method comprises: construction of an instrutmented object oriented model of the
problem domain; construction of an idealised representations of socio-technical
workflows that direct the specify the behaviours of actors in the problem
domain; annotation of the work flows with fuzzers to introduce variance;
execution of the workflows in scenarios; and measurement of outcomes in the
scenarios through observation of the problem domain state.

Each stage of the method is described in more detail below, using the task of
developing software as an illustrative example (we will present the full case
study in Section \ref{sec:evaluation}). Source code for this case study is also
available from the GitHub repository for the study
\citep{storer2016softdev-workflow-scm}.

%%%%%%%%%%%%%%%%%%%%%%%%%%%%%%%%%%%%%%%%%%%%%%%%%%%%%%%%%%%%%%%%%%%%%%%%%%%%%%%%

\subsection{Problem Domain Modelling}

%%%%%%%%%%%%%%%%%%%%%%%%%%%%%%%%%%%%%%%%%%%%%%%%%%%%%%%%%%%%%%%%%%%%%%%%%%%%%%%%

The initial step in the fuzzi moss method is to develop a model of the target
socio-technical problem domain.  The model represents the invariant `physics' of
the problem domain that cannot be altered because it represent hard constraints
in the real world.  The model physics may represent constraints imposed by
natural laws, such as the conservation of energy; deliberate technical system
design, such as an access control client that limits access to a software system
to those who know an appropriate username and password combination; or emergent
phenomena, such as the tendency for the implementation of new features in a
software system to introduce bugs.

We have adopted an object-oriented approach to modelling the elements of the
problem domain \citep{UMLtextbook}, so that domain artifacts are represented as
instances of classes.  The state of a particular scenario in a problem domain is
therefore represented by a collection of inter-related Python objects
constructed from the classes specified in the domain model.  Figure
\ref{fig:feature-class-diagram} shows part of a class diagram for the software
development problem domain case study.  The diagram shows classes for system
feature (representing user-facing specifications of the system's functionality;
code chunks, representing the implementation details of the system and bugs
introduced into chunks during the completion of features.  Figure
\ref{fig:feature-instance-diagram} illustrates an example state for the same
problem domain using an object diagram, showing a feature implemented with two
chunks, one of which contains a bug.

\begin{figure}
  \centering

  \begin{subfigure}{\linewidth}
    
    \caption{Class Diagram}
  \end{subfigure}

  \begin{subfigure}{\linewidth}
    
    \caption{Object diagram}
  \end{subfigure}

  
  \caption{Object oriented models of a software development problem domain,
    using the UML notation.}
  \label{fig:feature-class-diagram}
\end{figure}

The state of the problem domain can only be altered through explicit actions on
the domain (referred to as domain actions here), that are modelled as operations
on the domain classes.  In the example given in Figure \ref{fig:class-diagram},
the state of a feature can be altered by invoking the \lstinline!extend()!
operation.  This adds a new chunk to the feature's implementation and also has
two side effects. The addition of new chunks may introduce bugs into the newly
created chunk, as well as create dependencies between the new chunk and other
chunks already in the system as a whole.  Consequently, operations are also
provided for refactoring (to reduce dependencies) and debugging (to remove bugs)
the feature.

Very few restrictions are placed on the design of the problem domain operations.
Operations can accept a variety of arguments, modify the class state, invoke
operations on other problem domain classes and return values as desired.
However The operations on the problem domain should be modelled as atomic and
independent, allowing them to be safely invoked in any combination.

The object oriented problem domain model is implemented as a collection of
Python classes.  Model state is represented by python instance attributes.
Operations are implemented as Python methods in the appropriate classes.
Problem domain classes can be extended with supplementary properties using the
Python \lstinline!@property! decorator in order to improve readability as
normal.  The state of the problem domain can therefore be accessed and read as
desired for measuring behaviour in scenarios, or instrumented to record relevant
events for later analysis.

%%%%%%%%%%%%%%%%%%%%%%%%%%%%%%%%%%%%%%%%%%%%%%%%%%%%%%%%%%%%%%%%%%%%%%%%%%%%%%%%

\subsection{Workflow Modelling}

%%%%%%%%%%%%%%%%%%%%%%%%%%%%%%%%%%%%%%%%%%%%%%%%%%%%%%%%%%%%%%%%%%%%%%%%%%%%%%%%

The next stage of the method concerns the development of descriptions of
socio-technical behaviours of actors in the problem domain, called workflows.
These workflows orchestrate the sequence of actions by an actor on the artifacts
in the problem domain.  Behaviours are modelled as \emph{idealised} work flows,
i.e. the behaviour desired or expected of an actor in a socio-technical system
without concern for contingencies.  Workflows are modelled as activity diagrams,
which enables the construction of a hierarchy of workflow descriptions,
supporting modularity and reuse.

Two related work flows are illustrated in Figure \ref{fig:workflow-partial}.
Figure \ref{fig:workflow-tdd} illustrates a Test Driven Development workflow, in
which a software developer specifies an implementation, creates a test case,
implements and debugs the feature and finally refactors the functional
implementation.  Figure \ref{fig:workflow-refactoring} illustrates the more fine
grained workflow for refactoring.  A developer, continues to refactor the
feature until the measured coupling rate is less than a desired maximum.  Both
workflows are executed by invoking operations on a \lstinline!Developer!
instance.

\begin{figure}
  \centering

  \begin{subfigure}{\linewidth}


    \caption{Test driven development}
  \end{subfigure}

  \begin{subfigure}{\linewidth}
    
    \caption{Refactoring}
  \end{subfigure}
  
  \caption{Partial socio-technical workflows for software development expressed
    as UML activity diagrams.}

  \label{fig:workflow-partial}
\end{figure}


The workflow models are implemented as collections of Python functions.  As for
the domain actions, implementation of the workflow functions is
flexible. Workflow functions can be parameterised as and may also return values
as desired.  Workflow functions can be encapsulated into classes to support
model maintenance, although this is not necessary for the method itself.  One
restriction that is made is that workflow functions should not contain nested
function definitions, as these may be inadvertently fuzzed by the Fuzzi Moss
mechanism.

%%%%%%%%%%%%%%%%%%%%%%%%%%%%%%%%%%%%%%%%%%%%%%%%%%%%%%%%%%%%%%%%%%%%%%%%%%%%%%%%

\subsection{Specifying the Fuzzers for a Workflow}

%%%%%%%%%%%%%%%%%%%%%%%%%%%%%%%%%%%%%%%%%%%%%%%%%%%%%%%%%%%%%%%%%%%%%%%%%%%%%%%%

The workflow functions implemented in Python are denoted as eligible for fuzzing
by applying the \lstinline!@fuzz! decorator provided in the Fuzzi\_Moss
package. The decorator's constructor accepts one argument, a pointer to a
\emph{fuzzer} function, that will be used in the fuzzing mechanism described
below.  Python's decorator mechanism intercepts invocations of the decorated
function and allows these to be substituted with an alternative function to be
invoked.  The Fuzzi\_Moss decorator exploits this mechanism by defining and
returning a \lstinline!wrap()!  function that is able to dynamically fuzz the
decorated function each time it is invoked.  The wrap function accepts the same
set of arguments as the decorated function and should be expected to return the
same values. The decorator's \lstinline!__call__()!  method, which intercepts
the invocation of the decorated function, defines the wrap function as follows.

\begin{enumerate}

\item The abstract syntax tree (AST) for the decorated function is constructed,
  using the Python \lstinline!inspect! package to recover the function's source
  code and the \lstinline!ast!  package to build the AST from the source.  This
  reference AST is then cached for future reference.

\item A copy of the AST is made.  An AST visitor (called a transformer) is then
  constructed and given the decorator's fuzzer attribute as an argument.  The
  copied syntax tree is then passed to the visitor.

\item The visitor identifies the function definition node in the AST and applies
  the fuzzer to the function definition's body.  The body is represented as a
  list of Python statements in the AST.  The function body is replaced by the
  value returned by the fuzzer.

\item Finally, the now fuzzed AST is compiled to Python byte code.  The byte
  code of the fuzzed function is then substituted for the decorated function's
  byte code and the function is invoked with its original parameters.  Control
  is then passed back to the Python interpreter, allowing any returned values to
  be passed back as normal.

\end{enumerate}

An example of the application of a fuzzer (identity) is shown in Figure
\ref{fig:fuzz}.

 \begin{figure}
   \centering
  \begin{lstlisting}
@fuzz(identity)
def _refactor_feature(developer, feature):
    while len(feature.dependencies) > \
        self.target_dependencies_per_feature:
        developer.refactor(feature)
  \end{lstlisting}
  
  \caption{Example fuzz decorator applied to a software development workflow.}
  \label{fig:fuzz}
\end{figure}

The generic definition of the fuzzer function (accept and return a list of
Python statement ASTs) allows a user to define a wide range of fuzzing
mechanisms to suit their problem domain.  In addition, a suite of fuzzers is
also provided with Fuzzi\_Moss that allow the modular construction of more
complex fuzzer behaviours.  The behaviour of each of the fuzzers available in
Fuzzi\_Moss are described in the following sub sections.

%%%%%%%%%%%%%%%%%%%%%%%%%%%%%%%%%%%%%%%%%%%%%%%%%%%%%%%%%%%%%%%%%%%%%%%%%%%%%%%%

\subsection{Simple Fuzzers}

%%%%%%%%%%%%%%%%%%%%%%%%%%%%%%%%%%%%%%%%%%%%%%%%%%%%%%%%%%%%%%%%%%%%%%%%%%%%%%%%

A simple fuzzer accepts and returns a list of \lstinline!ast.Statement! objects
and can be declared for use directly within a fuzz decorator.  The following
simple fuzzers are defined in Fuzzi\_Moss.

\begin{FunctionList}

\item\lstinline!identity! returns the input list of statements.  The identity
  fuzzer is used as a default for the fuzz decorator, but is also useful when
  building composite fuzzers.

\item\lstinline!replace_steps_with_passes! returns a list of \lstinline!pass!
  statements of the same length as the input.  Replacing statements with pass is
  safer than removing the statement, since a Python function must be defined
  with at least one statement.

\item\lstinline!duplicate_steps! returns a list containing the input sequence
  repeated twice.

\item\lstinline!shuffle_steps! returns a randomly shuffled list of the input.  A
  Python Random object \lstinline!fuzzi_moss_random! is used as a random source.

\item \lstinline!swap_if_blocks! switches the body and orelse blocks of all if
  statements in the input.

\end{FunctionList}

We have already seen the application of simple fuzzers in Figure \ref{fig:fuzz}.
Note that the fuzzer is supplied to the fuzz decorator as a function pointer,
rather than as an evaluated function call.

%%%%%%%%%%%%%%%%%%%%%%%%%%%%%%%%%%%%%%%%%%%%%%%%%%%%%%%%%%%%%%%%%%%%%%%%%%%%%%%%

\subsection{Fuzzing Filters}

%%%%%%%%%%%%%%%%%%%%%%%%%%%%%%%%%%%%%%%%%%%%%%%%%%%%%%%%%%%%%%%%%%%%%%%%%%%%%%%%

Sometimes it is desirable to restrict the application of simple fuzzers to
particular portions of the body of a workflow function.  In this situation the
\lstinline!filter_steps()! fuzzer can be used.  Filter steps takes two
arguments: a filter function pointer and a fuzzer function pointer.  These are
used to define a nested \lstinline!_filter_step! fuzzer that behaves like a
fuzzer as described above.

The filter fuzzer accepts a list of statements and returns a list of tuples.
Each tuple defines gives the start and end index for a block in the input.  Each
block is then fuzzed using the supplied fuzzer.  Blocks not specified by the
filter are not affected by the fuzzer, such that the returned sequence of
statements may contain a mix of fuzzed and un-fuzzed statements.

The following filters are available for use in conjunction with a fuzzer.


\begin{FunctionList}

\item \lstinline!choose_last_step! returns the start and end index of the last
  statement in the input, i.e. \lstinline!input[-2:-1]!

\item \lstinline!choose_random_steps(n)!  returns $n$ length 1 sub-blocks
  randomly selected from the input steps.

\item \lstinline!exclude_control_structures(target)! permits the exclusion of
  control structure statements by type as specified by the target input
  argument. Supported structures are \lstinline!for!, \lstinline!while!,
  \lstinline!if!, \lstinline!try!-\lstinline!except! and \lstinline!return!.

\end{FunctionList}

Several pre-defined filtering fuzzers are implemented in Fuzzi\_Moss combining
some of the predefined filters and simple fuzzers.  These provide commonly
required fuzzers, including removing or duplicating the last step or a random
step. A \lstinline!choose_identity! filter is also included for completeness.

%%%%%%%%%%%%%%%%%%%%%%%%%%%%%%%%%%%%%%%%%%%%%%%%%%%%%%%%%%%%%%%%%%%%%%%%%%%%%%%%

\subsection{Composite Fuzzers}

%%%%%%%%%%%%%%%%%%%%%%%%%%%%%%%%%%%%%%%%%%%%%%%%%%%%%%%%%%%%%%%%%%%%%%%%%%%%%%%%

More complex fuzzers can be assembled using composite fuzzing functions that are
implemented in a similar manner to filtering.  


\begin{FunctionList}
\item \lstinline!in_sequence(fuzzers)! applies each fuzzer found in the input
  list of fuzzers in sequence to a function body.

\item \lstinline!choose_from(distribution)! selects a fuzzer to apply at random
  from the supplied probability distribution.  The distribution is defined as a
  list of weight, fuzzer tuples.

\item \lstinline!on_condition_that(condition, fuzzer)! %
  applies the specified fuzzer if the specified condition holds.  The condition
  may be a literal boolean value, a function pointer or a Python expression.
  Lambda expressions are not supported.

\item \lstinline!recurse_into_nested_steps(target_structures)! identifies
  control structure statements and applies the supplied fuzzer to their body
  blocks.  The recursion can be limited to particular control structures using
  the option \lstinline!target_structures! argument.
\end{FunctionList}

An example of applying a composite fuzzer to a workflow is shown in Figure
\ref{fig:composite}.  The figure shows a workflow for enhancing the quality of a
software system.  A developer adds a test to a feature in order to detect the
presence of bugs, or prevent the introduction of bugs during refactoring.  The
developer then performs some debugging work, followed by some refactoring.  The
fuzzer applied to the workflow is the \lstinline!choose_from!.  The supplied
distribution will fuzz the workflow 5\% of the time on average, removing one
step at random when it does so.  The figure therefore shows a convenient way of
modelling the occasional random omission of steps in a workflow.

\begin{figure}
  \centering

\begin{lstlisting}
@fuzz(
  choose_from([
    (0.95, identity),
    (0.05, remove_random_step)
  ])
)
def _enhance_system_quality(
    self, feature, developer):

    developer.add_test(feature)
    self._debug_feature(developer, feature)
    self._refactor_feature(developer, feature) 
\end{lstlisting}
  
  \caption{Application of a composite fuzzer to a workflow.}
  \label{fig:composite}
\end{figure}

%%%%%%%%%%%%%%%%%%%%%%%%%%%%%%%%%%%%%%%%%%%%%%%%%%%%%%%%%%%%%%%%%%%%%%%%%%%%%%%%

\subsection{Control Structure Fuzzers}

%%%%%%%%%%%%%%%%%%%%%%%%%%%%%%%%%%%%%%%%%%%%%%%%%%%%%%%%%%%%%%%%%%%%%%%%%%%%%%%%

The final set of fuzzers provide for manipulation of control structure
conditions and iterators.  

\begin{FunctionList}

\item \lstinline!replace_condtion_with(condition)! replaces conditions
  discovered in control structure statements with the specified condition.  The
  condition may be a literal boolean value, a function pointer or a Python
  expression defined in a string.  Lambda expressions are not supported.

\item \lstinline!replace_for_iterator_with(iterator)! replaces iterators
  discovered in for loops with the specified iterator.  The replacement iterator
  must be specified as a list containing numerical or string literal types only.

\end{FunctionList}

%%%%%%%%%%%%%%%%%%%%%%%%%%%%%%%%%%%%%%%%%%%%%%%%%%%%%%%%%%%%%%%%%%%%%%%%%%%%%%%%

\subsection{Configuration Options}

%%%%%%%%%%%%%%%%%%%%%%%%%%%%%%%%%%%%%%%%%%%%%%%%%%%%%%%%%%%%%%%%%%%%%%%%%%%%%%%%

The Fuzzi\_Moss package exposes variables in the \verb!__init__.py! file.

  This
  object can be replaced in order to seed fuzzing for repeatable
  experimentation.

%%%%%%%%%%%%%%%%%%%%%%%%%%%%%%%%%%%%%%%%%%%%%%%%%%%%%%%%%%%%%%%%%%%%%%%%%%%%%%%%

\section{Evaluation}

%%%%%%%%%%%%%%%%%%%%%%%%%%%%%%%%%%%%%%%%%%%%%%%%%%%%%%%%%%%%%%%%%%%%%%%%%%%%%%%%

%%%%%%%%%%%%%%%%%%%%%%%%%%%%%%%%%%%%%%%%%%%%%%%%%%%%%%%%%%%%%%%%%%%%%%%%%%%%%%%%

\section{Conclusions}

%%%%%%%%%%%%%%%%%%%%%%%%%%%%%%%%%%%%%%%%%%%%%%%%%%%%%%%%%%%%%%%%%%%%%%%%%%%%%%%%

\bibliographystyle{plainnat}
\bibliography{lib}


%%%%%%%%%%%%%%%%%%%%%%%%%%%%%%%%%%%%%%%%%%%%%%%%%%%%%%%%%%%%%%%%%%%%%%%%%%%%%%%%

\end{document}