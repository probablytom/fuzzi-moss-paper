%%%%%%%%%%%%%%%%%%%%%%%%%%%%%%%%%%%%%%%%%%%%%%%%%%%%%%%%%%%%%%%%%%%%%%%%%%%%%%%%

\documentclass{sig-alternate}

\usepackage[T1]{fontenc}

\usepackage[lighttt]{lmodern}

\usepackage[british]{babel}%
\usepackage{cleveref}%
\usepackage{csquotes}%
\usepackage[shortcuts]{extdash}%
\usepackage[numbers]{natbib}

\usepackage{listings}
\usepackage{subcaption}

\lstset{language=python}
\lstset{basicstyle=\ttfamily\small}
\lstset{keywordstyle=\ttfamily\bfseries}
\lstset{frame=single}

\newenvironment{FunctionList}{%
\lstset{basicstyle=\ttfamily\bfseries\small}
\begin{list}{}{\leftmargin=5pt}
}{\end{list}\lstset{basicstyle=\ttfamily\small}}

\newcommand{\picalc}{\(\pi\)-calculus }

\hyphenation{poli-cies}
%\hyphenation{tech-ni-cal}

%%%%%%%%%%%%%%%%%%%%%%%%%%%%%%%%%%%%%%%%%%%%%%%%%%%%%%%%%%%%%%%%%%%%%%%%%%%%%%%%

\title{Simulating Variance in Socio-Technical Behaviours using Executable Workflow Fuzzing}

%\author{Tom Wallis \and Tim Storer}

% \institute{Tom Wallis
%   \at School of Computing Science, University of Glasgow, Glasgow \\
%   \email{twallisgm@gmail.com}
%   \and
%   Tim Storer
%   \at School of Computing Science, University of Glasgow, Glasgow \\
%   \email{timothy.storer@glasgow.ac.uk}
%   }  
  
% \date{Received: XXX / Accepted: XXX}

% \date{}    Do we want a date on the paper? Don't know how these things are done. 

%%%%%%%%%%%%%%%%%%%%%%%%%%%%%%%%%%%%%%%%%%%%%%%%%%%%%%%%%%%%%%%%%%%%%%%%%%%%%%%%

\begin{document}

%%%%%%%%%%%%%%%%%%%%%%%%%%%%%%%%%%%%%%%%%%%%%%%%%%%%%%%%%%%%%%%%%%%%%%%%%%%%%%%%

\maketitle

%%%%%%%%%%%%%%%%%%%%%%%%%%%%%%%%%%%%%%%%%%%%%%%%%%%%%%%%%%%%%%%%%%%%%%%%%%%%%%%%

\begin{abstract}
  The engineering of large scale, complex software based socio-technical systems
  is still very much a craft, dependent on methods based on trial, error and
  subsequent revision.  A particular difficulty is the lack of modelling tools
  that support the analysis and prediction of actor behaviours in and around
  socio-technical systems.  Existing notations such as activity diagrams,
  business process modelling languages or Petri Nets assume that behaviour can
  be described as idealised workflows.  Unfortunately, the behaviour in a
  socio-technical system is highly contingent and subject to considerable
  variability as actors react to changing conditions and identify potential
  optimisations to their practices.  As a consequence, existing techniques
  result in models that either lack sufficient detail to capture the effect of
  subtle contingencies; are too narrow to make useful assessments about the
  larger system; are unable to capture evolution in behaviours; or are so
  complex that analysis and interpretation becomes intractable.

  This paper presents Fuzzi Moss, a novel method for simulating the effect of
  contingent behaviour in socio-technical systems using software code fuzzing.
  A socio-technical system is represented as an object-oriented domain model
  compromising one or more classes.  Behaviours that operate on the system model
  state are described separately as a set of idealised executable workflows.
  These workflows are then annotated with code fuzzers, declarations of how the
  workflow could be dynamically altered (removing steps, duplicating steps,
  introducing new steps for example) when the workflow is executed.  A
  simulation is then configured by executing the fuzzed workflow on an instance
  of the domain model.  Results collected from the domain model state inform
  predictions as to how an idealised workflow will perform in the presence of
  contingent behaviour in the real world.
 
  We have implemented a proof of concept tool to evaluate our method. Fuzzi Moss
  is as a small Python library, comprising a decorator for annotating executable
  workflows with desired fuzzers, a mechanism for dynamically inspecting the
  structure of a workflow and applying fuzzers during execution and a library of
  core fuzzers.  We evaluate our approach to simulating socio-technical
  behaviours by applying Fuzzi Moss to a socio-technical system case study.  The
  results of the case study demonstrate the feasibility of the fuzzing method.
  Finally, We explore the wider applications of fuzzing in socio-technical
  systems modelling and discuss the next steps for the research.
\end{abstract}

%%%%%%%%%%%%%%%%%%%%%%%%%%%%%%%%%%%%%%%%%%%%%%%%%%%%%%%%%%%%%%%%%%%%%%%%%%%%%%%%

\section{Introduction}
\label{sec:introduction}

%%%%%%%%%%%%%%%%%%%%%%%%%%%%%%%%%%%%%%%%%%%%%%%%%%%%%%%%%%%%%%%%%%%%%%%%%%%%%%%%

Modelling and simulating in large scale, complex socio-technical systems
development projects is an on-going research challenge \citep{baxter08learning}.
Socio-technical systems are characterised by the interactions between a variety
of actor types within a system's scope, including individual human operators,
organisational units and technical artefacts
\citep{Susman1976,elhassan07role,sommerville08socio}. These systems also
incorporates a complex variety of structures that direct or influence actor
and system behaviour, including workflow descriptions, organisational
hierarchies, regulatory or legal standards, organisational policies and culture,
interpersonal relationships and actor responsibilities
\citep{bade07structures,pentland05organisational}. Examples of such systems with
complex workflows involving multiple actors include emergency vehicle dispatch
\citep{robinson96limited}, electronic voting systems
\citep{bryans04towards,lock07observations}, patient care in a neo-natal unit
\citep{baxter07evaluating} and electronic stock exchange infrastructures
\citep{cftc-sec10findings}.  Systems of this form confound traditional
approaches to modelling, simulating and predicting behaviour for several
reasons.

\begin{itemize}

\item Socio-technical systems are typically very large, constituted of
  autonomous actors and components, each with their own behaviour that may be
  influenced by other actors in the system
  \cite{crabtree00ethnomethodologically}.  This intrinsic scale and complexity
  makes refinement, reduction and abstraction, the conventional tools of the
  systems engineer, difficult to apply, since individual behaviours are
  difficult to classify and aggregate.  One study has advocated the development
  of systems engineering methods based on the metaphor of city maintenance,
  rather than individual building construction \citep{feiler06ultra} to meet
  this system-of-systems challenge, although no methods are presented for making
  this transition.

\item The behaviour is contingent on unpredictable circumstances, including both
  factors in the environment and concerning the system actors.  For example, the
  time and manner in which a task, such as developing a new feature for a
  software system, is completed may vary considerably between actors with
  different training and experiences.  Similarly, the decision to work on a task
  at all may depend on unpredictable and uncontrollable external circumstances
  (such as a power outage).  In these circumstances, actors may also take it
  upon themselves to complete tasks outwith expected workflows in order to
  discharge their responsibilities, by working from a nearby cafe for example,
  even if this violates organisational security policies.  As
  \citet{besnard03human} note, such adaptations often make the human actors the
  dependable parts of a socio-technical system.

\item Behaviour is continually evolving, as the autonomous actors in a system
  adapt to new circumstances, discover optimisations to their workflows, adapt
  the workflow to suit local organisational priorities or take shortcuts
  \citep{bonen79evolutionary,Lyytinen2008,anderson04heterogeneous}.  As a
  consequence, the \emph{de facto} behaviour exhibited within a system may differ
  from that envisaged by system architects in idealised workflows.  For example,
  a ward manager in a hospital may delay releasing beds for re-allocation by
  wider hospital management in the anticipation that these will be required by
  incoming patients later in the day \citep{dewsbury07responsibility}.  This
  evolution of practice may quickly invalidate expected models of behaviour.

\end{itemize}

Due to these challenges, modelling socio-technical system behaviours using
conventional techniques will typically either result in a model that is
tractable, but lacks the necessary detail of the underlying system to provide
informative results; so narrow in scope as to be uninformative about the
behaviour of the wider system of interest; or so large and complex as to be
intractable for analysis, whether manual or automated.  The design and
construction of systems at this scale is still very much a craft, lacking the
methods and tools to support modelling and predictive simulation available in
other engineering disciplines.

The contribution of this paper is to present and evaluate a novel approach to
simulating variance in socio-technical behaviours on overall system performance,
based on the automated \emph{fuzzing} of executable workflow descriptions.  The
paper presents a method for describing socio-technical systems as
object-oriented domain models, implemented in the Python programming language.
These models are manipulated by system actors, who are directed by idealised
workflow descriptions, encoded as collections of Python functions.  Workflow
variance is simulated by automatically and dynamically fuzzing the workflow
descriptions using code fuzzing.

A proof of concept fuzzing library, Fuzzi Moss is developed and presented.  To
evaluate the proposed approach an example case study, software development
workflows was created using Fuzzi Moss.  The case study compares the
performance of different software development processes (resources
expended, features implemented and system quality) when following idealised
workflows and when following workflows that have been subject to code fuzzing
induced variance.

The rest of this paper is structured as follows.  Section \ref{sec:related}
discusses related work, covering existing techniques for modelling
socio-technical workflows and other applications of code fuzzing in software
engineering.  Section \ref{sec:fuzzi-moss} presents the method for constructing
models of socio-technical systems, associated workflows and denoting desired
fuzzings.  Where relevant, this section also discusses details of the
implementation details for Fuzzi Moss.  Section \ref{sec:evaluation} presents
the case study evaluation of the method and Section \ref{sec:conclusions}
discusses conclusions and future work, as well as noting the potential for
applying fuzzing to other forms of socio-technical models.

%%%%%%%%%%%%%%%%%%%%%%%%%%%%%%%%%%%%%%%%%%%%%%%%%%%%%%%%%%%%%%%%%%%%%%%%%%%%%%%%

\section{Related Work}

\label{sec:related}

%%%%%%%%%%%%%%%%%%%%%%%%%%%%%%%%%%%%%%%%%%%%%%%%%%%%%%%%%%%%%%%%%%%%%%%%%%%%%%%%

This section presents a literature review of the development of models and of
behaviours in socio-technical systems.  The difficulties of developing modelling
techniques that accommodate the inherent scale, complexity, contingency and
dynamism of socio-technical systems are highlighted.  In addition, existing
applications of software fuzzing are reviewed with respect to their relevance to
the present work.

Graphical notations have received considerable attention, perhaps due to their
perceived efficacy in communicating requirements between users, customers and
system architects.  These modelling languages include workflow based approaches
such as UML activity diagrams \citep{omg2010omguml}, BPMN \citep{omg2011omgbpmn},
YAWL \citep{hofstede2010yawl} and OBASHI \cite{obashimethodology}; and goal
based approaches such as KaOS \citep{Werneck2009}, \emph{i*} \citep{yu1995} and
responsibility modelling \citep{sommerville09responsibility}.

Activity diagrams are perhaps the most commonly known workflow language, due to
incorporation in the UML standard \citep{omg2010omguml}.  The notation supports
the modelling of the flow of control across a directed graph of activities, with
arcs representing transitions in control.  Additional nodes are provided for
denoting entry and exit points, as well as decision branches.  The notation is
based on the Petri Net formalism and includes support for concurrent flows
through the chart, as well as workflow forking and merging. The semi-formal
nature of the UML standard enable the automatic parsing of graphical models,
using CASE tools such as the Eclipse Modelling Framework \citep{EMFManual}.  An
advantage of this approach is that models can be used for negotiation between
project stakeholders, whilst also being used for simulations to predict system
behaviour.

The Business Process Model and Notation (BPMN) is an alternative OMG standard
for modelling workflows, with similar core notation and semantics for modelling
workflows \citep{omg2011omgbpmn}.  Unlike activity diagrams, however, BPMN
provides a richer notation for expressing more complex aspects of activities,
such as differentiating between tasks, activities and transactions; triggering
and orchestrating concurrent activities using messages; the identification of
information resources need to realise an activity; and the orchestration of
activities across organisational boundaries \citep{White2004}.  The notation is
intended to support the generation of executable business processes expressed as
web services, however, it can also be employed in other workflow contexts.

Yet Another Workflow Language (YAWL) provides similar capabilities to activity
diagrams for modelling workflows, as well as being supported by CASE tools for
graphical modelling \citep{hofstede2010yawl}.  However, unlike activity diagrams,
YAWL is based on the \picalc\citep{Aalst2004}.  The notation also provides for a
richer range of workflow requirements than activity diagrams, including
sophisticated forking and merging rules, separation between workflow
specifications and executions and resourcing and data requirements.

The OBASHI (Ownership, Business, Application, System, Hardware, Infrastructure)
methodology and notation \citep{obashimethodology} is designed for modelling
business processes across enterprise infrastructures.  The notation is intended
for capturing the movement of data through a business process and revealing the
associated dependencies on underlying infrastructure such as software systems,
servers and network communications.  The language also provides a means for
mapping these flows to higher level concerns, such as business rationale and
ultimate organisational owner. In contrast to other workflow notations, flows
are based on the movement of data rather than control.

%%%%

Describing socio-technical behaviour using workflow notations can be difficult,
because of the basic assumption that all contingencies in a workflow can be
completely described at a given level of granularity, and that more complex
details can be encapsulated within coarser grained activities with well defined
interfaces.  As argued in Section \ref{sec:introduction}, socio-technical
behaviours are inherently complex, contingent and evolutionary, making such
refinement based techniques difficult to apply.  As
\citet{israilidis13ignorance} have argued, the unknowns in a socio-technical
system may be far more significant than the knowns. Several authors have
therefore discussed alternative techniques for modelling socio-technical systems
with support for contingent behaviour
\citep{yu1995,dardenne93goal,Herrmann1999,sommerville09deriving}.

Both \emph{i*}\citet{yu1995} and KaOS \citet{dardenne93goal} are goal oriented
notations for modelling socio-technical systems \citep{Werneck2009}.  In
contrast to workflows, goal oriented approaches primarily capture the intents of
actors (what they are seeking to achieve).  Goals can be de-composed into a
sub-goal hierarchy using logical operators to express the form of
decomposition. Goals can also be annotated with strategies and/or resource
requirements to support automated analysis.  \citeauthor{yu1995} argued that
socio-technical systems should be viewed as collections of collaborating actors,
each with their own (potentially conflicting) objectives.  Eliciting and
analysing the actors intents allows the inter-dependencies between actors and
the overall behaviour of the system to be understood, without the need for
explicit models of individual workflows.

Other authors have extended goal oriented approaches to provide greater
flexibility.  \citet{sommerville09deriving} argued that stakeholders often
struggle to express their behaviour within a socio-technical system in terms of
goals.  Instead, \citeauthor{sommerville09deriving} argue that the concept of
\emph{responsibilities}, the duties held by an actor in a system, are a more
intuitive means of describing system behaviours that also capture a variety of
contingent behaviours.  A notation for expressing the relationships between
responsibilities and resources in order to identify dependencies within a system
is provided.  Earlier work on responsibility modelling also provided mechanisms
with annotating responsibilities with indicative workflows, expressing the means
by which responsibilities \emph{could} be executed
\citep{dewsbury07responsibility}.

Despite providing for contingency, a limitation of the goal and responsibility
approaches is the need for complete model descriptions. \citet{Herrmann1999}
introduced techniques for annotating goal oriented system models in the SeeMe
notation with vagueness.  The notation enables a modeller to denote where
vagueness may be present in a model due to abstraction (i.e. consistent
vagueness) and due to omission (inconsistent vagueness).  In addition,
\citeauthor{Herrmann1999} provide notation for indicating that a model is
thought to be complete, containing all pertinent details.  However, the
annotations are not accompanied by a formal semantics, or other means of
supporting automated analysis.

We are not aware of other applications of fuzzing techniques to modelling
contingent behaviours in socio-technical systems.  However, software code
fuzzing (or mutation) is employed in software quality assurance in order to
automatically generate program variants.  \emph{Mutation operators} in such
applications may alter the value of literals, swap arithmetic or other
operators, or change the ordering of arguments to a function call, for example.
Applying different combinations of mutation operators creates a population of
mutants of the target program.  One application of this technique is mutation
testing, in which the generation of program variants is used to simulate the
introduction of defects and evaluate the effectiveness of an application's test
suite in detecting regressions \citep{demillo78hints}.  A test is considered to
have detected a mutant if the application of the test to the mutant fails.  A
test suite that detects a higher proportion of mutants is considered to have
good coverage of the target program.

The effectiveness of mutant generation is significantly influenced by choice of
mutant operators to apply, since the search space of potential mutants to be
tested is very large and many mutants will reveal the same test suite
deficiencies \citep{takanen08fuzzing}.  Generation of mutants based on an
understanding of a system's specification allows mutant generation to be focused
on a system's intended behaviour.  It can therefore be expected that applying
code fuzzing to simulating socio-technical behaviours requires an understanding
of the likely variants to behaviour that may occur in a workflow in order to
generate realistic simulations.

There are a variety of existing tools that incorporate fuzzing functionality for
mutation testing, including PiTest \citep{coles14pitest} for Java and MutPy
\citep{mutpy26} and PyMuTester \citep{pymuttester} for Python.
\citet{storer15ringneck-repos} has also developed a tool for mutation testing
Maven component assembly specifications. All these tools work by constructing
and then manipulating abstract syntax trees of target programs.  The result is a
population of statically generated mutant programs that can be evaluated using
the target program's own test suite.  A disadvantage of this mechanism (for the
purposes of modelling socio-technical systems) is that the mutants are generated
statically, prior to program execution.  Our own implementation of code fuzzing
is motivated by the desire to simulate dynamic contingent behaviour, that can
vary from the idealised model each time a fuzzed step in a workflow is executed.


%%%%%%%%%%%%%%%%%%%%%%%%%%%%%%%%%%%%%%%%%%%%%%%%%%%%%%%%%%%%%%%%%%%%%%%%%%%%%%%%

\section{Method and Implementation}
\label{sec:fuzzi-moss}

%%%%%%%%%%%%%%%%%%%%%%%%%%%%%%%%%%%%%%%%%%%%%%%%%%%%%%%%%%%%%%%%%%%%%%%%%%%%%%%%

In this section we introduce our approach to modelling idealised workflows in
socio-technical systems and introducing variance in their execution through
dynamic fuzzing. The method comprises: construction of an object oriented model
of the problem domain; construction of an idealised representation of
socio-technical workflows that specify the behaviours of actors in the problem
domain; annotation of the workflows with fuzzers to introduce variance;
execution of the workflows in scenarios; and measurement of outcomes in the
scenarios through observation of the problem domain state. Each stage of the
method is described in more detail below, using the task of developing software
as an illustrative example the full case study is presented in Section
\ref{sec:evaluation}).

Implementation details of the the proof of concept library, Fuzzi Moss, are also
presented.  Source code for the Fuzzi Moss library is available from the
project's GitHub repository \citep{wallis2016fuzzi-moss-scm}.  The library was
implemented in the Python programming language.  The Python syntax is designed
to be readable, meaning that domain models and workflows could be expressed in
structured natural language.  Further, Python is an interpreted object-oriented
language, that treats function definitions as first class constructs.  This made
the development of a proof of concept fuzzing library convenient.  The core
implementation is less than 100 lines of program code, including blank lines and
source code documentation, whilst the library of utility fuzzers is implemented
in less than 300 lines. Source code for the case study is also available in a
separate GitHub repository \citep{storer2016softdev-workflow-scm} and comprises
approximately 1000 lines of program code.  Some of the example code fragments
shown below have been modified to ease explanation (removing self arguments from
function definitions, for example).

%%%%%%%%%%%%%%%%%%%%%%%%%%%%%%%%%%%%%%%%%%%%%%%%%%%%%%%%%%%%%%%%%%%%%%%%%%%%%%%%

\subsection{Problem Domain Modelling}

%%%%%%%%%%%%%%%%%%%%%%%%%%%%%%%%%%%%%%%%%%%%%%%%%%%%%%%%%%%%%%%%%%%%%%%%%%%%%%%%

The initial step in the Fuzzi Moss method is to develop a model of the target
socio-technical problem domain.  The model represents the invariant `physics' of
the problem domain that cannot be altered because it represent hard constraints
in the real world.  The model may represent constraints imposed by
natural laws, such as the conservation of energy; deliberate technical system
design, such as an access control client that limits access to a software system
to those who know an appropriate username and password combination; or emergent
phenomena that can be modelled stochastically, such as the tendency for the
implementation of new features in a software system to introduce bugs.

We have adopted an object-oriented approach to modelling the elements of the
problem domain \citep{bennett06object}, so that domain artefacts are represented
as instances of classes.  The state of a particular scenario in a problem domain
is therefore represented by a collection of inter-related Python objects
constructed from the classes specified in the domain model.  Figure
\ref{fig:feature-class-diagram} shows part of a class diagram for the software
development problem domain case study.  The diagram shows classes for system
features, representing user-facing specifications of the system's functionality;
code chunks, representing the implementation details of the system; and bugs
introduced into chunks during the completion of features.  Figure
\ref{fig:feature-instance-diagram} illustrates an example state for the same
problem domain using an object diagram, showing a feature implemented with two
chunks, one of which contains a bug.

\begin{figure}
  \centering

  \begin{subfigure}{\linewidth}
    \centering
    \includegraphics{floats/class-diagram-1}
    \caption{Class Diagram}
    \label{fig:feature-class-diagram}
  \end{subfigure}

  \vspace{10pt}

  \begin{subfigure}{\linewidth}
    \centering 
    \includegraphics{floats/object-diagram-1}
    \caption{Object diagram}
    \label{fig:feature-instance-diagram}
  
  \end{subfigure}

  \vspace{10pt}

  
  \caption{Object oriented models of a software development problem domain,
    using the UML notation.}

\end{figure}

The state of the problem domain can only be altered through explicit actions on
the domain (referred to as domain actions here), that are modelled as operations
on the domain classes.  In the example given in Figure
\ref{fig:feature-class-diagram}, the state of a feature can be altered by
invoking the \lstinline!extend()!  operation.  This adds a new chunk to the
feature's implementation and also has two side effects. The addition of new
chunks may introduce bugs into the newly created chunk, as well as create
dependencies between the new chunk and other chunks already in the system as a
whole.  Consequently, operations are also provided for refactoring to reduce
dependencies between chunks, and debugging, to remove bugs from the feature.
System features can be operated, in turn causing a random sequence of feature
chunks to be operated.  When this happens, there is the potential for the system
to fail, due to the manifestation of a bug in a chunk.

Very few restrictions are placed on the design of the problem domain operations.
Operations can accept a variety of arguments, modify the class state, invoke
operations on other problem domain classes and return values as desired.
However The operations on the problem domain should be modelled as atomic and
independent, allowing them to be safely invoked in any combination.

The object oriented problem domain model is implemented as a collection of
Python classes.  Model state is represented by python instance attributes.
Operations are implemented as Python methods in the appropriate classes.
Problem domain classes can be extended with supplementary properties using the
Python \lstinline!@property! decorator in order to improve readability as
normal.  The state of the problem domain can therefore be accessed and read as
desired for measuring behaviour in scenarios, or instrumented to record relevant
events for later analysis.

%%%%%%%%%%%%%%%%%%%%%%%%%%%%%%%%%%%%%%%%%%%%%%%%%%%%%%%%%%%%%%%%%%%%%%%%%%%%%%%%

\subsection{Workflow Modelling}

%%%%%%%%%%%%%%%%%%%%%%%%%%%%%%%%%%%%%%%%%%%%%%%%%%%%%%%%%%%%%%%%%%%%%%%%%%%%%%%%

The next stage of the method concerns the development of descriptions of
socio-technical behaviours of actors in the problem domain, using workflows.
Workflows direct the sequence of actions by an actor on the artefacts in the
problem domain.  Behaviours are modelled as \emph{idealised} workflows,
i.e. the behaviour desired or expected of an actor in a socio-technical system
without concern for contingencies.  Workflows are modelled as activity diagrams,
which enables the construction of a hierarchy of workflow descriptions,
supporting modularity and reuse.

Two related workflows are illustrated in Figure \ref{fig:workflow-partial}.
Figure \ref{fig:workflow-tdd} illustrates a Test Driven Development workflow, in
which a software developer specifies an implementation, creates a test case,
implements and debugs the feature and finally refactors the functional
implementation.  Figure \ref{fig:workflow-refactoring} illustrates the more fine
grained workflow for refactoring.  A developer, continues to refactor the
feature until the measured coupling rate is less than a desired maximum.  Both
workflows are executed by invoking operations on a \lstinline!Developer!
instance that tracks the cost of performing actions on the development project.

\begin{figure*}
  \centering

  \begin{subfigure}[b]{.45\linewidth}
    \centering
    \includegraphics{floats/tdd-workflow-1}

    \caption{Test driven development}
    \label{fig:workflow-tdd}
  \end{subfigure}
  \begin{subfigure}[b]{.45\linewidth}
    \centering
    \includegraphics{floats/refactoring-workflow-1}
    
    \caption{Refactoring}
    \label{fig:workflow-refactoring}
  \end{subfigure}
  
\

  \caption{Partial socio-technical workflows for software development expressed
    as UML activity diagrams.}

  \label{fig:workflow-partial}
\end{figure*}

We currently limit the use of activity diagram notation to begin, end, activity
and conditional branches, with transitions indicating the flow of control
between steps.  Fuzzi Moss does not currently support concurrent behaviours that
would be represented by fork and join nodes, although this feature is discussed
in Section \ref{sec:conclusions}.  The workflow models are implemented as
collections of Python functions.  As for the domain actions, implementation of
the workflow functions is flexible. Workflow functions can be parameterised
and may also return values as desired.  Workflow functions can be encapsulated
into classes to support model maintenance, although this is not necessary for
the method itself.  One restriction that is made is that workflow functions
should not contain nested function definitions, as these may be inadvertently
fuzzed by the Fuzzi Moss mechanism.

%%%%%%%%%%%%%%%%%%%%%%%%%%%%%%%%%%%%%%%%%%%%%%%%%%%%%%%%%%%%%%%%%%%%%%%%%%%%%%%%

\subsection{Specifying the Fuzzers for a Workflow}

%%%%%%%%%%%%%%%%%%%%%%%%%%%%%%%%%%%%%%%%%%%%%%%%%%%%%%%%%%%%%%%%%%%%%%%%%%%%%%%%

The workflow functions implemented in Python are denoted as eligible for fuzzing
by applying the \lstinline!@fuzz! decorator provided in the Fuzzi Moss
package. The decorator's constructor accepts one argument, a pointer to a
\emph{fuzzer} function, that will be used in the fuzzing mechanism described
below.  An example of the application of a fuzzer (identity) is shown in Figure
\ref{fig:fuzz}. Python's decorator mechanism intercepts invocations of the
decorated function and allows these to be substituted with an alternative
function to be invoked.  The Fuzzi Moss decorator exploits this mechanism by
defining and returning a \lstinline!wrap()!  function that is able to
dynamically fuzz the decorated function each time it is invoked.  The wrap
function accepts the same set of arguments as the decorated function and should
be expected to return the same values. The decorator's \lstinline!__call__()!
method, which intercepts the invocation of the decorated function, defines the
wrap function as follows.

\begin{enumerate}

\item The abstract syntax tree (AST) for the decorated function is constructed,
  using the Python \lstinline!inspect! package to recover the function's source
  code and the \lstinline!ast!  package to build the AST from the source.  This
  reference AST is then cached for future reference.

\item A copy of the AST is made.  An AST visitor (called a transformer) is then
  constructed and given the decorator's fuzzer attribute as an argument.  The
  copied syntax tree is then passed to the visitor.

\item The visitor identifies the function definition node in the AST and applies
  the fuzzer to the function definition's body.  The body is represented as a
  list of Python statements in the AST.  The function body is replaced by the
  value returned by the fuzzer.

\item Finally, the now fuzzed AST is compiled to Python byte code.  The byte
  code of the fuzzed function is then substituted for the decorated function's
  byte code and the function is invoked with its original parameters.  Control
  is then passed back to the Python interpreter, allowing any returned values to
  be passed back as normal.

\end{enumerate}


 \begin{figure}
   \centering
  \begin{lstlisting}
@fuzz(identity)
def _refactor_feature(developer, feature):
    while len(feature.dependencies) > \
        self.target_dependencies_per_feature:
        developer.refactor(feature)
  \end{lstlisting}
  
  \caption{Example fuzz decorator applied to a software development workflow.}
  \label{fig:fuzz}
\end{figure}

The generic definition of the fuzzer function (accept and return a list of
Python statement ASTs) allows a user to define a wide range of fuzzing
mechanisms to suit their problem domain.  To demonstrate this flexibility, a
suite of fuzzers is provided in Fuzzi Moss that allow the modular construction
of more complex fuzzer behaviours.  The behaviour of each of the fuzzers
available in Fuzzi Moss are described in the following sub sections.

%%%%%%%%%%%%%%%%%%%%%%%%%%%%%%%%%%%%%%%%%%%%%%%%%%%%%%%%%%%%%%%%%%%%%%%%%%%%%%%%

\subsection{Simple Fuzzers}

%%%%%%%%%%%%%%%%%%%%%%%%%%%%%%%%%%%%%%%%%%%%%%%%%%%%%%%%%%%%%%%%%%%%%%%%%%%%%%%%

A simple fuzzer accepts and returns a list of \lstinline!ast.Statement! objects
and can be declared for use directly within a fuzz decorator.  The following
simple fuzzers are defined in Fuzzi Moss.

\begin{FunctionList}

\item\lstinline!identity! returns the input list of statements.  The identity
  fuzzer is used as a default for the fuzz decorator, but is also useful when
  building composite fuzzers.

\item\lstinline!replace_steps_with_passes! returns a list of \lstinline!pass!
  statements of the same length as the input.  Replacing statements with pass is
  safer than removing the statement, since a Python function must be defined
  with at least one statement.

\item\lstinline!duplicate_steps! returns a list containing the input sequence
  repeated twice.

\item\lstinline!shuffle_steps! returns a randomly shuffled list of the input.  A
  Python Random object \lstinline!fuzzi_moss_random! is used as a random source.

\item \lstinline!swap_if_blocks! switches the body and orelse blocks of all if
  statements in the input.

\end{FunctionList}

The application of simple fuzzers was demonstrated in Figure \ref{fig:fuzz}.
Note that the fuzzer is supplied to the fuzz decorator as a function pointer,
rather than as an evaluated function call.

%%%%%%%%%%%%%%%%%%%%%%%%%%%%%%%%%%%%%%%%%%%%%%%%%%%%%%%%%%%%%%%%%%%%%%%%%%%%%%%%

\subsection{Fuzzing Filters}

%%%%%%%%%%%%%%%%%%%%%%%%%%%%%%%%%%%%%%%%%%%%%%%%%%%%%%%%%%%%%%%%%%%%%%%%%%%%%%%%

Sometimes it is desirable to restrict the application of simple fuzzers to
particular portions of the body of a workflow function.  In this situation the
\lstinline!filter_steps()! fuzzer can be used.  Filter steps takes two
arguments: a filter function pointer and a fuzzer function pointer.  These are
used to define a nested \lstinline!_filter_step! fuzzer that behaves like a
simple fuzzer as described above.

A filter function accepts a list of statements and returns a list of tuples.
Each tuple denotes the start and end index for a block of statements in the
input.  Each block is then fuzzed using the supplied fuzzer.  Blocks not
specified by the filter are not affected by the fuzzer, such that the returned
sequence of statements may contain a mix of fuzzed and un-fuzzed statements.

The following filters are available for use in conjunction with a fuzzer.

\begin{FunctionList}

\item \lstinline!choose_last_step! returns the start and end index of the last
  statement in the input, i.e. \lstinline!input[-2:-1]!

\item \lstinline!choose_random_steps(n)!  returns $n$ length 1 sub-blocks
  randomly selected from the input steps.

\item \lstinline!exclude_control_structures(target)! permits the exclusion of
  control structure statements by type as specified by the target input
  argument. Supported structures are \lstinline!for!, \lstinline!while!,
  \lstinline!if!, \lstinline!try!-\lstinline!except! and \lstinline!return!.

\end{FunctionList}

An invert filter is also provided which inverts the selection provided by its
argument. Several pre-defined filtering fuzzers are implemented in Fuzzi Moss
combining some of the predefined filters and simple fuzzers.  These provide
commonly required fuzzers, including removing or duplicating the last step or a
random step. A \lstinline!choose_identity! filter is also included for
completeness.  Figure \ref{fig:filter} illustrates the application of a filter
steps fuzzer that randomly shuffles all but the last step.

\begin{figure}
  \centering

\begin{lstlisting}
@fuzz(
  filter_steps(
    invert(choose_last_step),
    shuffle_steps
  )
)
def work(self, system, developer, schedule):

  self.complete_specification(schedule, system)
  self.implement_features(developer, system)
  self.implement_test_suite(developer, system)
  self.debug_system(developer, system)
  self.refactor_system(developer, system)

\end{lstlisting}

\caption{Using filters to control the application of step replacement to all but
  the last step in a Waterfall software development workflow.}
  \label{fig:filter}
\end{figure}

%%%%%%%%%%%%%%%%%%%%%%%%%%%%%%%%%%%%%%%%%%%%%%%%%%%%%%%%%%%%%%%%%%%%%%%%%%%%%%%%

\subsection{Composite Fuzzers}

%%%%%%%%%%%%%%%%%%%%%%%%%%%%%%%%%%%%%%%%%%%%%%%%%%%%%%%%%%%%%%%%%%%%%%%%%%%%%%%%

More complex fuzzers can be assembled using composite fuzzing functions that are
implemented in a similar manner to filtering.  


\begin{FunctionList}
\item \lstinline!in_sequence(fuzzers)! applies each fuzzer found in the input
  list of fuzzers in sequence to a function body.

\item \lstinline!choose_from(distribution)! selects a fuzzer to apply at random
  from the supplied probability distribution.  The distribution is defined as a
  list of weight, fuzzer tuples.

\item \lstinline!on_condition_that(condition, fuzzer)! %
  applies the specified fuzzer if the specified condition holds.  The condition
  may be a literal Boolean value, a function pointer or a Python expression.
  Lambda expressions are not supported.

\item \lstinline!recurse_into_nested_steps(target)! identifies
  control structure statements and applies the supplied fuzzer to their body
  blocks.  The recursion can be limited to particular control structures using
  the option \lstinline!target! argument.
\end{FunctionList}

An example of applying a composite fuzzer to a workflow is shown in Figure
\ref{fig:composite}.  The figure shows a workflow for enhancing the quality of a
software system.  A developer adds a test to a feature in order to detect the
presence of bugs, or prevent the introduction of bugs during refactoring.  The
developer then performs some debugging work, followed by some refactoring.  The
fuzzer applied to the workflow is the \lstinline!choose_from!.  The supplied
distribution will fuzz the workflow 5\% of the time on average, removing one
step at random when it does so.  The figure therefore shows a convenient way of
modelling the occasional random omission of steps in a workflow.

\begin{figure}
  \centering

\begin{lstlisting}
@fuzz(
  choose_from([
    (0.95, identity),
    (0.05, remove_random_step)
  ])
)
def _enhance_system_quality(
    self, feature, developer):

  developer.add_test(feature)
  self._debug_feature(developer, feature)
  self._refactor_feature(developer, feature) 
\end{lstlisting}
  
  \caption{Application of a composite fuzzer to a workflow.}
  \label{fig:composite}
\end{figure}

%%%%%%%%%%%%%%%%%%%%%%%%%%%%%%%%%%%%%%%%%%%%%%%%%%%%%%%%%%%%%%%%%%%%%%%%%%%%%%%%

\subsection{Control Structure Fuzzers}

%%%%%%%%%%%%%%%%%%%%%%%%%%%%%%%%%%%%%%%%%%%%%%%%%%%%%%%%%%%%%%%%%%%%%%%%%%%%%%%%

The final set of fuzzers provide for manipulation of control structure
conditions and iterators.  Both of these fuzzers inspect the input list of
statements for compatible statement types to fuzz and apply the fuzzing to every
one found.

\begin{FunctionList}

\item \lstinline!replace_condition_with(condition)! replaces conditions
  discovered in control structure statements with the specified condition.  The
  condition may be a literal Boolean value, a function pointer or a Python
  expression defined in a string.  Lambda expressions are not supported.

\item \lstinline!replace_for_iterator_with(iterator)! replaces iterators
  discovered in for loops with the specified iterator.  The replacement iterator
  must be specified as a list containing numerical or string literal types only.

\end{FunctionList}

Control structure fuzzers are useful for introducing variability in decision
making into a workflow.  For example, a developer may decide to stop working on
refactoring a feature before the number of dependencies on other parts of a
system are reduced below the specified level, even if this is required by the
idealised workflow.  Similarly, a developer may skip the application of a task
to a feature through an error of omission.

%%%%%%%%%%%%%%%%%%%%%%%%%%%%%%%%%%%%%%%%%%%%%%%%%%%%%%%%%%%%%%%%%%%%%%%%%%%%%%%%

\subsection{Defining and Evaluating Scenarios}

%%%%%%%%%%%%%%%%%%%%%%%%%%%%%%%%%%%%%%%%%%%%%%%%%%%%%%%%%%%%%%%%%%%%%%%%%%%%%%%%

Experiments in Fuzzi Moss are created by defining scenarios with initial
conditions (available resources, participants and so on), as well as a schedule
of work items.  The schedule of work items is specified as part of the problem
domain, so may be altered during the execution of a workflow, or as a result of
external events during a simulation.  For example, the specified set of features
for a software system may be altered as development proceeds and requirements
become better understood.

By convention, workflows are organised as Python classes, and contain a
top-level \lstinline!work()! method to initiate activity.  Invoking this method
causes the simulation to be executed.  Additional behaviour may also be
simulated in order to evaluate the behaviour of the workflow.  For example, a
software system may be operated within a simulation after a development workflow
has been executed in order to estimate quality assurance characteristics such as
mean operations to failure.

The Fuzzi Moss package exposes variables that can be used to configure the
global behaviour of the fuzz decorator and support the management of
experiments. Specifically:

\begin{itemize}
\item A Python random object, \lstinline!fuzzi_moss_random!.  This object is
  used as a source of randomness where required by the Fuzzi Moss fuzzers.  The
  object can be seeded or even completely replaced as desired by the modeller.

\item A Boolean variable, \lstinline!enable_fuzzings! is provided, which can be
  used to control whether fuzzings are applied.  This feature is useful for
  executing the fuzzed workflows in their idealised state, since this provides
  for comparison with fuzzed workflows.

\end{itemize}

The dynamic application of fuzzings to socio-technical behaviours in Fuzzi Moss,
mean that different sets of fuzzings may be applied to different runs of the
same scenario.  Many problem domains will also incorporate some form of
probabilistic functionality in order to represent stochastic elements of a
problem domain.  Therefore, it is anticipated that the characteristics of a
workflow should be measured across multiple runs of the same scenario.

%%%%%%%%%%%%%%%%%%%%%%%%%%%%%%%%%%%%%%%%%%%%%%%%%%%%%%%%%%%%%%%%%%%%%%%%%%%%%%%%

\section{Evaluation}
\label{sec:evaluation}

%%%%%%%%%%%%%%%%%%%%%%%%%%%%%%%%%%%%%%%%%%%%%%%%%%%%%%%%%%%%%%%%%%%%%%%%%%%%%%%%

This section describes the results of a case study that was undertaken to
evaluate the efficacy of executable workflow fuzzing for modelling, simulating
and predicting behaviours in software systems.  Software development was the
problem domain chosen for the case study, because the familiarity of the authors
with the domain and associated workflows mitigated the risk of infidelity in the
domain model.  This allowed the evaluation to focus on the efficacy of the
fuzzing method, rather than the validity of the case study model. The problem
domain model is illustrated in Figure \ref{fig:full-class-diagram}.  The
relationships between features, chunks and bugs have already been described in
Section \ref{sec:fuzzi-moss}.  Aspects of the model that have not yet been
described are given below.

\begin{figure*}
  \centering
  \includegraphics{floats/full-class-diagram-1}
  \

  \

  \caption{Full class diagram of the software development problem domain.}
  \label{fig:full-class-diagram}
\end{figure*}


The \lstinline!Test! class represents unit tests that are developed against
system specifications detailed by the \lstinline!Feature! class.  The detection
of bugs in a feature by tests is set probabilistically, but is deterministic
(once set, a test will either always or never reveal a bug when exercised).  By
contrast the introduction of bugs into chunks by modification; the relationship
between bugs and chunk operation; and the introduction of dependencies between
chunks are stochastic and modelled by probabilistic functions.

Software systems are characterised by a collection of features and associated
tests.  Like features, software systems can be operated.  When this happens,
features are selected randomly and their own \lstinline!operate()!  operation
invoked.  This feature selection and operation continues until either a maximum
trace size is reached, an incomplete feature is operated, or a bug is
manifested.  Software systems also record logs of traces of feature operations
and other characteristics implemented as Python properties, including mean
operations to failure of historic trace runs.  Developers perform the work on a
software system in terms of development, debugging and refactoring features.
All of these tasks consume available person time, measured in person time units
(ptu), which is tracked as a \lstinline!Developer! class attribute.  Developers
are associated with software systems by the \lstinline!SoftwareProject! class,
which encapsulates a particular run of a software development workflow and
subsequent operation.  Finally, software projects for the same scenario are
grouped together in the \lstinline!SoftwareProjectGroup! class in order to
obtain averaged metrics across multiple runs of the same simulation.

Two software development workflows were selected for evaluation in the case
study: Waterfall \citep{benington83production} and Test Driven Development (TDD)
\citep{beck02test}.  Again, these workflows were selected due to their
familiarity to the authors.  To gain additional assurance, unit tests were
developed for each of the problem domain and workflow classes to document the
expected behaviour of each class independently (using mocks to replace
dependencies) and to gain assurance that the exhibited behaviour without fuzzing
was that intended by the authors.

The implementation of the two workflows, decorated with fuzzers is illustrated
in Figure \ref{fig:workflow-impl}.  Although the two workflows describe
different behaviours, the implementation has been chosen to enable comparison
between the workflow structures and fuzzers chosen.  Fuzzers were applied based
on the judgement of the authors regarding the typical variance that occurs in
software development activities, in terms of omitted or re-ordered steps.
Following implementation, it was hypothesised that:

\begin{itemize}

\item The waterfall workflow would achieve higher completion rates for features
  than test driven development, given the upfront implementation of
  functionality.

\item TDD would realise higher software quality (expressed as mean operations to
  failure) due to the emphasis on delivering working software.

\item Waterfall would consume less resources in the context of excess resources,
  due to the termination of the workflow as soon as planned quality objectives
  are met.

\item The behaviour of TDD would be less susceptible to socio-technical variance
  due to the application of workflow fuzzing, due to it's iterative nature.

\end{itemize}


\begin{figure*}
  \centering
  
  \begin{subfigure}[b]{.45\linewidth}
\begin{lstlisting}[basicstyle=\ttfamily\scriptsize]
@fuzz(choose_from(
 [(0.95, identity), (0.05, remove_random_step)]
 )
)
def work(system, developer, schedule):
 _complete_specification(schedule, system)
 _implement_features(developer, system)
 _implement_test_suite(developer, system)
 _debug_system(developer, system)
 _refactor_system(developer, system)

@fuzz(choose_from(
 [(0.95, identity),
  (0.05, remove_random_step)]
 )
)
def _complete_specification(
  schedule, system):
  for feature_size in schedule:
    system.add_feature(feature_size)

@fuzz(choose_from(
 [(0.99, identity),
  (0.01, replace_condition_with(False))]
 )
)
def _implement_features(
 developer, system):
  for feature in system.features:
    while not feature.is_implemented:
      developer.extend_feature(feature)

@fuzz(choose_from(
 [(0.95, identity),
  (0.05, replace_condition_with(False))]
 )
)
def _implement_test_suite(
  developer, system):
  for feature in system.features:
    while feature.test_coverage < \
      target_test_coverage_per_feature:
      developer.add_test(feature)

@fuzz(
 choose_from(
 [(0.99, identity), 
  (0.01, replace_condition_with(False))]
 )
)
def _debug_system(developer, system):
  for test in system.tests:
    while True:
      try:
        test.exercise()
        break
      except BugEncounteredException as e:
        developer.debug(test.feature, e.bug)

@fuzz(
  choose_from(
  [(0.99, identity), 
   (0.01, replace_condition_with(False))]
 )
)
def _refactor_system(developer, system):
  for feature in system.features:
    while len(feature.dependencies) > \
      target_dependencies_per_feature:
      developer.refactor(feature)






''''''
\end{lstlisting}
    \caption{Waterfall}
  \end{subfigure}
  \hfill
  \begin{subfigure}[b]{.45\linewidth}
\begin{lstlisting}[basicstyle=\ttfamily\scriptsize]
@fuzz(
  recurse_into_nested_steps(
    target_structures={ast.For, ast.TryExcept},
    fuzzer=filter_steps(
      exclude_control_structures(),
      fuzzer=choose_from(
      [(0.95, identity), 
       (0.05, remove_random_step)])
    )
  )
)
def work(system, developer, schedule):

  for feature_size in schedule:
    try:
      feature = \
        system.add_feature(feature_size)
      _ensure_sufficient_tests(developer, feature)
      _complete_feature(developer, feature)
      _refactor_feature(developer, feature)
    except DeveloperExhaustedException:
      system.features.remove(feature)

  while True:
    try:
      feature = choice(system.features)
      _enhance_system_quality(feature, developer)
    except DeveloperExhaustedException:
      break

@fuzz(choose_from(
  [(0.95, identity), 
   (0.05, replace_condition_with(False))]))
def _ensure_sufficient_tests(developer, feature):
  while feature.test_coverage <\
    target_test_coverage_per_feature:
    developer.add_test(feature)

@fuzz(choose_from(
  [(0.99, identity), 
   (0.01, replace_condition_with(False))]))
def _complete_feature(developer, feature):
  while not feature.is_implemented:
    developer.extend_feature(feature)
    _debug_feature(developer, feature)

@fuzz(choose_from(
  [(0.95, identity), 
   (0.05, remove_random_step)]))
def _enhance_system_quality(feature, developer):
  developer.add_test(feature)
  _debug_feature(developer, feature)
  _refactor_feature(developer, feature)

@fuzz(choose_from(
 [(0.99, identity), 
  (0.01, replace_condition_with(False))]
))
def _debug_feature(developer, feature):
  while True:
    try:
      feature.exercise_tests()
      break
    except BugEncounteredException as e:
      developer.debug(feature, e.bug)

@fuzz(
 choose_from(
  [(0.99, identity), 
   (0.01, replace_condition_with(False))]
))
def _refactor_feature(developer, feature):
  while len(feature.dependencies) >\
   target_dependencies_per_feature:
    developer.refactor(feature)

''''''
\end{lstlisting}

    \caption{Test Driven Development}
  \end{subfigure}

  \caption{Workflow implementations in Python with fuzzers for Waterfall and Test Driven software development.}
  \label{fig:workflow-impl}
\end{figure*}

A default scenario was configured in which a single developer was tasked with
building a software system with a schedule comprising three features, of three,
five and seven chunks in order of priority.  Both workflows aimed to achieve a
target test coverage of 100\% of code chunks and a inter-feature dependency rate
of 0.  Dependencies between chunks within features were unregulated.  Each
system was constructed following the specified workflow behaviour, applying any
specified fuzzings.  Then, each resulting system was operated 50 times, allowing
for a maximum trace of 750 features per operation. Each overall build-operate
sequence was executed 10 times, resulting in 500 system operation traces per
configuration.

The workflows scenarios were evaluated in twelve configurations, parameterised
by workflow type, resource availability and the presence of fuzzing.  Initially,
the experiment was calibrated to determine an adequate resourcing level for the
Waterfall methodology to complete, by executing the Waterfall workflow and
determining an average resource consumption.  This was found to be approximately
250ptu.  Then, two further resource parameters were specified: inadequate
resources (50ptu) and excess resources (500ptu).  Each of these three resource
configurations were run with workflow fuzzing disabled and enabled.  Average
mean times to failure (feature operation trace length), remaining person time
resources, completed features and simulation run time were recorded and
tabulated as shown in Table \ref{tab:results}.

\begin{table}
 \caption{%
    Results for twelve simulation runs of Fuzzi Moss on the software development
    problem domain.  The table shows workflow type, resources allocated (excess,
    adequate, inadequate), whether fuzzings were applied, mean feature operations
    to failure, remaining person unit time available at end of development phase,
    average features asserted as implemented, and simulation runtime.
  }
  \label{tab:results}

  \centering
  \begin{tabular}{|l|r|l|r|r|r|r|l|} \hline
    Workflow & \begin{tabular}{@{}c@{}}Res.\\ PTU\end{tabular}& Fuzz  & MOF & \begin{tabular}{@{}c@{}}Rem.\\ PTU\end{tabular} & Feat. & \begin{tabular}{@{}c@{}}RT\\PTU\end{tabular}\\ \hline

    Waterfall & 50 & No & 14 & 1 & 3.0 & 3 \\
    Waterfall & 250 & No & 72 & 93 & 3.0 & 5 \\
    Waterfall & 500 & No & 72 & 343 & 3.0 & 6 \\
    Waterfall & 50 & Yes & 10 & 4 & 3.0 & 2 \\
    Waterfall & 250 & Yes & 114 & 136 & 2.4 & 5 \\
    Waterfall & 500 & Yes & 114 & 386 & 2.4 & 5 \\
    TDD & 50 & No & 259 & 1 & 1.5 & 8 \\
    TDD & 250 & No & 680 & 1 & 3.0 & 28 \\
    TDD & 500 & No & 750 & 1 & 3.0 & 56 \\
    TDD & 50 & Yes & 259 & 1 & 1.5 & 10 \\
   TDD & 250 & Yes & 513 & 1 & 3.0 & 34 \\
    TDD & 500 & Yes & 750 & 1 & 3.0 & 70 \\ \hline
 

  \end{tabular}
\end{table}

As expected, without fuzzing, the TDD workflows consume all available resource,
in contrast to the plan driven approach followed by Waterfall.  These results
confirm the hypothesis that TDD is a `greedy' workflow, absorbing available
resources allocated to a project (\citet{sommerville10software} has argued that
this characteristic makes agile like methodologies such as TDD difficult to
develop contracts for).  Note that the similarity in results between the
adequate and excess resource requirements for Waterfall is likely explained by
the workflow terminating once the minimum resources required (250) are consumed,
regardless of the excess availability.

The results also confirm the hypotheses that without fuzzing, the simulated TDD
workflow achieves higher software quality than the simulated Waterfall workflow,
at the cost of a reduction of features incorporated in the system.  TDD achieves
a mean operation to failure that is higher than Waterfall even when comparing
execution of constrained TDD resources with unconstrained Waterfall (259 vs
114).  This difference is explained by the far greater presence of bugs and
greater complexity of systems built following the Waterfall driven approach, as
well as the presence of incomplete features in the systems built following the
Waterfall workflow.  This distinction is particularly marked in the context of
constrained resources, in which test driven development achieves a far higher
mean operations to failure (259 vs 14), but fewer features are delivered (1.5 vs
3.0).

The result of applying fuzzing to the workflows reveals both expected and
unexpected behaviours.  On the one hand, the results support the hypothesis that
TDD is relatively resistant to workflow fuzzing, since the
measured characteristics (MOF and Features implemented) are not dramatically
affected by the application of fuzzing.  In contrast, the measurements for the
Waterfall model are changed considerably.  However, unexpectedly, fuzzing does
not appear to have reduced quality, but rather to have altered the
characteristics to be closer to the behaviour of TDD, with
reduced features implemented and greater software quality.  This result may be
explained by the relatively small number of features to be implemented.  In this
circumstance, forgetting to implement a feature (simulated by a fuzzer replacing
this step with a pass) will cause the number of dependencies to be reduced and
hence the likelihood of manifesting a bug.

Finally, the results show the extent to which Fuzzi Moss increases the runtime
for a simulation.  This is most noticeable for the excess resource simulations
that operate systems to the specified trace limit.  In the TDD scenario, Fuzzi
Moss adds 20\% to the runtime of the simulation.  Much of this additional cost
will be attributable to the dynamic recompilation of workflow functions each
time a fuzzer is applied.  Further investigation with more complex workflow
models and large scenarios is required to assess this cost and determine whether
optimisations are necessary.


%%%%%%%%%%%%%%%%%%%%%%%%%%%%%%%%%%%%%%%%%%%%%%%%%%%%%%%%%%%%%%%%%%%%%%%%%%%%%%%%

\section{Conclusions}
\label{sec:conclusions}

%%%%%%%%%%%%%%%%%%%%%%%%%%%%%%%%%%%%%%%%%%%%%%%%%%%%%%%%%%%%%%%%%%%%%%%%%%%%%%%%

This paper has presented and evaluated the use of executable workflow fuzzing to
the problem of modelling and simulating variance in socio-technical system
behaviours.  The paper described a proof-of-concept workflow fuzzing tool, Fuzzi
Moss, and applied it in a case study of software development workflows.  The
workflow tool was demonstrated to introduce realistic variance into idealised
workflows in accordance with expectations of software development workflow in
practice.

The proof of concept has created a substantial research agenda in the
application of fuzzing techniques to socio-technical system modelling.  Within
the scope of the Fuzzi Moss project, several immediate next steps are proposed:

\begin{itemize}

\item Evaluate Fuzzi Moss further by extending the current case study to
  incorporate more complex aspects of software development (such as change
  management). Other case studies such as the e-counting system described in
  detail by \citet{lock07observations} will also be investigated.  A more
  complex case study will allow the identification of required new fuzzers, as
  well as areas in which the methodology can be optimised.

\item The implementation of support for concurrent workflows.  Fuzzi Moss
  simulations are currently executed as a single thread.  However, real world
  socio-technical systems are most conveniently modelling as a collection of
  concurrently interacting workflows.  The modular implementation of Fuzzi Moss
  should make this extension straight forward.

\item Experimentation with the Fuzzi Moss API (aspects) and extension of fuzz
  operator capability.  In some respects, the current specifications of fuzzers
  using function decorators is unsatisfactory, since it prevents flexible
  experimentation with different fuzzer configurations.  An aspect-oriented
  style approach \citep{filman01aspect}, with problem domain models oblivious to
  the application of fuzzers may be more appropriate for this purpose.

\item The development of more rigorous methods for identifying and validating
  suitable fuzzers for application to workflows.  One possibility is the
  development of recommendations based on the internal structure of a workflow.
  An alternative option is methods that allow the interactive specification of
  workflows with project stakeholders.  Workflows could be developed through the
  narration of a scenario by a stakeholder, similar to the act of literate
  programming \cite{knuth84literate}.

\end{itemize}

More widely, fuzzing techniques may be applicable for other socio-technical
modelling formalisms.  For example, fuzzing goal oriented models such as
\emph{i*} could be used to simulate the shifting goals of actors over time as
priorities and focus varies.  Similarly, fuzzing enterprise modelling techniques
such as OBASHI could provide means of assessing the resilience of critical
infrastructures when subject to unexpected behaviours, similar to HAZOPS like
techniques that have been applied manually to responsibility models
\cite{lock09modelling}. The proof of concept in this research demonstrates the
potential for the development of realistic simulations of socio-technical
systems with predictive power.  The availability of such tools would do much to
progress the current craft of large scale systems engineering.

%%%%%%%%%%%%%%%%%%%%%%%%%%%%%%%%%%%%%%%%%%%%%%%%%%%%%%%%%%%%%%%%%%%%%%%%%%%%%%%% 

\bibliographystyle{abbrvnat}
\bibliography{lib}


%%%%%%%%%%%%%%%%%%%%%%%%%%%%%%%%%%%%%%%%%%%%%%%%%%%%%%%%%%%%%%%%%%%%%%%%%%%%%%%%


\end{document}